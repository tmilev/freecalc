\documentclass{article}
\usepackage{amsmath, amsfonts, amssymb, verbatim, hyperref, ifthen}
\usepackage{amsthm}
\usepackage{multicol}
\usepackage{longtable}
\usepackage{etoolbox}
\usepackage{comment}
\usepackage{cleveref}
\usepackage{enumitem}

\crefformat{footnote}{#2\footnotemark[#1]#3}

\newtoggle{solutions}
\newtoggle{solutionsExtra}
\newtoggle{answers}

\usepackage[latin1]{inputenc}
\usepackage{etex}
\usepackage{ifthen}
\usepackage[strings]{underscore}
\usepackage{tikz}
\usetikzlibrary{calc}
\usepackage{bbding}
\let\Cross\relax
\let\Square\relax
\usepackage{amsmath}
\usepackage{amssymb}
\usepackage{cancel}
\usepackage{comment}
\usepackage{multirow}
\usepackage{multicol}
\usepackage{psfrag}
\usepackage{rotating}
\usepackage{fp}
\usepackage{calc}
\usepackage{bm}
\usepackage[all,cmtip]{xy}
\RequirePackage{xstring}
\usepackage{times}
\usepackage[english]{babel}
\usepackage{longtable}
\usepackage{graphicx}
\usepackage{verbatim}
\usepackage{array}
\usepackage[breakwords]{truncate}


\newcommand{\freecalcBaseFolder}{../..}

\providecommand{\autopstpdfConflictResolutionTemporary}{
\usepackage[
dvips={-o -Ppdf}, 
pspdf={
-dNOSAFER
%-dAutoRotatePages=/None %<-breaks in windows:%
}, 
pdfcrop={}, 
%crop=off%without crop=off breaks in windows
]{auto-pst-pdf}
}


%%%%%%%%%%%%%%%%%%%%%%%%%%%%%%%%%%%%%%%%%%
%
% List of commands in this document
%
%
% \logdiffbaseandexp
% \logdifftwouponedown
% \productrulefofx
% \quotientruley
% \limitradical  (broken)
% \limitsub
% \chainruley
% \chainrulefofx
% \chainruleStyleOne
% \chainruleStyleTwo
% \chainruleStyleThree
% \infinitelimit
% \limitfactor
% \newtonsmethod
% \constantmultiple
% \chainruletwice
% \youWillNotBeTested
% \optionalDisplay  %Dummy command needed for compatibility with Calculus notes.
% \Arcsin
% \Arccos
% \Arctan
% \Arccot
% \diff
%%%%%%%%%%%%%%%%%%%%%%%%%%%%%%%%%%%%%%%%%%

\newcommand*{\minwidthbox}[2]{%
  \relax
  \ifmmode
    \mathpalette{\minwidthboxmath{#1}{#2}}{}%
  \else
    \makebox[{\ifdim#1<\width\width\else#1\fi}]{#2}%
  \fi
}
\newcommand*{\minwidthboxmath}[4]{%
  % #1: minimum width
  % #2: box contents
  % #3: math style
  % #4: unused
  \mbox{\minwidthbox{$#3#1$}{#2}}%
}

\newcommand{\diff}{{\normalfont \text{d}}}
\newcommand{\ds}{\displaystyle}
\newtheorem{question}{Question}
\newtheorem{emptyTheorem}{}
\newtheorem{mathematicalRule}{Rule}
\newtheorem{observation}{Observation}
\newtheorem{algorithm}{Algorithm}
\newtheorem{proposition}{Proposition}
\newtheorem{remark}{Remark}
\newcommand{\youWillNotBeTested}{\begin{frame}You will not be tested on the material in the following slide.\end{frame}}
\DeclareMathOperator{\Vol}{Vol}

\DeclareMathOperator{\Arcsin}{\sin^{-1}}
\DeclareMathOperator{\Arccos}{\cos^{-1}}
\DeclareMathOperator{\Arctan}{\tan^{-1}}
\DeclareMathOperator{\Arccot}{{\cot^{-1}}}
\DeclareMathOperator{\Arcsec}{{\sec^{-1}}}
\DeclareMathOperator{\Arccsc}{{\csc^{-1}}}
\DeclareMathOperator{\sech}{sech}
\DeclareMathOperator{\csch}{csch}

\DeclareMathOperator{\maclaurin}{{\normalfont{Mc}}}
\newcommand{\taylor}{{\normalfont{T}}}

\newcommand{\optionalDisplay}[1]{#1}
\renewcommand{\Im}{\mathrm{Im}}
\renewcommand{\Re}{\mathrm{Re}}

%\DeclareMathOperator{\Re}{Re}
%\DeclareMathOperator{\Im}{Im}
\newcommand{\fcv}[1]{{\bf #1}} %this command stands for freecalc Vector
\DeclareMathOperator{\curl}{\fcv{curl}}
\DeclareMathOperator{\divg}{div}
\DeclareMathOperator{\proj}{\fcv{proj}}
\DeclareMathOperator{\orth}{\fcv{orth}}
\DeclareMathOperator{\grad}{\fcv{grad}}
\newcommand{\RR}{{\mathbb{R}}}
\newcommand{\cR}{{\mathcal{R}}}
\newcommand{\cD}{{\mathcal{D}}}
\newcommand{\cP}{{\mathcal{P}}}
\newcommand{\alertNoH}[2]{\alert<handout:0|#1>{#2}}
\newcommand{\fcUncoverAlert}[2]{\uncover<#1->{\alertNoH{#1}{#2}}}
\newcommand{\uncoverAlert}[2]{\uncover<#1->{\alertNoH{#1}{#2}}}
\newcommand{\rectangle}{{%
  \ooalign{$\sqsubset\mkern2mu$\cr$\mkern1mu\sqsupset$\cr}%
}}
\newcommand{\worksheet}[1]{\uncover<1->{#1}}
\newcommand{\turnOnWorksheetMode}{\renewcommand{\worksheet}[1]{\uncover<handout:0|1->{##1}}}
%Code from user cfr from stackexchange, from discussion:
%http://tex.stackexchange.com/questions/259138/beamer-class-creating-3-modes-presentation-handout-and-custom
%\makeatletter
%\newcommand{\turnOnWorksheetMode}{%
%\gdef\beamer@currentmode{worksheet}%
%\def\animate<##1>{\transduration<##1| handout:0| worksheet:0| trans:0>{0}}%
%}
%\makeatother
\newcommand{\fcAnswerNoH}[2]{
\FPeval{\fcResult}{clip(#1-1)}
\uncover<handout:0|\fcResult>{\alertNoH{\fcResult}{\textbf{?} }} \worksheet{\uncover<handout:0| #1->{\alertNoH{#1}{\!\!\!#2}}}
}
\newcommand{\fcAnswer}[2]{%
\uncover<handout:0|\the\numexpr#1-1\relax>{\alertNoH{{\the\numexpr#1-1\relax}}{{\textbf{?}}}}\worksheet{\uncover<#1->{{\alertNoH{#1}{\!\!\!#2}}}}%
}%

\newcommand{\fcAnswerUncover}[3]{%
\FPeval{\fcResult}{clip(#2-1)}%
\uncover<handout:0|#1-\fcResult>{\alertNoH{\fcResult}{\textbf{?}}}\worksheet{\uncover<#2->{\alertNoH{#2}{\!\!\!#3}}}
%\makebox[\widthof{#3}][c]{\only<handout:0|#1-\fcResult>{\alertNoH{\fcResult}{\textbf{?}}} \only<#2->{\alertNoH{#2}{\!\!\!#3}}}%
}
\newcommand{\fcAnswerUncoverNew}[4]{%
\uncover<handout:0|.(#1)-.(#2)>{\alertNoH{.(#2)}{\textbf{?}}}\worksheet{\uncover<.(#3)->{\alertNoH{.(#3)}{\!\!\!#4}}}
%\makebox[\widthof{#3}][c]{\only<handout:0|#1-\fcResult>{\alertNoH{\fcResult}{\textbf{?}}} \only<#2->{\alertNoH{#2}{\!\!\!#3}}}%
}

\newcommand{\fcAnswerUncoverNoH}[3]{
\FPeval{\fcResult}{clip(#2-1)}
\uncover<handout:0|#1-\fcResult>{\alertNoH{\fcResult}{\textbf{?}}}\worksheet{\uncover<handout:0|#2->{\alertNoH{#2}{\!\!\!#3}}}
}

\newcommand{\fcQuestion}[2]{%
\FPeval{\fcResult}{clip(#1+1)}%
\uncover<#1->{\alertNoH{ #1,\fcResult}{#2}}%
}
\newcommand{\fcEvalToInt}[1]{\FPeval{\fcResult}{clip(#1)}\fcResult}
\newcommand{\refBad}[3]{%
\ifthenelse{\equal{#1}{??}}%
{#2}%
{#3}%
}%example usage: \refBad{\ref{eqMacLaurinDef}}{their definition}{their definition (Definition \ref{eqMacLaurinDef})}
\newcommand{\fcCancel}[2]{%
\FPeval{\fcResult}{clip(#1-1)}%
\only<handout:0|-\fcResult>{{#2}} \only<#1->{{\alertNoH{#1}{{\cancel{{\alertNoH{0}{{#2}}}}}}}}
\vphantom{{\cancel{#2}}}%
}
%<-WARNING: the superflous-looking \alertNoH{0} is needed:
% for some unknown to me reason it causes LaTeX to add the correct amount of spacing.

%code blocks regular expression that replaces all strings of the form \alert<handout:0| a> by \alertNoH{a}:
%Find:
%\\alert<[^|^0]*0|\([^>]*\)>
%Replace:
%\\alertNoH{\1}
%code blocks regular expression that replaces all strings of the form \alert<a> but not containing | by \alertNoH{a}:
%Find:
%\\alert<\([^|^>]*\)>
%Replace:
%\\alertNoH{\1}

\newcommand{\fcLicenseContent}{
These lecture slides and their \LaTeX{} source code are licensed to you under the Creative Commons license CC BY 3.0. You are free
\begin{itemize}
\item to Share - to copy, distribute and transmit the work,
\item to Remix - to adapt, change, etc., the work,
\item to make commercial use of the work,
\end{itemize}
as long as you reasonably acknowledge the original project.
\begin{itemize}
\item Latest version of the .tex sources of the slides: \url{https://github.com/tmilev/freecalc}
\item Should the link be outdated/moved, search for  ``freecalc project''.
\item Creative Commons license CC BY 3.0:
\url{https://creativecommons.org/licenses/by/3.0/us/}
and the links therein.
\end{itemize}
}

\newcommand{\fcDeterminantThreeByThree}[9]{%
\left|\begin{array}{ccc}
\alertNoH{.(1),.(4)}{#1} & \alertNoH{.(2),.(5)}{#2}  &  \alertNoH{.(3),.(6)}{#3}\\
\alertNoH{.(3),.(5)}{#4}  & \alertNoH{.(1),.(6)}{#5} &  \alertNoH{.(2),.(4)}{#6}\\
\alertNoH{.(2),.(6)}{#7}  & \alertNoH{.(3),.(4)}{#8}  & \alertNoH{.(1),.(5)}{#9}\\
\end{array}\right|
}%

\newcommand{\fcLicense}{
\begin{frame}
\frametitle{License to use and redistribute}
\fcLicenseContent
\end{frame}
}
\newcommand{\onlyNoH}[2]{\only<handout:0|#1>{#2}}

%\newcommand{\fcUncover}[2]{{\newcommand{\tempExpander}{#1}\uncover<\tempExpander>{#2}}

%Produces a picture which illustrates graphically the solution of
%sin x = sin a (a=known angle).
%argument1: the number of the first frame
%argument2: the angle a
%argument3: the label of the segment representing sin a
%argument4: the label of the angle a
%argument5: the label of the other angle that solves the problem
%Example:
%\uncover<10>{}% to ensure uncovering of the whole picture
%\fcPictureSolvingSinXequalsSinA{2}{240}{$-\frac{\sqrt{3}}{2}$}{$240^\circ$}{$300^\circ$}
\newcommand{\fcPictureSolvingSinXequalsSinA}[5]{
\begin{pspicture}(-1.2, -1.2)(1.2, 1.2)
\tiny
\pstVerb{25 dict begin}
\pstVerb{%
/theBaseAngle #2\space def
/theOtherAngle theBaseAngle 360 mod 180 gt{540 theBaseAngle sub}{180 theBaseAngle sub}ifelse def
/theYcoord theBaseAngle sin def
/theXbase theBaseAngle cos def
/theXother theOtherAngle cos def
}%
\fcAxesStandard{-1.2}{-1.2}{1.2}{1.2}%
\parametricplot[linecolor=\fcColorGraph]{0}{360}{t cos t sin}%
\uncover<#1->{%
\psline[linecolor=blue, linewidth=2pt](0,0)(! 0 theYcoord )%
\rput[l](! 0 theYcoord 2 div){#3}%
}%
%\uncover<#3, \the\numexpr#3+1\relax>{\psline[linecolor=blue, linewidth=2pt](0,0)(! 0 theYcoord)}%
\uncover<handout:0|\the\numexpr#1+4\relax>{%
\parametricplot[linecolor=blue, linewidth=2pt, plotpoints=500, arrows=->]{0}{theBaseAngle 360 add}{t cos 0.35 t 3000 div add mul t sin 0.35 t 3000 div add mul}%
}%
\uncover<\the\numexpr#1+2\relax,\the\numexpr#1+3\relax>{\parametricplot[arrows=->, linecolor=purple, linewidth=2pt]{0}{theBaseAngle }{t cos 0.3 mul t sin 0.3 mul}}%
\uncover<\the\numexpr#1+2\relax->{%
%\fcPerpendicular{[theXbase theYcoord]}{[0 1]}{0.1}%
%\fcPerpendicular{[theXother theYcoord]}{[0 1]}{0.1}%
\psline[arrows=->](0,0)(! theXbase 1.2 mul theYcoord 1.2 mul)%
\parametricplot[arrows=->, linecolor=purple]{0}{theBaseAngle }{t cos 0.3 mul t sin 0.3 mul}%
\rput(! theBaseAngle 2 div dup cos 0.4 mul exch sin 0.4 mul){\fcAnswer{\the\numexpr#1+3\relax}{#4}}%
}%
\uncover<handout:0|\the\numexpr#1+7\relax>{%
\parametricplot[linecolor=brown, linewidth=2pt, plotpoints=500, arrows=->]{0}{theOtherAngle 360 add}{t cos 0.55 t 3000 div add mul t sin 0.55 t 3000 div add mul}%
}%
\uncover<\the\numexpr#1+5\relax,\the\numexpr#1+6\relax>{\parametricplot[arrows=->, linewidth=2pt, linecolor=orange]{0}{theOtherAngle}{t cos 0.5 mul t sin 0.5 mul}}%
\uncover<\the\numexpr#1+5\relax->{%
%\fcPerpendicular{[theXbase theYcoord]}{[0 1]}{0.1}%
%\fcPerpendicular{[theXother theYcoord]}{[0 1]}{0.1}%
\psline[arrows=->](0,0)(! theXother 1.2 mul theYcoord 1.2 mul)%
\parametricplot[arrows=->, linecolor=orange]{0}{theOtherAngle}{t cos 0.5 mul t sin 0.5 mul}%
\rput(! theOtherAngle 2 div dup cos 0.6 mul exch sin 0.6 mul){\fcAnswer{\the\numexpr#1+6\relax}{#5}}%
}%
\uncover<\the\numexpr#1+1\relax->{%
%\fcPerpendicular{[theXbase theYcoord]}{[0 1]}{0.1}%
%\fcPerpendicular{[theXother theYcoord]}{[0 1]}{0.1}%
\psline[linecolor=green, linewidth=2pt](! theXother theYcoord)(! theXbase theYcoord)%
\fcFullDot{theXother}{theYcoord}%
\fcFullDot{theXbase}{theYcoord}%
}%
\pstVerb{end}%
\end{pspicture}
}

%Produces a picture which illustrates graphically the solution of
%cos x = cos a (a=known angle).
%argument1: the number of the first frame
%argument2: the angle a
%argument3: the label of the segment representing sin a
%argument4: the label of the angle a
%argument5: the label of the other angle that solves the problem
\newcommand{\fcPictureSolvingCosXequalsCosA}[5]{
\begin{pspicture}(-1.2, -1.2)(1.2, 1.2)
\tiny
\pstVerb{25 dict begin}
\pstVerb{%
/theBaseAngle #2\space def
/theOtherAngle theBaseAngle -1 mul def
/theXcoord theBaseAngle cos def
/theYbase theBaseAngle sin def
/theYother theOtherAngle sin def
}%
\fcAxesStandard{-1.2}{-1.2}{1.2}{1.2}%
\parametricplot[linecolor=\fcColorGraph]{0}{360}{t cos t sin}%
\uncover<#1->{%
\psline[linecolor=green, linewidth=2pt](0,0)(! theXcoord 0)%
\rput[t](! theXcoord 2 div -0.02){#3}%
}%
%\uncover<#3, \the\numexpr#3+1\relax>{\psline[linecolor=blue, linewidth=2pt](0,0)(! 0 theYcoord)}%
\uncover<handout:0|\the\numexpr#1+4\relax>{%
\parametricplot[linecolor=blue, linewidth=2pt, plotpoints=500, arrows=->]{0}{theBaseAngle 360 add}{t cos 0.35 t 3000 div add mul t sin 0.35 t 3000 div add mul}%
}%
\uncover<\the\numexpr#1+2\relax,\the\numexpr#1+3\relax>{\parametricplot[arrows=->, linecolor=purple, linewidth=2pt]{0}{theBaseAngle }{t cos 0.3 mul t sin 0.3 mul}}%
\uncover<\the\numexpr#1+2\relax->{%
%\fcPerpendicular{[theXbase theYcoord]}{[0 1]}{0.1}%
%\fcPerpendicular{[theXother theYcoord]}{[0 1]}{0.1}%
\psline[arrows=->](0,0)(! theXcoord 1.2 mul theYbase 1.2 mul)%
\parametricplot[arrows=->, linecolor=purple]{0}{theBaseAngle }{t cos 0.3 mul t sin 0.3 mul}%
\rput(! theBaseAngle 2 div dup cos 0.4 mul exch sin 0.4 mul){\fcAnswer{\the\numexpr#1+3\relax}{#4}}%
}%
\uncover<handout:0|\the\numexpr#1+7\relax>{%
\parametricplot[linecolor=brown, linewidth=2pt, plotpoints=500, arrows=->]{0}{theOtherAngle 360 sub}{t cos 0.55 t 3000 div add mul t sin 0.55 t 3000 div add mul}%
}%
\uncover<handout:0|\the\numexpr#1+8\relax>{%
\parametricplot[linecolor=brown, linewidth=2pt, plotpoints=500, arrows=->]{0}{theOtherAngle 360 add}{t cos 0.55 t 3000 div add mul t sin 0.55 t 3000 div add mul}%
}%
\uncover<\the\numexpr#1+5\relax,\the\numexpr#1+6\relax>{\parametricplot[arrows=->, linewidth=2pt, linecolor=orange]{0}{theOtherAngle}{t cos 0.5 mul t sin 0.5 mul}}%
\uncover<\the\numexpr#1+5\relax->{%
%\fcPerpendicular{[theXbase theYcoord]}{[0 1]}{0.1}%
%\fcPerpendicular{[theXother theYcoord]}{[0 1]}{0.1}%
\psline[arrows=->](0,0)(! theXcoord 1.2 mul theYother 1.2 mul)%
\parametricplot[arrows=->, linecolor=orange]{0}{theOtherAngle}{t cos 0.5 mul t sin 0.5 mul}%
\rput(! theOtherAngle 2 div dup cos 0.63 mul exch sin 0.63 mul){\fcAnswer{\the\numexpr#1+6\relax}{#5}}%
}%
\uncover<\the\numexpr#1+1\relax->{%
%\fcPerpendicular{[theXbase theYcoord]}{[0 1]}{0.1}%
%\fcPerpendicular{[theXother theYcoord]}{[0 1]}{0.1}%
\psline[linecolor=blue, linewidth=2pt](! theXcoord theYother)(! theXcoord theYbase)%
\fcFullDot{theXcoord}{theYother}%
\fcFullDot{theXcoord}{theYbase}%
}%
\pstVerb{end}%
\end{pspicture}
}

%
%  An example of logarithmic differentiation of a function with a
%  variable base and exponent.
%  #1 is the base.
%  #2 is the exponent.
%  #3 is the derivative of the natural logarithm of the base.
%  #4 is the derivative of the exponent.
%  #5 is (base)(exponent)' + (exponent)(base)' after simplification.
%
\newcommand{\logdiffbaseandexp}[5]{
\begin{example}[Variable base and exponent]
\abovedisplayskip=0pt
\belowdisplayskip=0pt
\abovedisplayshortskip=0pt
\belowdisplayshortskip=0pt
\begin{align*}
\text{Differentiate}\quad \alertNoH{ 13}{y} %
 & \alertNoH{ 13}{=} %
\alertNoH{ 13}{%
#1^{#2}%
}.%
\uncover<2->{%
\intertext{
Take logarithms of both sides:%
}
}%
\uncover<2->{%
\ln y
}%
 & \uncover<2->{ = } %
\uncover<2->{%
\ln #1^{\alertNoH{ 3}{#2}}%
}\\%
\uncover<3->{%
\alertNoH{ 4-5}{\ln y}%
}%
 & \uncover<3->{ = } %
\uncover<3->{%
\alertNoH{ 6-7}{%
\alertNoH{ 3}{#2} \ln #1%
}.}%
\uncover<4->{%
\intertext{
Differentiate implicitly with respect to $x$:%
}%
}%
\fcAnswer{5}{\frac{1}{y} y'}%
 & \uncover<4->{ = } %
\fcAnswerUncover{4}{7}{%
\left( #2 \right) \alertNoH{ 8-9}{\frac{\diff}{\diff x} \left( \ln #1 \right)} + \left( \ln #1 \right)\alertNoH{ 10-11}{\frac{\diff}{\diff x}\left( #2 \right)} %
}\\%
\uncover<8->{%
\frac{1}{\alertNoH{12}{y}} y'%
}%
 & \uncover<8->{ = } %
\uncover<8->{%
( #2 ) \alertNoH{8-9}{\left( \fcAnswerUncover{8}{9}{ #3 }\right)} + \left( \ln #1 \right) \alertNoH{ 10-11}{ \left( \fcAnswerUncover{8}{11}{ #4 } \right) }
}\\%
\uncover<12->{%
y'%
}%
 & \uncover<12->{ = } %
\uncover<12->{%
\alertNoH{ 12-13}{y} \left( #5 \right)%
}\\%
 & \uncover<13->{ = } %
\uncover<13->{%
\alertNoH{ 13}{#1^{#2}} \left( #5 \right).%
}%
\end{align*}
\end{example}
}


%
%  An example of logarithmic differentiation of a function.
%  It looks as follows:
%
%  Differentiate y = (#1 #2)/#3.
%  Take logarithms of both sides:
%  ln y = ln((#1 #2)/#3)
%  ln y = ln#1 + ln#2 - ln#3
%  ln y = #4 + #5 - #6
%  Differentiate implicitly with respect to x:
%  (1/y)y' = #7 + #8 - #9
%  y' = y(#7 + #8 - #9)
%  y' = ((#1 #2)/#3)(#7 + #8 - #9)
%
\newcommand{\logdifftwouponedown}[9]{
\begin{example}[Logarithmic Differentiation%
]
\abovedisplayskip=0pt
\belowdisplayskip=0pt
\abovedisplayshortskip=0pt
\belowdisplayshortskip=0pt
\begin{align*}
\text{Differentiate}\quad \alertNoH{ 18}{y} %
 & \alertNoH{ 18}{=} %
\alertNoH{ 18}{%
\frac{#1 #2}{#3}%
}.%
\uncover<2->{%
\intertext{
Take logarithms of both sides:%
}
}%
\uncover<2->{%
\ln y
}%
 & \uncover<2->{ = } %
\uncover<2->{%
\ln \frac{\alertNoH{ 3-4}{#1}\alertNoH{ 5-6}{#2}}{\alertNoH{ 7-8}{#3}}%
}\\%
\uncover<2->{%
\ln y
}%
 & \uncover<2->{ = } %
\uncover<2->{%
\ln \alertNoH{ 3-4}{#1} + \ln \alertNoH{ 5-6}{#2} -  \ln \alertNoH{ 7-8}{#3}%
}\\%
\uncover<3->{%
\alertNoH{ 9-10}{\ln y}%
}%
 & \uncover<3->{ = } %
\uncover<3->{%
\alertNoH{ 3-4,11-12}{%
\left( \uncover<4->{#4}\right) %
}%
\alertNoH{ 5-6}{%
\uncover<6->{+} \alertNoH{ 13-14}{\left( \uncover<6->{#5}\right)} %
}%
\alertNoH{ 7-8}{%
\uncover<8->{-} \alertNoH{ 15-16}{\left( \uncover<8->{#6}\right)} %
}%
}%
\uncover<9->{%
\intertext{
Differentiate implicitly with respect to $x$:%
}%
}%
\uncover<10->{%
\alertNoH{ 10}{\frac{1}{\alertNoH{ 17}{y}} y'}%
}%
 & \uncover<9->{ = } %
\uncover<9->{%
\alertNoH{ 11-12}{\left( \uncover<12->{#7} \right)} + %
\alertNoH{ 13-14}{\left( \uncover<14->{#8} \right)} - %
\alertNoH{ 15-16}{\left( \uncover<16->{#9} \right)} %
}\\%
\uncover<17->{%
y'%
}%
 & \uncover<17->{ = } %
\uncover<17->{%
\alertNoH{ 17-18}{y} \left( #7 + #8 - #9 \right)%
}\\%
 & \uncover<18->{ = } %
\uncover<18->{%
\alertNoH{ 18}{\frac{#1 #2}{#3}} \left( #7 + #8 - #9 \right)%
}%
\end{align*}
\end{example}
}


%
%  An example of a derivative with the Product Rule, using the symbol f(x).
%  It looks as follows:
%
%  Differentiate f(x) = #1 #2.
%  Product Rule: f'(x) = (#1)(d/dx)(#2) + (#2)(d/dx)(#1)
%   = (#1)(#4) + (#2)(#3)
%   = #5.
%
%  #6 appears in the subtitle of the example.
%
\newcommand{\productrulefofx}[6]{%
\begin{example}[Product Rule%
\ifthenelse{\equal{#6}{0}}%
{}%
{, #6}%
]%
\abovedisplayskip=0pt
\belowdisplayskip=0pt
\abovedisplayshortskip=0pt
\belowdisplayshortskip=0pt
\begin{align*}
\text{Differentiate}\quad f(x) & = \alertNoH{2}{ #1}\alertNoH{3}{ #2.}\\%
\uncover<2->{%
\text{Product Rule:}\quad f'(x)%
}%
& \uncover<2->{%
 =  \alertNoH{ 6-7}{\frac{\diff}{\diff x}\left( \alertNoH{2}{#1} \right)}\left( \alertNoH{3}{#2} \right)+\left( \alertNoH{2}{#1} \right) \alertNoH{ 4-5}{\frac{\diff}{\diff x}\left( \alertNoH{3}{#2} \right)} %
}\\%
& \uncover<4->{%
 = \alertNoH{ 6-7}{\left( \fcAnswerUncover{4}{7}{#3} \right)}\left( #2 \right)+ \left( #1 \right) \alertNoH{ 4-5}{\left(\fcAnswer{5}{ #4 }\right)}  %
}\\%
& \uncover<8->{%
 = #5.%
}%
\end{align*}
\end{example}
}


%
%  An example of a derivative with the Constant Multiple Rule.
%  It looks as follows:
%
%  Find the derivative of #1 = #2.
%   #1 = (#3)(#4).
%   d#1/dx = (d/dx)((#3)(#4))
% Constant Multiple Rule: = (#3)(d/dx)(#4)
%   = (#3)(#5)
%   = #6.
%
%  #7 appears in the subtitle of the example.
%
\newcommand{\constantmultiple}[7]{%
\begin{example}[Constant Multiple Rule%
\ifthenelse{\equal{#7}{0}}%
{}%
{, #7}%
]%
\abovedisplayskip=0pt
\belowdisplayskip=0pt
\abovedisplayshortskip=0pt
\belowdisplayshortskip=0pt
\begin{align*}
\text{Find the derivative of}\quad #1 & = #2.\\%
\uncover<2->{%
#1 %
}%
& \uncover<2->{%
 = \left( #3\right)\left( #4\right).
}\\%
\uncover<3->{%
\frac{\diff #1}{\diff x} %
}%
& \uncover<3->{%
 = \frac{\diff}{\diff x}\left[ \alertNoH{ 4}{\left( #3\right)}\left( #4\right)\right]
}\\%
\uncover<4->{%
\text{Constant Multiple Rule:}\quad %
}%
& \uncover<4->{%
 =  \alertNoH{ 4}{\left( #3\right)}\alertNoH{ 5-6}{\frac{\diff}{\diff x}\left( #4\right)}
}\\%
& \uncover<5->{%
 =  \left( #3\right)\alertNoH{ 5-6}{\left( \fcAnswer{6}{#5}\right)}
}\\%
& \uncover<7->{%
 =  #6.
}%
\end{align*}
\end{example}
}


%
%  An example of a derivative with the Quotient Rule, using the symbol y.
%  It looks as follows:
%
%  Differentiate y = #1 / #2.
%  Quotient Rule: dy/dx = ((#2)(d/dx)(#1)-(#1)(d/dx)(#2))/(#2)^2
%   = ((#2)(#3)-(#1)(#4))/(#2)^2
%   = #5
%   = #6.
%
%  #7 appears in the subtitle of the example.
%
\newcommand{\quotientruley}[7]{%
\begin{example}[Quotient Rule%
\ifthenelse{\equal{#7}{0}}%
{}%
{, #7}%
]%
\abovedisplayskip=0pt
\belowdisplayskip=0pt
\abovedisplayshortskip=0pt
\belowdisplayshortskip=0pt
\begin{align*}
\text{Differentiate}\quad y & = \frac{\alertNoH{2}{ #1}}{\alertNoH{3}{#2}}.%
\uncover<2->{%
\intertext{Quotient Rule:}%
}%
%&\\%
\uncover<2->{%
\frac{\diff y}{\diff x}%
}%
& \uncover<2->{%
 = \frac%
{ \alertNoH{ 4-5}{\frac{\diff}{\diff x}\left( \alertNoH{2}{ #1} \right)}\left( \alertNoH{3}{#2} \right) - \left( \alertNoH{2}{#1} \right) \alertNoH{ 6-7}{\frac{\diff}{\diff x}\left( \alertNoH{3}{#2} \right)}}%
{\left( \alertNoH{3}{#2}\right)^2}%
}\\%
& \uncover<4->{%
 = \frac%
{\alertNoH{ 4-5}{\left(\fcAnswer{5}{ #3 }\right)}\left( #2 \right)  - \left( #1 \right) \alertNoH{ 6-7}{\left( \fcAnswerUncover{4}{7}{#4} \right)}}%
{\left( #2\right)^2}%
}\\%
& \uncover<8->{%
 = #5%
}\\%
& \uncover<9->{%
 = #6.%
}%
\end{align*}
\end{example}
}

%
%  An example of an indefinite integral with the Substitution Rule.
%  It looks as follows:
%
%  Find \int (#1, with nothing substituted for UU and VV).
%  Let u = #2
%  Then du = #3.
%  Therefore #4 = #5.
%  Substitute: \int (#1, with the alert command for u and du
%          substituted for UU and VV respectively)
%  = \int (#6, with the alert command for u and du substituted for UU and VV)
%  = (#7, with u substituted for UU) + C
%  = (#8, with #2 substituted for UU) + C
%
%  #9 appears in the subtitle of the example.
%
\newcommand{\subrule}[9]{%
\begin{example}[Substitution Rule%
\ifthenelse{\equal{#9}{0}}%
{}%
{, #9}%
]%
\abovedisplayskip=0pt
\belowdisplayskip=0pt
\abovedisplayshortskip=0pt
\belowdisplayshortskip=0pt
\begin{align*}
\text{Find}\quad \int %
 \noexpandarg\exploregroups\StrSubstitute{\StrSubstitute{#1}{UU}{3}}{VV}{6-7}\noexploregroups\expandarg. & \\%
\uncover<2->{%
\text{Let}\quad\alertNoH{2-3, 8, 14}{u}%
}%
& \uncover<2->{%
\alertNoH{ 2-3,8,14}{ = \fcAnswer{3}{#2.}}%
}\\%
\uncover<4->{%
\text{Then}\quad \alertNoH{ 4-5,7}{\diff u}%
}%
& \uncover<4->{%
\alertNoH{ 4-5}{ = \fcAnswer{5}{#3}}%
}\\%
\uncover<6->{%
\alertNoH{ 6-7,10}{#4}%
}%
& \uncover<6->{%
\alertNoH{ 6-7,10}{ = \fcAnswer{7}{#5.}}%
}\\%
\uncover<8->{%
\text{Substitute:}\quad \int%
 \noexpandarg\exploregroups\StrSubstitute{\StrSubstitute{#1}{UU}{8}}{VV}{9,10}\noexploregroups\expandarg}%
& \uncover<8->{= \alertNoH{ 11-12}{\int\noexpandarg\exploregroups\StrSubstitute{\StrSubstitute{#6}{UU}{8}}{VV}{10}\noexploregroups\expandarg %
}}\\%
& \uncover<11->{%
= \fcAnswer{12}{\noexpandarg\exploregroups \StrSubstitute{#7}{UU}{\alertNoH{ 14}{u}}\noexploregroups\expandarg} \uncover<13->{\alertNoH{ 13}{+C}}%
}\\%
& \uncover<14->{%
 = \noexpandarg\exploregroups \StrSubstitute{#8}{UU}{\alertNoH{14}{#2}}\noexploregroups\expandarg +C.%
}%
\end{align*}
\end{example}
}

%
%  An example of a definite integral with the Substitution Rule.
%  There are nine arguments to the function.  The ninth is a string of four
%  groups of the form {AA}{BB}{CC}{DD} where AA is the lower limit of
%  integration, BB is the upper limit of integration, CC is the lower limit
%  of integration with respect to u, and DD is the upper limit of integration
%  with respect to u.
%  It looks as follows:
%
%  Find \int_{AA}^{BB} (#1, with nothing substituted for UU and VV).
%  Let u = #2
%  Then du = #3.
%  #4 = #5.
%  When x = AA, u = CC.
%  When x = BB, u = DD.
%  Substitute: \int_{AA}^{BB} (#1, with the alert command for u and du
%          substituted for UU and VV respectively)
%  = \int_{CC}^{DD} (#6, with the alert command for u and du substituted for UU and VV)
%  = [#7, with u substituted for UU]_{CC}^{DD}
%  = #8.
%
%
\newcommand{\subruledefbounds}[9]{%
\begin{example}[Substitution Rule, Definite Integral%
]%
\abovedisplayskip=0pt
\belowdisplayskip=0pt
\abovedisplayshortskip=0pt
\belowdisplayshortskip=0pt
\begin{align*}
\text{Find}\quad \int%
_{\StrMid{#9}{1}{1}}%
^{\StrMid{#9}{2}{2}} %
 \noexpandarg\exploregroups\StrSubstitute{\StrSubstitute{#1}{UU}{3}}{VV}{6-7}\noexploregroups\expandarg. & \\%
\uncover<2->{%
\text{Let}\quad\alertNoH{ 2-3,8-12}{u}%
}%
& \uncover<2->{%
\alertNoH{ 2-3,8-12}{ = \uncover<3->{#2.}}%
}\\%
\uncover<4->{%
\text{Then}\quad \alertNoH{ 4-5}{\diff u}%
}%
& \uncover<4->{%
\alertNoH{ 4-5}{ = \uncover<5->{#3}}%
}\\%
\uncover<6->{%
\alertNoH{ 6-7,13}{#4}%
}%
& \uncover<6->{%
\alertNoH{ 6-7,13}{ = \uncover<7->{#5.}}%
}\\%
\uncover<8->{%
\alertNoH{ 8-9,14}{\text{When } x = \StrMid{#9}{1}{1}, \quad u }%
}%
& \uncover<8->{%
\alertNoH{ 8-9,14}{ = \uncover<9->{\StrMid{#9}{3}{3}.}}%
}\\%
\uncover<10->{%
\alertNoH{ 10-11,15}{\text{When } x = \StrMid{#9}{2}{2}, \quad u }%
}%
& \uncover<10->{%
\alertNoH{ 10-11,15}{ = \uncover<11->{\StrMid{#9}{4}{4}.}}%
}\\%
\uncover<12->{%
\text{Substitute:}\quad \int%
_{\alertNoH{ 14}{\StrMid{#9}{1}{1}}}%
^{\alertNoH{ 15}{\StrMid{#9}{2}{2}}} %
 \noexpandarg\exploregroups\StrSubstitute{\StrSubstitute{#1}{UU}{12}}{VV}{13}\noexploregroups\expandarg}%
& \uncover<12->{= \alertNoH{ 16-17}{{\int}%
_{\uncover<14->{\alertNoH{ 14}{
\StrMid{#9}{3}{3}}}}%
^{\uncover<15->{
\alertNoH{ 15}{
\StrMid{#9}{4}{4}}}} %
\noexpandarg\exploregroups\StrSubstitute{\StrSubstitute{#6}{UU}{12}}{VV}{13}\noexploregroups\expandarg %
}}\\%
& \uncover<16->{\alertNoH{ 16-17}{%
 = {\left[ \uncover<17->{%
\noexpandarg\exploregroups\StrSubstitute{#7}{UU}{u}\noexploregroups\expandarg %
}\right]}_{\StrMid{#9}{3}{3}}^{\StrMid{#9}{4}{4}}%
}}\\%
& \uncover<18->{%
 = #8.
}%
\end{align*}
\end{example}
}


%
%  An example of a definite integral with the Substitution Rule.
%  There are nine arguments to the function.  The ninth is a string of two
%  groups of the form {AA}{BB} where AA is the lower limit of
%  integration and BB is the upper limit of integration.
%  It looks as follows:
%
%  Find \int_{AA}^{BB} (#1, with nothing substituted for UU and VV).
%  Let u = #2
%  Then du = #3.
%  #4 = #5.
%  Substitute: \int (#1, with the alert command for u and du
%          substituted for UU and VV respectively)
%  = \int (#6, with the alert command for u and du substituted for UU and VV)
%  = #7, with u substituted for UU
%  = #8.
%  Therefore int_{AA}^{BB} (#1, with nothing substituted for UU and VV)
%      = [#8]_{AA}^{BB}
%  = #9.
%
%
\newcommand{\subruledefvar}[9]{%
\begin{example}[Substitution Rule, Definite Integral%
]%
\abovedisplayskip=0pt
\belowdisplayskip=0pt
\abovedisplayshortskip=0pt
\belowdisplayshortskip=0pt
\begin{align*}
\text{Find}\quad \int%
_{\StrMid{#9}{1}{1}}%
^{\StrMid{#9}{2}{2}} %
 \noexpandarg\exploregroups\StrSubstitute{\StrSubstitute{#1}{UU}{3}}{VV}{6-7}\noexploregroups\expandarg. & \\%
\uncover<2->{%
\text{Let}\quad\alertNoH{ 2-3,8,12}{u}%
}%
& \uncover<2->{%
\alertNoH{ 2-3,8,12}{ = \uncover<3->{#2.}}%
}\\%
\uncover<4->{%
\text{Then}\quad \alertNoH{ 4-5}{\diff u}%
}%
& \uncover<4->{%
\alertNoH{ 4-5}{ = \uncover<5->{#3}}%
}\\%
\uncover<6->{%
\alertNoH{ 6-7,9}{#4}%
}%
& \uncover<6->{%
\alertNoH{ 6-7,9}{ = \uncover<7->{#5.}}%
}\\%
\uncover<8->{%
\text{Substitute:}\quad \int%
 \noexpandarg\exploregroups\StrSubstitute{\StrSubstitute{#1}{UU}{8}}{VV}{9}\noexploregroups\expandarg}%
& \uncover<8->{= \alertNoH{ 10-11}{{\int}%
\noexpandarg\exploregroups\StrSubstitute{\StrSubstitute{#6}{UU}{8}}{VV}{9}\noexploregroups\expandarg %
}}\\%
& \uncover<10->{%
 \alertNoH{ 10-11}{ = \uncover<11->{%
\noexpandarg\exploregroups{\StrSubstitute{#7}{UU}{\alertNoH{ 12}{u}}}\noexploregroups\expandarg%
}}%
  \uncover<12->{%
 = \noexpandarg\exploregroups{\StrSubstitute{#7}{UU}{\alertNoH{ 12}{#2}}}\noexploregroups\expandarg.%
}%
}\\%
\uncover<13->{%
\text{Therefore}\quad \int%
_{\StrMid{#9}{1}{1}}%
^{\StrMid{#9}{2}{2}} %
 \noexpandarg\exploregroups\StrSubstitute{\StrSubstitute{#1}{UU}{0}}{VV}{0}\noexploregroups\expandarg}%
& \uncover<13->{%
 = \left[%
 \noexpandarg\exploregroups{\StrSubstitute{#7}{UU}{#2}}\noexploregroups\expandarg%
\right]%
_{\StrMid{#9}{1}{1}}%
^{\StrMid{#9}{2}{2}} %
}\\%
& \uncover<14->{%
 = #8.
}%
\end{align*}
\end{example}
}

%
%  An example of a derivative with the Chain Rule, using the symbol y.
%  It looks as follows:
%
%  Differentiate y = #1.
%  Let u = #2
%  Then y = #3
%  Chain Rule: dy/dx = (dy/du)(du/dx)
%  = (#4, with u substituted for UU)(#5)
%  = #6, with #2 substituted for UU
%
%  #7 appears in the subtitle of the example.
%
\newcommand{\chainruley}[7]{%
\begin{example}[Chain Rule%
\ifthenelse{\equal{#7}{0}}%
{}%
{, #7}%
]%
\abovedisplayskip=0pt
\belowdisplayskip=0pt
\abovedisplayshortskip=0pt
\belowdisplayshortskip=0pt
\begin{align*}
\text{Differentiate}\quad y & = #1.\\%
\uncover<2->{%
\text{Let}\quad\alertNoH{ 2-3,8-10}{u}%
}%
& \uncover<2->{%
\alertNoH{ 2-3,8-10}{ = \fcAnswer{3}{\worksheet{#2}.}}%
}\\%
\uncover<4->{%
\text{Then}\quad \alertNoH{ 6-7}{y}%
}%
& \uncover<4->{%
\alertNoH{ 6-7}{ = \fcAnswer{4}{\worksheet{#3}.}}%
}\\%
\uncover<5->{%
\text{Chain Rule:}\quad%
\frac{\diff y}{\diff x}%
}%
& \uncover<5->{%
 = \alertNoH{ 6-7}{\frac{\diff y}{\diff u}}%
\alertNoH{ 8-9}{\frac{\diff u}{\diff x}}%
}\\%
& \uncover<6->{%
 = \alertNoH{ 6-7}{\left( \fcAnswer{7}{\worksheet{  \noexpandarg\exploregroups\StrSubstitute{#4}{UU}{\alertNoH{ 10}{u}}\noexploregroups\expandarg}} \right)}%
\alertNoH{ 8-9}{\left( \fcAnswer{9}{\worksheet{#5}}\right)}%
}\\%
& \uncover<10->{ =  \worksheet{ \noexpandarg\exploregroups \StrSubstitute{#6}{UU}{\alertNoH{ 10}{#2}}.\noexploregroups\expandarg}}%
\end{align*}
\end{example}
}





%
%  An example of a derivative with the Chain Rule, using the symbol f(x).
%  It looks as follows:
%
%  Differentiate f(x) = #1.
%  Let h(x) = #2
%  Let g(x) = #3
%  Then f(x) = g(h(x))
%  f'(x) = g'(h(x))h'(x)
%  = (#4, with h(x) substituted for UU)(#5)
%  = #6, with #2 substituted for UU
%
%  #7 appears in the subtitle of the example.
%
\newcommand{\chainrulefofx}[7]{%
\begin{example}[Chain Rule%
\ifthenelse{\equal{#7}{0}}%
{}%
{, #7}%
]%
\abovedisplayskip=0pt
\belowdisplayskip=0pt
\abovedisplayshortskip=0pt
\belowdisplayshortskip=0pt
\begin{align*}
\text{Differentiate}\quad f(x) & = #1.\\%
\uncover<2->{%
\text{Let}\quad\alertNoH{ 2-3,9-11}{h(x)}%
}%
& \uncover<2->{%
\alertNoH{ 2-3,9-11}{ = \fcAnswerNoH{3}{#2.}}%
}\\%
\uncover<2->{%
\text{Let}\quad\alertNoH{ 4-5,7-8}{g(x)}%
}%
& \uncover<2->{%
\alertNoH{ 4-5,7-8}{ = \fcAnswerUncover{2}{5}{#3.}}%
}\\%
\uncover<2-| handout:0>{%
\text{Then}\quad f(x)%
}%
& \uncover<2-| handout:0>{%
 = g(h(x)).%
}\\%
\uncover<6-| handout:0>{%
\text{Chain Rule:}\quad%
f'(x)%
}%
& \uncover<6-| handout:0>{%
 = \alertNoH{ 7-8}{g'(h(x))}%
\alertNoH{ 9-10}{h'(x)}%
}\\%
& \uncover<7-| handout:0>{%
=}\uncover<7-| handout:0>{\alertNoH{ 7-8}{\left( \fcAnswerNoH{8}{\noexpandarg\exploregroups\StrSubstitute{#4}{UU}{\alertNoH{ 11}{h(x)}}\noexploregroups\expandarg}\right)}%
\alertNoH{ 9-10}{\left( \fcAnswerUncoverNoH{7}{10}{#5}\right)}%
}\\%
& \uncover<11-| handout:0>{=} \uncover<11-| handout:0>{%
 \noexpandarg \exploregroups \StrSubstitute{#6}{UU}{\alertNoH{ 11}{#2}}.\noexploregroups \expandarg%
}%
\end{align*}
\end{example}
}

%
%  Similar to chainrulefofx but in different style.
%  It looks as follows:
%
%  Recall the chain rule (...).
%******************************
%  Differentiate f(x) = #1.
%  h(x) = #2
%  Let g(u) = #3
%  Then g'(u)=#4
%  Then f(x) = g(u)
%  f'(x) = g'(u)h'(x)
%  = (#4, with h(x) substituted for UU)(#5)
%  = #6, with #2 substituted for UU
%
%  #7 appears in the subtitle of the example.
%
\newcommand{\chainruleStyleOne}[7]{%
{\renewcommand{\arraystretch}{1.2}
$
\begin{array}{rclll}
\alertNoH{1-}{\left(g(h(x))\right)'}&\alertNoH{1-}{=}&\alertNoH{1-}{g'(h(x))\cdot  h'(x)}&& \text{(notation 1)} {~~~~~~~~~~~~~~~~~~~~~~~~~~~~~~~~~~~~} \\
(g(u))'&\alertNoH{0}{=}&g'(u) u'&\text{where } u=h(x)& \text{(notation 2)}\\
\displaystyle\frac{\diff y}{\diff x} &\alertNoH{0}{=}& \displaystyle\frac{\diff y}{\diff u}  \frac{\diff u}{\diff x} &\text{where } y=g(u)& \text{(notation 3)}\quad.\\
\end{array}
$
}
\begin{example}[Chain Rule, Notation 1%
\ifthenelse{\equal{#7}{0}}%
{}%
{, #7}%
]%
\[
\begin{array}{rrcl}
\text{Differentiate } & f(x) & =& #1.\\%
\uncover<2->{%
\text{Let}&\alertNoH{2-3,9-11}{h(x)}%
}%
&\uncover<2-| handout:0>{\alertNoH{2-3, 9-11}{ = }} &\displaystyle \uncover<2-| handout:0>{%
\alertNoH{2-3,9-11}{ \fcAnswerNoH{3}{#2.}}%
}\\%
\uncover<2->{%
\text{Let}&\alertNoH{4-5,7-8}{g(u)}%
}
&\uncover<2->{\alertNoH{4-5,7-8}{=}}&\displaystyle
\uncover<2->{\alertNoH{4-5,7-8}{ \fcAnswerUncover{2}{5}{\uncover<5-| handout:0>{#3.}}}%
}\\%
\uncover<2-| handout:0>{%
\text{Then}& f(x)
}%
&\uncover<2-| handout:0>{{=}}&\uncover<2-| handout:0>{%
 g(h(x)).%
}\\%
\uncover<6->{%
\text{Chain Rule:} &
f'(x)%
}%
&\uncover<6->{=}& \uncover<6->{%
 \alertNoH{ 7-8}{g'(h(x))}%
\alertNoH{ 9-10}{h'(x)}%
}\\%
&&\uncover<7->{=}& \displaystyle
\uncover<7->{\alertNoH{ 7-8}{ \left( \fcAnswerUncoverNoH{7}{8}{\noexpandarg \exploregroups \StrSubstitute{#4}{UU}{\alertNoH{ 11}{h(x)}} \noexploregroups\expandarg}\right)}%
\alertNoH{ 9-10}{\left( \fcAnswerUncoverNoH{7}{10}{#5}\right)}%
}\\%
&&\uncover<11-| handout:0>{=}&\displaystyle \uncover<11-| handout:0>{%
 \noexpandarg \exploregroups \StrSubstitute{#6}{UU}{\alertNoH{ 11}{#2}}.\noexploregroups \expandarg%
}%
\end{array}
\]
\end{example}
}

%
%  Similar to chainrulefofx but in different style.
%  It looks as follows:
%
%  Recall the chain rule (...).
%******************************
%  Differentiate f(x) = #1.
%  Let u= #2
%  Let g(u) = #3
%  Then g'(u)=#4
%  Then f(x) = g(u)
%  f'(x) = g'(u)h'(x)
%  = (#4, with h(x) substituted for UU)(#5)
%  = #6, with #2 substituted for UU
%
%  #7 appears in the subtitle of the example.
%
\newcommand{\chainruleStyleTwo}[7]{%
{\renewcommand{\arraystretch}{1.2}
$
\begin{array}{rclll}
\alertNoH{0}{\left(g(h(x))\right)'}&\alertNoH{0}{=}&g'(h(x))  \cdot  h'(x)&& \text{(notation 1)} {~~~~~~~~~~~~~~~~~~~~} \\
\alertNoH{1-}{(g(u))'}&\alertNoH{1-}{=}&\alertNoH{1-}{g'(u) u'}&\text{where } u=h(x)& \text{(notation 2)}\\
\displaystyle\frac{\diff y}{\diff x} &\alertNoH{0}{=}& \displaystyle\frac{\diff y}{\diff u}  \frac{\diff u}{\diff x} &\text{where } y=g(u)& \text{(notation 3)}\quad.\\
\end{array}
$
}
\begin{example}[Chain Rule, Notation 2%
\ifthenelse{\equal{#7}{0}}%
{}%
{, #7}%
]%
\[
\begin{array}{rrcl}
\text{Differentiate } & f(x) & =& #1.\\%
\uncover<2->{%
\text{Let}&\alertNoH{2-3,9-11}{u}%
}%
&\uncover<2->{\alertNoH{2-3,9-11}{=}}&\displaystyle \uncover<2->{%
\alertNoH{2-3,9-11}{ \fcAnswerNoH{3}{#2.}}%
}\\%
\uncover<2->{%
\text{Let}&\alertNoH{4-5,7-8}{g(u)}%
}
&\uncover<2->{\alertNoH{4-5,7-8}{=}}&\displaystyle
\uncover<2->{\alertNoH{4-5,7-8}{\fcAnswerUncoverNoH{2}{5}{ #3.}}%
}\\%
\uncover<2->{%
\text{Then}& f(x)
}%
&\uncover<2->{{=}}&\uncover<2->{%
 g(u).%
}\\%
\uncover<6->{%
\text{Chain Rule:} &
f'(x)%
}%
&\uncover<6->{=}& \uncover<6->{%
 \alertNoH{ 7-8}{g'(u)}%
\alertNoH{ 9-10}{u'}%
}\\%
&& \uncover<7-|handout:0>{=}&\displaystyle \uncover<7-|handout:0>{\alertNoH{7-8}{\left( \fcAnswerUncoverNoH{7}{8}{\noexpandarg\exploregroups\StrSubstitute{#4}{UU}{\alertNoH{11}{u}}\noexploregroups\expandarg}\right)}%
\alertNoH{9-10}{\left( \fcAnswerUncoverNoH{7}{10}{#5}\right)}%
}\\%
&& \uncover<11-|handout:0>{ = }&\displaystyle \uncover<11-| handout:0>{%
 \noexpandarg \exploregroups \StrSubstitute{#6}{UU}{\alertNoH{11}{#2}}.\noexploregroups \expandarg%
}%
\end{array}
\]
\end{example}
}


%
%  Similar to chainrulefofx but in different style.
%  It looks as follows:
%
%  Recall the chain rule (...).
%******************************
%  Differentiate f(x) = #1.
%  h(x) = #2
%  Let g(u) = #3
%  Then f(x) = g(u)
%  f'(x) = g'(u)h'(x)
%  = (#4, with h(x) substituted for UU)(#5)
%  = #6, with #2 substituted for UU
%
%  #7 appears in the subtitle of the example.
%
\newcommand{\chainruleStyleThree}[7]{%
{\renewcommand{\arraystretch}{1.2}
$
\begin{array}{rclll}
\alertNoH{0}{\left(g(h(x))\right)'}&\alertNoH{0}{=}&g'(h(x))  \cdot  h'(x)&& \text{(notation 1)} {~~~~~~~~~~~~~~~~~~~~} \\
(g(u))'&\alertNoH{0}{=}&g'(u) u'&\text{where } u=h(x)& \text{(notation 2)}\\
\displaystyle\alertNoH{1-}{\frac{\diff y}{\diff x}}&\alertNoH{1-}{=}&\displaystyle\alertNoH{1-}{\frac{\diff y}{\diff u}  \frac{\diff u}{\diff x}} &\text{where } y=g(u)& \text{(notation 3)}\quad.\\
\end{array}
$
}
\begin{example}[Chain Rule, Notation 3%
\ifthenelse{\equal{#7}{0}}%
{}%
{, #7}%
]%
\[
\begin{array}{rrcl}
\text{Differentiate } & y & =& #1.\\%
\uncover<2->{%
\text{Let}&\alertNoH{2-3,9-11}{u}%
}%
&\uncover<2->{\alertNoH{2-3,9-11}{=}}& \displaystyle \uncover<2->{%
\alertNoH{2-3,9-11}{ \fcAnswerNoH{3}{#2.}}%
}\\%
\uncover<2->{%
\text{Then}&\alertNoH{4-5,7-8}{y}%
}
&\uncover<2->{\alertNoH{4-5,7-8}{=}}&\displaystyle
\uncover<2->{\alertNoH{4-5,7-8}{\fcAnswerUncoverNoH{2}{5}{ #3.}}%
}\\%
\uncover<6->{%
\text{Chain Rule:} &
\displaystyle \frac{\diff y}{\diff x}%
}%
&\uncover<6->{=}&\displaystyle  \uncover<6->{%
 \alertNoH{7-8}{\frac{\diff y}{\diff u}}%
\alertNoH{9-10}{\frac{\diff u}{\diff x}}%
}\\%
&& \uncover<7->{ =&\displaystyle  \alertNoH{7-8}{ \left( \fcAnswerUncoverNoH{7}{8}{\noexpandarg \exploregroups \StrSubstitute{#4}{UU}{\alertNoH{ 11}{u}} \noexploregroups\expandarg}\right)}%
\alertNoH{9-10}{\left( \fcAnswerUncoverNoH{7}{10}{#5}\right)}}%
\\%
&&\uncover<11->{=}&\displaystyle \uncover<11-| handout:0>{%
\noexpandarg \exploregroups \StrSubstitute{#6}{UU}{\alertNoH{ 11}{#2}}.\noexploregroups \expandarg%
}%
\end{array}
\]
\end{example}
}

%
%  An example of an infinite limit calculation.
%  There are nine arguments to the function.  The ninth is a string of six
%  plus and minus signs.  Let AA, BB, CC, DD, EE, and FF denote these plus
%  and minus signs.  Then the output of the function looks as follows:
%
%  Find lim_{x \to #1^AA} (#2, with x substituted for UU)/(#3, with x substituted for UU).
%  Plug in #1.
%  (#2, with (#1) substituted for UU)/(#3, with (#1) substituted for UU) = #4/0.
%  The numerator is non-zero and the denominator is zero.
%  Therefore the answer is DNE, infty, or -infty.
%  Factor: (#3, with x substituted for UU)/(#4, with x substituted for UU) = (#5 #6)/(#7 #8)
%  \to ((BB)(CC))/((DD)(EE))
%  = (FF).
%  Therefore lim_{x \to #1^AA} (#2, with x substituted for UU)/(#3, with x substituted for UU) = FF infty.
%
\newcommand{\infinitelimit}[9]{%
\begin{example}[Infinite Limit]%
\abovedisplayskip=0pt
\belowdisplayskip=0pt
\abovedisplayshortskip=0pt
\belowdisplayshortskip=0pt
\begin{align*}
\text{Find}\quad \lim_{x\to #1^{\StrMid{#9}{1}{1}}}
\frac%
{\noexpandarg\StrSubstitute{#2}{UU}{x}\expandarg}%
{\noexpandarg\StrSubstitute{#3}{UU}{x}\expandarg}%
& \\%
\uncover<2->{%
\text{Plug in $#1$:}\quad%
\frac%
{\alertNoH{ 2-3}{\noexpandarg\StrSubstitute{#2}{UU}{(#1)}\expandarg}}%
{\alertNoH{ 4-5}{\noexpandarg\StrSubstitute{#3}{UU}{(#1)}\expandarg}}%
}%
& \uncover<2->{= \frac{\fcAnswer{3}{#4}}{ \fcAnswerUncover{2}{5}{ 0}}}%
\uncover<6->
Therefore the answer is DNE, $\infty$, or $-\infty$.}
}%
\uncover<7->{%
\text{Factor:}\quad
}%
\uncover<7->{%
\lim_{x\to #1^{\StrMid{#9}{1}{1}}}%
\frac%
{\alertNoH{ 8-9}{\noexpandarg\StrSubstitute{#2}{UU}{x}\expandarg}}%
{\alertNoH{ 10-11}{\noexpandarg\StrSubstitute{#3}{UU}{x}\expandarg}}%
}%
& \uncover<8->{%
 = \lim_{x\to #1^{\StrMid{#9}{1}{1}}}%
\frac%
{%
\fcAnswer{9}{%
\alertNoH{ 12-13}{%
#5%
}%
\alertNoH{ 14-15}{%
#6%
}%
}%
}{%
\fcAnswerUncover{8}{11}{%
\alertNoH{ 16-17}{%
#7%
}%
\alertNoH{ 18-19}{%
#8%
}%
}%
}%
}\\%
& \uncover<12->{%
 \to \alertNoH{ 20-21}{\frac%
{%
\alertNoH{ 12-13}{( \fcAnswerUncover{12}{13}{%
\StrMid{#9}{2}{2}%
})}%
\alertNoH{ 14-15}{(\fcAnswerUncover{12}{15}{%
\StrMid{#9}{3}{3}%
})}%
}{%
\alertNoH{ 16-17}{(\fcAnswerUncover{12}{17}{%
\StrMid{#9}{4}{4}%
})}%
\alertNoH{ 18-19}{(\fcAnswerUncover{12}{19}{%
\StrMid{#9}{5}{5}%
})}%
}%
}%
}\\%
& \uncover<20->{\alertNoH{ 20-21}{ = \fcAnswer{21}{(\alertNoH{22}{ \StrMid{#9}{6}{6}})}}}\\%
\uncover<22->{%
\text{Therefore}\quad\lim_{x\to #1^{\StrMid{#9}{1}{1}}}%
\frac%
{\noexpandarg\StrSubstitute{#2}{UU}{x}\expandarg}%
{\noexpandarg\StrSubstitute{#3}{UU}{x}\expandarg}%
}%
& \uncover<22->{ = } \uncover<handout:0| 22->{ \alertNoH{ 22}{\StrMid{#9}{6}{6}}\infty.}
\end{align*}
\end{example}
}




%
%  An example of a limit calculation with factoring.
%
%  It looks as follows.
%
%  Find lim_{x \to #1} (#2, with x substituted for UU)/(#3, with x substituted for UU).
%  Plug in #1.
%  (#2, with (#1) substituted for UU)/(#3, with (#1) substituted for UU) = 0/0.
%  Zero over zero gives no information.
%  Factor: (#2, with x substituted for UU)/(#3, with x substituted for UU) = ((#4, with x substituted for UU) #6)/((#5, with x substituted for UU) #6)
%  = (#4, with x substituted for UU)/(#5, with x substituted for UU)
%  Plug in #1: = (#4, with (#1) substituted for UU)/(#5, with (#1) substituted for UU)
%  = #7
%  = #8
%
\newcommand{\limitfactor}[8]{%
\begin{example}[Limit with Factoring]%
\abovedisplayskip=0pt
\belowdisplayskip=0pt
\abovedisplayshortskip=0pt
\belowdisplayshortskip=0pt
\begin{align*}
\text{Find}\quad \lim_{x\to #1}
\frac%
{\noexpandarg\StrSubstitute{#2}{UU}{x}\expandarg}%
{\noexpandarg\StrSubstitute{#3}{UU}{x}\expandarg}%
& \\%
\uncover<2->{%
\text{Plug in $#1$:}\quad%
\frac%
{\alertNoH{2-3}{\noexpandarg\StrSubstitute{#2}{UU}{(#1)}\expandarg}}%
{\alertNoH{4-5}{\noexpandarg\StrSubstitute{#3}{UU}{(#1)}\expandarg}}%
}%
& \uncover<2->{%
= \frac%
{\fcAnswerUncoverNoH{2}{3}{0}}%
{\fcAnswerUncoverNoH{2}{5}{0}}%
}%
\uncover<6->{%
\intertext{Zero over zero is undefined, so we can't use direct substitution.}
}%
\uncover<7->{%
\text{Factor:}\quad%
\lim_{x\to #1} \frac%
{\alertNoH{8-9}{\noexpandarg\StrSubstitute{#2}{UU}{x}\expandarg}}%
{\alertNoH{10-11}{\noexpandarg\StrSubstitute{#3}{UU}{x}\expandarg}}%
}%
& \uncover<8->{%
 = \lim_{x\to #1} \frac%
{%
\fcAnswerUncoverNoH{8}{9}{%
(\noexpandarg\StrSubstitute{#4}{UU}{x}\expandarg)%
\fcCancel{12}{#6}%
}%
}{%
\fcAnswerUncoverNoH{8}{11}{%
(\noexpandarg\StrSubstitute{#5}{UU}{x}\expandarg)%
\fcCancel{12}{#6}%
}%
}%
}\\%
& \uncover<12->{%
 = \lim_{x\to #1} \frac%
{\uncover<12->{\worksheet{ \noexpandarg\StrSubstitute{#4}{UU}{ \alertNoH{ 13}{x}}\expandarg}}}%
{\uncover<12->{\worksheet{ \noexpandarg\StrSubstitute{#5}{UU}{ \alertNoH{ 13}{x}}\expandarg}}}%
}\\%
\uncover<13->{%
\text{Plug in $#1$:}\quad%
\lim_{x\to #1} \frac%
{\noexpandarg\StrSubstitute{#2}{UU}{x}\expandarg}%
{\noexpandarg\StrSubstitute{#3}{UU}{x}\expandarg}%
}%
& \uncover<13->{%
 = \frac%
{\uncover<13->{\worksheet{ \noexpandarg\StrSubstitute{#4}{UU}{( \alertNoH{ 13}{#1})} \expandarg}}}%
{\uncover<13->{\worksheet{\noexpandarg\StrSubstitute{#5}{UU}{(\alertNoH{ 13}{#1})}\expandarg}}}%
}\\%
& \uncover<14->{%
= \uncover<14->{\worksheet{#7}}%
}\\%
& \uncover<15->{%
= \uncover<14->{\worksheet{#8.}}%
}%
\end{align*}
\end{example}
}




%
%  An example of a limit calculation with a conjugate radical.
%
%  It looks as follows.
%
%  Find lim_{x \to #1} (#2, with x substituted for UU)/(#3, with x substituted for UU).
%  Plug in #1.
%  (#2, with (#1) substituted for UU)/(#3, with (#1) substituted for UU) = 0/0.
%  Zero over zero gives no information.
%  Factor: (#2, with x substituted for UU)/(#3, with x substituted for UU) = ((#4, with x substituted for UU) #6)/((#5, with x substituted for UU) #6)
%  = (#4, with x substituted for UU)/(#5, with x substituted for UU)
%  Plug in #1: = (#4, with (#1) substituted for UU)/(#5, with (#1) substituted for UU)
%  = #7
%  = #8
%
\newcommand{\limitradical}[9]{%
\begin{example}[Limit with Conjugate Radical]%
\abovedisplayskip=0pt
\belowdisplayskip=0pt
\abovedisplayshortskip=0pt
\belowdisplayshortskip=0pt
\begin{align*}
& \text{Find}\quad \lim_{x\to #1}
\frac%
{\noexpandarg\StrSubstitute{#2}{UU}{x}\expandarg}%
{\noexpandarg\StrSubstitute{#3}{UU}{x}\expandarg}%
 \\%
\uncover<2->{%
& \text{Plug in $#1$:}\quad%
\frac%
{\alertNoH{ 2-3}{\noexpandarg\StrSubstitute{#2}{UU}{(#1)}\expandarg}}%
{\alertNoH{ 4-5}{\noexpandarg\StrSubstitute{#3}{UU}{(#1)}\expandarg}}%
}%
 \uncover<2->{%
= \frac%
{\uncover<3->{\alertNoH{ 3}{0}}}%
{\uncover<5->{\alertNoH{ 5}{0}}}%
}%
\uncover<6->{%
\intertext{Zero over zero gives no information.  Use a conjugate radical.}
}%
& \uncover<7->{%
\lim_{x\to #1} \frac%
{\noexpandarg\StrSubstitute{#2}{UU}{x}\expandarg}%
{\alertNoH{ 7-8}{\noexpandarg\StrSubstitute{#3}{UU}{x}\expandarg}}%
\cdot %
\frac%
{\uncover<8->{\alert<8>{\noexpandarg\StrSubstitute{#4}{UU}{x}\expandarg}}}%
{\uncover<8->{\alert<8>{\noexpandarg\StrSubstitute{#4}{UU}{x}\expandarg}}}%
}\\%
& \uncover<9->{%
 = \lim_{x\to #1} \frac%
{(\noexpandarg\StrSubstitute{#2}{UU}{x}\expandarg)%
\left(\noexpandarg\StrSubstitute{#4}{UU}{x}\expandarg\right)}%
{#5}%
}\\%
& \uncover<10->{%
 = \lim_{x\to #1} \frac%
{(\alert<11-12>{\noexpandarg\StrSubstitute{#2}{UU}{x}\expandarg})%
\left(\noexpandarg\StrSubstitute{#4}{UU}{x}\expandarg\right)}%
{\alert<13-14>{#6}}%
}\\%
\uncover<11->{%
\text{Factor:}\quad%
}%
& \uncover<11->{%
 = \lim_{x\to #1} \frac%
{\uncover<12->{\alert<12>{(\noexpandarg\StrSubstitute{#7}{UU}{x}\expandarg)(x-#1)}}%
\left(\noexpandarg\StrSubstitute{#4}{UU}{x}\expandarg\right)}%
{\uncover<14->{\alert<14>{(\noexpandarg\StrSubstitute{#8}{UU}{x}\expandarg)(x-#1)}}}%
}\\%
& \uncover<15->{%
 = \lim_{x\to #1} \frac%
{(\noexpandarg\StrSubstitute{#7}{UU}{x}\expandarg)%
\left(\noexpandarg\StrSubstitute{#4}{UU}{x}\expandarg\right)}%
{\noexpandarg\StrSubstitute{#8}{UU}{x}\expandarg}%
}\\%
\uncover<16->{%
\text{Plug in $#1$:}\quad%
}%
& \uncover<16->{%
 = \frac%
{(\noexpandarg\StrSubstitute{#7}{UU}{(#1)}\expandarg)%
\left(\noexpandarg\StrSubstitute{#4}{UU}{(#1)}\expandarg\right)}%
{\noexpandarg\StrSubstitute{#8}{UU}{(#1)}\expandarg}%
}\\%
& \uncover<17->{%
#9.
}%
\end{align*}
\end{example}
}


%
%  An example of a limit calculation with direct substitution.
%
%  It looks as follows.
%
%  Find lim_{x \to #1} (#2, with x substituted for UU)/(#3, with x substituted for UU).
%  Plug in #1.
%  (#2, with (#1) substituted for UU)/(#3, with (#1) substituted for UU) = 0/0.
%  Zero over zero gives no information.
%  Factor: (#2, with x substituted for UU)/(#3, with x substituted for UU) = ((#4, with x substituted for UU) #6)/((#5, with x substituted for UU) #6)
%  = (#4, with x substituted for UU)/(#5, with x substituted for UU)
%  Plug in #1: = (#4, with (#1) substituted for UU)/(#5, with (#1) substituted for UU)
%  = #7
%  = #8
%
\newcommand{\limitsub}[7]{%
\begin{example}[%
\ifthenelse{\equal{#6}{0}}%
{Limit in Which Direct Substitution Doesn't Work}%
{Limit with Direct Substitution}%
]%
\abovedisplayskip=0pt
\belowdisplayskip=0pt
\abovedisplayshortskip=0pt
\belowdisplayshortskip=0pt
\begin{align*}
\text{Find}\quad \lim_{x\to #1}
\frac%
{\noexpandarg\StrSubstitute{#2}{UU}{x}\expandarg}%
{\noexpandarg\StrSubstitute{#3}{UU}{x}\expandarg}%
& \\%
\uncover<2->{%
\text{Plug in $#1$:}\quad%
\frac%
{\alertNoH{ 2-3}{\noexpandarg\StrSubstitute{#2}{UU}{(#1)}\expandarg}}%
{\alertNoH{ 4-5}{\noexpandarg\StrSubstitute{#3}{UU}{(#1)}\expandarg}}%
}%
& \uncover<2->{%
= \frac%
{\uncover<3->{\alertNoH{ 3}{#4}}}%
{\uncover<5->{\alertNoH{ 5}{#5}}}%
}\\%
\ifthenelse{\equal{#6}{0}}%
{ }%
{&}%
\uncover<6->{%
\ifthenelse{\equal{#6}{0}}%
{\intertext{Dividing by zero is undefined, so we can't use direct substitution.}}%
{ = #7.}%
}%
\ifthenelse{\equal{#6}{0}}%
{ }%
{ \text{Therefore}= #7.}%
\end{align*}
\end{example}
}



%
%  An example Newton's Method.
%
%  It looks as follows.
%
%  Starting with x_1 = #1, find the third approximation x_3 to the root of the equation #2.
%
%  f(x) = (#3, with x substituted for UU).
%  f'(x) = (#4, with x substituted for UU).
%  Newton's Method: x_{n+1} = x_n - f(x_n)/f'(x_n) = x_n - (#3, with x_n substituted for UU)/(#4, with x_n substituted for UU).
%
%  x_2 = x_1 - (#3, with x_1 substituted for UU)/(#4, with x_1 substituted for UU)     x_3 = x_2 - (#3, with x_2 substituted for UU)/(#4, with x_2 substituted for UU)
%   = (#1) - (#3, with (#1) substituted for UU)/(#4, with (#1) substituted for UU)     = (#5) - (#3, with (#5) substituted for UU)/(#4, with (#5) substituted for UU)
%  = #5.      = #6.
%
\newcommand{\newtonsmethod}[8]{%
\begin{example}[Newton's Method%
\ifthenelse{\equal{#8}{0}}%
{}%
{, #8}%
]%
\ifthenelse{\equal{#7}{0}}%
{%
Starting with $x_1 = #1$, find the third approximation $x_3$ to the root of the equation $#2$.
}%
{#7}%
\abovedisplayskip=0pt
\belowdisplayskip=10pt
\abovedisplayshortskip=0pt
\belowdisplayshortskip=0pt
\begin{align*}
\uncover<2->{%
\alertNoH{ 2-3,7}{f(x)}%
& \alertNoH{ 2-3,7}{ = \uncover<3->{\noexpandarg \exploregroups \StrSubstitute{#3}{UU}{x}.\noexploregroups \expandarg}}%
}\\%
\uncover<4->{%
\alertNoH{ 4-5,8}{f'(x)}%
& \alertNoH{ 4-5,8}{ = \uncover<5->{\noexpandarg \exploregroups \StrSubstitute{#4}{UU}{x}.\noexploregroups \expandarg}}%
}\\%
\uncover<6->{%
\text{Newton's Method:}\quad %
x_{n+1} & = x_n - \frac{\alertNoH{ 7}{f(x_n)}}{\alertNoH{ 8}{f'(x_n)}}%
}
\uncover<7->{%
 = x_n - \frac%
{\alertNoH{ 7}{\noexpandarg \exploregroups \StrSubstitute{#3}{UU}{x_n}\noexploregroups \expandarg}}%
{\alertNoH{ 8}{\uncover<8->{\noexpandarg \exploregroups \StrSubstitute{#4}{UU}{x_n}\noexploregroups \expandarg}}}%
}
\end{align*}
\begin{align*}
\uncover<9->{%
x_2 %
}%
& \uncover<9->{%
 = \alertNoH{ 10}{x_1} - \frac%
{\noexpandarg \exploregroups \StrSubstitute{#3}{UU}{\alertNoH{ 10}{x_1}}\noexploregroups \expandarg}%
{\noexpandarg \exploregroups \StrSubstitute{#4}{UU}{\alertNoH{ 10}{x_1}}\noexploregroups \expandarg}%
}%
& \uncover<12->{%
x_3 %
}%
& \uncover<12->{%
 = \alertNoH{ 13}{x_2} - \frac%
{\noexpandarg \exploregroups \StrSubstitute{#3}{UU}{\alertNoH{ 13}{x_2}}\noexploregroups \expandarg}%
{\noexpandarg \exploregroups \StrSubstitute{#4}{UU}{\alertNoH{ 13}{x_2}}\noexploregroups \expandarg}%
}\\%
& \uncover<10->{%
 = \alertNoH{ 10}{(#1)} - \frac%
{\noexpandarg \exploregroups \StrSubstitute{#3}{UU}{\alertNoH{ 10}{(#1)}}\noexploregroups \expandarg}%
{\noexpandarg \exploregroups \StrSubstitute{#4}{UU}{\alertNoH{ 10}{(#1)}}\noexploregroups \expandarg}%
}%
& %
& \uncover<13->{%
 = \alertNoH{ 13}{(#5)} - \frac%
{\noexpandarg \exploregroups \StrSubstitute{#3}{UU}{\alertNoH{ 13}{(#5)}}\noexploregroups \expandarg}%
{\noexpandarg \exploregroups \StrSubstitute{#4}{UU}{\alertNoH{ 13}{(#5)}}\noexploregroups \expandarg}%
}\\%
& \uncover<11->{%
 = #5.%
}%
& %
& \uncover<14->{%
 = #6.
}%
\end{align*}
\end{example}
}


%
%  An example of a derivative using the Chain Rule twice, using dy/dx.
%  It looks as follows:
%
%  Differentiate: y = #1.
%		  dy\dx  = d\dx(#1)
%  Chain Rule:     = (#2) (d/dx)(#3)
%  Chain Rule:     = (#2)(#4) d/dx(#5)
%  #7 [optional]    = (#2)(#3)(#6)
%                             = (#8)
%                             = (#9)    [optional]
%

\newcommand{\chainruletwice}[9]{%
\begin{example}[Using the Chain Rule twice]%
\abovedisplayskip=0pt
\belowdisplayskip=0pt
\abovedisplayshortskip=0pt
\belowdisplayshortskip=0pt
\begin{align*}
\text{Differentiate:}\quad y & = #1.\\%
\uncover<2->{\frac{\diff y}{\diff x} & = \alertNoH{3-5}{\frac{\diff}{\diff x}\left( #1\right)}}\\%
\uncover<4->{\text{Chain Rule:} \ \ \quad &= \alertNoH{4-5}{\left(\fcAnswerNoH{5}{#2} \right)\alertNoH{6-8}{\frac{\diff}{\diff x} \left(\uncover<4-| handout:0>{#3}\right)}}} \\%
\uncover<7->{\text{Chain Rule:} \ \ \quad &= \left(\uncover<7-| handout:0>{#2}\right) \alertNoH{7-8}{\left(\fcAnswerNoH{8}{#4}\right) \alertNoH{9-10}{\frac{\diff}{\diff x}\left( \uncover<7-| handout:0>{#5} \right)}}}\\%
\uncover<9->{\uncover<10->{\ifthenelse{\equal{#7}{}}{}{\text{#7 :} \ \ \quad}}& = \left(\uncover<9-| handout:0>{#2} \right) \left(\uncover<9-| handout:0>{#4}\right)\alertNoH{9-10}{\left( \fcAnswerNoH{10}{#6} \right) }} \\%
\uncover<11->{& = \uncover<11-| handout:0>{#8 \ifthenelse{\equal{#9}{}}{.}{\\}}}%
\ifthenelse{\equal{#9}{}}{}{\uncover<12->{& = \uncover<12-| handout:0>{#9.}}}
\end{align*}
\end{example}
}
 %warning this path is relative to the file that uses the \usepackage command, not relative to the style file.
\usepackage{etex, ifthen}
\providecommand{\autopstpdfConflictResolutionTemporary}{
\usepackage[
dvips={-o -Ppdf}, 
pspdf={
-dNOSAFER
%-dAutoRotatePages=/None %<-breaks in windows:%
}, 
pdfcrop={}, 
%crop=off%without crop=off breaks in windows
]{auto-pst-pdf}
}
\autopstpdfConflictResolutionTemporary
\usepackage{pst-plot}
\usepackage{pst-math}

\makeatletter
\begingroup
\catcode `P=12  % digits and punct. catcode
\catcode `T=12  % digits and punct. catcode
\lowercase{%
\def\x{\def\rem@pt##1.##2PT{##1\ifnum##2>\z@.##2\fi}}}
\expandafter\endgroup\x%
\newcommand{\stripPoints}[1]{\expandafter\rem@pt\the#1}
\makeatother

\definecolor{darkgreen}{rgb}{0,0.5,0}
\newcommand{\fcShiftX}{0}
\newcommand{\fcShiftY}{0}
\newcommand{\fcXLabel}{$x$}
\newcommand{\fcYLabel}{$y$}
\newcommand{\fcZLabel}{$z$}
\newcommand{\fcDelta}{0.5}
\newcommand{\fcZBufferNumXIntervals}{20}
\newcommand{\fcZBufferNumYIntervals}{20}
\newcommand{\fcZBufferRowColumnsUnderInvestigation}{(empty) (empty)}
\newcommand{\fcZBufferUparameterPointUnderInvestigation}{(empty) (empty)}
\newcommand{\fcZBufferVparameterPointUnderInvestigation}{(empty) (empty)}
%\newcommand{\fcContourDebugged}{-1}
\newcommand{\fcStartXIId}{0}
\newcommand{\fcStartYIId}{0}
\newcommand{\fcIterationsX}{9\space}
\newcommand{\fcIterationsY}{9\space}
\newcommand{\fcIterationsU}{9\space}
\newcommand{\fcIterationsV}{9\space}
\newcommand{\fcNumCountourSegmentsPatchU}{10\space}
\newcommand{\fcNumCountourSegmentsPatchV}{10\space}
\newcommand{\fcScreenStyle}{z}
\newcommand{\fcColorLine}{black}
\newcommand{\fcColorAngle}{red}
\newcommand{\fcUseMidpointImplicitPlots}{false\space}
\newcommand{\fcForceForeground}{false}
\newcommand{\fcLineWidth}{1}
\newcommand{\fcAngleLineWidth}{1pt}
\newcommand{\fcScale}{1\space}
\newcommand{\fcArrows}{}
\newcommand{\fcPlotPoints}{200}
\newcommand{\fcLineStyle}{0}
\newcommand{\fcDashLength}{1}
\newcommand{\fcContourOptions}{
[(\fcArrows) (->) eq (\fcArrows) (<-) eq (\fcArrows) (<->) eq or or [\fcGetColorCode{\fcColorLine}] \fcLineWidth [\fcDashes] (\fcLineStyle)]
}
\newcommand{\fcShowGridImplicitIId}{false}
\newcommand{\fcDashes}{[\fcDashLength\space \fcDashLength] 0}
\newcommand{\fcColorPatchUV}{1 0.5 0.5}
\newcommand{\fcColorPatchVU}{0.7 0.2 0.2}
\newcommand{\fcFastPatchSort}{false}
\newcommand{\fcDashesCode}{%
(\fcLineStyle) (dashed) eq (\fcLineStyle) (dashed ) eq or %
{\fcDashes\space setdash}%
{[] 0 setdash}%
ifelse\space %
}
\newcommand{\fcDashesCodeVirtual}{%
(\fcLineStyle) (dashed) eq (\fcLineStyle) (dashed ) eq or %
{\fcDashes\space setdashVirtual}%
{[] 0 setdashVirtual}%
ifelse\space %
}
\newcommand{\fcScreen}{[-1 1 -0.75] -1} %default projection plane. Renew this command to change projection plane.
\newcommand{\fcLightSource}{[2 -1 2] 1\space} %This command is not used yet, reserved for the future! Lightsource location, constant = brightness.
%Z-depth lighting description. Suppose we rescale the image linearly so that the z-depth of all points in the image vary between 0.5 and -0.5. Points that are more distant have higher z-depth (0.5 are most distant from the viewing screen). Then points at depth t are colored with [user-given rgb color triple with colors between 0 and 1] - [t t t]*fcLightDifference. Should one of the rgb numbers become negative, it is set to 0. In particular points at depth 0 are colored with the user-given patch color, points in the distance are darker and points nearer are lighter. If fcLightDifference is set to 1, then the most distant points are 0.5 units darker than the middle depth points and the points closest to the viewer are 0.5 units ligher.
\newcommand{\fcLightDifference}{0.5\space}

\newcommand{\fcMaxNumPatchesToUseShading}{1}

\newcommand{\fcSet}[1]{\setkeys{fcGraphics}{#1}}

\makeatletter %needed for define@key command.
\define@key{pstricks,pst-plot}{xLabel}[]{}
\define@key{pstricks,pst-plot}{yLabel}[]{}
\define@key{pstricks,pst-plot}{zLabel}[]{}
\define@key{fcGraphics}{Delta}[\renewcommand{\fcDelta}{1}]{\renewcommand{\fcDelta}{#1}}
\define@key{fcGraphics}{shiftX}[\renewcommand{\fcShiftX}{0}]{\renewcommand{\fcShiftX}{#1}}
\define@key{fcGraphics}{shiftY}[\renewcommand{\fcShiftY}{0}]{\renewcommand{\fcShiftY}{#1}}
\define@key{fcGraphics}{startX}[\renewcommand{\fcStartXIId}{0}]{\renewcommand{\fcStartXIId}{#1}}
\define@key{fcGraphics}{startY}[\renewcommand{\fcStartYIId}{0}]{\renewcommand{\fcStartYIId}{#1}}
\define@key{fcGraphics}{colorUV}[\renewcommand{\fcColorPatchUV}{1 0.5 0.5}]{\renewcommand{\fcColorPatchUV}{#1\space}}
\define@key{fcGraphics}{colorVU}[\renewcommand{\fcColorPatchVU}{0.7 0.2 0.2}]{\renewcommand{\fcColorPatchVU}{#1\space}}
\define@key{fcGraphics}{iterationsU}[\renewcommand{\fcIterationsU}{9\space}]{\renewcommand{\fcIterationsU}{#1\space}}
\define@key{fcGraphics}{iterationsV}[\renewcommand{\fcIterationsU}{9\space}]{\renewcommand{\fcIterationsV}{#1\space}}
\define@key{fcGraphics}{forceForeground}[\renewcommand{\fcForceForeground}{false\space}]{\renewcommand{\fcForceForeground}{#1\space}}
\define@key{fcGraphics}{showGridImplicitIId}[\renewcommand{\fcShowGridImplicitIId}{false\space}]{\renewcommand{\fcShowGridImplicitIId}{#1\space}}
\define@key{fcGraphics}{useMidpointImplicitPlots}[\renewcommand{\fcUseMidpointImplicitPlots}{false\space}]{\renewcommand{\fcUseMidpointImplicitPlots}{#1\space}}
\define@key{fcGraphics}{iterationsX}[\renewcommand{\fcIterationsX}{9\space}]{\renewcommand{\fcIterationsX}{#1\space}}
\define@key{fcGraphics}{scale}[\renewcommand{\fcScale}{1}]{\renewcommand{\fcScale}{#1\space}}
%\define@key{fcGraphics}{debugContour}[\renewcommand{\fcContourDebugged}{-1\space}]{\renewcommand{\fcContourDebugged}{#1\space}}
\define@key{fcGraphics}{iterationsY}[\renewcommand{\fcIterationsY}{9\space}]{\renewcommand{\fcIterationsY}{#1\space}}
\define@key{fcGraphics}{screenStyle}[\renewcommand{\fcScreenStyle}{z}]{\renewcommand{\fcScreenStyle}{#1}}
\define@key{fcGraphics}{xLabel}[\renewcommand{\fcXLabel}{$x$}]{\renewcommand{\fcXLabel}{#1}}
\define@key{fcGraphics}{yLabel}[\renewcommand{\fcYLabel}{$y$}]{\renewcommand{\fcYLabel}{#1}}
\define@key{fcGraphics}{zLabel}[\renewcommand{\fcZLabel}{$z$}]{\renewcommand{\fcZLabel}{#1}}
\define@key{fcGraphics}{linecolor}[\renewcommand{\fcColorLine}{black}]{\renewcommand{\fcColorLine}{#1}}
\define@key{fcGraphics}{anglecolor}[\renewcommand{\fcColorAngle}{red}]{\renewcommand{\fcColorAngle}{#1}}
\define@key{fcGraphics}{linewidth}[\renewcommand{\fcLineWidth}{1}]{\renewcommand{\fcLineWidth}{#1}}
\define@key{fcGraphics}{anglelinewidth}[\renewcommand{\fcAngleLineWidth}{1pt}]{\renewcommand{\fcAngleLineWidth}{#1}}
\define@key{fcGraphics}{linestyle}[\renewcommand{\fcLineStyle}{0}]{\renewcommand{\fcLineStyle}{#1\space}}
\define@key{fcGraphics}{fastsort}[\renewcommand{\fcFastPatchSort}{false}]{\renewcommand{\fcFastPatchSort}{#1}}
\define@key{fcGraphics}{dashes}[\renewcommand{\fcDashes}{[\fcDashLength\space \fcDashLength] 0}]{\renewcommand{\fcDashes}{#1}}
\define@key{fcGraphics}{arrows}[\renewcommand{\fcArrows}{}]{\renewcommand{\fcArrows}{#1}}
\makeatother %undoes \makeatletter.


\newcommand{\fcHollowDot}[2]{
\pscircle*[fillcolor=white, linecolor=red](! #1 #2){0.07}
\pscircle*[fillcolor=white, linecolor=white](! #1  #2){0.04}
}

\newcommand{\fcFullDot}[3][linecolor=red]{
\setkeys{fcGraphics}{#1}
\pscircle*[#1](! #2 #3){! 0.07 \fcScale\space mul}
}

\newcommand{\fcFullDotCode}{
gsave
\fcCoordsPStricksToPS [0.02 0] \fcCoordsPStricksToPS pop 0 360 arc 1 0 0 setrgbcolor fill stroke
grestore
}

\newcommand{\fcHollowDotCode}{
gsave
\fcCoordsPStricksToPS [0.04 0] \fcCoordsPStricksToPS pop 0 360 arc 0 1 0 setrgbcolor stroke
grestore
}

\newcommand{\fcHashString}{
2 dict begin
/theString exch def
/counter -1 def
0
theString length {
/counter counter 1 add def
theString counter get cvi counter 1 add mul add
}repeat
20 string cvs
end
}

\newcommand{\fcToString}{
1 dict begin
/ToString
{
5 dict begin
cvlit
/theData exch def
theData type (arraytype) eq{
([)
theData{ToString ( )} forall
(])
theData length 2 mul 2 add \fcConcatenateMultiple
}
{theData 200 string cvs}
ifelse
end
} def %
ToString
end\space
}

\newcommand{\fcConcatenate}{ %Code taken from stackexchange, many thanks!
exch dup length
2 index length add string
dup dup 4 2 roll copy length
4 -1 roll putinterval
}

\newcommand{\fcConcatenateMultiple}{ %Code taken from stackexchange, many thanks!
%Usage: (s1) (s2) (s3) ... (sN) n  \fcConcatenateMultiple  (s1s2s3...sN)
% s1 s2 s3 .. sN n                   % first sum the lengths
dup 1 add  % s1 s2 s3 .. sN n n+1
copy       % s1 s2 s3 .. sN n  s1 s2 s3 .. sN n
0 exch     % s1 s2 s3 .. sN n  s1 s2 s3 .. sN 0 n
{exch length add} repeat % s1 s2 s3 .. sN  n   len  % then allocate string
string exch          % s1 s2 s3 .. sN str   n
0 exch               % s1 s2 s3 .. sN str  off  n
-1 1 {               % s1 s2 s3 .. sN str  off  n  % copy each string
  2 add -1 roll       % s2 s3 .. sN str  off s1  % bottom to top
  3 copy putinterval  % s2 s3 .. sN str' off s1
  length add          % s2 s3 .. sN str' off+len(s1)
                      % s2 s3 .. sN str' off'
} for                               % str' off'
pop  % str'
}

\newcommand{\fcIntervalOnRealLineFromMinusInfty}[5][]{%
\pscurve[linecolor=blue, #1](! #2\space 1.3)(! #3\space -0.1 add 1.3)(! #3\space -0.1 add 1.3)(! #3 \space 0)%
\fcXTickWithLabel{#3}{#4}%
\rput[b](! #2\space #3\space add 2 div 0.2){#5}%
}%

\newcommand{\fcIntervalOnRealLineToPlusInfty}[5][]{%
\pscurve[linecolor=blue, #1](! #3\space 1.3)(! #2\space 0.1 add 1.3)(! #2\space 0.1 add 1.3)(! #2 \space 0) 
\fcXTickWithLabel{#2}{#4}
\rput[b](! #2\space #3\space add 2 div 0.2){#5}
}%

\newcommand{\fcIntervalOnRealLine}[6][]{%
\pscurve[linecolor=blue, #1](! #2\space 0)(! #2 \space 0.1 add 1.3)(! #2 \space 0.1 add 1.3)(! #3\space 0.1 sub 1.3 )(! #3\space 0.1 sub 1.3 )(! #3\space 0)
\rput[b](! #2\space #3\space add 2 div 0.2){#6}
\fcXTickWithLabel{#2}{#4}
\fcXTickWithLabel{#3}{#5}
}%

\newcommand{\fcRiemannSum}[6][linecolor=\fcColorGraph]{%
\pstVerb{%
15 dict begin
/numIntervals #5\space def
/leftEnd #2\space def
/rightEnd #3\space def
/pointRatio #6\space def
/DeltaX rightEnd leftEnd sub numIntervals div def
/function {#4} def
}%
\multido{\ra=0+1}{#5}{%
\pstVerb{/iteration \ra\space def
/currentLeft iteration DeltaX mul leftEnd add def
/x pointRatio DeltaX mul currentLeft add def
}%
\psline*[linecolor=cyan](! currentLeft 0)(! currentLeft DeltaX add 0)(! currentLeft DeltaX add function)(! currentLeft function) (! currentLeft 0)%
\psline[linecolor=blue](! currentLeft 0)(! currentLeft DeltaX add 0)(! currentLeft DeltaX add function)(! currentLeft function) (! currentLeft 0)%
}%
\psplot[#1]{leftEnd}{rightEnd}{function}%
\pstVerb{end}%
}

\newcommand{\fcRectangularRiemannSumCode}{
/theRiemannSumFigure exch def
theRiemannSumFigure \fcInitializeAndSetupFilesCode
graphicsCached not{
theRiemannSumFigure \fcArrayToStack
30 dict begin
/theFunction exch def
/yIterations exch def
/xIterations exch def
/yMax exch def
/xMax exch def
/yMin exch def
/xMin exch def
/DeltaX xMax xMin sub xIterations div def
/DeltaY yMax yMin sub yIterations div def
/theSideColor exch def
/theContourColor exch def
/x xMin def
xIterations{
/y yMin def
yIterations{
/xOld x def
/yOld y def
/x x DeltaX 2 div add def
/y y DeltaY 2 div add def
/z theFunction def
[xOld yOld 0]
[xOld DeltaX add yOld 0]
[xOld yOld DeltaY add 0]
[xOld yOld z]
theContourColor
theSideColor
true
false
[\fcDashes]
%(GOT TO HERE) ==
\fcBoxIIIdFilledCode
%(GOT TO HERE2) ==
/x xOld def
/y yOld DeltaY add def
}repeat
/x x DeltaX add def
}repeat
end
}if
}

%Arguments:
% argument2,argument3: xMin, yMin
% argument4,argument5: xMax, yMax
% argument6,argument7: xIterations, yIterations
% argument8=function
%example
%\begin{pspicture}(-2.2,-2.3)(4,4)
%\fcRectangularRiemannSum[colorUV=cyan, linecolor=blue]{0}{0.8}{2}{1.2}{5}{1}{x 2 sub dup mul y 2 sub dup mul add 2 div}%
%\end{pspicture}
\newcommand{\fcRectangularRiemannSum}[8][]{%
\setkeys{fcGraphics}{#1}%
\pscustom{%
\code{%
[[\fcGetColorCode{\fcColorLine}] [\fcGetColorCode{\fcColorPatchUV}]  #2\space #3\space #4\space #5 \space #6\space #7\space {#8}\space]
\fcRectangularRiemannSumCode
}%
}%
}

\newcommand{\fcTextCode}{
/Times-Roman findfont
4 scalefont
setfont
newpath
\fcCoordsPStricksToPS moveto
show
stroke
}

\newcommand{\fcHollowDotBlue}[2]{
\pscircle*[fillcolor=white, linecolor=blue](#1, #2){0.07}
\pscircle*[fillcolor=white, linecolor=white](#1, #2){0.04}
}
\newcommand{\fcFullDotBlack}[2]{
\pscircle*[fillcolor=white, linecolor=black](#1, #2){0.07}
}
\newcommand{\fcFullDotBlue}[2]{
\pscircle*[fillcolor=white, linecolor=blue](! #1 #2){0.07}
}
\newcommand{\fcXTickColored}[2]{\psline[linecolor=#1](#2, -0.1)(#2,0.1)}

\newcommand{\fcTickSize}{0.1}

\newcommand{\fcXTick}[1]{\psline(! #1\space -\fcTickSize)(! #1 \space \fcTickSize)}
\newcommand{\fcYTick}[1]{\psline(! -\fcTickSize\space #1)(! \fcTickSize\space #1)}
\newcommand{\fcXYTick}[2]{\fcXTick{#1} \fcYTick{#2}}

\newcommand{\fcXTickWithLabel}[2]{\fcXTick{#1}\rput[t](! #1\space -\fcTickSize\space 2 mul){#2}}
\newcommand{\fcYTickWithLabel}[2]{\fcYTick{#1}\rput[r](! -0.2\space #1){#2}}

\newcommand{\fcLabelNumberXaxis}[1]{\fcXTickWithLabel{#1}{#1}}
\newcommand{\fcLabelNumberYaxis}[1]{\fcYTickWithLabel{#1}{#1}}

\newcommand{\fcLabelNumberXYaxes}[2]{\fcLabelNumberXaxis{#1} \fcLabelNumberYaxis{#2} }

\newcommand{\fcLabelXOne}{\fcLabelNumberXaxis{1} }
\newcommand{\fcLabelYOne}{\fcLabelNumberYaxis{1} }

\newcommand{\fcLabelOnXaxis}[2]{\fcXTick{#1}\rput[t](#1,-0.2){#2}}
\newcommand{\fcLabelOnYaxis}[2]{\fcYTick{#1}\rput[r](-0.2, #1){#2}}

\newcommand{\fcLabels}[1][$x$]{%
  \def\ArgpsXAxisLabel{{#1}}%
  \fcLabelsRelay
}
\newcommand\fcLabelsRelay[3][$y$]{\rput[t](! #2 -0.1){\ArgpsXAxisLabel}\rput[r](! -0.1 #3){#1}}

\newcommand{\fcLabelsWithOnes}[2]{\psline(1, -0.1)(1,0.1) \rput[t](1, -0.2 ) { $1$} \psline(-0.1, 1)(0.1, 1) \rput[r](-0.2, 1 ) { $1$} \fcLabels{#1}{#2}}

\newcommand{\fcDefaultXLabel}{$x$}
\newcommand{\fcDefaultYLabel}{$y$}

\newcommand{\fcBoundingBox}[4]{%
\psframe*[linecolor=white](! #1\space #2)(! #3\space #4)%
\psline[linecolor=black!1](! #1 #2 )(! #1 #2 0.01 add)%
\psline[linecolor=black!1](! #3 #4 )(! #3 #4 0.01 add)%
}
\newcommand{\fcAxesStandardNoFrame}[5][]{%
\psline[arrows=<->, linecolor=black, linewidth=1pt, #1](! #2\space 0)(! #4\space 0)%
\psline[arrows=<->, linecolor=black, linewidth=1pt, #1](! 0\space #3)(! 0\space #5)% \fcLabels[\fcDefaultXLabel][\fcDefaultYLabel]{#3}{#4}%
}%

\newcommand{\fcAxesStandard}[5][]{%
\psframe*[linecolor=white](! #2\space #3)(! #4 \space 0.1 add #5 \space 0.1 add)%
\fcAxesStandardNoFrame[#1]{#2}{#3}{#4}{#5}%
}%
\newcommand{\fcColorTangent}{blue}
\newcommand{\fcColorGraph}{red}
\newcommand{\fcColorAreaUnderGraph}{cyan}
\newcommand{\fcColorNegativeAreaUnderGraph}{orange}

\newcommand{\fcMachine}[2]{
\pscustom*[linecolor=#2]{
\psline(1,1.1)(1,0.1)(1.5,0.1)(2, 0.6)(2.5, 0.6)(2.5, -0.6)(2, -0.6)(1.5,-0.1)(1,-0.1)(1,-1.1)(-1,-1.1)(-1,-0.1)(-1.5,-0.1)(-2, -0.6)(-2.5, -0.6)(-2.5, 0.6)(-2, 0.6)(-1.5,0.1)(-1,0.1)(-1,1.1)
}
\pscircle*[linecolor=white](0,0){0.3}
\rput(0,0){#1}
}

%command format
%first argument gives you formula for the direction field in
%postscript notation, for example x y add.
%second and third argument give the starting x,y coordinates
\newcommand{\fcDirectionFieldOneTangent}[6]{%
\pstVerb{%
3 dict begin%
/x #2 \space def%
/y #3 \space def%
/F #1 \space def%
}%
\psline[#6](! x F ATAN 57.295 mul cos #4 mul sub y F ATAN 57.295 mul sin #4 mul sub)(! x F ATAN 57.295 mul cos #4 mul add y F ATAN 57.295 mul sin #4 mul add)%
\pscircle*[linecolor=red!60](! x y){#5}%
\pstVerb{%
end%
}%
}

\newcommand{\fcDirectionFieldOneTangentDefault}[3]{%
\fcDirectionFieldOneTangent{#1}{#2}{#3}{0.3}{0.03}{linecolor=blue}%
}

%command format
%first argument gives you formula for the direction field in
%postscript notation, for example x y add.
%second and third argument give the starting x,y coordinates
%fourth coordinate gives the delta x=delta y
%fifth argument gives the number of iterations delta x
%sixth argument gives the number of iterations delta y
%seventh argument gives the length of the vector
%eighth  argument gives the circle radius
%ninth argument gives the arguments of the psline command
\newcommand{\fcDirectionFieldFull}[9]{%
\multido{\ra=#2+#4}{#5}{%
\multido{\rb=#3+#4}{#6}{%
\fcDirectionFieldOneTangent{#1}{\ra}{\rb}{#7}{#8}{#9}%
}%end multido
}%end multido
}%end newcommand

\newcommand{\fcDirectionFieldDefault}[5]{%
\fcDirectionFieldFull{#1}{#2}{#3}{#4}{#5}{#5}{0.2}{0.02}{linecolor=blue}%
}%
\newcommand{\fcDirectionFieldDefaultRange}[1]{%
\fcDirectionFieldFull{#1}{-4}{-4}{0.5}{21}{21}{0.2}{0.02}{linecolor=blue}%
}

\newcommand{\fcMatrixTimesMatrix}{
10 dict begin
/matrixRight exch def
/matrixLeft exch def
/rowCounter -1 def
[
matrixLeft length {
/rowCounter rowCounter 1 add def
/columnCounter -1 def
[
matrixRight 0 get length{
/columnCounter columnCounter 1 add def
/thirdCounter -1 def
/accum 0 def
matrixLeft rowCounter get length{
/thirdCounter thirdCounter 1 add def
/accum accum
matrixLeft rowCounter get thirdCounter get
matrixRight thirdCounter get columnCounter get
mul add def
}repeat
accum
}repeat
]
}repeat
]
end
}

\newcommand{\fcMatrixTimesVector}{
10 dict begin
/theVector exch def
/theMatrix exch def
/rowCounter -1 def
[
theMatrix length {
/rowCounter rowCounter 1 add def
/columnCounter -1 def
/accum 0 def
theVector length{
/columnCounter columnCounter 1 add def
/accum accum
theMatrix rowCounter get columnCounter get theVector columnCounter get mul
add def
}repeat
accum
}repeat
]
end
}

\newcommand{\fcVectorProjectOntoVector}{%
\fcVectorNormalize dup 3 1 roll \fcVectorScalarVector \fcVectorTimesScalar%
} %

%fcAngleIIId Arguments:
%first optional: pstricks options
%second: vector describing arm of first angle
%third: vector describing arm of second angle
%fourth: radius of arc representing the angle
\newcommand{\fcAngleIIId}[4][]{%
\pstVerb{%
3 dict begin%
/firstV #2 \fcVectorNormalize def%
/orthonormalV #3 dup firstV  \fcVectorProjectOntoVector \fcVectorMinusVector \fcVectorNormalize def%
/theAngle firstV #3\space \fcVectorNormalize \fcVectorScalarVector arccos def%
}%
\parametricplot[#1]{0}{theAngle}{firstV t cos #4 mul \fcVectorTimesScalar orthonormalV t sin #4 mul \fcVectorTimesScalar \fcVectorPlusVector \fcCoordsIIIdToPStricks}%
\pstVerb{end}%
}

%argument 1: options
%arguments 2,3: start, end angle in degrees
%argument 4: arm length
%argument 5: label
\newcommand{\fcAngleDegrees}[5][linecolor=red]{%
\parametricplot[#1]{#2}{#3}{t cos #4\space mul t sin #4\space mul}%
\rput(! #2\space #3\space add 2 div dup cos #4\space 1.2 mul mul exch sin #4\space 1.2 mul mul){#5}%
}

% angle from three points
% 1st argument = options
% 2,3,4th argument = the three points
% 5th argument = radius of the arc.
% example: \fcAngleFromThreePoints{[0 1]}{[3 5]}{[ 2 7]}{0.2}
\newcommand{\fcAngleFromThreePoints}[5][ ]{%
\setkeys{fcGraphics}{#1}%
\pstVerb{%
25 dict begin %
/angleVertex #3\space def
/initialArmPoint #2\space def
/terminalArmPoint #4\space def
/arcRadius #5\space def
angleVertex \fcCheckIsArray not initialArmPoint \fcCheckIsArray not terminalArmPoint \fcCheckIsArray not or or {
(ERROR: at least one of the input to the fcAngleFromThreePoints command is not a vector. ) ==
(Your inputs are: ) ==
angleVertex ==
initialArmPoint ==
terminalArmPoint ==
}{
/firstV initialArmPoint angleVertex \fcVectorMinusVector \fcVectorNormalize def%
/secondV terminalArmPoint angleVertex \fcVectorMinusVector \fcVectorNormalize def%
/orthonormalV secondV dup firstV  \fcVectorProjectOntoVector \fcVectorMinusVector \fcVectorNormalize def%
/theAngle firstV secondV \fcVectorNormalize \fcVectorScalarVector arccos def%
}ifelse
}%
\parametricplot[linewidth=\fcAngleLineWidth, linecolor=\fcColorAngle]{0}{theAngle}{firstV t cos arcRadius mul \fcVectorTimesScalar orthonormalV t sin arcRadius mul \fcVectorTimesScalar \fcVectorPlusVector angleVertex \fcVectorPlusVector \fcArrayToStack}%
\psline[#1](! initialArmPoint  \fcArrayToStack)(! angleVertex \fcArrayToStack)(! terminalArmPoint \fcArrayToStack)%
\pstVerb{end}%
}%

\newcommand{\fcAngleBetweenVectors}[5][linecolor=\fcColorGraph]{%
\pstVerb{%
3 dict begin%
/firstV #2 \fcVectorNormalize def%
/orthonormalV #3 dup firstV  \fcVectorProjectOntoVector \fcVectorMinusVector \fcVectorNormalize def%
/theAngle firstV #3\space \fcVectorNormalize \fcVectorScalarVector arccos def%
}%
\parametricplot[#1]{0}{theAngle}{firstV t cos #4 mul \fcVectorTimesScalar orthonormalV t sin #4 mul \fcVectorTimesScalar \fcVectorPlusVector \fcArrayToStack}%
\rput(! firstV theAngle 2 div cos #4 mul \fcVectorTimesScalar orthonormalV theAngle 2 div sin #4 mul \fcVectorTimesScalar \fcVectorPlusVector \fcArrayToStack){#5}
\pstVerb{end}%
}

\makeatletter
\newcommand{\fcAngle}[5][linecolor=\fcColorGraph]{%
\ifPst@algebraic{%
\parametricplot[#1, algebraic=true]{#2}{#3}{#4*cos(t)| #4*sin(t)}%
\rput(! #2\space #3\space add 2 div 57.29578 mul cos #4\space 0.2 add mul #2\space #3\space add 2 div 57.29578 mul sin #4\space 0.2 add mul){#5}%
}%
\else%
\parametricplot[#1, algebraic=false]{#2}{#3}{t 57.29578 mul cos #4\space mul t 57.29578 mul sin #4\space mul}%
\rput(! #2\space #3\space add 2 div 57.29578 mul cos #4\space 0.2 add mul #2\space #3\space add 2 div 57.29578 mul sin #4\space 0.2 add mul){#5}%
\fi%
}
\makeatother

\newcommand{\fcDistance}{ \fcVectorMinusVector \fcVectorNorm\space}

%Indicates length with options #1
%from point (argument2, argument3) to point (argument4, argument5)
%with label argument6
\newcommand{\fcLengthIndicator}[6][]{%
\pstVerb{5 dict begin
/oV [#3 #5 sub #4 #2 sub] def 
/nV oV \fcVectorNormalize def
/nVscaled nV 0.05 \fcVectorTimesScalar def
}%
\psline[linecolor=red, #1](! #2 #3)(! #4 #5)%
\psline[linecolor=red, #1, arrows=none](! [#2 #3] nVscaled \fcVectorMinusVector \fcArrayToStack)(! [#2 #3] nVscaled \fcVectorPlusVector \fcArrayToStack)%
\psline[linecolor=red, #1, arrows=none](! [#4 #5] nVscaled \fcVectorMinusVector \fcArrayToStack)(! [#4 #5] nVscaled \fcVectorPlusVector \fcArrayToStack)%
\rput(! #2 #4 add 0.5 mul #3 #5 add 0.5 mul){\colorbox{white}{#6}}%
\pstVerb{end}%
}%

\newcommand{\fcLengthIndicatorTwo}[6][t]{%
\pstVerb{5 dict begin
/pointA [#2\space #3] def
/pointB [#4\space #5] def
}%
\psline[arrows=|-|](! pointA \fcArrayToStack)(! pointB \fcArrayToStack)%
\rput[#1](! pointA pointB \fcVectorPlusVector 0.5 \fcVectorTimesScalar \fcArrayToStack){#6}%
\pstVerb{end}%
%\rput[#1](0 ,0){#1}%
}

\makeatletter
\newcommand{\fcDrawPolar}[4][linecolor=\fcColorGraph]{%
\ifPst@algebraic{%
\parametricplot[#1]{#2}{#3}{(#4) *cos(t) | (#4) * sin(t)}%
}%
\else%
\parametricplot[#1]{#2}{#3}{#4 t 57.29578 mul cos mul #4 t 57.29578 mul sin mul}%
\fi%
}
\makeatother

\newcommand{\fcPolarWedge}[4][fillstyle=solid, linecolor=blue, fillcolor=\fcColorAreaUnderGraph]{%
\pstVerb{%
2 dict begin%
/theta {t 57.295779513 mul} def%
/r {#4} def%
}%
\pscustom[#1]{%
\psline(0,0)%
(! 1 dict begin /t #2\space def theta cos r mul theta sin r mul end)%
(! 1 dict begin /t #3\space def theta cos r mul theta sin r mul end)%
(0,0)%
}%
\pstVerb{end}%
}%

\newcommand{\fcPolarWedgeSequence}[4]{%
\multido{\ra=#1+#2}{#3}{%
\fcPolarWedge{\ra}{\ra\space #2 add}{#4}%
}%
}

\newcommand{\fcRegularNgon}[3][linecolor=\fcColorGraph]{%
\multido{\ra=0+1}{#2}{%
\psline[#1](! \ra \space #2 div 360 mul cos #3 mul \ra \space #2 div 360 mul sin #3 mul)(! \ra \space 1 add #2 div 360 mul cos #3 mul \ra \space 1 add #2 div 360 mul sin #3 mul)%
}%end multido
}

\newcommand{\fcEvaluateT}[2]{%
1 dict begin /t #1 def #2 end
}

\newcommand{\fcPolylineAlongCurve}[5][linecolor=\fcColorGraph]{%
\multido{\ra=0+1}{#2}{%
\psline[#1](! \fcEvaluateT{\ra\space #2 div #3 mul 1 \ra \space #2 div sub #4 mul add}{#5})(! \fcEvaluateT{\ra\space 1 add #2 div #3 mul 1 \ra \space 1 add #2 div sub #4 mul add}{#5})%
\rput(! \fcEvaluateT{\ra\space #2 div #3 mul 1 \ra \space #2 div sub #4 mul add}{#5}){\fcFullDot{0}{0}}%
}%
\rput(! \fcEvaluateT{#3}{#5}){\fcFullDot{0}{0}}%
}

\newcommand{\fcPolylineAlongCurveWithLabels}[6][linecolor=\fcColorGraph]{%
\fcPolylineAlongCurve[#1]{#2}{#3}{#4}{#5}%
\multido{\ia=0+1}{#2}{%
\rput[b](! \fcEvaluateT{\ia\space #2 div #3 mul 1 \ia \space #2 div sub #4 mul add}{#5} 0.1 add){${#6}_{\ia}$}%
}%
\rput[b](! \fcEvaluateT{#3}{#5}){${#6}_{#2}$}%
}

\newcommand{\fcVectorNormalize}{ %
1 dict begin %
/theV exch def % theV is our vector
theV 1 theV \fcVectorNorm div \fcVectorTimesScalar %
end %
} %pushes elements of array onto the stack

\newcommand{\fcArrayToStack}{ %
aload pop
} %pushes elements of array onto the stack

\newcommand{\fcSpliceArrayOperationArray}{ %
5 dict begin %
/theOp exch def %
/secondV exch def %
/firstV exch def %
/counter 0 def %
/dimension firstV length def %
[dimension {firstV counter get secondV counter get theOp /counter counter 1 add def } repeat] %
end %
} %splices two arrays and operation, for example [a b] [c d] {op} -> [a c op b d op]

\newcommand{\fcSpliceArrayOperation}{ %
4 dict begin %
/theOp exch def %
/firstV exch def %
/counter 0 def %
/dimension firstV length def %
[ dimension {firstV counter get theOp /counter counter 1 add def } repeat ] %
end %
} %splices array with operation. [a b] {op} -> [a op b op]

\newcommand{\fcArrayOperation}{ %
4 dict begin %
/theOp exch def %
/firstV exch def %
/counter 0 def%
/dimension firstV length def %
dimension {firstV counter get /counter counter 1 add def} repeat %
dimension 1 sub {theOp} repeat %
end %
} %applies operation n-1 times to array. Example: [a b c] {op} -> a b c op op

\newcommand{\fcVectorScalarVector}{%
{mul} \fcSpliceArrayOperationArray {add}\fcArrayOperation
} %Scalar product two vectors

\newcommand{\fcVectorPlusVector}{%
{add} \fcSpliceArrayOperationArray %
} %Adds two vectors

\newcommand{\fcVectorMinusVector}{%
{sub} \fcSpliceArrayOperationArray %
} %Adds two vectors

\newcommand{\fcVectorTimesScalar}{ %
2 dict begin %
/theScalar exch def %
/theV exch def %
theV {theScalar mul} \fcSpliceArrayOperation %
end %
} %

\newcommand{\fcVectorTripleProduct}{%
\fcVectorCrossVector \fcVectorScalarVector\space %
}

\newcommand{\fcVectorCrossVector}{ %
8 dict begin %
/vectB exch def %
/vectA exch def %
vectA \fcArrayToStack %
/a3 exch def %The three coordinates of Vector a
/a2 exch def %
/a1 exch def %
vectB \fcArrayToStack %
/b3 exch def %The three coordinates of Vector b
/b2 exch def %
/b1 exch def %
[a2 b3 mul a3 b2 mul sub a3 b1 mul a1 b3 mul sub a1 b2 mul a2 b1 mul sub] %the cross product of a and b
end %
}

\newcommand{\fcVectorNorm}{%
dup \fcVectorScalarVector sqrt %
} %

\newcommand{\fcVectorNormSquared}{%
dup \fcVectorScalarVector %
} %

\newcommand{\fcMarkClean}{
mark\space
}

\newcommand{\fcMarkCleanCheck}{
counttomark 0 ne {(ERROR: procedure did not clean up properly. Printing stack: ) print pstack == error}if pop
}

\newcommand{\fcProjectOntoScreen}{%
3 dict begin %
\fcScreen\space %
/theD exch def %
/theNormal exch def %
/theV exch def %
theV theNormal theD theV theNormal \fcVectorScalarVector sub theNormal \fcVectorNormSquared div \fcVectorTimesScalar \fcVectorPlusVector %
end %
} %Projection of point onto a plane. First argument is point, second argument is plane normal, third argument is the scalar product you need to have with the normal to be in the plane. Format: [1 2 3] [4 5 6] 7, corresponds to projecting the point (1,2,3) onto the plane 4x+5y+6z=7

\newcommand{\fcCoordsIIIdToPStricks}{%
%(nput to fcCoordsIIId to pstricks: ) ==
%dup ==
5 dict begin %
/theV exch def %
/theVprojected theV \fcProjectOntoScreen [0 0 0] \fcProjectOntoScreen  \fcVectorMinusVector def%
/theNormalizedNormal \fcScreen\space pop \fcVectorNormalize def %
(\fcScreenStyle) (z) eq %
{ %
/theYUnitV [0 0 1] \fcProjectOntoScreen [0 0 0] \fcProjectOntoScreen \fcVectorMinusVector \fcVectorNormalize def %
/theXUnitV theNormalizedNormal theYUnitV \fcVectorCrossVector def %
} %
{ %
(\fcScreenStyle) (x) eq %
{
/theXUnitV [1 0 0] \fcProjectOntoScreen [0 0 0] \fcProjectOntoScreen \fcVectorMinusVector \fcVectorNormalize def %
/theYUnitV theXUnitV theNormalizedNormal \fcVectorCrossVector def%
}
{
/theYUnitV \fcScreenStyle \fcProjectOntoScreen [0 0 0] \fcProjectOntoScreen \fcVectorMinusVector \fcVectorNormalize def%
/theXUnitV theNormalizedNormal theYUnitV \fcVectorCrossVector def%
} ifelse%
}%
ifelse %
%(normalized normal: ) == theNormalizedNormal ==
%(y unit v) == theYUnitV ==
%(x unit v: ) == theXUnitV ==
theVprojected theXUnitV \fcVectorScalarVector theVprojected theYUnitV \fcVectorScalarVector
end\space%
}

\newcommand{\fcCoordsIIIdToPS}{%
[ exch \fcCoordsIIIdToPStricks ] \fcCoordsPStricksToPS
}

\newcommand{\fcBoxIIId}[5][]{%
\pstVerb{%
4 dict begin%
/visibleCorner #2 def%
/vectorOne #3 #2 \fcVectorMinusVector def%
/vectorTwo #4 #2 \fcVectorMinusVector def%
/vectorThree #5 #2 \fcVectorMinusVector def%
}%
\fcPolyLineIIId[#1]{visibleCorner dup vectorOne \fcVectorPlusVector dup vectorTwo \fcVectorPlusVector dup vectorOne \fcVectorMinusVector dup vectorTwo \fcVectorMinusVector visibleCorner}%
\fcPolyLineIIId[#1]{visibleCorner dup vectorOne \fcVectorPlusVector dup vectorThree \fcVectorPlusVector dup vectorOne \fcVectorMinusVector dup vectorThree \fcVectorMinusVector}%
\fcPolyLineIIId[#1]{visibleCorner vectorTwo \fcVectorPlusVector dup vectorThree \fcVectorPlusVector dup vectorTwo \fcVectorMinusVector}%
\fcPolyLineIIId[#1, linestyle=dashed]{visibleCorner vectorOne  vectorTwo vectorThree \fcVectorPlusVector \fcVectorPlusVector \fcVectorPlusVector dup vectorOne \fcVectorMinusVector}%
\fcPolyLineIIId[#1, linestyle=dashed]{visibleCorner vectorOne  vectorTwo vectorThree \fcVectorPlusVector \fcVectorPlusVector \fcVectorPlusVector dup vectorTwo \fcVectorMinusVector}%
\fcPolyLineIIId[#1, linestyle=dashed]{visibleCorner vectorOne  vectorTwo vectorThree \fcVectorPlusVector \fcVectorPlusVector \fcVectorPlusVector dup vectorThree \fcVectorMinusVector}%
\pstVerb{end}%
}

\newcommand{\fcParallelogramIIIdCode}{
4 dict begin
/v2 exch def
/v1 exch def
/v0 exch def
/secondRun false def
2 {
newpath
v0 \fcCoordsPStricksToPS moveto
v0 v1 \fcVectorPlusVector \fcCoordsPStricksToPS lineto
v0 v1 v2 \fcVectorPlusVector \fcVectorPlusVector \fcCoordsPStricksToPS lineto
v0 v2 \fcVectorPlusVector \fcCoordsPStricksToPS lineto
v0 \fcCoordsPStricksToPS lineto
closepath
secondRun not {fill} if
stroke
/secondRun true def
}repeat
end
}

\newcommand{\fcPatchMakeFromThreeCorners}{
5 dict begin
/options exch def
/v2 exch def
/v1 exch def
/v0 exch def
/v3 v1 v2 \fcVectorPlusVector v0 \fcVectorMinusVector def
[v0 v1 v2 v3 [v0 v1 v1 v3 v3 v2 v2 v0] options]
end
}

\newcommand{\fcZDepth}{
\fcScreen\space pop \fcVectorScalarVector
}

\newcommand{\fcBoxIIIdFilledCode}{
%input order
% corner0 corner1 corner2 corner3
15 dict begin
/currentDashes exch def
/contourIsDashedIndependentOfVisibility exch def
/sidesVisible exch def
/colorSides exch def
/colorContour exch def
/options [colorSides colorSides true true colorContour contourIsDashedIndependentOfVisibility currentDashes] def
/corner3 exch def
/corner2 exch def
/corner1 exch def
/corner0 exch def
/v1 corner1 corner0 \fcVectorMinusVector def
/v2 corner2 corner0 \fcVectorMinusVector def
/v3 corner3 corner0 \fcVectorMinusVector def
%the following code selects the corner closest to the viewing screen
v1 \fcScreen\space pop \fcVectorScalarVector 0 lt
{/corner0 corner0 v1 \fcVectorPlusVector def /v1 v1 -1 \fcVectorTimesScalar def }if
v2 \fcScreen\space pop \fcVectorScalarVector 0 lt
{/corner0 corner0 v2 \fcVectorPlusVector def /v2 v2 -1 \fcVectorTimesScalar def }if
v3 \fcScreen\space pop \fcVectorScalarVector 0 lt
{/corner0 corner0 v3 \fcVectorPlusVector def /v3 v3 -1 \fcVectorTimesScalar def }if
%the closest corner is selected, we are recomputing the box corners
/corner1 corner0 v1 \fcVectorPlusVector def
/corner2 corner0 v2 \fcVectorPlusVector def
/corner3 corner0 v3 \fcVectorPlusVector def
[
corner0 corner1 corner2 options \fcPatchMakeFromThreeCorners
corner0 corner2 corner3 options \fcPatchMakeFromThreeCorners
corner0 corner3 corner1 options \fcPatchMakeFromThreeCorners
%corner1 corner1 v2 \fcVectorPlusVector corner1 v3 \fcVectorPlusVector options \fcPatchMakeFromThreeCorners
%corner2 corner2 v3 \fcVectorPlusVector corner2 v1 \fcVectorPlusVector options \fcPatchMakeFromThreeCorners
%corner3 corner3 v1 \fcVectorPlusVector corner3 v2 \fcVectorPlusVector options \fcPatchMakeFromThreeCorners
]
/LeftGreaterThanRight {\fcPatchGetPoint \fcZDepth exch \fcPatchGetPoint \fcZDepth gt} def
\fcMergeSort
sidesVisible{dup {\fcPatchPaintFilledDirectly }forall }if
{\fcPatchPaintContourDirectly}forall
/cornerop corner0 v1 v2 v3 \fcVectorPlusVector \fcVectorPlusVector \fcVectorPlusVector def
\fcLineFormatCode
currentDashes \fcArrayToStack setdash
[v1 v2 v3]
{
newpath
cornerop \fcCoordsIIIdToPS moveto
cornerop exch \fcVectorMinusVector \fcCoordsIIIdToPS lineto
stroke
}forall
end
}

\newcommand{\fcBoxIIIdFilledNew}[5][]{%
\setkeys{fcGraphics}{#1}%
\pscustom{%
\code{%
\fcSetUpGraphicsToScreen %
#2\space #3\space #4\space #5\space[ \fcGetColorCode{\fcColorLine} ] [\fcGetColorCode{\fcColorPatchUV}] true (\fcLineStyle) (dashed) eq [\fcDashes] \fcBoxIIIdFilledCode%
}%
}%
}

%input order
%options corner0 corner1 corner2 corner3
\newcommand{\fcBoxIIIdHollowNew}[5][]{%
\setkeys{fcGraphics}{#1}%
\pscustom{%
\code{%
\fcSetUpGraphicsToScreen %
#2\space #3\space #4\space #5\space[ \fcGetColorCode{\fcColorLine} ] [\fcGetColorCode{\fcColorPatchUV}] false (\fcLineStyle) (dashed) eq [\fcDashes] \fcBoxIIIdFilledCode%
}%
}%
}

\newcommand{\fcBoxIIIdFilled}[5][]{%
\pscustom*[#1]{%
\fcPolyLineIIId{4 dict begin%
/visibleCorner #2 def%
/vectorOne #3 #2 \fcVectorMinusVector def%
/vectorTwo #4 #2 \fcVectorMinusVector def%
/vectorThree #5 #2 \fcVectorMinusVector def %
visibleCorner vectorOne \fcVectorPlusVector dup vectorTwo \fcVectorPlusVector dup vectorOne \fcVectorMinusVector dup vectorThree \fcVectorPlusVector dup vectorTwo \fcVectorMinusVector dup vectorOne \fcVectorPlusVector visibleCorner vectorOne \fcVectorPlusVector end %
}%
}%
}

\newcommand{\fcParallelogramIIId}[4][linecolor=cyan!30]{%
\pscustom*[#1]{%
\fcParallelogramHollowIIId{#2}{#3}{#4}%
}%
}

\newcommand{\fcParallelogramHollowIIId}[4][]{ %
\fcPolyLineIIId[#1]{3 dict begin /corner #2 def /vectorOne #3 #2 \fcVectorMinusVector def /vectorTwo #4 #2 \fcVectorMinusVector def corner dup vectorOne \fcVectorPlusVector dup vectorTwo \fcVectorPlusVector dup vectorOne \fcVectorMinusVector corner end
}%
}

\newcommand{\fcParallelogramHalfVisibleIIId}[4][]{%
\pstVerb{3 dict begin /corner #2 def /vectorOne #3 #2 \fcVectorMinusVector def /vectorTwo #4 #2 \fcVectorMinusVector def}%
\fcPolyLineIIId[#1]{corner vectorOne \fcVectorPlusVector corner dup vectorTwo \fcVectorPlusVector}%
\fcPolyLineIIId[#1,linestyle=dashed]{corner vectorOne \fcVectorPlusVector dup vectorTwo \fcVectorPlusVector dup vectorOne \fcVectorMinusVector}%
\pstVerb{end}%
}

\newcommand{\fcPolyLineIIId}[2][linecolor=black]{%
\listplot[#1]{ [#2] {\fcCoordsIIIdToPStricks} \fcSpliceArrayOperation \fcArrayToStack}%
}

%\newcommand{\fcCoordsPStricksToPS}{\fcArrayToStack \fcConvertPSYUnit exch \fcConvertPSXUnit exch\space }
\makeatletter
\newcommand{\fcCoordsPStricksToPS}{\fcArrayToStack \tx@ScreenCoor\space }
\makeatother

\newcommand{\fcLine}[3][]{%
\setkeys{fcGraphics}{#1}
\pscustom{%
\code{%
\fcLineFormatCode
newpath %
#2\space \fcCoordsPStricksToPS moveto %
#3\space \fcCoordsPStricksToPS lineto %
stroke %
}%
}%
}

\newcommand{\fcEllipsoidInScene}[2][iterationsU=22, iterationsV=22]{%
\setkeys{fcGraphics}{#1}%
\pstVerb{%
/theIIIdObjects%
[theIIIdObjects \fcArrayToStack [0 0 180 360%
{ #2\space 6 dict begin%
  /c exch def%
  /b exch def%
  /a exch def%
  /z1 exch def%
  /y1 exch def%
  /x1 exch def%
  [ u sin v cos mul a mul x1 add %
    u sin v sin mul b mul y1 add %
    u cos c mul z1 add%
  ]%
  end%
}%
{true}%
\fcIterationsU\space \fcIterationsV\space %
\fcPatchOptions %
\fcContourOptions %
(surface)%
]%
]%
def%
}%
}

\newcommand{\fcLineFormatCode}{\fcDashesCode \fcLineWidth\space setlinewidth \fcGetColorCode{\fcColorLine} setrgbcolor %(fcDashesCode: \fcDashesCode ) == %
}
\newcommand{\fcLineFormatCodeVirtual}{\fcDashesCodeVirtual \fcLineWidth\space setlinewidthVirtual \fcGetColorCode{\fcColorLine} setrgbcolorVirtual %(fcDashesCodeVirtual: \fcDashesCodeVirtual ) == %
}

\newcommand{\fcCurveCode}{%
%(calling fcCurveCode) == %
5 dict begin %
%newpath 0 0 moveto 1000 1000 lineto stroke
/theCurve exch def %
%theCurve == %
/tMin exch def%
/tMax exch def%
/Delta tMax tMin sub \fcPlotPoints \space 1 sub div def %
/t tMin def %
\fcLineFormatCode %
newpath %
theCurve \fcCoordsPStricksToPS moveto %
\fcPlotPoints\space 1 sub {/t t Delta add def theCurve \fcCoordsPStricksToPS lineto %
} repeat %
stroke %
end\space%
}

\newcommand{\fcCurve}[4][]{%
\setkeys{fcGraphics}{#1}%
\pstVerb{#2\space #3\space {#4} \space \fcCurveCode}%
}

\newcommand{\fcLineIIId}[3][linecolor=black]{%
\psline[#1](! #2 \space \fcCoordsIIIdToPStricks)(! #3 \space \fcCoordsIIIdToPStricks)%
}

\newcommand{\fcAxesIIIdFull}[4][linecolor=black, arrows=->]{%
\fcAxesIIId[#1]{#2}{#3}{#4}%
\fcLineIIId[#1]{[0 0 0]}{[#2\space -1 mul 0 0]}%
\fcLineIIId[#1]{[0 0 0]}{[0 #3\space -1 mul 0]}%
\fcLineIIId[#1]{[0 0 0]}{[0 0 #4\space -1 mul]}%
} %

\newcommand{\fcAxesIIIdInScene}[4][linecolor=black, arrows=->]{%
\setkeys{fcGraphics}{#1}%
\fcLineIIIdInScene[#1]{[0 0 0]}{[#2 0 0]}%
\fcLineIIIdInScene[#1]{[0 0 0]}{[0 #3 0]}%
\fcLineIIIdInScene[#1]{[0 0 0]}{[0 0 #4]}%
\rput[l](! [#2 0 0] \fcCoordsIIIdToPStricks){~\fcXLabel}%
\rput[l](! [0 #3 0] \fcCoordsIIIdToPStricks){~\fcYLabel}%
\rput[r](! [0 0 #4] \fcCoordsIIIdToPStricks){\fcZLabel~}%
}%

\newcommand{\fcAxesIIIdFullInScene}[7][linecolor=black, arrows=<->]{%
\setkeys{fcGraphics}{#1}%
\fcLineIIIdInScene[#1]{[0 0 0]}{[#2 0 0]}%
\fcLineIIIdInScene[#1]{[0 0 0]}{[#5 0 0]}%
\fcLineIIIdInScene[#1]{[0 0 0]}{[0 #3 0]}%
\fcLineIIIdInScene[#1]{[0 0 0]}{[0 #6 0]}%
\fcLineIIIdInScene[#1]{[0 0 0]}{[0 0 #4]}%
\fcLineIIIdInScene[#1]{[0 0 0]}{[0 0 #7]}%
}

\newcommand{\fcAxesIIIdSymmetricInScene}[4][linecolor=black, arrows=->]{%
\setkeys{fcGraphics}{#1}%
\fcLineIIIdInScene[#1]{[0 0 0]}{[#2 0 0]}%
\fcLineIIIdInScene[#1]{[0 0 0]}{[0 #3 0]}%
\fcLineIIIdInScene[#1]{[0 0 0]}{[0 0 #4]}%
\fcLineIIIdInScene[#1]{[0 0 0]}{[#2\space-1 mul 0 0]}%
\fcLineIIIdInScene[#1]{[0 0 0]}{[0 #3\space-1 mul 0]}%
\fcLineIIIdInScene[#1]{[0 0 0]}{[0 0 #4\space-1 mul]}%
}

\newcommand{\fcAxesIIId}[4][linecolor=black, arrows=->]{%
\setkeys{fcGraphics}{#1}%
\fcLineIIId[#1]{[0 0 0]}{[#2 0 0]}%
\fcLineIIId[#1]{[0 0 0]}{[0 #3 0]}%
\fcLineIIId[#1]{[0 0 0]}{[0 0 #4]}%
\rput[l](! [#2 0 0] \fcCoordsIIIdToPStricks){~\fcXLabel}%
\rput[l](! [0 #3 0] \fcCoordsIIIdToPStricks){~\fcYLabel}%
\rput[r](! [0 0 #4] \fcCoordsIIIdToPStricks){\fcZLabel~}%
}

\newcommand{\fcDotIIId}[2][linecolor=\fcColorGraph]{%
\pscircle*[#1](! #2 \fcCoordsIIIdToPStricks){0.07} %
} %

\newcommand{\fcPutIIId}[3][]{ \rput[#1](! #2 \fcCoordsIIIdToPStricks) {#3}%
} %

\newcommand{\fcPaintCone}{ %
\fcArrayToStack %
15 dict begin %
/c exch def %
/b exch def %
/a exch def %
/z1 exch def %
/y1 exch def %
/x1 exch def %
/zmax exch def %
/zmin exch def %
}

\newcommand{\fcZBufferRowColumn}{ %
%input: vector on the top of the stack.
%output: row column of point in the z-buffer.
\fcCoordsIIIdToPStricks %
2 dict begin %
/rowIndex exch \space getZBufferYmin sub getZBufferYmax getZBufferYmin sub div zBufferNumRows mul floor cvi def %
/columnIndex exch getZBufferXmin sub getZBufferXmax getZBufferXmin sub div zBufferNumCols mul floor cvi def %
rowIndex zBufferNumRows ge {/rowIndex rowIndex 1 sub def}if %
columnIndex zBufferNumCols ge {/columnIndex columnIndex 1 sub def}if %
rowIndex zBufferNumRows ge {(ERROR: bad row index!!!) == rowIndex ==}if %
columnIndex zBufferNumCols ge {(ERROR: bad column index: ) == columnIndex == }if %
rowIndex 0 lt {/rowIndex rowIndex 1 add def}if %
columnIndex 0 lt {/columnIndex columnIndex 1 add def}if %
rowIndex 0 lt {(ERROR: bad row index!!!) == rowIndex ==}if %
columnIndex 0 lt {(ERROR: bad column index: ) == columnIndex ==  }if %
rowIndex columnIndex %
end %
}

\newcommand{\fcPointIsBehindOrInFrontOfPatch}[1]{ %
%(entering fcPointIsBehindOrInFrontOfPatch) ==
%a patch is assumed to be on the top of the stack
12 dict begin %
/thePatch exch def
/point exch def
thePatch \fcPatchGetInBounds
{
/v0 thePatch \fcPatchGetvZero def %
/v1 thePatch \fcPatchGetvOne def %
/v2 thePatch \fcPatchGetvTwo def %
/v3 thePatch \fcPatchGetvThree def %
/normalLeft v1 v0 \fcVectorMinusVector \fcScreen\space pop \fcVectorCrossVector def %
/normalRight v3 v2 \fcVectorMinusVector \fcScreen\space pop \fcVectorCrossVector def %
/normalBottom v2 v0 \fcVectorMinusVector \fcScreen\space pop \fcVectorCrossVector def %
/normalTop v3 v1 \fcVectorMinusVector \fcScreen\space pop \fcVectorCrossVector def %
/patchNormal v1 v0 \fcVectorMinusVector v2 v0 \fcVectorMinusVector \fcVectorCrossVector def %
point v0 \fcVectorMinusVector normalLeft \fcVectorScalarVector %
v2 point \fcVectorMinusVector normalRight \fcVectorScalarVector %
mul 0 ge { %
point v0 \fcVectorMinusVector normalBottom \fcVectorScalarVector %
v1 point \fcVectorMinusVector normalTop \fcVectorScalarVector %
mul 0 ge %
{ %
point v0 \fcVectorMinusVector patchNormal \fcVectorScalarVector %
\fcScreen\space pop patchNormal \fcVectorScalarVector %
mul #1{0.00001  gt}{-0.00001 lt}ifelse %
{true}{false}ifelse %
}{false}ifelse %
}{false}ifelse %
}{false}ifelse
end %
}

\newcommand{\fcIsInForeground}{ %
15 dict begin %
/theNeighborhood exch def %
/thePoint theNeighborhood 0 get def %
%(neighborhood: ) print
%theNeighborhood ==
%(thePoint:) print
%thePoint ==
thePoint \fcZBufferRowColumn %
/column exch def %
/row exch def %
/theZBuffEntry theZBuffer row get column get def %
/result true def %
/counterZBuff -1 def %
theZBuffEntry length { %
/counterZBuff counterZBuff 1 add def %
/theZbuffPatchIndex theZBuffEntry counterZBuff get def %
/theZbuffPatch thePatchCollection theZbuffPatchIndex get def %
theZbuffPatchIndex theNeighborhood \fcContains not{ %
%(patch) == theZbuffPatch == (is not contained in neighborhood ) ==
%theNeighborhood ==
thePoint theZbuffPatch \fcPointIsBehindOrInFrontOfPatch{true}
{/result false def
exit
}
{ %(point is in front of patch) ==
}
ifelse
}
{ %(patch coincides with zbuff patch) ==
}
ifelse %
}repeat %
result %
end %
}

\newcommand{\fcZBufferBoundingBoxPatch}{ %
5 dict begin %
/thePatch exch def
/v3 thePatch \fcPatchGetvThree def %
/v2 thePatch \fcPatchGetvTwo def %
/v1 thePatch \fcPatchGetvOne def %
/v0 thePatch \fcPatchGetvZero def %
v0 \fcZBufferBoundingBoxPoint %
v1 \fcZBufferBoundingBoxPoint %
v2 \fcZBufferBoundingBoxPoint %
v3 \fcZBufferBoundingBoxPoint %
v1 v2 \fcVectorPlusVector v0 \fcVectorMinusVector \fcZBufferBoundingBoxPoint %
end %
}

\newcommand{\fcZBufferBoundingBoxPoint}{ %
%Account bounding box:
dup \fcScreen\space pop  \fcVectorNormalize \fcVectorScalarVector
dup getZmin lt {dup setZmin}{}ifelse
dup getZmax gt {setZmax}{pop} ifelse
\fcCoordsIIIdToPStricks %
dup dup getZBufferYmin lt {setZBufferYmin}{pop}ifelse %
dup getZBufferYmax gt {setZBufferYmax}{pop}ifelse %
dup dup getZBufferXmin lt {setZBufferXmin}{pop}ifelse %
dup getZBufferXmax gt {setZBufferXmax}{pop}ifelse \space%
}

\newcommand{\fcZBufferEllipsoid}{ %
%(calling fcZBufferEllipsoid with input: ) == dup == %
\fcArrayToStack
6 dict begin
/c exch def %
/b exch def %
/a exch def %
/z1 exch def %
/y1 exch def %
/x1 exch def %
[a -1 mul x1 add b -1 mul y1 add c -1 mul z1 add a x1 add b y1 add c z1 add] %
end
}

\newcommand{\fcSegmentBoundingBox}{ %
\fcZBufferBoundingBoxPoint %
\fcZBufferBoundingBoxPoint %
}

\newcommand{\fcZBufferPaintCellContainingPoint}{
10 dict begin
/thePoint exch def
thePoint \fcZBufferRowColumn
/column exch def
/row exch def
\fcZBufferComputeDeltaXDeltaY
/lowerLeftX getZBufferXmin DeltaX column mul add def
/lowerLeftY getZBufferYmin DeltaY row mul add def
gsave
0.6 0.6 1 setrgbcolor %
newpath
[lowerLeftX lowerLeftY] \fcCoordsPStricksToPS moveto
[lowerLeftX DeltaX add lowerLeftY] \fcCoordsPStricksToPS lineto
[lowerLeftX DeltaX add lowerLeftY DeltaY add] \fcCoordsPStricksToPS lineto
[lowerLeftX lowerLeftY DeltaY add] \fcCoordsPStricksToPS lineto
[lowerLeftX lowerLeftY] \fcCoordsPStricksToPS lineto
stroke
grestore
end
}

\newcommand{\fcZBufferComputeDeltaXDeltaY}{
/DeltaX getZBufferXmax getZBufferXmin sub zBufferNumCols div def %
/DeltaY getZBufferYmax getZBufferYmin sub zBufferNumRows div def %
}

\newcommand{\fcPaintZbuffForDebug}{ %
6 dict begin %
\fcZBufferComputeDeltaXDeltaY
gsave %
0.1 setlinewidth %
/x getZBufferXmin def %
0.5 0.5 0.5 setrgbcolor %
zBufferNumRows {newpath [x getZBufferYmin] \fcCoordsPStricksToPS moveto [x getZBufferYmax] \fcCoordsPStricksToPS lineto stroke /x x DeltaX add def}repeat %
/y getZBufferYmin def %
zBufferNumCols { newpath [getZBufferXmin y] \fcCoordsPStricksToPS moveto [getZBufferXmax y] \fcCoordsPStricksToPS lineto stroke /y y DeltaY add def}repeat %
/y getZBufferYmin DeltaY 2 div add def %
/counterY 0 def %
zBufferNumRows { %
/x getZBufferXmin DeltaX 2 div add def %
/counterX 0 def %
zBufferNumCols { %
theZBuffer counterY get counterX get length 0 gt{ %
[x y] \fcFullDotCode %
} if %
/x x DeltaX add def %
/counterX counterX 1 add def %
}repeat %
/y y DeltaY add def %
/counterY counterY 1 add def %
}repeat %
grestore %
end %
}

\newcommand{\fcStartIIIdScene}{%
\pstVerb{20 dict begin /theIIIdObjects [] def \fcBufferInitializePartOne\space
}%
}%

\newcommand{\fcZBufferPrint}{ %
(printing Zbuffer...) == %
getZBufferXmin == %
getZBufferXmax == %
getZBufferYmin == %
getZBufferYmax == %
theZBuffer == %
}

\newcommand{\fcZBufferLoad}{ %
\fcZBufferInitialize %
/theZBuffer exch def %
setZBufferYmax %
setZBufferYmin %
setZBufferXmax %
setZBufferXmin %
}

\newcommand{\fcBufferInitializePartOne}{
/ZBufferRectangle [0 0 0 0] def %
/ZMinMax [0 0] def%
/setZBufferXmin {ZBufferRectangle exch 0 exch put} def %
/setZBufferYmin {ZBufferRectangle exch 1 exch put} def %
/setZmin        {ZMinMax          exch 0 exch put} def %
/setZBufferXmax {ZBufferRectangle exch 2 exch put} def %
/setZBufferYmax {ZBufferRectangle exch 3 exch put} def %
/setZmax        {ZMinMax          exch 1 exch put} def %
/getZBufferXmin {ZBufferRectangle 0 get} def %
/getZBufferYmin {ZBufferRectangle 1 get} def %
/getZmin        {ZMinMax          0 get} def %
/getZBufferXmax {ZBufferRectangle 2 get} def %
/getZBufferYmax {ZBufferRectangle 3 get} def %
/getZmax        {ZMinMax          1 get} def %
}

\newcommand{\fcZBufferInitialize}{ %
graphicsFileAvailable graphicsCached not and {%
graphicsFile (\fcBufferInitializePartOne) writestring
}if%
/zBufferNumCols \fcZBufferNumXIntervals\space def
/zBufferNumRows \fcZBufferNumYIntervals\space def
thePatchCollection length zBufferNumCols zBufferNumRows mul lt
{ /zBufferNumCols thePatchCollection length sqrt round cvi def
  zBufferNumCols 3 lt {/zBufferNumCols 3 def}if
  /zBufferNumRows zBufferNumCols def
  (There are only a few patches in the scene, I decreased z-buffer size to ) print zBufferNumCols == (rows and columns. ) print
}if
/getZBufferDeltaX {getZBufferXmax getZBufferXmin sub zBufferNumCols div} def %
/getZBufferDeltaY {getZBufferYmax getZBufferYmin sub zBufferNumRows div} def %
/theZBuffer [zBufferNumRows {[zBufferNumCols{[]} repeat]}repeat] def %
}

\newcommand{\fcPaintCachedFile}{
graphicsFile run
}

\newcommand{\fcGetPointOnPlaneThatProjectsToXYCode}{
15 dict begin
/y exch def
/x exch def
%(made it to here ) ==
/vZeroProj [vZero \fcCoordsIIIdToPStricks] def
/dirVectProjOne [vOne \fcCoordsIIIdToPStricks] vZeroProj \fcVectorMinusVector def
/dirVectProjTwo [vTwo \fcCoordsIIIdToPStricks] vZeroProj \fcVectorMinusVector def
%(dir vect proj one: ) ==
%dirVectProjOne ==
dirVectProjOne \fcArrayToStack
/a21 exch def
/a11 exch def
dirVectProjTwo \fcArrayToStack
/a22 exch def
/a12 exch def
/theDet a11 a22 mul a12 a21 mul sub def
theDet 0 eq {vZero}{
/theVproj [x y] vZeroProj \fcVectorMinusVector def
/theLinCombi
[ [a22 a12 -1 mul] 1 theDet div \fcVectorTimesScalar [a21 -1 mul a11] 1 theDet div \fcVectorTimesScalar ] theVproj \fcMatrixTimesVector  def
%(the lin combi: ) ==
%theLinCombi ==
%(counttomark produces: ) ==
%counttomark ==
vOne vZero \fcVectorMinusVector theLinCombi 0 get \fcVectorTimesScalar
vTwo vZero \fcVectorMinusVector theLinCombi 1 get \fcVectorTimesScalar
\fcVectorPlusVector
vZero
\fcVectorPlusVector
%(and so the vector is... ) ==
%dup ==
}ifelse
%(printing stack at end of get point that projects ...) ==
%pstack
end
}

\newcommand{\fcByteToHexStringCode}{
1 dict begin
/theByte exch def
theByte 255 gt {/theByte 255 def} if
theByte 0 lt {/theByte 0 def} if
%Believe it or not, after about 10 hours of googling,
% I came to the conclusion that this is the simplest implementation
%of a conversion of a byte to its binary representation.
%Note that while there may be a language mechanism to do that
%I wasn't able to find it after 10 hour of search, while the solution below takes about 15 minutes to type (10 minutes with a good text editor), which more or less shows that the solution below is the best.
theByte 0  eq {<00>}if
theByte 1  eq {<01>}if
theByte 2  eq {<02>}if
theByte 3  eq {<03>}if
theByte 4  eq {<04>}if
theByte 5  eq {<05>}if
theByte 6  eq {<06>}if
theByte 7  eq {<07>}if
theByte 8  eq {<08>}if
theByte 9  eq {<09>}if
theByte 10 eq {<0A>}if
theByte 11 eq {<0B>}if
theByte 12 eq {<0C>}if
theByte 13 eq {<0D>}if
theByte 14 eq {<0E>}if
theByte 15 eq {<0F>}if
theByte 16 eq {<10>}if
theByte 17 eq {<11>}if
theByte 18 eq {<12>}if
theByte 19 eq {<13>}if
theByte 20 eq {<14>}if
theByte 21 eq {<15>}if
theByte 22 eq {<16>}if
theByte 23 eq {<17>}if
theByte 24 eq {<18>}if
theByte 25 eq {<19>}if
theByte 26 eq {<1A>}if
theByte 27 eq {<1B>}if
theByte 28 eq {<1C>}if
theByte 29 eq {<1D>}if
theByte 30 eq {<1E>}if
theByte 31 eq {<1F>}if
theByte 32 eq {<20>}if
theByte 33 eq {<21>}if
theByte 34 eq {<22>}if
theByte 35 eq {<23>}if
theByte 36 eq {<24>}if
theByte 37 eq {<25>}if
theByte 38 eq {<26>}if
theByte 39 eq {<27>}if
theByte 40 eq {<28>}if
theByte 41 eq {<29>}if
theByte 42 eq {<2A>}if
theByte 43 eq {<2B>}if
theByte 44 eq {<2C>}if
theByte 45 eq {<2D>}if
theByte 46 eq {<2E>}if
theByte 47 eq {<2F>}if
theByte 48 eq {<30>}if
theByte 49 eq {<31>}if
theByte 50 eq {<32>}if
theByte 51 eq {<33>}if
theByte 52 eq {<34>}if
theByte 53 eq {<35>}if
theByte 54 eq {<36>}if
theByte 55 eq {<37>}if
theByte 56 eq {<38>}if
theByte 57 eq {<39>}if
theByte 58 eq {<3A>}if
theByte 59 eq {<3B>}if
theByte 60 eq {<3C>}if
theByte 61 eq {<3D>}if
theByte 62 eq {<3E>}if
theByte 63 eq {<3F>}if
theByte 64 eq {<40>}if
theByte 65 eq {<41>}if
theByte 66 eq {<42>}if
theByte 67 eq {<43>}if
theByte 68 eq {<44>}if
theByte 69 eq {<45>}if
theByte 70 eq {<46>}if
theByte 71 eq {<47>}if
theByte 72 eq {<48>}if
theByte 73 eq {<49>}if
theByte 74 eq {<4A>}if
theByte 75 eq {<4B>}if
theByte 76 eq {<4C>}if
theByte 77 eq {<4D>}if
theByte 78 eq {<4E>}if
theByte 79 eq {<4F>}if
theByte 80 eq {<50>}if
theByte 81 eq {<51>}if
theByte 82 eq {<52>}if
theByte 83 eq {<53>}if
theByte 84 eq {<54>}if
theByte 85 eq {<55>}if
theByte 86 eq {<56>}if
theByte 87 eq {<57>}if
theByte 88 eq {<58>}if
theByte 89 eq {<59>}if
theByte 90 eq {<5A>}if
theByte 91 eq {<5B>}if
theByte 92 eq {<5C>}if
theByte 93 eq {<5D>}if
theByte 94 eq {<5E>}if
theByte 95 eq {<5F>}if
theByte 96 eq  {<60>}if
theByte 97 eq  {<61>}if
theByte 98 eq  {<62>}if
theByte 99 eq  {<63>}if
theByte 100 eq {<64>}if
theByte 101 eq {<65>}if
theByte 102 eq {<66>}if
theByte 103 eq {<67>}if
theByte 104 eq {<68>}if
theByte 105 eq {<69>}if
theByte 106 eq {<6A>}if
theByte 107 eq {<6B>}if
theByte 108 eq {<6C>}if
theByte 109 eq {<6D>}if
theByte 110 eq {<6E>}if
theByte 111 eq {<6F>}if
theByte 112 eq {<70>}if
theByte 113 eq {<71>}if
theByte 114 eq {<72>}if
theByte 115 eq {<73>}if
theByte 116 eq {<74>}if
theByte 117 eq {<75>}if
theByte 118 eq {<76>}if
theByte 119 eq {<77>}if
theByte 120 eq {<78>}if
theByte 121 eq {<79>}if
theByte 122 eq {<7A>}if
theByte 123 eq {<7B>}if
theByte 124 eq {<7C>}if
theByte 125 eq {<7D>}if
theByte 126 eq {<7E>}if
theByte 127 eq {<7F>}if
theByte 128 eq {<80>}if
theByte 129 eq {<81>}if
theByte 130 eq {<82>}if
theByte 131 eq {<83>}if
theByte 132 eq {<84>}if
theByte 133 eq {<85>}if
theByte 134 eq {<86>}if
theByte 135 eq {<87>}if
theByte 136 eq {<88>}if
theByte 137 eq {<89>}if
theByte 138 eq {<8A>}if
theByte 139 eq {<8B>}if
theByte 140 eq {<8C>}if
theByte 141 eq {<8D>}if
theByte 142 eq {<8E>}if
theByte 143 eq {<8F>}if
theByte 144 eq {<90>}if
theByte 145 eq {<91>}if
theByte 146 eq {<92>}if
theByte 147 eq {<93>}if
theByte 148 eq {<94>}if
theByte 149 eq {<95>}if
theByte 150 eq {<96>}if
theByte 151 eq {<97>}if
theByte 152 eq {<98>}if
theByte 153 eq {<99>}if
theByte 154 eq {<9A>}if
theByte 155 eq {<9B>}if
theByte 156 eq {<9C>}if
theByte 157 eq {<9D>}if
theByte 158 eq {<9E>}if
theByte 159 eq {<9F>}if
theByte 160 eq {<A0>}if
theByte 161 eq {<A1>}if
theByte 162 eq {<A2>}if
theByte 163 eq {<A3>}if
theByte 164 eq {<A4>}if
theByte 165 eq {<A5>}if
theByte 166 eq {<A6>}if
theByte 167 eq {<A7>}if
theByte 168 eq {<A8>}if
theByte 169 eq {<A9>}if
theByte 170 eq {<AA>}if
theByte 171 eq {<AB>}if
theByte 172 eq {<AC>}if
theByte 173 eq {<AD>}if
theByte 174 eq {<AE>}if
theByte 175 eq {<AF>}if
theByte 176 eq {<B0>}if
theByte 177 eq {<B1>}if
theByte 178 eq {<B2>}if
theByte 179 eq {<B3>}if
theByte 180 eq {<B4>}if
theByte 181 eq {<B5>}if
theByte 182 eq {<B6>}if
theByte 183 eq {<B7>}if
theByte 184 eq {<B8>}if
theByte 185 eq {<B9>}if
theByte 186 eq {<BA>}if
theByte 187 eq {<BB>}if
theByte 188 eq {<BC>}if
theByte 189 eq {<BD>}if
theByte 190 eq {<BE>}if
theByte 191 eq {<BF>}if
theByte 192 eq {<C0>}if
theByte 193 eq {<C1>}if
theByte 194 eq {<C2>}if
theByte 195 eq {<C3>}if
theByte 196 eq {<C4>}if
theByte 197 eq {<C5>}if
theByte 198 eq {<C6>}if
theByte 199 eq {<C7>}if
theByte 200 eq {<C8>}if
theByte 201 eq {<C9>}if
theByte 202 eq {<CA>}if
theByte 203 eq {<CB>}if
theByte 204 eq {<CC>}if
theByte 205 eq {<CD>}if
theByte 206 eq {<CE>}if
theByte 207 eq {<CF>}if
theByte 208 eq {<D0>}if
theByte 209 eq {<D1>}if
theByte 210 eq {<D2>}if
theByte 211 eq {<D3>}if
theByte 212 eq {<D4>}if
theByte 213 eq {<D5>}if
theByte 214 eq {<D6>}if
theByte 215 eq {<D7>}if
theByte 216 eq {<D8>}if
theByte 217 eq {<D9>}if
theByte 218 eq {<DA>}if
theByte 219 eq {<DB>}if
theByte 220 eq {<DC>}if
theByte 221 eq {<DD>}if
theByte 222 eq {<DE>}if
theByte 223 eq {<DF>}if
theByte 224 eq {<E0>}if
theByte 225 eq {<E1>}if
theByte 226 eq {<E2>}if
theByte 227 eq {<E3>}if
theByte 228 eq {<E4>}if
theByte 229 eq {<E5>}if
theByte 230 eq {<E6>}if
theByte 231 eq {<E7>}if
theByte 232 eq {<E8>}if
theByte 233 eq {<E9>}if
theByte 234 eq {<EA>}if
theByte 235 eq {<EB>}if
theByte 236 eq {<EC>}if
theByte 237 eq {<ED>}if
theByte 238 eq {<EE>}if
theByte 239 eq {<EF>}if
theByte 240 eq {<F0>}if
theByte 241 eq {<F1>}if
theByte 242 eq {<F2>}if
theByte 243 eq {<F3>}if
theByte 244 eq {<F4>}if
theByte 245 eq {<F5>}if
theByte 246 eq {<F6>}if
theByte 247 eq {<F7>}if
theByte 248 eq {<F8>}if
theByte 249 eq {<F9>}if
theByte 250 eq {<FA>}if
theByte 251 eq {<FB>}if
theByte 252 eq {<FC>}if
theByte 253 eq {<FD>}if
theByte 254 eq {<FE>}if
theByte 255 eq {<FF>}if
end\space
}

\newcommand{\fcColorToColorStringCode}{
5 dict begin
/theColor exch def
/theColor [theColor {cvi} forall] def
theColor 0 get
theColor 0 get 255 gt {pop 255}if
theColor 0 get 0 lt {pop 0}if
\fcByteToHexStringCode %dup ==
theColor 1 get
theColor 1 get 255 gt {pop 255}if
theColor 1 get 0 lt {pop 0}if
\fcByteToHexStringCode %dup ==
theColor 2 get
theColor 2 get 255 gt {pop 255}if
theColor 2 get 0 lt {pop 0} if
\fcByteToHexStringCode %dup ==
3 \fcConcatenateMultiple
end
}

\newcommand{\fcColorFromColorBaseAndPoint}{
5 dict begin
\fcScreen\space pop \fcVectorNormalize \fcVectorScalarVector
/theScalarProd exch def
/currentColor exch def
/theColorDifference
/getZmax where {pop
getZmax getZmin sub dup 0 eq {(Zmax equal Zmin, something is wrong!) == pop 1}if
theScalarProd getZmin sub exch div 0.5 sub \fcLightDifference\space mul
}{0}ifelse
def
currentColor [theColorDifference theColorDifference theColorDifference] \fcVectorMinusVector
end\space
}

\newcommand{\fcShFillPlaneFromLightSourceGetColorFromXYCode}{
\fcGetPointOnPlaneThatProjectsToXYCode\space
currentColor exch
\fcColorFromColorBaseAndPoint \space 255 \fcVectorTimesScalar
\fcColorToColorStringCode\space
%(theColor: ) ==
%dup ==
%(string representing color: ) ==
%dup ==
}


%calling format:
% [ vZero vOne vTwo [r g b]] \fcShFillPlaneFromLightSourceCode
% where vZero vOne vTwo are three points on the patch
% and [r b g] is the red-blue-green color of the patch
\newcommand{\fcShFillPlaneFromLightSourceCode}{
20 dict begin
\fcArrayToStack
/currentColor exch def
/vTwo exch def
/vOne exch def
/vZero exch def
/directionOne vOne vZero \fcVectorMinusVector def
/directionTwo vTwo vZero \fcVectorMinusVector def
/numXiterations 4 def
/numYiterations 4 def
/DeltaX getZBufferXmax getZBufferXmin sub numXiterations 1 sub div def
/DeltaY getZBufferYmax getZBufferYmin sub numYiterations 1 sub div def
gsave % save graphics state
%debug code start
%fill
%newpath
%[getZBufferXmin getZBufferYmin] \fcCoordsPStricksToPS  moveto
%[getZBufferXmin getZBufferYmax] \fcCoordsPStricksToPS lineto
%[getZBufferXmax getZBufferYmax] \fcCoordsPStricksToPS lineto
%[getZBufferXmax getZBufferYmin] \fcCoordsPStricksToPS lineto
%[getZBufferXmin getZBufferYmin] \fcCoordsPStricksToPS lineto
%debug code end
clip % clip to constructed path
%newpath % clear out current path
% Define the shading and function dictionaries
<<
/Domain [ [getZBufferXmin getZBufferYmin] \fcCoordsPStricksToPS  [getZBufferXmax  getZBufferYmax] \fcCoordsPStricksToPS 3 -1 roll exch]% Domain box: a large rectangular box that bounds the shading. Nothing will be painted outside of that box.
%Format of the box: [xMin xMax yMin yMax]
/Matrix [1 0 0 1 0 0] % additional affine transformation of bounded box.
/ShadingType 1 %sampling shading.
/ColorSpace /DeviceRGB
/Function <<
/FunctionType 0
/Order 1 %order=3 use cubic spline interpolation.
%order=1 use linear interpolation.
/Domain [ [getZBufferXmin getZBufferYmin] \fcCoordsPStricksToPS  [getZBufferXmax  getZBufferYmax] \fcCoordsPStricksToPS 3 -1 roll exch]% must be at least as large as the previously defined domain.
/Range [0 1 0 1 0 1]%has something to do with how each individual color channel is interpolated - exactly what  is that something can be found by trial and error or by decoding the adobe manual on the shfill operator. Having read the latter for many hours, I strongly recommend the trial and error method.
/BitsPerSample 8 % means the RGB color is 24 bit (8 bits per channel)
/Size [numXiterations numYiterations] % number of sampling points within the domain box
%/Size [1 1] % number of sampling points within the domain box
%dup ==
%(about to enter data source) ==
/DataSource
/counterX -1 def
numXiterations {
/counterX counterX 1 add def
/counterY -1 def
numYiterations
{/counterY counterY 1 add def
getZBufferXmin counterX DeltaX mul add getZBufferYmin counterY DeltaY mul add \fcShFillPlaneFromLightSourceGetColorFromXYCode
%(after getting color: ) ==
%pstack
}repeat
}repeat
%(got to here) ==
%pstack
numXiterations numYiterations mul
%(got to before concatenation) == pstack
\fcConcatenateMultiple
%(the stack so far: ) == pstack
%filter
%currentfile
%def
%(the shading dictionary is complete, printing stack: ) ==
%pstack
>>
>>
%(got to before shfill, stack: ) ==
%pstack
shfill
grestore
end
}

\newcommand{\fcSetUpGraphicsToScreen}{
%(SETTING UP graphcis SCREEN ONLY!!!) ==
/movetoVirtual {moveto} def
/linetoVirtual {lineto} def
/strokeVirtual {stroke} def
/closepathVirtual {closepath} def
/newpathVirtual {newpath} def
/fillVirtual {fill} def
/shfillPlaneFromLightSourceVirtual {
shfillPlaneFromLightSource} def
/arrowVirtual {} def
/setrgbcolorVirtual {setrgbcolor} def
/setlinewidthVirtual {setlinewidth} def
/plotArrowHeadVirtual {\fcArrowHeadPlotCode} def
/setdashVirtual {setdash} def
}

\newcommand{\fcSetUpGraphicsToFileAndScreen}{
graphicsFile ( /plotArrowHeadVirtual {\fcArrowHeadPlotCode} def ) writestring
/storeNumberPairToGraphicsFile{
\fcMarkClean 3 1 roll
20 string cvs exch 20 string cvs graphicsFile exch writestring graphicsFile ( ) writestring graphicsFile exch writestring graphicsFile ( ) writestring
\fcMarkCleanCheck
} def
%(About to define movetoVirtual !!!!) ==
/movetoVirtual {
2 copy moveto
storeNumberPairToGraphicsFile
graphicsFile (moveto ) writestring
} def
/setlinewidthVirtual {
dup setlinewidth
20 string cvs graphicsFile exch writestring graphicsFile ( setlinewidth ) writestring
} def
/linetoVirtual {
2 copy lineto
storeNumberPairToGraphicsFile
graphicsFile (lineto ) writestring
} def
/strokeVirtual {
stroke
graphicsFile (stroke ) writestring
} def
/newpathVirtual {
newpath
graphicsFile (newpath ) writestring
} def
/closepathVirtual {
closepath
graphicsFile ( closepath ) writestring
} def
/fillVirtual {
fill
graphicsFile ( fill ) writestring
} def
/shfillPlaneFromLightSourceVirtual
{ dup
(shfill virtual input: ) == dup ==
\fcToString
graphicsFile exch writestring
graphicsFile ( shfillPlaneFromLightSource ) writestring
shfillPlaneFromLightSource
} def
/setrgbcolorVirtual {
3 copy
setrgbcolor %
graphicsFile 4 -1 roll \fcToString writestring graphicsFile ( ) writestring
graphicsFile 3 -1 roll \fcToString writestring graphicsFile ( ) writestring
graphicsFile exch \fcToString writestring
graphicsFile ( setrgbcolor ) writestring
} def
/setdashVirtual {
2 copy setdash
20 string cvs exch \fcToString graphicsFile exch writestring graphicsFile ( ) writestring graphicsFile exch writestring graphicsFile ( setdash ) writestring
} def
/plotArrowHeadVirtual {
2 copy \fcArrowHeadPlotCode\space
\fcToString exch \fcToString graphicsFile exch writestring graphicsFile ( ) writestring graphicsFile exch writestring graphicsFile ( plotArrowHeadVirtual ) writestring
}def
}

\newcommand{\fcSetupFilesGraphicsIIIdCode}{
\fcSetUpGraphicsToScreen
/graphicsFileName (xgraphicsCacheSafeToDelete) (\fcScreen) [[1 0] \fcCoordsPStricksToPS] \fcToString [[0 1] \fcCoordsPStricksToPS] \fcToString 6 -1 roll \fcToString \fcHashString /fileNameId where  {pop /fileNameId load \fcHashString}{()} ifelse (.txt) 7 \fcConcatenateMultiple def
}

%defines graphicsFileAvailable to be true if file can be created, false else.
\newcommand{\fcSetupFilesGraphicsNameAlreadyDefinedCode}{
/graphicsCached false def
/graphicsFileAvailable true def
/graphicsFile (file not open) def
errordict begin
/invalidfileaccess
{ userdict begin /graphicsFileAvailable false def end
(ERROR: failed to open file for caching large IIId rendering operations. ) print
(This is not fatal but will cause your graphics to compile incredibly slowly. I am TURNING OFF Z-buffering since you will have to wait forever for the computation. Your 3d graphics will be drawn by contours only! ) print
(To fix this, compile with pdflatex --shell-escape. ) print
(Make sure the file is executed in a folder with write priviledges. \string\n) print
pop pop
} def
/undefinedfilename{
(File doesn't exist. ) print
 pop pop
} def
end
(Opening file ) print graphicsFileName print (. ) print
/graphicsFile graphicsFileName (r) file def
graphicsFile type (filetype) eq {
/graphicsCached true def
\fcPaintCachedFile
graphicsFile closefile
}
{ /graphicsFile graphicsFileName (w) file def
  graphicsFile type (filetype) ne{
  /graphicsFileAvailable false def
  }
  { \fcSetUpGraphicsToFileAndScreen
    (SET UP graphics TO FILE AND SCREEN) ==
  }
  ifelse
}ifelse
}

\newcommand{\fcInitializeFunctionsCode}{%
/shfillPlaneFromLightSource {\fcShFillPlaneFromLightSourceCode} def %
}

\newcommand{\fcInitializeAndSetupFilesCode}{
\fcInitializeFunctionsCode
\fcSetupFilesGraphicsIIIdCode
\fcSetupFilesGraphicsNameAlreadyDefinedCode
}

\newcommand{\fcComputePatchesAndContours}{
/totalNumPatches 0 def
/totalNumContours 0 def
\fcMarkClean
theIIIdObjects {\fcObjectGetNumPatches /totalNumPatches exch totalNumPatches add def} forall
\fcMarkCleanCheck
\fcMarkClean
theIIIdObjects {\fcObjectGetNumContours /totalNumContours exch totalNumContours add def} forall
\fcMarkCleanCheck
/thePatchCollection [totalNumPatches {(empty)}repeat] def
/theContourCollection [totalNumContours{(empty)} repeat] def
(\string\n) print (Expected number of patches, contours: ) print
thePatchCollection length 20 string cvs print (, ) print
theContourCollection length 20 string cvs print
(. Computing patches and contours ...) print
/numPatchesComputedSoFar 0 def
/numContoursComputedSoFar 0 def
/counter -1 def
theIIIdObjects length {
/counter counter 1 add def
theIIIdObjects counter get \fcObjectComputePatchesAndContours
theIIIdObjects counter get \fcObjectGetNumPatches
/numPatchesComputedSoFar exch numPatchesComputedSoFar add def
theIIIdObjects counter get \fcObjectGetNumContours
/numContoursComputedSoFar exch numContoursComputedSoFar add def
} repeat
thePatchCollection{(empty) eq {(ERROR: declared patch not computed!)}if }forall
theContourCollection{(empty) eq {(ERROR: declared contour not computed!)}if }forall
(... computing patches and contours done!\string\n) print
}

\newcommand{\fcPaintPatches}{
thePatchIndices {\fcPaintPatchIndexFilledDirectly} forall
}

\newcommand{\fcPaintPatchLabels}{
/counter -1 def
0 0 0 setrgbcolorVirtual
thePatchCollection length
{/counter counter 1 add def
counter thePatchCollection counter get \fcPatchPaintLabel
}repeat
}

\newcommand{\fcPaintPatchSortOrder}{
/counter -1 def
0 0 0 setrgbcolorVirtual
thePatchIndices length
{/counter counter 1 add def
counter thePatchCollection thePatchIndices counter get get \fcPatchPaintLabel
}repeat
}

\newcommand{\fcPaintContours}{
theContourCollection{\fcPaintContour} forall %
}

\newcommand{\fcProcessCurrentZBuffer}{
/counterI -1 def
%(currentZBuffer is: ) ==
%currentZBuffer ==
currentZBuffer length {
/counterI counterI 1 add def
/counterJ -1 def
currentZBuffer length{
/counterJ counterJ 1 add def
counterI counterJ ne{
/leftIndex currentZBuffer counterI get def
/rightIndex currentZBuffer counterJ get def
rightIndex leftIndex \fcLeftPatchIsBehind {
rightIndex thePatchIncidenceGraph leftIndex get \fcContains not {
thePatchIncidenceGraph leftIndex
[ thePatchIncidenceGraph leftIndex get \fcArrayToStack rightIndex
]
put
}if
}if
}if
}repeat
}repeat
}

\newcommand{\fcComputePatchOrder}{
/thePatchIncidenceGraph [thePatchCollection length {[]}repeat] def
graphicsFileAvailable {
/rowCounter -1 def
(computing patch order... ) print
(processing z-buffer row, column:) print
theZBuffer length {
/rowCounter rowCounter 1 add def
/columnCounter -1 def
theZBuffer rowCounter get length {
/columnCounter columnCounter 1 add def
/currentZBuffer theZBuffer rowCounter get columnCounter get def
% % % % %
(\string\n) print rowCounter == columnCounter ==
(out of: ) print theZBuffer length == theZBuffer rowCounter get length ==
% % % % %
\fcProcessCurrentZBuffer
}repeat
}repeat
(... computing patch order done ) print
}if
}

\newcommand{\fcSortPatchIndices}{
(sorting a total of ) print thePatchCollection length == ( patches... ) print
fastPatchSort
{ (\string\n Sorting patches by their z-depth. This may be inaccurate but is fast. ) print
  /counter -1 def
  /thePatchIndices [thePatchCollection length {/counter counter 1 add def counter} repeat] def
  20 dict begin
  /LeftGreaterThanRight
  { 4 dict begin
  	/rightPatch exch thePatchCollection exch get def
  	/leftPatch  exch thePatchCollection exch get def
    rightPatch \fcPatchGetForcedForegroundStatus leftPatch \fcPatchGetForcedForegroundStatus not and {true}{
    leftPatch \fcPatchGetForcedForegroundStatus rightPatch \fcPatchGetForcedForegroundStatus not and
    {false}{
  	leftPatch  \fcPatchGetPoint \fcScreen\space pop \fcVectorScalarVector
  	rightPatch \fcPatchGetPoint \fcScreen\space pop \fcVectorScalarVector
  	lt
  	}ifelse}ifelse
    end
  } def
  thePatchIndices \fcMergeSort
  end
  /thePatchIndices exch def
  (\string\n Sorting patches done. ) print
}
{
(\string\n Sorting patches via the partial order induced by their visibility. This is slow but somewhat accurate. ) print
%thePatchIncidenceGraph shall have one entry for each patch.
%The entry will consist of a list of all patch indices of
%patches that are behind our patch.
/thePatchIndices [thePatchCollection length {(empty)} repeat] def
20 dict begin
\fcComputePatchOrder
(patch incidence graph:) ==
thePatchIncidenceGraph ==
/accountPatch {
  /patchIndex exch def
  (accounting patch: ) print
  patchIndex ==
  /numAccountedLastCycle numAccountedLastCycle 1 add def
  accountedPatches patchIndex true put
  thePatchIndices numAccountedSoFar patchIndex put
  /numAccountedSoFar numAccountedSoFar 1 add def
} def
/accountedPatches [thePatchCollection length {false} repeat] def
/numAccountedSoFar 0 def
{
/numAccountedLastCycle 0 def
/patchIndex -1 def
thePatchIncidenceGraph length {
/patchIndex patchIndex 1 add def
accountedPatches patchIndex get not
{/patchIsNext true def
/patchBehindIndex -1 def
/patchesBehindCurrent thePatchIncidenceGraph patchIndex get def
patchesBehindCurrent length{
/patchBehindIndex patchBehindIndex 1 add def
accountedPatches patchesBehindCurrent patchBehindIndex get get not{
/patchIsNext false def exit
}if
}repeat
patchIsNext{
patchIndex accountPatch
}if
}if
}repeat
numAccountedLastCycle 0 eq{
(We have cyclically overlapping patches!) ==
/patchIndex -1 def
thePatchIncidenceGraph length {
/patchIndex patchIndex 1 add def
accountedPatches patchIndex get not
{ patchIndex accountPatch exit
}if
}repeat
}if
numAccountedSoFar thePatchIndices length eq{exit} if
}loop
(sorting patches done) print
%thePatchIndices ==
%(thePatchCollection: ) ==
%thePatchCollection ==
end
}ifelse
}

\newcommand{\fcFileStoreBoundingBoxCode}{
graphicsFileAvailable graphicsCached not and {%
graphicsFile (/ZBufferRectangle ) writestring
graphicsFile ZBufferRectangle \fcToString writestring
graphicsFile ( def ) writestring
graphicsFile (/ZMinMax ) writestring
graphicsFile ZMinMax \fcToString writestring
graphicsFile ( def ) writestring
}if%
}

\newcommand{\fcFinishIIIdScene}[1][fastsort=false]{%
\setkeys{fcGraphics}{#1}%
\pscustom{%
\code{%
%print the objects we are about to paint:
theIIIdObjects length 0 gt { %
%(about to process IIId scene given by: ) print %
%theIIIdObjects == %
} if %
70 dict begin %
theIIIdObjects \fcInitializeAndSetupFilesCode
graphicsCached not{
/fastPatchSort \fcFastPatchSort\space def
% % % % % % % % % % % % % % % % % % %
\fcComputePatchesAndContours
% % % % % % % % % % % % % % % % % % %
\fcZBufferInitialize %
(computing bounding box IIId scene... ) print %
theContourCollection {\fcZBufferBoundingBoxPatchContour} forall
thePatchCollection {\fcZBufferBoundingBoxPatch} forall
%extend slightly the bounding box to take care of floating point errors at the
%border
getZBufferXmin 0.1 sub setZBufferXmin %
getZBufferXmax 0.1 add setZBufferXmax %
getZBufferYmin 0.1 sub setZBufferYmin %
getZBufferYmax 0.1 add setZBufferYmax %
getZmin 0.1 sub setZmin
getZmax 0.1 add setZmax
\fcFileStoreBoundingBoxCode
(bounding box computed: ) == ZBufferRectangle == %
(Depth interval: ) == ZMinMax == %
% % % % % % % % % % % % % % % % % % % % % % % % % %
(computing z-buffer IIId scene... ) print %
/counter -1 def
thePatchCollection length { /counter counter 1 add def counter \fcZBufferPatchIndex} repeat %
(z buffer computed.) print %
% % % % % % % % % % % % % % % % % % % % % % % % % %
\fcSortPatchIndices
\fcPaintPatches
\fcPaintContours
%\fcPaintPatchSortOrder
%\fcPaintPatchLabels
%[
%thePatchCollection { \fcPatchGetPoint \fcScreen \space pop \fcVectorScalarVector } forall
%] ==
% % % % % %
%(z buffer sorted) print %
% % % % % % % % % % % % % % % % % % % % % % % % % %
%\fcZBufferPrint %
%(painting patches) ==
%\fcZBufferPaintPatches
% %
%\fcPaintZbuffForDebug %
% %
end %
(done, printing stack to make sure no trash is left) == %
pstack %
} %if graphics is not cached
if
}%
}%
\pstVerb{end}%
}%

\newcommand{\fcObjectGetNumContours}{%
\fcArrayToStack %
1 dict begin %
/HandlerNotFound true def %
HandlerNotFound{dup (surface) eq {pop \fcSurfaceGetNumContours /HandlerNotFound false def} if} if %
HandlerNotFound{dup (curve) eq {pop \fcCurveInit 1 /HandlerNotFound false def} if} if %
HandlerNotFound{dup (triangle) eq {pop \fcTriangleInSceneInit 1 /HandlerNotFound false def} if} if %
HandlerNotFound{== (ERROR: OBJECT GET NUMBER OF CONTOURS HANDLER NOT FOUND)} if %
end %
}%

\newcommand{\fcObjectGetNumPatches}{%
\fcArrayToStack %
1 dict begin %
/HandlerNotFound true def %
HandlerNotFound{dup (surface) eq {pop \fcSurfaceGetNumPatches /HandlerNotFound false def} if} if %
HandlerNotFound{dup (curve) eq {pop \fcCurveInit 0 /HandlerNotFound false def} if} if %
HandlerNotFound{dup (triangle) eq {pop \fcTriangleInSceneInit 1 /HandlerNotFound false def} if} if %
HandlerNotFound{== (ERROR: OBJECT NUMBER OF PATCHES HANDLER NOT FOUND)} if %
end %
}%

\newcommand{\fcObjectComputePatchesAndContours}{%
\fcMarkClean exch
\fcArrayToStack %
1 dict begin %
/HandlerNotFound true def %
HandlerNotFound{dup (surface) eq {pop \fcSurfaceComputePatchesAndContours /HandlerNotFound false def} if} if %
HandlerNotFound{dup (curve) eq {pop \fcCurveComputeContour /HandlerNotFound false def} if} if %
HandlerNotFound{dup (triangle) eq {pop \fcTriangleComputePatchesAndContours /HandlerNotFound false def} if} if %
HandlerNotFound{== (ERROR: OBJECT PATCH-CONTOUR HANDLER NOT FOUND)} if %
end %
\fcMarkCleanCheck
}%

\newcommand{\fcZBufferBoundingBoxPatchContour}{ %
2 dict begin %
/theContour exch def %
/counter -1 def %
theContour length 1 sub { /counter counter 1 add def theContour counter get 0 get \fcZBufferBoundingBoxPoint}repeat %
end %
}

\newcommand{\fcCheckIsArray}{type (arraytype) eq\space}

\newcommand{\fcPaintPointForegroundData}{
20 dict begin %
gsave
/theNeighborhood exch def %
/thePoint theNeighborhood 0 get def %
(theNeighborhood:)==
theNeighborhood ==
thePoint \fcZBufferRowColumn
/column exch def
/row exch def
/theZBufferCurrent theZBuffer row get column get def
/counter 0 def
/numPatchesInNeighborhood 0 def %
theNeighborhood length 1 sub {
/counter counter 1 add def
1.8 setlinewidth
0.5 1 0.5 setrgbcolor
theNeighborhood counter get theZBufferCurrent \fcContains not
{3 setlinewidth 1 0 0 setrgbcolor
}if
theNeighborhood counter get \fcPatchPaintContourDirectly
theNeighborhood counter get type (arraytype) eq
{/numPatchesInNeighborhood numPatchesInNeighborhood 1 add def}if
}repeat
/counter -1 def
theZBufferCurrent length {
/counter counter 1 add def %
1.1 setlinewidth
1 0.7 0.7 setrgbcolor
thePoint thePatchCollection theZBufferCurrent counter get get \fcPointIsBehindOrInFrontOfPatch{true}
{1 0 0 setrgbcolor
}if
thePatchCollection theZBufferCurrent counter get get \fcPatchPaintContourDirectly
}repeat
thePoint \fcZBufferPaintCellContainingPoint %
[thePoint \fcCoordsIIIdToPStricks] theNeighborhood \fcIsInForeground
{\fcFullDotCode}{\fcHollowDotCode}ifelse %
0 0 0 setrgbcolor
(nsize: ) numPatchesInNeighborhood 20 string cvs \fcConcatenate [thePoint \fcCoordsIIIdToPStricks ] \fcTextCode
(bsize: ) theZBufferCurrent length 20 string cvs \fcConcatenate [thePoint \fcCoordsIIIdToPStricks 0.2 sub ] \fcTextCode
grestore
end
}

\newcommand{\fcZBufferRowColumnIsInvestigated}{
\fcZBufferRowColumnsUnderInvestigation\space column eq exch row eq and
}

\newcommand{\fcArrowHeadAndTailPlotCode}{
2 copy
10 dict begin
/pointRight exch def
/pointLeft exch def
newpath %
pointLeft \fcCoordsPStricksToPS moveto %
pointRight \fcCoordsPStricksToPS lineto %
stroke %
end
\fcArrowHeadPlotCode
}

\makeatletter
\newcommand{\fcArrowHeadPlotCode}{
10 dict begin
/pointRight exch [ exch \fcCoordsPStricksToPS ] def
/pointLeft  exch [ exch \fcCoordsPStricksToPS ] def
 gsave
/directionVector [pointRight pointLeft \fcVectorMinusVector \fcArrayToStack] def
directionVector \fcVectorNorm 0 ne{
/directionVector directionVector \fcVectorNormalize def
}if
/xPS directionVector 0 get def
/yPS directionVector 1 get def
[yPS  xPS -1 mul  -1 xPS mul -1 yPS mul pointRight \fcArrayToStack] concat
%[1 0 0 1  pointRight 0 get \fcCoordsIIIdToPS ] concat
 newpath
 false 0.4 1.5 0.4 0.5 \tx@Arrow
 closepath
 stroke
 grestore
end
}
\makeatother

\newcommand{\fcPointOnContourGetVisibility}{
dup length 1 sub get\space
}

\newcommand{\fcContourGetUseArrows}{dup length 1 sub get 0 get\space}
\newcommand{\fcContourGetColor}    {dup length 1 sub get 1 get\space}
\newcommand{\fcContourGetWidth}    {dup length 1 sub get 2 get\space}
\newcommand{\fcContourGetDashes}   {dup length 1 sub get 3 get\space}
\newcommand{\fcContourGetLineStyle}{dup length 1 sub get 4 get\space}

\newcommand{\fcPaintContour}{ %
20 dict begin %
/theContour exch def %
theContour \fcContourGetLineStyle (none) ne{
/numSegments theContour length 2 sub def %
/counter -1 def %
/rightIsInForeground theContour 0 get \fcIsInForeground def %
/style (none) def %
/useArrows theContour \fcContourGetUseArrows def
/currentDashes {theContour \fcContourGetDashes \fcArrayToStack} def
theContour \fcContourGetWidth setlinewidthVirtual
theContour \fcContourGetColor \fcArrayToStack setrgbcolorVirtual
numSegments{ %
  /counter counter 1 add def %
  /pointLeft theContour counter get def %
  /pointRight theContour counter 1 add get def %
  /leftIsInForeground rightIsInForeground def %
  /rightIsInForeground pointRight \fcIsInForeground def %
  /leftIsVisible pointLeft \fcPointOnContourGetVisibility def %
  /rightIsVisible pointRight \fcPointOnContourGetVisibility def %
  /oldStyle style def %
  /style leftIsInForeground rightIsInForeground or {(normal)}{(dashed)}ifelse def %
  /setStyle {style (normal) eq{[] 0 setdashVirtual}{currentDashes setdashVirtual}ifelse } def
  leftIsVisible rightIsVisible and{
	counter 0 eq{newpathVirtual}if
    style oldStyle ne{strokeVirtual setStyle %
    newpathVirtual pointLeft 0 get \fcCoordsIIIdToPS movetoVirtual %
    }if %
    pointRight 0 get \fcCoordsIIIdToPS linetoVirtual
  }if
  leftIsVisible rightIsVisible not and{
    strokeVirtual
    /style (invisible) def
  }if
  leftIsVisible not rightIsVisible and{
    setStyle
    newpathVirtual pointRight 0 get \fcCoordsIIIdToPS movetoVirtual %
  }if
}repeat %
strokeVirtual %
useArrows{[] 0 setdashVirtual [pointLeft 0 get \fcCoordsIIIdToPStricks] [pointRight 0 get \fcCoordsIIIdToPStricks] plotArrowHeadVirtual}if %
}if %
end %
}

\newcommand{\fcCurveInit}{
/contourOptions exch def
/theCurve exch def %
/tMax exch def %
/tMin exch def %
/tIterations \fcPlotPoints\space def %
}

\newcommand{\fcCurveComputeContour}{
30 dict begin %
\fcCurveInit
/DeltaT tMax tMin sub tIterations div def %
/t tMin def %
theContourCollection numContoursComputedSoFar
[
tIterations { %
[theCurve true]
/t t DeltaT add def %
}repeat %
contourOptions
]
put
end %
}

\newcommand{\fcComputeSurfacePatch}{
/oldU u def %
/oldV v def %
/inBounds true def
[ %start of patch data structure
theSurface %(x(u,v), y(u,v), z(u,v))
theRestrictions not {/inBounds false def}if
/u u DeltaU add def %
theSurface %(x(u+Delta,v), y(u+Delta,v), z(u+Delta,v))
theRestrictions not {/inBounds false def}if
/u oldU def %
/v v DeltaV add def %
theSurface %(x(u,v+Delta), y(u,v+Delta), z(u,v+Delta))
theRestrictions not {/inBounds false def}if
/u u DeltaU add def %
theSurface %(x(u+Delta,v+Delta), y(u+Delta,v+Delta), z(u+Delta,v+Delta))
theRestrictions not {/inBounds false def}if
[
/u oldU def
/v oldV def
numContourUSegmentsPerPatch {theRestrictions{theSurface}if /u u DeltaDeltaU add def} repeat
/u oldU DeltaU add def
/v oldV def
numContourVSegmentsPerPatch {theRestrictions{theSurface}if /v v DeltaDeltaV add def} repeat
/v oldV DeltaV add def
numContourVSegmentsPerPatch {theRestrictions{theSurface}if /u u DeltaDeltaU sub def} repeat
/u oldU def
numContourVSegmentsPerPatch {theRestrictions{theSurface}if /v v DeltaDeltaV sub def} repeat
]
[colorUVpatch % front color of patch
colorVUpatch % back color of patch
forceForeground %
inBounds %
colorVUpatch %patch built-in contour color, used when drawing patch contour directly
false %is contour dashed independent of visibility?
[] %dashes not used
]
(patch) %
] %patch data structure complete
/u oldU def %
/v oldV def %
}

\newcommand{\fcSurfaceGetNumContours}{
30 dict begin
\fcSurfaceInit %
uIterations 1 add vIterations 1 add add
end
}

\newcommand{\fcSurfaceGetNumPatches}{
30 dict begin
\fcSurfaceInit %
uIterations vIterations mul
end
}

\newcommand{\fcSurfaceInit}{%
/contourOptions exch def%
/patchOptions exch def%
/forceForeground patchOptions 2 get def%
/colorUVpatch patchOptions 0 get def%
/colorVUpatch patchOptions 1 get def%
/vIterations exch def %
/uIterations exch def %
/theRestrictions exch def
/theSurface exch def %
/vMax exch def %
/uMax exch def %
/vMin exch def %
/uMin exch def %
}

\newcommand{\fcSurfaceComputePatchesAndContours}{%
30 dict begin %
{ %begin loop, used to simulate a jump instruction
%exiting the loop = jumping at loop end
\fcSurfaceInit
/u uMin def
/v vMin def
mark
theSurface type (arraytype) ne {(\string\n ERROR: surface must be an array. \string\n) print cleartomark exit}if cleartomark
\fcMarkClean
/DeltaU uMax uMin sub uIterations div def %
/DeltaV vMax vMin sub vIterations div def %
/getPatchIndex {exch vIterations mul add numPatchesComputedSoFar add} def
/accountPatch {getPatchIndex dup thePatchCollection exch get (empty) eq {thePatchCollection exch 3 -1 roll put}{pop pop}ifelse
} def
/numContourVSegmentsPerPatch \fcNumCountourSegmentsPatchV \space def
/numContourUSegmentsPerPatch \fcNumCountourSegmentsPatchU \space def
/DeltaDeltaV DeltaV numContourVSegmentsPerPatch div def %
/DeltaDeltaU DeltaU numContourUSegmentsPerPatch div def %
% % % % % % % % % % % % % % % % % % % % %
%process uv-contours
/u uMin def %
/counterU -1 def %
uIterations 1 add{ %
/counterU counterU 1 add def %
/counterV -1 def %
/v vMin def %
/patchLeftIndex (empty) def %
/patchRightIndex (empty) def %
[
vIterations { %
/counterV counterV 1 add def
/patchLeftBackIndex patchLeftIndex def %
/patchRightBackIndex patchRightIndex def %
/patchLeftIndex counterU uIterations lt
{\fcComputeSurfacePatch counterU counterV accountPatch counterU counterV getPatchIndex}
{(empty)}
ifelse def %
/patchRightIndex counterU 0 ne {counterU 1 sub counterV getPatchIndex}
{(empty)} ifelse def %
/vOld v def %
[theSurface patchLeftIndex patchRightIndex patchLeftBackIndex patchRightBackIndex theRestrictions]
numContourUSegmentsPerPatch 1 sub{/v v DeltaDeltaV add def [theSurface patchLeftIndex patchRightIndex theRestrictions]} repeat
/v vOld DeltaV add def %
}repeat %
[theSurface patchLeftIndex patchRightIndex theRestrictions]
contourOptions
]
/thePatchContour exch def %
%/contourIsUnderInvestigation \fcZBufferVparameterPointUnderInvestigation\space pop counterU eq def
%/indexPointUnderInvestigation
%\fcZBufferVparameterPointUnderInvestigation\space exch pop def
theContourCollection counterU numContoursComputedSoFar add thePatchContour put
/uOld u def %
/u u DeltaU add def %
} repeat %
% % % % % % % % % % % % % % % % % % % % %
%process vu-contours
/v vMin def %
/counterV -1 def %
vIterations 1 add{ %
  /counterV counterV 1 add def %
  /counterU -1 def
  /u uMin def %
  /patchLeftIndex (empty) def %
  /patchRightIndex (empty) def %
  [
  uIterations { %
  /counterU counterU 1 add def
  /patchLeftBackIndex patchLeftIndex def %
  /patchRightBackIndex patchRightIndex def %
  /patchLeftIndex counterV vIterations lt
  {counterU counterV getPatchIndex}
  {(empty)}
  ifelse
  def %
  /patchRightIndex counterV 0 ne {counterU counterV 1 sub getPatchIndex} {(empty)} ifelse def %
  /uOld u def %
  [theSurface patchLeftIndex patchRightIndex patchLeftBackIndex patchRightBackIndex theRestrictions]
  numContourVSegmentsPerPatch 1 sub{/u u DeltaDeltaU add def [theSurface patchLeftIndex patchRightIndex theRestrictions]} repeat
  /u uOld DeltaU add def %
  }repeat %
  [theSurface patchLeftIndex patchRightIndex theRestrictions]
  contourOptions
  ]
  /thePatchContour exch def %
  %/contourIsUnderInvestigation \fcZBufferUparameterPointUnderInvestigation\space pop counterV  eq def
  %/indexPointUnderInvestigation
  %\fcZBufferUparameterPointUnderInvestigation\space exch pop def
  theContourCollection uIterations 1 add counterV add numContoursComputedSoFar add thePatchContour put
  /vOld v def %
  /v v DeltaV add def %
} repeat %
\fcMarkCleanCheck
exit
}loop %this loop is used to simulate a jump instruction
end %
}%

\newcommand{\fcGetColorCode}[1]{%
2 dict begin%
/theColor {0 0 0} def%
/colorNotFound true def%
(#1) (black) eq (#1) (black ) eq or{/theColor {0 0 0} def /colorNotFound false def}if%
(#1) (white) eq (#1) (white ) eq or{/theColor {1 1 1} def /colorNotFound false def}if%
(#1) (red) eq (#1) (red ) eq or{/theColor {1 0 0} def /colorNotFound false def}if%
(#1) (blue) eq (#1) (blue ) eq or{/theColor {0 0 1} def /colorNotFound false def}if%
(#1) (green) eq (#1) (green ) eq or{/theColor {0 1 0} def /colorNotFound false def}if%
(#1) (brown) eq (#1) (brown ) eq or{/theColor {0.6484375 0.1640625 0.1640625} def /colorNotFound false def}if%
(#1) (orange) eq (#1) (orange ) eq or{/theColor {1 0.647058824 0} def /colorNotFound false def}if%
(#1) (cyan) eq (#1) (cyan ) eq or {/theColor {0 1 1} def /colorNotFound false def}if %
(#1) (pink) eq (#1) (pink ) eq or {/theColor {1 0.75390625 0.796875} def /colorNotFound false def}if %
(#1) (gray) eq (#1) (gray ) eq or {/theColor {0.5 0.5 0.5} def /colorNotFound false def}if %
colorNotFound{/theColor {#1} def}if %
theColor %
end%
}%

\newcommand{\fcLineIIIdInScene}[3][arrows=,]{%
\fcCurveIIIdInScene[#1]{0}{1}{#3 t \fcVectorTimesScalar #2 1 t sub \fcVectorTimesScalar \fcVectorPlusVector}%
}

%Format of curve: we store the curve in the format [tmin tmax [x y z] arrows red green blue (curve)]
%Example use:
%\begin{pspicture}(-1,-1)(1,1)
%\fcStartIIIdScene
%\fcCurveIIIdInScene[linecolor=orange]{0}{1}{[t t sqrt 0 ]}
%\fcFinishIIIdScene[true]
%\end{pspicture}
\newcommand{\fcCurveIIIdInScene}[4][]{%
\setkeys{fcGraphics}{#1}%
\pstVerb{%
[theIIIdObjects \fcArrayToStack [#2\space #3{#4} \fcContourOptions (curve)] ]/theIIIdObjects exch def}%
}%

\newcommand{\fcSegmentCodeNoFirstPoint}{
5 dict begin
/right exch def
/left exch def
/Delta 1 \fcNumCountourSegmentsPatchU div def
/t Delta def
\fcNumCountourSegmentsPatchU {left 1 t sub \fcVectorTimesScalar right t \fcVectorTimesScalar \fcVectorPlusVector /t t Delta add def} repeat
end
}

\newcommand{\fcTriangleComputePatchesAndContours}{
25 dict begin
\fcTriangleInSceneInit
\fcMarkClean
/currentPatch
[ vertex0 vertex1 vertex2 vertex1 vertex2 \fcVectorPlusVector 0.5
\fcVectorTimesScalar
[ vertex0 dup vertex1 \fcSegmentCodeNoFirstPoint vertex1 vertex2 \fcSegmentCodeNoFirstPoint vertex2 vertex0 \fcSegmentCodeNoFirstPoint]
patchOptions (patch)
] def
/currentContour
[
currentPatch \fcPatchGetContour {[exch true]}forall
contourOptions
]
def
thePatchCollection numPatchesComputedSoFar currentPatch put
theContourCollection numContoursComputedSoFar currentContour put
\fcMarkCleanCheck
end
}

\newcommand{\fcTriangleInSceneInit}{
/contourOptions exch def
/patchOptions exch def
/vertex2 exch def
/vertex1 exch def
/vertex0 exch def
}

\newcommand{\fcBoxIIIdInScene}[5][]{%
\fcPatchInScene[#1]{#2}{#4}{#3}%
\fcPatchInScene[#1]{#2}{#5}{#4}%
\fcPatchInScene[#1]{#2}{#3}{#5}%
\fcPatchInScene[#1]{#3}{#3 #4 #2 \fcVectorMinusVector \fcVectorPlusVector}{#3 #5 #2 \fcVectorMinusVector \fcVectorPlusVector}%
\fcPatchInScene[#1]{#4}{#4 #5 #2 \fcVectorMinusVector \fcVectorPlusVector}{#4 #3 #2 \fcVectorMinusVector \fcVectorPlusVector}%
\fcPatchInScene[#1]{#5}{#5 #3 #2 \fcVectorMinusVector \fcVectorPlusVector}{#5 #4 #2 \fcVectorMinusVector \fcVectorPlusVector}%
}

%We give the patch by its corners v0, v1, v2.
\newcommand{\fcTriangleInScene}[4][]{%
\setkeys{fcGraphics}{#1}%
\pstVerb{%
/theIIIdObjects [theIIIdObjects \fcArrayToStack [#2\space #3\space #4\space \fcPatchOptions \fcContourOptions (triangle)] ] def%
}%
}

%Format of patch: we give the patch by its corners v0, v1, v2. The patch is spanned by v1-v0 and v2-v0
\newcommand{\fcPatchInScene}[4][]{%
\fcSurfaceInScene[#1, iterationsU=1, iterationsV=1]{0}{0}{1}{1}{%
3 dict begin %
/v0 #2\space def %
/t1 #3\space v0 \fcVectorMinusVector def %
/t2 #4\space v0 \fcVectorMinusVector def %
v0 %
t1 u \fcVectorTimesScalar %
t2 v \fcVectorTimesScalar %
\fcVectorPlusVector \fcVectorPlusVector %
end %
}{}%
}

%Format of surface: we store the surface in the format
%[umin vmin umax vmax
%[x(u,v) y(u,v) z(u,v)] %coordinate functions
% restrictions %boolean function in u,v, when the function evaluates to false the surface is not drawn. A great tool for cutting surfaces.
%uIterations vIterations   %number of curvilinear u,v-segments
%[red green blue] %color of patches whose u,v-side is visible
%[red green blue] %color of patches whose v,u-side is visible
%[red green blue] %color of contours
% foregroundstatus %true or false
%(surface)].
%Example:
%\begin{pspicture}
%\fcStartIIIdScene
%\fcSurfaceInScene[arrows=(none), linecolor=red]{0}{0}{1}{360}{[u u sqrt v cos mul u sqrt v sin mul]}{}
%\fcFinishIIIdScene[true]
%\end{pspicture}
\newcommand{\fcSurfaceInScene}[7][arrows=none]{%
\setkeys{fcGraphics}{#1}%
\pstVerb{%
[theIIIdObjects \fcArrayToStack [#2\space #3\space #4\space #5\space {#6} {#7} length 0 ne {{#7}}{{true}}ifelse \fcIterationsU\space \fcIterationsV%
\fcPatchOptions%
\fcContourOptions%
(surface)] ]/theIIIdObjects exch def}%
}%

\newcommand{\fcCurveIIIdNoSceneCode}{%
15 dict begin%
/theCurve exch def%
/tMax exch def%
/tMin exch def%
/numPoints \fcPlotPoints\space def%
/Delta tMax tMin sub numPoints 1 sub div def%
/t tMin def %
\fcLineFormatCode %
newpath %
theCurve \fcCoordsIIIdToPS moveto %
numPoints 1 sub {/t t Delta add def theCurve \fcCoordsIIIdToPS lineto } repeat %
stroke %
end %
}%

\newcommand{\fcZBufferBoundingBoxPolyline}{ %
{\fcZBufferBoundingBoxPoint} forall
}

\newcommand{\fcContains}{ %
%format: theElement theArray -> true if theElement is in the array, false else.
3 dict begin %
/theArray exch def %
/theElement exch def %
false %
/counter 0 def %
theArray length { %
theElement theArray counter get \fcAreEqual {pop true exit}if %
/counter counter 1 add def %
}repeat %
end %
}

\newcommand{\fcAreEqual}{ %
1 dict begin
/areEqual
{5 dict begin
/left exch def
/right exch def
left type  right type ne{false}
{ left type (arraytype) ne{ left right eq}
{ left length right length ne{false}
{ /counter 0 def %
true %
left length { left  counter get right counter get areEqual not{pop false exit }if /counter counter 1 add def }repeat
}ifelse
}ifelse
}ifelse
end
} def %
areEqual
end
}

\newcommand{\fcZBufferAccountPatchIndexAtXY}{ %
5 dict begin %
/currentArray theZBuffer row get column get def %
/counter 0 def %
thePatchIndex currentArray \fcContains not
{theZBuffer row get column [currentArray \fcArrayToStack thePatchIndex ] put}if
end %
}

\newcommand{\fcMergeSort}{
2 dict begin
/theArray exch def
/mergeSortStartIndexLength {
10 dict begin
/theLength exch def
/startIndex exch def
{ theLength 1 eq
  { [theArray startIndex get]
  	exit
  } if
  theLength 2 eq
  { theArray startIndex get theArray startIndex 1 add get
    LeftGreaterThanRight
    { [theArray startIndex 1 add get theArray startIndex get]
    }
    { [theArray startIndex get theArray startIndex 1 add get]
    } ifelse
    exit
  } if
  /leftLength theLength 2 div cvi def
  /rightLength theLength leftLength sub def
  /leftSorted startIndex leftLength mergeSortStartIndexLength def
  /rightSorted startIndex leftLength add rightLength mergeSortStartIndexLength def
  /counterLeft 0 def
  /counterRight 0 def
  [ theLength{
      counterLeft leftLength ge{
        /leftNext false def
      }{
        counterRight rightLength ge{
          /leftNext true def
        }{
          /leftNext leftSorted counterLeft get rightSorted counterRight get  LeftGreaterThanRight not def
        } ifelse
      } ifelse
      leftNext{
        leftSorted counterLeft get /counterLeft counterLeft 1 add def
      }{
        rightSorted counterRight get /counterRight counterRight 1 add def
      }ifelse
    } repeat
  ]
  exit
}loop
end
} def
theArray length 0 gt {
  0 theArray length mergeSortStartIndexLength
}{
  theArray
} ifelse
end
}

\newcommand{\fcBubbleSort}{
5 dict begin
/a exch def
/n a length 1 sub def
n 0 gt {
% at this point only the n+1 items in the bottom of a remain to be sorted
% the largest item in that block is to be moved up into position n
n {
0 1 n 1 sub {
/i exch def
a i get a i 1 add get LeftGreaterThanRight {
% if a[i] > a[i+1] swap a[i] and a[i+1]
a i 1 add
a i get
a i a i 1 add get
% set new a[i] = old a[i+1]
put
% set new a[i+1] = old a[i]
put
} if
} for
/n n 1 sub def
} repeat
} if
end
}

\newcommand{\fcGaussianElimination}{
15 dict begin
/theMatrix exch def
/numRows theMatrix length def
/numCols theMatrix 0 get length def
/rowIndex 0 def
/columnIndex -1 def
{ /columnIndex columnIndex 1 add def
  columnIndex numCols ge {exit}if
  /candidateIndex -1 def
  /counter rowIndex 1 sub def
  numRows rowIndex sub {
    /counter counter 1 add def
    theMatrix counter get columnIndex get 0 ne {/candidateIndex counter def exit}if
  }repeat
  candidateIndex -1 ne{
    /temp theMatrix rowIndex get def
    theMatrix rowIndex theMatrix candidateIndex get put
    theMatrix candidateIndex temp put
    /pivotRow theMatrix rowIndex get def
    /theCoeff 1 pivotRow columnIndex get div def
    /pivotRow pivotRow  theCoeff \fcVectorTimesScalar def
    theMatrix rowIndex pivotRow put
    /counter -1 def
    numRows {
      /counter counter 1 add def
      counter rowIndex ne{
      theMatrix counter get dup columnIndex get pivotRow exch \fcVectorTimesScalar \fcVectorMinusVector theMatrix exch counter exch put
      }if
    }repeat
    /rowIndex rowIndex 1 add def
  }if
  rowIndex numRows ge {exit}if
}loop
theMatrix
end\space
}

\newcommand{\fcLeftSegmentIsBehindSegmentsAreSkew}{
%Let l1 l2 be the endpoints of the left segment,
%r1, re - the endpoints of the right.
%Let pl1, pl2, pr1, pr2 denote the projections onto the viewing
%screen of the corresponding endpoints.
%Let
%(a,c) be the vector pl1-pl2
%(b,d) be the vector pr2-pr1
%(e,f) be the vector pr2-pl2
%We need to solve the system
%(a b) (s)= (e)
%(c d) (t)= (f)
%The determinant of the system must be non-zero, else the segments are non-skew.
% the projections of the segments intersect if 0 <= s<=1
% and 0<=t<=1.
%In that case the point on the left segment that projects onto the point of interest is s*l1 +(1-s)l2. The point on the right segment that projects onto the point of interest is t*r1 +(1-t)*r2
%
25 dict begin
\fcArrayToStack
/r1 exch def
/r2 exch def
\fcArrayToStack
/l1 exch def
/l2 exch def
/pr1 [r1 \fcCoordsIIIdToPStricks] def
/pr2 [r2 \fcCoordsIIIdToPStricks] def
/pl1 [l1 \fcCoordsIIIdToPStricks] def
/pl2 [l2 \fcCoordsIIIdToPStricks] def
pl1 pl2 \fcVectorMinusVector \fcArrayToStack
/c exch def
/a exch def
pr2 pr1 \fcVectorMinusVector \fcArrayToStack
/d exch def
/b exch def
pr2 pl2 \fcVectorMinusVector \fcArrayToStack
/f exch def
/e exch def
/theMatrix [[a b e][c d f]] \fcGaussianElimination def
/a theMatrix 0 get 0 get def
/d theMatrix 1 get 1 get def
/e theMatrix 0 get 2 get def
/f theMatrix 1 get 2 get def
a 0 eq d 0 eq or {false}{
/s e a div def
/t f d div def
s 0 gt s 1 lt t 0 gt t 1 lt and and and
{
/leftPoint l1 s \fcVectorTimesScalar l2 1 s sub \fcVectorTimesScalar \fcVectorPlusVector def
/rightPoint r1 t \fcVectorTimesScalar r2 1 t sub \fcVectorTimesScalar \fcVectorPlusVector def
leftPoint  \fcScreen\space pop \fcVectorScalarVector
rightPoint \fcScreen\space pop \fcVectorScalarVector
ge
}
{false}
ifelse
}ifelse
end
}

\newcommand{\fcLeftPatchIsBehind}{
2 dict begin
/rightIndex exch def
/leftIndex exch def
leftIndex rightIndex \fcLeftPatchIsBehindOrPatchesIntersect
{
rightIndex leftIndex \fcLeftPatchIsBehindOrPatchesIntersect not
}
{false}ifelse
end
}

\newcommand{\fcLeftPatchIsBehindOrPatchesIntersect}{
%this function compares two patches:
40 dict begin
/rightIndex exch def
/leftIndex exch def
/rightPatch thePatchCollection rightIndex get def
/leftPatch thePatchCollection leftIndex get def
rightPatch \fcPatchGetForcedForegroundStatus leftPatch \fcPatchGetForcedForegroundStatus not and {true}{
leftPatch \fcPatchGetForcedForegroundStatus rightPatch \fcPatchGetForcedForegroundStatus not and
{false}{
/result false def
{ leftPatch \fcPatchGetContour {rightPatch \fcPointIsBehindOrInFrontOfPatch{true} {/result true def exit}if }forall
result {exit}if
rightPatch \fcPatchGetContour {leftPatch \fcPointIsBehindOrInFrontOfPatch{false} {/result true def exit}if }forall
exit
}loop
result
}ifelse
}ifelse
end
}

\newcommand{\fcLeftPatchIsBehindOrPatchesIntersectOLDAndINCORRECT}{
%this function compares two patches:
40 dict begin
/rightIndex exch def
/leftIndex exch def
/rightPatch thePatchCollection rightIndex get def
/leftPatch thePatchCollection leftIndex get def
rightPatch \fcPatchGetForcedForegroundStatus leftPatch \fcPatchGetForcedForegroundStatus not and {true}{
leftPatch \fcPatchGetForcedForegroundStatus rightPatch \fcPatchGetForcedForegroundStatus not and
{false}{
/lv0 leftPatch \fcPatchGetvZero def
/lv1 leftPatch \fcPatchGetvOne def
/lv2 leftPatch \fcPatchGetvTwo def
/lv3 leftPatch \fcPatchGetvThree def
/rv0 rightPatch \fcPatchGetvZero def
/rv1 rightPatch \fcPatchGetvOne def
/rv2 rightPatch \fcPatchGetvTwo def
/rv3 rightPatch \fcPatchGetvThree def
/ls0 [lv0 lv1] def
/ls1 [lv1 lv3] def
/ls2 [lv2 lv3] def
/ls3 [lv0 lv2] def
/rs0 [rv0 rv1] def
/rs1 [rv1 rv3] def
/rs2 [rv2 rv3] def
/rs3 [rv0 rv2] def
/result false def
4 leftIndex eq  3 rightIndex eq and {(4 behind 3?) == }if
{ lv0 rightPatch \fcPointIsBehindOrInFrontOfPatch{true} {/result true def 4 leftIndex eq  3 rightIndex eq and {(4 IS behind 3 for vertex reasons) == }if exit }if
  lv1 rightPatch \fcPointIsBehindOrInFrontOfPatch{true} {/result true def 4 leftIndex eq 3 rightIndex eq and {(4 IS behind 3 for vertex reasons) == }if exit }if
  lv2 rightPatch \fcPointIsBehindOrInFrontOfPatch{true} {/result true def 4 leftIndex eq 3 rightIndex eq and {(4 IS behind 3 for vertex reasons) == }if exit }if
  lv3 rightPatch \fcPointIsBehindOrInFrontOfPatch{true} {/result true def 4 leftIndex eq 3 rightIndex eq and {(4 IS behind 3 for vertex reasons) == }if exit }if
  [ls0 ls1 ls2 ls3]
  { [rs0 rs1 rs2 rs3]
    {
    3 leftIndex eq  4 rightIndex eq and {(3 behind 4? 2nd) == pstack (STACK) ==}if
    %3 leftIndex eq  3 rightIndex eq and {(ere be i) == pstack }if
    %(printint stack) == pstack (printing stack done ) ==
      exch dup 3 -1 roll \fcLeftSegmentIsBehindSegmentsAreSkew{/result true def exit}if
    }forall
    pop
    result {exit}if
  }forall
  exit
}loop
3 leftIndex eq 4 rightIndex eq and {(and the 34 RESULT result is...) == result ==}if
result
}ifelse
}ifelse
end
}

\newcommand{\fcZBufferPaintPatches}{
10 dict begin
gsave
/row -1 def %
theZBuffer length{ %
/row row 1 add def %
/column -1 def %
theZBuffer row get length{ %
/column column 1 add def %
/currentZBuffer theZBuffer row get column get def %
currentZBuffer {\fcPaintPatchIndexFilledDirectly} forall %
} repeat%
}repeat %
grestore
end
}

\newcommand{\fcPatchGetNormal}{
1 dict begin
/thePatch exch def
thePatch \fcPatchGetvOne thePatch \fcPatchGetvZero \fcVectorMinusVector
thePatch \fcPatchGetvTwo thePatch \fcPatchGetvZero \fcVectorMinusVector
\fcVectorCrossVector
end
}

\newcommand{\fcPatchOptions}{%
[[\fcGetColorCode{\fcColorPatchUV}] [\fcGetColorCode{\fcColorPatchVU}] \fcForceForeground \space true [\fcGetColorCode{\fcColorLine}] ]%
}%

\newcommand{\fcPatchGetvZero}{0 get\space}
\newcommand{\fcPatchGetvOne}{1 get\space}
\newcommand{\fcPatchGetvTwo}{2 get\space}
\newcommand{\fcPatchGetvThree}{3 get \space}
\newcommand{\fcPatchGetContour}{4 get\space}
\newcommand{\fcPatchGetColorUV}{5 get 0 get \fcArrayToStack \space}
\newcommand{\fcPatchGetColorVU}{5 get 1 get \fcArrayToStack \space}
\newcommand{\fcPatchGetForcedForegroundStatus}{5 get 2 get\space}
\newcommand{\fcPatchGetInBounds}{5 get 3 get\space}
\newcommand{\fcPatchGetColorContour}{5 get 4 get \fcArrayToStack \space}
\newcommand{\fcPatchGetIsDashed}{5 get 5 get \space}
\newcommand{\fcPatchGetDashes}{5 get 6 get \fcArrayToStack \space}

\newcommand{\fcPaintPatchIndexFilledDirectly}{
thePatchCollection exch get \fcPatchPaintFilledDirectly
}

\newcommand{\fcPatchGetPoint}{
1 dict begin
/thePatch exch def %
thePatch \fcPatchGetvZero
thePatch \fcPatchGetvOne
thePatch \fcPatchGetvTwo
thePatch \fcPatchGetvThree
\fcVectorPlusVector
\fcVectorPlusVector
\fcVectorPlusVector
0.25 \fcVectorTimesScalar
end
}

\newcommand{\fcPatchPaintLabel}{
3 dict begin
/thePatch exch def
20 string cvs
/Times-Roman findfont
4 scalefont
setfont
newpath
thePatch \fcPatchGetPoint \fcCoordsIIIdToPS moveto
show
stroke
end
}

\newcommand{\fcPlotArrow}{
2 copy
exch
newpath
\fcCoordsIIIdToPS moveto
\fcCoordsIIIdToPS lineto
stroke
exch [ exch \fcCoordsIIIdToPStricks ] exch [exch \fcCoordsIIIdToPStricks ] plotArrowHeadVirtual
}

\newcommand{\fcPatchPaintNormal}{
1 dict begin
/thePatch exch def
2 setlinewidth
0 0 0 setrgbcolorVirtual
thePatch \fcPatchGetPoint
thePatch \fcPatchGetPoint thePatch \fcPatchGetNormal \fcVectorPlusVector
\fcPlotArrow
1 1 0 setrgbcolorVirtual
thePatch \fcPatchGetvZero
thePatch \fcPatchGetvOne
\fcPlotArrow
0 1 1 setrgbcolorVirtual
thePatch \fcPatchGetvZero
thePatch \fcPatchGetvTwo
\fcPlotArrow
end
}

\newcommand{\fcPatchPaintFilledDirectly}{
2 dict begin
/thePatch exch def %
thePatch type (arraytype) eq{
thePatch \fcPatchGetContour length 0 gt{
/currentVisibleColor [
thePatch \fcPatchGetNormal \fcScreen\space pop \fcVectorScalarVector 0 gt
{ thePatch \fcPatchGetColorVU }
{ thePatch \fcPatchGetColorUV }
ifelse]
def
0.5 setalpha
newpathVirtual
thePatch \fcPatchGetContour 0 get \fcCoordsIIIdToPS movetoVirtual
thePatch \fcPatchGetContour{\fcCoordsIIIdToPS linetoVirtual}forall
closepathVirtual
/thePatchCollection where{pop thePatchCollection length \fcMaxNumPatchesToUseShading\space le}{false}ifelse {
[thePatch \fcPatchGetvZero thePatch \fcPatchGetvOne thePatch \fcPatchGetvTwo currentVisibleColor]
shfillPlaneFromLightSourceVirtual
}
{currentVisibleColor
%(current visible color: ) == dup ==
thePatch \fcPatchGetPoint\space
%(current point: ) == dup ==
\fcColorFromColorBaseAndPoint\space
%(final color: ) == dup ==
\fcArrayToStack setrgbcolorVirtual
fillVirtual}
ifelse
strokeVirtual
%thePatch \fcPatchPaintNormal
}if
}if
end
}

\newcommand{\fcPatchPaintContourDirectly}{
1 dict begin
/thePatch exch def %
thePatch type (arraytype) eq{
thePatch \fcPatchGetIsDashed {thePatch \fcPatchGetDashes setdashVirtual}{[] 0 setdashVirtual}ifelse
thePatch \fcPatchGetContour length 0 gt{
thePatch \fcPatchGetColorContour setrgbcolorVirtual
newpathVirtual
thePatch \fcPatchGetContour 0 get \fcCoordsIIIdToPS movetoVirtual
thePatch \fcPatchGetContour{\fcCoordsIIIdToPS linetoVirtual}forall
closepathVirtual
strokeVirtual
}if
}if
end
}

\newcommand{\fcZBufferPatchIndex}{ %
15 dict begin %
/thePatchIndex exch def %
/thePatch thePatchCollection thePatchIndex get def
%thePatch \fcPatchPaintContourDirectly
/secondPoint thePatch \fcPatchGetvTwo def %
/firstPoint  thePatch \fcPatchGetvOne def %
/basePoint   thePatch \fcPatchGetvZero def %
/firstDirection firstPoint basePoint \fcVectorMinusVector def %
/secondDirection secondPoint basePoint \fcVectorMinusVector def %
/iterationsFirst [firstDirection \fcCoordsIIIdToPStricks\space pop getZBufferDeltaX div firstDirection \fcCoordsIIIdToPStricks\space exch pop getZBufferDeltaY div] \fcVectorNorm 2 mul 2 add round cvi def %
/iterationsSecond [secondDirection \fcCoordsIIIdToPStricks pop  getZBufferDeltaX div secondDirection \fcCoordsIIIdToPStricks exch pop getZBufferDeltaY div] \fcVectorNorm 2 mul 2 add round cvi def
/s 0 def %
iterationsFirst{ %
/firstComponent firstDirection s iterationsFirst 1 sub div  \space\fcVectorTimesScalar basePoint \fcVectorPlusVector def%
/t 0 def %
iterationsSecond{ %
secondDirection t iterationsSecond 1 sub div \fcVectorTimesScalar %
firstComponent \fcVectorPlusVector \fcZBufferRowColumn %
/column exch def %
/row exch def %
\fcZBufferAccountPatchIndexAtXY %
/t t 1 add def %
}repeat
/s s 1 add def %
}repeat
end %
}

%example: 
%\begin{pspicture}(-1,-1.2)(2.8,3)
%\tiny
%\fcAxesIIId{3}{3}{3}
%\fcCurve{0}{5}{[t t t]}
%\end{pspicture}
\newcommand{\fcCurveIIId}[4][linecolor=\fcColorGraph]{%
\parametricplot[#1]{#2}{#3}{#4 \fcCoordsIIIdToPStricks}%
}

\newcommand{\fcZeroVector}{[exch {0} repeat]}

\newcommand{\fcPerpendicularComputeHeelCode}[3]{%
7 dict begin%
/thePoint #1\space def%
/heelSize #3\space def %
mark #2 \space%
counttomark 1 eq {%
/directionUnitVector exch \fcVectorNormalize def%
/basePoint thePoint length \fcZeroVector def%
}{%
/basePoint exch def%
/directionUnitVector exch basePoint \fcVectorMinusVector \fcVectorNormalize def%
} ifelse %
pop%
/heel directionUnitVector thePoint basePoint \fcVectorMinusVector directionUnitVector \fcVectorScalarVector \fcVectorTimesScalar basePoint \fcVectorPlusVector def%
/perpendicularUnitVector thePoint heel \fcVectorMinusVector \fcVectorNormalize def %
/polyLineInput {%
heel directionUnitVector heelSize \fcVectorTimesScalar \fcVectorMinusVector %
dup perpendicularUnitVector heelSize \fcVectorTimesScalar \fcVectorPlusVector %
heel perpendicularUnitVector heelSize \fcVectorTimesScalar \fcVectorPlusVector%
} def%
}

%argument 1: options
%arguments 2: tip of perpendicular in format [a b]
%argument 3: direction vector of line through origin to drop perpendicular to.
%argument 4: heel size 
%example: \fcPerpendicular[linecolor=blue]{[1 1]}{[1 0]}{0.1}
\newcommand{\fcPerpendicular}[4][ ]{%
\setkeys{fcGraphics}{#1}%
\pstVerb{\fcPerpendicularComputeHeelCode{#2}{#3}{#4}}%
\psline[#1](! thePoint \fcArrayToStack)(! heel \fcArrayToStack)%
\listplot[linecolor=\fcColorAngle, linewidth=\fcAngleLineWidth]{ [polyLineInput] {\fcArrayToStack} \fcSpliceArrayOperation \fcArrayToStack}%
\pstVerb{end}%
}

\newcommand{\fcPerpendicularIIId}[4][]{%
\pstVerb{\fcPerpendicularComputeHeelCode{#2}{#3}{#4}}%
\fcLineIIId[#1]{thePoint}{heel}%
\fcPolyLineIIId[linecolor=red]{polyLineInput}%
\pstVerb{end}%
}%

\newcommand{\fcPlotIIId}[7][]{%
\fcPlotIIIdXconst[#1]{#2}{#3}{#4}{#5}{#6}{#7}%
\fcPlotIIIdYconst[#1]{#2}{#3}{#4}{#5}{#6}{#7}%
}

\newcommand{\fcPlotIIIdXconst}[7][]{%
\setkeys{fcGraphics}{#2}%
\multido{\ra=0+1}{\fcIterationsX}{%
\pstVerb{%
3 dict begin %
/x \ra \space #3 mul \fcIterationsX \space \ra \space sub 1 sub  #5\space mul add \fcIterationsX\space 1 sub div def%
/ymin #4 def%
/ymax #6 def%
}%
\parametricplot[#1]{ymin}{ymax}{%
1 dict begin /y t def  [x y #7] \fcCoordsIIIdToPStricks end%
}%
\pstVerb{end}%
}%end multido
}

\newcommand{\fcPlotIIIdYconst}[7][]{%
\setkeys{fcGraphics}{#2}%
\multido{\ra=0+1}{\fcIterationsY}{%
\pstVerb{%
3 dict begin%
/y \ra \space #4 mul \fcIterationsY \space \ra \space sub 1 sub  #6\space mul add \fcIterationsY\space 1 sub div def%
/xmin #3 def%
/xmax #5 def%
}%
\parametricplot[#1]{xmin}{xmax}{%
1 dict begin /x t def  [x y #7] \fcCoordsIIIdToPStricks end%
}%
\pstVerb{end}%
}%end multido
}

\newcommand{\fcSurfaceIIIdUConst}[7][]{%
\setkeys{fcGraphics}{#2}%
\multido{\ra=0+1}{\fcIterationsX}{%
\pstVerb{%
3 dict begin%
/u \ra \space #3 mul \fcIterationsX \space \ra \space sub 1 sub  #5\space mul add \fcIterationsX\space 1 sub div def%
/vmin #4 def%
/vmax #6 def%
}%
\parametricplot[#1]{vmin}{vmax}{%
1 dict begin /v t def #7 \fcCoordsIIIdToPStricks end%
}%
\pstVerb{end}%
}%end multido
}

\newcommand{\fcSurfaceIIIdVConst}[7][]{%
\setkeys{fcGraphics}{#2}%
\multido{\ra=0+1}{\fcIterationsY}{%
\pstVerb{%
3 dict begin%
/v \ra \space #4 mul \fcIterationsY \space \ra \space sub 1 sub  #6\space mul add \fcIterationsY\space 1 sub div def%
/umin #3 def%
/umax #5 def%
}%
\parametricplot[#1]{umin}{umax}{%
1 dict begin /u t def #7 \fcCoordsIIIdToPStricks end%
}%
\pstVerb{end}%
}%end multido
}

\newcommand{\fcSurfaceDirectDraw}[7][]{%
\fcSurfaceIIIdUConst[#1]{#2}{#3}{#4}{#5}{#6}{#7}%
\fcSurfaceIIIdVConst[#1]{#2}{#3}{#4}{#5}{#6}{#7}%
}%

%Use: \fcVectorFieldCenteredArrow[options]{startX}{startY}{iterationsX}{iterationsY}{Delta}{y -x}
\newcommand{\fcVectorFieldCenteredArrow}[7][linecolor=blue]{%
\multido{\ra=#2+#6}{#4}{%
\multido{\rb=#3+#6}{#5}{%
\pstVerb{%
4 dict begin%
/x \ra\space def%
/y \rb\space def %
#7\space%
/vY exch def%
/vX exch def%
}%
\psline[#1](! x vX 2 div sub y vY 2 div sub)(! x vX 2 div add y vY 2 div add)%
\pscircle*[linecolor=red](! x y){0.02}%
\pstVerb{end}%
}%end multido
}%end multido
}%

%Use: \fcVectorField[options]{startX}{startY}{iterationsX}{iterationsY}{Delta}{y -x}
\newcommand{\fcVectorField}[7][linecolor=blue]{%
\setkeys{fcGraphics}{#1}%
\multido{\ra=#2+#6}{#4}{%
\multido{\rb=#3+#6}{#5}{%
\pstVerb{%
6 dict begin%
/x \ra\space def%
/y \rb\space def %
#7\space %
/vY exch def%
/vX exch def%
}%
\pscustom{%
\code{%
vX 0 ne vY 0 ne or{
\fcLineFormatCode
[x y] [x vX add y vY add] \fcArrowHeadAndTailPlotCode
}if%
}%
}%
\pscircle*[linecolor=red](! x y){0.02}%
\pstVerb{end}%
}%end multido
}%end multido
}%

%example
%\fcImplicitIId[linestyle=solid, linecolor=red, showGridImplicitIId=false, useMidpointImplicitPlots=false]{-6}{-6}{48}{48}{0.25}{0.25}{y y mul y y mul 3 sub mul x x mul x x mul 5 sub mul sub} 
%First two arguments: x and y coordinate of lower left corner of picture
% next two arguments: number of grid rectangles in x direction and in y direction
% next two arguments: width and height of grid rectangles
% final argument: function given of x and y (written in postscript)
\newcommand{\fcImplicitIId}[8][]{%
%\fcGrid[#1, linestyle=dashed]{#2}{#3}{#4}{#5}{#6}{#7}{0}%
\setkeys{fcGraphics}{#1}%
\pscustom{%
\code{%
50 dict begin
\fcSetUpGraphicsToScreen
/graphicsFileName (xgraphicsCacheSafeToDelete) (#1 #2 #3 #4 #5 #6 #7 #8) \fcHashString (.txt) 3 \fcConcatenateMultiple def
\fcSetupFilesGraphicsNameAlreadyDefinedCode
graphicsCached not {
/theEquation {#8} \space def
/startX #2\space def
/startY #3\space def
/iterationsX #4\space def
/iterationsY #5\space def
/DeltaX #6\space def
/DeltaY #7\space def
graphicsFileAvailable not
{ (Implicit function plot: graphics file cache not available. I am DECREASING THE NUMBER OF ITERATIONS in the X and Y direction to 100x100 else this will take way too long. ) ==
iterationsX 100 gt {/DeltaX DeltaX iterationsX mul 100 div def /iterationsX 100 def} if 
iterationsY 100 gt {/DeltaY DeltaY iterationsY mul 100 div def /iterationsY 100 def} if 
}if
/interpolate {20 dict begin
/rightPt exch def
/leftPt exch def
/rightVal rightPt \fcArrayToStack /y exch def /x exch def theEquation def
/leftVal leftPt \fcArrayToStack /y exch def /x exch def theEquation def
leftVal 0 eq rightVal 0 eq and{leftVal}{
leftVal rightVal mul 0 le{
leftVal 0 lt {/leftVal leftVal -1 mul def}if
rightVal 0 lt {/rightVal rightVal -1 mul def}if
\fcUseMidpointImplicitPlots
{leftPt 0.5 \fcVectorTimesScalar
rightPt 0.5 \fcVectorTimesScalar
\fcVectorPlusVector
}
{
rightVal leftVal add 0 eq
{ leftPt
}
{leftPt  rightVal  rightVal leftVal add div \fcVectorTimesScalar
rightPt leftVal rightVal leftVal add div \fcVectorTimesScalar
\fcVectorPlusVector
}ifelse
}ifelse
}if
}ifelse
end}
def
/processOneTriangle {
20 dict begin
/firstV exch def
/secondV exch def
/thirdV exch def
firstV secondV interpolate
secondV thirdV interpolate
thirdV firstV interpolate
end
} def
%%%%%%%%%%%%%%%%%%%just the grid
\fcShowGridImplicitIId {
[1 1] 0 setdashVirtual
0.5 setlinewidthVirtual
0 0 0 setrgbcolorVirtual
/counterX -1 def
iterationsX {
/counterX counterX 1 add def
/counterY -1 def
iterationsY {
/counterY counterY 1 add def
%
/BottomLeft [startX DeltaX counterX mul add startY DeltaY counterY mul add] def
/BottomRight [startX DeltaX counterX 1 add mul add startY DeltaY counterY mul add] def
/TopLeft [startX DeltaX counterX mul add startY DeltaY counterY 1 add mul add] def
/TopRight [startX DeltaX counterX 1 add mul add startY DeltaY counterY 1 add mul add] def
newpathVirtual
BottomLeft \fcCoordsPStricksToPS  movetoVirtual
BottomRight \fcCoordsPStricksToPS linetoVirtual
TopRight \fcCoordsPStricksToPS linetoVirtual
BottomLeft \fcCoordsPStricksToPS linetoVirtual
TopLeft \fcCoordsPStricksToPS linetoVirtual
TopRight \fcCoordsPStricksToPS linetoVirtual
strokeVirtual
}repeat
}repeat
}if
%%%%%%%%%%%%%%%%%%%%%%%%%%%%%%
\fcLineFormatCodeVirtual
/counterX -1 def
iterationsX {
/counterX counterX 1 add def
/counterY -1 def
iterationsY {
/counterY counterY 1 add def
%
/BottomLeft [startX DeltaX counterX mul add startY DeltaY counterY mul add] def
/BottomRight [startX DeltaX counterX 1 add mul add startY DeltaY counterY mul add] def
/TopLeft [startX DeltaX counterX mul add startY DeltaY counterY 1 add mul add] def
/TopRight [startX DeltaX counterX 1 add mul add startY DeltaY counterY 1 add mul add] def
/firstPair  [BottomLeft BottomRight TopRight processOneTriangle ] def
/secondPair [BottomLeft TopRight    TopLeft  processOneTriangle ] def
firstPair length 2 eq
{newpathVirtual
firstPair 0 get \fcCoordsPStricksToPS movetoVirtual
firstPair 1 get \fcCoordsPStricksToPS linetoVirtual
strokeVirtual
}if
secondPair length 2 eq
{newpathVirtual
secondPair 0 get \fcCoordsPStricksToPS movetoVirtual
secondPair 1 get \fcCoordsPStricksToPS linetoVirtual
strokeVirtual
}if
}repeat
}repeat
}if
end
}}%
}%

%example \fcGrid{-2}{-2}{4}{4}{1}{1}{}
%Arguments in order of appearance:
% [options]
%xStart yStart numIntervalsX numIntervalsY DeltaX DeltaY Stickypiece
%Stickypiece is optional, leave at 0 or empty if you don't care about it.
%Stickypiece adds a shift within the grid.
\newcommand{\fcGrid}[8][]{%
\setkeys{fcGraphics}{#1}%
\pscustom{%
\code{%
20 dict begin
/startX #2\space def
/startY #3\space def
/iterationsX #4\space def
/iterationsY #5\space def
/DeltaX #6\space def
/DeltaY #7\space def
/stickyPiece (#8) () eq (#8) ( ) eq or {0}{#8}ifelse def
\fcLineFormatCode
/counterX -1 def
iterationsX 1 add{
/counterX counterX 1 add def
/currentX1 counterX DeltaX mul startX add def
/currentY1 0        stickyPiece sub DeltaY mul startY add def
/currentY2 iterationsY stickyPiece add DeltaY mul startY add def
/currentX2 currentX1 def
newpath
[currentX1 currentY1] \fcCoordsPStricksToPS moveto
[currentX2 currentY2] \fcCoordsPStricksToPS lineto
stroke
} repeat
/counterY -1 def
iterationsY 1 add{
/counterY counterY 1 add def
/currentX1 0        stickyPiece sub DeltaX mul startX add def
/currentY1 counterY DeltaY mul startY add def
/currentY2 currentY1 def
/currentX2 iterationsX stickyPiece add DeltaX mul startX add def
newpath
[currentX1 currentY1] \fcCoordsPStricksToPS moveto
[currentX2 currentY2] \fcCoordsPStricksToPS lineto
stroke
} repeat
end
}%
}%
}

\newcommand{\fcVectorFieldAlongSurfaceInScene}[8][ ]{%
\setkeys{fcGraphics}{#1}%
\fcSurfaceInScene[#1]{#2}{#3}{#4}{#5}{#6}{#7}%
\pstVerb{%
/theIIIdObjects [theIIIdObjects \fcArrayToStack
30 dict begin
/theField {#8} def
/theSurfaceObject exch def
theSurfaceObject \fcArrayToStack pop \fcSurfaceInit
/DeltaU uMax uMin sub uIterations div def
/DeltaV vMax vMin sub vIterations div def
/counterU -1 def %
uIterations{
/counterU counterU 1 add def
/counterV -1 def %
vIterations{
/counterV counterV 1 add def
/u DeltaU counterU 0.5 add mul uMin add def
/v DeltaV counterV 0.5 add mul vMin add def
/lineStart theSurface def
/lineEnd theSurface theField \fcVectorPlusVector def
[0 1 [lineEnd lineStart {1 t sub \fcVectorTimesScalar exch t \fcVectorTimesScalar \fcVectorPlusVector} \fcArrayToStack] cvx \fcContourOptions (curve)]
}repeat
}repeat
end
] def
}%
}
 %warning this path is relative to the file that uses the \usepackage command, not relative to the style file.

\renewcommand{\Re}{\mathrm{Re~}}
\renewcommand{\Im}{\mathrm{Im~}}
\newcommand{\doublebrace}[4]{\left\{\begin{array}{ll} #1 & #2 \\#3 & #4  \end{array} \right.}
\newcommand{\triplebrace}[6]{\left\{\begin{array}{ll} #1 & #2 \\#3 & #4  \\#5 & #6\end{array} \right.}

\newenvironment{solutionEnvironment}%
{\begin{proof}[\bfseries\upshape Solution]\renewcommand{\qedsymbol}{}}%
{\end{proof}}%
\newcommand{\solution}[1]{\iftoggle{solutions}{\begin{solutionEnvironment}#1\end{solutionEnvironment}}{}}
%solutionExtra is identical to the solution command, except that it useses a different toggle. 
%The solutionExtra command should only be used for
%problems that have already extremely similar counterparts solved. 
%solutionExtra serves to archive LaTeX'ed solution which 
%are to never be given to the students.
\newcommand{\solutionExtra}[1]{\iftoggle{solutionsExtra}{\begin{solutionEnvironment}#1\end{solutionEnvironment}}{}}

\newcommand{\homeworkEnd}{\end{enumerate}\end{document}}
\newcommand{\homeworkStart}[2]{\title{\course \\ Homework \ #1}\date{%
\ifthenelse{\equal{#2}{}}{}{%
Due #2 at \deadline}}%
\begin{document}\maketitle\begin{enumerate}[]
}%
\newcommand{\points}[1]{\stepcounter{enumi}\item[ ({\bf #1 mark\ifthenelse{\equal{#1}{1}}{}{s}}) \arabic{enumi}.]}
\newcommand{\pointsii}[1]{\stepcounter{enumii}\item[ ({\bf #1 mark\ifthenelse{\equal{#1}{1}}{}{s}}) (\alph{enumii})]}
%Default answer command
\newcommand{\answer}[1]{\iftoggle{answers}{ \hfill{~} \rotatebox{180}{\tiny answer: #1}}{}} %
\newcommand{\hiddenanswer}[1]{\iftoggle{solutions}{ %
\iftoggle{answers}{ %
\hfill{~} \rotatebox{180}{\tiny answer: #1}}
{}} %
{}} %
%I am using \renewcommand{\answer} to redo the command to suit my taste; feel free to modify the above line any way you please.


% Default course name
\newcommand{\course}{Freecalc}
% Default deadline
\newcommand{\deadline}{ (to be announced)}

\newcommand{\fcProblemRef}{\arabic*}


\toggletrue{solutions}
%\togglefalse{solutions}
\toggletrue{answers}
\newtheorem{problem}{Problem}

\newcommand{\hide}[1]{}

\renewcommand{\fcProblemRef}{\theproblem.\theenumi}
\renewcommand{\fcSubProblemRef}{\theenumi.\theenumii}

\begin{document}
\begin{center}
\Large
Master Problem Sheet \\ Calculus I \\ 
\end{center}

%\noindent \textbf{Name:} \hfill{~}
%\begin{tabular}{c|c|c|c|c|c|c|c|c||c}
%Problem&1 &2&3&4&5&6&7&8& $\sum$\\ \hline
%Score&&&&&&&&&\\ \hline
%Max&20&20&20&20&20&10&20&20&150
%\end{tabular}




This master problem sheet contains all freecalc problems on the topics studied in Calculus I. For a list of contributors/authors of the freecalc project (and in particular, the present problem collection) see file contributors.tex.


\fcLicenseContent



\tableofcontents

\section{Functions, Basic Facts}\label{secMPSfunctionBasics}
\subsection{Understanding function notation}
\begin{problem}
(Stewart, 7ed., page 21, problems 27-30)
Evaluate the difference and simplify your answer.
\begin{multicols}{2}
\begin{enumerate}
\item $\frac{f(3+h)-f(3)}{h}$, where $f(x)=4+3x-x^2$.
\answer{$-3-h$}
\item $\frac{f(a+h)-f(a)}{h}$, where $f(x)= x^3$.
\answer{$ 3a^2+3ah+h^2$}
\item $\frac{f(x)-f(a)}{x-a}$, where $f(x)=\frac{1}{x}$.
\answer{$-\frac{1}{ax}$.}
\item $\frac{f(x)-f(1)}{x-1}$, where $f(x)=\frac{x+3}{x+1}$.
\answer{$-\frac{1}{x+1}$.}
\end{enumerate}
\end{multicols}

\end{problem}
\subsection{Domains and ranges}
\begin{problem}
Find the implied domain of the function.
\begin{multicols}{2}
\begin{enumerate}[ref={\fcProblemRef}]
\item $\displaystyle f(x)=\frac{x+4}{x^2-4}$. 

\answer{\begin{tabular}{l} $x\neq \pm 2$, \\alternatively:\\ $x\in (-\infty, -2)\cup (-2,2)\cup (2,\infty)$\end{tabular} }
\item $\displaystyle f(x)=\frac{2x^3-5}{x^2+5x+6}$.

\answer{\begin{tabular}{l} $x\neq -2,-3$, \\alternatively:\\ $x\in (-\infty, -3)\cup (-3,-2)\cup (-2,\infty)$\end{tabular} }
\item $\displaystyle f(t)=\sqrt[3]{3t-1}$.

\answer{$x\in \mathbb R$ (the domain is all real numbers) }
\item $\displaystyle g(t)=\sqrt{5-t}-\sqrt{1+t}$.

\answer{$x\in [-1,5]$.}
\item $\displaystyle h(x)=\frac{1}{\sqrt[6]{x^2-7x}}$.

\answer{$x\in (-\infty, 0)\cup (7,\infty)$.}
\item $f\displaystyle (u)=\frac{u+1}{1+\frac{1}{u+1}}$.

\answer{
\begin{tabular}{l}
$u\neq -1, -2$ or \\
$u\in (-\infty, -2)\cup (-2, -1)\cup (-1,\infty).$.
\end{tabular}
}
\item $\displaystyle F(x)=\sqrt{10-{\sqrt{x}}}$.

\answer{$x\in [0,100]$}
\end{enumerate}
\end{multicols}
\end{problem}
\subsection{Piecewise Defined Functions}
\begin{problem}
Write down a formula for a function whose graphs is given below. The graphs are up to scale. Please note that there is more than one way to write down a correct answer.
\begin{multicols}{2}
\begin{enumerate}[ref={\fcProblemRef}]
\item ~
\psset{xunit=0.4cm, yunit=0.4cm}
\begin{pspicture}(-1.2,-1.2)(6.7,5.9)
\tiny
\fcAxesStandard{-1}{-1}{6.5}{5.5}
\fcXTickWithLabel{1}{$1$}
\fcYTickWithLabel{1}{$1$}
\fcLabels{6.5}{5.5}
\fcGrid[linestyle=dashed]{0}{0}{6}{5}{1}{1}{}
\psline[linecolor=red](0,3)(2, 0)(5, 4)
\fcHollowDot{0}{3}
\end{pspicture}

\answer{ $y= \left\{\begin{array}{ll} 
-\frac{3}{2}x+3 &\text{if } 0< x\leq 2 \\ 
\frac{4}{3}x-\frac{8}{3}& \text{if } 2<x\leq 6 \end{array} \right.$}
\item ~
\psset{xunit=0.4cm, yunit=0.4cm}
\begin{pspicture}(-3.7,-3.7)(3.7,3.9)
\tiny
\fcAxesStandard{-3.5}{-3.5}{3.5}{3.5}
\fcXTickWithLabel{1}{$1$}
\fcYTickWithLabel{1}{$1$}
\fcLabels{3.5}{3.5}
\fcGrid[linestyle=dashed]{-3}{-3}{6}{6}{1}{1}{}
\psline[linecolor=red](-3,-1)(1, 2)(2, -2)
\fcHollowDot{1}{2}
\end{pspicture}

\answer{ $y= \left\{\begin{array}{ll} 
\frac{3}{4}x+\frac{5}{4} &\text{if } -3\leq x < 1\\ 
-4x +6& \text{if }1<x\leq 2\end{array} \right.$}
\item ~
\psset{xunit=0.4cm, yunit=0.4cm}
\begin{pspicture}(-3.7,-3.7)(3.7,3.9)
\tiny
\fcAxesStandard{-3.5}{-3.5}{3.5}{3.5}
\fcXTickWithLabel{1}{$1$}
\fcYTickWithLabel{1}{$1$}
\fcLabels{3.5}{3.5}
\fcGrid[linestyle=dashed]{-3}{-3}{6}{6}{1}{1}{}
\psline[linecolor=red](-2,1)(2, -2)(3, 3)
\fcHollowDot{3}{3}
\end{pspicture}

\answer{ $y= \left\{\begin{array}{ll} 
-\frac{3}{4}x-\frac{1}{2} & \text{if } -2\leq x<2\\ 
5x-12 & \text{if } 2\leq x<3
\end{array} \right.$}
\item ~
\psset{xunit=0.4cm, yunit=0.4cm}
\begin{pspicture}(-3.7,-3.7)(3.7,3.9)
\tiny
\fcAxesStandard{-3.5}{-3.5}{3.5}{3.5}
\fcXTickWithLabel{1}{$1$}
\fcYTickWithLabel{1}{$1$}
\fcLabels{3.5}{3.5}
\fcGrid[linestyle=dashed]{-3}{-3}{6}{6}{1}{1}{}
\psline[linecolor=red](-2,2)(-1, -1)(0, 1)(1,-1)(2,2)
\end{pspicture}

\answer{ $y= \left\{\begin{array}{ll} 
-3x-4&\text{if } -2\leq x<-1  \\ 
2x+1&\text{if } -1\leq x<0  \\ 
-2x+1&\text{if } 0\leq x<1  \\ 
3x-4&\text{if } 1\leq x\leq 2  \\ 
\end{array} \right.$}
\end{enumerate}
\end{multicols}

\end{problem}

\begin{problem}
Write down formulas for function whose graphs are as follows. The graphs are up to scale. All arcs are parts of circles.

\begin{enumerate}[ref={\fcProblemRef}]
\item ~%warning ~ sign needed for a latex bug
\tiny
\psset{xunit=0.4cm, yunit=0.4cm}
\begin{pspicture}(-5,-1.4)(5,5)
\tiny
\psaxes{->}(0,0)(-4.5,-1)(4.5,4)
\psplot[linecolor=red]{-2}{2}{4 x x mul sub sqrt }
\psline[linecolor=red](2,0)(4,1)
\psline[linecolor=red](-2,0)(-4,1)
\rput[b](4.5,0.1 ){$x$}
\rput[l](0.1,4 ){$y$}
\fcFullDot{4}{1}
\rput[b](4, 1.1){$(4, 1)$}
\fcFullDot{-4}{1}
\rput[b](-4, 1.1){$(-4, 1)$}

\end{pspicture}
\end{enumerate}

\end{problem}

\begin{problem}
(Stewart, 7ed., page 21, problems 45, 46, 49)
Plot the piecewise defined functions.
\begin{multicols}{2}
\begin{enumerate}
\item $G(x)=\frac{3x+|x|}x$.
\item $g(x)=|x|-x$.
\item $f(x)=\doublebrace{x+2}{x\leq -1}{x^2}{x\geq -1}$.
\end{enumerate}
\end{multicols}

\end{problem}

\subsection{Function composition}
\begin{problem}
Find the functions $f\circ g$, $g\circ f$, $f\circ f$ and $g\circ g$ and their implied domains. The answer key has not been proofread, use with caution.

\begin{enumerate}
\item $f(x)=x^2+1$, $g(x)=x+1$. 
\answer{\begin{tabular}{l} Domain, all 4 cases: $x\in \mathbb R$ (all reals)\\ in some order: $(1+x)^{2}+1, (x)^{2}+2, ((x)^{2}+1)^{2}+1, 2+x$\end{tabular} }
\item $f(x)=\sqrt{x+1}$, $g(x)=x+1$. 
\answer{\begin{tabular}{l} Domain of $f\circ g$ is $x\geq -2$. Domain of $g\circ f$ is $x\geq -1$ \\Domain of $f\circ f$ is $ x\geq -1$. Domain of $g\circ g$ is all reals ($x\in \mathbb R$). \\ in some order:$\sqrt{2+x}, 1+\sqrt{1+x}, \sqrt{1+\sqrt{1+x}}, 2+x$\end{tabular}}
\item $f(x)= 2x$, $g(x)= \tan x$. 

You are not required to find the domain.
\answer{\begin{tabular}{l}
Domain  $f\circ f$: all reals ($x\in \mathbb R$). Domain $g\circ f$: $x\neq (2k+1)\frac{\pi}{2}$ for all $k\in \mathbb Z$\\ Domain $ f\circ g$: $x\neq (4k+1)\frac{\pi}{4}$, $x\neq (4k+3)\frac{\pi}{4}$ for all $k\in \mathbb Z$\\
Domain $g\circ g$: $x\neq (2k+1)\frac{\pi}{2}$ and $x\neq k\pi+ \arctan \left(\frac{\pi}{2}\right)$ for all $k\in \mathbb Z$
\\
in some order:$2 \tan{}x, \tan{}(2 x), 4 x, \tan{}(\tan{}x) $
\end{tabular}
}
\item $f(x)=\frac{x+1}{x-1}$, $g(x)=\frac{x-1}{x+1}$.
\answer{ 
\begin{tabular}{l}
Domain $ f\circ f$: $x\neq 1$. Domain $g\circ g$: $x\neq 0$, $x\neq -1$\\
Domain $f\circ g$: $x \neq -1$. Domain $g\circ f$: $x\neq 0$, $x\neq 1$\\
in some order: $- x, \frac{1}{x}, x, -\frac{1}{x} $
\end{tabular}
}
\end{enumerate}

\end{problem}
\begin{problem}
Compute the composite functions $(f\circ g)(x)$, $(g\circ f)(x)$. Simplify your answer to a single fraction. Find the domain of the composite function. The answer key has not been fully proofread, use with caution. 

\begin{enumerate}[ref={\fcProblemRef}]
\item $\displaystyle f{}({{x}})=\frac{x+2}{x-2},
g{}({{x}})=\frac{x-1}{x+2}$.

\answer{
\begin{tabular}{rl}
$(f\circ g)(x)= \frac{3+3 x}{-5- x}$& $x\neq -2, -5$\\ 
$(g\circ f)(x)=\frac{4}{-2+3 x}$& $x\neq 2, \frac{2}{3}$ 
\end{tabular}
}
\item 
$\displaystyle f{}({{x}})=\frac{x+1}{3x-2}, g{}({{x}})= \frac{x-2}{x-1}
$.

\answer{
\begin{tabular}{rl}
$(f\circ g)(x)= \frac{-3+2 x}{-4+x}
$ & $x\neq 4, 1$ \\
$(g\circ f)(x)=\frac{5-5 x}{3-2 x}$
& $x\neq \frac{2}{3}, \frac{3}{2}$
\end{tabular}
}
\item 
$\displaystyle f{}({{x}})=\frac{2x+1}{3x-1},
g{}({{x}})=\frac{x-2}{2x-1}
$.

\answer{
\begin{tabular}{rl}
$(f\circ g)(x)=\frac{-5+4 x}{-5+x}
$ &$x\neq 5, \frac{1}{2}$ \\
$(g\circ f)(x)=\frac{3-4 x}{3+x}
$ &$x\neq -3, \frac{1}{3}$
\end{tabular}
}
\item 
$\displaystyle f{}({{x}})=\frac{x+1}{x-2},
g{}({{x}})=\frac{x+2}{2x-1}
$.

\answer{
\begin{tabular}{rl}
$(f\circ g)(x)= \frac{1+3 x}{4-3 x}
$&$x\neq \frac{4}{3}, \frac{1}{2}$\\ 
$(g\circ f)(x)=\frac{-3+3 x}{4+x}
$&$x\neq -4, 2$
\end{tabular}
}
\item 
$\displaystyle f{}({{x}})=\frac{5x+1}{4x-1},
g{}({{x}})=\frac{4x-1}{3x+1}
$.

\answer{
\begin{tabular}{rl}
$(f\circ g)(x)= \frac{-4+23 x}{-5+13 x}
$&$x\neq \frac{5}{13}, -\frac{1}{3 }$\\
$(g\circ f)(x)=\frac{5+16 x}{2+19 x}
$&$x\neq -\frac{2}{19}, \frac{ 1}{4}$
\end{tabular}
}


\item 
$\displaystyle  f(x)= \frac{3x-5}{x-2}$, $\displaystyle g(x)=\frac{x-2 }{x-4} $. 

\answer{ 
\begin{tabular}{rl}
$(f\circ g)(x)=\frac{-2 x+14}{- x+6}$&$x\neq 6, 4$\\
$(g\circ f)(x)=\frac{x-1}{- x+3}$&$x\neq 3,2$
\end{tabular}
}

\item 
$\displaystyle  f(x)= \frac{x-3}{x+2}$, $\displaystyle g(y)=\frac{y+3 }{y-4} $. 

\answer{ 
\begin{tabular}{rl}
$(f\circ g)(x)=\frac{-2x+15}{3 x-5}$&$x\neq \frac{5}{3}, 4$\\ 
$(g\circ f)(x)=\frac{4 x+3}{-3 x-11}$&$x\neq -\frac{11}{3}, -2$
\end{tabular}
}

\end{enumerate}

\end{problem}
\subsection{Linear Transformations and Graphs of Functions}
\begin{problem}
\begin{problem}Graph the functions by hand, by applying consecutively the transformations learned in class.
\begin{multicols}{2}
\begin{enumerate}
\item $y=\frac{1}{x}$.
\item $y=\frac{1}{x+1}$.
\item $y=\frac{1}{2x+1}$.
\item $y=\frac{3}{2x+1}$.
\item $y=\frac{3+x}{2x+1}$.
\item $y=\left|\frac{3+x}{2x+1}\right|$.
\end{enumerate}
\end{multicols}
\end{problem}
\end{problem}

\begin{problem}
Sketch by hand approximately the given function. The function is obtained by transforming linearly the graph of a known function. The known function has been sketched for you by computer.

\begin{enumerate}[ref={\fcProblemRef}]
\item $\displaystyle f(x)=-\frac{1}{2}x^2+1$.

\begin{pspicture}(-3.5,-3.5)(3.5,3.5)
\fcAxesStandard{-3.5}{-3.5}{3.5}{3.5}
\fcGrid[linestyle=dashed]{-3}{-3}{6}{6}{1}{1}{}
\newcommand{\theFun}{x x mul}
\newcommand{\theFunTwo}{x x mul -0.5 mul 1 add}
\psplot[linecolor=gray]{-1.75}{1.75}{\theFun}
\rput[t](0,0){$y=x^2$}
%\psplot[linecolor=\fcColorGraph]{-3}{3}{\theFunTwo}
%\rput[tl](-1.9,-1.5){$y=-\frac{1}{2}x^2+1$}
\end{pspicture}

\answer{
\psset{xunit=0.2cm, yunit=0.2cm}
\begin{pspicture}(-3.5,-3.5)(3.5,3.5)
\fcAxesStandard{-3.5}{-3.5}{3.5}{3.5}
\fcGrid[linestyle=dashed]{-3}{-3}{6}{6}{1}{1}{}
\newcommand{\theFun}{x x mul}
\newcommand{\theFunTwo}{x x mul -0.5 mul 1 add}
\psplot[linecolor=gray]{-1.75}{1.75}{\theFun}
\psplot[linecolor=\fcColorGraph]{-3}{3}{\theFunTwo}
\end{pspicture}
}
\item $\displaystyle f(x)=\frac{1}{2}x^2+x-1 $.

\begin{pspicture}(-3.5,-3.5)(3.5,3.5)
\fcAxesStandard{-3.5}{-3.5}{3.5}{3.5}
\fcGrid[linestyle=dashed]{-3}{-3}{6}{6}{1}{1}{}
\newcommand{\theFun}{x x mul}
\newcommand{\theFunTwo}{x x mul 0.5 mul x add 1 add}
\psplot[linecolor=gray]{-1.75}{1.75}{\theFun}
\rput[t](0,0){$y=x^2$}
%\psplot[linecolor=\fcColorGraph]{-3}{1.3}{\theFunTwo}
%\rput[tl](-2,1.5){$y=\frac{1}{2}x^2+x-1$}
\end{pspicture}

\answer{
\psset{xunit=0.2cm, yunit=0.2cm}
\begin{pspicture}(-3.5,-3.5)(3.5,3.5)
\fcAxesStandard{-3.5}{-3.5}{3.5}{3.5}
\fcGrid[linestyle=dashed]{-3}{-3}{6}{6}{1}{1}{}
\newcommand{\theFun}{x x mul}
\newcommand{\theFunTwo}{x x mul 0.5 mul x add 1 add}
\psplot[linecolor=gray]{-1.75}{1.75}{\theFun}
\psplot[linecolor=\fcColorGraph]{-3}{1.3}{\theFunTwo}
\end{pspicture}
}
\item \label{problemSketch1/(2x-1)+1fromgraphof1/x} $\displaystyle f(x)=\frac{1}{2x-1}+1$.

\begin{pspicture}(-3.5,-3.5)(3.5,3.5)
\fcAxesStandard{-3.5}{-3.5}{3.5}{3.5}
\fcGrid[linestyle=dashed]{-3}{-3}{6}{6}{1}{1}{}
\newcommand{\theFun}{1 x div}
\newcommand{\theFunTwo}{1 x 2 mul -1 add div 1 add}
\psplot[linecolor=gray]{-3}{-0.32}{\theFun}
\psplot[linecolor=gray]{0.32}{3}{\theFun}
\rput[bt](1.2,0.5){$~y=\frac{1}{x}$}
%\psplot[linecolor=\fcColorGraph]{-3}{0.37}{\theFunTwo}
%\psplot[linecolor=\fcColorGraph]{0.7}{3}{\theFunTwo}
%\rput[tl](-2,1.5){$y=\frac{1}{2x-1}+1$}
\end{pspicture}

\answer{
\psset{xunit=0.2cm, yunit=0.2cm}
\begin{pspicture}(-3.5,-3.5)(3.5,3.5)
\fcAxesStandard{-3.5}{-3.5}{3.5}{3.5}
\fcGrid[linestyle=dashed]{-3}{-3}{6}{6}{1}{1}{}
\newcommand{\theFun}{1 x div}
\newcommand{\theFunTwo}{1 x 2 mul -1 add div 1 add}
\psplot[linecolor=gray]{-3}{-0.32}{\theFun}
\psplot[linecolor=gray]{0.32}{3}{\theFun}
%\rput[bt](1.2,0.5){$~y=\frac{1}{x}$}
\psplot[linecolor=\fcColorGraph]{-3}{0.37}{\theFunTwo}
\psplot[linecolor=\fcColorGraph]{0.7}{3}{\theFunTwo}
%\rput[tl](-2,1.5){$y=\frac{1}{2x-1}+1$}
\end{pspicture}
}
\item $\displaystyle f(x)=\frac{\frac{1}{2}x + \frac{1}{4}}{x-\frac{1}{2}}+\frac{1}{2}$.

\begin{pspicture}(-3.5,-3.5)(3.5,3.5)
\fcAxesStandard{-3.5}{-3.5}{3.5}{3.5}
\fcGrid[linestyle=dashed]{-3}{-3}{6}{6}{1}{1}{}
\newcommand{\theFun}{1 x div}
\newcommand{\theFunTwo}{1 x 2 mul -1 add div 1 add}
\psplot[linecolor=gray]{-3}{-0.32}{\theFun}
\psplot[linecolor=gray]{0.32}{3}{\theFun}
\rput[bt](1.2,0.5){$~y=\frac{1}{x}$}
%\psplot[linecolor=\fcColorGraph]{-3}{0.37}{\theFunTwo}
%\psplot[linecolor=\fcColorGraph]{0.7}{3}{\theFunTwo}
%\rput[tl](-2,1.5){$y=\frac{1}{2x-1}+1$}
\end{pspicture}

\answer{
\psset{xunit=0.2cm, yunit=0.2cm}
\begin{pspicture}(-3.5,-3.5)(3.5,3.5)
\fcAxesStandard{-3.5}{-3.5}{3.5}{3.5}
\fcGrid[linestyle=dashed]{-3}{-3}{6}{6}{1}{1}{}
\newcommand{\theFun}{1 x div}
\newcommand{\theFunTwo}{1 x 2 mul -1 add div 1 add}
\psplot[linecolor=gray]{-3}{-0.32}{\theFun}
\psplot[linecolor=gray]{0.32}{3}{\theFun}
%\rput[bt](1.2,0.5){$~y=\frac{1}{x}$}
\psplot[linecolor=\fcColorGraph]{-3}{0.37}{\theFunTwo}
\psplot[linecolor=\fcColorGraph]{0.7}{3}{\theFunTwo}
%\rput[tl](-2,1.5){$y=\frac{1}{2x-1}+1$}
\end{pspicture}
}
\item $\displaystyle f(x)= -\sqrt{2x-1}-1$

\begin{pspicture}(-3.5,-3.5)(3.5,3.5)
\fcAxesStandard{-3.5}{-3.5}{3.5}{3.5}
\fcGrid[linestyle=dashed]{-3}{-3}{6}{6}{1}{1}{}
\newcommand{\theFun}{x sqrt}
\newcommand{\theFunTwo}{2 x mul 1 sub sqrt -1 mul -1 add}
\psplot[linecolor=gray]{0}{3}{\theFun}
\rput[bt](1.2,0.5){$~y=\sqrt{x}$}
%\psplot[linecolor=\fcColorGraph]{0.5}{3}{\theFunTwo}
\end{pspicture}

\answer{
\psset{xunit=0.2cm, yunit=0.2cm}
\begin{pspicture}(-3.5,-3.5)(3.5,3.5)
\fcAxesStandard{-3.5}{-3.5}{3.5}{3.5}
\fcGrid[linestyle=dashed]{-3}{-3}{6}{6}{1}{1}{}
\newcommand{\theFun}{x sqrt}
\newcommand{\theFunTwo}{2 x mul 1 sub sqrt -1 mul -1 add}
\psplot[linecolor=gray]{0}{3}{\theFun}
\rput[bt](1.2,0.5){$~y=\sqrt{x}$}
%\psplot[linecolor=\fcColorGraph]{0.5}{3}{\theFunTwo}
\end{pspicture}
}
\item $\displaystyle f(x)= -\sqrt{-2x-1}+1$

\begin{pspicture}(-3.5,-3.5)(3.5,3.5)
\fcAxesStandard{-3.5}{-3.5}{3.5}{3.5}
\fcGrid[linestyle=dashed]{-3}{-3}{6}{6}{1}{1}{}
\newcommand{\theFun}{x sqrt}
\newcommand{\theFunTwo}{-2 x mul 1 sub sqrt -1 mul 1 add}
\psplot[linecolor=gray]{0}{3}{\theFun}
\rput[bt](1.2,0.5){$~y=\sqrt{x}$}
%\psplot[linecolor=\fcColorGraph]{-3}{-0.5}{\theFunTwo}
\end{pspicture}

\answer{
\psset{xunit=0.2cm, yunit=0.2cm}
\begin{pspicture}(-3.5,-3.5)(3.5,3.5)
\fcAxesStandard{-3.5}{-3.5}{3.5}{3.5}
\fcGrid[linestyle=dashed]{-3}{-3}{6}{6}{1}{1}{}
\newcommand{\theFun}{x sqrt}
\newcommand{\theFunTwo}{-2 x mul 1 sub sqrt -1 mul 1 add}
\psplot[linecolor=gray]{0}{3}{\theFun}
\rput[bt](1.2,0.5){$~y=\sqrt{x}$}
\psplot[linecolor=\fcColorGraph]{-3}{-0.5}{\theFunTwo}
\end{pspicture}
}
\end{enumerate}
\end{problem}
\solution{\ref{problemSketch1/(2x-1)+1fromgraphof1/x}.

We are asked to plot $\displaystyle f(x)=\frac{1}{2x-1}+1$ by establishing a connection with the graph of $g(x)=\frac{1}{x}$. To do that we have to compose $g$ with a sequence of linear transformations to obtain $f(x)$. There are two natural ways to do that; we show both by presenting two different solutions.


\noindent\textbf{Solution I.} 
We show how to get from $g(x)=\frac{1}{x}$ by composing $g$ with a sequence of linear transformations.
\[
\begin{array}{r|rcll|l}
g(x)&=&\frac{1}{x} \\
\text{Define } h(x) \text{ via:}& h(x)&=&\displaystyle g(x+1)= \frac{1}{x-1} \\
\text{Define } k(x) \text{ via:}& k(x)&=&\displaystyle h(2x)= \frac{1}{2x-1} \\
\text{Therefore }& f(x)&=&\displaystyle k(x)+1= \frac{1}{2x-1}+1 \\
\end{array}
\]


We plot consecutively the functions $g(x)$, $h(x)$, $k(x)$ and $f(x)$. We start from the given graph of $g(x)$.

\psset{xunit=0.7cm, yunit=0.7cm}
\begin{pspicture}(-3.5,-3.5)(3.5,3.5)
\fcAxesStandard{-3.5}{-3.5}{3.5}{3.5}
\fcGrid[linestyle=dashed]{-3}{-3}{6}{6}{1}{1}{}
\newcommand{\theFun}{1 x div}
\newcommand{\theFunTwo}{1 x 2 mul -1 add div 1 add}
\psplot[linecolor=\fcColorGraph]{-3}{-0.32}{\theFun}
\psplot[linecolor=\fcColorGraph]{0.32}{3}{\theFun}
\rput[bt](1.2,0.5){$~y=g(x)=\frac{1}{x}$}
%\psplot[linecolor=\fcColorGraph]{-3}{0.37}{\theFunTwo}
%\psplot[linecolor=\fcColorGraph]{0.7}{3}{\theFunTwo}
%\rput[tl](-2,1.5){$y=\frac{1}{2x-1}+1$}
\end{pspicture}
\raisebox{2cm}{ $\stackrel{\begin{array}{l} \text{shift graph of }g(x)\\ 1 \text{ unit right} \end{array} }{\longrightarrow}$}
\begin{pspicture}(-3.5,-3.5)(3.5,3.5)
\fcAxesStandard{-3.5}{-3.5}{3.5}{3.5}
\fcGrid[linestyle=dashed]{-3}{-3}{6}{6}{1}{1}{}
\newcommand{\theFun}{1 x div}
\newcommand{\theFunTwo}{1 x  -1 add div}
\psplot[linecolor=gray]{-3}{-0.32}{\theFun}
\psplot[linecolor=gray]{0.32}{3}{\theFun}
\rput[bl](-3,-2.95){$~y=h(x)=g(x-1)=\frac{1}{x-1}$}
\psplot[linecolor=\fcColorGraph]{-3}{0.68}{\theFunTwo}
\psplot[linecolor=\fcColorGraph]{1.32}{3}{\theFunTwo}
%\rput[tl](-2,1.5){$y=\frac{1}{2x-1}+1$}
\end{pspicture}
\raisebox{2cm}{ $\stackrel{\begin{array}{l} \text{compress graph of }h(x)\\ \text{by factor of }2 \\\text{horizontally} \end{array} }{\longrightarrow}$}

\begin{pspicture}(-3.5,-3.5)(3.5,3.5)
\fcAxesStandard{-3.5}{-3.5}{3.5}{3.5}
\fcGrid[linestyle=dashed]{-3}{-3}{6}{6}{1}{1}{}
\newcommand{\theFun}{1 x  -1 add div}
\newcommand{\theFunTwo}{1 x 2 mul -1 add div}
\psplot[linecolor=gray]{-3}{-0.68}{\theFun}
\psplot[linecolor=gray]{1.32}{3}{\theFun}
\rput[bl](-3,-2.95){$~y=k(x)=h(2x)=\frac{1}{2x-1}$}
\psplot[linecolor=\fcColorGraph]{-3}{0.34}{\theFunTwo}
\psplot[linecolor=\fcColorGraph]{0.66}{3}{\theFunTwo}
%\rput[tl](-2,1.5){$y=\frac{1}{2x-1}+1$}
\end{pspicture}
\raisebox{2cm}{ $\stackrel{\begin{array}{l} \text{shift graph of }k(x)\\ 1 \text{ unit vertically }\end{array} }{\longrightarrow}$}
\begin{pspicture}(-3.5,-3.5)(3.5,3.5)
\fcAxesStandard{-3.5}{-3.5}{3.5}{3.5}
\fcGrid[linestyle=dashed]{-3}{-3}{6}{6}{1}{1}{}
\newcommand{\theFun}{1 x 2 mul -1 add div}
\newcommand{\theFunTwo}{1 x 2 mul -1 add div 1 add}
\psplot[linecolor=gray]{-3}{0.34}{\theFun}
\psplot[linecolor=gray]{0.66}{3}{\theFun}
\rput[bl](-3,-2.95){$~y=f(x)=h(x)+1=\frac{1}{2x-1}+1$}
\psplot[linecolor=\fcColorGraph]{-3}{0.38}{\theFunTwo}
\psplot[linecolor=\fcColorGraph]{0.74}{3}{\theFunTwo}
%\rput[tl](-2,1.5){$y=\frac{1}{2x-1}+1$}
\end{pspicture}
\noindent\textbf{Solution II.} 
In the previous solution we used horizontal stretch to transform the graph of $h(x)$ to the graph of $k(x)=h(2x)$. However algebra suggests a another to transform the graph of $g(x)$ to the graph of $f(x)$, this time using a vertical stretch. Indeed, the equality 
\[
\displaystyle f(x)=\frac{1}{2x-1}+1=\frac{1}{2}\cdot \frac{1}{x-\frac{1}{2}}+1
\]
suggests a different way to obtain the graph of $f(x)$ from the graph of $g(x)$.
\[
\begin{array}{r|rcll|l}
g(x)&=&\frac{1}{x} \\
\text{Define } l(x) \text{ via:}& l(x)&=&\displaystyle g\left(x-\frac{1}{2}\right)= \frac{1}{x-\frac{1}{2}} \\
\text{Define } k(x) \text{ via:}& k(x)&=&\displaystyle \frac{1}{2}h(x)= \frac{1}{2} \cdot\frac{1}{\left(x-\frac{1}{2}\right)}= \frac{1}{\left(2x-1\right)} \\
\text{Therefore }& f(x)&=&\displaystyle k(x)+1= \frac{1}{2x-1}+1 \\
\end{array}
\]


We plot consecutively the functions $g(x)$, $h(x)$, $k(x)$ and $f(x)$. We start from the given graph of $g(x)$.

\psset{xunit=0.7cm, yunit=0.7cm}
\begin{pspicture}(-3.5,-3.5)(3.5,3.5)
\fcAxesStandard{-3.5}{-3.5}{3.5}{3.5}
\fcGrid[linestyle=dashed]{-3}{-3}{6}{6}{1}{1}{}
\newcommand{\theFun}{1 x div}
\newcommand{\theFunTwo}{1 x 2 mul -1 add div 1 add}
\psplot[linecolor=\fcColorGraph]{-3}{-0.32}{\theFun}
\psplot[linecolor=\fcColorGraph]{0.32}{3}{\theFun}
\rput[bt](1.2,0.5){$~y=g(x)=\frac{1}{x}$}
%\psplot[linecolor=\fcColorGraph]{-3}{0.37}{\theFunTwo}
%\psplot[linecolor=\fcColorGraph]{0.7}{3}{\theFunTwo}
%\rput[tl](-2,1.5){$y=\frac{1}{2x-1}+1$}
\end{pspicture}
\raisebox{2cm}{ $\stackrel{\begin{array}{l} \text{shift graph of }g(x)\\\frac{1}{2} \text{ units right} \end{array} }{\longrightarrow}$}
\begin{pspicture}(-3.5,-3.5)(3.5,3.5)
\fcAxesStandard{-3.5}{-3.5}{3.5}{3.5}
\fcGrid[linestyle=dashed]{-3}{-3}{6}{6}{1}{1}{}
\newcommand{\theFun}{1 x div}
\newcommand{\theFunTwo}{1 x  -0.5 add div}
\psplot[linecolor=gray]{-3}{-0.32}{\theFun}
\psplot[linecolor=gray]{0.32}{3}{\theFun}
\rput[bl](-3,-2.95){$y=l(x)=g\left(x-\frac{1}{2}\right)=\frac{1}{x-\frac{1}{2}}$}
\psplot[linecolor=\fcColorGraph]{-3}{0.18}{\theFunTwo}
\psplot[linecolor=\fcColorGraph]{0.82}{3}{\theFunTwo}
%\rput[tl](-2,1.5){$y=\frac{1}{2x-1}+1$}
\end{pspicture}
\raisebox{2cm}{ $\stackrel{\begin{array}{l} \text{compress graph of }h(x)\\ \text{by factor of }2 \\\text{vertically} \end{array} }{\longrightarrow}$}

\begin{pspicture}(-3.5,-3.5)(3.5,3.5)
\fcAxesStandard{-3.5}{-3.5}{3.5}{3.5}
\fcGrid[linestyle=dashed]{-3}{-3}{6}{6}{1}{1}{}
\newcommand{\theFun}{1 x  -0.5 add div}
\newcommand{\theFunTwo}{1 x 2 mul -1 add div}
\psplot[linecolor=gray]{-3}{0.18}{\theFun}
\psplot[linecolor=gray]{0.82}{3}{\theFun}
\rput[bl](-3,-2.95){$~y=k(x)=\frac{1}{2}l(x)=\frac{1}{2x-1}$}
\psplot[linecolor=\fcColorGraph]{-3}{0.34}{\theFunTwo}
\psplot[linecolor=\fcColorGraph]{0.66}{3}{\theFunTwo}
%\rput[tl](-2,1.5){$y=\frac{1}{2x-1}+1$}
\end{pspicture}
\raisebox{2cm}{ $\stackrel{\begin{array}{l} \text{shift graph of }k(x)\\ 1 \text{ unit vertically }\end{array} }{\longrightarrow}$}
\begin{pspicture}(-3.5,-3.5)(3.5,3.5)
\fcAxesStandard{-3.5}{-3.5}{3.5}{3.5}
\fcGrid[linestyle=dashed]{-3}{-3}{6}{6}{1}{1}{}
\newcommand{\theFun}{1 x 2 mul -1 add div}
\newcommand{\theFunTwo}{1 x 2 mul -1 add div 1 add}
\psplot[linecolor=gray]{-3}{0.34}{\theFun}
\psplot[linecolor=gray]{0.66}{3}{\theFun}
\rput[bl](-3,-2.95){$~y=f(x)=h(x)+1=\frac{1}{2x-1}+1$}
\psplot[linecolor=\fcColorGraph]{-3}{0.38}{\theFunTwo}
\psplot[linecolor=\fcColorGraph]{0.74}{3}{\theFunTwo}
%\rput[tl](-2,1.5){$y=\frac{1}{2x-1}+1$}
\end{pspicture}
}


\section{Trigonometry}\label{secMPStrigonometry}
\subsection{Angle conversion}
\begin{problem}
Convert from degrees to radians.
\begin{multicols}{3}
\begin{enumerate}
\item $15^\circ$.
\item $30^\circ$.
\item $36^\circ$.
\item $45^\circ$.
\item $60^\circ$.
\item $75^\circ$.
\item $90^\circ$.
\item $120^\circ$.
\item $135^\circ$.
\item $150^\circ$.
\item $180^\circ$.
\item $225^\circ$.
\item $270^\circ$.
\item $305^\circ$.
\item $360^\circ$.
\item $405^\circ$.
\item $1200^\circ$.
\item $-900^\circ$.
\item $-2014^\circ$.
\end{enumerate}
\end{multicols}

\end{problem}
\begin{problem}
(Textbook page A32, problems 7-12). 
Convert from radians to degrees.
\begin{multicols}{3}
\begin{enumerate}
\item $4\pi$.
\item $-7/2\pi$.
\item $5/12\pi$.
\item $8/3\pi$.
\item $-3/8\pi$.
\item $5$.
\end{enumerate}
\end{multicols}
\end{problem}
\subsection{Trigonometry identities}
\begin{problem}
\begin{problem}
(Textbook page A32-, problems 45, 46, 47, 48, 49, 50, 51, 52, 56, 57, 58).
\begin{multicols}{3}
\begin{enumerate}
\item $\sin \theta\cot \theta =\cos \theta$.
\item $(\sin x +\cos x)^2=1+\sin(2x)$.
\item $\sec y - \cos y= \tan y \sin y$.
\item $\tan^2 \alpha-\sin^2 \alpha=\tan^2\alpha\sin^2\alpha$.
\item $\cot^2\theta+\sec^2\theta=\tan^2\theta+\csc^2\theta$.
\item $2\csc 2t= \sec t \csc t$.
\item $\tan 2\theta =\frac{2\tan \theta}{1-\tan^2\theta} $.
\item $\frac{1}{1-\sin \theta}+ \frac{1}{1+\sin \theta}=2\sec^2\theta$.
\item $\tan x + \tan y = \frac{\sin (x+y)}{\cos x \cos y}$.
\item $\sin 3\theta +\sin \theta = 2 \sin 2\theta \cos \theta $.
\item $\cos 3\theta = 4\cos^3\theta-3\cos \theta $.
\end{enumerate} 
\end{multicols}
\end{problem}
\end{problem}

\subsection{Trigonometry equations}
\begin{problem}
\begin{problem}(Textbook page A33, problems 65-72).
Find all values of $x$ in the interval $[0,2\pi]$ that satisfy the 
equation.
\begin{multicols}{3}
\begin{enumerate}
\item $2\cos x - 1=0$.
\item $3\cot^2 x= 1$.
\item $2\sin^2 x= 1$.
\item $|\tan x|=1 $.
\item $\sin 2x = \cos x $.
\item $2\cos x +\sin 2x=0$.
\item $\sin x =\tan x$.
\item $2+\cos 2x = 3 \cos x$.
\end{enumerate}
\end{multicols}
\end{problem}

\end{problem}
\solution{ \ref{problemSolve2cos^2x-(1+sqrt(2))cosx+sqrt(2)/2=0}
Set $\cos x=u$. Then 
\[
2\cos^2 x- (1+\sqrt{2})\cos x+\frac{\sqrt 2}2=0 
\] 
becomes 
\[2u^2-(1+\sqrt{2})u+\frac{\sqrt{2}}{2}=0.
\] 
This is a quadratic equation in $u$ and therefore has solutions
\[
\begin{array}{rcl}
u_1, u_2\displaystyle &=& \displaystyle \frac{ 1+\sqrt{2}\pm\sqrt{ (1+\sqrt{2})^2-4 \sqrt{2} } }4\\
&=&\displaystyle \frac{1+\sqrt{2}\pm\sqrt{1-2\sqrt{2}+2} }4\\
&=&\displaystyle \frac{1+\sqrt{2}\pm \sqrt{(1-\sqrt{2})^2}}4\\
&=&\displaystyle \frac{1+\sqrt{2}\pm (1-\sqrt{2}) }4=\doublebrace{\frac{1}2 }{ \mathrm{or}} {\frac{\sqrt{2}}{2}}{}
\end{array}
\]
Therefore $u=\cos x= \frac12$ or $u=\cos x=\frac{\sqrt{2}}2$, and, as $x$ is in the interval $[0,2\pi]$, we get $x=\frac{\pi}{3}, \frac{5\pi}{3}$ (for $\cos x=\frac12$) or $x=\frac{\pi }4 ,\frac{7\pi}4$ (for $\cos x=\frac{\sqrt{2}}{2}$).
} % end solution

\section{Limits and Continuity}
\subsection{Limits as $x$ tends to a number}\label{secMPSlimitsXtendsToNumer}
\begin{problem}
Evaluate the limit if it exists.
\begin{multicols}{2}
\begin{enumerate}[ref={\fcProblemRef}]
\item \label{problemlim(xto2)(x^2-5x+6)/(x-2)}
$\displaystyle\lim\limits_{x\to 2}\frac{x^2-5x+6}{x-2} $. 

\answer{$-1$}
\item $\displaystyle\lim\limits_{x\to 3}\frac{x^2-3x}{x^2-2x-3} $.

\answer{$\frac{3}{4}$}
\item \label{problemlimxto-2(2x^2+x-6)/(x^2-4)}

$\displaystyle \lim_{x\to -2} \frac{2x^2+x-6}{x^2-4}$
\answer{$\frac{7}{4}$}
\item $\displaystyle\lim\limits_{x\to 2}\frac{x^2-5x-6}{x-2} $.

\answer{DNE}
\item $\displaystyle\lim\limits_{x\to -1}\frac{x^2-3x}{x^{2}-2x-3} $.

\answer{DNE}

\item $\displaystyle\lim\limits_{x\to -2}\frac{x^2-4}{2x^2+5x+2} $.

\answer{$\frac{4}{3}$}

\item $\displaystyle\lim\limits_{x\to -1}\frac{2x^2+3x+1}{3x^2-2x-5} $.

\answer{$\frac{1}{8}$}

\item \label{problemlimxto-4(x^2+7x+12)/(x^2+6x+8)}
$\displaystyle \lim_{x \to -4}\frac{x^{2}+7 x+12}{x^{2}+6 x+8}$.

\answer{$\frac{1}{2}$}


\item $\displaystyle\lim\limits_{h\to 0}\frac{(-3+h)^2-9}{h} $.

\answer{$-6$}

\item $\displaystyle\lim\limits_{h\to 0}\frac{(-2+h)^3+8}{h} $.

\answer{$12$}
\item $\displaystyle\lim\limits_{x\to -3}\frac{x+3}{x^3+27} $.

\answer{$\frac{1}{27}$}

\item $\displaystyle\lim\limits_{x\to 1}\frac{x^4-1}{x^3-1} $.

\answer{$\frac{4}{3}$}
\item $\displaystyle\lim\limits_{h\to 0}\frac{\sqrt{4+h}-2}{h} $.

\answer{$\frac{1}{4}$}
\item $\displaystyle\lim\limits_{x\to 3} \frac{\sqrt{5x+1}-4}{x-3}$.

\answer{$\frac{5}{8}$}

\item $\displaystyle\lim\limits_{x\to -3} \frac{\sqrt{x^2+16}-5}{x+3}$.

\answer{$-\frac{3}{5}$}

\item $\displaystyle\lim\limits_{x\to -3} \frac{\frac{1}{3}+ \frac{1}{x}} {3+x}$.

\answer{$-\frac{1}{9}$}
\item $\displaystyle\lim\limits_{x\to -2} \frac{x^2+4x+4}{x^4-16}$.

\answer{$0$}
\item $\displaystyle\lim\limits_{x\to 0} \frac{\sqrt{1+x}- \sqrt{1-x}}{x}$.

\answer{$1$}
\item $\displaystyle\lim\limits_{x\to 0}\left(\frac{1}x -\frac{1}{x^2+x}\right)$.

\answer{$1$}
\item $\displaystyle\lim\limits_{x\to 9} \frac{3-\sqrt{x}}{9x-x^2}$.

\answer{$\frac{1}{54}$}
\item $\displaystyle\lim\limits_{h \to 0}\frac{(2+h)^{-1}-2^{-1}}{h} $.

\answer{$-\frac{1}{4}$}

\item $\displaystyle\lim\limits_{x\to 0} \left(\frac{1}{x\sqrt{1+x}}-\frac{1}{x} \right)$.

\answer{$-\frac{1}{2}$}

\item 
$\displaystyle\lim\limits_{h\to 0}\frac{(x+h)^3-x^3}{h} $.

\answer{$3x^2$}

\item 
\label{problemlim_hto0_(1/(x+h)^2-1/x^2)/h} $\displaystyle\lim\limits_{h\to 0}\frac{\frac{1}{(x+h)^2}-\frac{1}{x^2}}{h} $.

\answer{$-\frac{2}{x^3}$}

\item 
\label{problemlimhto0(1/(2+h)^2-1/4)/h}
$ \displaystyle \lim_{h\to 0} \frac{\frac{1}{(2+h)^2}-\frac{1}{4}}{h}  $.

\answer{$-\frac{1}{4}$}
\item 
\label{problemlimhto0(1/(1+h)^2-1)/h}
$ \displaystyle \lim_{h\to 0} \frac{\frac{1}{(1+h)^2}-1}{h}  $.

\answer{$-2$}
\end{enumerate}
\end{multicols}

\end{problem}
\solution{\ref{problemlim(xto2)(x^2-5x+6)/(x-2)}

$
\begin{array}{rcll|l}
\displaystyle 
\displaystyle \lim\limits_{x\to 2}\frac{x^2-5x+6}{x-2} &=&\displaystyle \lim\limits_{x\to 2}\frac{(x-3)\cancel{(x-2)}}{\cancel{x-2}} &&\text{factor and cancel}\\
&=&\displaystyle 2-3=-1
\end{array}
$
}
\solution{\ref{problemlimxto-2(2x^2+x-6)/(x^2-4)}

$\begin{array}{rcll|l}
\displaystyle \lim\limits_{x\to -2} \frac{2x^2+x-6}{x^2-4}&=&  \displaystyle \lim\limits_{x\to -2}\frac{ (2x -3)\cancel{( x+ 2 ) }}{ (x-2)\cancel{(x+2)}} &&\text{factor and cancel}\\ 
&=&\displaystyle  \frac{(2(-2)-3)}{-2-2} &&\text{substitute}\\
&=&\displaystyle \frac{7}{4}
\end{array}
$

}
\solution{\ref{limproblem(xto-2)(x^2-4)/(2x^2+5x+2)}

$
\begin{array}{rcll|l}
\displaystyle 
\displaystyle \lim\limits_{x\to 2}\frac{x^2-4}{2x^2+5x+2} &=&\displaystyle \lim\limits_{x\to -2} \frac{(x-2)\cancel{(x+2)}}{(2x+1) \cancel{(x+2)}} &&\text{factor and cancel}\\
&=&\displaystyle \frac{(-2)-2}{2(-2)+1}=\frac{4}{3}.
\end{array}
$
}
\solution{
\ref{problemlim(xto-1)(2x^2+3x+1)/(3x^2-2x-5)}

$
\begin{array}{rcll|l}
\displaystyle \lim\limits_{x\to-1}\frac{2x^2+3x+1}{3x^2-2x-5} &=&\displaystyle \lim\limits_{x\to -1}\frac{(2x+1)\cancel{(x+1)}}{(3x-5)\cancel{(x+1)}}&&\text{factor and cancel}\\
&=&\displaystyle \frac{2(-1)+1}{3(-1)-5} =\frac{1}{8} 
\end{array}
$
}
\solution{\ref{problemlimxto-4(x^2+7x+12)/(x^2+6x+8)}.

$\begin{array}{rcll|l}
\displaystyle \lim_{x \to -4}\frac{x^{2}+7 x+12}{x^{2}+6 x+8}&=& \displaystyle \lim_{x \to -4}\frac{(x+3)(\cancel{x+4})}{(x+2)(\cancel{x+4})} &&\text{factor}\\
&=&\displaystyle \frac{-4+3}{-4+2}=-\frac{1}{2}.
\end{array}
$

}
\solution{ \ref{problemlim_hto0_(1/(x+h)^2-1/x^2)/h}

$
\begin{array}{rcl}
\displaystyle \lim\limits_{h\to 0}\frac{\frac{1}{(x+h)^2}-\frac{1}{x^2}}{h} &=&\displaystyle \lim\limits_{h\to 0}\frac{x^2-(x+h)^2}{hx^2(x+h)^2}=\lim\limits_{h\to 0} \frac{x^2-(x^2+2xh+h^2)}{hx^2(x+h)^2}\\
&=&\displaystyle \lim\limits_{h\to 0}\frac{\cancel{h}(-2x+h)}{\cancel{h}x^2(x+h)^2}= \frac{-2x+0}{x^2(x+0)^2}=-\frac{2}{x^3}.
\end{array}
$
}

\solution{\ref{problemlimhto0(1/(2+h)^2-1/4)/h}.

\textbf{Variant I.}

$\begin{array}{rcll|l}
\displaystyle \lim_{h\to 0} \frac{\frac{1}{(2+h)^2}-\frac{1}{4}}{h}&=&\displaystyle \lim_{h\to 0}\frac{\frac{4-(2+h)^2}{4(2+h)^2}}{h}\\
&=&\displaystyle \lim_{h\to 0} \frac{4- (4+4h+h^2)}{4h(2+h)^2}\\
&=&\displaystyle \lim_{h\to 0} \frac{-4h-h^2}{4h(2+h)^2}\\
&=&\displaystyle \lim_{h\to 0} \frac{\cancel{h}(-4-h) }{4\cancel{h}(2+h)^2}&&\text{substitute }h=0\\
&=&\displaystyle \frac{-4-0}{4(2+0)^2}\\
&=&\displaystyle -\frac{1}{4}
\end{array}
$

\textbf{Variant II.}

$\begin{array}{rcll|l}
\displaystyle \lim_{h\to 0} \frac{\frac{1}{(2+h)^2}-\frac{1}{4}}{h}&=&\displaystyle \frac{\diff }{\diff x}\left(\frac{1}{x^2}\right)_{|x=2}\\
&=&\displaystyle \left(\frac{-2}{x^3}\right)_{|x=2}\\
&=&\displaystyle -\frac{1}{4}
\end{array}
$

}


\solution{\ref{problemlimhto0(1/(1+h)^2-1)/h}.

\textbf{Variant I.}

$\begin{array}{rcll|l}
\displaystyle \lim_{h\to 0} \frac{\frac{1}{(1+h)^2}-1}{h}&=&\displaystyle \lim_{h\to 0}\frac{\frac{1-(1+h)^2}{ (1+h)^2}}{h}\\
&=&\displaystyle \lim_{h\to 0} \frac{1- (1+2h+h^2)}{h(1+h)^2}\\
&=&\displaystyle \lim_{h\to 0} \frac{-2h-h^2}{h(1+h)^2}\\
&=&\displaystyle \lim_{h\to 0} \frac{\cancel{h}(-2-h) }{\cancel{h}(1+h)^2}&&\text{substitute }h=0\\
&=&\displaystyle \frac{-2-0}{(1+0)^2}\\
&=&\displaystyle -2.
\end{array}
$

\textbf{Variant II.}

$\begin{array}{rcll|l}
\displaystyle \lim_{h\to 0} \frac{\frac{1}{(1+h)^2}-1}{h}&=&\displaystyle \frac{\diff }{\diff x}\left(\frac{1}{x^2}\right)_{|x=1}&&\text{derivative definition}\\
&=&\displaystyle \left(\frac{-2}{x^3}\right)_{|x=1}\\
&=&\displaystyle -2.
\end{array}
$

}
\begin{problem}
Evaluate the limit if it exists.
\begin{enumerate}
\item $\displaystyle \lim\limits_{x\to 1} \frac{3x^2+4x-7}{x^3-x}$ \answer{$5 $.}
\item $\displaystyle \lim\limits_{x\to -1} \frac{2x^2-3x-5}{x^3+1}$ \answer{$ -\frac{7}{3}$.}
\end{enumerate}

\end{problem}
\begin{problem}
(Textbook page 69, problems 3-9). 
Evaluate the limits. Justify your computations.
\begin{multicols}{3}
\begin{enumerate}
\item $\displaystyle\lim\limits_{x\to 3} 5x^3-3x^2+x-6$.
\answer{$105$}
\item $\displaystyle\lim\limits_{x\to -1} (x^4-3x)(x^2+5x+3)$.
\answer{$-4$}
\item $\displaystyle\lim\limits_{t\to -2} \frac{t^4- 2}{2t^2 -3t +2} $.
\answer{$\frac78$}
\item $\displaystyle\lim\limits_{u\to -2}\sqrt{u^4+3u +6}$.
\answer{$4$}
\item $\displaystyle\lim\limits_{x \to 8}(1+ \sqrt[3]{x} )(2- 6x^2 + x^3)$.
\answer{$390$}
\item $\displaystyle\lim\limits_{t \to 2}\left( \frac{t^2 - 2 }{ t^3-3t+5} \right)^2$.
\answer{$\frac{4}{49}$}
\item $\displaystyle\lim\limits_{x\to 2}\sqrt{ \frac{2x^2 + 1}{ 3x-2}}$.
\answer{$\frac32$}
\end{enumerate}
\end{multicols}

\end{problem}
\subsection{Limits involving $\infty $}
\subsubsection{Limits as $x\to\pm \infty$}\label{secMPSlimitsXtoInfty}
\begin{problem}
Find the limit or show that it does not exist. If the limit does not exist, indicate whether it is $\pm\infty$, or neither. The answer key has not been proofread, use with caution.
\begin{multicols}{3}
\begin{enumerate}[ref={\fcProblemRef}]
\item $\displaystyle \lim\limits_{x\to\infty }\frac{x-2}{2x+1}$.

\answer{$\frac12$}
\item $\displaystyle \lim\limits_{x\to\infty }\frac{1-x^2}{x^3-x-1}$.

\answer{$ 0$}
\item $\displaystyle \lim\limits_{x\to-\infty }\frac{x-2}{x^2+5}$.

\answer{$ 0$}
\item \label{problemlimxtominusinfty(3x^3+2)/(2x^3-4x+5)} $\displaystyle \lim\limits_{x\to-\infty }\frac{3x^3+2}{2x^3-4x+5}$.

\answer{$ \frac{3}{2}$}
\item $\displaystyle \lim\limits_{x\to\infty }\frac{\sqrt{x}+x^2}{\sqrt{x}-x^2}$.

\answer{$-1$}

\item $\displaystyle \lim\limits_{x\to\infty }\frac{3-x\sqrt{x}}{2x^{\frac{3}{2}}-2}$.

\answer{$-\frac12$}
\item $\displaystyle \lim\limits_{x\to\infty }\frac{(2x^2+3)^2}{(x-1)^2(x^2+1)}$.

\answer{$ 4$}
\item $\displaystyle \lim\limits_{x\to\infty }\frac{x^2-3}{\sqrt{x^4+3}}$.

\answer{$1$}

\item \label{problemlimxtoinftysqrt(x^2+1)/(x+1)}
$ \displaystyle \lim_{x\to -\infty} \frac{\sqrt{x^2+1}}{x+1} $. 

\answer{$-1$}


\item $\displaystyle \lim\limits_{x\to\infty }\frac{\sqrt{16x^6-3x}}{x^3+2}$.

\answer{$4$}
\item \label{problemLimitxtominusinftysqrt(16x^6-3x)/(x^3+2)}
$\displaystyle \lim\limits_{x\to-\infty }\frac{\sqrt{16x^6-3x}}{x^3+2}$.

\answer{$-4$}

\item \label{problemlimxtoinfty(sqrt(3x^2+2x+1)/(x+1))}

$ \displaystyle \lim_{x\to \infty} \frac{\sqrt{3x^2+2x+1}}{x+1} $. 

\answer{$\sqrt{3}$}
\item $\displaystyle \lim\limits_{x\to\infty}\sqrt{4x^2+x}-2x$.

\answer{$\frac{1}{4}$}
\item $\displaystyle \lim\limits_{x\to-\infty} x+\sqrt{x^2+3x} $.

\answer{$-\frac{3}{2} $}
\item $\lim\limits_{x \to\infty}\sqrt{x^2+2x}- \sqrt{ x^2 -2x} $. 

\answer{ $2 $.}
\item \label{problemlim(xto-infty)sqrt(x^2+x)-sqrt(x^2-x)} $\lim\limits_{x\to -\infty}\sqrt{x^2+x}- \sqrt{ x^2-x}$. 

\answer{$-1 $.}

\item $\displaystyle \lim\limits_{x\to \infty}\sqrt{ x^2+ax}- \sqrt{x^2+bx}$.

\answer{$\frac{a-b}2$}
\item $\displaystyle \lim\limits_{x\to\infty}\cos x$.

\answer{DNE}
\item $\displaystyle \lim\limits_{x\to\infty}\frac{x^4+x}{x^3-x+2}$.

\answer{$\infty$}
\item $\displaystyle \lim\limits_{x\to\infty}\sqrt{x^2+1}$.

\answer{$\infty$}
\item $\displaystyle \lim\limits_{x\to-\infty}(x^4+x^5)$.

\answer{$-\infty$}
\item $\displaystyle \lim\limits_{x\to-\infty}\frac{\sqrt{1+x^6}}{1+x^2}$.

\answer{$\infty$}
\item $\displaystyle \lim\limits_{x\to\infty}(x-\sqrt{x})$.

\answer{$\infty$}
\item $\displaystyle \lim\limits_{x\to\infty}(x^2-x^3)$.

\answer{$-\infty$}
\item $\displaystyle \lim\limits_{x\to\infty}x\sin x$.

\answer{DNE}
\item $\displaystyle \lim\limits_{x\to\infty}\sqrt{x}\sin x$.

\answer{DNE}
\end{enumerate}
\end{multicols}
\end{problem}
\solution{\ref{problemlimxtominusinfty(3x^3+2)/(2x^3-4x+5)}.


\[
\begin{array}{rcll|l}
\displaystyle \lim_{x\to-\infty} \frac{3x^3+2}{2x^3-4x+5}&=&\displaystyle \lim_{x\to-\infty} \frac{\left(3x^3+2\right) \frac{1}{x^3}}{ \left(2x^3-4x +5\right)\frac{1}{ x^3}}&& \begin{array}{l} \text{Divide top}\\ \text{and bottom}\end{array}\\
&=&\displaystyle \lim_{x\to-\infty} \frac{3+\frac{2}{x^3}}{2-\frac{4}{x^2}+\frac{5}{x^3}} && \begin{array}{l} \text{by highest term}\\ \text{in denominator}\end{array}\\
&=&\displaystyle\frac{3+0}{2-0+0}=\frac{3}{2}.
\end{array}
\]
}

\solution{\ref{problemlimxtoinftysqrt(x^2+1)/(x+1)}

$ 
\begin{array}{rcll|l}
\displaystyle \lim_{x\to -\infty} \frac{\sqrt{x^2+1}}{x+1} &=&\displaystyle \lim_{x\to- \infty} \frac{\frac{1}{x}\sqrt{x^2+1}}{\frac{1}{x}(x+1)}=\displaystyle \lim_{x\to- \infty} \frac{-\frac{1}{\sqrt{x^2}}\sqrt{x^2+1}}{\frac{1}{x}(x+1)}&& x=-\sqrt{x^2}, \text{ whenever } x<0\\
&=&\displaystyle \lim_{x\to- \infty} \frac{-\sqrt{\frac{x^2+1}{x^2}}}{1+\frac{1}{x}}= \lim_{x\to- \infty} \frac{-\sqrt{1+\underbrace{\frac{1}{x^2}}_{\to 0}}}{1+\underbrace{\frac{1}{x}}_{x\to 0}}\\
&=&1.
\end{array}
$
}
\solution{\ref{problemLimitxtominusinftysqrt(16x^6-3x)/(x^3+2)}.

$
\begin{array}{rcll|l}
\displaystyle 
\lim\limits_{x\to-\infty }\frac{\sqrt{16x^6-3x}}{x^3+2}&=&\displaystyle \lim\limits_{x\to-\infty } \frac{\sqrt{x^6\left(16-\frac{3}{x^5}\right)}}{x^3+2}\\
&=&\displaystyle \lim\limits_{x\to-\infty } \frac{\sqrt{x^6}\sqrt{\left(16-\frac{3}{x^5}\right)}}{x^3 +2} &&\displaystyle\sqrt{x^6}={\color{red}{-}} x^3\text{ because }x<0 \text{ as }x\to-\infty\\
&=&\displaystyle \lim\limits_{x\to-\infty } \frac{{\color{red}{-}}x^3 \sqrt{\left(16-\frac{3}{x^5}\right)}}{x^3 +2} &&\text{Divide by highest order term in denominator} \\
&=&\displaystyle \lim\limits_{x\to-\infty } \frac{-x^3 \sqrt{\left(16-\frac{3}{x^5}\right)}}{x^3 +2} \\
&=&\displaystyle \lim\limits_{x\to-\infty } \frac{-\cancel{x^3} \sqrt{\left(16-\frac{3}{ x^5}\right)}\frac{1}{\cancel{x^3}}}{\left(x^3 +2\right)\frac{1}{x^3}} \\
&=&\displaystyle \lim\limits_{x\to-\infty } \frac{-\sqrt{\left(16-\underbrace{\frac{3}{ x^5}}_{\to 0}\right)} }{1+\underbrace{\frac{2}{x^3}}_{\to 0}} \\
&=&\displaystyle \frac{-\sqrt{16}}{1}=-4.
\end{array}
$
}
\solution{\ref{problemlimxtoinfty(sqrt(3x^2+2x+1)/(x+1))}

$\begin{array}{rcll|l}
\displaystyle \lim_{x\to \infty} \frac{\sqrt{3x^2+2x+1}}{x+1} &=&\displaystyle \lim_{x\to \infty} \frac{\frac{1}{x} \sqrt{3x^2+2x+1}}{\frac{1}{x}\left(x+1\right) } \\
&=&\displaystyle\lim_{x\to \infty} \frac{ \sqrt{\frac{3x^2+2x+1}{x^2 }}}{\left(1+\frac{1}{x}\right) }\\
&=&\displaystyle \lim_{x\to \infty} \frac{ \sqrt{3+\frac{2}{x}+\frac{1}{x^2 }}}{\left(1+\frac{1}{x}\right) }\\
&=&\displaystyle\frac{\sqrt{3+0+0}}{1+0}\\
&=&\displaystyle \sqrt{3}
\end{array}
$
}

\solution{\ref{problemlim(xto-infty)sqrt(x^2+x)-sqrt(x^2-x)}.
\[ \begin{array}{rcl}
\displaystyle\lim_{x\to -\infty} \sqrt{x^2+x}-\sqrt{x^2-x} &=&\displaystyle\lim_{x\to -\infty} \left(\sqrt{x^2+x}-\sqrt{x^2-x}\right) \frac{ \left(\sqrt{x^2+x}+\sqrt{x^2-x}\right) }{\left( \sqrt{x^2+x}+\sqrt{x^2-x}\right)}
\\
&=&\displaystyle \lim_{x\to -\infty} \frac{x^2+x-(x^2-x) }{\sqrt{x^2+x}+\sqrt{x^2-x} } = \lim_{x\to -\infty} \frac{2x \frac{1}{x} }{\left(\sqrt{x^2+x}+\sqrt{x^2-x} \right)\frac{1}{x}} 
\\&=&\displaystyle \lim_{x\to -\infty} \frac{2}{\frac{\sqrt{x^2+x}}x+\frac{\sqrt{x^2-x}}x }= \lim_{x\to -\infty} \frac{2}{ {\color{red}-} \sqrt{\frac{x^2+x}{x^2}} {\color{red}-} \sqrt{\frac{x^2-x}{x^2}} }\\
&=&\displaystyle \lim_{x\to -\infty} \frac{2}{ - \sqrt{1+\frac1x} - \sqrt{1-\frac1x} }=\frac{2}{-\sqrt{1+0}-\sqrt{1-0}}=-1.
\end{array}
\]
The sign highlighted in red arises from the fact that, for negative $x$, we have that $ x={\color{red}-}\sqrt{x^2}$.
}



\subsubsection{Limits involving vertical asymptote}\label{secMPSlimitsVerticalAsymptote}
\begin{problem}
Show the following limits do not exist and compute whether they evaluate to $\infty $, $-\infty$, or neither. 
\begin{multicols}{3}
\begin{enumerate}
\item $\displaystyle\lim_{x\to 3^+} \frac{x^{2}+x-1}{x^2-2x-3} $.
\answer{$\infty$.}
\item $\displaystyle\lim_{x\to 3^-} \frac{x^{2}+x-1}{x^2-2x-3} $.
\answer{$-\infty$.}
\item $\displaystyle\lim_{x\to 1^+} \frac{x^2+1}{\sqrt{x^2+3 }-2} $.
\answer{$\infty$.}
\item $\displaystyle\lim_{x\to 1^-} \frac{x^2+1}{\sqrt{x^2+3 }-2} $.
\answer{$-\infty$.}
\item $\displaystyle\lim_{x\to 2^+} \frac{\sqrt{x^3-8}}{ -x^2+x+2} $.
\answer{$-\infty$.}
\item $\displaystyle\lim_{x\to -1^+} \frac{\sqrt[3]{x^2+2x+1}}{ x^2-2x-3} $.
\answer{$-\infty$.}

\end{enumerate}
\end{multicols}

\end{problem}
\begin{problem}
Evaluate the limit if it exists.
\begin{enumerate}
\item $\displaystyle \lim\limits_{x\to 3^+} \frac{\sqrt{\frac{x^2}{9}-1 }}{2x^2 -3x-9 }$. \answer{$ \infty$.}
\solution{ 
We have that 
\[
\begin{array}{rcl}
\displaystyle \lim\limits_{x\to 3^+}\frac{\sqrt{\frac{x^2}9-1} }{2x^2-3x-9}&=&\displaystyle  \lim\limits_{x\to 3^+} \frac{\sqrt{(\frac{x}3-1)(\frac{x}3+1)} }{2(x+\frac{3}2)(x-3)}= \lim\limits_{x\to 3^+}
\frac{\left(\frac{1}3 (x-3)(\frac{x}3+1) \right)^{ \frac{1}2}}{ 2(x+\frac{3}2)(x-3)} \\~\\
&=&\displaystyle  \lim\limits_{x\to 3^+} \frac{\sqrt{\frac{1}3\left(\frac{x}3+1\right)} }{2\left(x+\frac{3}2\right)(x-3)^{\frac12}}= \lim\limits_{x\to 3^+} \frac{\sqrt{\underbrace{ \frac{1}3\left(\frac{x}3+1\right)}_{\to\frac23 }} }{\underbrace{2\left(x+\frac{3}2\right)}_{\to 9}\underbrace{(x-3)^{\frac12}}_{\to 0^+}}=\infty,
\end{array}
\]
where the latest term is $+\infty$ because it is of the form $\frac{(+)}{(+)(+)}$.
}
\item $\displaystyle \lim\limits_{x\to -2^-} \frac{\sqrt{\frac{x^2}{4}-1 }}{2x^2 +3x-2 }$. \answer{$ \infty$.}
\end{enumerate}

\end{problem}
\solution{\ref{problemlimxto3+sqrt(x^2/9-1)/(2x^2-3x-9)}.
We have that 
\[
\begin{array}{rcl}
\displaystyle \lim\limits_{x\to 3^+}\frac{\sqrt{\frac{x^2}9-1} }{2x^2-3x-9}&=&\displaystyle  \lim\limits_{x\to 3^+} \frac{\sqrt{(\frac{x}3-1)(\frac{x}3+1)} }{2(x+\frac{3}2)(x-3)}= \lim\limits_{x\to 3^+}
\frac{\left(\frac{1}3 (x-3)(\frac{x}3+1) \right)^{ \frac{1}2}}{ 2(x+\frac{3}2)(x-3)} \\~\\
&=&\displaystyle  \lim\limits_{x\to 3^+} \frac{\sqrt{\frac{1}3\left(\frac{x}3+1\right)} }{2\left(x+\frac{3}2\right)(x-3)^{\frac12}}= \lim\limits_{x\to 3^+} \frac{\sqrt{\underbrace{ \frac{1}3\left(\frac{x}3+1\right)}_{\to\frac23 }} }{\underbrace{2\left(x+\frac{3}2\right)}_{\to 9}\underbrace{(x-3)^{\frac12}}_{\to 0^+}}=\infty,
\end{array}
\]
where the latest term is $+\infty$ because it is of the form $\frac{(+)}{(+)(+)}$.
}
\subsubsection{Find the Horizontal and Vertical Asymptotes}\label{secMPShorAndVertAsymptotes}
\begin{problem}
Find the horizontal and vertical asymptotes of the graph of the function. Check your work by plotting the function.

\begin{multicols}{2}
\begin{enumerate}[ref={\fcProblemRef}]
\item 
\label{problemAsymptotesy=(2x/(sqrt(x^2+x+3)-3))} $\displaystyle y=\frac{2x}{\sqrt{x^2+x+3}-3}$. 

\answer{Vertical: $x=2, x=-3$, horizontal: $y=2, y=-2$}

\item $\displaystyle y=\frac{3x^2}{\sqrt{x^2+2x+10}-5}$. 

\answer{Vertical: $x=3, x=-5$, horizontal: none.}
\item $\displaystyle y=\frac{3x+1}{x-2}$.

\answer{vertical: $x=2$, horizontal: $y=3$}
\item \label{problemAsymptotesy=(x^2-1)/(2x^2-3x-2)}
$\displaystyle y=\frac{x^2-1}{2x^2-3x-2}$.

\answer{vertical: $x=2, x=-\frac{1}{2}$, horizontal: $y=\frac{1}{2}$}

\item $\displaystyle y=\frac{2x^2-2x-1}{x^2+x-2}$.

\answer{vertical: $x=1, x=-2$, horizontal: $y=2$}

\item \label{problemAsymptotesy=(-5x^2-3x+5)/(x^2-2x-3)}
$\displaystyle 
f(x)=\frac{-5 x^{2}-3 x+5}{x^{2}-2 x-3}
$

\answer{vertical: $x=-1$, $x=3$, horizontal: $y=-5$}



\item $\displaystyle y=\frac{1+x^4}{x^2-x^4}$.

\answer{vertical: $x=0, x=1, x=-1$, horizontal: $y=-1$}
\item $\displaystyle y=\frac{x^3-x}{x^2-7x+6}$.

\answer{vertical: $x=6$, no horizontal asymptote}
\item $\displaystyle y=\frac{x-9}{\sqrt{4x^2+3x+3}}$.

\answer{no vertical asymptote, horizontal: $y=\pm\frac12$}

\item $\displaystyle y=\frac{\sqrt{x^2+1}- x }{x}$.

\answer{vertical: $x=0$, horizontal: $y=0$, $y=-2$}
\item \label{problemAsymptotesy=x/(sqrt(x^2+3) -2x)}
$\displaystyle 
f(x)= \frac{x}{\sqrt{x^2+3} -2x}
$

\answer{vertical: $x=1$, horizontal: $y=-\frac{1}{3}$, $y=-1$}

\end{enumerate}
\end{multicols}
\end{problem}
\solution{\ref{problemAsymptotesy=(2x/(sqrt(x^2+x+3)-3))}
\textbf{Vertical asymptotes.} A function $f(x)$ has a vertical asymptote at $x=a$ if $\lim\limits_{x\to a} f(x)=\pm \infty$. 

The function is algebraic, and therefore has a finite limit at every point it is defined (i.e., no asymptote). Therefore the function can have vertical asymptotes only for those $x$ for which $f(x)$ is not defined. The function is not defined for $\sqrt{x^2+x+3}-3=0$, which has two solutions, $x=2$ and $x=-3$. These are precisely the vertical asymptotes: indeed, 
\[
\lim\limits_{x\to 2^+} \frac{2x}{\sqrt{x^2+x+3}-3}=\infty \quad \quad \quad 
\lim\limits_{x\to 2^-} \frac{2x}{\sqrt{x^2+x+3}-3}=-\infty 
\]
and
\[
\lim\limits_{x\to -3^+} \frac{2x}{\sqrt{x^2+x+3}-3}=\infty \quad \quad \quad 
\lim\limits_{x\to -3^-} \frac{2x}{\sqrt{x^2+x+3}-3}=-\infty 
\]

\textbf{Horizontal asymptotes.} A function $f(x)$ has a horizontal asymptote if $\lim\limits_{x\to \pm\infty} f(x)$ exists. If that limit exists, and is some number, say, $N$, then $y=N$ is the equation of the corresponding asymptote.

Consider the limit $x\to -\infty$. We have that 
\[
\begin{array}{rcll|l}
\displaystyle \lim\limits_{x\to -\infty} \frac{2x}{ \sqrt{x^2+3x+3} - 3}&=&\displaystyle \lim\limits_{x\to - \infty} \frac{ 2}{ \frac{ \sqrt{ x^2 + x+3}}x-\frac3x}\\
&=&\displaystyle  \lim\limits_{x\to - \infty} \frac{2}{-\sqrt{\frac{ x^2 +3x+3}{x^2}}-\frac3x}  && \frac{1}{x} =- \sqrt{\frac{1}{x^2} } \text{ when } x<0\\
&=&\displaystyle \lim\limits_{x\to - \infty} \frac{2}{-\sqrt{1+ \frac{3}{x} + \frac{3}{x^2}}-\frac3x}\\
&=& \displaystyle \frac{\lim\limits_{x\to - \infty} 2}{-\sqrt{ \lim \limits_{x \to - \infty} 1+\lim\limits_{x\to - \infty} \frac{3}{x} + \lim\limits_{x \to - \infty} \frac{3}{x^2}}-\lim\limits_{x\to - \infty} \frac3x}\\
&=&\displaystyle \frac{2}{-\sqrt{1+0+0}-0}\\
&=&\displaystyle -2\quad . 
\end{array}
\]
Therefore $y=-2$ is a horizontal asymptote. 

The case $x\to \infty$, is handled similarly and yields that $y=2$ is a horizontal asymptote.

A computer generated graph confirms our computations.

\psset{xunit=0.2cm, yunit=0.2cm}
\begin{pspicture}(-16, -20)(16,17)
\tiny
\fcAxesStandard{-15}{-19.32133}{15}{16.190354}
\fcXTick{10}
\rput[t](10, -0.6){$10$}
%Function formula: \frac{2 x}{\sqrt{x^{2}+x+3}-3}
\psplot[linecolor=\fcColorGraph, plotpoints=1000]{2.344}{15}{x  2 mul    -3  3 x add   x  2 exp   add    0.5 exp   add   div  }
%Function formula: \frac{2 x}{\sqrt{x^{2}+x+3}-3}
\psplot[linecolor=\fcColorGraph, plotpoints=1000]{-2.6}{1.78}{x  2 mul    -3  3 x add   x  2 exp   add    0.5 exp   add   div  }
%Function formula: \frac{2 x}{\sqrt{x^{2}+x+3}-3}
\psplot[linecolor=\fcColorGraph, plotpoints=1000]{-15}{-3.4}{x  2 mul    -3  3 x add   x  2 exp   add    0.5 exp   add   div  }
\psline[linestyle=dotted](-3,-19.3)(-3,16.1)
\psline[linestyle=dotted](2,-19.3)(2,16.1)
\psline[linestyle=dashed, linecolor=blue](-15, 2)(15, 2)
\psline[linestyle=dashed, linecolor=blue](-15, -2)(15, -2)
\rput[b](-8, 2.6){$y=2$}
\rput[t](8, -2.6){$y=-2$}
\rput[bl](5,5){$y=\frac{2x}{\sqrt{x^2+x+3}-3}$}
\rput[l](2.6,-8){$x=2$}
\rput[r](-3.6,8){$x=-3$}
\end{pspicture}

}


\solution{\ref{problemAsymptotesy=(x^2-1)/(2x^2-3x-2)}



\textbf{Vertical asymptotes.} A function $f(x)$ has a vertical asymptote at $x=a$ if $\lim\limits_{x\to a} f(x)=\pm \infty$. 

The function is algebraic, and therefore has a finite limit at every point it is defined (i.e., no asymptote). Therefore the function can have vertical asymptotes only for those $x$ for which $f(x)$ is not defined. The function is not defined for $2x^2-3x-2=0$, which has two solutions, $x=2$ and $x=-\frac{1}{2}$. These are precisely the vertical asymptotes: indeed, 
\[
\begin{array}{rcll|l}
\displaystyle 
\lim\limits_{x\to 2^+} \frac{x^2-1}{2x^2-3x-2}&=&\displaystyle  \lim\limits_{x\to 2^+} \frac{x^2-1}{2(x-2) \left(x+\frac{1}{2}\right)} = \infty &&\text{Limit of form }\frac{(+)}{(+)(+)}\\
\displaystyle 
\lim\limits_{x\to 2^-} \frac{x^2-1}{2x^2-3x-2}&=&\displaystyle  \lim\limits_{x\to 2^-} \frac{x^2-1}{2(x-2) \left(x+\frac{1}{2}\right)} = -\infty &&\text{Limit of form }\frac{(+)}{(-)(+)}\\
\end{array}
\]
and
\[
\begin{array}{rcll|l}
\displaystyle 
\lim\limits_{x\to -\frac{1}{2}^+} \frac{x^2-1}{2x^2-3x-2}&=&\displaystyle  \lim\limits_{x\to -\frac{1}{2}^+} \frac{x^2-1}{2(x-2) \left(x+\frac{1}{2}\right)} = \infty &&\text{Limit of form }\frac{(-)}{(+)(-)}\\
\displaystyle 
\lim\limits_{x\to -\frac{1}{2}^-} \frac{x^2-1}{2x^2-3x-2}&=&\displaystyle  \lim\limits_{x\to -\frac{1}{2}^-} \frac{x^2-1}{2(x-2) \left(x+\frac{1}{2}\right)} = -\infty &&\text{Limit of form }\frac{(-)}{(-)(-)}\\
\end{array}
\]

\textbf{Horizontal asymptotes.} A function $f(x)$ has a horizontal asymptote if $\lim\limits_{x\to \pm\infty} f(x)$ exists. If that limit exists, and is some number, say, $N$, then $y=N$ is the equation of the corresponding asymptote.

We have that 
\[
\begin{array}{rcll|l}\renewcommand{\arraystretch}{1.6}
\displaystyle \lim\limits_{x\to \infty} \frac{x^2-1}{2x^2-3x-2} &=&\displaystyle \lim\limits_{x\to \infty} \frac{\left(x^2-1\right)\frac{1}{x^2}}{\left(2x^2-3x-2\right)\frac{1}{x^2}}&&\text{Divide by highest term in den.}\\
&=&\displaystyle  \displaystyle \lim\limits_{x\to \infty} \frac{1-\frac{1}{x^2}}{2-\frac{3}{x}-\frac{2}{x^2}} \\
&=&\displaystyle  \displaystyle  \frac{\lim\limits_{x\to \infty}1-\lim\limits_{x\to \infty}\frac{1}{x^2}}{\lim\limits_{x\to \infty}2-\lim\limits_{x\to \infty}\frac{3}{x}-\lim\limits_{x\to \infty}\frac{2}{x^2}}&&\text{Step may be skipped}\\
&=& \displaystyle \frac{1-0}{2-0-0}\\
&=&\displaystyle \frac{1}{2}\\
\end{array}
\]
A similar computation shows that 
\[
\begin{array}{rcll|l}\renewcommand{\arraystretch}{1.6}
\displaystyle \lim\limits_{x\to -\infty} \frac{x^2-1}{2x^2-3x-2} 
&=&\displaystyle \frac{1}{2}\\
\end{array}
\]

Therefore $y=\frac{1}{2}$ is the only horizontal asymptote, valid in both directions ($x\to \pm \infty$). 


A computer generated graph confirms our computations.

\psset{xunit=0.2cm, yunit=0.2cm}
\begin{pspicture}(-16, -20)(16,17)
\tiny
\fcAxesStandard{-15}{-19.32133}{15}{16.190354}
\fcXTick{10}
\newcommand{\theFun}{x x mul 1 sub 2 x x mul mul -3 x mul -2 add add div }
\psplot[linecolor=\fcColorGraph, plotpoints=1000]{-15}{-0.508}{\theFun}
\psplot[linecolor=\fcColorGraph, plotpoints=1000]{-0.49}{1.969}{\theFun}
\psplot[linecolor=\fcColorGraph, plotpoints=1000]{2.04}{15}{\theFun}
\psline[linestyle=dotted](-0.5,-19.3)(-0.5,16.1)
\psline[linestyle=dotted](2,-19.3)(2,16.1)
\psline[linestyle=dashed, linecolor=blue](-15, 0.5)(15, 0.5)
\rput[b](-8, 0.6){$y=\frac{1}{2}$}
\rput[bl](5,2){$y= \frac{x^2-1}{2x^2-3x-2}$}
\rput[l](2.6,-8){$x=2$}
\rput[r](-0.6,8){$x=-\frac{1}{2}$}
\end{pspicture}
}

\solution{\ref{problemAsymptotesy=(-5x^2-3x+5)/(x^2-2x-3)}

\textbf{Vertical asymptotes.} The function is rational, and therefore has a finite limit (and therefore no vertical asymptote) at every point it its domain. The function is not defined for $x^2-2x-3=0$, which has two solutions, $x=-1$ and $x=3$. These are precisely the vertical asymptotes: indeed, 
\[
\begin{array}{rcll|l}
\displaystyle 
\lim\limits_{x\to -1^+} \frac{-5x^2-3x+5}{x^2-2x-3}&=&\displaystyle  \lim\limits_{x\to -1^+} \frac{-5x^2-3x+5}{(x+1) \left(x-3\right)} = -\infty &&\text{Limit of form }\frac{(+)}{(+)(-)}\\
\displaystyle 
\lim\limits_{x\to -1^-} \frac{-5x^2-3x+5}{x^2-2x-3}&=&\displaystyle  \lim\limits_{x\to -1^-} \frac{-5x^2-3x+5}{(x+1) \left(x-3\right)} = \infty &&\text{Limit of form }\frac{(+)}{(-)(-)}\\
\end{array}
\]
and
\[
\begin{array}{rcll|l}
\displaystyle 
\lim\limits_{x\to 3^+} \frac{-5x^2-3x+5}{x^2-2x-3}&=&\displaystyle  \lim\limits_{x\to 3^+} \frac{-5x^2-3x+5}{(x+1) \left(x-3\right)} = -\infty &&\text{Limit of form }\frac{(-)}{(+)(+)}\\
\displaystyle 
\lim\limits_{x\to 3^-} \frac{-5x^2-3x+5}{x^2-2x-3}&=&\displaystyle  \lim\limits_{x\to 3^-} \frac{-5x^2-3x+5}{(x+1) \left(x-3\right)} = \infty &&\text{Limit of form }\frac{(-)}{(+)(-)}\\
\end{array}
\]

\textbf{Horizontal asymptotes.} 
\[
\begin{array}{rcll|l}\renewcommand{\arraystretch}{1.6}
\displaystyle \lim\limits_{x\to \pm\infty} \frac{-5x^2-3x+5}{x^2-2x-3} &=&\displaystyle \lim\limits_{x\to \pm\infty} \frac{\left(-5x^2-3x+5\right)\frac{1}{x^2}}{\left(x^2-2x-3\right)\frac{1}{x^2}}&&\text{Divide by highest term in den.}\\
&=&\displaystyle  \displaystyle \lim\limits_{x\to \pm\infty} \frac{-5-\frac{3}{x}+\frac{5}{x^2}}{1-\frac{2}{x}-\frac{3}{x^2}} \\
&=&\displaystyle  \displaystyle  \frac{-\lim\limits_{x\to \pm\infty}5-\lim\limits_{x\to \pm\infty}\frac{3}{x}+\lim\limits_{x\to \pm\infty}\frac{5}{x^2}}{\lim\limits_{x\to \pm\infty}1-\lim\limits_{x\to \pm\infty}\frac{2}{x}-\lim\limits_{x\to \pm\infty}\frac{3}{x^2}}&&\text{Step may be skipped}\\
&=& \displaystyle \frac{-5-0+0}{1-0-0}\\
&=&\displaystyle -5.\\
\end{array}
\]

Therefore $y=-5$ is the only horizontal asymptote, valid in both directions ($x\to \pm \infty$). 


A computer generated graph confirms our computations.

\psset{xunit=0.2cm, yunit=0.2cm}
\begin{pspicture}(-16, -20.9)(16,17.2)
\tiny
\fcAxesStandard{-15}{-20.8}{15}{17.5}
\fcXTick{10}
\newcommand{\theFun}{x x -5 mul mul -3 x mul 5 add add x x mul -2 x mul -3 add add div\space}
\psplot[linecolor=\fcColorGraph, plotpoints=1000]{-15}{-1.04}{\theFun}
\psplot[linecolor=\fcColorGraph, plotpoints=1000]{-0.96}{2.95}{\theFun}
\psplot[linecolor=\fcColorGraph, plotpoints=1000]{3.05}{15}{\theFun}
\psline[linestyle=dotted](-1,-19.3)(-1,16.1)
\psline[linestyle=dotted](3,-19.3)(3,16.1)
\psline[linestyle=dashed, linecolor=blue](-15, -5)(15, -5)
\rput[t](-8, -5.2){$y=-5$}
\rput[bl](5,2){$y= \frac{-5x^2-3x+5}{x^2-2x-3}$}
\rput[l](3.2,-8){$x=3$}
\rput[r](-0.6,8){$x=-1$}
\end{pspicture}
}

\solution{\ref{problemAsymptotesy=x/(sqrt(x^2+3) -2x)}

\textbf{Vertical asymptotes.} A function $f(x)$ has a vertical asymptote at $x=a$ if $\lim\limits_{x\to a} f(x)=\pm \infty$. 

The function is algebraic, and therefore has a finite limit at every point it is defined (i.e., no asymptote). Therefore the function can have vertical asymptotes only for those $x$ for which $f(x)$ is not defined. The function is not defined for 

\[
\begin{array}{rcll|l}
\sqrt{x^2+3}-2x&=&0\\
\sqrt{x^2+3}&=&2x&&\begin{array}{l} \text{square both sides}\\\text{may introduce extraneous solutions} \end{array}\\
x^2+3&=&4x^2\\
3x^2-3&=&0\\
3(x-1)(x+1)&=&0\\
x=1 \quad &\text{or}& \cancel{ x=-1}\\
&&x=-1 \text{ is extraneous:}\\
&& \sqrt{(-1)^2+3}-(-1)2=4\neq 0
\end{array}
\]

$x=-1$ is indeed a vertical asymptote:
\[
\lim\limits_{x\to 1^+}  \frac{x}{\sqrt{x^2+3} -2x}=\infty \quad \quad \quad 
\lim\limits_{x\to 1^-}  \frac{x}{\sqrt{x^2+3} -2x}=-\infty .
\]
\textbf{Horizontal asymptotes.} 
\[
\begin{array}{rcll|l}
\displaystyle \lim\limits_{x\to -\infty}  \frac{x}{\sqrt{x^2+3} -2x}&=&\displaystyle \lim\limits_{x\to - \infty} \frac{1}{\frac{\sqrt{x^2+3}}{x} -2} \\
&=& \displaystyle \lim\limits_{x\to - \infty} \frac{1}{-\sqrt{\frac{x^2+3}{x^2}} -2}   && \frac{1}{x} =- \sqrt{\frac{1}{x^2} } \text{ when } x<0\\
&=&\displaystyle \lim\limits_{x\to - \infty} \frac{1}{-\sqrt{1+\frac{3}{x^2}} -2}  \\
&=& \displaystyle \frac{1}{-\sqrt{1+0}-2}\\
&=&\displaystyle -\frac{1}{3}.\\
\displaystyle \lim\limits_{x\to -\infty}  \frac{x}{\sqrt{x^2+3} -2x}&=&\displaystyle \lim\limits_{x\to  \infty} \frac{1}{\frac{\sqrt{x^2+3}}{x} -2} \\
&=& \displaystyle \lim\limits_{x\to  \infty} \frac{1}{\sqrt{\frac{x^2+3}{x^2}} -2}   && \frac{1}{x} = \sqrt{\frac{1}{x^2} } \text{ when } x>0\\
&=&\displaystyle \lim\limits_{x\to  \infty} \frac{1}{\sqrt{1+\frac{3}{x^2}} -2}  \\
&=& \displaystyle \frac{1}{\sqrt{1+0}-2}\\
&=&\displaystyle -1.\\
\end{array}
\]
Therefore $y=-\frac{1}{3}$ and $y=-1$ are the two horizontal asymptotes. 


A computer generated graph confirms our computations.

\psset{xunit=0.2cm, yunit=0.2cm}
\begin{pspicture}(-16, -20)(16,17)
\tiny
\fcAxesStandard{-15}{-19.32133}{15}{16.190354}
\fcXTick{10}
\rput[t](10, -0.6){$10$}
\newcommand{\theFun}{x x x mul 3 add sqrt -2 x mul add div}
%Function formula: \frac{2 x}{\sqrt{x^{2}+x+3}-3}
\psplot[linecolor=\fcColorGraph, plotpoints=1000]{1.036}{15}{\theFun }
%Function formula: \frac{2 x}{\sqrt{x^{2}+x+3}-3}
\psplot[linecolor=\fcColorGraph, plotpoints=1000]{-15}{0.961}{\theFun }
\psline[linestyle=dotted](1,-19.3)(1,16.1)
\psline[linestyle=dashed, linecolor=blue](! -15 -1  3 div)(!15 -1 3 div)
\psline[linestyle=dashed, linecolor=blue](-15, -1)(15, -1)
\rput[b](-8, 0.2){$y=-\frac{1}{3}$}
\rput[t](8, -2){$y=-1$}
\rput[bl](5,5){$y=\frac{x}{\sqrt{x^2+3} -2x}$}
\rput[l](1.6,-8){$x=1$}
\end{pspicture}


}



\subsection{Limits - All Cases - Problem Collection}
\begin{problem}
%(Problem contributed by Gabe Cunningham) 
Find the following limits, or show that they do not exist:
\begin{multicols}{2}
\begin{enumerate}
\item ${\displaystyle \lim_{x \to 2} \frac{x^2-4}{x^2-x-2}}$
\answer{$\frac43$}
\item ${\displaystyle \lim_{x \to -\infty} \frac{5x^3+x-1}{2x^3-7}}$
\answer{$\frac{5}{2}$}
\item ${\displaystyle \lim_{x \to 1^{+}} \frac{x-3}{x-1}}$
\answer{$-\infty$}
\item ${\displaystyle \lim_{h \to 0} \frac{2(x+h)^3 - 2x^3}{h}}$
\answer{$6x^2$}
\item ${\displaystyle \lim_{x \to \infty} \frac{\sqrt{9x^2-2}}{x+4}}$
\answer{$3$}
\item ${\displaystyle \lim_{x \to -1} \frac{2x+3}{x+1}}$
\answer{Does not exist}
\end{enumerate}
\end{multicols}

\end{problem}
\subsection{Continuity}
\subsubsection{Continuity to evaluate limits}
\begin{problem}
Use continuity to evaluate the limits. The answer key has not been proofread, use with caution.

\begin{itemize}
\item $\displaystyle \lim_{x\to \frac{\pi}{4}} x\tan x  .$

\answer{$\frac{\pi}{4} $}
\item $\displaystyle \lim_{x\to 0} \frac{1}{1-\sqrt{3+\cos x}}$

\answer{$-1$}
\item $\displaystyle \lim_{x\to 0 } \tan (x+ \sin x)$

\answer{$0$}
\item $\displaystyle \lim_{x\to \pi} \cos(\ln x \sin x)$

\answer{$1$}
\end{itemize}
\end{problem}

\subsubsection{Conceptual problems} \label{secMPScontinuityConceptual}
\begin{problem}
\begin{problem}(Textbook, page 93, problem 63-64) 
For which values of $x$ is $f$ continuous?
\begin{itemize}
\item $f(x)=\doublebrace{0}{\mathrm{if~} x\mathrm{~is~rational}}{1}{\mathrm{if~}x~\mathrm{is~irrational}}$
\item $f(x)=\doublebrace{0}{\mathrm{if~} x\mathrm{~is~rational}}{x}{\mathrm{if~}x~\mathrm{is~irrational}}$
\end{itemize}
\end{problem}
\begin{problem} This problem is of higher difficulty and will not be used for tests. For which values of $x$ is $f$ continuous?
\[f(x)=\doublebrace{\frac{1}{q^2}}{\mathrm{if~}x\mathrm{~is~rational~and~} x=\frac{p}{q} }{0}{\mathrm{if~}x~\mathrm{is~irrational}}\]
where in the first item $p,q$ are relatively prime integers (i.e., integers without a common divisor).
\end{problem}
\end{problem}
\begin{problem}
Show that $f(x)$ is continuous at all irrational points and discontinuous at all rational ones.
\[
f(x)=\doublebrace{\frac{1}{q^2}}{\mathrm{if~}x\mathrm{~is~rational~and~} x=\frac{p}{q} }{0}{\mathrm{if~}x~\mathrm{is~irrational}}
\]
where in the first item $p,q$ are relatively prime integers (i.e., integers without a common divisor).

\end{problem}
\subsubsection{Continuity and Piecewise Defined Functions} \label{secMPScontinuityPiecewise}
\begin{problem}
Find the (implied) domain of $f(x)$. Extend the definition of $f$ at $x=3$ to make $f$ continuous at $3$.
\begin{multicols}{2}
\begin{enumerate}
\item $f(x)=\frac{x^2-x-6}{x-3}$.

\answer{
\begin{tabular}{l}
Implied domain: $x\in (-\infty, 3)\cup (3,\infty)$. \\
Extend $f(x)$ to $\bar f(x)=x+2$.
\end{tabular}
}
\item $f(x)=\frac{x^3-27}{x^2-9}$.

\answer{
\begin{tabular}{l}
Implied domain: $x\in (-\infty, -3)\cup(-3,3)\cup (3,\infty)$. \\
Extend $f(x)$ to $\bar f(x)=\frac{x^2+3x+9}{x+3}$ with domain $x\in (-\infty, -3)\cup(-3,\infty)$.
\end{tabular}
}
\end{enumerate}
\end{multicols}
\end{problem}
\begin{problem}
(Textbook, page 92, problem 41-43)
Find the numbers $x$ for which $f$ is discontinuous. At which of these numbers is $f$ continuous from the right, from the left, or neither? 
\begin{multicols}{2}
\begin{enumerate}
\item $f(x)=\triplebrace{1+x^2}{\mathrm{if~} x\leq 0 } {2-x}{\mathrm{if~} 0<x\leq 2}{(x-2)^2}{\mathrm{if~} x>2}$.
\item $f(x)=\triplebrace{x+1}{\mathrm{if~} x\leq 1 }{\frac{1}{x}}{\mathrm{if~}1<x<3}{\sqrt{x-3}}{\mathrm{if~} x\geq 3}$.
\item $f(x)=\triplebrace{x+2}{\mathrm{if~} x<0}{2x^2}{\mathrm{if~} 0\leq x\leq 1 }{2-x}{\mathrm{if~}x>1 }$
\end{enumerate}
\end{multicols}

\end{problem}
\begin{problem}
Find the values of $a$ and $b$ that make $f$ continuous everywhere.
\begin{enumerate}[ref={\fcProblemRef}]
\item 
$\displaystyle
f(x)= \left\{\begin{array}{ll}
\displaystyle 1 & {\text{if~}x<0}\\ 
\displaystyle {ax+b}& {\text{if~}0\leq x<1}\\ 
{2x}& \text{if~}{x\geq 1}\end{array}\right. .
$
\item
$\displaystyle
f(x)= \left\{\begin{array}{ll}
\displaystyle {\frac{x^2-1}{x-1}} & {\text{if~}x<1}\\ 
\displaystyle {ax^2-bx+3}& {\text{if~}1\leq x<3}\\ 
{2x-a+b}& \text{if~}{x\geq 3}\end{array}\right. .
$

\end{enumerate}

\end{problem}
\subsection{Intermediate Value Theorem}\label{secMPSintermediateValueTheorem}
\begin{problem}
Use the Intermediate Value Theorem to show that there is a real number solution of the given equation in the specified interval. 


\begin{multicols}{2}
\begin{enumerate}[ref={\fcProblemRef}]
\item 
$x^5+x-3=0$ where $x\in (1,2)$.

\item $\sqrt[4]{x}=1-x$ where $x\in \mathbb R$ (i.e., $x$ is an arbitrary real number).
\item $\cos x=2x$, where $x\in (0,1)$.
\item 
$\sin x=x^2-x-1$, where $x\in \mathbb R$ (i.e., $x$ is an arbitrary real number).

\item 
$\cos x=x^4$, where $x\in \mathbb R$ (i.e., $x$ is an arbitrary real number).

\item $x^5-x^2+x+3=0$, where $x\in \mathbb R$.



\end{enumerate}
\end{multicols}

\end{problem}
\input{../../modules/continuity/homework/IVT-problems-solutions}
\begin{problem}
~
\begin{enumerate}[ref={\fcProblemRef}]
\item \label{problemIVTtoshowx^2+13x+14=sinx-has-solutions} (1) Solve the equation $x^2+13x+41=1$.  (2) Use the intermediate value theorem to prove that the equation $x^2+13x+41=\sin  x$ has at least two solutions, lying between the two numbers found in (1).
\item (1) Solve the equation $x^2-15x+55=1$.  (2) Use the intermediate value theorem to prove that the equation $x^2-15x+55=\cos  x$ has at least two solutions, lying between the two numbers found in (1).
\end{enumerate}

\end{problem}
\solution{\ref{problemIVTtoshowx^2+13x+14=sinx-has-solutions}.
\noindent (1)
\begin{eqnarray*}
x^2+13x+41&=&1\\
x^2+13x+40&=&0\\
(x+5)(x+8)&=&0\quad .
\end{eqnarray*}
Therefore the two solutions are $x_1=-5$ and $x_2=-8$.

\noindent (2) Consider the function
\[
f(x)=x^2+13x+41-\sin x\quad.
\]
Our strategy for proving $f(x)=0$ has a solution consists in finding a number $a$ such that $f(a)<0$ and a number $b$ such that $f(b)>0$, and then using the Intermediate Value Theorem (IVT) with $N=0$.

Let
\[
g(x)=x^2+13x+41,
\]
and so $f(x)=g(x)-\sin x$. We have no techniques for evaluating $\sin x$ without calculator, but we do have all knowledge necessary to evaluate $g(x)$. Indeed, from high school we know that the lowest point of the parabola $g(x)$ is located at $x=-\frac{13}2=-6.5$. Then $g(-6.5)= -1.25$. Therefore
\[
f(-6.5)=g(-6.5)-\sin(-6.5)= g(-6.5)+\sin (6.5)=-1.25+\sin 6.5 \leq -0.25,
\]
where for the very last inequality we use the fact that $\sin 6.5< 1 $ (remember $\sin t\leq 1$ for all real values of $t$).

On the other hand,
\[f(-5)= g(-5)-\sin (-5) = 1+\sin 5> 0\]
as $\sin 5 >-1$ (remember $\sin t\geq -1$ for all real values of $t$). Therefore $f(-5)>0$ and $f(-6.5)<0$ and by the Intermediate Value Theorem (IVT) $f(x)=0$ has a solution in the interval $x\in (-6.5, -5)$.

Proving $f(x)=0$ has a solution in the interval $x\in (-8, -6.5)$ is similar and we leave it to the student.

Below is a computer generated plot of the function with the use of which we can visually verify our answer.

\psset{xunit=1cm, yunit=1cm}
\begin{pspicture}(-9, -5)(1,5)
\psframe*[linecolor=white](-9,-5)(1,5)
\tiny
\psaxes[ticks=none, labels=none]{<->}(0,0)(-9,-4.5)(1,4.5)
\fcLabels{1}{5}
\fcXTickWithLabel{-6.5}{$-6.5$}
\fcXTickWithLabel{-8}{$-8$}
\fcXTickWithLabel{-5}{$-5$}
%Function formula: (x)^{2}+40+13 (x)
\rput(-4.5,-2){$y=x^{2}+13 x+40$}
\psplot[linecolor=grey!30, plotpoints=1000]{-9}{-4}{x 13 mul 40 x 2 exp add add }
\rput(-6.5,3){$y=x^{2}+41+13 x- \sin x$}
\psplot[linecolor=\fcColorGraph, plotpoints=1000]{-9}{-4}{x 57.29578 mul sin -1 mul x 13 mul 41 x 2 exp add add add }

\end{pspicture}
}
\section{Inverse Functions}\label{secMPSInverseFunctions}

\subsection{Problems Using Rational Functions Only}
\begin{problem}
% begin homework inverse-functions3
Find the inverse function. You are asked to do the algebra only; you are not asked to determine the domain or range of the function or its inverse. 
\begin{enumerate}[ref={\fcProblemRef}]
\item $f(x)= 3x^2+4x-7$, where $x\geq -\frac{2}{3}$.

\answer{$f^{-1}(x)= -\frac{2}3+\frac{\sqrt{25+3x}}{3}, \quad x\geq -\frac{25}{3}$}
\item $f(x)= 2x^2+3x-5$, where $x\geq -\frac{3}{4}$.

\answer{$f^{-1}(x)=-\frac{3}{4}+\frac{\sqrt{49+8x}}{4}, \quad x\geq -\frac{49}{8}$}
\item $\displaystyle f(x)= \frac{2x+5}{x-4}$, where $x\neq 4$.

\answer{$f^{-1}(x)=\frac{4x+5}{x-2}, \quad x\neq 2$}
\pointsii{3} 
\label{problemFindInversef=(3x+5)/(2x-4)} $\displaystyle f(x)= \frac{3x+5}{2x-4}$, where $x\neq 2$.

\hiddenanswer{$\displaystyle f^{-1}(x) = \frac{ 4 x +5}{2x-3}, \quad x\neq \frac{3}{2}$}

\item \label{problemFindIversef=(5x+6)/(4x+5)}  $\displaystyle f(x)= \frac{5x+6}{4x+5}$, where $x\neq -\frac{5}{4}$.

\answer{$f^{-1}(x)= \frac{-5x+6}{4x-5}$, $x\neq \frac{5}{4}$}

\item  $\displaystyle f(x)= \frac{2x-3}{-3x+4}$, where $x\neq \frac{4}{3}$..

\answer{$f^{-1}(x)=\frac{4x+3}{3x+2}  $, $x\neq -\frac{2}{3}$}
\end{enumerate}
% end homework inverse-functions3

\end{problem}
\solution{\ref{problemFindInversef=(3x+5)/(2x-4)}
This is a concise solution written in form suitable for test taking.
\[
\begin{array}{rcl}
y & =& \displaystyle \frac{3x+5}{2x-4} \\
y(2x-4) & =& 3x+5 \\
2xy-4y & =& 3x+5 \\
2xy-3x & =& 4y+5 \\
x(2y-3) & =& 4y+5 \\
x & =& \displaystyle  \frac{4y+5}{2y-3} \\
\text{Therefore}\quad \displaystyle  f^{-1}(y) & =& \displaystyle  \frac{5+4y}{2y -3 } \\
\displaystyle f^{-1}(x) & =& \displaystyle \frac{5+4x}{2x-3}.
\end{array}
\]
}%
\solution{
\ref{problemFindIversef=(5x+6)/(4x+5)}. Set $f(x)=y$. Then
\[
\begin{array}{rcl}
y&=&\displaystyle \frac{5x+6}{4x+5}\\
y(4x+5)&=&5x+6\\
x(4y-5)&=&-5y+6\\
x&=&\displaystyle \frac{-5y+6}{4y-5} .
\end{array}
\]
Therefore the function $\displaystyle x=g(y)=\frac{-5y+6}{4y-5}$ is the inverse of $f(x)$. We write $g=f^{-1}$. The function $g=f^{-1}$ is defined for $\displaystyle y\neq \frac{5}{4}$. For our final answer we relabel the argument of $g$ to $x$:

\[
g(x)=f^{-1}(x)= \frac{-5x+6}{4x-5}\quad .
\]

Let us check our work. In order for $f$ and $g$ to be inverses, we need that $g(f(x))$ be equal to $x$.
\[
g(f(x))=  \frac{-5f(x) +6}{4f(x)-5}=  \frac{-5\frac{(5x+6)}{4x+5} +6}{4\frac{(5x+6)}{4x+5}-5}= \frac{-5(5x+6) +6(4x+5)}{4(5x+6)-5(4x+5)}=\frac{-x}{-1}=x\quad ,
\]
as expected.
}




\begin{problem}
Find the inverse function $f^{-1}$. Plot roughly by hand $y=f(x)$. Using the plot of $y=f(x)$, plot roughly by hand $f^{-1}(x)$. Indicate the relationship between the graph of $f(x)$ and $f^{-1}(x)$.
\begin{enumerate}
\item $f(x)= x^2+2x-2$,\quad \quad \quad $ x\geq -1$. 
\answer{
$f^{-1}(x)=\sqrt{x+3}-1$
\psset{xunit=0.2cm, yunit=0.2cm}
\begin{pspicture}(-3, -5)(6,5) 
\psframe*[linecolor=white](-3,-5)(6,5) 
\tiny 
\psaxes[ticks=none, labels=none]{<->}(0,0)(-3,-4.5)(6,4.5)
\psLabels{6}{5}
%Function formula: (x+3)^{1/2}-1 
\psplot[linecolor=\psColorGraph, plotpoints=1000]{-3}{6}{-1 3 x add 0.5 exp add }
%Function formula: x^{2}+2 x-2 
\psplot[linecolor=\psColorGraph, plotpoints=1000]{-1}{2}{-2 x 2 mul add x 2 exp add }
\end{pspicture} 
}
\item $f(x)= x^2+x-2$, \quad \quad \quad $ x\geq -\frac{1}{2}$.

\answer{
$f^{-1}(x)=\frac{ \sqrt{4 x+9}-1}2$
\psset{xunit=0.2cm, yunit=0.2cm}
\begin{pspicture}(-2.25, -5)(4,5) 
\psframe*[linecolor=white](-2.25,-5)(4,5) 
\tiny 
\psaxes[ticks=none, labels=none]{<->}(0,0)(-2.25,-4.5)(4,4.5)
\psLabels{4}{5}
%Function formula: 1/2 (4 x+9)^{1/2}-1/2 
\psplot[linecolor=\psColorGraph, plotpoints=1000]{-2.25}{4}{-0.5 9 x 4 mul add 0.5 exp 0.5 mul add }
%Function formula: x^{2}+x-2 
\psplot[linecolor=\psColorGraph, plotpoints=1000]{-0.5}{2}{-2 x add x 2 exp add }
\end{pspicture} 
}
\end{enumerate}
\end{problem}

\subsection{Problems Involving Exponents, Logarithms}
\begin{problem}
% begin homework inverse-functions2
Find the inverse function and its domain. 

\begin{multicols}{2}
\begin{enumerate}[ref={\fcProblemRef}]
\item \label{problemFindInversey=ln(x+3)} $\displaystyle y=\ln (x+3)$.

\answer{$f^{-1}(x)=e^x-3$}


\item \label{problemFindInversey=4ln(x-3)-4} $\displaystyle y=4 \ln{}\left({{x}}-3\right)-4$.

\answer{$f^{-1}(x)=e^{\frac{x+4}{4} x}+3$}

\item $y=2 \ln{}\left(-2 {{x}}+4\right)+1$

\answer{$ f^{-1}(x)=-\frac{1}{2} e^{\frac{1}{2} x-\frac{1}{2}}+2$}

\item  $f(x)=e^{x^3}$.

\answer{$f^{-1}(x)=\sqrt[3]{\ln x}, \quad x>0$}
\item \label{problemFindInversey=(lnx)^2} $\displaystyle y=(\ln x)^2$, $x\geq 1$.

\answer{$f^{-1}(x)=e^{\sqrt{x}}, \quad x\geq 0 $}

\pointsii{5} 
\label{problemFindInversey=e^x/(1+2e^x)}  $\displaystyle y=\frac{e^x}{1+2e^x}$.

\hiddenanswer{$f^{-1}(x)= \ln \left(\frac{x}{1-2x}\right) $, \quad $x\in \left(0, \frac12\right) $}

\item \label{problemFindInversef=2^(2x)+2^x-2} $f(x)=2^{2x}+2^{x}-2$.

\answer{$f^{-1}(x) =\log_2\frac{-1+\sqrt{9+4x}}{2}, \quad x\geq -2$}
\end{enumerate}
\end{multicols}
% end homework inverse-functions2

\end{problem}
\solution{\ref{problemFindInversey=ln(x+3)}
\begin{align*}
y & = \ln (x+3) \\
e^y & = e^{\ln (x+3)} \\
e^y & = x + 3 \\
e^y - 3 & = x \\
\text{Therefore} \quad f^{-1}(y) & = e^y - 3.
\end{align*}
The domain of $e^y$ is all real numbers, so the domain of $f^{-1}$ is all real numbers.  
}%

\solution{\ref{problemFindInversey=4ln(x-3)-4} 


\[
\renewcommand{\arraystretch}{1.6}
\begin{array}{rcll|l}
\displaystyle 4\ln (x-3)-4&=&\displaystyle y\\
\displaystyle 4\ln(x-3)&=&\displaystyle y+4\\
\displaystyle \ln(x-3)&=&\displaystyle \frac{y+4}{4}&&\text{exponentiate}\\
\displaystyle e^{\ln(x-3)}&=&\displaystyle e^{\frac{y+4}{4}}\\
\displaystyle x-3&=&\displaystyle e^{\frac{y+4}{4}}\\
\displaystyle f^{-1}(y)= x&=&\displaystyle e^{\frac{y+4}{4}}+3\\
\displaystyle f^{-1}(x)&=&\displaystyle e^{\frac{x+4}{4}}+3&& \text{relabel.}\\
\end{array}
\]
The domain of $f^{-1}$ is all real numbers (no restrictions on the domain).
}


\solution{ \ref{problemFindInversey=(lnx)^2}
\[ 
\begin{array}{rcll|l}y&=&(\ln x)^2 &&\mathrm{take~ \sqrt{~} ~on~both~sides,~ } y\geq 0 \\ \sqrt{y}&=&\ln x&&\mathrm{ ~exponentiate} \\ e^{\sqrt{y}}&=&e^{\ln x}=x \\ f^{-1}(y)&=&e^{\sqrt{y}} \\f^{-1}(x)&=&e^{\sqrt{x}} \end{array}
\]
}

\solution{\ref{problemFindInversey=e^x/(1+2e^x)}
\begin{align*}
y & = \frac{e^x}{1+2e^x} \\
y(1+2e^x) & = e^x \\
y & = e^x(1-2y) \\
\frac{y}{1-2y} & = e^x \\
\ln\frac{y}{1-2y} & = \ln e^x \\
\ln\frac{y}{1-2y} & = x \\
\text{Therefore} \quad f^{-1}(y) & = \ln\frac{y}{1-2y}.
\end{align*}
The natural logarithm function is only defined for positive input values.  
Therefore the domain is the set of all $y$ for which 
\begin{align*}
\frac{y}{1-2y} & > 0.
\end{align*}
This inequality holds if the numerator and denominator are both positive or both negative.  
This happens if either
\begin{enumerate}
\item  $y > 0$ and $y < \frac{1}{2}$, or 
\item  $y < 0$ and $y > \frac{1}{2}$.
\end{enumerate}
The latter option is impossible, so the domain is $\{ y \in \mathbb{R} \ | \ 0 < y < \frac{1}{2}\}$.  
}%


\section{Logarithms and Exponent Basics}\label{secMPSLogarithmsExponentsBasics}
\subsection{Exponents Basics}
\begin{problem}
% begin homework exponent-simplfy
Express each of the following as a single power.  

\begin{enumerate}
\item   $\displaystyle\frac{2^5\cdot 2^7}{2\sqrt{2}}$

\pointsii{2} $\displaystyle\frac{3^2\cdot 3^{-1}}{3^3\cdot \sqrt{3^3}}$

\solution{%
\begin{align*}
\frac{3^2\cdot 3^{-1}}{3^3\cdot\sqrt{3^3}} & = \frac{3^2\cdot 3^{-1}}{3^3\cdot (3^3)^{\frac{1}{2}}} \\
 & = \frac{3^2\cdot 3^{-1}}{3^3\cdot 3^{\frac{3}{2}}} \\
 & = \frac{3^{2-1}}{3^{3+\frac{3}{2}}} \\
 & = \frac{3^{1}}{3^{\frac{9}{2}}} \\
 & = 3^{1-\frac{9}{2}} \\
 & = 3^{-\frac{7}{2}}.
\end{align*}
}%

\item   $\displaystyle \frac{\pi^3}{\pi^{-1}\sqrt{\pi^5}}$

\end{enumerate}
% end homework exponent-simplify

\end{problem}
\solution{\ref{problemSimplify3^23^(-1)/(3^3sqrt(3^3))}.

\begin{align*}
\frac{3^2\cdot 3^{-1}}{3^3\cdot\sqrt{3^3}} & = \frac{3^2\cdot 3^{-1}}{3^3\cdot (3^3)^{\frac{1}{2}}} \\
 & = \frac{3^2\cdot 3^{-1}}{3^3\cdot 3^{\frac{3}{2}}} \\
 & = \frac{3^{2-1}}{3^{3+\frac{3}{2}}} \\
 & = \frac{3^{1}}{3^{\frac{9}{2}}} \\
 & = 3^{1-\frac{9}{2}} \\
 & = 3^{-\frac{7}{2}}.
\end{align*}
}%


\subsection{Logarithm Basics}
\begin{problem}
% begin homework logarithms-basic2
Use the definition of a logarithm to evaluate each of the following without using a calculator. The answer key has not been proofread, use with caution.

\begin{enumerate}[ref={\fcProblemRef}]
\item   $\displaystyle \log_2 16$


\answer{$4$}
\item   $\displaystyle\log_3 \left(\frac{1}{9}\right)$

\answer{$-2$}
\item   $\displaystyle\log_{10} 1000$

\answer{$3$}
\item   $\displaystyle\log_{6} 36^{-\frac{2}{3}}$

\answer{$-\frac{4}{3}$}
\item   $\displaystyle\log_{2} (8\sqrt{2})$

\answer{$\frac{7}{2}$}
\item   $\displaystyle\log_{\frac{1}{2}} (4)$

\answer{$-2$}
\item   $\displaystyle\log_{\frac{1}{9}} (\sqrt{3})$

\answer{$ -\frac{1}{4}$}
\end{enumerate}
% end homework logarithms-basic2

\end{problem}
\solution{\ref{problemSimplifylog_7(49^x/343^y)}.

\[
\begin{array}{rcl}
\log_7\left(\frac{49^x}{343^y}\right) & =& \log_749^x - \log_7343^y \\
 & =& x\log_749 - y\log_7343 \\
\text{However }49 = 7^2\text{ and }343=7^3,\text{ therefore }
\log_7\left(\frac{49^x}{343^y}\right) & =& 2x-3y.
\end{array}
\]
}%
\begin{problem}
% begin homework logarithms-combine
Express each of the following as a single logarithm. If possible, compute the logarithm without using a calculator. The answer key has not been proofread, use with caution.

\begin{enumerate}[ref={\fcProblemRef}]
\item   $\ln 4 + \ln 6 - \ln 5$.

\answer{$\ln \left(\frac{24}{5} \right) $}
\item \label{problem2ln(2)-3ln(3)+4ln(4)} $2\ln 2 - 3\ln 3 + 4\ln 4$.

\answer{$ \ln \left( \frac{1024}{27}\right)$}
\item   $\ln 36 - 2\ln 3 - 3\ln 2$.

\answer{$-\ln 2=\ln \left(\frac{1}{2}\right) $}

\item $\log_2(24)-\log_{4}9$.

\answer{$3$}

\item $\log_7(24)+\log_{\frac{1}{7}}3-\log_{49} (64)$.

\answer{$0$}
\item $\log_3(24)+\log_{3}\left(\frac{3}{8}\right)$.

\answer{$ 2$}

\end{enumerate}
% end homework logarithms-combine

\end{problem}
\solution{\ref{problem2ln(2)-3ln(3)+4ln(4)}.
\begin{align*}
2\ln 2 - 3\ln 3 + 4\ln 4 & = \ln 2^2 - \ln 3^3 + \ln 4^4 \\
 & = \ln 4 - \ln 27 + \ln 256 \\
 & = \ln \Big( \frac{4}{27}\Big) + \ln 256 \\
 & = \ln \Big( \frac{4\cdot 256}{27}\Big) \\
 & = \ln \Big( \frac{1024}{27}\Big).
\end{align*}

$\frac{1024}{27}$ is not a rational power of $e$, therefore $ \ln \left( \frac{1024}{27}\right)$ is not a rational number and there are no further simplifications of the answer (except possibly a numerical approximation with a calculator or equivalent). 
}%

\solution{\ref{problemlog_7(24)+log_(1/7)3-log_(49)64}

\[
\renewcommand{\arraystretch}{2}
\begin{array}{@{}r@{}c@{}l@{}l|l}
\displaystyle \log_{7}{(24)}+\log_{\frac{1}{7}}{(3)}-\log_{49}{(64)}&=&\displaystyle  \log_{7}{(24)}+ \frac{\log_{7}{(3)}}{ \log_{7}{\left(\frac{1}{7}\right)}}- \frac{\log_{7}{ (64)}}{\log_{7}{(49)}} && \text{common base}\\
&=&\displaystyle \log_{7}{(24)}+ \frac{\log_{7}{(3)}}{-1} -\frac{\log_{7}{(64)}}{2}&&\text{simplify logarithms}\\
&=&\displaystyle \log_{7}{(24)}-\log_{7}{(3)}-\frac{1}{2}\log_{7}{(64)}\\
&=&\displaystyle \log_{7}{\left(\frac{24}{3}\right)}- \log_{7}{ \left(64^{ \frac{ 1}{2}}\right) }&&\renewcommand{\arraystretch}{1.2} \begin{array}{@{}l}\text{rule: } \log_ax-\log_ay=\log_a\left( \frac{x}{y}\right)  \\ \text{rule: } \log_ax^r=r\log_ax \end{array} \\
&=&\displaystyle \log_{7}{\left(8\right)}-\log_{7}\left(\sqrt{64} \right)\\
&=&\displaystyle \log_78-\log_78 =0 &&\text{alternatively: }\\
&=&\displaystyle \log_7\left(\frac{8}{8}\right)\\
&=&\displaystyle \log_7(1)\\
&=&0.
\end{array}
\]
}
\begin{problem}
Find the exact value of each expression.
\begin{multicols}{2}
\begin{enumerate}[ref={\fcProblemRef}]
\item $\displaystyle \log_5 125$. 

\answer{$3$}
\item $\displaystyle \log_3 \frac{1}{27}$. 

\answer{$-3$}
\item $\displaystyle \ln \left(\frac{1}{e}\right) $. 

\answer{$-1$}
\item $\displaystyle \log_{10}\sqrt{10}$. 

\answer{$\displaystyle \frac12$}
\item $\displaystyle e^{\ln 4.5}$.  

\answer{$4.5$}
\item $\displaystyle \log_{10} 0.0001 $.  

\answer{$4$}
\item $\displaystyle \log_{1.5}2.25$.  

\answer{$2$}
\item $\displaystyle \log_5 4- \log_5 500$.  

\answer{$-3$}
\item $\displaystyle \log_2 6 - \log_2 15 +\log_2 20$. 

 \answer{$3$}
\item $\displaystyle \log_3 100- \log_3 18 - \log _3 50 $.  

\answer{$-2$}
\item $\displaystyle e^{-2\ln 5}$.  

\answer{$\frac{1}{25}$}
\item $\displaystyle \ln \left(\ln e^{e^{10}}\right)$. 

\answer{$10$}
\item  \label{problemSimplifylog_7(49^x/343^y)}  $\displaystyle \log_7\left(\frac{49^x}{343^y}\right)$

\answer{$2x-3y$}
%solution moved to separate file.

\end{enumerate}
\end{multicols}
\end{problem}
\solution{\ref{problemSimplifylog_7(49^x/343^y)}.

\[
\begin{array}{rcl}
\log_7\left(\frac{49^x}{343^y}\right) & =& \log_749^x - \log_7343^y \\
 & =& x\log_749 - y\log_7343 \\
\text{However }49 = 7^2\text{ and }343=7^3,\text{ therefore }
\log_7\left(\frac{49^x}{343^y}\right) & =& 2x-3y.
\end{array}
\]
}%

\subsection{Some Problems Involving Logarithms}
\begin{problem}
Solve each equation for $x$. If available, use a calculator to give an ($\approx$) answer in decimal notation. If available, use a calculator to verify your approximate solutions.
\begin{multicols}{2}
\begin{enumerate}[ref={\fcProblemRef}]
\item $e^{7-4x}=7$.

\answer{$\frac{7-\ln 7 }{4}\approx 1.263522 $}
\item $\ln (2x-9)=2$.

\answer{$\frac{e^2+9}{2}\approx 8.194528 $}
\item $\ln (x^2-2)=3$.

\answer{$\pm \sqrt{e^3+2}\approx \pm 4.699525 $}
\item  \label{problem2^(x-3)=5} $2^{x-3}=5$.

\answer{$\log_2 5+3= \frac{\ln 5}{\ln 2}+3 \approx 5.321928 $}
\item \label{problemlnx+ln(x-1)=1} $\ln x+\ln (x-1)=1$.

\answer{$\frac{1}{2}\left(1+\sqrt{1+4e}\right)\approx 2.223$}
\item $e^{2x+1}=t$.

\answer{$\frac{\ln t-1}{2}$}
\item $\log_2(m x)=c$.

\answer{$\frac{2^c}{m}$}
\item \label{probleme-e^(-2x)=1} $e- e^{-2x}=1$.

\answer{$-\frac12\ln (e-1)\approx -0.271$}
\item $8(1+e^{-x})^{-1}=3$.

\answer{$-\ln \frac53 =\ln \frac35 \approx -0.510826 $}
\item $\ln (\ln x)=1$.

\answer{$e^e\approx 15.154$}
\item $e^{e^x}=10$.

\answer{$\ln (\ln 10)\approx 0.834$}
\item $\ln(2x+1)=3-\ln x$.

\answer{$\frac{-1+\sqrt{1+8e^3}}{4}\approx 2.928878 $}
\item $e^{2x}-4e^x+3=0$.

\answer{$x=\ln 3\approx 1.098612, ~~~, x=0$}


\item $e^{4x}+3e^{2x}-4=0$. 

\answer{$x=0$}
\item $e^{2x}-e^x-6=0$.

\answer{$x=\ln 3$}
\item 
\label{problemSolve4^(3x)-2^(3x+2)-5}

$4^{3x}-2^{3x+2}-5=0$. 

\answer{$x=\frac{\log_{2}5}{3}$}


\item \label{problemSolve32^x+2(1/2)^(x-1)-7=0}
$3\cdot 2^{x}+2 \left(\frac{1}{2}\right)^{x-1}-7=0$. 

\answer{$x=0 \text{ or } 2-\log_2  3= 2-\frac{\ln 3}{\ln 2}$}



\end{enumerate}
\end{multicols}


\end{problem}

\solution{\ref{problem2^(x-3)=5}
\[\begin{array}{rcll|l}
\displaystyle 2^{x-3} &=& 5 &&\displaystyle  \text{take } \log_2 \\
x-3&=&\displaystyle  \log_2(5) &&\text{add } 3 \text{ to both sides}\\
x&=&\displaystyle \log_2(5)+3 &&\text{answer is complete} \\
&=&\displaystyle \frac{\ln 5}{\ln 2}+3 && \text{optional step: convert to }\ln\\
&\approx &5.321928095 &&\text{calculator}
\end{array}
\]
}

\solution{ \ref{probleme-e^(-2x)=1}

\[
\begin{array}{rcll|l}
\displaystyle e-e^{-2x}&=&1\\
\displaystyle e^{-2x}&=&e-1&& \text{apply }\ln\\
\displaystyle \ln e^{-2x}&=&\displaystyle \ln(e-1)\\
-2x&=&\displaystyle\ln(e-1)\\
x&=&\displaystyle-\frac{1}{2}\ln(e-1)\\
&\approx& -0.270662427&&\text{calculator}
\end{array}
\]

}

\solution{\ref{problemlnx+ln(x-1)=1} %
\[
\begin{array}{rcl}
\displaystyle \ln x + \ln (x-1) & =& 1 \\
\displaystyle \ln \left(x^2-x\right) & =& 1 \\
\displaystyle e^{\ln (x^2-x)} & =& e^1 \\
\displaystyle x^2-x & =& e \\
\displaystyle x^2-x-e & =& 0 \\
\text{Quadratic formula:}\quad x & = &\displaystyle \frac{-(-1)\pm \sqrt{(-1)^2-4(1)(-e)}}{2(1)} \\
& =&\displaystyle  \frac{1\pm \sqrt{1+4e}}{2}.
\end{array}
\]
However $\frac{1-\sqrt{1+4e}}{2}$ is negative, so $\ln\left( \frac{1-\sqrt{1 + 4e}}{2} \right)$ is undefined.  
Hence the only solution is $x = \frac{1+\sqrt{1+4e}}{2}\approx 2.2229$.  
}%

\solution{\ref{problemSolve4^(3x)-2^(3x+2)-5}

\[
\begin{array}{rcll|l}
\displaystyle 4^{3x}-2^{3x+2}-5&=&0 \\
\displaystyle 4^{3x}-4\cdot 2^{3x}-5&=&0&&\text{Set } \begin{array}{rcl}\displaystyle 2^{3x}&=&u\\ \displaystyle 4^{3x}&=&u^2\end{array} \\
\displaystyle u^2-4u-5&=&0\\
\displaystyle (u-5)(u+1)&=&0\\
\displaystyle u=5&\text{or}& u=-1\\
\displaystyle 2^{3x}=5&&\displaystyle 2^{3x}=-1\\
\displaystyle 3x=\log_2(5)&&\text{no real solution}\\
\displaystyle x=\frac{\log_2 5}{3}\\
\text{Calculator: }x\approx 0.773976
\end{array}
\]
}

\solution{\ref{problemSolve32^x+2(1/2)^(x-1)-7=0}
\[
\begin{array}{rcll|l}
\displaystyle 3\cdot 2^{x}+2 \displaystyle \left(\frac{1}{2}\right)^{x-1}-7&=&0\\
3\cdot 2^{x}+2\displaystyle  \left(\frac{1}{2} \right)^{x}\left( \frac{1}{2}\right)^{-1} -7&=&0\\
\displaystyle 3\cdot 2^{x}+4 \left(\frac{1}{2} \right)^{x} -7&=&0&&\text{Set } 2^x=u \\
\displaystyle 3u+\frac{4}{u}-7&=&0&&\text{Multiply by }u\\
\displaystyle 3u^2-7u+4&=&0\\
\displaystyle (u-1)(3u-4)&=&0\\
\displaystyle u=1&\text{or}&\displaystyle 3u-4=0\\
\displaystyle 2^x=1&&\displaystyle u=\frac{4}{3}\\
x=0&&\displaystyle 2^x=\frac{4}{3}\\
&&\displaystyle x=\log_2\frac{4}{3}=\log_2 4- \log_2 3\\
&&\displaystyle x=2-\log_2 3\\
\text{Calculator:}&&x\approx 0.415037
\end{array}
\]
}


\section{Derivatives}
\subsection{Derivatives and Function Graphs: basics}\label{secMPSderivativesFunGraphsBasics}
\begin{problem}
\label{problemMatchGraphToDerivativeGraphProblem2}
~\\
\psset{xunit=0.5cm, yunit=0.5cm}
\begin{tabular}{ccccc}
\multicolumn{5}{l}{Match each of the following function plots:}\\
$1.$&$2.$&$3.$&$4.$&$5.$\\
\begin{pspicture}(-3.1,-4.3)(3.1,4.1)
\fcAxesStandard{-3}{-4}{3}{4}
\fcGrid[linestyle=dashed, linewidth=0.5, linecolor=gray]{-3}{-4}{6}{8}{1}{1}{}
% 1/4x^4-2x^2
\psplot[linecolor=red]{-3}{3}{x x x x 0.25 mul mul mul mul -2 x x mul mul add}
\end{pspicture}
&
\begin{pspicture}(-3.1,-4.3)(3.1,4.1)
\fcAxesStandard{-3}{-4}{3}{4}
\fcGrid[linestyle=dashed, linewidth=0.5, linecolor=gray]{-3}{-4}{6}{8}{1}{1}{}
%sqrt(9-x^2)
\psplot[linecolor=red]{-3}{3}{x x -1 mul mul 9 add sqrt}
\end{pspicture}
&
\begin{pspicture}(-3.1,-4.3)(3.1,4.1)
\fcAxesStandard{-3}{-4}{3}{4}
\fcGrid[linestyle=dashed, linewidth=0.5, linecolor=gray]{-3}{-4}{6}{8}{1}{1}{}
%1/4 (\frac{1}{5} x^{5}-\frac{10}{3} x^{3}+9 x)
\psplot[linecolor=red]{-3}{3}{0.2 x x x x x mul mul mul mul mul -10 3 div x x x mul mul mul 9 x mul add add 0.25 mul}
\end{pspicture}
&
\begin{pspicture}(-3.1,-4.3)(3.1,4.1)
\fcAxesStandard{-3}{-4}{3}{4}
\fcGrid[linestyle=dashed, linewidth=0.5, linecolor=gray]{-3}{-4}{6}{8}{1}{1}{}
%cos(x*pi)
\psplot[linecolor=red]{-3}{3}{x 180 mul cos}
\end{pspicture}
&
\begin{pspicture}(-3.1,-4.3)(3.1,4.1)
\fcAxesStandard{-3}{-4}{3}{4}
\fcGrid[linestyle=dashed, linewidth=0.5, linecolor=gray]{-3}{-4}{6}{8}{1}{1}{}
%\int 1/10(x-1)(x-2)(x-3)(x+3) (x+2)(x+1)dx
\psplot[linecolor=red]{-3}{3}{x -3.6 mul x 3 exp 1.633333 mul add x 5 exp -0.28 mul add x 7 exp 0.014286 mul add}
\end{pspicture}
\\
\multicolumn{5}{l}{to their derivative plots:}\\
$(a)$&$(b)$&$(c)$&$(d)$&$(e)$\\
\begin{pspicture}(-3.1,-4.3)(3.1,4.1)
\fcAxesStandard{-3}{-4}{3}{4}
\fcGrid[linestyle=dashed, linewidth=0.5, linecolor=gray]{-3}{-4}{6}{8}{1}{1}{}
% -x/sqrt(9-x^2)
%matches sqrt(9-x^2)
\psplot[linecolor=blue]{-2.9}{2.9}{-1 x mul 9 x x -1 mul mul add sqrt div}
\end{pspicture}
&
\begin{pspicture}(-3.1,-4.3)(3.1,4.1)
\fcAxesStandard{-3}{-4}{3}{4}
\fcGrid[linestyle=dashed, linewidth=0.5, linecolor=gray]{-3}{-4}{6}{8}{1}{1}{}
% 3pi/4* (-sin)(x*pi)
%matches cos(x*pi)
\psplot[linecolor=blue, plotpoints=400]{-3}{3}{3.141592654 x 180 mul sin mul -1 mul}
\end{pspicture}
&
\begin{pspicture}(-3.1,-4.3)(3.1,4.1)
\fcAxesStandard{-3}{-4}{3}{4}
\fcGrid[linestyle=dashed, linewidth=0.5, linecolor=gray]{-3}{-4}{6}{8}{1}{1}{}
% (x+2)(x-2)x 
%matches 1/4x^4-2x^2
\psplot[linecolor=blue]{-2.38}{2.38}{x 2 add x -2 add x mul mul}
\end{pspicture}
&
\begin{pspicture}(-3.1,-4.3)(3.1,4.1)
\fcAxesStandard{-3}{-4}{3}{4}
\fcGrid[linestyle=dashed, linewidth=0.5, linecolor=gray]{-3}{-4}{6}{8}{1}{1}{}
%1/10(x-1)(x-2)(x-3)(x+3) (x+2)(x+1)
%matches \int 1/10(x-1)(x-2)(x-3)(x+3) (x+2)(x+1)dx
\psplot[linecolor=blue, plotpoints=400]{-3}{3}{1 10 div x x x mul mul dup mul mul -7 5 div x x mul dup mul mul 49 10 div x x mul mul -18 5 div add add add}
\end{pspicture}
&
\begin{pspicture}(-3.1,-4.3)(3.1,4.1)
\fcAxesStandard{-3}{-4}{3}{4}
\fcGrid[linestyle=dashed, linewidth=0.5, linecolor=gray]{-3}{-4}{6}{8}{1}{1}{}
% 1/4(x-1)(x-3)(x+3) (x+1)
%matches 1/4 (\frac{1}{5} x^{5}-\frac{10}{3} x^{3}+9 x)
\psplot[linecolor=blue, plotpoints=400]{-3}{3}{x 1 sub x 3 sub x 1 add x 3 add 0.25 mul mul mul mul}
\end{pspicture}
\end{tabular}
\end{problem}
\begin{problem}
\solution{\ref{problemMatchGraphToDerivativeGraphProblem2}

(1) matches (c) because (1) has three local extrema and (c) is the only derivative graph with three zeros. 

(2) matches (a) because (2) has one local extrema and (a) is the only derivative graph with one zero. 

(3) matches (e) because (3) has four local extrema ($\pm1$ and $\pm3$) and (e) is the only derivative graph with four zeros. 

(4) matches (b) because (4) has seven local extrema and (b) is the only derivative graph with seven zeros. 

(5) matches (d) because (5) has six local extrema and (d) is the only derivative graph with six zeros. 
}
\end{problem}
\begin{problem}
~\\
\psset{xunit=0.5cm, yunit=0.5cm}
\begin{tabular}{cccc}
\multicolumn{4}{l}{Match each of the following function plots:}\\
$1.$&$2.$&$3.$&$4.$\\
\begin{pspicture}(-3.1,-4.3)(3.1,4.1)
\fcAxesStandard{-3}{-4}{3}{4}
\fcGrid[linestyle=dashed, linewidth=0.5, linecolor=gray]{-3}{-4}{6}{8}{1}{1}{}
%Function formula: -8 ((x) ((x) (x)))+2 (x)
\psplot[linecolor=red, plotpoints=1000]{-0.9}{0.9}{x 2 mul x x mul x mul -8 mul add }
\end{pspicture}
&
\begin{pspicture}(-3.1,-4.3)(3.1,4.1)
\fcAxesStandard{-3}{-4}{3}{4}
\fcGrid[linestyle=dashed, linewidth=0.5, linecolor=gray]{-3}{-4}{6}{8}{1}{1}{}
%Function formula: -2+x
\psplot[linecolor=red, plotpoints=1000]{1}{3}{x -2 add } %Function formula: - (x)
\psplot[linecolor=red, plotpoints=1000]{-1}{1}{x -1 mul } %Function formula: 2+x
\psplot[linecolor=red, plotpoints=1000]{-3}{-1}{x 2 add }
\end{pspicture}
&
\begin{pspicture}(-3.1,-4.3)(3.1,4.1)
\fcAxesStandard{-3}{-4}{3}{4}
\fcGrid[linestyle=dashed, linewidth=0.5, linecolor=gray]{-3}{-4}{6}{8}{1}{1}{}
%Function formula: - ((1)/((x)^{2}+1))
\psplot[linecolor=red, plotpoints=1000]{-3}{3}{1 1 x 2 exp add div -1 mul }
\end{pspicture}
&
\begin{pspicture}(-3.1,-4.3)(3.1,4.1)
\fcAxesStandard{-3}{-4}{3}{4}
\fcGrid[linestyle=dashed, linewidth=0.5, linecolor=gray]{-3}{-4}{6}{8}{1}{1}{}
%Function formula: - (((x)^{2}) ((x) (x)))+(x)^{2}
\psplot[linecolor=red, plotpoints=1000]{-1.59}{1.59}{x 2 exp x x mul x 2 exp mul -1 mul add }
\end{pspicture}
\\
\multicolumn{4}{l}{to their derivative plots:}\\
$(a)$&$(b)$&$(c)$&$(d)$\\
\begin{pspicture}(-3.1,-4.3)(3.1,4.1)
\fcAxesStandard{-3}{-4}{3}{4}
\fcGrid[linestyle=dashed, linewidth=0.5, linecolor=gray]{-3}{-4}{6}{8}{1}{1}{}
%Function formula: (x)/(((x)^{2}+1)^{2})
\psplot[linecolor=blue, plotpoints=1000]{-3}{3}{x 1 x 2 exp add 2 exp div }
\end{pspicture}
&
\begin{pspicture}(-3.1,-4.3)(3.1,4.1)
\fcAxesStandard{-3}{-4}{3}{4}
\fcGrid[linestyle=dashed, linewidth=0.5, linecolor=gray]{-3}{-4}{6}{8}{1}{1}{}
%Function formula: -24 ((x) (x))+2
\psplot[linecolor=blue, plotpoints=1000]{-0.485}{0.485}{2 x x mul -24 mul add }
\end{pspicture}
&
\begin{pspicture}(-3.1,-4.3)(3.1,4.1)
\fcAxesStandard{-3}{-4}{3}{4}
\fcGrid[linestyle=dashed, linewidth=0.5, linecolor=gray]{-3}{-4}{6}{8}{1}{1}{}
%Function formula: -4 ((x)^{3})+2 (x)
\psplot[linecolor=blue, plotpoints=1000]{-1.15}{1.15}{x 2 mul x 3 exp -4 mul add }
\end{pspicture}
&
\begin{pspicture}(-3.1,-4.3)(3.1,4.1)
\fcAxesStandard{-3}{-4}{3}{4}
\fcGrid[linestyle=dashed, linewidth=0.5, linecolor=gray]{-3}{-4}{6}{8}{1}{1}{}
%Function formula: 1
\psplot[linecolor=blue, plotpoints=1000]{1}{3}{1}
\fcHollowDotBlue{-1}{1}
%Function formula: -1
\fcHollowDotBlue{-1}{-1}
\psplot[linecolor=blue, plotpoints=1000]{-1}{1}{-1}
\fcHollowDotBlue{1}{-1}
%Function formula: 1
\fcHollowDotBlue{1}{1}
\psplot[linecolor=blue, plotpoints=1000]{-3}{-1}{1}
\end{pspicture}
\end{tabular}

Give reasons for your choices. Can you guess formulas that would give a similar (or precisely the same) graph, and confirm visually your guess using a graphing device?

\end{problem}
\subsection{Product and Quotient Rules}\label{secMPSproductQuotientRules}

\begin{problem}
(Textbook, page 136, 1-44).
Compute the derivative.
\begin{multicols}{2}
\begin{enumerate}
\item $f(x)=2^{40}$.

\answer{$0$}
\item $f(x)=\pi^2$.

\answer{$0$}
\item $f(t)=2-\frac{2}{3}t$.

\answer{$-\frac{2}{3}$}
\item $F(x)=\frac{3}{4}x^8$.

\answer{$6 x^{7}$}
\item $f(x)=x^3-4x+6$.

\answer{$-4+3 x^{2} $}
\item $f(t)=\frac{1}{2}t^6-3t^4+t$.

\answer{$ 3 t^{5}-12 t^{3}+1$}
\item $g(x)=x^2(1-2x)$. 

\solution{
There are two approaches: 
1. Uncover the parenthesis, and then differentiate:

$
\left(x^2(1-2x)\right)'= \left(x^2-2x^3\right)'=2x-6x^2
$

2. Use first the product rule and then simplify:
$
\begin{array}{rcl}
\left(x^2(1-2x)\right)'&=& (x^2)'(1-2x)+x^2(1-2x)'\\
&=&2x(1-2x)+x^2(-2)\\
&=& 2x-4x^2-2x^2\\
&=& 2x-6x^2.
\end{array}
$

Of course, both approaches lead to the same answer.
}


\answer{$ 2 x-6 x^{2}$}
\item $h(x)=(x-2)(2x+3)$.

\answer{$ 4x-1$}
\item $g(t)=2t^{-3/4}$.

\answer{$-\frac{3}{2} t^{-\frac{7}{4}} $}
\item $B(y)=c y^{-6}$.

\answer{$-6 c y^{-7} $}
\item $A(s)=-\frac{12}{s^5}$.
\answer{$60 s^{-6}$}

\end{enumerate}
\end{multicols}

\end{problem}
\solution{\ref{problemDifferentiatexsquaredtimes1minus2x}
Approach 1. Uncover the parenthesis, and then differentiate:

$
\left(x^2(1-2x)\right)'= \left(x^2-2x^3\right)'=2x-6x^2
$

Approach 2. Use first the product rule and then simplify:
$
\begin{array}{rcl}
\left(x^2(1-2x)\right)'&=& (x^2)'(1-2x)+x^2(1-2x)'\\
&=&2x(1-2x)+x^2(-2)\\
&=& 2x-4x^2-2x^2\\
&=& 2x-6x^2.
\end{array}
$

Of course, both approaches lead to the same answer.
}

\begin{problem}
Compute the derivative.
\begin{multicols}{2}
\begin{enumerate}[ref={\fcProblemRef}]
\item $\displaystyle y=x^{\frac53}-x^{\frac23}$.

\answer{$ \frac53 x^{\frac23}-2/3 x^{-\frac13}$}
\item $\displaystyle f(x)=\sqrt{x}-x$.

\answer{$-1+\frac{1}{2} x^{-\frac{1}{2}} $}
\item $\displaystyle y=\sqrt{x}(x-1)$.

\answer{$ \frac{3}{2} x^{\frac{1}{2}}- \frac{1}{2}  x^{-\frac{1}{2}}$}
\item $\displaystyle f(x)=(2x+1)^2$.

\answer{$4+8 x $}
\item $\displaystyle f(x)=4\pi x^2$.

\answer{$8 \pi x$}
\item  $\displaystyle y=\frac{ x^2+4x+3}{\sqrt{x}}$.

\answer{$ 2 x^{-\frac{1}{2}}+\frac{3}{2} x^{\frac{1}{2}}-\frac{3}{2} x^{-\frac{3}{2}}$}

\item $\displaystyle y=\frac{\sqrt{x}+x}{x^2}$.

\answer{$- x^{-2}-\frac{3}{2} x^{-\frac{5}{2}} $}

\item $\displaystyle f(x)=\left(x+x^{-1}\right)^3$.

\answer{$3x^{2}+3-3x^{-2}-3x^{-4} $}
\item $\displaystyle f(x)=\sqrt 2 x +\sqrt{5x}$.

\answer{$ \sqrt{2}+\frac{\sqrt5}{2}  x^{-\frac{1}{2}}=\sqrt{2}+\frac{\sqrt5}{2\sqrt{x}}$}
\item $\displaystyle y=\sqrt[5]x+4\sqrt{x^5}$.
\answer{$10 x^{\frac{3}{2}}+\frac{1}{5} x^{-\frac{4 }{5}} $}
\item  \label{problemd/dx((sqrt(x)+1/sqrt[3](x))^2)} $\displaystyle y=\left(\sqrt{x}+ \frac{1}{ \sqrt[3]{x}}\right)^2$.


\answer{$1+\frac{1}{3} x^{-\frac{5}{6}}-\frac{2}{3} x^{-\frac{5}{3}} $}


\item $\displaystyle f(x)=(1+2x^2)(x-x^2)$.

\answer{$1-2 x+6 x^{2}-8 x^{3}$}
\item $\displaystyle f(x)=\frac{x^4-5x^3+ \sqrt{x}}{x^2}$.

\answer{$-5+2 x-\frac{3}{2} x^{-\frac{5}{2}} $}
\item $\displaystyle f(x)=(2x^3+3)(x^4-2x)$.

\answer{$-6-4 x^{3}+14 x^{6}$}
\item $\displaystyle f(x)=(1+x+x^2)(2-x^4)$.

\answer{$ 2+4 x-4 x^{3}-5 x^{4}-6 x^{5}$}
\item $\displaystyle g(y)= \left(\frac{1}{y^2}- \frac{3}{y^4} \right)(y+5y^3)$.

\answer{$5+9 y^{-4}+14 y^{-2} $}
\item $\displaystyle f(x)=(x^3-2x)(x^{-4}+x^{-2})$.

\answer{$1+6 x^{-4}+x^{-2}$}
\item $\displaystyle f(x)=\frac{1+2x}{3-4x}$.

\answer{$ 10 (3-4 x)^{-2}$}
\end{enumerate}
\end{multicols}
\end{problem}
\solution{\ref{problemd/dx((sqrt(x)+1/sqrt[3](x))^2)}
\[
\begin{array}{rcl}
\displaystyle \left(\left(\sqrt{x}+\frac{1}{\sqrt[3]{x}}\right)^2 \right)' &=&\displaystyle \left(\left(x^{\frac{1}{2}}+x^{- \frac{1}{3}}\right)^2 \right)'\\
&=&\displaystyle \left(\left(x^{\frac{1}{2}}\right)^2 +2 x^{\frac{1}{2}}x^{-\frac{1}{3}} + \left(x^{-\frac{1}{3}}\right)^2  \right)'\\
&=&\left(x +2 x^{\frac{1}{6}} + x^{-\frac{2}{3}}\right)'\\
&=&\displaystyle 1+2\cdot \frac{1}{6} x^{\frac{1}{6}-1} + \left(-\frac{2}{3} \right)x^{-\frac{2}{3}-1}\\
&=&\displaystyle 1+\frac{1}{3} x^{-\frac{5}{6}} -\frac{2}{3}x^{-\frac{5}{3}}.
\end{array}
\]
}

\begin{problem}
Compute the derivative (with respect to the implied variable).
\begin{multicols}{2}
\begin{enumerate}[ref={\fcProblemRef}]
\item $\displaystyle f(x)=\frac{x-3}{x+3}$.

\answer{$6 (3+x)^{-2} $}

\item $\displaystyle y=\frac{x^3}{1-x^2}$.

\answer{$ \frac{3 x^{2}- x^{4}}{(1- x^{2})^{2}}$}
\item $\displaystyle y=\frac{x+1}{x^3+x-2}$.

\answer{$\frac{-3-3 x^{2}-2 x^{3}}{(-2+x+x^{3})^{2}} $}
\item $\displaystyle y=\frac{x-1}{x^3+x-2}$.

\answer{$  \frac{-2 x-1}{\left(x^{2}+x+2\right)^{2}} $}
\item \label{problemd/dx((x+1)/(x^3+1))}

$\displaystyle f(x)= \frac{x+1}{x^3+1}$.

\answer{$\frac{-2x+1}{(x^2-x+2)^2}$}
\item $\displaystyle y=\frac{x^3-2x\sqrt{x}}{x}$.

\answer{$2 x- x^{-\frac{1}{2}}$}
\item \label{problemd/dt(t/(t-1)^2)} $\displaystyle y=\frac{t}{(t-1)^2}$.

\answer{$-\frac{t +1}{(t-1)^3} $}
\item $\displaystyle y=\frac{t^2+2}{t^4-3t^2+1}$.

\answer{$\frac{14 t-8 t^{3}-2 t^{5}}{(1-3 t^{2}+t^{4})^{2}} $}
\item $\displaystyle g(t)=\frac{t-\sqrt{t}}{t^{\frac{1}{3}}}$.

\answer{$-\frac{1}{6} t^{-\frac{5}{6}}+\frac{2}{3} t^{-\frac{1}{3}} $}
\item $\displaystyle y=a x^2+b x + c$.

\answer{$ b+2 a x$}
\item $\displaystyle y=A+\frac{B}x +\frac{C}{x^2}$.

\answer{$\frac{- Bx-2 C }{x^{3}}$}
\item $\displaystyle f(t)=\frac{2t}{2+\sqrt{t}}$.

\answer{$\frac{4+t^{\frac{1}{2}}}{\left( 2+t^{\frac{1 }{ 2}} \right)^{2}} $}
\item $\displaystyle y=\frac{c x}{1+c x}$.

\answer{$ c (1+c x)^{-2}$}
\item $\displaystyle y=\sqrt[3]{t}(t^2+t+t^{-1}) $.

\answer{$-\frac{2}{3} t^{-\frac{5}{3}}+\frac{4}{3} t^{\frac{1}{3}}+\frac{7}{3} t^{\frac{4}{3}} $}
\item $\displaystyle y=\frac{u^6-2u^3+5}{u^2}$.

\answer{$-2-10 u^{-3}+4 u^{3} $}
\item $\displaystyle f(x)=\frac{a x+b}{c x+ d}$.

\answer{$\frac{a d- b c}{(d+c x)^{2}}$}
\item $\displaystyle f(x)=\frac{1+x }{1+\frac{2}x}$. 

\answer{$\frac{x^{2}+4 x+2}{(2+x)^{2}}$}
\item $\displaystyle f(x)=\frac{1+x }{1+\frac{3}x}$. 

\answer{$\frac{x^{2}+6 x+3}{(3+x)^{2}}$}
\item $\displaystyle f(x)=\frac{x}{x+\frac{c}{x}}$.

\answer{$ \frac{2 x c}{(c+x^{2})^{2}}$}
\end{enumerate}
\end{multicols}

\end{problem}
\solution{\ref{problemd/dt(t/(t-1)^2)}
This can be differentiated more efficiently using the chain rule, however let us show how the problem can be solved directly using the quotient rule.
\[
\begin{array}{rcl}
\displaystyle  \left(\frac{t}{(t-1)^2}\right)'&=&\displaystyle \frac{(t)' (t-1)^2-t\left((t-1)^2\right)' }{(t-1)^4}\\
&=&\displaystyle \frac{(t-1)^2 - t \left(t^2-2t+1\right)' }{(t-1)^4}\\
&=&\displaystyle \frac{(t-1)^2 - t \left(2t-2\right) }{(t-1)^4}\\
&=&\displaystyle \frac{\cancel{(t-1)} \left((t-1) - 2t \right)}{(t-1)^{\cancel{4} ~3}}\\
&=&\displaystyle \frac{-t -1}{(t-1)^3}\\
&=&\displaystyle-\frac{t+1}{(t-1)^3}
\end{array}
\]


}


\solution{\ref{problemd/dx((x+1)/(x^3+1))}

\[\begin{array}{rcl}
\displaystyle \frac{\diff }{\diff x}\left(\frac{x+1}{x^3+1}\right)&=&\displaystyle \frac{\diff }{\diff x}\left(\frac{\cancel{x+1}}{\cancel{(x+1)}(x^2-x+1)}\right)\\
&=&\displaystyle \frac{\diff }{\diff x}\left(\frac{1}{x^2-x+1}\right)\\
\multicolumn{3}{l}{\textbf{Variant I: use quotient rule.}}\\
&=&\displaystyle \frac{ \frac{\diff }{\diff x}(1)\cdot (x^2-x+1)-1\cdot\frac{\diff }{\diff x}\left(x^2-x+1\right)}{\left(x^2-x+1\right)^2}\\
&=&\displaystyle \frac{-2x+1}{\left(x^2-x+1\right)^2}\\
\multicolumn{3}{l}{\textbf{Variant I: use chain rule.}}\\
&=&\displaystyle \frac{\diff }{\diff x}\left(\left(x^2-x+1\right)^{-1}\right)\\
&=&\displaystyle -(x^2-x+1)^{-2}\frac{\diff}{\diff x}(x^2-x+1)\\
&=&\displaystyle -(x^2-x+1)^{-2}(2x-1)\\
&=&\displaystyle \frac{-2x+1}{\left(x^2-x+1\right)^2}.
\end{array}
\]
}


\subsection{Basic Trigonometric Derivatives}\label{secMPStrigDerivatives}
\begin{problem}

Compute the derivative.
\begin{multicols}{2}
\begin{enumerate}
\item $\displaystyle f(x)= 2x^3 -3 \cos x$.

\answer{$ 6 x^2 +3 \sin x$}
\item $\displaystyle f(x)=\sqrt{x}\cos x$.

\answer{$ -x^{\frac{1}{2}}\sin x +\frac{1}{2}x^{-\frac{1}{2}} \cos x $}


\item $\displaystyle f(x)=\sin x +\frac{1}{3}\cot x$.

\answer{$\frac{-\frac{1}{3}+\cos x \sin^2x}{\sin^2x} = \cos x- \frac{1}{3} \csc^2 x $}
\item $\displaystyle y=2\sec x - \csc x$.

\answer{$ \frac{\cos^3 x+2 \sin^3x}{(\cos x \sin x)^{2}}$}
\item $\displaystyle y=\frac{1+\sin^2\theta}{\cos^3\theta}$.

\answer{$ y'=\displaystyle \frac{2 \sin{}\theta \cos^{2}{}\theta+3 \sin^{3}{}\theta+3 \sin{}\theta}{\cos^{4}{}\theta} $}
\item $\displaystyle g(t)=4 \sec t + \tan t-\csc t +3\cot t  $.

\answer{$4\sec t \tan t +\sec^2t +\csc t \cot t-3\csc^2 t $}

\item $\displaystyle y= c\cos t + t^2\sin t$.

\answer{$ - c \sin t+2 t \sin t+ t^{2}\cos t$}
\item $\displaystyle y=u(a\cos u + b \cot u)$.

\answer{$ \frac{- a u \sin^3 u+a \cos u \sin^2u- b u +b \cos u \sin u}{\sin^2u}$}
\item $\displaystyle y=\frac{x}{2-\tan x}$.

\answer{$ \frac{x - \cos x \sin x+2 \cos^2 x}{(2 \cos x- \sin x)^{2}}$}
\item $\displaystyle y=\sin \theta \cos \theta$.

\answer{$\cos (2\theta)= \cos^2\theta- \sin^2\theta$}
\item $\displaystyle f(\theta)=\frac{\sec \theta}{1+\sec \theta}$.

\answer{$\frac{\sin\theta}{(1+\cos\theta)^{2}} $}
\item $\displaystyle y=\frac{\cos x}{1-\sin x}$.

\answer{$\frac{1}{1- \sin x} $}
\item $\displaystyle y=\frac{t\sin t}{1+t}$.

\answer{$ \frac{\sin t+t \cos t+t^{2}\cos t }{(1+t)^{2}}$}
\item 
$\displaystyle y=\frac{1-\sec x}{\tan x}$.

\answer{$\frac{\cos x- 1}{\sin^2x} $}

\item $\displaystyle h(\theta)=\theta \csc \theta -\cot \theta$.

\answer{$\frac{1+\sin\theta- \theta \cos\theta}{\sin^2\theta}$}
\item $\displaystyle y=x^2\sin x\tan x$.

\answer{$\frac{2 x \cos{}x \sin^2{}x+2 x^{2} \sin{}x  \cos^2x+x^{2} \sin^3{}(x)}{\cos^2{}x} $}
\end{enumerate}
\end{multicols}

\end{problem}
\begin{problem}
Differentiate.

\begin{multicols}{2}
\begin{enumerate}
\item $\tan x$.

\answer{$\sec^2 x$}
\item $\cot x$.

\answer{$-\csc^2 x$}
\item $\sec x$.

\answer{$\sec x \tan x= \frac{\sin x}{\cos^2 x}$}
\item $\csc x$.

\answer{$-\csc x \cot x= -\frac{\cos x }{\sin^2x} $}
\item $\sec x\tan x$.

\answer{$\sec x \tan^2 x+\sec^3 x$}
\item $\sec x+\tan x$.

\answer{$\sec x(\tan x +\sec x) $}
\item $\sec^2 x$.

\answer{$2\tan x\sec^2 x$}
\item $\csc^2 x$.

\answer{$ -2\cot x\csc^2 x$}

\item \label{problemd/dx((secx)e^x)}
$\displaystyle f(x)=(\sec x )e^{x}$.

\answer{$\sec x\tan x e^x + \sec x e^x=(\tan x +1)(\sec x) e^x$}
\item \label{problemd/dx((tanx)e^x)}
$\displaystyle f(x)=(\tan x )e^{x}$.

\answer{$\sec^2x e^x + \tan x e^x$}

\item $\displaystyle \frac{\sin x}{x}$.

\answer{$\frac{x \cos{}x- \sin{}x}{x^{2}}$}

\item $\displaystyle \frac{\sin x}{e^x}$.

\answer{$\frac{\cos x -\sin x}{e^x}$}
\item $\displaystyle x(\cos x) e^x$.

\answer{$  \begin{array}{l}e^x( x \cos{}x-x  \sin{}x+ \cos{}x) \end{array}$}

\item $\displaystyle \frac{e^x}{\tan x}$.

\answer{$  e^x\left(\cot x-\csc^2 x\right)$}
\item $\displaystyle \frac{e^x}{\sec x} +\sec x$.

\answer{$e^x(\cos x-\sin x)+ \sec x \tan x $}

\end{enumerate}

\end{multicols}
\end{problem}
\subsection{Natural Exponent Derivatives}
% begin homework exponent-derivative
Differentiate each function.  

\begin{enumerate}
\item   $\displaystyle f(x) = \frac{e^x}{1+2e^x}$.  

\pointsii{3}  $r(t) = Ae^{-kt^2}$, where $A$ and $k$ are unknown constants.  

\solution{%
\begin{align*}
r & = Ae^{-kt^2}. \\
\text{Let}\quad u & = -kt^2. \\
\text{Then}\quad r & = Ae^u. \\
\text{Chain Rule}\quad \frac{\diff r}{\diff t} & = \frac{\diff r}{\diff u}\frac{\diff u}{\diff t} \\
 & = (Ae^u)(-2kt) \\
 & = -2Akte^{-kt^2}.
\end{align*}
}%

\item   $y = \frac{e^x}{2}(\sin x + \cos x)$.  
\end{enumerate}
% end homework exponent-derivative


\subsection{The Chain Rule}\label{secMPSchainRule}

\begin{problem}
% begin homework chain-rule1
Compute the derivative using the chain rule.
\begin{multicols}{2}
\begin{enumerate}[ref={\fcProblemRef}]
\item $\displaystyle f(x)=\sqrt{1+x^2}$

\answer{$x (x^{2}+1)^{-\frac{1}{2}}  $}
\item \label{problemd/dx(sqrt(3x^2-x+2))}
$\displaystyle f(x)=\sqrt{3 {{x}}^{2}-{{x}}+2}$.

\answer{$\frac{6x-1}{2\sqrt{3x^2-x+2}}$}


\item \label{problemDifferentialtexDivsqrt(1+2divx^2)}  $\displaystyle  f(x)=\frac{x }{\sqrt{1+\frac{2}{x^2}}}$.

\answer{$\frac{\pm x^2}{\sqrt{x^2+2}} $}

\item \label{problemd/dxsqrt(1-sqrt(x))}

$f(x)=\sqrt{1-\sqrt{x}}$.

\answer{$-\frac{1}{4} x^{-\frac{1}{2}} \left(- \sqrt{x}+1\right)^{-\frac{1}{2}} $}

\pointsii{3} \label{problemd/dx((cosx)^(1/2))} $y = (\cos x)^{\frac{1}{2}}$

\answer{$\frac{1}{2} x^{-\frac{1}{2}} \cos{}\left(\sqrt{x}\right)  = \frac{\cos \left(\sqrt{x}\right)}{2\sqrt{x}}$}

\item $\displaystyle f(x)=\sin^3 x$.

\answer{$ 3 \cos{}x \sin^{2}{}x$}
\pointsii{3} \label{problemd/dx((1+cosx)^2)}  $y = (1+\cos x)^2$.


\answer{$ -2 \cos{}x \sin{}x-2 \sin{}x =-\sin(2x)-2\sin x$}
\item   $\displaystyle f(x)=\frac{1}{\sin^3x}$.

\answer{$  -\frac{3 \cos{}x}{\sin^{4}{}x} $}
\item  $\displaystyle f(x)= \sqrt[3]{4+3\tan x}$.

\answer{$  (4+3\tan x)^{-\frac{2}{3}}\sec^2x $}
\item  $f(x)=(\cos x + 3\sin x)^4$.

\answer{$4(\cos x + 3\sin x)^3 (3\cos x-\sin x) $}

\pointsii{4} \label{problemd/dx(sin(sqrt(x)))}  $\displaystyle y = \sin \left( \sqrt{x}\right)$

\answer{$\frac{\cos\sqrt{x}}{2\sqrt{x}}$}
\item  $y = \cos\left( 4x\right)$

\answer{$-4 \sin{}\left(4 x\right)  $}

\item $\sec^2 (3x^2)$. 

\answer{$12 \frac{x\sin{}(3 x^{2}) }{\left(\cos{}\left(3 x^{2}\right)\right)^{3}}$}

\item $\csc^2 (3x^2)$. 

\answer{$-12 \frac{ x  \cos{}\left(3 x^{2}\right) }{\left(\sin{}\left(3 x^{2}\right)\right)^{3}}$}
\item $e^{2x}$.

\answer{$2e^{2x}$}

\item $e^{-x^2}$

\answer{$-2xe^{-x^2}$}

\item $e^{\sqrt{x}}$

\answer{$\frac{1}{2\sqrt{x}}e^{\sqrt{x}}$}

\item \label{problemd/dx(e^(-1/x))}

$\displaystyle f(x)=e^{-\frac{1}{x}}$.

\answer{$ \frac{e^{-\frac{1}{x}}}{x^2}$}

\item $5^{x}$.

\answer{$(\ln 5)5^x $}
\item $e^{2^x}$.

\answer{$e^{2^x}2^x (\ln 2) $}

\item $2^{3^x}$.

\answer{$2^{3^{x}} 3^{x} (\ln{}2)  (\ln{}3) $}
\item $3^{2^x}$.

\answer{$ 3^{2^{x}} 2^{x}(\ln{}2)(\ln{}3)$}
\pointsii{5} \label{problemd/dx(sqrt(sec(4x)))}  $y = \sqrt{\sec (4x)}$

\answer{ $  2\sqrt{\sec (4x)}\tan (4x) =  2(\sec (4x))^{\frac{3}{2}}\sin (4x). $}
\item \label{problemd/dx(x^2tan(5x))} $y = x^2\tan (5x)$

\answer{$2x\tan (5x) - 5x^2\sec^2 (5x)$}
\pointsii{5} \label{problemd/dx((1+sin(x^2))/(1+cos(x^2)))}  $\displaystyle y = \frac{1+\sin \left(x^2\right)}{1+\cos \left(x^2\right)}$.

\answer{ $ \frac{2x\left(1 + \cos \left(x^2\right) + \sin \left(x^2\right)\right)}{\left(1+\cos \left(x^2\right)\right)^2}$}
\end{enumerate}
\end{multicols}
% end homework chain-rule1

\end{problem}
\solution{\ref{problemd/dx(sqrt(3x^2-x+2))}

$\begin{array}{rcl}
\displaystyle \frac{\diff}{\diff x}\left(\sqrt{3 x^{2}-x+2}\right)&=&\displaystyle \frac{(3 x^{2}-x+2)'}{2\sqrt{3 x^{2}-x+2}}=\frac{6x-1}{2\sqrt{3 x^{2}-x+2}}.
\end{array}
$
}
\solution{\ref{problemDifferentialtexDivsqrt(1+2divx^2)}
\[
\begin{array}{rclr|r}
\displaystyle\left(\frac{x }{\sqrt{1+\frac{2}{x^2}}}\right)'&=&\displaystyle\frac{\sqrt{1+\frac{2}{x^2}}- x\left(\sqrt{1+\frac{2}{x^2}}\right)'}{1+\frac{2}{x^2}} =\frac{\sqrt{1+\frac{2}{x^2}}-  x\frac{\frac12}{\sqrt{1 +\frac{ 2}{ x^2 }}}  \left(\frac{2}{x^2}\right)'}{1+\frac{2}{x^2}}\\
&=& \displaystyle\frac{\sqrt{1+\frac{2}{x^2}}+  \frac{2}{x^2\sqrt{1 +\frac{ 2}{ x^2 }}} }{1+\frac{2}{x^2}} = \frac{x^2\left(1+\frac{2}{x^2}\right)+  2 }{x^2\left(1+\frac{2}{x^2}\right)^{\frac32}}= \frac{x^2+4}{x^2\left(1+\frac{2}{x^2}\right)^{\frac32}}
\end{array}
\]
Please note that this problem can be solved also by applying the transformation 
\[
\displaystyle  \frac{x}{\sqrt{1+\frac{2}{x^2}}}= \frac{x}{\sqrt{\frac{x^2+2}{x^2}}} =\frac{x}{\frac{1}{\pm x}\sqrt{x^2+2}} = \frac{\pm x^2}{\sqrt{x^2+2}}
\]
before differentiating, however one must not forget the $\pm $ sign arising from $\sqrt{x^2}=\pm x$. Our original approach resulted in more algebra, but did not have the disadvantage of dealing with the $\pm$ sign.
}

\solution{\ref{problemd/dxsqrt(1-sqrt(x))}

\[
\begin{array}{rcll|l}
\displaystyle \frac{\diff }{\diff x}\left(\sqrt{1-\sqrt{x}}\right)&= &\displaystyle \frac{\diff }{\diff x}\left(\left(1-x^{\frac{1}{2}} \right)^{ \frac{ 1}{2}}\right)&&\text{chain rule}\\
&=&\displaystyle \frac{1}{2}\left(1-x^{\frac{1}{ 2}}\right)^{- \frac{1}{ 2}}\frac{\diff }{\diff x}\left(1-x^{\frac{1}{2}}\right)\\
&=&\displaystyle -\frac{1}{4}x^{-\frac{1}{2}} \left(1-x^{\frac{1}{ 2}}\right)^{- \frac{1}{ 2}}
\end{array}
\]
}

\solution{ \ref{problemd/dx((cosx)^(1/2))}%
\begin{align*}
\text{Let } \quad u & = \cos x. \\
\text{Then } \quad y & = u^{\frac{1}{2}}. \\
\text{Chain Rule: } \quad \frac{\diff y}{\diff x} & = \frac{\diff y}{\diff u}\frac{\diff u}{\diff x} \\
 & = \left(\frac{1}{2}u^{-\frac{1}{2}}\right) (-\sin x) \\
 & = -\frac{1}{2} \sin x (\cos x)^{-\frac{1}{2}}.
\end{align*}
}%

\solution{\ref{problemd/dx((1+cosx)^2)} %
\begin{align*}
\text{Let } \quad u & = 1+\cos x. \\
\text{Then } \quad y & = u^{2}. \\
\text{Chain Rule: } \quad \frac{\diff y}{\diff x} & = \frac{\diff y}{\diff u}\frac{\diff u}{\diff x} \\
 & = (2u) (-\sin x) \\
 & = -2 \sin x (1+\cos x) \\
 & = -2\sin x -2 \sin x \cos x \\
 & = -2\sin x -\sin (2x). \quad \text{(This last step is optional.)}
\end{align*}
}%

\solution{\ref{problemd/dx(sin(sqrt(x)))} %
\begin{align*}
\text{Let } \quad u & = \sqrt{x}. \\
\text{Then } \quad y & = \sin u. \\
\text{Chain Rule: } \quad \frac{\diff y}{\diff x} & = \frac{\diff y}{\diff u}\frac{\diff u}{\diff x} \\
 & = (\cos u) \left(\frac{1}{2}u^{-\frac{1}{2}}\right) \\
 & = \frac{\cos\left(\sqrt{x}\right) }{2\sqrt{x}}.
\end{align*}
}%

\solution{\ref{problemd/dx(e^(-1/x))}
\[
\begin{array}{rcll|l}
\frac{\diff }{\diff x}\left(e^{-\frac{1}{x}}\right)&=&e^{-\frac{1}{x}} \frac{\diff }{\diff x}\left(-\frac{1}{x}\right)&&\text{chain rule}\\
&=&\displaystyle -e^{-\frac{1}{x}} \frac{\diff }{\diff x} \left(x^{-1}\right)\\
&=&\displaystyle x^{-2}e^{-\frac{1}{x}}\\
&=&\displaystyle \frac{e^{-\frac{1}{x}}}{x^2}
\end{array}
\]
}

\solution{\ref{problemd/dx(sqrt(sec(4x)))} %
\begin{align*}
\text{Chain Rule: } \quad \frac{\diff y}{\diff x} & = \left( \frac{1}{2}(\sec (4x))^{-\frac{1}{2}} \right) \frac{\diff}{\diff x}(\sec (4x)) \\
\text{Chain Rule: } \quad \frac{\diff y}{\diff x} & = \left( \frac{1}{2\sqrt{\sec (4x)}} \right) (\sec (4x) \tan (4x))\frac{\diff}{\diff x}(4x) \\
 & = \left( \frac{1}{2\sqrt{\sec (4x)}}\right) (\sec (4x) \tan (4x))(4) \\
 & =  \frac{2\sec (4x)\tan (4x)}{\sqrt{\sec (4x)}} \\
\intertext{There are many ways to simplify this answer, including both of the following.}
 & =  2\sqrt{\sec (4x)}\tan (4x). \\
 & =  2(\sec (4x))^{\frac{3}{2}}\sin (4x). 
\end{align*}
}%

\solution{\ref{problemd/dx(x^2tan(5x))} %
\begin{align*}
\text{Product Rule: } \quad \frac{\diff y}{\diff x} & = (x^2)\frac{\diff}{\diff x}(\tan (5x)) + (\tan (5x))\frac{\diff}{\diff x}(x^2) \\
\intertext{Use the Chain Rule to differentiate $\tan (5x)$ in the first term.}
\frac{\diff y}{\diff x} & = (x^2)(-5\sec^2 (5x) + (\tan (5x))(2x) \\
 & = 2x\tan (5x) - 5x^2\sec^2 (5x).
\end{align*}
}%


\solution{\ref{problemd/dx((1+sin(x^2))/(1+cos(x^2)))} %
\begin{align*}
\text{Quotient Rule: } \quad \frac{\diff y}{\diff x} & = \frac{\left(1+ \cos \left( x^2 \right)\right)\frac{\diff}{\diff y}(1+\sin \left(x^2\right) ) - (1+\sin \left(x^2\right))\frac{\diff}{\diff x}(1+\cos \left(x^2\right))}{(1+\cos \left(x^2\right))^2} \\
\intertext{By the Chain Rule, $\frac{\diff}{\diff x}(1+\cos \left(x^2\right)) = -2x\sin \left(x^2\right)$ and $\frac{\diff}{\diff x}(1+\sin \left(x^2\right)) = 2x\cos \left(x^2\right)$.}
\frac{\diff y}{\diff x} & = \frac{(1+\cos \left(x^2\right))(2x\cos \left(x^2\right)) - (1+\sin \left(x^2\right))(-2x\sin \left(x^2\right))}{(1+\cos \left(x^2\right))^2} \\
 & = \frac{2x\cos \left(x^2\right) + 2x\cos^2 \left(x^2\right) + 2x\sin \left(x^2\right) + 2x\sin^2 \left(x^2\right)}{(1+\cos \left(x^2\right))^2} \\
 & = \frac{2x(\cos^2 \left(x^2\right) + \sin^2 \left(x^2\right)) + 2x(\cos \left(x^2\right) + \sin \left(x^2\right))}{(1+\cos \left(x^2\right))^2} \\
\intertext{By the Pythagorean Identity, $\cos^2 \left(x^2\right) + \sin^2 \left(x^2\right) = 1$.}
\frac{\diff y}{\diff x} & = \frac{2x + 2x(\cos \left(x^2\right) + \sin \left(x^2\right))}{(1+\cos \left(x^2\right))^2} \\
 & = \frac{2x(1 + \cos \left(x^2\right) + \sin \left(x^2\right))}{(1+\cos \left(x^2\right))^2}.
\end{align*}
}%



\begin{problem}
Differentiate. The answer key has not been proofread, use with caution.
\begin{multicols}{3}
\begin{enumerate}[ref={\fcProblemRef}]
\item $\displaystyle f(x)=\sin (-5x)$. 

\answer{$ -5 \cos(-5 x)=-5\cos (5x)$}
\item $\displaystyle f(x)=\cot (2x)$. 

\answer{$-2\csc^2 (2x)$}
\item $\displaystyle f(x)=e^{-3x}$. 

\answer{$-3 e^{-3 x} $}
\item $\displaystyle f(x)=e^{\frac{1}x}$. 

\answer{$ -\frac{e^{\frac{1}{x}}}{x^2}$}
\item $\displaystyle f(x)=e^{\sqrt{x}}$. 

\answer{$ \frac{1}{2\sqrt{x}}e^{\sqrt{x}} $}
\item $\displaystyle f(x)=\ln (1+x) $

\answer{$\frac{1}{1+x} $}
\item $\displaystyle f(x)=\ln(1+x^3) $

\answer{$\frac{3x^2}{1+x^3}$}
\item $\displaystyle f(x)=\frac{1}{2}\ln\left(\frac{1+x}{1-x}\right) $

\answer{$\frac{1}{1-x^{2}}$}
\end{enumerate}
\end{multicols}

\end{problem}

\begin{problem}
Compute the derivative.
\begin{multicols}{2}
\begin{enumerate}
\item   $\displaystyle f(x)=\sqrt{1+x^2}$

\answer{$x (x^{2}+1)^{-\frac{1}{2}}  $}
\item \label{problemd/dx(cos(x))^(1/2)}  $\displaystyle f(x)=(\cos x)^{\frac{1}{ 2}}$

\answer{$-\frac{1}{2} \sin{}x (\cos{}x)^{-\frac{1}{2}}  $}
\item $\displaystyle f(x)=\sin^3 x$

\answer{$ 3 \cos{}x \sin^{2}{}x$}
\item \label{problemd/dx(1+cos(x))^2} $\displaystyle f(x)=(1+\cos x)^2$

\answer{$ -2 \cos{}x \sin{}x-2 \sin{}x =-\sin(2x)-2\sin x$}
\item   $\displaystyle f(x)=\frac{1}{\sin^3x}$

\answer{$  -\frac{3 \cos{}x}{\sin^{4}{}x} $}
\item  $\displaystyle f(x)= \sqrt[3]{4+3\tan x}$

\answer{$  (4+3\tan x)^{-\frac{2}{3}}\sec^2x $}
\item  $f(x)=(\cos x + 3\sin x)^4$

\answer{$4(\cos x + 3\sin x)^3 (3\cos x-\sin x) $}
\item \label{problemd/dx(sin(sqrt(x)))}  $f(x)=\sin\left(\sqrt{x}\right)$

\answer{$\frac{1}{2} x^{-\frac{1}{2}} \cos{}\left(\sqrt{x}\right)  = \frac{\cos \left(\sqrt{x}\right)}{2\sqrt{x}}$}
\item  $f(x)=\cos(4x)$

\answer{$-4 \sin{}\left(4 x\right)  $}
\item $\displaystyle f(x)= (x^4+3x^2-2)^5$.

\answer{$ \left(30 x +20 x^{3}\right) \left(-2+3 x^{2}+x^{4}\right)^{4}$}
\item $\displaystyle f(x)= (4x-x^2)^{100}$.

\answer{$(-200 x+400) \left(4 x- x^{2}\right)^{99}$}
\item $\displaystyle f(x)= \sqrt{1-2x}$.

\answer{$- (1-2 x)^{-\frac{1}{2}}$}
\item $\displaystyle f(x)= \frac{1}{(1+\sec x)^2}$.

\answer{$\frac{-2 \cos{}(x) \sin{}(x)}{(1+\cos{}(x))^{3}} =\frac{- \sin{}(2x)} {(1+\cos{}(x))^{3}} $}
\item $\displaystyle f(x)=\frac{1}{1+x^2} $.

\answer{$\frac{-2 x}{(1+x^{2})^{2}} $}
\item $\displaystyle f(x)= \sqrt[3]{1+\tan x}$.

\answer{$ 
\frac{1}{3}\left(1+\tan x \right)^{-\frac{2}{3}} \sec^2 x
$}
\item $\displaystyle f(x)=\cos (a^3+x^3) $.

\answer{$ -3 x^{2}\sin{}(a^{3}+x^{3}) $}
\item $\displaystyle f(x)= a^3+\cos^3 x$.

\answer{$ -3 \sin{}(x) (\cos{}(x))^{2}$}
\item $\displaystyle f(x)= x\sec (k x) $.

\answer{$\frac{\cos{}(k x)+k x \sin{}(k x) }{(\cos{}(k x))^{2}} $}
\item $\displaystyle f(\theta)= 3\cot (n\theta)$.

\answer{$ \frac{-3 n}{(\sin{}(n \theta))^{2}}$}
\item $\displaystyle f(x)= (2x - 3)^4 (x^2 + x + 1)^5$.

\answer{$ \left(-7-12 x+28 x^{2}\right)\left(-3+2 x\right)^{3} \left(1+x +x^{2} \right)^{4}$}
\item $\displaystyle f(x)= (x^2+1)^3(x^2+2)^6$.

\answer{$\left(24 x+18 x^{3}\right)\left(1+x^{2}\right)^{2} \left(2+x^{2}\right)^{5} $}
\item $\displaystyle f(t)= (t+1)^{\frac{2}{3}}(2t^2-1)^3$.

\answer{$ \left(\frac{40}{3} t^{2}+12 t-\frac{2}{3}\right)\left(2 t^{2}-1\right)^{2}\left(t+1\right)^{-\frac{1}{3}}$}
\item $\displaystyle f(t)= (3t-1)^4(2t+1)^{-3}$.

\answer{$(3 t-1)^{3}\frac{6 t+18}{(2 t+1)^{4}}$}
\item $\displaystyle f(x)=\left(\frac{x^2+1}{x^2-1} \right)^3 $.

\answer{$\frac{-12 x}{\left(x^{2}-1\right)^{2}} \left(\frac{x^{2}+1}{x^{2}-1}\right)^{2} $}
\item $\displaystyle f(s)= \sqrt{\frac{s^2+1}{s^2+4}}$.

\answer{$\frac{3 s}{\left(s^{2}+4\right)^{2}} \left(\frac{s^{2}+1}{s^{2}+4}\right)^{-\frac{1}{2}} $}

\end{enumerate}
\end{multicols}


\end{problem}
\solution{\ref{problemd/dx(cos(x))^(1/2)}%
\begin{align*}
\text{Let } \quad u & = \cos x. \\
\text{Then } \quad y & = u^{\frac{1}{2}}. \\
\text{Chain Rule: } \quad \frac{\diff y}{\diff x} & = \frac{\diff y}{\diff u}\frac{\diff u}{\diff x} \\
 & = \left(\frac{1}{2}u^{-\frac{1}{2}}\right) (-\sin x) \\
 & = -\frac{1}{2} \sin x (\cos x)^{-\frac{1}{2}}.
\end{align*}
}%


\solution{\ref{problemd/dx(1+cos(x))^2} %
\begin{align*}
\text{Let } \quad u & = 1+\cos x. \\
\text{Then } \quad y & = u^{2}. \\
\text{Chain Rule: } \quad \frac{\diff y}{\diff x} & = \frac{\diff y}{\diff u}\frac{\diff u}{\diff x} \\
 & = (2u) (-\sin x) \\
 & = -2 \sin x (1+\cos x) \\
 & = -2\sin x -2 \sin x \cos x \\
 & = -2\sin x -\sin (2x). \quad \text{(This last step is optional.)}
\end{align*}
}%

\solution{\ref{problemd/dx(sin(sqrt(x)))} %
\begin{align*}
\text{Let } \quad u & = \sqrt{x}. \\
\text{Then } \quad y & = \sin u. \\
\text{Chain Rule: } \quad \frac{\diff y}{\diff x} & = \frac{\diff y}{\diff u}\frac{\diff u}{\diff x} \\
 & = (\cos u) \left(\frac{1}{2}u^{-\frac{1}{2}}\right) \\
 & = \frac{\cos\sqrt{x}}{2\sqrt{x}}.
\end{align*}
}%


\begin{problem}
Differentiate. 
\begin{multicols}{2}
\begin{enumerate}
\item $\displaystyle f(x)=\sin (\tan (2x)) $.

\answer{$2\sec^2(2x) \cos (\tan 2x) $}
\item $\displaystyle f(x)=\sec^2(m x) $.


\answer{$ \frac{2 m \sin{}(m x) }{(\cos{}(m x))^{3}} $}
\item $\displaystyle f(x)= \sec^2 x+\tan^2 x$.

\answer{$\frac{4 \sin{}x}{\cos^{3}{}x} $}
\item $\displaystyle f(x)=x\sin\left( \frac{1}{x}\right) $.

\answer{$- x^{-1}\cos{}(x^{-1}) +\sin{}(x^{-1})$}
\item $\displaystyle f(x)= \left(\frac{1-\cos (2x)}{1+\cos (2x)}\right)^4$.

\answer{$\frac{16 \sin{}(2 x)}{(\cos{}(2 x)+1)^{2}} \left(\frac{- \cos{}(2 x)+1}{\cos{}(2 x)+1}\right)^{3} $}
\item $\displaystyle f(x)=\sqrt{\frac{x}{x^2+4}} $.

\answer{$ \frac{-\frac{1}{2} x^{2}+2}{(x^{2}+4)^{2}} \left(\frac{x}{x^{2}+4}\right)^{-\frac{1}{2}}$}
\item $\displaystyle f(t)= \cot^2(\sin t)$.

\answer{$ \frac{-2 \cos{}t \cos{}(\sin{}t)}{\sin^{3}{}(\sin{}t)}$}
\item $\displaystyle f(x)= \left(a x+\sqrt{x^2+b^2}\right)^{-2}$.

\answer{$\frac{-2 x\left(x^{2}+b^{2} \right)^{-\frac{ 1 }{2}} -2 a}{\left(\left(x^{ 2}+b^{2}\right)^{ \frac{1}{2}} +a x\right)^{3}} $}
\item $\displaystyle f(x)= \left(x^2+(1-3x)^5 \right)^3$.

\answer{
\begin{tabular}{l}
$\left(-45 (-3 x+1)^{4} +6 x \right) \left((-3 x+1)^{5}+x^{2}\right)^{2}$
\\
Using computer algebra:
\\
$(-3645x^{4}+4860x^{3}-2430x^{2}+546x -45)\left((-3 x+1)^{5}+x^{2}\right)^{2}$ \\
Using computer algebra full expansion:
\\
$\begin{array}{l}
-215233605x^{14}+1004423490x^{13}-2176250895x^{12}\\
+2903793624x^{11} -2666357595x^{10}+1782098820x^{9}\\
-893713176x^{8} +341444160x^{7} -99805041x^{6} \\ +22199676x^{5}-3697470x^{4}  +447132x^{3}\\
-37125x^{2}+1896x -45
\end{array}
$
\end{tabular} 
}
\item $\displaystyle f(x)=\sin (\sin (\sin x))$.

\answer{$\cos{}x \cos{}(\sin{}x) \cos{}(\sin{}(\sin{}x)) $}
\item $\displaystyle f(x)= \sqrt{x+\sqrt{x}}$.

\answer{$ \left(\frac{1}{2} +\frac{1}{4} x^{-\frac{1}{2}}\right) \left(x^{\frac{1}{2}}+x\right)^{-\frac{1}{2}}$}

\item $\displaystyle f(x)= \sqrt{x+\sqrt{x+\sqrt{x}}}$.

\answer{$\frac{1}{2} \left(\left(x^{\frac{1}{2}}+x\right)^{\frac{1}{2}}+x\right)^{-\frac{1}{2}} \left(\frac{1}{2} \left(x^{\frac{1}{2}}+x\right)^{-\frac{1}{2}} \left(\frac{1}{2} x^{-\frac{1}{2}}+1\right)+1\right) $}
\item $\displaystyle f(x)=(2r \sin (r x)+n)^p $.

\answer{$ p r(2 r \sin{}(r x)+n)^{p-1} \cos{}(r x) $}
\item $\displaystyle f(x)=\cos^4(\sin^3 x) $.

\answer{$-12 \cos{}x \sin^{2}{}x \sin{}(\sin^{3}{}x) \cos^{3}{}(\sin^{3}{}x) $}
\item $\displaystyle f(x)=\cos \sqrt{\sin (\tan (\pi x))} $.

\answer{$ \frac{-\frac{1}{2} \pi \cos{}(\tan{}(\pi x))  \sin{}\left(\sqrt{\sin (\tan{}(\pi x) )} \right)}{\sqrt{\sin{}(\tan{}(\pi x))} \cos^{2}{}(\pi x) }$}
\item $\displaystyle f(x)=\left(x+(x+\sin^2 x)^3 \right)^4 $.

\answer{$4 ((\sin^{2}{}x+x)^{3}+x)^{3} (3 (\sin^{2}{}x+x)^{2} (2 \sin{}x \cos{}x+1)+1) $}
\end{enumerate}
\end{multicols}
\end{problem}
\begin{problem}
Compute the second derivative.
\begin{multicols}{3}
\begin{enumerate}
\item $\displaystyle f(x)=\sin (-5x)$. 

\answer{$ 25 \sin{}(5 x)$}
\item $\displaystyle f(x)=e^{-3x}$. 

\answer{$9 e^{-3 x} $}
\item $\displaystyle f(x)=e^{\frac{1}x}$. 

\answer{$ 2 e^{x^{-1}} x^{-3}+e^{x^{-1}} x^{-4}$}
\item $\displaystyle f(x)=e^{\sqrt{x}}$. 

\answer{$ e^{x^{\frac{1}{2}}} x^{-\frac{3}{2}}+\frac{1}{4} e^{x^{\frac{1}{2}}} x^{-1}$}
\item $\displaystyle f(x)=\frac{e^{x}-e^{-x}}{e^x+e^{-x}} $

\answer{$\frac{-8 \left(- e^{- x}+e^{x}\right)}{\left(e^{- x}+e^{x}\right)^{3}} $}
\item $\displaystyle f(x)=\frac{1}2\ln \left(\frac{1+x}{1-x}\right) $

\answer{$\frac12\left(-\frac{1}{(x+1)^{2}}+\frac{1}{(- x+1)^{2}}\right)= \frac{ 2x}{\left(1-x^2\right)^2} $}
\end{enumerate}
\end{multicols}

\end{problem}
\subsection{Problem Collection All Techniques}
\begin{problem}
%(Problem contributed by Gabe Cunningham) 

Find the derivative of the following functions.
\begin{multicols}{2}
\begin{enumerate}
\item ${\displaystyle \frac{\sin x}{x^2}}$

\answer{${\displaystyle \frac{x \cos x - 2 \sin x}{x^3}}$}
\item ${\displaystyle e^{\sqrt{x^2 + 1}}}$

\answer{
$\begin{array}{l}\displaystyle e^{\sqrt{x^2 + 1}} \cdot \frac{1}{2}\left(x^2+ 1\right)^{ -\frac{ 1}{2}} \cdot 2x \\~\\
= \frac{x e^{\sqrt{x^2+1}} }{ \sqrt{ x^2 +1} } \end{array}$
} 
\item ${\displaystyle \ln \left(x-\frac{1}{x} \right)}$

\answer{${\displaystyle \frac{1}{x-\frac{1}{x}} \cdot \left(1 + \frac{1}{x^2}\right)}$} 
\item ${\displaystyle \sqrt[3]{x} \ln x}$

\answer{${\displaystyle \frac{1}{3} \frac{1}{\sqrt[3]{x^2}} \ln x + \frac{1}{\sqrt[3]{x^2}} }$} 
\item ${\displaystyle \cos(e^x)}$

\answer{${\displaystyle -\sin\left(e^x\right) \cdot e^x}$} 
\item ${\displaystyle \sin^3(2x)}$

\answer{${\displaystyle 3 \sin^2(2x) \cdot \cos(2x) \cdot 2 = 6 \sin^2(2x) \cos(2x)}$} 
\item ${\displaystyle f(x) = \int_x^1 (2+t^4)^5 \; \diff t}$

\answer{${\displaystyle -\left(2+x^4\right)^5}$} 
\item ${\displaystyle g(x) = \int_{0}^{x^3} \cos^2 t \; \diff t}$

\answer{${\displaystyle 3x^2 \cos^2\left(x^3\right)}$} 
\item Find $y'$ if $2x^2 + x + xy = 1$.

\answer{${\displaystyle y' = \frac{-4x-1-y}{x}}$} 
\item Find $y'$ if $x \sin y + y \sin x = 4$.

\answer{${\displaystyle y' = \frac{-\sin y - y \cos x}{\sin x + x \cos y}}$} 
\end{enumerate}
\end{multicols}

\end{problem}
\solution{\ref{problemDifferentiateFTC1int_x^1(2+t^4)^5dt} %(Contributed by student Anamaria Ronayne)
We recall that the Fundamental Theorem of Calculus part 1 states that $\frac{\diff}{\diff x}\left(\int_{a}^{x}h(t)dt\right)=h(x)$
where $a$ is a constant. We can rewrite the integral so it has $x$ as the upper limit:
\[
f(x)=\int_{x}^{1}(2+1^4)^5\diff t =-\int_{1}^{x}(2+1^4)^5\diff t\quad.
\]
Therefore
\[
\frac{\diff}{\diff x}\left( -\int_{1}^{x}(2+t^4)^5 \diff t\right)=- \frac{\diff }{\diff x}\left(\int_{1}^{x}(2+t^4)^5\diff t \right)\stackrel{\text{FTC part 1}}{=}
-(2+x^4)^5\quad .
\]

}

\subsection{Implicit Differentiation}\label{secMPSImplicitDifferentiation}
\begin{problem}
Express $\frac{\diff y}{\diff x}$ as a function of $x$ and $y$ by implicit differentiation. The answer key has not been proofread, use with caution.
\begin{multicols}{2}
\begin{enumerate}
\item $x^3+y^3=1$.

\answer{$\frac{\diff y}{\diff x}=-\frac{x^2}{y^2}$}
\item $ 2\sqrt x+\sqrt y=3$.

\answer{$\frac{\diff y}{\diff x}=-2\sqrt{\frac{ y}{x}}$}
\item $ x^2+x y-y^2=4$.

\answer{$\frac{\diff y}{\diff x}=\frac{-2x-y}{x-2y}$}
\item $ 2x^3+x^2y-x y^3=2$.

\answer{$\frac{\diff y}{\diff x}=\frac{y^{3}-6 x^{2}-2 x y}{-3 x y^{2}+x^{2}}$}
\item $ x^4(x+y)=y^2(3x-y)$.

\answer{$\frac{\diff y}{\diff x}= \frac{ -5x^4 -4x^3y +3y^2}{x^4- 6xy - 3y^2}$}
\item $ y^5+x^2y^3=1+x^4y $.

\answer{$\frac{\diff y}{\diff x}=\frac{ 4x^3y-2xy^3}{5y^4+3x^2y^2- x^4 }$}
\item $ y\cos x=x^2+y^2 $.

\answer{$\frac{\diff y}{\diff x}= \frac{ y\sin x+2x}{\cos x-2y}$}
\item $ \cos (x y)=1+\sin y$.

\answer{$\frac{\diff y}{\diff x}= -\frac{y\sin (xy)}{\cos y+x\sin(xy) }$}
\item $ 4\cos x\sin y=1$.

\answer{$\frac{\diff y}{\diff x}=\tan x \tan y$}
\item $ y\sin \left(x^2\right)=x\sin \left(y^2\right)$.

\answer{$\frac{\diff y}{\diff x}=\frac{-2xy\cos\left(x^2\right)+\sin \left(y^2\right)}{ - 2 x y \cos\left( y^2 \right)+\sin \left(x^2 \right) }$}
\item $ \tan \left(\frac{x}{y}\right)=x+y$.

\answer{$\frac{\diff y}{\diff x}=\frac{-y^2+y\sec^2 \left(\frac{y}{x}\right) }{y^2 + x\sec^2\left(\frac{x}{y}\right) }$}
\item $ \sqrt{x+y}=1+x^2y^2$.

\answer{$\frac{\diff y}{\diff x}= \frac{-\frac{1}{2} (y+x)^{-\frac{1}{2}}+8 x y^{2}}{\frac{1}{2} (y+x)^{-\frac{1}{2}}-8 x^{2} y} $}
\item $ \sqrt{xy}=1+x^2 y$.

\answer{$\frac{\diff y}{\diff x}=\frac{-\frac{1}{2} x^{-\frac{1}{2}} y^{\frac{1}{2}} -2 x y}{\frac{1}{2} x^{\frac12} y^{-\frac{1}{2}} +x^{2}} $}
\item $ x\sin y+y\sin x=1$.

\answer{$\frac{\diff y}{\diff x}=\frac{- y \cos{}x- \sin{}y}{x \cos{}y+\sin{}x}$}
\item $ y\cos x=1+\sin (x y)$.

\answer{$\frac{\diff y}{\diff x}=\frac{\cos{}(x y) y+y \sin{}x}{- \cos{}(x y) x+\cos{}x}$}
\item $ \tan (x-y)=\frac{y}{1+x^2}$.

\answer{$\frac{\diff y}{\diff x}=\frac{- (\sec{}(- y+x))^{2} x^{4}-2 (\sec{}(- y+x))^{2} x^{2}- (\sec{}(- y+x))^{2}-2 y x}{- (\sec{}(- y+x))^{2} x^{4}-2 (\sec{}(- y+x))^{2} x^{2}- (\sec{}(- y+x))^{2}- x^{2}-1}$}
\end{enumerate}
\end{multicols}
\end{problem}
\begin{problem}
Verify that the coordinates of the given point satisfy the given equation. Use implicit differentiation to find an equation of the tangent line to the curve at the given point. 
\begin{multicols}{2}
\begin{enumerate}[ref={\fcProblemRef}]
\item 
\label{problemImplicitTangentysin(2x)=xcos(2y)point(pi/2,pi/4)} $y\sin (2x)=x\cos (2y) $, $\left(\frac{\pi}{2}, \frac{\pi}{4}\right)$. 

\psset{xunit=0.3cm, yunit=0.3cm}
\begin{pspicture}(-5.8,-5.8)(5.8,5.8)
\fcAxesStandard{-5.5}{-5.5}{5.5}{5.5}
\fcLabels{5.5}{5.5}
\fcXTickWithLabel{1}{$1$}
\fcImplicitIId[linestyle=solid, linecolor=red, linewidth-=0.05, showGridImplicitIId=false]{-5}{-5}{1000}{1000}{0.01}{0.01}{2 x mul 180 mul 3.141592654 div sin y mul 2 y mul 180 mul 3.141592654 div cos x mul sub} 

\fcFullDot[linecolor=blue]{3.141592654 2 div}{3.141592654 4 div}
\end{pspicture}

\answer{$y=\frac{1}{2}x$}

\item $ \sin (x+y)=2x-2y$, $(\pi,\pi)$ . 

\answer{$\frac{1}{3} x+\frac{2}{3} \pi $}
\item 
$x^2+x y+y^2=3 $, $(1,-2)$ (ellipse). 

\psset{xunit=0.5cm, yunit=0.5cm}
\begin{pspicture}(-3.6,-3.8)(3.6,3.6)
\tiny
\fcAxesStandard{-3.5}{-3.5}{3.5}{3.5}
\fcLabels{3.5}{3.5}
\fcXTickWithLabel{1}{$1$}
\fcImplicitIId[linestyle=solid, linecolor=red, linewidth-=0.05, showGridImplicitIId=false]{-2}{-2}{400}{400}{0.01}{0.01}{x x mul x y mul add y y mul add 3 sub}
%\psline[linecolor=blue](-3.5,-2)(3.5,-2)
\fcFullDot[linecolor=blue]{1}{-2}
\end{pspicture}


\answer{$y=-2 $}

\item  $x^2+2x y-y^2+x=2 $, $(1,2)$ (hyperbola). 

\psset{xunit=0.4cm, yunit=0.4cm}
\begin{pspicture}(-5.6,-5.6)(5.6,5.6)
\tiny
\fcAxesStandard{-5.5}{-5.5}{5.5}{5.5}
\fcLabels{5.5}{5.5}
\fcXTickWithLabel{1}{$1$}
\fcImplicitIId[linestyle=solid, linecolor=red, linewidth-=0.05, showGridImplicitIId=false]{-5}{-5}{500}{500}{0.02}{0.02}{x x mul x y mul 2 mul add y y mul sub x add 2 sub}
%\psline[linecolor=blue](-3.5,-2)(3.5,-2)
\fcFullDot[linecolor=blue]{1}{2}
\end{pspicture}

\answer{$y= \frac{7}{2} x-\frac{3}{2}$}


\item  \label{problemImplicitTangenty^3+x^3+4xy=3/4}
$\displaystyle y^{3}+x^{3}+4 x y=\frac{3}{4}$, $\left(-\frac{1}{2},-\frac{1}{2}\right)$
\psset{xunit=0.7cm, yunit=0.7cm}
\begin{pspicture}(-2.4,-2.4)(2.4,2.4)
\fcAxesStandard{-2}{-2}{2}{2}
\fcImplicitIId[linecolor=red, linestyle=dashed, dashes={[1 1] 0}, showGridImplicitIId=false, useMidpointImplicitPlots=false]{-3}{-3}{120}{120}{0.05}{0.05}{y y y mul mul x x x mul mul add 4 x y mul mul add 0.75 sub} 
\fcFullDot{-0.5}{-0.5}
\end{pspicture}


\answer{$y=-x-1$}
\item  \label{problemImplicitTangenty^3+x^3+4xy=-4at(1,-1)}
$\displaystyle y^{3}+x^{3}+4 x y=-4$, $(1,-1)$

\psset{xunit=0.7cm, yunit=0.7cm}
\begin{pspicture}(-3.1,-3.1)(3.1,3.1)
\fcAxesStandard{-2}{-2}{2}{2}
\fcImplicitIId[linecolor=red, linestyle=dashed, dashes={[1 1] 0}, showGridImplicitIId=false, useMidpointImplicitPlots=false]{-3}{-3}{300}{300}{0.02}{0.02}{y y y mul mul x x x mul mul add 4 x y mul mul add 4 add} 
\fcFullDot{1}{-1}
\end{pspicture}

\answer{$ $}

\item $x^2+y^2=(2x^2+2y^2-x)^2 $, $(0,\frac{1}{2})$. 

\answer{$y= x+\frac{1}{2}$}
\item $x^{\frac{2}{3}}+y^{\frac{2}{3}}=4$, $(-3\sqrt{3},1)$. 

\answer{$y=\frac{1}{\sqrt{3}}x+4 $}
\item $2(x^2+y^2)^2 =25(x^2-y^2)$, $(3,1)$. 

\answer{$y= -\frac{9}{13} x+\frac{40}{13}$}
\item $y^2(y^2-4)=x^2(x^2-5) $, $(0,-2)$. 

\answer{$y=-2 $}
\item $x^{\frac{4}{3}}+y^{\frac{4}{3}}=10$ at $(-3\sqrt{3}, 1)$. 

\answer{$y=\sqrt{3}x+10$ }
\item $x^2y^3+x^3-y^2=1$ at $(1,1)$. 

\answer{$y=-5 x+6$}
\end{enumerate}
\end{multicols}

\end{problem}
\solution{\ref{problemImplicitTangentysin(2x)=xcos(2y)point(pi/2,pi/4)}

\psset{xunit=0.5cm, yunit=0.5cm}
\begin{pspicture}(-2,-2)(2,2)
\fcAxesStandard{-5.5}{-5.5}{5.5}{5.5}
\fcLabels{5.5}{5.5}
\fcXTickWithLabel{1}{$1$}
\fcImplicitIId[linestyle=solid, linecolor=red, linewidth-=0.05, showGridImplicitIId=false]{-5}{-5}{1000}{1000}{0.01}{0.01}{2 x mul 180 mul 3.141592654 div sin y mul 2 y mul 180 mul 3.141592654 div cos x mul sub} 

\psline[linecolor=blue](-5,-2.5)(5,2.5)
\fcFullDot[linecolor=blue]{3.141592654 2 div}{3.141592654 4 div}
\end{pspicture}


First we verify that the point $\displaystyle (x,y)=\left(\frac{\pi}{2}, \frac{\pi}{4}\right)$ indeed satisfies the given equation:

\[
\begin{array}{rcll|l}
\displaystyle y \sin (2x)_{|x=\frac{\pi}{2}, y=\frac{\pi}{4}}= \frac{\pi}{4}\sin \pi &=& \displaystyle 0 && \text{left hand side}\\
\displaystyle x \cos (2y)_{|x=\frac{\pi}{2}, y=\frac{\pi}{4}}= \frac{\pi}{2}\cos \left(\frac{\pi}{2} \right)&=&\displaystyle 0 && \text{right hand side}\\
\end{array}
\]
so the two sides of the equation are equal (both to $0$) when $x=\frac{\pi}{2}$ and $y=\frac{\pi}{4}$.

Since we are looking an equation of the tangent line, we need to find  $\frac{\diff y}{\diff x}_{|x=\frac{\pi}{2}, y= \frac{\pi}{4}}$ - that is, the derivative of $y$ at the point $x=\frac{\pi}{2}$, $y= \frac{\pi}{4}$. To do so we use implicit differentiation.
\[
\begin{array}{rcll|l}
\displaystyle y \sin (2x)&=&\displaystyle x\cos (2y)&&\frac{\diff }{\diff x}\\
\displaystyle \frac{\diff y}{\diff x} \sin (2x) +y \frac{\diff }{\diff x}\left(\sin (2x)\right)&=&\displaystyle  \cos (2y)+x\frac{\diff }{\diff x}(\cos (2y))\\
\displaystyle \frac{\diff y}{\diff x}\sin (2x)+2y\cos (2x)&= & \displaystyle \cos (2y)-2x \sin (2y) \frac{\diff y}{\diff x}\\
\displaystyle \frac{\diff y}{\diff x}(\sin (2x)+2x\sin (2y))&=&\displaystyle \cos (2y)-2y\cos (2x)&& \text{Set }x=\frac{ \pi}{2}, y=\frac{\pi}{4}\\
\displaystyle \frac{\diff y}{\diff x}_{|x=\frac{\pi}{2}, y=\frac{\pi}{4}} \left(\sin \pi+\pi \sin \left(\frac{\pi}{2}\right)\right)&=& \displaystyle \cos \left(\frac{\pi}{2}\right)-\frac{\pi}{2}\cos \pi\\
\displaystyle \pi \frac{\diff y}{\diff x}_{|x=\frac{\pi}{2}, y=\frac{\pi}{4}} &=& -\frac{\pi}{2}\cos \pi\\
\displaystyle \frac{\diff y}{\diff x}_{|x=\frac{\pi}{2}, y=\frac{\pi}{4}}&=& \displaystyle \frac{1}{2}.
\end{array}
\]
Therefore the equation of the line through $x=\frac{\pi}{2}, y=\frac{\pi}{4}$ is 
\[
\begin{array}{rcl}
\displaystyle y-\frac{\pi}{4}&=&\displaystyle \frac{1}{2}\left( x-\frac{\pi}{2} \right)\\
y&=&\displaystyle \frac{1}{2} x .
\end{array}
\]
}

\solution{\ref{problemImplicitTangenty^3+x^3+4xy=3/4}
\[
\begin{array}{rcll|l}
\displaystyle y^{3}+x^3+4x y&=& \displaystyle \frac{3}{4} &&\text{apply }\frac{\diff }{\diff x}\\
\displaystyle 3y^2 \frac{\diff y}{\diff x}+3x^2+4\left(y+x\frac{\diff y}{\diff x}\right)&=&
\displaystyle \frac{\diff y}{\diff x}\left(3y^2+4x\right)&=&\displaystyle -3x^2-4y\\
\displaystyle \frac{\diff y}{\diff x}&=&\displaystyle  \frac{-3x^2-4y}{ 3y^2+4x}&&\text{substitute }x=-\frac{1}{2}, y=-\frac{1}{2}\\
\displaystyle \frac{\diff y}{\diff x}_{|x=-\frac{1}{2}, y=-\frac{1}{2}}&=& \frac{-3\left(-\frac{1}{2}\right)^2-4\left(-\frac{1}{2}\right)}{ 3\left(-\frac{1}{2}\right)^2+4\left(-\frac{1}{2}\right)}\\
\displaystyle \frac{\diff y}{\diff x}_{|x=-\frac{1}{2}, y=-\frac{1}{2}}&=&-1
\end{array}
\]
Therefore the equation of the tangent is $y-\left(-\frac{1}{2}\right)=-(x-\left(-\frac{1}{2}\right))$ which simplifies to $y=-x-1$. A computer-generated plot visually confirms our computations.

\psset{xunit=0.7cm, yunit=0.7cm}
\begin{pspicture}(-2.4,-2.4)(2.4,2.4)
\fcAxesStandard{-2}{-2}{2}{2}
\fcImplicitIId[linecolor=red, linestyle=dashed, dashes={[1 1] 0}, showGridImplicitIId=false, useMidpointImplicitPlots=false]{-3}{-3}{120}{120}{0.05}{0.05}{y y y mul mul x x x mul mul add 4 x y mul mul add 0.75 sub} 
\psplot[linecolor=\fcColorTangent]{-3}{2}{x -1 mul -1 add}
\fcFullDot{-0.5}{-0.5}
\end{pspicture}
}

\solution{\ref{problemImplicitTangenty^3+x^3+4xy=-4at(1,-1)}

\[
\begin{array}{rcll|l}
\displaystyle y^{3}+x^3+4x y&=& \displaystyle -4 &&\text{apply }\frac{\diff }{\diff x}\\
\displaystyle 3y^2 \frac{\diff y}{\diff x}+3x^2+4\left(y+x\frac{\diff y}{\diff x}\right)&=&0\\
\displaystyle \frac{\diff y}{\diff x}\left(3y^2+4x\right)&=&\displaystyle -3x^2-4y\\
\displaystyle \frac{\diff y}{\diff x}&=&\displaystyle  \frac{-3x^2-4y}{ 3y^2+4x}&&\text{substitute }x=1, y=-1\\
\displaystyle \frac{\diff y}{\diff x}_{|x=1, y=-1}&=&\displaystyle  \frac{-3\left(1\right)^2-4\left(-1\right)}{ 3\left(-1\right)^2+4\left(1\right)} \\
\displaystyle \frac{\diff y}{\diff x}_{|x=-\frac{1}{2}, y=-\frac{1}{2}}&=& \displaystyle  \frac{1}{7}
\end{array}
\]
Therefore the equation of the tangent is $\displaystyle y-\left(- 1\right)= \frac{1}{7}\left(x-1\right)$ which simplifies to $\displaystyle  y=\frac{x}{7}-\frac{8}{7}$. A computer-generated plot visually confirms our computations.

\psset{xunit=0.7cm, yunit=0.7cm}
\begin{pspicture}(-3.1,-3.1)(3.1,3.1)
\fcAxesStandard{-2}{-2}{2}{2}
\fcImplicitIId[linecolor=red, linestyle=dashed, dashes={[1 1] 0}, showGridImplicitIId=false, useMidpointImplicitPlots=false]{-3}{-3}{300}{300}{0.02}{0.02}{y y y mul mul x x x mul mul add 4 x y mul mul add 4 add} 
\fcFullDot{1}{-1}
\psplot[linecolor=blue]{-3}{3}{x 7 div -8 7 div add}
\end{pspicture}
}

\subsection{Implicit Differentiation and Inverse Trigonometric Functions}
\begin{problem}
% begin homework implicit-inverse-trig1
The variables $x$ and $y$ are related by
\[
x\arctan y + y\arctan x = \frac{\pi}{2}.
\]

\begin{enumerate}
\item   Show that $(1,1)$ is on the graph of this relation.  

\solution{%
\begin{align*}
\text{LS} & = 1\cdot \arctan 1 + 1\cdot \arctan 1 & \text{RS} & = \frac{\pi}{2}. \\
 & = 1\cdot \frac{\pi}{4} + 1\cdot \frac{\pi}{4}  & & \\
 & = \frac{\pi}{2}.  & & 
\end{align*}
The fact that the left side equals the right side when we plug in $x = 1$ and $y = 1$ means that the point $(1,1)$ is on the graph of the relation.  
}%

\item   Find $\frac{\diff y}{\diff x}$ in terms of $x$ and $y$.  

\solution{%
Differentiate implicitly.
\begin{align*}
\big((x)\frac{\diff}{\diff x}(\arctan y) + (\arctan y)\frac{\diff}{\diff x}(x)\big)  + \big((y)\frac{\diff}{\diff x}(\arctan x) + (\arctan x)\frac{\diff}{\diff x}(y)\big)  & = 0 \\
x\cdot \frac{1}{1+y^2}\cdot \frac{\diff y}{\diff x} + (\arctan y)1 + y\cdot \frac{1}{1+x^2} + (\arctan x)\frac{\diff y}{\diff x} & = 0 \\
\frac{x}{1+y^2}\frac{\diff y}{\diff x} + \arctan y + \frac{y}{1+x^2} + \frac{\diff y}{\diff x}\arctan x & = 0.
\end{align*}
Rearrange to isolate $\frac{\diff y}{\diff x}$ on one side.  
\begin{align*}
\frac{x}{1+y^2}\frac{\diff y}{\diff x} +  \frac{\diff y}{\diff x}\arctan x & = - \frac{y}{1+x^2} - \arctan y \\
\big(\frac{x}{1+y^2} + \arctan x\big)\frac{\diff y}{\diff x} & = - \big(\frac{y}{1+x^2} + \arctan y\big) \\
\frac{\diff y}{\diff x} & = -\frac{\frac{y}{1+x^2}+\arctan y}{\frac{x}{1+y^2}+\arctan x}. 
\end{align*}

}%

\item   Find the equation of the tangent to the graph at $(1,1)$.  

\solution{%
To find the slope of the tangent, plug in $x=1,y=1$ to the formula for $\frac{\diff y}{\diff x}$.  

\begin{align*}
\frac{\diff y}{\diff x} & = -\frac{\frac{1}{1+1^2}+\arctan 1}{\frac{1}{1+1^2}+\arctan 1} \\
& = -1.
\end{align*}

Now use the point $(1,1)$ to find an equation for the tangent line.  
\begin{align*}
y - 1 & = (-1)(x-1) \\
y & = -x +2.
\end{align*}
}%

\end{enumerate}
% end homework implicit-inverse-trig1

\end{problem}
\begin{problem}
% begin homework implicit-inverse-trig2
The variables $x$ and $y$ are related by
\[
x^2y+xy^2+\arcsin x = \frac{\pi}{6}.
\]

\begin{enumerate}
\item   Find all points on the graph of this relation for which $x = 1/2$.  

\solution{%
Set $x = 1/2$ and solve for $y$.  
\begin{align*}
\big(\frac{1}{2}\big)^2y+\frac{1}{2}y^2 + \arcsin \frac{1}{2} & = \frac{\pi}{6} \\
\frac{1}{4}y + \frac{1}{2}y^2 + \frac{\pi}{6} & = \frac{\pi}{6} \\
\frac{1}{4}y + \frac{1}{2}y^2  & = 0 \\
\frac{1}{4}y(1 + 2y)  & = 0,
\end{align*}
so $y = 0$ or $y = -1/2$.  
Therefore $(1/2,0)$ and $(1/2,-1/2)$ are the points on the graph of the relation for which $x = 1/2$.  
}%

\item   Find $\frac{\diff y}{\diff x}$ in terms of $x$ and $y$.  

\solution{%
Differentiate implicitly.
\begin{align*}
\big((x^2)\frac{\diff}{\diff x}(y) + (y)\frac{\diff}{\diff x}(x^2)\big) + \big( (x)\frac{\diff}{\diff x}(y^2) + (y^2)\frac{\diff}{\diff x}(x)\big) + \frac{1}{\sqrt{1-x^2}} & = 0 \\
x^2\frac{\diff y}{\diff x} + y(2x) + x(2y)\frac{\diff y}{\diff x} + y^2 + \frac{1}{\sqrt{1-x^2}} & = 0 \\
x^2\frac{\diff y}{\diff x} + 2xy + 2xy\frac{\diff y}{\diff x} + y^2 + \frac{1}{\sqrt{1-x^2}} & = 0.
\end{align*}
Rearrange to isolate $\frac{\diff y}{\diff x}$ on one side.  
\begin{align*}
x^2\frac{\diff y}{\diff x} + 2xy\frac{\diff y}{\diff x} & = -y^2-2xy-\frac{1}{\sqrt{1-x^2}} \\
(x^2+2xy)\frac{\diff y}{\diff x} & = -\Big(y^2+2xy+\frac{1}{\sqrt{1-x^2}}\Big) \\
\frac{\diff y}{\diff x} & = -\frac{y^2+2xy+\frac{1}{\sqrt{1-x^2}}}{x^2+2xy}.
\end{align*}

}%

\item   Find the equation of the tangent to the graph at each of the points you found in the first part.  

\solution{%
To find the slope of the tangent at $(1/2,0)$, plug in $x=1/2,y=0$ to the formula for $\frac{\diff y}{\diff x}$.  

\begin{align*}
\frac{\diff y}{\diff x} & = -\frac{(0)^2+2(1/2)(0) + \frac{1}{\sqrt{1-(1/2)^2}}}{(1/2)^2+2(1/2)(0)} \\
& = -\frac{0+0+\frac{1}{\sqrt{3/4}}}{1/4 + 0} \\
& = -\frac{\frac{1}{\sqrt{3}/2}}{1/4} \\
& = -\frac{2}{\sqrt{3}}\cdot \frac{4}{1} \\
& = -\frac{8}{\sqrt{3}}.
\end{align*}

Now use the point $(1/2,0)$ to find an equation for the tangent line.  
\begin{align*}
y - 0 & = -\frac{8}{\sqrt{3}}(x-1/2) \\
y & = -\frac{8}{\sqrt{3}}x +\frac{4}{\sqrt{3}}.
\end{align*}

This is the equation for one of the tangent lines.  

To find the slope of the tangent at $(1/2,-1/2)$, plug in $x=1/2,y=-1/2$ to the formula for $\frac{\diff y}{\diff x}$.  

\begin{align*}
\frac{\diff y}{\diff x} & = -\frac{(-1/2)^2+2(1/2)(-1/2) + \frac{1}{\sqrt{1-(1/2)^2}}}{(1/2)^2+2(1/2)(-1/2)} \\
& = -\frac{\frac{1}{4}-\frac{1}{2}+\frac{1}{\sqrt{3/4}}}{1/4 -\frac{1}{2}} \\
& = -\frac{-\frac{1}{4}+\frac{2}{\sqrt{3}}}{-\frac{1}{4}} \\
& = \frac{-\frac{1}{4}+\frac{2}{\sqrt{3}}}{\frac{1}{4}} \\
& = 4\big(-\frac{1}{4}+\frac{2}{\sqrt{3}}\big) \\
& = -1 + \frac{8}{\sqrt{3}}.
\end{align*}

Now use the point $(1/2,-1/2)$ to find an equation for the tangent line.  
\begin{align*}
y - (-1/2) & = \Big(-1 + \frac{8}{\sqrt{3}}\Big)(x-1/2) \\
y  & = \Big(-1 + \frac{8}{\sqrt{3}}\Big)x +1/2 - \frac{4}{\sqrt{3}} - 1/2 \\
y  & = \Big(-1 + \frac{8}{\sqrt{3}}\Big)x - \frac{4}{\sqrt{3}},
\end{align*}
and this is the equation of the other tangent line.  
}%


\end{enumerate}
% end homework implicit-inverse-trig2

\end{problem}

\subsection{Derivative of non-Constant Exponent with non-Constant Base}\label{secMPSDerivativeNonConstExponent}
\begin{problem}
Differentiate
\begin{enumerate}
\item $x^{\sin x}$.
\solution{
$\displaystyle \left(x^{\sin x}\right)'=\left(e^{(\ln x)\sin x}\right)'= e^{(\ln x)\sin x} (\ln x\sin x)'=x^{\sin x}\left( \frac{\sin x}{x}+\ln x\cos x\right) $
}
\answer{$x^{\sin x}\left(\frac{\sin x}{x} +\cos x\ln x\right) $}
\item $x^{\tan x}$.
\answer{$x^{\tan x}\left(\frac{\tan x}{x}- (\ln x)\sec^2 x \right) $}
\end{enumerate}
\end{problem}
\solution{\ref{problemDifferentiatex^sinx}.
$\displaystyle \left(x^{\sin x}\right)'=\left(e^{(\ln x)\sin x}\right)'= e^{(\ln x)\sin x} (\ln x\sin x)'=x^{\sin x}\left( \frac{\sin x}{x}+\ln x\cos x\right) $.
}


\begin{problem}
Differentiate.
\begin{multicols}{2}
\begin{enumerate}
\item $10^{x^3}$. \answer{$3(\ln 10) x^{2} (10)^{x^{3}}$}
\item $2^{\tan x}$. \answer{ $(\ln 2) 2^{\tan x}  \sec^2 x $  }
\item $x^x $. \answer{$x^x(\log{}(x) +1)$}
\item $x^{x^x}$. \answer{$(\ln(x))^{2}  x^{x^{x}+x}+x^{x^{x}+x-1}+(\ln x) x^{x^{x}+x}$}
\item $(\sin x)^{\cos x}$. \answer{$\frac{- \ln(\sin{}x)  (\sin{}x)^{\cos{}x+2} +(\sin{}x)^{\cos{}x} \cos^{2}{}x}{\sin{}x}$}
\item $(\ln x)^{\ln x}$. \answer{$\ln{}(\ln{}(x)) x^{-1} (\ln{}(x))^{\ln{}(x)}+x^{-1} (\ln{}(x))^{\ln{}(x)}$}
\end{enumerate}
\end{multicols}
\end{problem}

\subsection{Related Rates}\label{secMPSrelatedRates}
\begin{problem}
\begin{enumerate}[ref={\fcProblemRef}]
% Related Rates
\item \label{problemRelatedRatesFinddr/dtIfds/dtIsKnownS-sphere} A spherical soap bubble is slowly shrinking. If its surface area is decreasing at a rate of 50 square millimeters per
second, how quickly is the radius decreasing when the surface area is 1000 square millimeters?

\answer{$\frac{\diff r}{\diff t} = - \frac{ 50}{ 8 \pi \sqrt{\frac{250}{\pi}}} =- \frac{ \sqrt{10 } }{8 \sqrt{\pi}}$.}

\item \label{problemRelatedRatesCarAlongEllipticalTrack}A car drives along an elliptical track. The track can be modeled by the equation $x^2 + 5y^2 = 14$, where $x$ and $y$ are 
measured in kilometers of distance from the center of the track. As the car passes the point $(3, 1)$, the $x$-coordinate is 
increasing at a rate of $1.5$ km/min. How quickly is the $y$-coordinate changing at that point?

\answer{$\frac{\diff y}{\diff t} = -\frac{9}{10}$ km/min.}
\item Gravel is being dumped so that the it forms a cone with circular base. The diameter of the base of the cone equals the hight of the cone at any given moment. Use related rates to approximate how fast is the height of the pile increasing when it is 2 meters tall? 
\end{enumerate}
\end{problem}
\solution{\ref{problemRelatedRatesCarAlongEllipticalTrack}. 
Let $t$ denote time, and let $t_0$ be the point in time in which the measurements take place. We are given that $\frac{\diff x}{\diff t}_{|t=t_0}=1.5 km/min$,  $x_{|t=t_0}=3$, $y_{|t=t_0}=1$. The problem asks us to find $\frac{\diff y}{\diff t}_{|t=t_0}$. 

Compute:
\[
\begin{array}{rcll|l}
\displaystyle x^2+5y^2&=&14&&\displaystyle \text{apply }\frac{\diff }{\diff t}\\
\displaystyle 2x \frac{\diff x}{\diff t}+10 y \frac{\diff y}{\diff t}&=&0 &&\displaystyle \text{fix time }t=t_0 \\
\displaystyle 2 \cdot 3 \cdot 1.5 +10 \cdot 1\cdot \frac{\diff y}{\diff t}_{|t=t_0}&=&0\\
\displaystyle 10\frac{\diff y}{\diff t}_{|t=t_0}&=&-9\\
\displaystyle \frac{\diff y}{\diff t}_{|t=t_0}&=&\displaystyle -\frac{9}{10}\quad .\\
\end{array}
\]
The measurement unit of $\diff y$ is $km$, the measurement unit of  $t$ is minutes, therefore the answer is $-\frac{9}{10}$km/min.

}
\begin{problem}
\begin{problem} (page 180)
If $V$ is the volume of a cube with edge length $x$ and the cube expands as time passes, find $\frac{dV}{dt}$ in terms of $\frac{dx}{dt}$.
\end{problem}
\begin{problem} (page 180)
Each side of a square is increasing at a rate of 6cm/s. At what rate is the area of the square increasing when the area of the square is 16 $cm^2$?
\end{problem}
\begin{problem}(page 180)
The radius of a ball is increasing at a rate of 4 mm/s. How fast is the volume increasing when the diameter is 80mm?
\end{problem}
\begin{problem}(page 181)
A street light is mounted at the top of a 4.5m tall pole. A man 180 cm tall walks away from the pole at a speed of 5km/h along a straight path. How fast is the tip of his shadow moving when he is 12m from the pole?
\end{problem}
\begin{problem}(page 181)
A boat is pulled into a dock by a rope attached to the bow of the boat and passing through a pulley on the dock that is 1m higher than the bow of the boat. If the rope is pulled in at a rate of 1m/s, how fast is the boat approaching the dock when it is 8m from the dock?
\end{problem}
\begin{problem}(page 182)
A Ferris wheel with a radius of 10m is rotating at a rate of one revolution every 2 minutes. How fast is a riding rising when his seat is 16 m above ground level?
\end{problem}
\begin{problem}(page 183)
The minute hand on a watch is 8mm long and the hour hand is 4mm long. How fast is the distance between the tips of the hands changing at one' clock?
\end{problem}
\end{problem}

\solution{
\ref{problemRelatedRatesCubeExpands}. The volume $V$ is given by $V=x^3$, therefore
\[
\frac{\diff V}{\diff t}= \frac{\diff }{\diff t}\left(x^3\right)= 3x^2\frac{\diff x}{\diff t}.
\]
}
\solution{\ref{problemRelatedRatesSquareExpands}
The area $A$ of the square is given by $A=x^2$, therefore
\[
\frac{\diff A}{\diff t}= \frac{\diff }{\diff t}\left(x^2\right)= 2x\frac{\diff x}{\diff t}.
\]
When $A=9 \text{cm}^2$, $x= 3\text{cm} $ ($=\sqrt{9\text{cm}^2}$), and so $\frac{\diff A}{\diff t}_{|x=3, \frac{\diff x}{\diff t}=1} = 2\cdot 3\text{cm} \cdot 1\text{cm/s} =6\text{cm}^2/\text{s}$.
}
\solution{\ref{problemRelatedRatesBallSurfaceAreaChanges}
Let $S$ denote the surface area and $r$ the radius of the ball. Then $S=4\pi r^2$. Let the point of time be $t_0$. We are given that $\frac{\diff S}{\diff t}_{|t=t_0}= 5 \text{cm}^2/\text{s}$. On the other hand, 

\[
\begin{array}{rcl}
\frac{\diff S}{\diff t}&=& \frac{\diff }{\diff t}\left(4\pi r^2\right)= 8\pi r \frac{\diff r}{\diff t}\\
\frac{1}{8\pi}\frac{\diff S}{\diff t}
\end{array}
\]
}
\solution{\ref{problemRelatedRatesWatchHands}
Let the angle between the two arrows be $\theta$. The cosine law states that  for a triangle with angle $\theta$ and sides $a, b, c$ we have that $c^2=a^2+b^2-2ab\cos \theta$ (where $c$ is the length of the side opposite to the angle $\theta$).

Then by the cosine law, the distance between the tips of the two hands is
1\[ 
y=\sqrt{5^2+10^2+2\cdot 5\cdot 10 \cos  \theta}=\sqrt{125+100 \cos \theta}.
\]

The short hand makes 1 full revolution every 12 hours, and the long hand makes 1 full revolution every 1 hour. Therefore the angle $\theta$ measured from the small hand to the long hand changes at the (constant) rate of $\frac{11}{12}$ revolutions per hour, or what is the same, at the rate $\frac{\diff  \theta}{\diff t} = \frac{11}{12} (2\pi)= \frac{ 11 }{ 6}\pi$.

The problem asks us to compute $\frac{d y}{d t}$ at two o'clock, i.e., at $t=2$. This is a straightforward computation using the chain rule:
\[
\begin{array}{rcll|l}
\displaystyle \frac{\diff y}{\diff t}&=&\displaystyle \frac{1}{2} \frac{-100\sin \theta}{ \sqrt{125+100\cos \theta}}\frac{\diff \theta}{\diff t}&&\text{at 2 o'clock } \theta=\frac{\pi}{3} \text{ and } \frac{\diff \theta}{\diff t}=\frac{11}{6} \pi\\~\\
\displaystyle {\frac{\diff y}{\diff t}}_{|t=2}&=&\displaystyle \frac{1}{2} \frac{-100 \sin \frac{\pi}{3} }{ \sqrt{125+100 \cos \frac{\pi}{3}}} \frac{11\pi}{6}=-\frac{55}{42}\sqrt{21} \pi \quad .
\end{array}
\]
The measurement unit of speed is mm/hour, so the distance is changing at the rate of $-\frac{55}{42}\sqrt{21}$ mm/hour.
}

\section{Graphical Behavior of Functions}
\subsection{Mean Value Theorem}\label{secMPS-MVT}

\begin{problem}
Recall that given a function $f$ satisfying certain conditions, the Mean value Theorem relates the slope of a secant line of the graph of $f$ to the slope of a tangent line to $f$ at some intermediate point(s) $c$.

Give the precise statement of the Mean Value Theorem. 

For the given functions and the given intervals, find the point(s) $c$ that satisfies the conclusion of the Mean Value Theorem.

The answer key has not been proofread, use with caution.
\begin{itemize}
\item $\displaystyle f(x)=x^2$, for the interval $[1,3]$.

\answer{$c=2$}
\item $\displaystyle f(x)=x^3-6x^2+11x-6$, for the interval $[1,3]$.

\answer{$c=2+\frac{1}{3}\sqrt{3}, 2-\frac{1}{3}\sqrt{3}$}
\item $\displaystyle f(x)=\ln x$, for the interval $[1,e]$.

\answer{$c= e-1 $}
\item $\displaystyle f(x)=\cos x$, for the interval $[0,\pi]$.

\answer{$c=\frac{\pi}{2} $}
\end{itemize}
\end{problem}

\begin{problem}
\begin{enumerate}[ref={\fcProblemRef}]
\item We have that $f$ is continuous in $[0,10]$ and differentiable in $(0,10)$. If $f(0)=0$, $f(10)=10$ and $|f'(x)|\leq 1$, how large can $f(x)$ be for $x$ in the interval $[0,10]$?

\answer{ $ 10 $}
\item We have that $f$ is continuous in $[0,10]$ and differentiable in $(0,10)$. If $f(0)=-3$, $f(2)=-13$ and $|f'(x)|\leq 5$, what is the smallest possible value of $f(1)$?

\answer{ $ -8 $}
\end{enumerate}
\end{problem}

\begin{problem}
Use the Intermediate Value theorem and the Mean Value Theorem/Rolle's Theorem to prove that the function has \textbf{exactly one} real root.
\begin{enumerate}
\item \label{problemIVTandMVTx^3+4x+7} $f(x)=x^3+4x+7$.
\item $f(x)= x^3 +x^2+x+1$.
\item \label{problemIVTandMVTcos3xdiv3+sinx-3x} $f(x)=\cos^3 \left({\frac{x}{3}}\right) +\sin x-  3x$.
\end{enumerate}

\end{problem}
\textbf{Solution \ref{problemIVTandMVTx^3+4x+7}.}  $f(-2) = -9$ and $f(1) = 12$. Since $f(x)$ is continuous and has both negative and positive outputs, it must have a zero. In other words, for some $c$ between $-2$ and $1$, $f(c) = 0$. If there were solutions $x = a$ and $x = b$,  then we would have $f(a) = f(b)$, and Rolle's Theorem would guarantee that for some $x$-value, $f'(x) = 0$. However, $f'(x) = 3x^2 + 4$, which always positive and therefore is never 0. Therefore there cannot be 2 or more solutions. 

The above can be stated informally as follows. Note that $f'(x) = 3x^2 + 4$, which is always positive. Therefore, the graph of $f$ is increasing from left to right. So once the graph crosses the $x$-axis, it can never turn around and cross again, so there can only be a single zero (that is, a single solution to $f(x) = 0$).


\textbf{Solution \ref{problemIVTandMVTcos3xdiv3+sinx-3x}.} $f(5)= \cos^3 \left( \frac{5}{ 3} \right) +\sin 5-15 \leq 2-15=-13<0 $ (because $\cos a, \sin b\in [-1,1]$ for arbitrary $a, b$). Similarly $f(-5)=\cos^3\left(-\frac{5}{3}\right) +\sin (-5)+15 \geq 15-2>0$. Therefore by the Intermediate Value Theorem $f(x)=0$ has at least one solution in the interval $[-5,5]$.

Suppose on the contrary to what we are trying to prove, $f(x)=0$ has two or more solutions; call the first 2 solutions $a,b$. That means that $f(a)=f(b)=0$, so by the Mean value theorem, there exists a $c\in (a,b)$ such that $f'(c)=(f(a)-f(b))/(a-b)=(0-0)/(a-b)=0$. On the other hand we may compute:
\[ 
f'(x)=-3+\cos x-\cos^{2}\left(\frac{x}3\right)\sin\left(\frac{x}{3}\right) \leq -1<0,
\] 
where the first inequality follows from the fact that $\sin x,\cos x\in [-1,1]$. So we got that $f'(c)=0$ for some $c$ but at the same time $f'(x)<0$ for all $x$, which is a contradiction. Therefore $f(x)=0$ has exactly one solution. 

\subsection{Maxima, Minima}\label{secMPSoneVariableMinMax}
\subsubsection{Closed Interval method}\label{secMPSclosedInterval}
\begin{problem}
Find the maximum and minimum values of $f$ on the given interval and the values of $x$ for which they are attained.
\begin{multicols}{2}
\begin{enumerate}[ref={\fcProblemRef}]
\item $\displaystyle f(x)=9+3x-x^2$, $x\in [0,4]$.

\answer{$f_{max}=f\left(\frac{3}{2}\right)=\frac{45}{4} $, $f_{min}=f\left(4\right)=5$}
\item $\displaystyle f(x)=5+4x-2x^3$, $x\in[-1,1] $.

\answer{$f_{max}=f\left(\frac{\sqrt{6}}{3} \right)= \frac{8}{9} \sqrt{6} +5  $, $f_{min} =f\left(-\frac{ \sqrt{6} }{3} \right)=5 - \frac{8 }{9}\sqrt{6} $}

\item $\displaystyle f(x)=2x^3-x^2-20x+1$, $x\in [-4,3]$.

\answer{$f_{max}=f\left( -\frac{5}{3}\right)=\frac{602}{27}$, $f_{min}=f\left(-4\right)=-63$}

\item $\displaystyle f(x)=3x^4-4x^3-12x^2+1$, $x\in [-2, 3]$.

\answer{$f_{max}=f\left(-2\right)=33$, $f_{min}=f\left(2\right)=-31 $}

\item \label{problemmaxminx^3-x^2-x+1over[-1,1]}
$f(x)=x^3-x^2-x+1,$  $x\in [-1,1]$.

\psset{xunit=2cm, yunit=2cm}
\begin{pspicture}(-1.5,-0.5)(1.5,1.5)
\tiny
\fcAxesStandard{-1.5}{-0.5}{1.5}{1.5}
\psplot[linecolor=red]{-1}{1}{x x x mul mul x x -1 mul mul -1 x mul 1 add add add}
\fcXTickWithLabel{1}{$1$}
\fcYTickWithLabel{1}{$1$}
\end{pspicture}
\item \label{problemmaxminx^3-x+1over[-2,1]}
$f(x)=x^3-x+1$,  $x\in[-2,1]$.

\answer{$f_{max}=f\left(-\frac{\sqrt{3}}{3}\right)= \frac{2}{9}\sqrt{3}+1, \quad f_{min}=f\left(-2\right)= -5 $}
\item $\displaystyle f(x)=(x^2-1)^3$, $x\in [-1, 2]$.

\answer{$f_{max}=f\left(2\right)=27$, $f_{min}=f\left(0\right)=-1$}
\item $\displaystyle f(x)=x+\frac{1}{x}$, $x\in [0.2,4 ]$.

\answer{$f_{max}=f\left(0.2\right)=\frac{26}{5}=5.2$, $f_{min}=f\left(1\right)=2$}
\item $\displaystyle f(x)=\frac{x}{x^2-x+1}$, $x\in [0,3 ]$.

\answer{$f_{max}=f\left(1\right)=1$, $f_{min}=f\left(-1\right)=-\frac{1}{3}$}
\item $\displaystyle f(t)=t\sqrt{4-t^2}$, $x\in [-1,2 ]$.

\answer{$f_{max}=f\left(\sqrt{2}\right)=2$, $f_{min}=f\left(-\sqrt{2}\right)=-2$}
\item $\displaystyle f(t)=\sqrt[3]{t}(8-t) $, $x\in [0,8 ]$.

\answer{$f_{max}=f\left(2\right)=6\sqrt[3]{2}$, $f_{min}=f\left(0\right)=f(8)=0$}
\item $\displaystyle f(t)=2\cos t+\sin (2t)$, $x\in [0,\frac{\pi}{2} ]$.

\answer{$f_{max}=f\left(\frac{\pi}{6}\right)=\frac{3}{2}\sqrt{3}$, $f_{min}=f\left(\frac{\pi}{2}\right)=0$}
\item $\displaystyle f(t)=t+\cot \left(\frac{t}{2}\right) $, $x\in [\frac{\pi}{4},\frac{7\pi}{4} ]$.

\answer{$f_{max}=f\left(\frac{3\pi}{2}\right)=\frac{3\pi}{2}-1$, $f_{min}=f\left(\frac{\pi}{2}\right)=\frac{\pi}{2}+1$}

\item $\displaystyle f(t)=t+\cot \left(\frac{t}{2}\right) $, $x\in [\frac{\pi}{4},\frac{7\pi}{4} ]$.
\item $\displaystyle f(x)=x e^{3 x}$, $x\in \left[-3, \frac{1}{6}\right]$.

\answer{$f_{max}=f\left( \frac{1}{6}\right)=\frac{e^{\frac{1}{2}}}{6}\approx 0.274787 $, $f_{min}=f\left( -\frac{1}{3}\right)= -\frac{1}{3e}\approx -0.122626 $}
\item $\displaystyle f(x)=\left(x-2\right) \left(x+1\right) e^{x} $, $x\in \left[-5,2\right]$.

\answer{$\begin{array}{l}
f_{max}=f\left(-\frac{\sqrt{13}}{2}-\frac{1}{2}
\right)= \left(\sqrt{13}+2\right) e^{\left(-\frac{\sqrt{13}}{2}-\frac{1}{2}\right)}\approx 0.560448\\
f_{min}=f\left(\frac{\sqrt{13}}{2}-\frac{1}{2} \right)= \left(-\sqrt{13}+2\right) e^{\left(\frac{\sqrt{13}}{2}-\frac{1}{2}\right)} \approx -5.907619
\end{array}
$}
\item $\displaystyle f(x)=$, $x\in \left[-3,3\right]$.

\answer{$\begin{array}{rcl}
f_{max}&=&f\left( \frac{\sqrt{3}}{2}-\frac{1}{2}
\right) =\left(\frac{\sqrt{3}}{2}+\frac{1}{2}\right) e^{\frac{\sqrt{3}}{2}-1}\approx 1.194743 \\
f_{min}&=&f\left(-\frac{\sqrt{3}}{2}-\frac{1}{2} \right)=\left(-\frac{\sqrt{3}}{2}+\frac{1}{2}\right) e^{-\frac{\sqrt{3}}{2}-1}\approx -0.056638 \end{array}$}
\item \label{problemmaxminxe^(2x)over[-2,1/2]}
$f(x)=x e^{2x}$, $x\in\left[ -2,\frac{1}{2}\right]$.

\answer{$ f_{max}=f\left(\frac{1}{2}\right)= \frac{e}{2}$, $f_{min}=f\left(-\frac{1}{2}\right)=-\frac{1}{2e}$}
\end{enumerate}
\end{multicols}

\end{problem}

\begin{problem}
% begin homework logarithm-physics
A particle moves in such a way that, after $t$ seconds, it is $s(t) = \ln (2-t+t^2)$ m to the right of the origin.  
\begin{enumerate}
\item  What is the closest it comes to the origin?  

\solution{%
\begin{align*}
s'(t) & = \frac{-1+2t}{2-t+t^2}. \\
\text{Set } \quad s'(t) & = 0. \\
\frac{-1+2t}{2-t+t^2} & = 0 \\
-1+2t & = 0 \\
t & = 1/2.
\end{align*}

Therefore the position function has a critical number at $t = 1/2$.  
The parabola $2-t+t^2$ has a global minimum at $t = 1/2$, and the natural logarithm function is an increasing function, so $\ln(2-t+t^2)$ also has a global minimum at $t=1/2$.  
The minimum value is $s(1/2) = \ln(2-1/2+(1/2)^2) = \ln(7/4)\approx 0.5596$ m.  
}%

\item  What is its acceleration when it is closest to the origin?  

\solution{%
\begin{align*}
v(t) = s'(t) & = \frac{-1+2t}{2-t+t^2}. \\
a(t) = s''(t) & = \frac{(2-t+t^2)(2) - (-1+2t)(-1+2t)}{(2-t+t^2)^2} \\
 & = \frac{(4-2t+2t^2) - (1-4t+4t^2)}{(2-t+t^2)^2} \\
 & = \frac{3+2t-2t^2}{(2-t+t^2)^2}. \\
\text{Plug in $t = 1/2$:} \quad a(1/2) & = \frac{3+2(1/2)-2(1/2)^2}{(7/4)^2} \\
 & = \frac{3+1-1/2}{49/16} \\
 & = \frac{7/2}{49/16} \\
 & = \frac{8}{7}.
\end{align*}
Therefore the particle is accelerating at a rate of $8/7\approx 1.1429$ m/s$^2$ when it is closest to the origin.  
}%

\item  For which values of $t$ is the position function $s(t)$ defined?  

\solution{%
The natural logarithm function is defined for all positive input values.  
The formula $y = 2-t+t^2$ is always positive.  
To see this, note that its graph is a parabola with discriminant $b^2-4ac = (-1)^2-4(1)(2) = -7$, which is negative.  
This means that the graph of $y = 2-t+t^2$ never touches the $x$-axis, and hence it is either always positive or always negative.  
Since $y(0) = 2$ is positive, this means all values are positive.  
Therefore $\ln(2-t+t^2)$ is defined for all input values $t$.  
}%

\end{enumerate}
% end homework logarithm-physics

\end{problem}

\subsubsection{Derivative tests}
\begin{problem}
Find the critical points of the function. Identify whether those are local maxima, minima, or neither. The answer key has not been proofread, use with caution.
\begin{multicols}{2}
\begin{enumerate}[ref={\fcProblemRef}]
\item $\displaystyle f(x)=\frac{x}{1+x^2}$.

\answer{$x=1 $, local \& global max, $x=-1$, local and global min}
\item $\displaystyle f(x)=x^3-x^2-x-1$.

\answer{$ x=-\frac{1}{3} $, local max, $x=1$, local min}
\item $\displaystyle f(x)=2x^3-x^2-20x+1$.

\answer{$x=-\frac{5}{3}$, local max $x=2 $, local min}
\item $\displaystyle f(x)=x+\frac{1}{x}$.

\answer{$x=-1$, local max, $x=1$, local min}
\item $\displaystyle f(x)=\frac{x-\frac{1}{2}}{x^{2}-2 x+\frac{7}{4}} $.

\answer{$x=-\frac{1}{2}$, local and global min, $x=\frac{3}{2}$, local and global max}
\end{enumerate}
\end{multicols}

\end{problem}

\subsubsection{Optimization}\label{secMPSoptimization}
\begin{problem}

\begin{enumerate}
\item Find the dimensions of a rectangle with area 1000 $m^2$ whose perimeter is as small as possible.
\item A box with an open top is to be constructed from a square piece of cardboard, 1m wide, by cutting out a square from each of the four corners and bending up the sides. Find the largest volume that such a box can have.
\item A right circular cylinder is inscribed in a sphere of radius $r$. Find the largest possible volume of such a cylinder.
\item A wedge of radius $2$ (depicted below) is folded into a cone cup. The volume varies depending on the angle of the wedge. Find the maximal possible volume of the cone cup and the angle of the wedge for which this maximal volume is achieved.
\psset{xunit=0.5cm, yunit=0.5cm}
\begin{pspicture}(-1.5, -1.5)(1.5,1.5) 
\tiny 
\pscustom*[linecolor=cyan!30]{ \psparametricplot[algebraic] {2.35619}{7.06858} {0+1*cos(t)| 0+1*sin(t)} \psline(0.707107, 0.707107)(0, 0)(-0.707107, 0.707107)}

\psparametricplot[algebraic,linecolor=blue]{2.35619}{7.06858}{cos(t)| sin(t)} 
\psline[linecolor=red](0.707107, 0.707107)(0, 0)(-0.707107, 0.707107)

\rput[t](0.4, 0.2){$r$}
\rput[lb](0.8,0.8){$B$}
\rput[rb](-0.8,0.8){$A$}
\rput[b](0,0.3){$O$}
\end{pspicture} 

\end{enumerate}

\end{problem}
\begin{problem}
\begin{enumerate}[ref={\fcProblemRef}]
% Optimization
\item \label{problemponthyperbolax^2-4y^2closestTo1,1} What is the $x$-coordinate of the point on the hyperbola $x^2 - 4y^2 = 16$ that is closest to the point $(1, 0)$?

\psset{xunit=0.3cm, yunit=0.3cm}
\begin{pspicture}(-10.500000, -5)(10.500000,5)
\psframe*[linecolor=white](-10.500000,-5)(10.500000,5)
\tiny
\fcAxesStandard{-10.000000}{-4.5}{10.000000}{4.5} %Function formula: - (1/4 x^{2}-4)^{1/2}
\psplot[linecolor=\fcColorGraph, plotpoints=1000]{-10.000000}{-4.000000}{-4 x 2 exp 0.25 mul add 0.5 exp -1 mul }
%Function formula: (1/4 x^{2}-4)^{1/2}
\psplot[linecolor=\fcColorGraph, plotpoints=1000]{-10.000000}{-4.000000}{-4 x 2 exp 0.25 mul add 0.5 exp }
%Function formula: - (1/4 x^{2}-4)^{1/2}
\psplot[linecolor=\fcColorGraph, plotpoints=1000]{4.000000}{10.000000}{-4 x 2 exp 0.25 mul add 0.5 exp -1 mul }
%Function formula: (1/4 x^{2}-4)^{1/2}
\psplot[linecolor=\fcColorGraph, plotpoints=1000]{4.000000}{10.000000}{-4 x 2 exp 0.25 mul add 0.5 exp }
\fcFullDot{1}{0}
\fcFullDot{4}{0}
\pscircle[linestyle=dotted](1,0){0.9}
\end{pspicture}

\answer{$x = 4$}

\item What is the $x$-coordinate of the point on the ellipse $x^2+4y^2=16$ closest to the point $(1,0)$?

\psset{xunit=0.3cm, yunit=0.3cm}
\begin{pspicture}(-4.500000, -5)(4.500000,5)
\psframe*[linecolor=white](-4.500000,-5)(4.500000,5)
\tiny
\psline[linecolor=red!1](-10,0)(-9.99,0)
\fcAxesStandard{-4.000000}{-4.5}{4.000000}{4.5} %Function formula: - (-1/4 x^{2}+4)^{1/2}
\psplot[linecolor=\fcColorGraph, plotpoints=1000]{-4.000000}{4.000000}{4 x 2 exp -0.25 mul add 0.5 exp -1 mul }
%Function formula: (-1/4 x^{2}+4)^{1/2}
\psplot[linecolor=\fcColorGraph, plotpoints=1000]{-4.000000}{4.000000}{4 x 2 exp -0.25 mul add 0.5 exp }
\pscircle[linestyle=dotted](1,0){0.574456}
\fcFullDot{1}{0}
\fcFullDot{1.333333}{1.885618}
\fcFullDot{1.333333}{-1.885618}
\end{pspicture}
\answer{$x=\frac43$}
\item \label{problemMaxVolumeBoxFixedAreaDoubleBottomNoLid} You want to build a rectangular box with a square base out of sheet metal. You are going to use 2 pieces of sheet metal for the bottom of the box to reinforce it, and only a single piece of sheet metal for all of the sides and the top. If you want to use no more than $36$ sq. ft. of material, what is the largest possible volume you can enclose?

\answer{12 cubic feet.}
\end{enumerate}

\end{problem}
\solution{\ref{problemponthyperbolax^2-4y^2closestTo1,1}

The distance function between an arbitrary point $(x,y)$ and the point $(1,0)$ is $d=\sqrt{(x-1)^2+(y-0)^2}$. On the other hand, when the point $(x,y)$ lies on the hyperbola we have $y^2= \frac{x^2 -16 }{4 }$. In this way, the problem becomes that of minimizing the distance function

\[
dist(x)=\sqrt{(x-1)^2+y^2}=\sqrt{(x-1)^2+\frac{x^2-16}{4}} \quad .
\]
This is a standard optimization problem: we need to find the critical endpoints, i.e., the points where $dist'=0$. As the square root function is an increasing function, the function $\displaystyle \sqrt{(x-1)^2+\frac{x^2-16}{4}}$ achieves its minimum when the function 
\[
l=dist^2=(x-1)^2+\frac{x^2-16}{4}\quad 
\]
does. $l$ is a quadratic function of $x$ and we can directly determine its minimimum via elementary methods. Alternatively, we find the critical points of $l$:
\[
\begin{array}{rcl}
\displaystyle l'&=&\displaystyle 0\\
\displaystyle 2(x-1)+\frac{x}{2} &=&\displaystyle 0\\
\displaystyle \frac{5}{2}x-2&=&0\\
\displaystyle x&=&\displaystyle \frac{4}{5}\quad .
\end{array}
\]
On the other hand, $x^2=16+4y^2$ and therefore $|x|\geq \sqrt{16} = 4$. Therefore $x\in (-\infty, -4]\cup [4,\infty)$. As $x= \frac{4}{5 }$ is outside of the allowed range, it follows that our either function attains its minimum at one of the endpoints $\pm 4$ or the function has no minimum at all. It is clear however that as $x$ tends to  $\infty$, so does $dist$. Therefore $dist$ attains its maximum for $x=4$ or $-4$ and $y=\pm\sqrt{(\pm4)^2-16}=0$. Direct check shows that $dist_{|x=4} =\sqrt{(4-1)^2 +\frac{4^2- 16}{4 }}=3$ and $dist_{|x=-4}=\sqrt{(-4-1)^2+\frac{4^2-16}{4}}=5$  so our function $dist$ has a minimal value of $3$ achieved when $x=4$, which is our final answer. Notice that this answer can be immediately given without computation by looking at the figure drawn for \ref{problemponthyperbolax^2-4y^2closestTo1,1}. Indeed, it is clear that there are no points from the hyperbola lying inside the dotted circle centered at $(1,0)$. Therefore the point where this circle touches the hyperbola must have the shortest distance to the center of the circle.
}

\solution{\ref{problemMaxVolumeBoxFixedAreaDoubleBottomNoLid}
Let $B$ denote the area of the base of the box, equal to the area of the top. Let $W$ denote the area of the four walls of the box (the four walls are all equal because the base of the box is a square). Then the surface area $S$ of material used will be 
\[
S=\underbrace{2B}_{\text{two pieces for the bottom}}+\underbrace{4W}_{4 \text{ walls}} +\underbrace{B}_{\text{the box lid}}=3B+4W\quad.
\] 
Let $x$ denote the length of the side of the square base and let $y$ denote the height of the box.  Then  
\[
B=x^2
\]
and 
\[
W=xy\quad .
\]
As the surface area $S$ is fixed to be $36$ square feet, we have that
\[
S=3B+4W=36= 3x^2 + 4xy\quad .
\]
As $y$ is positive, the above formula shows that $3x^2\leq 36$ and so $x\leq \sqrt{12}$. Let us now express $y$ in terms of $x$:
\[
\begin{array}{rcl}
3x^2+4xy&=&36\\
4xy&=&36-x^2\\
y&=&\displaystyle\frac{36-x^2}{4x}\quad .
\end{array}
\]
The problem asks us to maximize the volume $V$ of the box. The volume of the box equals the area of the base times the height of the box: 
\[
V=B\cdot y=yx^2 = \frac{(36-3x^2)}{4x}x^2=\frac{36x-3x^3}{4}\quad .
\]
As $x$ is non-negative, it follows that the domain for $x$ is:
\[
x\in [0, \sqrt{12}]\quad .
\]
To maximize the volume we find the critical points, i.e., the values of $x$ for which  $V'$ vanishes:

\[\begin{array}{rcl}
0&=&V' = \displaystyle\left(\frac{36x-3x^3}{4}\right)'\\
0&=&\displaystyle \frac{36- 9x^2}{4}\\
9x^2&=&36\\
x^2&=&4\\
x&=&\pm 2
\end{array}
\]
As $x$ measures length, $x=-2$ is not possible (outside of the domain for $x$). Therefore the only critical point is $x=2$. Direct check shows that at the endpoints $x=0$ and $x=\sqrt{12}$, we have that $V=0$. Therefore the maximal volume is achieved when $x=2$:
 
\[
V_{max}=V_{|x=2}= \frac{36(2)-3(2)^3}{4} =12\quad .
\]
 
}



\subsection{Function Graph Sketching}\label{secMPSfunctionGraphSketching}
\begin{problem}

Find the
\begin{multicols}{2}
\begin{itemize}
\item the implied domain of $f$,
\item $x$ and $y$ intercepts of $f$,
\item horizontal and vertical asymptotes,
\item intervals of increase and decrease,
\item local and global minima, maxima,
\item intervals of concavity,
\item points of inflection.
\end{itemize}
\end{multicols}
Label all relevant points on the graph. Show all of your computations.
\begin{enumerate}[ref={\fcProblemRef}]
\item $\displaystyle f(x)=\frac{x+\frac 1 2}{x^{2}+x+1}$

\psset{xunit=1cm, yunit=1cm}
\begin{pspicture}(-5, -1)(5,2)
\psframe*[linecolor=white](-5,-1)(5,2)
\tiny
\psaxes[ticks=none, labels=none]{<->}(0,0)(-5,-0.5)(5,1.5)
\fcLabels{5}{1.5}
%Function formula: \frac{x+1/2}{x^{2}+x+1}
\psplot[linecolor=\fcColorGraph, plotpoints=1000]{-5}{5}{0.5 x add 1 x add x 2 exp add div }
\end{pspicture}

\answer{
\begin{tabular}{l}
$y$-intercept: $\frac12$. $x$-intercept: $-\frac12$\\
Horizontal asymptote: $y=0$, vertical: none \\
local and global min at $x=\frac{ -1-\sqrt{3}}{2}$, local and global max at $x=\frac{ -1+\sqrt{3}}{2}$\\
Intervals of decrease: $ \left(-\infty, \frac{-1 -\sqrt{3} }{2}\right)\cup \left(\frac{-1 +\sqrt{3} }{2}, \infty\right) $, intervals of decrease $\left( \frac{ -1-\sqrt{3}}{2}, \frac{-1+ \sqrt{3}}{2}\right)$ \\
Concave down on $(-\infty, -2)\cup \left(-\frac12, 1\right)$, concave up on $\left(-2, -\frac12\right)\cup (1,\infty)$\\
Inflection points at: $x=-2$, $x= -\frac12$, $x=1 $ \\
\end{tabular}
}

\item \label{problemSketchCurve(2x^2-5x+9/2)/(x^2-3 x+3)} $\displaystyle f(x)=\frac{2 x^{2}-5 x+\frac{9}{2}}{x^{2}-3 x+3}$. \textbf{For this problem, indicate only the $x$-coordinates of the local maxima/minima and inflection points; you do not need to compute the $y$-coordinates of those points.} 

Computation shows that 
$\displaystyle
f'(x)=\frac{- x^{2}+3 x-\frac{3}{2} }{ \left(x^2- 3 x+3\right)^2}
$
and that 
$\displaystyle f''(x)=\frac{(2x-3)x(x-3)}{\left(x^2- 3 x+3\right)^3}$; you may use those computations without further justification. 
 
\psset{xunit=0.6cm, yunit=0.6cm}
\begin{pspicture}(-5, -1)(5,4)
\psframe*[linecolor=white](-5,-1)(5,4)
\tiny
\psaxes[ticks=none, labels=none]{<->}(0,0) (-5,-0.5) (5, 3.5)
\fcLabels{5}{3.5}
%Function formula: \frac{2 x^{2}-5 x+9/2}{x^{2}-3 x+3}
\psplot[linecolor=\fcColorGraph, plotpoints=1000]{-5}{5 } {4.5 x -5 mul add x 2 exp 2 mul add 3 x -3 mul add x 2 exp add div }
\end{pspicture}

\answer{
\begin{tabular}{l}
$y$-intercept: $\frac32$\\
horizontal asymptote: $y=2$, vertical: none\\
increasing on
$\left(\frac{3-\sqrt{3}}2, \frac{3+ \sqrt{3}}2 \right) $, decreasing on $\left(-\infty, \frac{3-\sqrt{3}}2\right)\cup \left(\frac{3+\sqrt{3}}2, \infty\right) $\\
local and global min at $x=\frac{3-\sqrt{3}}2$, local and global max at $x=\frac{3+\sqrt{3}}2$\\
concave up on $\left(0, \frac32\right)\cup \left(3, \infty \right)$, concave down $\left(-\infty, 0\right)\cup \left(\frac32, 3\right)$\\
inflection points at $x=0,x=\frac32, x=3$
\end{tabular}
}


\item $\displaystyle f(x)=\frac{2 \sqrt{- x^{2}+1}+ 1} {\sqrt{- x^{2}+1}+1}$,  $\displaystyle f(x)=\frac{1}{\sqrt{- x^{2} +1}+1}$

\psset{xunit=1cm, yunit=1cm}
\begin{pspicture}(-1, -1)(1.2,3)
\psframe*[linecolor=white](-1,-5)(1,5)
\tiny
\psaxes[ticks=none, labels=none]{<->}(0,0)(-1,-0.5)(1,2.5)
\fcLabels{1}{2.5}
%Function formula: \frac{2 (- x^{2}+1)^{1/2}+1}{(- x^{2}+1)^{1/2}+1}
\rput(1,3){}
\psplot[linecolor=brown, plotpoints=1000]{-1}{1}{1 1 x 2 exp -1 mul add 0.5 exp 2 mul add 1 1 x 2 exp -1 mul add 0.5 exp add div }
%Function formula: \frac{1}{(- x^{2}+1)^{1/2}+1}
\rput(1,3){}
\psplot[linecolor=\fcColorGraph, plotpoints=1000]{-1}{1}{1 1 1 x 2 exp -1 mul add 0.5 exp add div }
\end{pspicture}

The two functions are plotted simultaneously in the $x,y$-plane. Indicate which part of the graph is the graph of which function.

\answer{
\begin{tabular}{l}
For $f(x)=\frac{2 \sqrt{- x^{2}+1}+1}{ \sqrt{- x^{2}+1}+1}$: \\
$y$-intercept: $x=\frac{3}2$, no $x$ intercept\\
no asymptotes\\
increasing on $[-1, 0]$, decreasing on $[0, 1]$ \\
global and local max at $x=0$, global and local min at $x=\pm 1$.\\
concave down on $[-1,1]$\\
no inflection points
\end{tabular}
}
\answer{
\begin{tabular}{l}
For $f(x)=\frac{1}{\sqrt{- x^{2}+1}+1}$: \\
$y$-intercept: $x=\frac{1}2$, no $x$ intercept\\
no asymptotes\\
decreasing on $[-1, 0]$, increasing on $[0, 1]$ \\
global and local min at $x=0$, global and local max at $x=\pm 1$.\\
concave up on $[-1,1]$\\
no inflection points
\end{tabular}
}
\item $\displaystyle f(x)=\frac{e^x+e^{-x}}{e^x-e^{-x}}$

\psset{xunit=0.5cm, yunit=0.5cm}
\begin{pspicture}(-4, -5)(4,5)
\psframe*[linecolor=white](-4,-5)(4,5)
\tiny
\psaxes[ticks=none, labels=none]{<->}(0,0)(-4,-4.5)(4,4.5)
\fcLabels{4}{5}
%Function formula: \frac{e^{- x}+e^{x}}{- e^{- x}+e^{x}}
\psplot[linecolor=\fcColorGraph, plotpoints=1000]{0.2}{4}{2.718281828 x exp 2.718281828 x -1 mul exp add 2.718281828 x exp 2.718281828 x -1 mul exp -1 mul add div }
%Function formula: \frac{e^{- x}+e^{x}}{- e^{- x}+e^{x}}
\psplot[linecolor=\fcColorGraph, plotpoints=1000]{-4}{-0.2}{2.718281828 x exp 2.718281828 x -1 mul exp add 2.718281828 x exp 2.718281828 x -1 mul exp -1 mul add div }
\end{pspicture}
\item $\displaystyle f(x)=\frac{- e^{- x}+e^{x}}{e^{- x}+e^{x}}$

\psset{xunit=1cm, yunit=1cm}
\begin{pspicture}(-4, -1.5)(4,1.5)
\psframe*[linecolor=white](-4,-1.5)(4,1.5)
\tiny
\psaxes[ticks=none, labels=none]{<->}(0,0)(-4,-1.1)(4,1.1)
\fcLabels{4}{1.1}
%Function formula: \frac{- e^{- x}+e^{x}}{e^{- x}+e^{x}}
\psplot[linecolor=\fcColorGraph, plotpoints=1000]{-4}{4}{2.718281828 x exp 2.718281828 x -1 mul exp -1 mul add 2.718281828 x exp 2.718281828 x -1 mul exp add div }
\end{pspicture}
\item $\displaystyle f(x)=\ln{}\left(\frac{{{x}}+1}{- {{x}}+1}\right)$

\psset{xunit=1cm, yunit=1cm}
\begin{pspicture}(-1, -4.2)(1,4.2)
\psframe*[linecolor=white](-1,-5)(1,5)
\tiny
\psaxes[ticks=none, labels=none]{<->}(0,0)(-1.3,-4)(1.3,4)
\fcLabels{1.3}{4}
%Function formula: \log{}(\frac{x+1}{- x+1})
\psplot[linecolor=\fcColorGraph, plotpoints=1000]{-0.94}{0.94}{1 x add 1 x -1 mul add div ln }

\end{pspicture}

\item $\displaystyle f(x)=\frac{x^{2}+3 x+1}{x^{2}+2 x}$. \textbf{For this problem, indicate only the $x$-coordinates of the local maxima/minima and inflection points; you do not need to compute the $y$-coordinates of those points.} 

Computation shows that 
$\displaystyle
f'(x)= \frac{- x^{2}-2 x-2}{\left(x^{2}+2 x\right)^{2}} 
$
and that 
$\displaystyle f''(x)= \frac{2 x^{3}+6 x^{2}+12 x+8}{\left(x^{2}+2 x\right)^{3}} =\frac{\left(x+1\right) \left(2 x^{2}+4 x+8\right)}{\left(x^{2}+2 x\right)^{3}} $; you may use those computations without further justification. 

\psset{xunit=0.7cm, yunit=0.7cm}
\begin{pspicture}(-5, -4.7)(5,5)
\psframe*[linecolor=white](-5,-4.7)(5,5)
\tiny
\psaxes[ticks=none, labels=none]{<->}(0,0)(-5,-4.5)(5,4.5)
\fcLabels{5}{5}
%Function formula: \frac{x^{2}+3 x+1}{x^{2}+2 x}
\psplot[linecolor=\fcColorGraph, plotpoints=1000]{0.1}{5}{1 x 3 mul add x 2 exp add x 2 mul x 2 exp add div }
%Function formula: \frac{x^{2}+3 x+1}{x^{2}+2 x}
\psplot[linecolor=\fcColorGraph, plotpoints=1000]{-1.9}{-0.1}{1 x 3 mul add x 2 exp add x 2 mul x 2 exp add div }
%Function formula: \frac{x^{2}+3 x+1}{x^{2}+2 x}
\psplot[linecolor=\fcColorGraph, plotpoints=1000]{-5}{-2.1}{1 x 3 mul add x 2 exp add x 2 mul x 2 exp add div }
\end{pspicture}

\answer{
\begin{tabular}{l}
$y$-intercept: none, $x$-intercepts: $\frac{-3\mp\sqrt{5}}2$ \\
horizontal asymptote: $y=1$, vertical: $x=-2$ and $x=0$\\
always decreasing\\
no local/global minima/maxima\\
concave down on $\left(-\infty,-2\right)\cup \left(-1,0 \right)$, concave up on $\left(-2, -1\right)\cup \left(0, \infty\right)$\\
inflection point at $x=-1$
\end{tabular}
}


\item \label{problemSketch(x+1)/(x^2+2x+4)} $\displaystyle f(x)=\frac{x+1}{x^2+2x+4}$
\psset{xunit=1.4cm, yunit=1.4cm}
\begin{pspicture}(-5, -1.2)(3.4,1.2)
\psframe*[linecolor=white](-4.5,-1)(4.5,1)
\tiny
\fcAxesStandard{-5}{-1}{3}{1} %Function formula: \frac{x+1}{x^{2}+2 x+4}
\psplot[linecolor=\fcColorGraph, plotpoints=1000]{-5}{3}{1 x add 4 x 2 mul add x 2 exp add div }
\end{pspicture}

\answer{
\begin{tabular}{l}
$y$-intercept: $\frac14$, $x$-intercept: $-1$\\
horizontal asymptote: $y=0$, vertical: none\\
increasing on
$\left(-1-\sqrt{3}, -1+\sqrt{3}  \right) $, decreasing on $\left(-\infty, -1-\sqrt{3}\right)\cup \left(-1+\sqrt{3}, \infty\right) $\\
local and global min at $x=-1-\sqrt{3}$, local and global max at $x=-1+\sqrt{3}$\\
concave up on $\left(-4, -1\right)\cup \left(2, \infty \right)$, concave down $\left(-\infty, -4\right)\cup \left(-1, 2\right)$\\
inflection points at $x=-4,x=-1, x=2$
\end{tabular}
}
\item $\displaystyle f(x)= \frac{3 x^{3}-30 x^{2}+97 x-99}{x^{2}-6 x+8} $. \textbf{For this problem, do not find the $x$-intercepts of the function. Indicate only the $x$-coordinates of the local maxima/minima and inflection points; you do not need to compute the $y$-coordinates of those points.} 

Computation shows that 
$\displaystyle
f'(x)=\frac{3 x^{4}-36 x^{3}+155 x^{2}-282 x+182}{\left(x^{2}-6 x+8\right)^{2}}=\frac{\left(x^{2}-6 x+7\right) \left(3 x^{2}-18 x+26\right)}{\left(x^{2}-6 x+8\right)^{2}}
$
and that 
$\displaystyle f''(x)=\frac{2 x^{3}-18 x^{2}+60 x-72}{\left(x^{2}-6 x+8\right)^{3}}=\frac{\left(x-3\right) \left(2 x^{2}-12 x+24\right)}{\left(x^{2}-6 x+8\right)^{3}}$; you may use those computations without further justification. 

\psset{xunit=0.15cm, yunit=0.15cm}
\begin{pspicture}(-11,-21)(12,11)
\fcAxesStandard{-10}{-20}{10}{10}
\newcommand{\theFun}{3 x x x mul mul mul -30 x x mul mul 97 x mul -99 add add add x x mul -6 x mul 8 add add div}
\psplot[plotpoints=500, linecolor=\fcColorGraph]{-7}{1.97}{\theFun}
\psplot[plotpoints=500, linecolor=\fcColorGraph]{2.02}{3.98}{\theFun}
\psplot[plotpoints=500, linecolor=\fcColorGraph]{4.03}{10}{\theFun}
\end{pspicture}



\answer{
\begin{tabular}{l}
$y$-intercept $\frac{-99}{8}$, $x$-intercept: not requested \\
horizontal asymptote: none, vertical: $x=2$ and $x=4$\\
increasing on $\left(- \infty, -\sqrt{2}+3\right)\cup \left(-\frac{\sqrt{3}}{3}+3, \frac{\sqrt{3}}{3}+3\right)\cup \left(\sqrt{2}+3, \infty\right)
$, decreasing on the complement intervals\\
local maxima at $x=-\sqrt{2}+3$, $ x=\frac{\sqrt{3}}{3}+3$\\
local minima at $x=-\frac{\sqrt{3}}{3}+3$, $ x=\sqrt{2}+3$\\
concave up on $\left(\left(2, 3\right)\cup \left(4, \infty\right)\right)$\\ 
concave down on $\left(-\infty, 2\right)\cup \left(3, 4\right)$\\
inflection point at $x=3$
\end{tabular}
}

\end{enumerate}

\end{problem}
\solution{\ref{problemSketchCurve(2x^2-5x+9/2)/(x^2-3 x+3)}

\textbf{Domain.} We have that $f$ is not defined only when we have division by zero, i.e.,  if $x^2-3x+3$ equals zero. However, the roots of $x^{2}-3x+3$ are not real numbers: they are $\frac{3\pm \sqrt{3^2-4\cdot 3 }}{2}= \frac{3\pm \sqrt{-3}}{2}$, and therefore $x^2-3x+3$ can never equal zero. Alternatively, completing the square shows that the denominator is always positive:
\[
x^2-3x+3=x^2-2\cdot \frac{3}{2} x+\frac{9}{4}-\frac{9}{4}+3=\left(x-\frac{3}{2}\right)^2+\frac{3}{4} >0 
\]
Therefore the domain of $f$ is all real numbers.

\textbf{$x$, $y$-intercepts.}  The $y$-intercept of $f$ equals by definition $\displaystyle f(0)= \frac{ 2\cdot 0^2-5\cdot 0+ \frac{9}{2}}{0^2-3\cdot 0 + 3}=\frac{\frac{9}{2}}{3}= \frac{3}{2}$. The $x$ intercept of $f$ is those values of $x$ for which $f(x)=0$. The graph of $f$ shows no such $x$, and that is confirmed by solving the equation $f(x)=0$:

\[
\begin{array}{rcll|l}
f(x)&=&0\\
\displaystyle \frac{2x^2-5x+\frac{9}{2}}{x^2-3 x+3}&=&0&&\text{Mult. by }x^2-3 x+3\\
\displaystyle 2x^2-5x+\frac{9}{2}&=&0\\
\displaystyle x_1, x_2&=&\displaystyle \frac{5 \pm \sqrt{25- 4\cdot 2\cdot \frac{9}{2}}}{4}=\frac{5\pm \sqrt{-9}}{4}\quad ,
\end{array}
\]
so there are no real solutions (the number $\sqrt{-9}$ is not real).

\textbf{Asymptotes.} Since $f$ is defined for all real numbers, its graph has no vertical asymptotes. To find the horizontal asymptote(s), we need to compute the limits $\lim\limits_{x\to \infty } f(x)$ and $\lim\limits_{x\to -\infty} f (x)$. The two limits are equal, as the direct computation below shows:
\[
\begin{array}{rcll|l}
\displaystyle \lim_{x\to \pm\infty} \frac{2x^2-5x+\frac{9}{2}}{x^2-3 x+3}&=& \displaystyle  \lim_{x\to \pm\infty}\frac{\left(2x^2-5x+ \frac{9}{2}\right)\frac{1}{x^2}}{\left(x^2-3 x+3\right)\frac{1}{x^2}} &&\begin{array}{l}\text{Divide by leading}\\ \text{monomial in denominator}\end{array}\\
&=&\displaystyle\lim_{x\to \pm \infty}\frac{2-\frac{5}{x} +\frac{9}{2x^2}}{1-\frac{3}{x}+\frac{3}{x^2}}\\
&=&\displaystyle \frac{2-0+0}{1-0+0}\\
&=& 2
\end{array}
\]
Therefore the graph of $f(x)$ has a single horizontal asymptote at $y=2$.

\textbf{Intervals of increase and decrease.}
The intervals of increase and decrease of $f$ are governed by the sign of $f'$. We compute:

\[
\begin{array}{rcl}
f'(x)&=&\displaystyle \left(\frac{2x^2-5x+\frac{9}{2} }{x^2- 3 x+3} \right)' \\
&=&\displaystyle \frac{\left(2x^2-5x+\frac{9}{2}\right)'\left(x^2- 3 x+3\right)-\left(2x^2-5x+\frac{9}{2}\right)\left(x^2- 3 x+3\right)' }{ \left(x^2- 3 x+3\right)^2}\\
&=&\displaystyle \frac{- x^{2}+3 x-\frac{3}{2} }{ \left(x^2- 3 x+3\right)^2}
\end{array}
\]
As the denominator is a square, the sign of $f'$ is governed by the sign of $- x^{2}+3 x-\frac{3}{2}$. To find where $- x^{2}+3 x-\frac{3}{2}$ changes sign, we compute the zeroes of this expression:

\[
\begin{array}{rcll|l}
\displaystyle - x^{2}+3 x-\frac{3}{2}&=&0&& \text{Mult. by }-2\\
\displaystyle  2x^{2}-6 x+3&=&0\\
x_1, x_2&=&\displaystyle \frac{ 6\pm \sqrt{36-24 }}{4}=\frac{6\pm \sqrt{12}}{4}\\
x_1, x_2&=&\displaystyle \frac{3\pm \sqrt{3}}{2} 
\end{array}
\]
Therefore the quadratic $- x^{2}+3 x-\frac{3}{2}$ factors as 
\begin{equation}
\label{eq1problemSketch(2x^2-5x+9/2)/(x^2-3 x+3)}
-(x-x_1)(x-x_2)=-\left(x-\left(\frac{3- \sqrt{3}}{2} \right)\right)\left(x-\left(\frac{3+ \sqrt{3}}{2}\right)\right)
\end{equation} 

The points $x_1, x_2$ split the real line into three intervals: $\left(-\infty, \frac{3- \sqrt{3}}{2}\right)$, $\left(\frac{3- \sqrt{3}}{2}, \frac{3+ \sqrt{3}}{2} \right)$ and $\left(\frac{3+ \sqrt{3}}{2}, \infty \right)$, and each of the factors of \eqref{eq1problemSketch(2x^2-5x+9/2)/(x^2-3 x+3)} has constant sign inside each of the intervals. If we choose $x$ to be a very negative number, it follows that $-(x-x_1)(x-x_2)$ is a negative, and therefore $ f'(x)$ is negative for $x\in(-\infty, \frac{3- \sqrt{3}}{2})$. For $x\in (\frac{3- \sqrt{3}}{2}, \frac{3+ \sqrt{3}}{2})$, exactly one factor of $f'$ changes sign and therefore $f'(x)$ is positive in that interval; finally only one factor of $f'(x)$ changes sign in the last interval so $f'(x)$ is negative on $(\frac{3+ \sqrt{3}}{2}, \infty )$.

Our computations can be summarized in the following table. 

\begin{tabular}{|lll|}\hline
Interval & $f'(x)$ & $f(x)$   \\\hline
$\left(-\infty, \frac{3- \sqrt{3}}{2}\right)$ & $-$& $\searrow $ \\\hline
$\left(\frac{3- \sqrt{3}}{2}, \frac{3+ \sqrt{3}}{2} \right)$ &$+$&$\nearrow$\\\hline
$\left( \frac{3+ \sqrt{3}}{2}, \infty\right)$&$-$&$\searrow$ \\\hline
\end{tabular}

\textbf{Local and global minima and maxima. } The table above shows that $f(x)$ changes from decreasing to increasing at $x=x_1=\frac{3- \sqrt{3}}{2}$ and therefore $f$ has a local minimum at that point. The table also shows that $f(x)$ changes from increasing to decreasing at $ x=x_2=\frac{3+ \sqrt{3}}{2}$ and therefore $f$ has a local maximum at that point. The so found local maximum and local minimum turn out to be global: there are two things to consider here. First, no other finite point is critical and thus cannot be maximum or minimum - however this leaves out the possibility of a maximum/minimum ``at infinity''. This possibility can be quickly ruled out by looking at the graph of $f$. To do so via algebra, compute first $f(x_1)$ and $f(x_2)$:

\[
\begin{array}{rcl}
\displaystyle f(x_1)= f\left(\frac{3- \sqrt{3}}{2} \right)&=& \displaystyle \frac{2\left(\frac{3- \sqrt{3}}{2} \right)^2-5\left(\frac{3- \sqrt{3}}{2} \right)+\frac{9}{2} }{\left(\frac{3- \sqrt{3}}{2} \right)^2- 3 \left(\frac{3- \sqrt{3}}{2} \right)+3}=2-\frac{\sqrt{3}}{3} \\

\displaystyle f(x_2)= f\left(\frac{3+ \sqrt{3}}{2} \right)&=& \displaystyle \frac{2\left(\frac{3+ \sqrt{3}}{2} \right)^2-5\left(\frac{3+ \sqrt{3}}{2} \right)+\frac{9}{2} }{\left(\frac{3+ \sqrt{3}}{2} \right)^2- 3 \left(\frac{3+ \sqrt{3}}{2} \right)+3}=2+\frac{\sqrt{3}}{3}\quad . 
\end{array}
\]
On the other hand, while computing the horizontal asymptotes, we established that $\lim\limits_{x\to\pm \infty}f(x)=2$. This implies that all $x$ sufficiently far away from $x=0$, we have that $f(x)$ is close to $2 $. Therefore $f(x)$ is larger than $f(x_1)$ and smaller than $f(x_2)$ for all sufficiently far away from $x=0$. This rules out the possibility for a maximum or a minimum ``at infinity'', as claimed above.

\textbf{Intervals of concavity. } 
The intervals of concavity of $f$ are governed by the sign of $f''$. The second derivative of $f$ is:
\[
\begin{array}{rcll|@{}l}
f''(x)&=&\displaystyle (f'(x))'= \left( \frac{- x^{2}+3 x-\frac{3}{2} }{ \left(x^2- 3 x+3\right)^2 } \right)'\\
&=&\displaystyle \left(- x^{2}+3 x-\frac{3}{2} \right)' \left(\frac{1 }{\left(x^2- 3 x+3\right)^2}\right)+ \left(- x^{2}+3 x-\frac{3}{2} \right)\left(\frac{1}{\left(x^2- 3 x+3\right)^2}\right)' &&\begin{array}{@{}l}\text{second differentiation:}\\\text{chain rule }\end{array}\\
&=&\displaystyle (-2x+3)\left(\frac{1}{\left(x^2- 3 x+3\right)^2} \right)+\left(- x^{2}+3 x-\frac{3}{2} \right)(-2)\frac{\left(x^2- 3 x+3\right)'}{\left(x^2- 3 x+3\right)^{3}}\\
&=&\displaystyle (-2x+3)\left(\frac{1}{\left(x^2- 3 x+3\right)^2}\right) +\left(2x^{2}-6 x+3 \right) \frac{(2x-3)}{\left(x^2- 3 x+3\right)^{3}}&&\text{factor out }\frac{(2x-3)}{\left(x^2- 3 x+3\right)^2}\\
&=&\displaystyle \frac{(2x-3)}{\left(x^2- 3 x+3\right)^2}\left(-1+\frac{(2x^{2}-6 x+3)}{\left(x^2- 3 x+3\right)}\right)\\
&=&\displaystyle \frac{(2x-3)}{\left(x^2- 3 x+3\right)^2}\left(\frac{-\left(x^2- 3 x+3\right)+(2x^{2}-6 x+3)}{\left(x^2- 3 x+3\right)} \right)\\
&=&\displaystyle \frac{(2x-3)(x^{2}-3 x )}{\left(x^2- 3 x+3\right)^3}\\
&=&\displaystyle \frac{(2x-3)x(x-3)}{\left(x^2- 3 x+3\right)^3}
\end{array}
\]
When computing the domain of $f$, we established that the denominator of the above expression is always positive. Therefore $f''(x)$ changes sign when the terms in the numerator change sign, namely, at $x=0$, $x=\frac{3}{2}$ and $x=3$. 

Our computations can be summarized in the following table. In the table, we use the $\cup$ symbol to denote that the function is concave up in the indicated interval, and $\cap$ to denote that the function is concave down.

\begin{tabular}{|lll|}\hline
Interval & $f''(x)$ & $f(x)$   \\\hline
$(-\infty, 0)$ & $-$& $\cap$ \\\hline
$(0, \frac{3}{2})$ &$+$&$\cup$\\\hline
$(\frac{3}{2}, 3)$&$-$&$\cap$ \\\hline
$(3, \infty)$&$+$&$\cup$ \\\hline
\end{tabular}

\textbf{Points of inflection.} The preceding table shows that $f''(x)$ changes sign at $0, \frac{3}{2}, 3$ and therefore the points of inflection are located at $x=0, x=\frac{3}{2}$ and $x=3$, i.e., the points of inflection are $\left(0, f(0)\right)= \left(0, \frac{3}{2} \right) $, $\left(\frac {3}{2}, f\left(\frac{3}{2}\right)\right) =\left(\frac{3}{2}, 2\right)$, $\left(3, f(3)\right)=\left(3, \frac{5}{2}\right)$.

We can command our graphing device to use the so computed information to label the graph of the function. Finally, we can confirm visually that our function does indeed behave in accordance with our computations.

\psset{xunit=0.6cm, yunit=0.6cm}
\begin{pspicture}(-5, -1)(5.2,4)
\psframe*[linecolor=white](-5,-1)(5,4)
\tiny
\psaxes[ticks=none, labels=none]{<->}(0,0) (-5,-0.5) (5, 3.5)
\fcLabels{5}{3.5}
%Function formula: \frac{2 x^{2}-5 x+9/2}{x^{2}-3 x+3}
\psplot[linecolor=\fcColorGraph, plotpoints=1000]{-5}{5 } {4.5 x -5 mul add x 2 exp 2 mul add 3 x -3 mul add x 2 exp add div }
\fcFullDot[linecolor=green]{0}{3 2 div}
\rput[r](-2, 0.2){infl.: $\left(0, \frac{3}{2}\right)$}
\psline[linestyle=dotted, arrows=->](-2, 0.2)(-0.05, 1.45)
\fcFullDot[linecolor=green]{3 2 div }{2}
\rput[r](1.2, 0.2){infl.: $\left(\frac{3}{2},2 \right)$}
\psline[linestyle=dotted, arrows=->](1.2, 0.2)(1.45, 1.95)
\fcFullDot[linecolor=green]{3 }{5 2 div}
\rput[l](2, 0.2){infl.: $\left(3, \frac{5}{2}\right)$}
\psline[linestyle=dotted, arrows=->](2, 0.2)(2.95, 2.45)
\fcFullDot[linecolor=blue]{3 3 sqrt add 2 div}{2 3 sqrt 3 div add}
\rput(2, 3){$\left(\frac{3+\sqrt{3}}{2}, 2+\frac{\sqrt{3}}{3} \right)$}
\fcFullDot[linecolor=blue]{3 3 sqrt sub 2 div}{2 3 sqrt 3 div sub}
\rput[r](-2, 3){$\left(\frac{3-\sqrt{3}}{2}, 2-\frac{\sqrt{3}}{3} \right)$}
\psline[linestyle=dotted, arrows=->](-2, 3)(! 3 3 sqrt sub 2 div 0.05 add 2 3 sqrt 3 div sub 0.05 add)
\end{pspicture}

}

\solution{\ref{problemSketch(x+1)/(x^2+2x+4)} 

\textbf{This problem is very similar to Problem \ref{problemSketchCurve(2x^2-5x+9/2)/(x^2-3 x+3)}. We recommend to the student to solve the problem first ``with closed textbook'' and only then to compare with the present solution.}

\textbf{Domain.} As $f$ is a quotient of two polynomials (rational function), its implied domain is all $x$ except those for which we get division by zero for $f$. Consequently the domain of $f$ is all $x$ for which $x ^2+2x+4=0$. However, the polynomial $x^2+2x+4$ has no real roots - its roots are $\displaystyle \frac{-2\pm \sqrt{4-16} }{2}=-1\pm \sqrt{-3}$, and therefore the domain of $f$ is all real numbers. Alternatively, we can complete the square: $x^2+2x+4=(x+1)^2+3$ and so $x^2+2x+4$ is positive for all values of $x$. 

\textbf{$x$, $y$-intercepts.} The $y$-intercept of $f$ equals by definition $\displaystyle f(0)= \frac{ 0+ 1}{0^2+2\cdot 0 + 4}=\frac{1}{4}$. The $x$ intercept of $f$ is those values of $x$ for which $f(x)=0$. We compute

\[
\begin{array}{rcl}
\displaystyle f(x)&=&0\\
\displaystyle \frac{x+1}{x^2+2x+4}&=&0\\
x+1&=&0\\
x&=&-1\quad ,
\end{array}
\]
and the $x$-intercept of $f$ is $x=-1$. 

\textbf{Asymptotes.} The line $x=a$ is a vertical asymptote when $\lim\limits_{x\to a^{\pm}}f(x)=\pm \infty$; as $f$ is defined for all real numbers, this implies that there are no vertical asymptotes. 
 
The line $y=L$ is a horizontal asymptote if $\lim\limits_{x\to\pm \infty}f(x)$ exists and equals $L$. We compute:
\[
\lim\limits_{x\to \infty} f(x)=\lim\limits_{x\to \infty} \frac{(x+1)\frac{1}{x^2}}{ (x^2+ 2x +4)\frac{1}{x^2}} = \lim\limits_{x\to \infty}\frac{\frac{1}{x}+\frac{1}{x^2}}{ 1+ \frac{2}{x} +\frac{4}{x^2}}=\frac{0+0}{1+0+0}=0
\]
Therefore $y=0$ is a horizontal asymptote for $f$. An analogous computation shows that $\lim\limits_{x\to\pm \infty}f(x)=0$ and so $y=0$ is the only horizontal asymptote of $f$.

\textbf{Intervals of increase and decrease.} 
The intervals of increase and decrease of $f$ are governed by the sign of $f'$. We compute:
\[
\begin{array}{rcll|l}
f'(x)&=&\displaystyle \left(\frac{x+1}{x^2+2x+4}\right)' &&\text{qutotient rule}\\
&=&\displaystyle \frac{(x+1)'\left(x^2+ 2x+4\right)- (x+1)\left( x^2 +2 x+4\right)'}{\left(x^2+2x+4 \right)^2}\\
&=&\displaystyle\frac{ x^2+2x+4-(x+1)(2x+2)}{\left(x^2+2x+4 \right)^2}\\
&=&\displaystyle \frac{x^2+2x+4-\left( 2x^2+ 4x+ 2 \right)}{ \left( x^2 +2x+4 \right)^2}\\
&=&\displaystyle \frac{-x^2-2x+2}{\left(x^2+2x+4 \right)^2}
\end{array}
\] 
As $x^2+2x+4$ is positive, the sign of $f'$ is governed by the sign of $-x^2+2x+2$. To find out where $-x^2+2x+2$ changes sign, we compute the zeroes of this expression:
\[\begin{array}{rcll|l}
-x^2-2x+2&=&0\\
x^2+2x-2&=&0 &&\text{use the quadratic formula}\\
x_1, x_2&=& -1\pm \sqrt{3}\quad .
\end{array}
\]
Therefore the quadratic $-x^2+2x+2$ factors as 
\begin{equation}
\label{eq1problemSketch(x+1)/(x^2+2x+4)}
-(x-x_1)(x-x_2)=-\left(x-\left(-1-\sqrt{3}\right)\right)\left(x-\left(-1+\sqrt{3}\right)\right)
\end{equation} 
The points $x_1, x_2$ split the real line into three intervals: $(-\infty, -1-\sqrt{3})$, $(-1-\sqrt{3}, -1+\sqrt{3})$ and $(-1+ \sqrt{3}, \infty )$, and each of the factors of \eqref{eq1problemSketch(x+1)/(x^2+2x+4)} has constant sign inside each of the intervals. If we choose $x$ to be a very negative number, it follows that $-(x-x_1)(x-x_2)$ is a negative, and therefore $ f'(x)$ is negative for $x\in(-\infty, -1-\sqrt{3})$. For $x\in (-1-\sqrt{3}, -1+\sqrt{3})$, exactly one factor of $f'$ changes sign and therefore $f'(x)$ is positive in that interval; finally only one factor of $f'(x)$ changes sign in the last interval so $f'(x)$ is negative on $(-1+ \sqrt{3}, \infty )$.

Our computations can be summarized in the following table. 

\begin{tabular}{|lll|}\hline
Interval & $f'(x)$ & $f(x)$   \\\hline
$(-\infty, -1-\sqrt{3})$ & $-$& $\searrow $ \\\hline
$(-1-\sqrt{3}, -1+\sqrt{3})$ &$+$&$\nearrow$\\\hline
$( -1+\sqrt{3}, \infty)$&$-$&$\searrow$ \\\hline
\end{tabular}

\textbf{Local and global minima and maxima. } The table above shows that $f(x)$ changes from decreasing to increasing at $x=x_1=-1-\sqrt{3}$ and therefore $f$ has a local minimum at that point. The table also shows that $f(x)$ changes from increasing to decreasing at $ x=x_2=-1+\sqrt{3}$ and therefore $f$ has a local maximum at that point. The so found local maximum and local minimum turn out to be global: indeed, no other finite point is critical and thus cannot be maximum or minimum; on the other hand $\lim\limits_{x\to\pm \infty}f(x)=1$ and this implies that all $x$ sufficiently far away from $x=0$ have that $f(x)$ is close to $0$, and therefore $f(x)$ is larger than $f(x_1)$ and smaller than $f(x_2)$ for all $x$.

\textbf{Intervals of concavity. } 
The intervals of concavity of $f$ are governed by the sign of $f''$. The second derivative of $f$ is:
\[
\begin{array}{rcll|l}
f''(x)&=&\displaystyle (f'(x))'= \left(\frac{-x^2-2x+2}{\left(x^2+2x+4 \right)^2}\right)'\\
&=&\displaystyle (-x^2-2x+2)'\left(\frac{1}{(x^2+2x+4)^2}\right)+(-x^2-2x+2)\left(\frac{1}{(x^2+2x+4)^2}\right)' &&\begin{array}{l}\text{use chain rule }\\\text{for second differentiation}\end{array}\\
&=&\displaystyle (-2x-2)\left(\frac{1}{(x^2+2x+4)^2}\right)+(-x^2-2x+2)(-2)\frac{(x^2+2x+4)'}{(x^2+2x+4)^{3}}\\
&=&\displaystyle -(2x+2)\left(\frac{1}{(x^2+2x+4)^2}\right) +(2x^2+4x-4)\frac{(2x+2)}{(x^2+2x+4)^{3}}&&\text{factor out }\frac{(2x+2)}{(x^2+2x+4)^2}\\
&=&\displaystyle \frac{(2x+2)}{(x^2+2x+4)^2}\left(-1+\frac{(2x^2+4x-4)}{(x^2+2x+4)}\right)\\
&=&\displaystyle \frac{(2x+2)}{(x^2+2x+4)^2}\left(\frac{-(x^2+2x+4)+(2x^2+4x-4)}{(x^2+2x+4)} \right)\\
&=&\displaystyle \frac{(2x+2)(x^{2}+2 x-8)}{(x^2+2x+4)^3}&& \text{factor } (x^2+2x-8)\\
&=&\displaystyle \frac{(2x+2)(x+4)(x-2)}{(x^2+2x+4)^3}
\end{array}
\]
As we previously established, the denominator of the above expression is always positive. Therefore the expression above changes sign when the terms in the numerator change sign, namely, at $x=-1$, $x=-4$ and $x=2$. 

Our computations can be summarized in the following table. 

\begin{tabular}{|lll|}\hline
Interval & $f''(x)$ & $f(x)$   \\\hline
$(-\infty, -4)$ & $-$& $\cap$ \\\hline
$(-4, -1)$ &$+$&$\cup$\\\hline
$(-1, 2)$&$-$&$\cap$ \\\hline
$(2, \infty)$&$+$&$\cup$ \\\hline
\end{tabular}

\textbf{Points of inflection.} The preceding table shows that $f''(x)$ changes sign at $-4, -1, 2$ and therefore the points of inflection are located at $x=-4, x=-1$ and $x=2$, i.e., the points of inflection are $\left(-4, -\frac{1}{4}\right)$, $\left(-1, 0\right)$, $\left(2, \frac{1}{4}\right)$.

}

\begin{problem}
\begin{enumerate}
\item Sketch the graph of $y = x^4 - 8x^2 + 8$ by determining the intervals of increase and decrease, finding the local mins and maxes, determining where the graph is concave up and concave down, and plotting a few key points.

\answer{
\begin{tabular}{l}
Check your graph with a calculator or online graphing program. \\
Local max at 0, local mins at 2 and -2. Concave down between $-\sqrt{4/3}$ and $\sqrt{4/3}$, and concave up otherwise.
\end{tabular}
}

\item Sketch the graph of $y = \frac{x-1}{x^2-9}$ by graphing any vertical and horizontal asymptotes, finding the $x$- and $y$-intercepts, and then sketching a graph that fits this information.

\answer{
\begin{tabular}{l}
Check your graph with a calculator or online graphing program. \\
Vertical asymptotes at $x = 3$ and $x = -3$. \\ Horizontal
asymptote at $y = 0$. \\
$y$-intercept of $\frac{1}{9}$; $x$-intercept of $1$.
\end{tabular}
}
\item \label{problemSketch(4x^2+10x+5)/(2x+1)}

Consider the function $\displaystyle f(x)=\frac{4 x^{2}+10 x+5}{2 x+1} $. Computation shows that $\displaystyle f'(x)=\frac{8 x^{2}+8 x}{\left(2 x+1\right)^{2}}$ and $\displaystyle f''(x)=\frac{8}{\left(2 x+1\right)^{3}} $.

\begin{itemize}
\item Find the intervals of increase and intervals of decrease of $f$.

\item Find the local maxima and minima of $f$. 
\item Find where the function is concave up and where it is concave down.
\item Sketch the function $f(x)$ roughly by hand. Make sure that your plot matches your computations from the preceding parts of the problem.

You may use the provided grid and coordinate system. From the previous page, we recall that $\displaystyle f(x)=\frac{4 x^{2}+10 x+5}{2 x+1} $, $\displaystyle f'(x)=\frac{8 x^{2}+8 x}{\left(2 x+1\right)^{2}}$ and $\displaystyle f''(x)=\frac{8}{\left(2 x+1\right)^{3}} $.

The 4 points plotted on the grid are known to lie on the curve.

\psset{xunit=1cm, yunit=1cm}
\begin{pspicture}(-3.1,-3)(3.1,12.1)
\fcAxesStandard{-3}{-3}{3}{12}
\fcGrid[linestyle=dashed, linewidth=0.5, linecolor=gray]{-3}{-3}{6}{15}{1}{1}{}
\rput[t](0.9,-0.2){$1$}
\fcLabels{3}{12}
\fcFullDot{-3}{1 dict begin /x -3 def x x mul 4 mul 10 x mul 5 add add 2 x mul 1 add div end}
\fcFullDot{-0.59}{1 dict begin /x -0.59 def x x mul 4 mul 10 x mul 5 add add 2 x mul 1 add div end}
\fcFullDot{-0.44}{1 dict begin /x -0.44 def x x mul 4 mul 10 x mul 5 add add 2 x mul 1 add div end}
\fcFullDot{3}{1 dict begin /x 3 def x x mul 4 mul 10 x mul 5 add add 2 x mul 1 add div end}

%\psplot{-3}{-0.59}{x x mul 4 mul 10 x mul 5 add add 2 x mul 1 add div}
%\psplot{-0.44}{3}{x x mul 4 mul 10 x mul 5 add add 2 x mul 1 add div}
\end{pspicture}
\end{itemize}
\item \label{problemSketch(2 x^2-4 x+2)/(x^2-2 x)}
Consider the function $\displaystyle f(x)=\frac{2 x^{2}-4 x+2}{x^{2}-2 x} $.

\begin{itemize}
\item Find the vertical asymptotes of $f$. \textbf{For this particular sub-question, and for this sub-question alone, no justification is required (just write the answer).}

\item Computation shows that $\displaystyle f'(x)=\frac{-4 x+4}{\left(x^{2}-2 x\right)^{2}}$. Find the intervals of increase and decrease of $f$.
\item Find the local maxima and minima of $f$. 
\item Computation shows that $\displaystyle f''(x)=\frac{12 x^{2}-24 x+16}{\left(x^{2}-2 x\right)^{3}} $. Find where the function is concave up and where it is concave down.
\item Sketch the function $f(x)$ roughly by hand. Make sure that your plot matches your computations from the preceding parts of the problem.


You may use the provided grid and coordinate system. We recall that \\
$\displaystyle f(x)=\frac{2 x^{2}-4 x+2}{x^{2}-2 x} $,\\ 
$\displaystyle f'(x)=\frac{-4 x+4}{\left(x^{2}-2 x\right)^{2}}$,\\ $\displaystyle f''(x)=\frac{12 x^{2}-24 x+16}{\left(x^{2}-2 x\right)^{3}}$.

The points plotted below are known to lie on the curve.

\psset{xunit=0.8cm, yunit=0.8cm}
\begin{pspicture}(-3.2,-9.2)(5.2,13.2)
\fcAxesStandard{-3}{-8}{5}{13}
\fcGrid[linestyle=dashed, linewidth=0.5, linecolor=gray]{-3}{-9}{8}{22}{1}{1}{}
\rput[t](0.9,-0.2){$1$}
\fcLabels{5}{13}
\newcommand{\theFun}{2 x x mul mul -4 x mul 2 add add x x mul -2 x mul add div\space}
\fcFullDot{-3}{1 dict begin /x -3 def \theFun end}
\fcFullDot{-0.1}{1 dict begin /x -0.1 def \theFun end}
\fcFullDot{0.1}{1 dict begin /x 0.1 def \theFun end}
\fcFullDot{1.9}{1 dict begin /x 1.9 def \theFun end}
\fcFullDot{2.1}{1 dict begin /x 2.1 def \theFun end}
\fcFullDot{5}{1 dict begin /x 5 def \theFun end}

%\psplot{-3}{-0.1}{\theFun}
%\psplot{0.1}{1.9}{\theFun}
%\psplot{2.1}{5}{\theFun}
\end{pspicture}

\end{itemize}
\end{enumerate}

\end{problem}

\section{Linearizations and Differentials} \label{secMPSLinearizationAndDifferentials}
\begin{problem}
~\begin{enumerate}[ref={\fcProblemRef}]
\item Find the linearization of $f(x) = \sqrt{x}$ at $a = 100$ and use it to approximate
$\sqrt{99.8}$.

\answer{$L(x) = 10 + 0.05(x-100)$. Therefore $\sqrt{99.8} \approx L(99.8) = 9.99$.}
 
\item Find the linearization of $f(x)=\sqrt{8+x}$ at $a=1$ and use it to approximate $\sqrt{9.02}$.

\answer{ $f(x)\approx 3+ \frac16 (x-1)=\frac{1}{6} x+\frac{17}{6}$. Therefore $\sqrt{9.02}\approx \frac{901}{300} \approx 3.003333$}
\item Find the linearization of $f(x)=\sqrt[3]{8+x}$ at $a=0$ and use it to approximate $\sqrt[3] {7.97}$.

\answer{ $\sqrt[3]{8+x}\approx \frac{1}{12}x+2$. Therefore $\sqrt[3]{7.97}\simeq \frac{799}{400} =1.9975$}

\item Find the linearization of $f(x)=\ln x$ at $a=1$ and use it to approximate $\ln 1.01$.

\answer{ $f(x)\approx f(1)+f'(1)(x-1)=x-1 $, $\ln 1.01\approx 0.01$. }
\item Use a linear approximation to estimate $(1.001)^9$. 

\answer{$(1.001)^9 \approx 1.009$.}
\item \label{problem-linearization-estimate0.9999power2014} Use a linear approximation to estimate $(0.9999)^{2014}$. 

\answer{$(0.9999)^{2014} \approx 0.7986$.}

\end{enumerate}

\end{problem}
\solution{\ref{problem-linearization-estimate0.9999power2014}
Let $f(x)=x^{2014}$. We are looking to approximate $(0.9999)^{2014}= f(0.9999)$. As $f(1)=1^{2014}=1$ is easy to compute, is makes sense to use linear approximation at $a=1$ to approximate $(0.9999)^{2014}$. We have that 
\[
f'(x)=2014x^{2013} \quad .
\]
Therefore the linear approximation of $f(x)=x^{2014}$ at $a=1$ is:
\[
f(x)\approx f(1) +f'(1)(x-1)= 1^{2014}+2014 \cdot 1^{2013}(x-1)=1+ 2014(x-1)=2014x-2013 \quad .
\]
Therefore 
\[
f(0.9999)\approx 2014\cdot 0.9999 -2013=1\cdot 0.9999 +2013(0.9999-1)=0.9999-2013\cdot 0.0001= 0.9999-0.2013=0.7986
\]


A computation with computer shows that $0.999^{2014}=0.817577\dots $. While our approximation of $0.7986$ is less than perfect, it is within the same order of magnitude. We study techniques for estimating errors in linear approximations later.
}

\section{Integration Basics}
\subsection{Riemann Sums}\label{secMPSRiemannSums}
\begin{problem}
Estimate the integral using a Riemann sum using the indicated sample points and interval length.
\begin{enumerate}[ref={\fcProblemRef}]
% Riemann sums
\item \label{problemRiemannSum-sqrt(8x+1)} $\displaystyle \int_0^4 \left(\sqrt{8x+1}\right)\diff x$. Use four intervals of equal width, choose the sample point to be the left endpoint of each interval. 

\answer{ $\Delta x = 1$ and $f(x) = \sqrt{8x+1}$. Thus ${\displaystyle \int_0^4 f(x) \diff x \approx 9 + \sqrt{17}}$.}

\item $\displaystyle \int_0^6 \frac{1}{x^2+1} \diff x$. Use three intervals of equal width, choose the sample point to be the left endpoint. 

\answer{ $\Delta x = 2$ and $f(x) = \frac{1}{x^2+1}$. Thus ${\displaystyle \int_0^6 f(x) \diff x \approx \frac{214}{85}}$.}
\item \label{problemRiemannSum-1div1plusxsquared} $\displaystyle \int\limits_{-3.5}^{-0.5} \frac{\diff x}{x^2+1} $. Use three intervals of equal width, choose the sample point to be the midpoint of each interval. 

\answer{ $\Delta x = 1$ and $f(x) = \frac{1}{x^2+1}$. Thus $\displaystyle \int \limits_{-3.5}^{-0.5} f(x) \diff x  \approx \Delta x\left(f{} \left(-3 \right)+ f{}\left( -2\right)+f{}\left(-1\right)\right)=\frac{4}{5}=0.8$.}
\item $\displaystyle\int_{0}^2 \frac{\diff x}{1+x+x^3}$. Use $\Delta x=\frac{1}2 $ and right endpoint sampling points.

\answer{$ \frac{1}{2}\left(\frac{8}{13}+\frac{1}{3}+\frac{8}{47}+\frac{1}{11}\right)=\frac{12197}{20163}\approx 0.604920$}

\item $\displaystyle\int_{-2}^{0} \frac{\diff x}{1+x+x^2}$. Use $\Delta x=\frac23 $ and left endpoint sampling points.

\answer{$\frac23\left(\frac{1}{3}+\frac{9}{13}+\frac{9}{7}\right)=\frac{1262}{819}\approx 1.540904$}

\item $\displaystyle \int\limits_0^2 \frac{\diff x}{1+x^3}$. Use four intervals of equal width, choose the sample point to be the left endpoint of each interval. 

\answer{ $\Delta x = 0.5$ and $f(x) = \frac{1}{1+x^3}$. Thus $\displaystyle \int\limits_0^2 f(x) \diff x  \approx \Delta x\left(f{}\left(0\right)+f{}\left(1\right)+f{}\left(\frac{1}{2}\right)+f{}\left(\frac{3}{2}\right)\right)=\frac{1649}{1260}\approx 1.30873$.}

\item $\displaystyle \int\limits_{-2}^{0} \frac{\diff x}{x^4+1} $. Use four intervals of equal width, choose the sample point to be the right endpoint. 

\answer{ $\Delta x = 0.5$ and $f(x) = \frac{1}{1+x^3}$. Thus $\displaystyle \int\limits_0^2 f(x) \diff x  \approx \Delta x\left(f{}\left(-\frac{3}{2}\right)+f{}\left(-1\right)+f{}\left(-\frac{1}{2}\right)+f{}\left(0\right)\right)=\frac{8595}{6596}\approx 1.303062$.}

\item  \label{problemRiemannSum1/(3x^2+1)from-1to0with3intervalsLeftEndpt}
$\displaystyle\int_{-1}^0\frac{1}{3 {{x}}^{2}+1}\diff x
$. Use \textbf{$3$ intervals} of equal width, choose the sampling points to be the \textbf{left endpoints} of each interval. 
Simplify your answer to a rational number (single fraction of two integers).

\answer{ $\Delta x = \frac{1}{3}$ and $f(x) =\frac{1}{3 {{x}}^{2}+1}$. Thus $\displaystyle \int\limits_{-1}^0 f(x) \diff x$  is approximated by $\Delta x \left(f{}\left(-1\right)+f{}\left( -\frac{2}{3}\right)+f{}\left( -\frac{1}{3}\right)\right)=\frac{10}{21}$.}

\end{enumerate}



\end{problem}
\solution{\ref{problemRiemannSum-sqrt(8x+1)}. The interval $[0,4]$ is subdivided into $n=4$ intervals, therefore the length of each is $\Delta x=1$. The intervals are therefore
\[
[0,1], [1,2], [2,3], [3,4]\quad .
\]
The problem asks us to use the left endpoints of each interval as sampling points. Therefore our sampling points are $0,1,2,3$. Therefore the Riemann sum we are looking for is
\[
\Delta x\left(f(0)+f(1)+f(2)+f(3) \right)=1\cdot \left(\sqrt{8\cdot 0+1}+\sqrt{8\cdot 1+1}+\sqrt{8\cdot 2+1}+\sqrt{8\cdot 3+1}\right)= 9+\sqrt{17}\approx 13.1231
\]

\hfil \hfil \psset{xunit=1cm, yunit=1cm}
\begin{pspicture}(-0.9, -0.9)(4.4,6.233433)
\tiny
\psline*[linecolor=\fcColorAreaUnderGraph, linewidth=0.1pt](0.000000, 0.000000)(0.000000, 1.000000)(1.000000, 1.000000)(1.000000, 0.000000)(0.000000, 0.000000)
\psline*[linecolor=\fcColorAreaUnderGraph, linewidth=0.1pt](1.000000, 0.000000)(1.000000, 3.000000)(2.000000, 3.000000)(2.000000, 0.000000)(1.000000, 0.000000)
\psline*[linecolor=\fcColorAreaUnderGraph, linewidth=0.1pt](2.000000, 0.000000)(2.000000, 4.123106)(3.000000, 4.123106)(3.000000, 0.000000)(2.000000, 0.000000)
\psline*[linecolor=\fcColorAreaUnderGraph, linewidth=0.1pt](3.000000, 0.000000)(3.000000, 5.000000)(4.000000, 5.000000)(4.000000, 0.000000)(3.000000, 0.000000)
\psline[linecolor=blue, linewidth=0.1pt](0.000000, 0.000000)(0.000000, 1.000000)(1.000000, 1.000000)(1.000000, 0.000000)(0.000000, 0.000000)
\psline[linecolor=blue, linewidth=0.1pt](1.000000, 0.000000)(1.000000, 3.000000)(2.000000, 3.000000)(2.000000, 0.000000)(1.000000, 0.000000)
\psline[linecolor=blue, linewidth=0.1pt](2.000000, 0.000000)(2.000000, 4.123106)(3.000000, 4.123106)(3.000000, 0.000000)(2.000000, 0.000000)
\psline[linecolor=blue, linewidth=0.1pt](3.000000, 0.000000)(3.000000, 5.000000)(4.000000, 5.000000)(4.000000, 0.000000)(3.000000, 0.000000)
%Function formula: (8 x+1)^{1/2}
\psplot[linecolor=\fcColorGraph, plotpoints=1000]{0}{4}{ 1 x 8 mul add 0.5 exp }
\psaxes(0,0)(-0.65,-0.65)(4.15,5.883433)
\end{pspicture}
}

\solution{
\ref{problemRiemannSum-1div1plusxsquared}. The interval $[-3.5,-0.5]$ is subdivided into $n=3$ intervals, therefore the length of each is $\Delta x=\frac{-0.5-(-3.5)}{3}=\frac{3}{3}= 1$. The intervals are therefore
\[
[-3.5,-2.5], [-2.5,-1.5], [-1.5,-0.5]\quad .
\]
The problem asks us to use the midpoint of each interval as a sampling point. Therefore our sampling points are $-3,-2,-1$. Therefore the Riemann sum we are looking for is
\[
\Delta x\left(f(-3)+f(-2)+f(-1) \right)=1\cdot \left( \frac{1}{10}+\frac{1}{5}+\frac{1}{2}\right)= 0.8\quad .
\]

\hfil \hfil 
\psset{xunit=1cm, yunit=1cm}
\begin{pspicture}(-3.9, -0.9)(1.4,1.499857)
\tiny

\psline*[linecolor=\fcColorAreaUnderGraph, linewidth=0.1pt](-3.500000, 0.000000)(-3.500000, 0.100000)(-2.500000, 0.100000)(-2.500000, 0.000000)(-3.500000, 0.000000)
\psline*[linecolor=\fcColorAreaUnderGraph, linewidth=0.1pt](-2.500000, 0.000000)(-2.500000, 0.200000)(-1.500000, 0.200000)(-1.500000, 0.000000)(-2.500000, 0.000000)
\psline*[linecolor=\fcColorAreaUnderGraph, linewidth=0.1pt](-1.500000, 0.000000)(-1.500000, 0.500000)(-0.500000, 0.500000)(-0.500000, 0.000000)(-1.500000, 0.000000)
\psline[linecolor=blue, linewidth=0.1pt](-3.500000, 0.000000)(-3.500000, 0.100000)(-2.500000, 0.100000)(-2.500000, 0.000000)(-3.500000, 0.000000)
\psline[linecolor=blue, linewidth=0.1pt](-2.500000, 0.000000)(-2.500000, 0.200000)(-1.500000, 0.200000)(-1.500000, 0.000000)(-2.500000, 0.000000)
\psline[linecolor=blue, linewidth=0.1pt](-1.500000, 0.000000)(-1.500000, 0.500000)(-0.500000, 0.500000)(-0.500000, 0.000000)(-1.500000, 0.000000)
\rput[t](-3.500000,-0.03){$-\frac{7}{2}$}\rput[t](-2.500000,-0.03){$-\frac{5}{2}$}\rput[t](-1.500000,-0.03){$-\frac{3}{2}$}\rput[t](-0.500000,-0.03){$-\frac{1}{2}$}
%Function formula: (x^{2}+1)^{-1}
\psplot[linecolor=\fcColorGraph, plotpoints=1000]{-3.5}{1}{ 1 x 2 exp add -1 exp }
\psaxes[ticks=none, labels=none, arrows=<-> ](0,0)(-3.65,-0.65)(1.15,1.149857)
\fcLabels{1.15}{1.149857}
\end{pspicture}
}
\solution{\ref{problemRiemannSum1/(3x^2+1)from-1to0with3intervalsLeftEndpt}

$\Delta x = \frac{1}{3}$ and $f(x) =\frac{1}{3 {{x}}^{2}+1}$. Thus $\displaystyle \int\limits_{-1}^0 f(x) \diff x$  is approximated by $\Delta x \left(f{}\left(-1\right)+f{}\left(-\frac{2}{3}\right)+f{}\left(-\frac{1}{3}\right)\right)=\frac{10}{21}$.

}


\subsection{Antiderivatives}\label{secMPSantiderivatives}
\begin{problem}
Find all antiderivatives of the functions.
\begin{multicols}{3}
\begin{enumerate}[ref={\fcProblemRef}]
\item $\displaystyle f(x)=\sqrt {3}+\pi^2$.

\answer{$\displaystyle x\left(\pi^{2} +\sqrt{3}\right)+C$}
\item $\displaystyle f(x)=x-5$.

\answer{$\displaystyle \frac{x^2}{2}-5x+C$}

\item $\displaystyle f(x)= x^2-2x+6$.

\answer{ $\frac{x^3}{3} -x^2+6x+C$ }

\item $\displaystyle f(x)=\frac{x(x+1)}{2} $.

\answer{$\frac{1}{6} x^{3}+\frac{1}{4} x^{2}+C$}
\item $\displaystyle f(x)=x(x+1)(2x+1)$.

\answer{$\frac{1}{2} x^{4}+x^{3}+\frac{1}{2} x^{2}+C$}
\item $\displaystyle f(x)=7x^{\frac{2}{7}}+x^{-\frac{4}{7}}$.

\answer{$\frac{49}{9} x^{\frac{9}{7}}+\frac{7}{3} x^{\frac{3}{7}}+ C$}
\item $\displaystyle f(x)=x^{2.4}-2x^{\sqrt{3}-1}$.

\answer{$\frac{5}{17} x^{\frac{17}{5}}-\frac{2\sqrt{3} x^{\sqrt{3}}}{3} +C$}
\item $\displaystyle f(x)=\frac{8}{x^7}$.

\answer{$-\frac{4}{3} x^{-6}+C$}
\item $\displaystyle f(x)=\frac{x+1}{x^3}$.

\answer{$- x^{-1}-\frac{1}{2} x^{-2}+C$}
\item $\displaystyle f(x)=\frac{1}{x}$.

\answer{$\ln |x|+C$}
\item $\displaystyle f(x)=\frac{x^2+1}{x}$.

\answer{$\frac{1}{2} x^{2}+\ln|x|+C $}
\item $\displaystyle f(x)=\frac{5-4x^3+2x^6}{x^4}$.

\answer{$\frac{2}{3} x^{3}-\frac{5}{3} x^{-3}-4 \ln|x|+C $}
\item $\displaystyle g(x)=\frac{1+\sqrt{x}+x}{\sqrt{x^3}}$.

\answer{$2 x^{\frac{1}{2}}-2 x^{-\frac{1}{2}}+\ln|x|+C $}
\item $\displaystyle f(t)=3\sin t-4\cos t$.

\answer{$-3\cos t -4\sin t +C $}
\item $\displaystyle f(\theta)=\sec^2\theta$.

\answer{$\tan \theta +C $}
\item $\displaystyle f(\theta)=\csc^2\theta$.

\answer{$ -\cot\theta +C $}

\item $\displaystyle f(t)=\sec t \tan t +\csc t \cot t$.

\answer{$\sec t-\csc t +C $}
\item $\displaystyle f(x)=\frac{2+x\cos x}{x}$.

\answer{$2\ln |x|+\sin x $}
\end{enumerate}
\end{multicols}
\end{problem}
\begin{problem}
\begin{enumerate}
\item Find $f(x)$ if $f'(x) = 3 + \frac{1}{x}$ and $f(1) = 2$.

\answer{$f(x) = 3x + \ln |x| - 1$}
\item Find $f(x)$ if $f'(x) = x - \sin x$ and $f(0) = 7$.

\answer{$f(x) = \frac{x^2}{2} + \cos x + 6$}
\end{enumerate}

\end{problem}
\begin{problem}
Verify by differentiation that the formula is correct.
\begin{multicols}{2}
\begin{enumerate}
\item $\displaystyle \int\sqrt{1+x^2}\diff x=  \frac{1}{2}\left( x \sqrt{1+x^2} +\ln \left(x+\sqrt{1+x^2}\right)+C \right) $.
\item $\displaystyle \int \sin^2x \diff x=-\frac{1}{4} \sin{}(2 x)+\frac{1}{2} x+C$.
\item $\displaystyle \int \sin^3 x \diff x =\frac{1}{3} \cos^{3}{}x- \cos{}x+C$.
\item $\displaystyle \int \frac{x}{\sqrt{1+x}}\diff x= \frac{2}{3}(x-2)\sqrt{1+x}+C$
\end{enumerate}
\end{multicols}
\end{problem}

\subsection{Basic Definite Integrals} \label{secMPSBasicDefiniteIntegrals}
\begin{problem}
Evaluate the integral (definite or indefinite).
\begin{multicols}{3}
\begin{enumerate}[ref={\fcProblemRef}]
\item $\displaystyle \int\limits_{-2}^{3} \left(x^2-1 \right)  \diff x$.

\answer{$\left[ \frac{1}{3} x^{3}- x \right]_{-2}^3=\frac{20}{3}$}

\item $\displaystyle \int\limits_{1}^{2} \left(4x^3+3x^2+2x+1\right)  \diff x$.

\answer{$\left[ x^{4}+x^{3}+x^{2}+x\right]_{1}^{2}=26$}
\item $\displaystyle \int\limits_{0}^{2}(x-1)(x^2+1)  \diff x$.

\answer{$\left[\frac{1}{4} x^{4}-\frac{1}{3} x^{3}+\frac{1}{2} x^{2}- x \right]_{0}^{2} = \frac{4}{3}$}
\item $\displaystyle \int\limits_{-1}^{1} \left( \frac{x(x+1) }{ 2} \right)^2  \diff x$.

\answer{$\left[\frac{1}{20} x^{5}+\frac{1}{8} x^{4}+\frac{1}{12} x^{3} \right]_{-1}^{1}=\frac{4}{15}$}
\item $\displaystyle \int\limits_{0}^{1}(1+x^2)^3 dx$.

\answer{$\left[\frac{1}{7} x^{7}+\frac{3}{5} x^{5} + x^{3} + x \right]_{0}^{1}=\frac{96}{35}$}
\item $\displaystyle \int \limits_{1}^{2} \left(\frac{1}{x} - \frac{4}{x^2} \right)  \diff x$.

\answer{$\left[ 4 x^{-1}+\ln x\right]_{1}^{2}=\ln 2-2$}
\item $\displaystyle \int\limits_{1}^{4}\sqrt{x}(1+x) \diff x$.

\answer{$\left[\frac{2}{5} x^{\frac{5}{2}}+\frac{2}{3} x^{\frac{3}{2}} \right]_{1}^{4}=\frac{256}{15}$}

\item $\displaystyle \int\limits_{1}^{4} \sqrt{ \frac{6 }{x }} \diff x$.

\answer{$\left[2 \sqrt{6} \sqrt{x} \right]_{1}^{4}=2 \sqrt{6} $}
\item $\displaystyle \int \limits_{1}^{4} \frac{ \frac{ 1}{ \sqrt{x}}+1+x}{ \sqrt{x}}  \diff x$.

\answer{$\left[ \frac{2}{3} x^{\frac{3}{2}}+2 \sqrt{x}+\ln{}\left|x\right|\right]_{1}^{4}=\ln{}\left(4\right)+\frac{20}{3}$}

\item $\displaystyle \int \limits_{1}^{8} \frac{1+x}{ \sqrt[3]{x}} \diff x$.

\answer{$\left[\frac{3}{5} x^{\frac{5}{3}}+\frac{3}{2} x^{\frac{2}{3}}  \right]_{1}^{8}=\frac{231}{10}$}
\item $\displaystyle \int\limits_{1}^{64} \frac{\frac{1}{ \sqrt[3]{x}} +\sqrt[3]{x}}{ \sqrt{x}}\diff x$.

\answer{$\left[\frac{6}{5} x^{\frac{5}{6}}+6 x^{\frac{1}{6}}  \right]_{1}^{64}=\frac{216}{5}$}
\item $\displaystyle \int\limits_0^{1} \left(\sqrt[5]{x^6} + \sqrt[6]{x^5}\right) \diff x $.

\answer{$\left[\frac{5}{11} x^{\frac{11}{5}}+\frac{6}{11} x^{\frac{11}{6}} \right]_{0}^{1}=1$}

\item $\displaystyle \int\limits_{1}^{2} \left(x + \frac{1}{x} \right)^2 \diff x$.

\answer{$\left[frac{1}{3} x^{3}- x^{-1}+2 x  \right]_{1}^{2}= \frac{29}{6}$}
\item $\displaystyle \int\limits_{1}^{2} \left(x + \frac{1}{x} \right)^3 \diff x$.

\answer{$\left[\frac{1}{4} x^{4}+\frac{3}{2} x^{2}-\frac{1}{2} x^{-2}+3 \ln{}x  \right]_{1}^{2}=\frac{69}{8}+3 \ln{}2 $}
\item $\displaystyle \int\limits_{1}^{2} \left(\sqrt{x} + \frac{1}{\sqrt{x}} \right)^2 \diff x$.

\answer{$\left[ \frac{1}{2} x^{2}+\ln{}x+2 x \right]_{1}^{2}=\frac{7}{2} +\ln{}\left(2\right) $}
\item $\displaystyle \int\limits_{1}^{2} \left(\sqrt{x} +\frac{1}{\sqrt{x}} \right)^3 \diff x$.

\answer{$\left[\frac{2}{5} x^{\frac{5}{2}}+2 x^{\frac{3}{2}}+6 \sqrt{x}-2 x^{-\frac{1}{2}} \right]_{1}^{2}= \frac{53}{5}\sqrt{2}-\frac{32}{5} $}
\item $\displaystyle \int\limits_{0}^{2}|x-1| \diff x$.

\answer{$1$}
\item \label{problemIntegralAbsoluteValuexminushalf} $\displaystyle \int\limits_{0}^{1} \left|x-\frac{1}{2}\right| \diff x$.

\answer{$\frac{1}{4}$}
\item $\displaystyle \int\limits_{-1}^{1}(x-3|x|) \diff x$.

\answer{$-3$}
\item $\displaystyle \int\limits_{\frac{\pi}{4}}^{\frac{\pi}{2}} \csc^2\theta \diff \theta$.

\answer{$\left[ \right]_{}^{}=$}
\item $\displaystyle \int\limits_{0}^{\frac{\pi}{4}}\frac{1-\cos^2\theta}{\cos^2\theta} \diff \theta$.

\answer{$\left[ \right]_{}^{}=$}
\item $\displaystyle \int\limits_{0}^{\frac{\pi}{4}}\frac{\sin^2\theta}{\cos^2\theta} \diff \theta$.

\answer{$\left[ \right]_{}^{}=$}
\item $\displaystyle \int\limits_{0}^{\frac{\pi}{4}}\tan^2\theta \diff \theta$.

\answer{$\left[ \right]_{}^{}=$}
\item $\displaystyle \int\limits_{0}^{\frac{\pi}{3}} \frac{\sin \theta +\sin \theta \tan^2\theta}{\sec^2\theta} \diff \theta$.

\answer{$\left[ \right]_{}^{}=$}
\item $\displaystyle \int\limits_{0}^{\pi} (\sin \theta -\cos \theta) \diff \theta$.

\answer{$\left[ \right]_{}^{}=$}
\item $\displaystyle \int\limits_{0}^{\pi}|\sin x| \diff x$.
\end{enumerate}
\end{multicols}

\end{problem}
\solution{\ref{problemIntegralAbsoluteValuexminushalf}
\[
\begin{array}{rcll|l}
\displaystyle \int\limits_{0}^{1} \left|x-\frac{1}{2}\right| \diff x
&=&\displaystyle  \int\limits_{0}^{\frac{1}{2}}  \left|x-\frac{1}{2}\right| \diff x+\int\limits_{\frac{1}{2}}^1  \left|x-\frac{1}{2}\right| \diff x&&\begin{array}{l}\left|x-\frac{1}{2}\right| =\frac{1}{2}-x \text{ when }x\leq \frac{1}{2} \\
\left|x-\frac{1}{2}\right| =x-\frac{1}{2} \text{ when }x\geq \frac{1}{2}
\end{array}\\
&=&\displaystyle  \int\limits_{0}^{\frac{1}{2}} \left(\frac{1}{2}-x\right) \diff x+\int \limits_{\frac{1}{2}}^1 \left(x-\frac{1}{2}\right) \diff x\\
&=&\displaystyle \left[-\frac{x^2}{2}+\frac{x}{2}\right]_{0}^{\frac{1}{2}} +\left[\frac{x^2}{2}-\frac{x}{2}\right]_{\frac{1}{2}}^{1}\\
&=&\displaystyle \left(-\frac{1}{8}+\frac{1}{4}\right) +\left(\frac{1}{2} -\frac{1}{2}-\left(\frac{1}{8}-\frac{1}{4} \right)\right)\\
&=&\displaystyle \frac{1}{4}
\end{array}
\]
}
\begin{problem}
Integrate (definite or indefinite).
\begin{enumerate}
\item $\displaystyle\int\limits_{1}^{8} \frac{t-t^{\frac{1}{3}}+ 2}{ t^{\frac{4}{3}}} \diff t\quad .$

\answer{$- \ln8+\frac{15}{2}$}
\item $\displaystyle\int\limits_{1}^{4} \left(x+\sqrt{x}\right)^2 dx\quad .$

\answer{$\frac{533}{10}$}
\item \label{problemIntegrate(sqrt[3]x-x^(1/2)+1)/xdx}
$\displaystyle\int \frac{\sqrt[3]{x}-x^{\frac{1}{2}}+1 }{x}\diff x$.

\answer{$ -2 \sqrt{x}+3 x^{\frac{1}{3}}+\ln \left|x\right| +C$}

\item \label{problemint(sqrt[3](x)-1)/xdx}

$\displaystyle\int \frac{\sqrt[3]{x}-1 }{x}\diff x$.

\answer{$3x^{\frac{1}{3} }-\ln |x|+C$}
\end{enumerate}

\end{problem}
\subsection{Fundamental Theorem of Calculus Part I}\label{secMPSFTCpart1}
\begin{problem}
Differentiate $f(x)$ using the Fundamental Theorem of Calculus part 1.
\begin{multicols}{2}
\begin{enumerate}[ref={\fcProblemRef}]
\item $\displaystyle f(x) = \int\limits_1^x \sin \left(t^2\right)  \diff t$

\answer{$\displaystyle \sin \left(x^2\right)$} 
\item \label{problemd/dxint_1^x(t-sqrt(t))dt}

$\displaystyle f(x)=\int_{1}^x\left(t-\sqrt{t}\right)\diff t $.

\answer{$x-\sqrt{t}$}
\item \label{problemDifferentiateFTC1int_x^1(2+t^4)^5dt}  ${\displaystyle f(x) = \int\limits_x^1 (2+t^4)^5 \; \diff t}$

\answer{${\displaystyle -\left(2+x^4\right)^5}$} 

\item $\displaystyle f(x)=\int\limits_{0}^{x^2} t^2\diff t $.

\answer{$f'(x)=2x^5$ }

\item \label{problemd/dx(int_(ln x)^(e^x)t^3dt)} $\displaystyle f(x)=\int\limits_{\ln x}^{e^x} t^3\diff t $.

\answer{$f'(x)=e^{4x}-\frac{(\ln x)^3}{x}$ }

\item \label{problemd/dxint_1^x(sqrt(t)-t^(1/3))dt}
$\displaystyle f(x)=\int_{1}^x\left(\sqrt{t}- \sqrt[3]{t}\right)\diff t$.

\answer{$ \sqrt{x}-\sqrt[3]{x}$}
\item \label{problemd/dxint_1^(1/(x+1))sin(t^2)dt}

$\displaystyle f(x)=\int_{1}^{\frac{1}{x+1} }\sin \left(  t^2\right) \diff t$.

\answer{}
\item \label{problemd/dxint_1^(1/(1+x))cos(t^2)dt}

$\displaystyle f(x)= \int_{1}^{\frac{1}{x+1} }\cos  \left(  t^2\right) \diff t$.

\answer{$-\frac{1}{(x+1)^2}\cos \left( \frac{1}{(x+1)^2}\right) $}
\item ${\displaystyle f(x) = \int_{0}^{x^3} \cos^2 t \; \diff t}$

\answer{${\displaystyle 3x^2 \cos^2\left(x^3\right)}$} 

\end{enumerate}
\end{multicols}
\end{problem}
\solution{\ref{problemd/dxint_1^x(t-sqrt(t))dt}
\[
\begin{array}{rcll|l}
\displaystyle \frac{\diff }{\diff x}\left(\int_{1}^x\left(t-\sqrt{t}\right)\diff t\right)&=& x-\sqrt{x}. &&\text{FTC, part 1}
\end{array}
\]
}

\solution{\ref{problemDifferentiateFTC1int_x^1(2+t^4)^5dt} %(Contributed by student Anamaria Ronayne)
We recall that the Fundamental Theorem of Calculus part 1 states that $\frac{\diff}{\diff x}\left(\int_{a}^{x}h(t)dt\right)=h(x)$
where $a$ is a constant. We can rewrite the integral so it has $x$ as the upper limit:
\[
f(x)=\int_{x}^{1}(2+1^4)^5\diff t =-\int_{1}^{x}(2+1^4)^5\diff t\quad.
\]
Therefore
\[
\frac{\diff}{\diff x}\left( -\int_{1}^{x}(2+t^4)^5 \diff t\right)=- \frac{\diff }{\diff x}\left(\int_{1}^{x}(2+t^4)^5\diff t \right)\stackrel{\text{FTC part 1}}{=}
-(2+x^4)^5\quad .
\]

}

\solution{\ref{problemd/dx(int_(ln x)^(e^x)t^3dt)}

\[
\begin{array}{rcl}
f'(x)&=&\displaystyle \frac{\diff }{\diff x}\left( \int\limits_{\ln x}^{e^x} t^3\diff t  \right)\\
&=&\displaystyle \frac{\diff }{\diff x}\left(\int \limits_{\ln x}^{0} t^3\diff t+ \int\limits_{0}^{e^x} t^3\diff t  \right)\\
&=&\displaystyle \frac{\diff }{\diff x}\left(- \int \limits_{0}^{\ln x} t^3\diff t+ \int\limits_{0}^{e^x} t^3\diff t  \right).
\end{array}
\]
The Fundamental Theorem of Calculus part I states that for an arbitrary constant $a$,  $\displaystyle \frac{\diff }{\diff u} \left(\int_{a}^{u}g(t)\diff t \right)=g(u) $ (for a continuous $g$). We use this two compute the two derivatives:
\[
\begin{array}{rcll|l}
\displaystyle \frac{\diff }{\diff x}\left( \int \limits_{0}^{\ln x} t^3\diff t\right)&=&\displaystyle \frac{\diff }{\diff x}\left(\int \limits_{0}^{u} t^3\diff t\right)&&\text{Set } u=\ln x\\
&=&\displaystyle  u^3 \cdot \frac{\diff u}{\diff x}\\
&=&\displaystyle \frac{(\ln x)^3}{x}\\
\displaystyle \frac{\diff }{\diff x}\left( \int \limits_{0}^{e^x} t^3\diff t\right)&= & \displaystyle \frac{\diff }{\diff x}\left( \int \limits_{0}^{w} t^3\diff t\right)&&\text{Set }w=e^{x}\\
&=&\displaystyle w^3\cdot \frac{\diff w}{\diff x}\\
&=&\displaystyle e^{3x}e^{x}=e^{4x}
\end{array}
\]
Finally, we combine the above computations to a single answer. 
\[
f'(x)=e^{4x}-\frac{(\ln x)^3}{x}.
\]
}


\solution{\ref{problemd/dxint_1^x(sqrt(t)-t^(1/3))dt}

\[\begin{array}{rcll|l}
\displaystyle \frac{\diff }{\diff x}\int_1^x\left(\sqrt{t}-\sqrt[3]{t }\right)\diff t&=&\sqrt{x}-\sqrt[3]{x}&&\text{FTC part I} \\
\end{array}
\]
}
\solution{\ref{problemd/dxint_1^(1/(x+1))sin(t^2)dt}
\[
\begin{array}{rcll|l}
\displaystyle \frac{\diff }{\diff x}\int_{1}^{\frac{1}{x+1}} \sin \left(t^2\right)\diff t&=&\displaystyle \frac{\diff}{\diff x}\int_1^{u}\sin (t^2)\diff t   &&u=\frac{1}{x+1}, \text{ use FTC part I, chain rule} \\
&=&\displaystyle \sin\left(u^2\right)\frac{\diff u}{\diff x}\\
&=&\displaystyle \sin\left(\frac{1}{(x+1)^2} \right)\frac{\diff }{\diff x}\left(\frac{1}{x+1} \right)\\
&=&\displaystyle \sin\left(\frac{1}{(x+1)^2} \right)\left(-\frac{1}{(x+1)^2} \right)\\
&=&\displaystyle -\frac{1}{(x+1)^2}\sin\left(\frac{1}{(x+1)^2} \right)\\
\end{array}
\]
}
\solution{\ref{problemd/dxint_1^(1/(1+x))cos(t^2)dt}
\[
\begin{array}{rcll|l}
\displaystyle \frac{\diff }{\diff x}\left(\int_{1}^{\frac{1}{x+1}}\cos\left(t^2\right)\diff t\right)&=&\displaystyle   \frac{\diff }{\diff x}\left(\int_{1}^{u}\cos\left(t^2\right)\diff t\right)&&\text{Set }\frac{1}{x+1}=u\\
&=&\displaystyle   \cos\left(u^2\right)\frac{\diff }{\diff x}(u) &&\text{FTC part I and Chain Rule}\\
&=&\displaystyle   -\frac{1}{(x+1)^2}\cos\left(\frac{1}{(x+1)^2}\right) \\
\end{array}
\]
}


\subsection{Integration with The Substitution Rule}
\label{secMPSintegrationSubstitutionRule}
\subsubsection{Substitution in Indefinite Integrals}
\label{secMPSintegrationSubstitutionRuleIndefinite}
\begin{problem}
Evaluate the indefinite integral. The answer key has not been proofread, use with caution.
\begin{multicols}{3}
\begin{enumerate}[ref={\fcProblemRef}]
\item \label{problemIntegrate(1+3x)^9} $\displaystyle\int (1+3x)^9 \diff x $.

\answer{$\displaystyle \frac{ (1+3x)^{10}}{30} +C$.}
\item $\displaystyle\int \left(\sqrt{2x+1}\right)\diff x $.

\answer{$\displaystyle \frac{(2x+1)^{\frac{3}{2}}}{3}+C$}
\item $\displaystyle\int (3x+2)^{2.4}\diff x $.

\answer{$\displaystyle \frac{(3x+2)^{3.4}}{10.2}+C$}
\item $\displaystyle\int (x-1)\sqrt{2x-x^2} \diff x $.

\answer{$\displaystyle -\frac{ \left(2x-x^2\right)^{\frac{3}{2 }}}{ 3} +C$}
\item $\displaystyle\int x\sqrt{1-x^2} \diff x $.

\answer{$\displaystyle -\frac{\left(1-x^2\right)^{\frac{3}{2}}}{3}+C$}
\item $\displaystyle\int \frac{1+x^2}{\sqrt{3x+x^3}}\diff x $.

\answer{$\displaystyle \frac{2}{3} \left(3x+x^3\right)^{\frac{1}{2}}+C$}
\item $\displaystyle\int (x^2+1)(x^3+3x)^5 \diff x $.

\answer{$\frac{\left(x^3+3x\right)^6}{18} +C$ }
\item $\displaystyle\int \frac{x^2}{\sqrt[3]{1+x^3}} \diff x $.
\answer{$\displaystyle \frac{1}{2} \left(1+x^3 \right)^{\frac{2}{3}} +C$}

\item $\displaystyle\int x^2\left(\sqrt{1+x}\right)\diff x $.


\answer{$\displaystyle \frac{2}{7} (x-1)^{\frac{7}{2}}-\frac{4}{5} (x-1)^{\frac{5}{2}}+\frac{2}{3} (x-1)^{\frac{3}{2}}+C$}
\item $\displaystyle\int x(2x+5)^{2014} \diff x $.


\answer{$\displaystyle \frac{1}{8064} (2 x+5)^{2016}-\frac{1}{1612} (2 x+5)^{2015}+C$}
\item $\displaystyle\int x^3\left(\sqrt{x^2+1}\right) \diff x $.

\answer{$\displaystyle \frac{1}{5} \left(x^{2}+1\right)^{\frac{5}{2}}-\frac{1}{3} \left(x^{2}+1\right)^{\frac{3}{2}}+C$}
\item $\displaystyle\int \sqrt{x}\sin \left(2+x^{\frac{3}{2}}\right) \diff x $.


\answer{$\displaystyle  -\frac{2}{3}\cos \left(2+x^{\frac{3}{2}} \right)+C$}
\item $\displaystyle\int \frac{\cos\left(\frac{\pi}{x}\right)}{x^2} \diff x $.

\answer{$\displaystyle -\frac{\sin\left(\frac{\pi}{x}\right) }{\pi} +C$}
\item $\displaystyle\int \csc^2(2t) \diff t$.

\answer{$\displaystyle -\frac{1}{2}\cot (2t)+C$}
\item $\displaystyle \int \sec (5t) \tan (5t) \diff t $.

\answer{$\displaystyle \frac{1}{5}\sec (5t) +C$}
\item $\displaystyle\int \frac{\cos t}{\sin t} \diff t $.

\answer{$\displaystyle \ln |\sin t|+C$}
\item $\displaystyle\int \tan t \diff t $.

\answer{$\displaystyle -\ln |\cos t|+C $}
\item $\displaystyle\int \cot (2t) \diff t $.

\answer{$\displaystyle  \frac{1}{2}\ln |\sin (2t)|+C$}
\item $\displaystyle\int \frac{\sin \sqrt{t}}{\sqrt{t}} \diff t $.

\answer{$\displaystyle -2\cos \left(\sqrt{t}\right)+C $}
\item $\displaystyle\int \sec^2t \tan^3t \diff t$.

\answer{$\displaystyle \frac{\tan^4 t}{4}+C$}
\item $\displaystyle\int \cos^4t\sin t \diff t$.

\answer{$\displaystyle -\frac{\cos^5 t}{5}$}
\item $\displaystyle\int \frac{\diff t}{\cos^2 t\sqrt{1+\tan t}} $.

\answer{$\displaystyle 2\sqrt{1+\tan t}+C$}
\item $\displaystyle\int \sqrt{\cot t} \csc^2 t\diff t $.

\answer{$\displaystyle -\frac{2}{3}\cot^{\frac{3}{2}} t+C$}
\item $\displaystyle\int \sin t \sec^2(\cos t)\diff t $.

\answer{$\displaystyle -\tan (\cos t)+C$}
\item $\displaystyle\int \sec^3 t \tan t\diff t $.

\answer{$\displaystyle \frac{\sec^3t}{3}+C$}
\item $\displaystyle\int t \sin\left(t^2\right) \diff t $.

\answer{$\displaystyle -\frac{1}{2}\cos \left(t^2\right)+C$}
\end{enumerate}
\end{multicols}


\end{problem}
\solution{\ref{problemIntegrate(1+3x)^9}
We present two solution variants. The variants are equivalent. The only difference between them is that they use two interchangeable notations for differentials. Both variants are acceptable both when taking tests and writing scientific texts. 

\noindent \textbf{Variant I} 
\[
\begin{array}{rcll|l}
\displaystyle\int (1+3x)^9\diff x&=&\displaystyle \int (1+3x)^9 \frac{\diff (3x)}{3} &&\begin{array}{rcl}\text{Set }\\
u&=&1+3x\\
\diff u &=&3\diff x\\
\diff x&=& \frac{1}{3}\diff u\\
\end{array}\\
&=&\displaystyle \int u^9 \frac{\diff u}{3} \\
&=&\displaystyle \frac{1}{3}\int u^9 \diff u \\
&=&\frac{1}{30}u^{10}+C=\frac{(1+3x)^{10}}{30}+C.
\end{array}
\]


\noindent \textbf{Variant II} This variant is equivalent to the previous but uses the differential notation.
\[
\begin{array}{rcll|l}
\displaystyle\int (1+3x)^9\diff x&=&\displaystyle \int (1+3x)^9 \frac{\diff (3x)}{3} &&\text{differentials are linear: } \diff (3x)= (3x)' \diff x= 3\diff x\\
&=&\displaystyle \int (1+3x)^9 \frac{\diff (1+3x)}{3} &&\text{differentials don't change when we add constants}\\
&=&\displaystyle \frac{1}{3}\int u^9 \diff u &&\text{Set } u=1+3x\\
&=&\frac{1}{30}u^{10}+C=\frac{(1+3x)^{10}}{30}+C.
\end{array}
\]
}
\begin{problem}
Evaluate the integral. The answer key has not been proofread, use with caution.
\begin{multicols}{3}
\begin{enumerate}[ref={\fcProblemRef}]
\item $\displaystyle\int \frac{\diff x}{3x+5} $.

\answer{$\frac{1}{3}\ln |3x+5|+C$}

\item $\displaystyle\int \frac{\diff x}{2-3x}$.

\answer{$-\frac{1}{3}\ln |2-3x|+C$}
\item $\displaystyle\int e^x\cos (e^x) \diff x$.

\answer{$ \sin \left(e^x\right)+C$}
\item $\displaystyle\int \frac{(\ln x)^3}{x} \diff x$.

\answer{$ \frac{(\ln x)^4}{4} +C$}

\item \label{probleminte^x(sqrt(e^x+1))dx} ${\displaystyle \int e^x \left(\sqrt{e^x + 1}\right) \diff x}$

\answer{${\displaystyle \frac23 (e^x + 1)^{\frac{3}{2}} + C}$}
\item $\displaystyle\int e^x\sqrt{1-e^x} \diff x$.

\answer{$ -\frac{2}{3}\left(1-e^x\right)^{\frac{3}{2}} +C$}
\item $\displaystyle\int e^{\sin t }\cos t \diff t$.

\answer{$ e^{\sin t}+C$}
\item $\displaystyle\int e^{\cot x}\csc^2x \diff x$.

\answer{$ -e^{\cot x}+C$}
\item $\displaystyle\int \frac{x}{1+x^2} \diff x$. 

\answer{$\frac{1}{2} \ln\left(x^2+1\right)+C $}

\item $\displaystyle\int \frac{x}{2+3x^2} \diff x$. 

\answer{$\frac{1}{6} \ln{}\left(x^{2}+\frac{2}{3}\right)+C$}

\item $\displaystyle\int \frac{x}{\sqrt{1-x^2}} \diff x$. 

\answer{$-\sqrt{1-x^2}+C $}
\item $\displaystyle\int \frac{\cos \left(\ln x\right)}{x} \diff x$.

\answer{$ \sin (\ln x)+C$}

\item \label{problemIntegrate(sin(lnx))/xdx}

$\displaystyle\int \frac{\sin (\ln x)}{x} \diff x$.

\answer{$-\cos(\ln x)+C  $}
\item \label{problemintsin(2x)/(2+cos^2x)dx} $\displaystyle\int \frac{\sin (2x)}{2+\cos^2x}\diff x$.

\answer{$-\ln (2+\cos^2x)+C $}
\item ${\displaystyle \int \frac{\cos x}{\sin x} \diff x}$

\answer{${\displaystyle \ln |\sin x| + C}$}
\item $\displaystyle\int \cot x \diff x$.

\answer{$\ln |\sin x|+C $}

\item $\displaystyle \int \cot \left(\frac{x}{2}\right) \diff x$

\answer{$2\ln\left|\sin \left(\frac{x}{2}\right) \right|+C$}

\item $\displaystyle\int \tan (2x) \diff x$.

\answer{$-\frac{1}{2}\ln|\cos (2x) | +C$}
\item ${\displaystyle \int \frac{x^4 + 3x}{x^2} \diff x}$

\answer{${\displaystyle \frac{x^3}{3} + 3 \ln |x| + C}$}

\item ${\displaystyle \int x^2 e^{x^3} \diff x}$

\answer{${\displaystyle \frac{1}{3}e^{x^3} + C}$}
\item \label{problemIntArctan(x)/(1+x^2)dx} $\displaystyle\int \frac{\Arctan x}{1+x^2} \diff x$. 

\answer{$ \frac{(\Arctan x)^2}{2}+C$}
\end{enumerate}
\end{multicols}
\end{problem}
\solution{\ref{probleminte^x(sqrt(e^x+1))dx}.
\[
\begin{array}{rcll|l}
\displaystyle\int e^x\sqrt{e^x+1} ~ \diff x&=& \displaystyle \int \sqrt{e^x+1} ~ \diff \left(e^x\right)\\
&=&\displaystyle \int \sqrt{e^x+1}~ \diff \left(e^x+1\right)
\displaystyle &&\text{Set }u=e^x+1\\
&=&\displaystyle \int \sqrt{u}~ \diff u\\
&=&\displaystyle \frac{2}{3}u^{\frac{3}{2}}+C\\
&=&\displaystyle \frac{2}{3}\left(e^x+1\right)^{\frac{3}{2}}+C
\end{array}
\]
}

\solution{\ref{problemintsin(2x)/(2+cos^2x)dx}
\[
\begin{array}{rcll|l}
\displaystyle\int \frac{\sin (2x)}{2+\cos^2x}\diff x&=&\displaystyle\int \frac{2\cos x \sin x  \diff x }{2+\cos^2x}&&\text{use } \sin (2x)=2\sin x \cos x\\
&=& \displaystyle\int \frac{2\cos x   \diff(-\cos x) }{2+\cos^2x}&&\text{use }\diff (\cos x)=-\sin x \diff x\\
&=&\displaystyle-\int \frac{2 u   \diff(u) }{2+u^2} &&\text{set }u=\cos x\\
&=&\displaystyle-\int \frac{   \diff\left(2+u^2\right) }{2+u^2}&&\text{use }\diff (u^2+2)=2u\diff u\\
&=&\displaystyle -\int \frac{   \diff z }{z}&&\text{set }z= 2+u^2\\
&=&\displaystyle -\ln |z| +C &&\text{Substitute back }z=u^2+2\\
&=&\displaystyle -\ln (u^2+2) +C &&\begin{array}{l} u^2+2\text{ is positive}\\\Rightarrow \text{omit the abs. value}\\ \text{Substitute back }u=\cos x \end{array}\\
&=&\displaystyle -\ln (\cos^2x+2) +C .
\end{array}
\]

}

\solution{\ref{problemIntegrate(sin(lnx))/xdx}
\[
\begin{array}{rcll|l}
\displaystyle \int\frac{\sin(\ln x)}{x}\diff x &=&\displaystyle \int \sin(\ln x)\diff \left(\ln x\right)&&u=\ln x\\
&=&\displaystyle \int \sin u\diff u\\
&=&-\cos u+C\\
&=&-\cos(\ln x)+C
\end{array}
\]
}


\subsubsection{Substitution in Definite Integrals}
\label{secMPSintegrationSubstitutionRuleDefinite}
\begin{problem}
Evaluate the definite integral. The answer key has not been proofread, use with caution.
\begin{enumerate}[ref={\fcProblemRef}]
\item $\displaystyle\int\limits_{e}^{e^3}\frac{\diff x}{x \sqrt[3]{\ln x}} $.

\answer{$ \frac{3}{2}\left( \sqrt[3]{9} -1\right)$}

\item $\displaystyle\int\limits_{0}^{1}xe^{-x^2} \diff x$.

\answer{$ \frac{1-e^{-1}}{2}$}
\item $\displaystyle\int\limits_{0}^{1}\frac{e^x+1}{e^x+x} \diff x$.

\answer{$\ln(e+1) $}
\item \label{problemIntx/(2x^2+1)} $\displaystyle\int\limits_{1}^{2} \frac{x}{2x^2+1 }  \diff x$.

\answer{$\frac14 \ln 3$}

\item \label{problemIntegratefrom-3to2_x/(1-x^2)dx}

$\displaystyle\int_{-3}^{-2} \frac{x}{1-x^2}\diff x$.

\answer{$\left[-\frac{1}{2}\ln \left|1-x^2\right| \right]_{-3}^{-2}=\frac{1}{2} \ln{}\left(\frac{8}{3}\right) $}
\item \input{\freecalcBaseFolder/modules/substitution-rule/homework/substitution-rule-definite-1-problem-6}

\item $\displaystyle\int\limits_{0}^{\frac{1}4}\frac{x }{\sqrt{1-3x^2}}\diff x$.

\answer{$\frac{1}3\left(1-\sqrt{\frac{13}{16}} \right)$}

\end{enumerate}

\end{problem}
\input{\freecalcBaseFolder/modules/substitution-rule/homework/substitution-rule-definite-1-problem-4-solution}
\solution{\ref{problemIntegratefrom-3to2_x/(1-x^2)dx}

\[\begin{array}{rcll|l}
\displaystyle \int_{-3}^{-2} \frac{x}{1-x^2}\diff x&=&\displaystyle \int\limits_{\tiny \begin{array}{rcl}x&=&-3\\ u&=&-8\end{array}}^{\tiny  \begin{array}{rcl}x&=&-2\\ u&=&-3\end{array}} \frac{1}{u}\left(-\frac{1}{2}\diff u\right)&& \begin{array}{rcl}
u&=&1-x^2\\
\diff u&=&-2x\diff x\\
x\diff x&=&-\frac{1}{2}\diff u
\end{array}\\
&=&\displaystyle -\frac{1}{2}\left[\ln |u|\right]_{-8}^{-3}\\~\\
&=&\displaystyle -\frac{1}{2} \left(\ln|3|-\ln|8| \right)\\~\\
&=&\displaystyle \frac{\ln\left|\frac{8}{3}\right|}{2}
\end{array}
\]

}
\input{\freecalcBaseFolder/modules/substitution-rule/homework/substitution-rule-definite-1-problem-6-solution}



\section{First Applications of Integration}
\subsection{Area Between Curves}\label{secMPSareaBetweenCurves}
\begin{problem}
\begin{enumerate}[ref={\fcProblemRef}]
% Area problems
\item 
\label{problemAreaBetweeny=2x^2,y=4+x^2} Find the area of the region bounded by the curves $y = 2x^2$ and $y = 4 + x^2$.

\answer{$\frac{32}{3}$}
\item \label{problemAreaBetweeny=2-x,x=4-y^2} Find the area of the region bounded by the curves $x = 4 - y^2$ and $y = 2 - x$.

\answer{$\frac92$}
\item \label{problemAreaBetweeny=x^2} Find the area of the region bounded by the curves $y=x^2$ and $y=2x^2+x-2$.

\answer{$\frac{9}{2}$}


\item \label{problemareabetweeny=x^2andy=2x^2+x-2}
\begin{itemize}
\item Sketch the region bounded by the curves $y=x^2$ and $y=2x^2+x-2$.

\psset{xunit=0.5cm, yunit=0.5cm}
\begin{pspicture}(-3.4,-3.4)(3,5.7)
\fcAxesStandardNoFrame{-3.5}{-3.5}{2.5}{5.5}
\fcGrid[linestyle=dashed, linewidth=0.5, linecolor=gray]{-3}{-3}{5}{8}{1}{1}{}
\rput[t](0.9,-0.2){$1$}
\fcLabels{3.5}{5.5}
%\psplot{-3}{2}{x x mul}
%\psplot{-3}{2}{x x mul 2 mul x -2 add add}
\end{pspicture}

\vskip 2cm


\item Find the area of the region.

\answer{$\frac{9}{2}$}
\end{itemize}
\item \label{problemAreaBetween-x^2+2x-1and-2x^2+3x+1}
~
\begin{itemize}
\item Sketch the region bounded by the curves $y=- x^{2}+2 x-1$ and $y=-2 x^{2}+3 x+1$. Make sure to indicate the points where the curves intersect.

\psset{xunit=0.5cm, yunit=0.5cm}
\begin{pspicture}(-3.5,-8.8)(3.7,5.7)
\fcAxesStandard{-3.5}{-8.4}{3.5}{5.5}
\fcGrid[linestyle=dashed, linewidth=0.5, linecolor=gray]{-2}{-8}{5}{13}{1}{1}{}
\rput[t](0.9,-0.2){$1$}
\fcLabels{3.5}{5.5}
%\psplot[linecolor=\fcColorGraph]{-1.3}{2.7}{x x -1 mul mul 2 x mul -1 add add}
%\psplot[linecolor=\fcColorGraph]{-1.3}{2.7}{x x -2 mul mul 3 x mul 1 add add}
\end{pspicture}
\item Find the area of the region.
\end{itemize}
\end{enumerate}

\end{problem}
\solution{\noindent \ref{problemAreaBetweeny=2-x,x=4-y^2}.
$x=4-y^2$ is a parabola (here we consider $x$ as a function of $y$). $y=-x+2$ implies that $x=2-y$ and so the two curves intersect when
\[
\begin{array}{rcl}
4-y^2&=&2-y\\
-y^2+y+2&=&0\\
-(y+1)(y-2)&=&0\\
y&=& -1\text{~or~}2\quad \quad .
\end{array}
\]
As $x=2-y$, this implies that $x=0$ when $y=2$ and $x=3$ when $y=-1$, or in other words the points of intersection are $(0,2)$ and $(3, -1)$. Therefore we the region is the one plotted below. Integrating with respect to $y$, we get that the area is
\[
\begin{array}{rcl}
A&=&\displaystyle \int\limits_{-1}^{2} \left|4-x^2-(-x+2) \right| \diff y = \int\limits_{-1}^2 \left(-y^2+y+2\right)\diff y \\
&=& \displaystyle \left[- \frac{y^3}3 +\frac{y^2}{2}+ 2y\right]_{-1}^2
=-\frac{8}{3}+2+4 -\left(-\frac{(-1)^3}{3} +\frac{ (-1)^2}{2}-2 \right)\\
&=&\displaystyle \frac{9}{2}\quad .
\end{array}
\]
\psset{xunit=0.5cm, yunit=0.5cm}
\begin{pspicture}(-3.500000, -5)(4.500000,5.5)
\psframe*[linecolor=white](-3.500000,-5)(4.500000,5)
\tiny
\pscustom*[linecolor=cyan]{
\psplot[linecolor=\fcColorGraph, plotpoints=1000]{0}{4}{4 x -1 mul add 0.5 exp }
\psplot[linecolor=\fcColorGraph, plotpoints=1000]{4}{3}{4 x -1 mul add 0.5 exp -1 mul }
}
\rput(-1.5,5){$y=- x+2$}
\psplot[linecolor=\fcColorGraph, plotpoints=1000]{-3.000000}{4.000000}{2 x -1 mul add }
%Function formula: - (- x+4)^{1/2}
\psplot[linecolor=\fcColorGraph, plotpoints=1000]{-3.000000}{4.000000}{4 x -1 mul add 0.5 exp -1 mul }
%Function formula: (- x+4)^{1/2}
\rput(2,2){$x=4-y^2$}
\psplot[linecolor=\fcColorGraph, plotpoints=1000]{-3.000000}{4.000000}{4 x -1 mul add 0.5 exp }
\psaxes[arrows=<->, ticks=none, labels=none](0,0) (-3.000000,-4.5)(4.5,4.5) %Function formula: - x+2
\end{pspicture}
}
\solution{\ref{problemareabetweeny=x^2andy=2x^2+x-2}

\textbf{Region plot.}
\psset{xunit=0.5cm, yunit=0.5cm}
\begin{pspicture}(-3,-3)(3,5.7)
\fcAxesStandard{-3.5}{-3.5}{2.5}{5.5}
\pscustom*[linecolor=\fcColorAreaUnderGraph]{%
\psplot{-2}{1}{x x mul}%
\psplot{1}{-2}{x x mul 2 mul x -2 add add}%
}%
\psplot{-3}{2}{x x mul}
\psplot{-3}{2}{x x mul 2 mul x -2 add add}
\fcGrid[linestyle=dashed, linewidth=0.5, linecolor=gray]{-3}{-3}{5}{8}{1}{1}{}
\rput[t](0.9,-0.2){$1$}
\fcLabels{3.5}{5.5}
\end{pspicture}

The intersection between the two parabolas are found via
\[
\begin{array}{rcl}
x^2&=&2x^2+x-2\\
x^2+x-2&=&0\\
(x-1)(x+2)&=&0\\
x=1&& x=-2\\
y=1&&y=4.
\end{array}
\]

\textbf{Area of the region.} 
\[
\begin{array}{rcll|l}
A&=&\displaystyle\int_{1}^{-2}\left|x^2-(2x^2+x-2) \right|\diff x&&x^2>(2x^2+x-2) \text{ for }x\in [-2,1] \text{ (from plot)}\\
&=&\displaystyle\int_{1}^{-2}\left(x^2-(2x^2+x-2) \right)\diff x\\
&=&\displaystyle \left[-\frac{1}{3} x^{3}-\frac{1}{2} x^{2}+2 x \right]_{-2}^1\\
&=&\displaystyle \frac{9}{2}.
\end{array}
\]
}
\solution{\ref{problemAreaBetween-x^2+2x-1and-2x^2+3x+1}

\textbf{Region plot.}

\psset{xunit=0.5cm, yunit=0.5cm}
\begin{pspicture}(-3.5,-8.8)(3.7,5.7)
\fcAxesStandard{-3.5}{-8.4}{3.5}{5.5}
\pscustom*[linecolor=cyan]{
\psplot{-1}{2}{x x -1 mul mul 2 x mul -1 add add}
\psplot{2}{-1}{x x -2 mul mul 3 x mul 1 add add}
}
\fcGrid[linestyle=dashed, linewidth=0.5, linecolor=gray]{-2}{-8}{5}{13}{1}{1}{}
\rput[t](0.9,-0.2){$1$}
\fcLabels{3.5}{7.5}
\psplot[linecolor=\fcColorGraph]{-1.3}{2.7}{x x -1 mul mul 2 x mul -1 add add}
\psplot[linecolor=\fcColorGraph]{-1.3}{2.7}{x x -2 mul mul 3 x mul 1 add add}
\end{pspicture}

The intersections between the two parabolas are found via
\[
\begin{array}{rcl}
-2x^2+3x+1&=&-x^2+2x-1\\
-x^2+x+2&=&0\\
-(x+1)(x-2)&=&0\\
x=-1&\text{or}& x=2\\
y=-4&&y=-1.
\end{array}
\]

\textbf{Area of the region.} 
\[
\begin{array}{rcll|l}
A&=&\displaystyle\int_{-1}^{2}\left|-2x^2+3x+1-(-x^2+2x-1) \right|\diff x&& \begin{array}{l} -2x^2+3x+1>-x^2+2x-1 \\ \text{ for }x\in [-1,2] \text{ (from plot)}\end{array}\\
&=&\displaystyle\int_{-1}^{2}\left(-2x^2+3x+1-(-x^2+2x-1) \right)\diff x\\
&=&\displaystyle\int_{-1}^{2}\left(-x^2+x+2 \right)\diff x\\
&=&\displaystyle \left[-\frac{1}{3} x^{3}+\frac{1}{2} x^{2}+2 x \right]_{-1}^2\\
&=&\displaystyle \left(-\frac{1}{3} 2^{3}+\frac{1}{2} 2^{2}+2 \cdot 2 \right)-\left( -\frac{1}{3} (-1)^{3}+\frac{1}{2} (-1)^{2}+2 (-1) \right)\\
&=&\displaystyle \frac{9}{2}.
\end{array}
\]
}



\subsection{Volumes of Solids of Revolution}\label{secMPSvolumesSolidsRevolution}
\subsubsection{Problems Geared towards the Washer Method}\label{secMPSvolumesSolidsRevolutionWashers}
\begin{problem}
\begin{enumerate}[ref={\fcProblemRef}]
% Volume problems
\item 
\label{problemVolumeRegionBoundedByy=2x^2-x+1,y=x^2+1rotatedAroundx=0} Consider the region bounded by the curves $y = 2x^2-x+1$ and $y =x^2+1$. What is the volume of the solid obtained by rotating this region about the line $x = 0$?

\answer{$\frac{2}{5}\pi$.} 
\item Consider the region bounded by the curves $y = 1-x^2$ and $y =0$. What is the volume of the solid obtained by
rotating this region about the line $y = 0$?

\answer{$\frac{16 \pi}{15}$}
 
\item Consider the region bounded by the curves $y = x^2$ and $x = y^2$. What is the volume of the solid obtained by
rotating this region about the line $x = 2$?

\answer{ $\frac{31 \pi}{30}$}
\item \label{problemVolumeAreay=-x^2+2andy=0rotatedAroundy=0andy=-3}
Set up \textsc{but do not evaluate} an integral to calculate the volume of the solid obtained by rotating the region bounded by $y=-x^2+2$ and $y=0$ about the given line. 

\begin{itemize}
\item The $x$ axis.
\item The line $y=-3$.
\end{itemize}

\item \label{problemVolumeRevolution-x^2+1aroundy=0andy=-4}
Set up \textsc{but do not evaluate} an integral to calculate the volume of the solid obtained by rotating the region bounded by $y=-x^2+1$ and $y=0$ about the given line. 
\begin{itemize}
\item The $x$ axis.
\item The line $y=-4$.
\end{itemize}


\end{enumerate}

\end{problem}
\solution{\ref{problemVolumeRegionBoundedByy=2x^2-x+1,y=x^2+1rotatedAroundx=0}
First, plot $y=2x^2-x+1$ and $y=x^2+1$. 

\psset{xunit=2cm, yunit=2cm}
\begin{pspicture}(-0.5,-0.5)(1.3,3)
\tiny
\fcAxesStandard{-1}{-0.4}{1.3}{3}
\pscustom*[linecolor=\fcColorAreaUnderGraph]{
\psplot[linecolor=\fcColorGraph]{0}{1}{x x 2 mul mul x sub 1 add}
\psplot[linecolor=\fcColorGraph]{1}{0}{x x mul 1 add}
}
\psplot[linecolor=\fcColorGraph]{-0.5}{1.2}{x x 2 mul mul x sub 1 add}
\psplot[linecolor=\fcColorGraph]{-0.5}{1.2}{x x mul 1 add}
\psline[](! 0.7 11 8 div)(! 0.8 11 8 div)
\psline[](! 0.7 25 16 div)(! 0.8 25 16 div)
\psline[](0.75, 0)(! 0.75 25 16 div)
\rput[br](! 0.75 25 16 div){$r_{outer}~$}
\rput[tl](! 0.8 11 8 div 0.05 sub){$~r_{inner}$}
\psline[linewidth=2pt, linecolor=blue](0, 0)(1,0)
\psline[linestyle=dashed](1,2)(1,0)
\fcFullDot[linecolor=blue]{0}{0}
\fcFullDot[linecolor=blue]{1}{0}
\end{pspicture}
\psset{xunit=2cm, yunit=2cm}
\begin{pspicture}(-1,-2.3)(3,3)
\renewcommand{\fcScreenStyle}{x}
\renewcommand{\fcScreen}{[-0.2 -0.2 -1] 0}
\fcStartIIIdScene
\fcAxesIIIdInScene{2.2}{2.2}{2.2}
\fcSurfaceInScene[iterationsV=7, iterationsU=4, linewidth=0.3, arrows=(none)]{0}{0}{1}{360}{[u 2 u u mul mul u sub 1 add v cos mul 2 u u mul mul u sub 1 add v sin mul]}{}
\fcSurfaceInScene[iterationsV=7, iterationsU=4, arrows=(none), linewidth=0.3, colorUV={1 0.5 0.5}, colorVU={1 0.5 0.5}]{0}{0}{1}{360}{[u u u mul 1 add v cos mul u u mul 1 add v sin mul]}{}
\fcSurfaceInScene[iterationsV=1, iterationsU=3, colorUV=cyan, colorVU=cyan, forceForeground=true]{ 0 }{0}{1}{1}{[u u u 2 mul mul u sub 1 add v mul u u mul 1 add 1 v sub mul add 0]}{}
\fcLineIIIdInScene[linewidth=2, linecolor=blue]{[0 0 0]}{[1 0 0]}
\fcFinishIIIdScene[true]
\fcDotIIId[linecolor=blue]{[0 0 0]}
\fcDotIIId[linecolor=blue]{[1 0 0]}
\end{pspicture}
\psset{xunit=2cm, yunit=2cm}
\begin{pspicture}(-1,-2)(3,3)
\renewcommand{\fcScreenStyle}{x}
\renewcommand{\fcScreen}{[-0.2 -0.2 -1] 0}
\fcStartIIIdScene
\fcAxesIIIdInScene{2.2}{2.2}{2.2}
\fcSurfaceInScene[arrows=(none), iterationsV=1, iterationsU=4, colorUV=cyan, colorVU=cyan, linewidth=0.3, forceForeground=true]{ 0 }{0}{1}{1}{[u u u 2 mul mul u sub 1 add v mul u u mul 1 add 1 v sub mul add 0]}{}
\fcSurfaceInScene[arrows=(none), iterationsV=20, iterationsU=1, colorUV=blue, colorVU=blue, linewidth=0.3]{0.75 0.75 mul 2 mul 0.75 sub 1 add}{0}{0.75 0.75 mul 1 add}{360}{[0.75 v cos u mul v sin u mul]}{}
\fcFinishIIIdScene[true]
\fcLineIIId[]{[0.75  0 0]}{[0.75 11 8 div 0]}
\fcPutIIId[br]{[0.75 25 16 div 0]}{$r_{outer}~$}
\fcPutIIId[tl]{[0.75 11 8 div 0.1 sub 0]}{$~r_{inner}$}
\fcLineIIId{[0 0 0]}{[1 0 0]}
\fcDotIIId[linecolor=blue]{[0 0 0]}
\fcDotIIId[linecolor=blue]{[1 0 0]}

\end{pspicture}

\noindent The two curves intersect when 
\[ 
\begin{array}{rcl}
2x^2-x+1&=&x^2+1\\
x^2-x&=&0\\
x(x-1)&=&0\\
x=0 &\text{or}& x= 1.
\end{array}
\]
Therefore the two points of intersection have $x$-coordinates between $x=0$ and $x=1$. Therefore we need we need to integrate the volumes of washers with inner radii $r_{inner}=2x^2-x+1 $, outer radii $r_{outer}=x^2+1$ and infinitesimal heights $\diff x$. The volume of an individual infinitesimal washer is then $ \pi(r^2_{outer}- r^2_{inner})\diff x$
\[
\begin{array}{rcl}
V&=&\displaystyle \int_{0}^1\pi\left(\left(x^2+1\right)^2- \left(2x^2-x+1\right)^2\right)\diff x\\
&=&\displaystyle \pi\int_{0}^1\left(-3 x^{4}+4 x^{3}-3 x^{2}+2 x \right)\diff x\\
&=&\displaystyle \pi\left[-\frac{3}{5} x^{5}+x^{4}- x^{3}+x^{2} \right]_0^1 \\
&=& \displaystyle \frac{2}{5} \pi.
\end{array}
\]
}
\solution{\ref{problemVolumeAreay=-x^2+2andy=0rotatedAroundy=0andy=-3}
First, we plot the 2d region. The two curves intersect when $-x^2+2=0$, i.e., when $x=\pm \sqrt{2}$


\hfil \hfil \begin{pspicture}(-6.2,-3.2)(6.2,3.2)\tiny
\pscustom*[linecolor=\fcColorAreaUnderGraph]{
\psplot[linecolor=\fcColorGraph]{2 sqrt -1 mul}{2 sqrt}{x x mul -1 mul 2 add}
}
\newcommand{\theFuN}{x x mul -1 mul 2 add\space}%
\psplot[linecolor=\fcColorGraph]{-2}{2}{x x mul -1 mul 2 add}
\psline[linecolor=\fcColorGraph](-2,0)(2,0)
\fcAxesStandardNoFrame{-2}{-3.1}{2}{3}
\rput[t](! 2 sqrt -0.1){$\sqrt{2}$}
\rput[t](! 2 sqrt -1 mul -0.1){$-\sqrt{2}$}
\psline[arrows=<->](1, 0)(! 1 1 dict begin /x 1 def \theFuN end)
\rput[l](1.1,0.3){cross-section rad., $y=0$}
\psline[arrows=<->](-1, -3)(! -1 1 dict begin /x -1 def \theFuN end)
\rput[r](-1.1,-1){cross-section rad., $y=-3$}
\psline[linecolor=green](-2,-3)(2,-3)
\end{pspicture}

\textbf{Rotation about $y=0$. }

Unless explicitly stated in the problem, a 3d plot of the solid is not required in the solution. Nevertheless generating such a plot helps to understand the problem. 

To generate a 3d plot of the solid, we draw the circular cross-sections of the solid of revolution. By hand, this can be done roughly by drawing ovals (circles look like ovals when observed at an angle) centered at the axis about which we revolve the 2d-region. We include a computer-generated plot below; the plot's precision is above what is expected on an exam.

\hfil \hfil \begin{pspicture}(-3,-3)(4.2,3.2)%
\newcommand{\theFun}{u u mul -1 mul 2 add\space}%
\renewcommand{\fcScreenStyle}{x}
\renewcommand{\fcScreen}{[-1 -0.2 -0.75] -1}
\fcStartIIIdScene%
\fcAxesIIIdFullInScene{-3}{-3}{-3}{3}{3}{3}%
\fcSurfaceInScene[arrows=(none), iterationsV=15, iterationsU=8, colorVU={1 0.5 0.5}]{2 sqrt -1 mul 0.001 add}{0}{2 sqrt -0.001 add}{360}{[u v cos \theFun mul v sin \theFun mul]}{}%
\fcSurfaceInScene[arrows=(none), iterationsV=4, iterationsU=3, colorUV={0.3 0.7 1}, forceForeground=true]{2 sqrt -1 mul }{0}{2 sqrt}{1}{[u \theFun v mul  0]}{}%
\fcFinishIIIdScene[true]%
\fcPutIIId{[3 0 0]}{$x$}
\fcPutIIId{[0 3 0]}{$y$}
\fcPutIIId{[0 0 3]}{$z$}
\end{pspicture}

The volume of a solid (and in particular, of a solid of revolution) is computed by integrating the area $A(x)=\pi(\text{radius cross-section})= \pi (-x^2+2)^2 $ of the cross-section of the solid. Therefore the volume $V$ equals
\[
\begin{array}{rcll|l}
V&=&\displaystyle \int_{a}^bA(x)\diff x\\
&=&\displaystyle\int_{-\sqrt{2}}^{\sqrt{2}}\pi (-x^2+2)^2  \diff x\\
&=&\displaystyle\pi\left[\frac{1}{5} x^{5}-\frac{4}{3} x^{3}+4 x\right]_{-\sqrt{2}}^{\sqrt 2}&&\text{step not required by problem}\\
&=&\displaystyle \pi \frac{64}{15}\sqrt{2}&&\text{step not required by problem.}
\end{array}
\]


\textbf{Rotation about $y=-3$. } The cross-section of this solid of revolution is a washer with inner radius $ 3$ and outer radius $-x^2+2-(-3)=5-x^2$. Therefore the area of the cross-section is $\pi (5-x^2)^2-\pi 3^2$ and the volume is computed via

\[
\begin{array}{rcll|l}
V&=&\displaystyle \int_{a}^bA(x)\diff x\\
&=&\displaystyle\int_{-\sqrt{2}}^{\sqrt{2}} \pi \left( (5-x^2)^2- 3^2\right)  \diff x\\
&=&\displaystyle\pi\left[\frac{1}{5} x^{5}-\frac{10}{3} x^{3}+16 x  \right]_{-\sqrt{2}}^{\sqrt 2}&&\text{step not required by problem}\\
&=&\displaystyle \pi  \frac{304}{15}\sqrt{2}&&\text{step not required by problem.}
\end{array}
\]
\hfil \hfil 
\begin{pspicture}(-4,-9)(5,4.2)%
\newcommand{\theFuN}{u u mul -1 mul 2 add\space}%
\renewcommand{\fcScreenStyle}{x}
\fcStartIIIdScene%
\fcAxesIIIdFullInScene{-3}{-9}{-4}{3}{3}{4}%
\renewcommand{\fcScreen}{[-1 -0.2 -0.75] -1}
\fcSurfaceInScene[arrows=(none), iterationsV=15, iterationsU=8, colorVU={0.7 0.2 0.2}, colorUV={0.7 0.2 0.2}]{2 sqrt -1 mul }{0}{2 sqrt -0.001 add}{360}{[u v cos 3 mul -3 add v sin 3 mul]}{}%
\fcSurfaceInScene[arrows=(none), iterationsV=15, iterationsU=8, linecolor=black,colorUV={1 0.5 0.5}, colorVU={1 0.5 0.5}]{2 sqrt -1 mul 0.01 add}{0}{2 sqrt -0.01 add}{360}{[u v cos \theFuN 3 add mul -3 add v sin \theFuN 3 add mul]}{}%
\fcSurfaceInScene[arrows=(none), iterationsV=1, iterationsU=8, colorUV={0.3 0.7 1}, forceForeground=true]{2 sqrt -1 mul }{0}{2 sqrt}{1}{[u \theFuN v mul  0]}{}%
\fcLineIIIdInScene[linecolor=green, linewidth=2]{[-6 -3 0]}{[6 -3 0]}
\fcCurveIIIdInScene[linecolor=red, arrows=(none), linewidth=2]{2 sqrt -1 mul}{2 sqrt}{[1 dict begin /u t def u \theFuN 0 end]}
\fcFinishIIIdScene[true]%
\fcPutIIId{[3 0 0]}{$x$}
\fcPutIIId{[0 3 0]}{$y$}
\fcPutIIId{[0 0 3]}{$z$}
\end{pspicture}
}
\solution{\ref{problemVolumeRevolution-x^2+1aroundy=0andy=-4}

First, we plot the 2d region. The two curves intersect when $-x^2+1=0$, i.e., when $x=\pm 1$.


\hfil \hfil \begin{pspicture}(-6.2,-4.2)(6.2,3.2)\tiny
\pscustom*[linecolor=\fcColorAreaUnderGraph]{
\psplot[linecolor=\fcColorGraph]{-1}{1}{x x mul -1 mul 1 add}
}
\newcommand{\theFuN}{x x mul -1 mul 1 add\space}%
\psplot[linecolor=\fcColorGraph]{-2}{2}{x x mul -1 mul 1 add}
\psline[linecolor=\fcColorGraph](-2,0)(2,0)
\fcAxesStandardNoFrame{-2}{-3.1}{2}{3}
\rput[t](! 1  -0.1){$1$}
\rput[t](! -1 -0.1){$-1$}
\psline[arrows=<->](0.5, 0)(! 0.5 1 dict begin /x 0.5 def \theFuN end)
\rput[l](0.6,0.3){cross-section rad., $y=0$}
\psline[arrows=<->](-0.5, -4)(! -0.5 1 dict begin /x -0.5 def \theFuN end)
\rput[r](-0.6,-2){cross-section rad., $y=-4$}
\rput[b](-2,-4){axis $y=-4$}
\psline[linecolor=green](-2,-4)(2,-4)
\end{pspicture}

\textbf{Rotation about $y=0$. }

\hfil \hfil \begin{pspicture}(-3,-3)(4.2,3.2)%
\newcommand{\theFun}{u u mul -1 mul 1 add\space}%
\renewcommand{\fcScreenStyle}{x}
\renewcommand{\fcScreen}{[-1 -1 -4] -1}
\fcStartIIIdScene%
\fcAxesIIIdFullInScene{-2}{-2}{-6}{2}{2}{6}%
\fcSurfaceInScene[arrows=(none), iterationsV=15, iterationsU=8, colorVU={1 0.5 0.5}]{-0.99}{0}{0.99}{360}{[u v cos \theFun mul v sin \theFun mul]}{}%
\fcSurfaceInScene[arrows=(none), iterationsV=4, iterationsU=3, colorUV={0.3 0.7 1}, forceForeground=true]{-1}{0}{1}{1}{[u \theFun v mul  0]}{}%
\fcFinishIIIdScene[true]%
\fcPutIIId{[3 0 0]}{$x$}
\fcPutIIId{[0 3 0]}{$y$}
\fcPutIIId{[0 0 3]}{$z$}
\end{pspicture}

The volume of a solid (and in particular, of a solid of revolution) is computed by integrating the area $A(x)=\pi(\text{radius cross-section})^2= \pi (-x^2+1)^2 $ of the cross-section of the solid. Therefore the volume $V$ equals
\[
\begin{array}{rcll|l}
V&=&\displaystyle \int_{a}^bA(x)\diff x\\
&=&\displaystyle\int_{-1}^{1}\pi (-x^2+1)^2  \diff x\\
&=&\displaystyle\pi\left[\frac{1}{5} x^{5}-\frac{2}{3} x^{3}+x\right]_{-1}^{1}&&\text{step not required by problem}\\
&=&\displaystyle \pi \frac{16}{15}&&\text{step not required by problem.}
\end{array}
\]


\textbf{Rotation about $y=-4$. } The cross-section of this solid of revolution is a washer with inner radius $ 4$ and outer radius $-x^2+1-(-4)=5-x^2$. Therefore the area of the cross-section is $\pi (5-x^2)^2-\pi 4^2$ and the volume is computed via

\[
\begin{array}{rcll|l}
V&=&\displaystyle \int_{a}^bA(x)\diff x\\
&=&\displaystyle\int_{-1}^{1} \pi \left( (5-x^2)^2- 4^2\right)  \diff x\\
&=&\displaystyle\pi\left[\frac{1}{5} x^{5}-\frac{10}{3} x^{3}+9 x   \right]_{-1}^{1}&&\text{step not required by problem}\\
&=&\displaystyle \frac{176}{15} \pi &&\text{step not required by problem.}
\end{array}
\]
\hfil \hfil 
\begin{pspicture}(-4,-9)(5,4.2)%
\newcommand{\theFuN}{u u mul -1 mul 1 add\space}%
\renewcommand{\fcScreenStyle}{x}
\fcStartIIIdScene%
\fcAxesIIIdFullInScene{-3}{-9}{-4}{3}{3}{4}%
\renewcommand{\fcScreen}{[-1 -1 -4] -1}%
\fcSurfaceInScene[arrows=(none), iterationsV=15, iterationsU=8, colorVU={0.7 0.2 0.2}, colorUV={0.7 0.2 0.2}]{1 -1 mul }{0}{1 -0.001 add}{360}{[u v cos 4 mul -4 add v sin 4 mul]}{}%
\fcSurfaceInScene[arrows=(none), iterationsV=15, iterationsU=8, linecolor=black,colorUV={1 0.5 0.5}, colorVU={1 0.5 0.5}]{-1 0.01 add}{0}{1 -0.01 add}{360}{[u v cos \theFuN 4 add mul -4 add v sin \theFuN 4 add mul]}{}%
\fcSurfaceInScene[arrows=(none), iterationsV=1, iterationsU=8, colorUV={0.3 0.7 1}, forceForeground=true]{-1}{0}{1}{1}{[u \theFuN v mul  0]}{}%
\fcLineIIIdInScene[linecolor=green, linewidth=2]{[-6 -4 0]}{[6 -4 0]}
\fcCurveIIIdInScene[linecolor=red, arrows=(none), linewidth=2]{-1}{1}{[1 dict begin /u t def u \theFuN 0 end]}
\fcFinishIIIdScene[true]%
\fcPutIIId{[3 0 0]}{$x$}
\fcPutIIId{[0 3 0]}{$y$}
\fcPutIIId{[0 0 3]}{$z$}
\end{pspicture}
}



\subsubsection{Problems Geared towards the Cylindrical Shells Method}\label{secMPSvolumesSolidsRevolutionShells}
\begin{problem}
\begin{enumerate}
% Volumes by cylindrical shells
\item Consider the region bounded by the curves $y=\sqrt{x}$, $x=0$, $y=2$. Use the method of cylindrical shells to find the volume of
the solid obtained by rotating this region about the $x$-axis. 
\answer{$8\pi$}
\item Consider the region bounded by the curves $y=x^2$ and $y=2-x^2$. Use the method of cylindrical shells to find the volume of the solid obtained by rotating this region about the line $x=1$.
\answer{$16\frac{\pi}{3}$}
\end{enumerate}

\end{problem}
\end{document}