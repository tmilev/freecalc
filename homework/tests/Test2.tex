\documentclass{article}
\ProvidesPackage{homework-problems-UMB}
\addtolength{\hoffset}{-3.5cm}
\addtolength{\textwidth}{6.8cm}
\addtolength{\voffset}{-3cm}
\addtolength{\textheight}{6cm}
\usepackage{../homework-problems} %warning folder paths are relative to the file that uses the includepackage

\renewcommand{\answer}[1]{\iftoggle{answers}{ \hfill{~} \rotatebox{180}{\tiny answer: #1}}{} }
\renewcommand{\pointsii}[1]{}
\renewcommand{\hiddenanswer}{\answer}
\renewcommand{\points}[1]{\item}
\renewcommand{\pointsii}[1]{\item}
\renewcommand{\Arctan}{\arctan}
\renewcommand{\Arcsin}{\arcsin}
\renewcommand{\Arccot}{\operatorname{arccot}}

\toggletrue{solutions}
\toggletrue{answers}
\renewcommand{\fcProblemRef}{\theproblem.\theenumi}
\renewcommand{\fcSubProblemRef}{\theproblem.\theenumi.\theenumii}
\newcommand{\answerBox}[1][]{Answer:\\
\fbox{\begin{minipage}{0.5\textwidth}
\ifx\reservedToken#1\reservedToken
~\\~\\~\\%
\else
#1%
\fi
\end{minipage}}}

\newcommand{\hide}[1]{}
\newtheorem{problem}{Problem}
\pagestyle{empty}
\begin{document}
\begin{center}
\Large
Test 2 \\ Math 141 Calculus II \\ \normalsize Summer 2017 \\ Instructor(s): Todor Milev
\end{center}
\noindent \textbf{Name:\underline{~~~~~~~~~~~~~~~~~~~~~~~} } \hfill{~}



\noindent The exam is closed textbook. \textbf{No electronic devices are allowed during the exam. }
\begin{problem}Compute the limit.
\[\displaystyle \lim_{x \to 0}\frac{\sin{}\left(-16 x\right)}{\ln{}\left(42 x+1\right)} \]
\end{problem}

\vfill
\answerBox 
\hfill \begin{tabular}{c|c|c|c|c||c}
Problem&1 &2&3&4& $\sum$\\ \hline
Score&&&&&\\ \hline
Max&10&10&10&10&40
\end{tabular} 

\newpage
\begin{problem}
Determine whether the integral is convergent. If not, indicate whether it converges to $\infty, -\infty$ or neither. If convergent, find its value.
\[\int _{0}^{9}\frac{\ln{}\left(\frac{1}{3} x\right)}{x^{5}} \diff x\]
\end{problem}
\newpage
\begin{problem}
Find the (arc)length of the curve given by: 
\[
\left|\begin{array}{rcl} x&=&t^{\frac{2}{3}}- \ln{}t\\ y&=&2\sqrt{6} t^{\frac{1}{3}}\\ \end{array}\right., t\in \left(1, 3\right)
\]
\end{problem}
\psset{xunit=0.4cm, yunit=0.4cm}
\begin{pspicture}(-3,-1)(5,12)
\fcAxesStandard{-3}{-1}{5}{12}
\fcXTickWithLabel{1}{$1$}
\parametricplot[plotpoints=500, linecolor=red]{0.03}{10}{t 2 3 div exp t ln sub 2 6 sqrt mul t 1 3 div exp mul}
\parametricplot[plotpoints=500, linecolor=blue, linewidth=4pt]{1}{3}{t 2 3 div exp t ln sub 2 6 sqrt mul t 1 3 div exp mul}
\end{pspicture} 
\vfill
\answerBox
\newpage
\begin{problem}
Find the area locked by the curve given in polar coordinates by $r=\sin{}\left(5 \theta\right)+4$
\end{problem}
\psset{xunit=0.4cm, yunit=0.4cm}
\begin{pspicture}(-6,-6)(6,6)
\newcommand{\theFun}{5 t mul sin 4 add dup t cos mul exch t sin mul}
\pscustom*[linecolor=cyan]{
\parametricplot[plotpoints=500]{0}{360}{\theFun }
}
\pscustom[linecolor=red]{
\parametricplot[plotpoints=500]{0}{360}{\theFun }
}
\fcAxesStandardNoFrame{-6}{-6}{6}{6}
\fcXTickWithLabel{1}{$1$}
\end{pspicture}
\vfill
\answerBox
\end{document}