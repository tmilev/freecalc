\documentclass{article}
\usepackage{amsmath, amsfonts, amssymb, verbatim, hyperref}
\usepackage{../pstricks-commands}
%\usepackage{pst-solides3d}
%\usepackage{pst-3dplot}
\usepackage{../homework-problems}


\renewcommand{\Re}{\mathrm{Re~}}
\renewcommand{\Im}{\mathrm{Im~}}
\newtheorem{problem}{Problem}

%\title{Exam II\\ Math 140 Calculus I \\Instructor: Todor Milev}
%\date{April 9 2012}
\begin{document}
\begin{center}
\Large
Review problems Test 2\\ Math 140 Calculus I \\ \normalsize Instructor: Todor Milev
\end{center}
%\textbf{Name:} 


\noindent The exam is closed books, no calculators will be allowed. The material covered will be the material from Lecture 8 to Lecture 17 (including Lectures 8 and 17). The time for work will be 50 minutes. The problems on the exam will be similar to the problems in the review sheet. You will be asked a theoretical question (see the last problem).
\begin{problem}(Lecture 8)
Solve the equation.
\begin{enumerate}
\item $e^{4x}+3e^{2x}-4=0$. \answer{$x=0$}
\item $4^{3x}-2^{3x+2}-5=0$. \answer{$x=\log_25$}
\end{enumerate}
\end{problem}
\begin{problem}(Lecture 8)
Find the inverse function $f^{-1}$. Plot roughly by hand $y=f(x)$. Using the plot of $y=f(x)$, plot roughly by hand $f^{-1}(x)$. Indicate the relationship between the graph of $f(x)$ and $f^{-1}(x)$.
\begin{enumerate}
\item $f(x)= x^2+2x-2$,\quad \quad \quad $ x\geq -1$. 
\answer{
$f^{-1}(x)=\sqrt{x+3}-1$
\psset{xunit=0.2cm, yunit=0.2cm}
\begin{pspicture}(-3, -5)(6,5) 
\psframe*[linecolor=white](-3,-5)(6,5) 
\tiny 
\psaxes[ticks=none, labels=none]{<->}(0,0)(-3,-4.5)(6,4.5)
\psLabels{6}{5}
%Function formula: (x+3)^{1/2}-1 
\psplot[linecolor=\psColorGraph, plotpoints=1000]{-3}{6}{-1 3 x add 0.5 exp add }
%Function formula: x^{2}+2 x-2 
\psplot[linecolor=\psColorGraph, plotpoints=1000]{-1}{2}{-2 x 2 mul add x 2 exp add }
\end{pspicture} 
}
\item $f(x)= x^2+x-2$, \quad \quad \quad $ x\geq -\frac{1}{2}$.

\answer{
$f^{-1}(x)=\frac{ \sqrt{4 x+9}-1}2$
\psset{xunit=0.2cm, yunit=0.2cm}
\begin{pspicture}(-2.25, -5)(4,5) 
\psframe*[linecolor=white](-2.25,-5)(4,5) 
\tiny 
\psaxes[ticks=none, labels=none]{<->}(0,0)(-2.25,-4.5)(4,4.5)
\psLabels{4}{5}
%Function formula: 1/2 (4 x+9)^{1/2}-1/2 
\psplot[linecolor=\psColorGraph, plotpoints=1000]{-2.25}{4}{-0.5 9 x 4 mul add 0.5 exp 0.5 mul add }
%Function formula: x^{2}+x-2 
\psplot[linecolor=\psColorGraph, plotpoints=1000]{-0.5}{2}{-2 x add x 2 exp add }
\end{pspicture} 
}
\end{enumerate}
\end{problem}

\begin{problem}(Lectures 10-16)
Compute the derivative of the function.
\begin{itemize}
\item $f(x)=\frac{1+x }{1+\frac{2}x}$.\answer{$\frac{x^{2}+4 x+2}{(2+x)^{2}}$}
\item $f(x)=\frac{1+x }{1+\frac{3}x}$.\answer{$\frac{x^{2}+6 x+3}{(3+x)^{2}}$}
\end{itemize}
\end{problem}

\begin{problem}(Lectures 10-16) Compute the derivative of the function.
\begin{enumerate}
\item $2^{3^x}$.
\answer{$2^{3^{x}} 3^{x} (\ln{}2)  (\ln{}3) $}
\item $3^{2^x}$.
\answer{$ 3^{2^{x}} 2^{x}(\ln{}2)(\ln{}3)$}
\end{enumerate}
\end{problem}
\begin{problem}Compute the derivative of the function.
\begin{enumerate}
\item $\sec^2 (3x^2)$. \answer{$12 \frac{x\sin{}(3 x^{2}) }{(\cos{}(3 x^{2}))^{3}}$}
\item $\csc^2 (3x^2)$. \answer{$-12 \frac{ x  \cos{}(3 x^{2}) }{(\sin{}(3 x^{2}))^{3}}$}
\end{enumerate}
\end{problem}

\begin{problem}(Lecture 15)
Use implicit differentiation to express $\frac{dy}{dx}$ via $y $ and $x$, where $x$ and $y$ satisfy the following relation.
\begin{enumerate}
\item  $x^4(x+y)=y^2(3x-y)$.
\item $2x^3+x^2y-xy^3=2$.
\end{enumerate}
\end{problem}

\begin{problem} (Lecture 15)
Use implicit differentiation to find an equation of the tangent line to the curve at the given point.
\begin{itemize}
\item $x^{2/3}+y^{2/3}=4$ at $(-3\sqrt{3}, 1)$. \answer{$y=\frac{\sqrt{3}}{3}x+4$ }
\item $x^2y^3+x^2-y^2=1$ at $(1,1)$. \answer{$y=-4x+5$}
\end{itemize}
\end{problem}

\begin{problem} (Lecture 17)
A wedge of radius 2 (depicted below) is folded into a cone cup. Find the maximum possible volume of the cone cup.
\psset{xunit=0.5cm, yunit=0.5cm}
\begin{pspicture}(-5, -5)(5,5) 
\psframe*[linecolor=white](-5,-5)(5,5) 
\tiny 
\pscustom*[linecolor=cyan!30]{ \psparametricplot[algebraic] {2.35619}{7.06858} {0+1*cos(t)| 0+1*sin(t)} \psline(0.707107, 0.707107)(0, 0)(-0.707107, 0.707107)}

\psparametricplot[algebraic,linecolor=blue]{2.35619}{7.06858}{cos(t)| sin(t)} 
\psline[linecolor=red](0.707107, 0.707107)(0, 0)(-0.707107, 0.707107)

\rput[t](0.4, 0.2){$r$}
\rput[lb](0.8,0.8){$B$}
\rput[rb](-0.8,0.8){$A$}
\rput[b](0,0.3){$O$}
\end{pspicture} 
\end{problem}

\begin{problem}(Lecture 14)
\begin{itemize}
\item State the quotient rule for computing the derivative of $\left(\frac{f}{g}\right)'$. Derive the quotient rule 
using the chain rule, the negative power rule and the product rule.
\item State the power rule for computing the derivative
$\left(x^r\right)'$ for an arbitrary real number $r$ and $x>0$. Derive the power rule using the chain rule, the rule $\left(e^{x}\right)'=e^x$, the constant multiple derivative rule and the logarithm derivative rule $(\ln x)'=\frac{1}x$.
\end{itemize}
\end{problem}

%\begin{tabular}{c|c|c|c|c|c|c|c|c||c}
%Problem&1 &2&3&4&5&6&7&8& $\sum$\\ \hline
%Score&&&&&&&&&\\ \hline
%Max&20&20&20&20&20&10&20&20&150
%\end{tabular} 
\end{document}