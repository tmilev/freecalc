\documentclass{article}
\usepackage{amsmath, amsfonts, amssymb, verbatim, hyperref}
\usepackage{auto-pst-pdf}
\usepackage{pst-plot}
%\usepackage{pst-solides3d}
%\usepackage{pst-3dplot}
\usepackage{../pstricks-commands}

\usepackage{../homework-problems}


\renewcommand{\Re}{\mathrm{Re~}}
\renewcommand{\Im}{\mathrm{Im~}}
\newtheorem{problem}{Problem}


%\title{Exam II\\ Math 140 Calculus I \\Instructor: Todor Milev}
%\date{April 9 2012}
\begin{document}
\begin{center}
\Large
Solution: problem 5, review problems Test 1\\ Math 140 Calculus I \\ \normalsize Instructor: Todor Milev
\end{center}

%\noindent \textbf{Name:} \hfill{~}
%\begin{tabular}{c|c|c|c|c|c|c|c|c||c}
%Problem&1 &2&3&4&5&6&7&8& $\sum$\\ \hline
%Score&&&&&&&&&\\ \hline
%Max&20&20&20&20&20&10&20&20&150
%\end{tabular} 


\begin{problem}~
\begin{enumerate}
\item (a) Solve the equation $x^2+13x+41=1$.  (b) Use the intermediate value theorem to prove that the equation 
\begin{equation}\label{eqProblemRevew5.1}
x^2+13x+41=\sin  x
\end{equation} has at least two solutions, lying between the two numbers found in (a).
\end{enumerate}
\end{problem}
\textbf{Solution.} 

\noindent (a) 
\begin{eqnarray*}
x^2+13x+41&=&1\\
x^2+13x+40&=&0\\
(x+5)(x+8)&=&0\quad .
\end{eqnarray*}
Therefore the two solutions are $x_1=-5$ and $x_2=-8$.

\noindent (b) Consider the function
\[
f(x)=x^2+13x+41-\sin x\quad. 
\]
Finding a solution to \eqref{eqProblemRevew5.1} is equivalent to finding a solution to $f(x)=0$. Our strategy for proving $f(x)=0$ has a solution consists in finding a number $a$ such that $f(a)<0$ and a number $b$ such that $f(b)>0$, and then using the Intermediate Value Theorem (IVT) with $N=0$. 

Let 
\[
g(x)=x^2+13x+41,
\]
and so $f(x)=g(x)-\sin x$. We have no techniques for evaluating $\sin x$ without calculator, but we do have all knowledge necessary to evaluate $g(x)$. Indeed, from high school we know that the lowest point of the parabola $g(x)$ is located at $x=-\frac{13}2=-6.5$. Then $g(-6.5)= -1.25$. Therefore 
\[
f(-6.5)=g(-6.5)-\sin(-6.5)= g(-6.5)+\sin (6.5)=-1.25+\sin 6.5 \leq -0.25, 
\]
where for the very last inequality we use the fact that $\sin 6.5< 1 $ (remember $\sin t\leq 1$ for all real values of $t$).

On the other hand, 
\[f(-5)= g(-5)-\sin (-5) = 1+\sin 5> 0\]
as $\sin 5 >-1$ (remember $\sin t\geq -1$ for all real values of $t$). Therefore $f(-5)>0$ and $f(-6.5)<0$ and by the Intermediate Value Theorem (IVT) $f(x)=0$ has a solution in the interval $x\in (-6.5, -5)$. 
 
Proving $f(x)=0$ has a solution in the interval $x\in (-8, -6.5)$ is similar and we leave it to the student. 

Below is a computer generated plot of the function with the use of which we can visually verify our answer.

\psset{xunit=1cm, yunit=1cm}
\begin{pspicture}(-9, -5)(1,5) 
\psframe*[linecolor=white](-9,-5)(1,5) 
\tiny 
\psaxes[ticks=none, labels=none]{<->}(0,0)(-9,-4.5)(1,4.5)
\psLabels{1}{5}
\psXTickWithLabel{-6.5}{$-6.5$}
\psXTickWithLabel{-8}{$-8$}
\psXTickWithLabel{-5}{$-5$}
%Function formula: (x)^{2}+40+13 (x) 
\rput(-4.5,-2){$y=x^{2}+13 x+40$} 
\psplot[linecolor=grey!30, plotpoints=1000]{-9}{-4}{x 13 mul 40 x 2 exp add add }
\rput(-6.5,3){$y=x^{2}+41+13 x- \sin x$} 
\psplot[linecolor=\psColorGraph, plotpoints=1000]{-9}{-4}{x 57.29578 mul sin -1 mul x 13 mul 41 x 2 exp add add add }

\end{pspicture} 
\end{document}