\documentclass{article}
\usepackage{amsmath, amsfonts, amssymb, verbatim, hyperref}
\usepackage{auto-pst-pdf}
\usepackage{pst-plot}
%\usepackage{pst-solides3d}
%\usepackage{pst-3dplot}
\usepackage{pstricks}
\usepackage{rotating}

\usepackage{multicol}
\addtolength{\hoffset}{-3.5cm}
\addtolength{\textwidth}{6.8cm}
\addtolength{\voffset}{-3.3cm}
\addtolength{\textheight}{6.3cm}
\renewcommand{\Re}{\mathrm{Re~}}
\renewcommand{\Im}{\mathrm{Im~}}
\newcommand{\doublebrace}[4]{\left\{\begin{array}{ll} #1 & #2 \\#3 & #4  \end{array} \right.}
\newcommand{\triplebrace}[6]{\left\{\begin{array}{ll} #1 & #2 \\#3 & #4  \\#5 & #6\end{array} \right.}
\newtheorem{problem}{Problem}
\newcommand{\psHollowDot}[2]{
\pscircle*[fillcolor=white, linecolor=red](#1, #2){0.07}
\pscircle*[fillcolor=white, linecolor=white](#1, #2){0.04}
}
\newcommand{\psHollowDotBlue}[2]{
\pscircle*[fillcolor=white, linecolor=blue](#1, #2){0.07}
\pscircle*[fillcolor=white, linecolor=white](#1, #2){0.04}
}
\newcommand{\psFullDot}[2]{
\pscircle*[fillcolor=white, linecolor=red](#1, #2){0.07}
}
\newcommand{\psFullDotBlack}[2]{
\pscircle*[fillcolor=white, linecolor=black](#1, #2){0.07}
}
\newcommand{\psLabelXOne}{\psline(1, -0.1)(1,0.1) \rput[t](1, -0.2 ) {\footnotesize $1$} }
\newcommand{\psLabelYOne}{\psline(-0.1, 1)(0.1, 1) \rput[r](-0.2, 1 ) {\footnotesize $1$} }

\title{
Review problems for May 9 Exam\\
Math 140 \\
}
\date{}
\begin{document}
\maketitle
\textbf{Instructor.} Todor Milev

The exam will be closed book, no calculators allowed. Try to solve all theoretical problems without using the lectures/textbook. If you get stuck, read the lectures/textbook, but close the textbook/lectures when going back to the problem. Finally, compare what you wrote with the lectures/textbook.

For problem answer key, look for updates on piazza.com.
\begin{problem} Find the maxima and minima of the function in the interval.
\begin{enumerate}
\item $f(x)=\frac{1-x}{2+x+x^{2}}$, $x\in [-2, 4]$. \hfill{~}  \rotatebox{180}{
answer: $\min= f(3) =-\frac{1}{7}$, $\max = f(-1)=1$.
}
\item $f(x)=\frac{2-x}{3+x+x^{2}} $, $x\in [-2, 6]$. \hfill{~}  \rotatebox{180}{
answer: $\min= f(5) =-\frac{1}{11}$, $\max = f(-1)=1$.
}
\end{enumerate}
\end{problem}
\begin{problem}~

\begin{enumerate}
\item $f(x)=x(x-1)^2(x-3)$, $x\in(-\infty, \infty) $.
\item $f(x)=x(x-2)^2(x-3)$, $x\in(-\infty, \infty) $.
\end{enumerate}
For the above examples do the following.
\begin{itemize}
\item Find the zeroes of the function. \hfill{~} 
\rotatebox{180}{
answer: $0,1, 3 $ and answer: $0,2,3$.
}
\item Find the intervals where the function is increasing and decreasing.  

 \hfill{~} 
\rotatebox{180}{answer
$(-\infty, \frac{11 - \sqrt{73}}{8})$-decreasing $(\frac{11 - \sqrt{73}}{8}, 1)  $-increasing, $(1, \frac{11 + \sqrt{73}}{8})$-decreasing, $(\frac{11 + \sqrt{73}}{8}, \infty)$-increasing.
}

 \hfill{~} 
\rotatebox{180}{answer:
$(-\infty, \frac{13 - \sqrt{73}}{8})$-decreasing $(\frac{13 - \sqrt{73}}{8}, 2)  $-increasing, $(2, \frac{13 + \sqrt{73}}{8})$-decreasing, $(\frac{13 + \sqrt{73}}{8}, \infty)$-increasing.
}
\item Find the values of $x$ for which the function has local minima and maxima. If they exist, find the absolute minima and maxima of the function.
\hfill{~} 
\rotatebox{180}{answer: local minima at 
$ x=\frac{11 \pm \sqrt{73}}{8})$, local maximum  at $1$,}

\hfill{~} 
\rotatebox{180}{
 absolute minimum $f(\frac{11 -\sqrt{73}}{8})= -\frac{827}{512}-\frac{73}{512} \sqrt{73}$, has no absolute maximum.
}


\hfill{~} 
\rotatebox{180}{answer:
local minima at 
$ x=\frac{13 \pm \sqrt{73}}{8})$, local maximum  at $2$,}

\hfill{~} 
\rotatebox{180}{
 absolute minimum $f(\frac{13 -\sqrt{73}}{8})=
 -\frac{827}{512}-\frac{73}{512} \sqrt{73}$, has no absolute maximum.
}


\item Find the intervals of concavity (up and down)  of the function. Find the inflection points of the function.

\hfill{~} 
\rotatebox{180}{answer:
$\left(-\infty, \frac{15 -\sqrt{57}}{12}\right)$-concave up, $\left( \frac{15 -\sqrt{57}}{12}, \frac{15 +\sqrt{57}}{12}\right)$-concave down, $\left(\frac{15 +\sqrt{57}}{12}\right)$-concave up. Inflection points at $\frac{15 \pm\sqrt{57}}{12}$.
}

\hfill{~} 
\rotatebox{180}{ answer:
$\left(-\infty, \frac{21 -\sqrt{57}}{12}\right)$-concave up, $\left( \frac{21 -\sqrt{57}}{12}, \frac{21 +\sqrt{57}}{12}\right)$-concave down, $\left(\frac{21 +\sqrt{57}}{12}\right)$-concave up. Inflection points at $\frac{21 \pm\sqrt{57}}{12}$.
}

\item Plot the function roughly. 
\rotatebox{180}{
answer:
\psset{xunit=0.5cm, yunit=0.5cm}
\begin{pspicture}(-5, -5)(5,5) 
\psframe*[linecolor=white](--5,-5)(5,5) 
\psaxes[ticks=none, labels=none]{<->}(0,0)(-1,-4.5)(4.5,4.5)\tiny
%Function formula: (x)^{4}+7 ((x)^{2})-3 (x)-5 ((x)^{3}) 
\rput(1,3){$y=x(x-1)^2(x-3) $} 
\psplot[linecolor=red, plotpoints=1000]{-0.5}{3.2}{x 3 exp -5 mul x -3 mul x 2 exp 7 mul x 4 exp add add add }
\end{pspicture} 
\begin{pspicture}(-5, -5)(5,5) 
\psframe*[linecolor=white](-5,-5)(5,5) 
\psaxes[ticks=none, labels=none]{<->}(0,0)(-1,-4.5)(4.5,4.5)\tiny
%Function formula: (x)^{4}+16 ((x)^{2})-7 ((x)^{3})-12 (x) 
\rput(1,3){$y=x(x-2)^2(x-3)$} 
\psplot[linecolor=red, plotpoints=1000]{-0.2}{3.5}{x -12 mul x 3 exp -7 mul x 2 exp 16 mul x 4 exp add add add }
\end{pspicture} 
}

\end{itemize}
\end{problem}
\begin{problem}~
\begin{enumerate}
\item 
Find the largest possible area of a triangle inscribed in circle of radius $r$ such that one of the sides of the triangle is a diameter of the circle.

\hfill{~} 
\rotatebox{180}{answer: $r^2$.}

\item Find the largest possible area of a trapezoid inscribed in circle of radius $r$ such that one of the bases of the trapezoid is a diameter of the circle.
\hfill{~} 
\rotatebox{180}{answer: $\frac{3 \sqrt{3}}{2}r^2 $.}

\end{enumerate}
\psset{xunit=0.5cm, yunit=0.5cm}
\begin{pspicture}(-5, -5)(5,5) 
\psframe*[linecolor=white](-5,-5)(5,5) 
\psplot[linecolor=black, plotpoints=1000]{-2}{2}{x 2 exp -1 mul 4 add 0.5 exp }
\psplot[linecolor=black, plotpoints=1000]{-2}{2}{x 2 exp -1 mul 4 add 0.5 exp -1 mul}
\psline[linecolor=red](-2,0)(2,0)(1, 1.732050808)(-2,0)
\pscircle*[linecolor=red](0,0){0.03}
\end{pspicture} 
\begin{pspicture}(-5, -5)(5,5) 
\psframe*[linecolor=white](-5,-5)(5,5) 
\psplot[linecolor=black, plotpoints=1000]{-2}{2}{x 2 exp -1 mul 4 add 0.5 exp }
\psplot[linecolor=black, plotpoints=1000]{-2}{2}{x 2 exp -1 mul 4 add 0.5 exp -1 mul}
\psline[linecolor=red](-2,0)(2,0)(1, 1.732050808)(-1, 1.732050808)(-2,0)
\pscircle*[linecolor=red](0,0){0.03}
\end{pspicture} 
\end{problem}
\begin{problem}
Evaluate the definite integral 
\begin{enumerate}
\item $\int\limits_{1}^{8} \frac{t^{2}-t^{2/3}+2}{t^{5/3}} dt\quad .$
\item $\int\limits_{1}^{32} \frac{t^{2}-t^{-2/5}-2}{t^{3/5}} dt\quad .$
\end{enumerate}
\end{problem}
\begin{problem}
Evaluate the indefinite integral 
\begin{enumerate}
\item $\int (\sin (3x+\pi/3)+e^{-x/3} ) ~dx$.
\item $\int (\cot (4x+\pi/4)+e^{-x/4} )~dx$.
\end{enumerate}
\end{problem}
\begin{problem}
Evaluate the indefinite integral 
\begin{enumerate}
\item $\int \tan x ~dx$. \hfill{~} 
\rotatebox{180}{answer: $-\ln |\sin x|+C$.}
\item $\int \cot x ~dx$. \hfill{~} 
\rotatebox{180}{answer: $\ln |\cos x|+C$.}
\end{enumerate}
\end{problem}
\begin{problem}
Evaluate the definite integral 
\begin{enumerate}
\item $\int\limits_{1}^{2} \frac{x}{(1-3x)(1+3x) }  ~dx$.
\hfill{~} 
\rotatebox{180}{answer: $\frac{1}{18} \ln \frac{8}{35}$.}
\item $\int\limits_{0}^1 \frac{x}{(2-x)(2+x) }  ~dx$.
\hfill{~} 
\rotatebox{180}{answer: $\frac{1}{2} \ln \frac{4}{3}$.}
\end{enumerate}
\end{problem}
\begin{problem}~
\begin{enumerate}
\item State and prove the Mean Value Theorem.
\item Define Riemann sum.
\item Define definite integral.
\item State the Fundamental Theorem of Calculus (both parts).
\item State the substitution rule.
\end{enumerate}
\end{problem}

\end{document}