\documentclass%
%[handout]
{beamer}
\newcommand{\currentLecture}{3}
\newcommand{\semester}{Summer 2015}
% % % % % % % %
% % % % % % % %
% % % % % % % %
%IMPORTANT
%compiles with
%pdflatex -shell-escape
%IMPORTANT
% % % % % % % %
% % % % % % % %
% % % % % % % %




\mode<presentation>
{
\useinnertheme{rounded}
\useoutertheme{infolines}
\usecolortheme{orchid}
\usecolortheme{whale}
}
\usepackage[english]{babel}
\usepackage[latin1]{inputenc}
\usepackage[all,cmtip]{xy}
\usepackage{times}
\ProvidesPackage{example-templates}
\usepackage{ifthen}
\usepackage{amsmath}
\usepackage{amssymb}
\usepackage{cancel}
\usepackage{comment}
\usepackage{enumerate}
\RequirePackage{xstring}

%%%%%%%%%%%%%%%%%%%%%%%%%%%%%%%%%%%%%%%%%%
%
% List of commands in this document
%
%
% \logdiffbaseandexp
% \logdifftwouponedown
% \chainruley
% \productrulefofx
% \quotientruley
% \limitradical  (broken)
% \limitsub
% \chainrulefofx
% \infinitelimit
% \limitfactor
% \newtonsmethod
% \constantmultiple
% \chainruletwice
% \youWillNotBeTested
% \Arcsin 
% \Arccos 
% \Arctan 
% \Arccot 
%
%%%%%%%%%%%%%%%%%%%%%%%%%%%%%%%%%%%%%%%%%%

\newcommand{\youWillNotBeTested}{\begin{frame}You will not be tested on the material in the following slide.\end{frame}}
\newcommand{\Arcsin}{\sin^{-1}}
\newcommand{\Arccos}{\cos^{-1}}
\newcommand{\Arctan}{\tan^{-1}}
\newcommand{\Arccot}{\cot^{-1}}

%
%  An example of logarithmic differentiation of a function with a 
%  variable base and exponent.
%  #1 is the base.
%  #2 is the exponent.
%  #3 is the derivative of the natural logarithm of the base.
%  #4 is the derivative of the exponent.
%  #5 is (base)(exponent)' + (exponent)(base)' after simplification.
%
\newcommand{\logdiffbaseandexp}[5]{
\begin{example}[Variable base and exponent]
\abovedisplayskip=0pt
\belowdisplayskip=0pt
\abovedisplayshortskip=0pt
\belowdisplayshortskip=0pt
\begin{align*}
\text{Differentiate}\quad \alert<handout:0| 13>{y} %
 & \alert<handout:0| 13>{=} %
\alert<handout:0| 13>{%
#1^{#2}%
}.%
\uncover<2->{%
\intertext{
Take logarithms of both sides:%
}
}%
\uncover<2->{%
\ln y
}%
 & \uncover<2->{ = } %
\uncover<2->{%
\ln #1^{\alert<handout:0| 3>{#2}}%
}\\%
\uncover<3->{%
\alert<handout:0| 4-5>{\ln y}%
}%
 & \uncover<3->{ = } %
\uncover<3->{%
\alert<handout:0| 6-7>{%
\alert<handout:0| 3>{#2} \ln #1%
}.}%
\uncover<4->{%
\intertext{
Differentiate implicitly with respect to $x$:%
}%
}%
\uncover<5->{%
\alert<handout:0| 5>{\frac{1}{y} y'}%
}%
 & \uncover<4->{ = } %
\uncover<7->{%
\alert<handout:0| 7>{%
\left( #2 \right) \alert<handout:0| 8-9>{\frac{\diff}{\diff x} \left( \ln #1 \right)} + \left( \ln #1 \right)\alert<handout:0| 10-11>{\frac{\diff}{\diff x}\left( #2 \right)} %
}%
}\\%
\uncover<8->{%
\frac{1}{\alert<handout:0| 12>{y}} y'%
}%
 & \uncover<8->{ = } %
\uncover<8->{%
( #2 ) \alert<handout:0| 8-9>{\left( \uncover<9->{ #3 }\right)} + \left( \ln #1 \right) \alert<handout:0| 10-11>{ \left( \uncover<11->{ #4 } \right) }
}\\%
\uncover<12->{%
y'%
}%
 & \uncover<12->{ = } %
\uncover<12->{%
\alert<handout:0| 12-13>{y} \left( #5 \right)%
}\\%
 & \uncover<13->{ = } %
\uncover<13->{%
\alert<handout:0| 13>{#1^{#2}} \left( #5 \right).%
}%
\end{align*}
\end{example}
}


%
%  An example of logarithmic differentiation of a function.
%  It looks as follows:
%
%  Differentiate y = (#1 #2)/#3.
%  Take logarithms of both sides:
%  ln y = ln((#1 #2)/#3)
%  ln y = ln#1 + ln#2 - ln#3
%  ln y = #4 + #5 - #6
%  Differentiate implicitly with respect to x:
%  (1/y)y' = #7 + #8 - #9
%  y' = y(#7 + #8 - #9)
%  y' = ((#1 #2)/#3)(#7 + #8 - #9)
%
\newcommand{\logdifftwouponedown}[9]{
\begin{example}[Logarithmic Differentiation%
]
\abovedisplayskip=0pt
\belowdisplayskip=0pt
\abovedisplayshortskip=0pt
\belowdisplayshortskip=0pt
\begin{align*}
\text{Differentiate}\quad \alert<handout:0| 18>{y} %
 & \alert<handout:0| 18>{=} %
\alert<handout:0| 18>{%
\frac{#1 #2}{#3}%
}.%
\uncover<2->{%
\intertext{
Take logarithms of both sides:%
}
}%
\uncover<2->{%
\ln y
}%
 & \uncover<2->{ = } %
\uncover<2->{%
\ln \frac{\alert<handout:0| 3-4>{#1}\alert<handout:0| 5-6>{#2}}{\alert<handout:0| 7-8>{#3}}%
}\\%
\uncover<2->{%
\ln y
}%
 & \uncover<2->{ = } %
\uncover<2->{%
\ln \alert<handout:0| 3-4>{#1} + \ln \alert<handout:0| 5-6>{#2} -  \ln \alert<handout:0| 7-8>{#3}%
}\\%
\uncover<3->{%
\alert<handout:0| 9-10>{\ln y}%
}%
 & \uncover<3->{ = } %
\uncover<3->{%
\alert<handout:0| 3-4,11-12>{%
\left( \uncover<4->{#4}\right) %
}%
\alert<handout:0| 5-6>{%
\uncover<6->{+} \alert<handout:0| 13-14>{\left( \uncover<6->{#5}\right)} %
}%
\alert<handout:0| 7-8>{%
\uncover<8->{-} \alert<handout:0| 15-16>{\left( \uncover<8->{#6}\right)} %
}%
}%
\uncover<9->{%
\intertext{
Differentiate implicitly with respect to $x$:%
}%
}%
\uncover<10->{%
\alert<handout:0| 10>{\frac{1}{\alert<handout:0| 17>{y}} y'}%
}%
 & \uncover<9->{ = } %
\uncover<9->{%
\alert<handout:0| 11-12>{\left( \uncover<12->{#7} \right)} + %
\alert<handout:0| 13-14>{\left( \uncover<14->{#8} \right)} - %
\alert<handout:0| 15-16>{\left( \uncover<16->{#9} \right)} %
}\\%
\uncover<17->{%
y'%
}%
 & \uncover<17->{ = } %
\uncover<17->{%
\alert<handout:0| 17-18>{y} \left( #7 + #8 - #9 \right)%
}\\%
 & \uncover<18->{ = } %
\uncover<18->{%
\alert<handout:0| 18>{\frac{#1 #2}{#3}} \left( #7 + #8 - #9 \right)%
}%
\end{align*}
\end{example}
}


%
%  An example of a derivative with the Product Rule, using the symbol f(x).
%  It looks as follows:
%  
%  Differentiate f(x) = #1 #2.
%  Product Rule: f'(x) = (#1)(d/dx)(#2) + (#2)(d/dx)(#1)
%   = (#1)(#4) + (#2)(#3)
%   = #5.
%
%  #6 appears in the subtitle of the example.
%
\newcommand{\productrulefofx}[6]{%
\begin{example}[Product Rule%
\ifthenelse{\equal{#6}{0}}%
{}%
{, #6}%
]%
\abovedisplayskip=0pt
\belowdisplayskip=0pt
\abovedisplayshortskip=0pt
\belowdisplayshortskip=0pt
\begin{align*}
\text{Differentiate}\quad f(x) & = #1 #2.\\%
\uncover<2->{%
\text{Product Rule:}\quad f'(x)%
}%
& \uncover<2->{%
 = \left( #1 \right) \alert<handout:0| 3-4>{\frac{\diff}{\diff x}\left( #2 \right)} + \left( #2 \right) \alert<handout:0| 5-6>{\frac{\diff}{\diff x}\left( #1 \right)}%
}\\%
& \uncover<3->{%
 = \left( #1 \right) \alert<handout:0| 3-4>{\left(\uncover<4->{ #4 }\right)} + \left( #2 \right) \alert<handout:0| 5-6>{\left( \uncover<6->{#3} \right)}%
}\\%
& \uncover<7->{%
 = #5.%
}%
\end{align*}
\end{example}
}


%
%  An example of a derivative with the Constant Multiple Rule. 
%  It looks as follows:
%  
%  Find the derivative of #1 = #2.
%   #1 = (#3)(#4).
%   d#1/dx = (d/dx)((#3)(#4))
% Constant Multiple Rule: = (#3)(d/dx)(#4)
%   = (#3)(#5)
%   = #6.
%
%  #7 appears in the subtitle of the example.
%
\newcommand{\constantmultiple}[7]{%
\begin{example}[Constant Multiple Rule%
\ifthenelse{\equal{#7}{0}}%
{}%
{, #7}%
]%
\abovedisplayskip=0pt
\belowdisplayskip=0pt
\abovedisplayshortskip=0pt
\belowdisplayshortskip=0pt
\begin{align*}
\text{Find the derivative of}\quad #1 & = #2.\\%
\uncover<2->{%
#1 %
}%
& \uncover<2->{%
 = \left( #3\right)\left( #4\right).
}\\%
\uncover<3->{%
\frac{\diff #1}{\diff x} %
}%
& \uncover<3->{%
 = \frac{\diff}{\diff x}\left[ \alert<handout:0| 4>{\left( #3\right)}\left( #4\right)\right]
}\\%
\uncover<4->{%
\text{Constant Multiple Rule:}\quad %
}%
& \uncover<4->{%
 =  \alert<handout:0| 4>{\left( #3\right)}\alert<handout:0| 5-6>{\frac{\diff}{\diff x}\left( #4\right)}
}\\%
& \uncover<5->{%
 =  \left( #3\right)\alert<handout:0| 5-6>{\left( \uncover<6->{#5}\right)}
}\\%
& \uncover<7->{%
 =  #6.
}%
\end{align*}
\end{example}
}


%
%  An example of a derivative with the Quotient Rule, using the symbol y.
%  It looks as follows:
%  
%  Differentiate y = #1 / #2.
%  Quotient Rule: dy/dx = ((#2)(d/dx)(#1)-(#1)(d/dx)(#2))/(#2)^2
%   = ((#2)(#3)-(#1)(#4))/(#2)^2
%   = #5
%   = #6.
%
%  #7 appears in the subtitle of the example.
%
\newcommand{\quotientruley}[7]{%
\begin{example}[Quotient Rule%
\ifthenelse{\equal{#7}{0}}%
{}%
{, #7}%
]%
\abovedisplayskip=0pt
\belowdisplayskip=0pt
\abovedisplayshortskip=0pt
\belowdisplayshortskip=0pt
\begin{align*}
\text{Differentiate}\quad y & = \frac{#1}{#2}.%
\uncover<2->{%
\intertext{Quotient Rule:}%
}%
%&\\%
\uncover<2->{%
\frac{\diff y}{\diff x}%
}%
& \uncover<2->{%
 = \frac%
{\left( #2 \right) \alert<handout:0| 3-4>{\frac{\diff}{\diff x}\left( #1 \right)} - \left( #1 \right) \alert<handout:0| 5-6>{\frac{\diff}{\diff x}\left( #2 \right)}}%
{\left( #2\right)^2}%
}\\%
& \uncover<3->{%
 = \frac%
{\left( #2 \right) \alert<handout:0| 3-4>{\left(\uncover<4->{ #3 }\right)} - \left( #1 \right) \alert<handout:0| 5-6>{\left( \uncover<6->{#4} \right)}}%
{\left( #2\right)^2}%
}\\%
& \uncover<7->{%
 = #5%
}\\%
& \uncover<8->{%
 = #6.%
}%
\end{align*}
\end{example}
}

%
%  An example of an indefinite integral with the Substitution Rule.
%  It looks as follows:
%  
%  Find \int (#1, with nothing substituted for UU and VV).
%  Let u = #2
%  Then du = #3.
%  Therefore #4 = #5.
%  Substitute: \int (#1, with the alert command for u and du 
%          substituted for UU and VV respectively) 
%  = \int (#6, with the alert command for u and du substituted for UU and VV)
%  = (#7, with u substituted for UU) + C
%  = (#8, with #2 substituted for UU) + C
%
%  #9 appears in the subtitle of the example.
%
\newcommand{\subrule}[9]{%
\begin{example}[Substitution Rule%
\ifthenelse{\equal{#9}{0}}%
{}%
{, #9}%
]%
\abovedisplayskip=0pt
\belowdisplayskip=0pt
\abovedisplayshortskip=0pt
\belowdisplayshortskip=0pt
\begin{align*}
\text{Find}\quad \int %
 \noexpandarg\exploregroups\StrSubstitute{\StrSubstitute{#1}{UU}{3}}{VV}{6-7}\noexploregroups\expandarg. & \\%
\uncover<2->{%
\text{Let}\quad\alert<handout:0| 2-3,8,13>{u}%
}%
& \uncover<2->{%
\alert<handout:0| 2-3,8,13>{ = \uncover<3->{#2.}}%
}\\%
\uncover<4->{%
\text{Then}\quad \alert<handout:0| 4-5>{\diff u}%
}%
& \uncover<4->{%
\alert<handout:0| 4-5>{ = \uncover<5->{#3}}%
}\\%
\uncover<6->{%
\alert<handout:0| 6-7,9>{#4}%
}%
& \uncover<6->{%
\alert<handout:0| 6-7,9>{ = \uncover<7->{#5.}}%
}\\%
\uncover<8->{%
\text{Substitute:}\quad \int%
 \noexpandarg\exploregroups\StrSubstitute{\StrSubstitute{#1}{UU}{8}}{VV}{9}\noexploregroups\expandarg}%
& \uncover<8->{= \alert<handout:0| 10-11>{\int\noexpandarg\exploregroups\StrSubstitute{\StrSubstitute{#6}{UU}{8}}{VV}{9}\noexploregroups\expandarg %
}}\\%
& \uncover<10->{\alert<handout:0| 10-11>{%
 = \uncover<11->{\noexpandarg\exploregroups \StrSubstitute{#7}{UU}{\alert<handout:0| 13>{u}}\noexploregroups\expandarg} \uncover<12->{\alert<handout:0| 12>{+C}}%
}}\\%
& \uncover<13->{%
 = \noexpandarg\exploregroups \StrSubstitute{#8}{UU}{\alert<handout:0| 13>{#2}}\noexploregroups\expandarg +C.%
}%
\end{align*}
\end{example}
}

%
%  An example of a definite integral with the Substitution Rule.
%  There are nine arguments to the function.  The ninth is a string of four 
%  groups of the form {AA}{BB}{CC}{DD} where AA is the lower limit of 
%  integration, BB is the upper limit of integration, CC is the lower limit
%  of integration with respect to u, and DD is the upper limit of integration
%  with respect to u.
%  It looks as follows:
%  
%  Find \int_{AA}^{BB} (#1, with nothing substituted for UU and VV).
%  Let u = #2
%  Then du = #3.
%  #4 = #5.
%  When x = AA, u = CC.
%  When x = BB, u = DD.
%  Substitute: \int_{AA}^{BB} (#1, with the alert command for u and du 
%          substituted for UU and VV respectively) 
%  = \int_{CC}^{DD} (#6, with the alert command for u and du substituted for UU and VV)
%  = [#7, with u substituted for UU]_{CC}^{DD}
%  = #8.
%
%
\newcommand{\subruledefbounds}[9]{%
\begin{example}[Substitution Rule, Definite Integral%
]%
\abovedisplayskip=0pt
\belowdisplayskip=0pt
\abovedisplayshortskip=0pt
\belowdisplayshortskip=0pt
\begin{align*}
\text{Find}\quad \int%
_{\StrMid{#9}{1}{1}}%
^{\StrMid{#9}{2}{2}} %
 \noexpandarg\exploregroups\StrSubstitute{\StrSubstitute{#1}{UU}{3}}{VV}{6-7}\noexploregroups\expandarg. & \\%
\uncover<2->{%
\text{Let}\quad\alert<handout:0| 2-3,8-12>{u}%
}%
& \uncover<2->{%
\alert<handout:0| 2-3,8-12>{ = \uncover<3->{#2.}}%
}\\%
\uncover<4->{%
\text{Then}\quad \alert<handout:0| 4-5>{\diff u}%
}%
& \uncover<4->{%
\alert<handout:0| 4-5>{ = \uncover<5->{#3}}%
}\\%
\uncover<6->{%
\alert<handout:0| 6-7,13>{#4}%
}%
& \uncover<6->{%
\alert<handout:0| 6-7,13>{ = \uncover<7->{#5.}}%
}\\%
\uncover<8->{%
\alert<handout:0| 8-9,14>{\text{When } x = \StrMid{#9}{1}{1}, \quad u }%
}%
& \uncover<8->{%
\alert<handout:0| 8-9,14>{ = \uncover<9->{\StrMid{#9}{3}{3}.}}%
}\\%
\uncover<10->{%
\alert<handout:0| 10-11,15>{\text{When } x = \StrMid{#9}{2}{2}, \quad u }%
}%
& \uncover<10->{%
\alert<handout:0| 10-11,15>{ = \uncover<11->{\StrMid{#9}{4}{4}.}}%
}\\%
\uncover<12->{%
\text{Substitute:}\quad \int%
_{\alert<handout:0| 14>{\StrMid{#9}{1}{1}}}%
^{\alert<handout:0| 15>{\StrMid{#9}{2}{2}}} %
 \noexpandarg\exploregroups\StrSubstitute{\StrSubstitute{#1}{UU}{12}}{VV}{13}\noexploregroups\expandarg}%
& \uncover<12->{= \alert<handout:0| 16-17>{{\int}%
_{\uncover<14->{\alert<handout:0| 14>{
\StrMid{#9}{3}{3}}}}%
^{\uncover<15->{
\alert<handout:0| 15>{
\StrMid{#9}{4}{4}}}} %
\noexpandarg\exploregroups\StrSubstitute{\StrSubstitute{#6}{UU}{12}}{VV}{13}\noexploregroups\expandarg %
}}\\%
& \uncover<16->{\alert<handout:0| 16-17>{%
 = {\left[ \uncover<17->{%
\noexpandarg\exploregroups\StrSubstitute{#7}{UU}{u}\noexploregroups\expandarg %
}\right]}_{\StrMid{#9}{3}{3}}^{\StrMid{#9}{4}{4}}%
}}\\%
& \uncover<18->{%
 = #8.
}%
\end{align*}
\end{example}
}


%
%  An example of a definite integral with the Substitution Rule.
%  There are nine arguments to the function.  The ninth is a string of two 
%  groups of the form {AA}{BB} where AA is the lower limit of 
%  integration and BB is the upper limit of integration.  
%  It looks as follows:
%  
%  Find \int_{AA}^{BB} (#1, with nothing substituted for UU and VV).
%  Let u = #2
%  Then du = #3.
%  #4 = #5.
%  Substitute: \int (#1, with the alert command for u and du 
%          substituted for UU and VV respectively) 
%  = \int (#6, with the alert command for u and du substituted for UU and VV)
%  = #7, with u substituted for UU
%  = #8.
%  Therefore int_{AA}^{BB} (#1, with nothing substituted for UU and VV)
%      = [#8]_{AA}^{BB}
%  = #9.
%
%
\newcommand{\subruledefvar}[9]{%
\begin{example}[Substitution Rule, Definite Integral%
]%
\abovedisplayskip=0pt
\belowdisplayskip=0pt
\abovedisplayshortskip=0pt
\belowdisplayshortskip=0pt
\begin{align*}
\text{Find}\quad \int%
_{\StrMid{#9}{1}{1}}%
^{\StrMid{#9}{2}{2}} %
 \noexpandarg\exploregroups\StrSubstitute{\StrSubstitute{#1}{UU}{3}}{VV}{6-7}\noexploregroups\expandarg. & \\%
\uncover<2->{%
\text{Let}\quad\alert<handout:0| 2-3,8,12>{u}%
}%
& \uncover<2->{%
\alert<handout:0| 2-3,8,12>{ = \uncover<3->{#2.}}%
}\\%
\uncover<4->{%
\text{Then}\quad \alert<handout:0| 4-5>{\diff u}%
}%
& \uncover<4->{%
\alert<handout:0| 4-5>{ = \uncover<5->{#3}}%
}\\%
\uncover<6->{%
\alert<handout:0| 6-7,9>{#4}%
}%
& \uncover<6->{%
\alert<handout:0| 6-7,9>{ = \uncover<7->{#5.}}%
}\\%
\uncover<8->{%
\text{Substitute:}\quad \int%
 \noexpandarg\exploregroups\StrSubstitute{\StrSubstitute{#1}{UU}{8}}{VV}{9}\noexploregroups\expandarg}%
& \uncover<8->{= \alert<handout:0| 10-11>{{\int}%
\noexpandarg\exploregroups\StrSubstitute{\StrSubstitute{#6}{UU}{8}}{VV}{9}\noexploregroups\expandarg %
}}\\%
& \uncover<10->{%
 \alert<handout:0| 10-11>{ = \uncover<11->{%
\noexpandarg\exploregroups{\StrSubstitute{#7}{UU}{\alert<handout:0| 12>{u}}}\noexploregroups\expandarg%
}}%
  \uncover<12->{%
 = \noexpandarg\exploregroups{\StrSubstitute{#7}{UU}{\alert<handout:0| 12>{#2}}}\noexploregroups\expandarg.%
}%
}\\%
\uncover<13->{%
\text{Therefore}\quad \int%
_{\StrMid{#9}{1}{1}}%
^{\StrMid{#9}{2}{2}} %
 \noexpandarg\exploregroups\StrSubstitute{\StrSubstitute{#1}{UU}{0}}{VV}{0}\noexploregroups\expandarg}%
& \uncover<13->{%
 = \left[%
 \noexpandarg\exploregroups{\StrSubstitute{#7}{UU}{#2}}\noexploregroups\expandarg%
\right]%
_{\StrMid{#9}{1}{1}}%
^{\StrMid{#9}{2}{2}} %
}\\%
& \uncover<14->{%
 = #8.
}%
\end{align*}
\end{example}
}

%
%  An example of a derivative with the Chain Rule, using the symbol y.
%  It looks as follows:
%  
%  Differentiate y = #1.
%  Let u = #2
%  Then y = #3
%  Chain Rule: dy/dx = (dy/du)(du/dx)
%  = (#4, with u substituted for UU)(#5)
%  = #6, with #2 substituted for UU
%
%  #7 appears in the subtitle of the example.
%
\newcommand{\chainruley}[7]{%
\begin{example}[Chain Rule%
\ifthenelse{\equal{#7}{0}}%
{}%
{, #7}%
]%
\abovedisplayskip=0pt
\belowdisplayskip=0pt
\abovedisplayshortskip=0pt
\belowdisplayshortskip=0pt
\begin{align*}
\text{Differentiate}\quad y & = #1.\\%
\uncover<2->{%
\text{Let}\quad\alert<handout:0| 2-3,8-10>{u}%
}%
& \uncover<2->{%
\alert<handout:0| 2-3,8-10>{ = \uncover<3-| handout:0>{#2.}}%
}\\%
\uncover<4->{%
\text{Then}\quad \alert<handout:0| 6-7>{y}%
}%
& \uncover<4->{%
\alert<handout:0| 6-7>{ = \uncover<4-| handout:0>{#3.}}%
}\\%
\uncover<5->{%
\text{Chain Rule:}\quad%
\frac{\diff y}{\diff x}%
}%
& \uncover<5->{%
 = \alert<handout:0| 6-7>{\frac{\diff y}{\diff u}}%
\alert<handout:0| 8-9>{\frac{\diff u}{\diff x}}%
}\\%
& \uncover<6->{%
 = \alert<handout:0| 6-7>{\left( \uncover<7-| handout:0>{\noexpandarg\exploregroups\StrSubstitute{#4}{UU}{\alert<handout:0| 10>{u}}\noexploregroups\expandarg}\right)}%
\alert<handout:0| 8-9>{\left( \uncover<9-| handout:0>{#5}\right)}%
}\\%
& \uncover<10->{ = } \uncover<10-| handout:0>{%
 \noexpandarg\exploregroups \StrSubstitute{#6}{UU}{\alert<handout:0| 10>{#2}}.\noexploregroups\expandarg%
}%
\end{align*}
\end{example}
}





%
%  An example of a derivative with the Chain Rule, using the symbol f(x).
%  It looks as follows:
%  
%  Differentiate f(x) = #1.
%  Let h(x) = #2
%  Let g(x) = #3
%  Then f(x) = g(h(x))
%  f'(x) = g'(h(x))h'(x)
%  = (#4, with h(x) substituted for UU)(#5)
%  = #6, with #2 substituted for UU
%
%  #7 appears in the subtitle of the example.
%
\newcommand{\chainrulefofx}[7]{%
\begin{example}[Chain Rule%
\ifthenelse{\equal{#7}{0}}%
{}%
{, #7}%
]%
\abovedisplayskip=0pt
\belowdisplayskip=0pt
\abovedisplayshortskip=0pt
\belowdisplayshortskip=0pt
\begin{align*}
\text{Differentiate}\quad f(x) & = #1.\\%
\uncover<2->{%
\text{Let}\quad\alert<handout:0| 2-3,9-11>{h(x)}%
}%
& \uncover<2->{%
\alert<handout:0| 2-3,9-11>{ = \uncover<3-| handout:0>{#2.}}%
}\\%
\uncover<2->{%
\text{Let}\quad\alert<handout:0| 4-5,7-8>{g(x)}%
}%
& \uncover<2->{%
\alert<handout:0| 4-5,7-8>{ = \uncover<5-| handout:0>{#3.}}%
}\\%
\uncover<2->{%
\text{Then}\quad f(x)%
}%
& \uncover<2->{%
 = g(h(x)).%
}\\%
\uncover<6->{%
\text{Chain Rule:}\quad%
f'(x)%
}%
& \uncover<6->{%
 = \alert<handout:0| 7-8>{g'(h(x))}%
\alert<handout:0| 9-10>{h'(x)}%
}\\%
& \uncover<7->{%
 = \alert<handout:0| 7-8>{\left( \uncover<8-| handout:0>{\noexpandarg\exploregroups\StrSubstitute{#4}{UU}{\alert<handout:0| 11>{h(x)}}\noexploregroups\expandarg}\right)}%
\alert<handout:0| 9-10>{\left( \uncover<10-| handout:0>{#5}\right)}%
}\\%
& \uncover<11->{ = } \uncover<11-| handout:0>{%
 \noexpandarg \exploregroups \StrSubstitute{#6}{UU}{\alert<handout:0| 11>{#2}}.\noexploregroups \expandarg%
}%
\end{align*}
\end{example}
}



%
%  An example of an infinite limit calculation. 
%  There are nine arguments to the function.  The ninth is a string of six
%  plus and minus signs.  Let AA, BB, CC, DD, EE, and FF denote these plus
%  and minus signs.  Then the output of the function looks as follows:
%  
%  Find lim_{x \to #1^AA} (#2, with x substituted for UU)/(#3, with x substituted for UU).
%  Plug in #1.
%  (#2, with (#1) substituted for UU)/(#3, with (#1) substituted for UU) = #4/0.
%  The numerator is non-zero and the denominator is zero.  
%  Therefore the answer is DNE, infty, or -infty.
%  Factor: (#3, with x substituted for UU)/(#4, with x substituted for UU) = (#5 #6)/(#7 #8)
%  \to ((BB)(CC))/((DD)(EE))
%  = (FF).
%  Therefore lim_{x \to #1^AA} (#2, with x substituted for UU)/(#3, with x substituted for UU) = FF infty.
%
\newcommand{\infinitelimit}[9]{%
\begin{example}[Infinite Limit]%
\abovedisplayskip=0pt
\belowdisplayskip=0pt
\abovedisplayshortskip=0pt
\belowdisplayshortskip=0pt
\begin{align*}
\text{Find}\quad \lim_{x\to #1^{\StrMid{#9}{1}{1}}}
\frac%
{\noexpandarg\StrSubstitute{#2}{UU}{x}\expandarg}%
{\noexpandarg\StrSubstitute{#3}{UU}{x}\expandarg}%
& \\%
\uncover<2->{%
\text{Plug in $#1$:}\quad%
\frac%
{\alert<handout:0| 2-3>{\noexpandarg\StrSubstitute{#2}{UU}{(#1)}\expandarg}}%
{\alert<handout:0| 4-5>{\noexpandarg\StrSubstitute{#3}{UU}{(#1)}\expandarg}}%
}%
& \uncover<2->{= \frac{\uncover<3->{\alert<handout:0| 3>{#4}}}{\uncover<5->{\alert<handout:0| 5>{0}}}}%
\uncover<6->
Therefore the answer is DNE, $\infty$, or $-\infty$.}
}%
\uncover<7->{%
\text{Factor:}\quad
}%
\uncover<7->{%
\lim_{x\to #1^{\StrMid{#9}{1}{1}}}%
\frac%
{\alert<handout:0| 8-9>{\noexpandarg\StrSubstitute{#2}{UU}{x}\expandarg}}%
{\alert<handout:0| 10-11>{\noexpandarg\StrSubstitute{#3}{UU}{x}\expandarg}}%
}%
& \uncover<8->{%
 = \lim_{x\to #1^{\StrMid{#9}{1}{1}}}%
\frac%
{%
\uncover<handout:0| 9->{\alert<handout:0| 9>{%
\alert<handout:0| 12-13>{%
#5%
}%
\alert<handout:0| 14-15>{%
#6%
}%
}}%
}{%
\uncover<handout:0| 11->{\alert<handout:0| 11>{%
\alert<handout:0| 16-17>{%
#7%
}%
\alert<handout:0| 18-19>{%
#8%
}%
}}%
}%
}\\%
& \uncover<12->{%
 \to \alert<handout:0| 20-21>{\frac%
{%
\alert<handout:0| 12-13>{(\uncover<handout:0| 13->{%
\StrMid{#9}{2}{2}%
})}%
\alert<handout:0| 14-15>{(\uncover<handout:0| 15->{%
\StrMid{#9}{3}{3}%
})}%
}{%
\alert<handout:0| 16-17>{(\uncover<handout:0| 17->{%
\StrMid{#9}{4}{4}%
})}%
\alert<handout:0| 18-19>{(\uncover<handout:0| 19->{%
\StrMid{#9}{5}{5}%
})}%
}%
}%
}\\%
& \uncover<20->{\alert<handout:0| 20-21>{ = \uncover<handout:0| 21->{\alert<handout:0| 22>{(\StrMid{#9}{6}{6})}}}}\\%
\uncover<22->{%
\text{Therefore}\quad\lim_{x\to #1^{\StrMid{#9}{1}{1}}}%
\frac%
{\noexpandarg\StrSubstitute{#2}{UU}{x}\expandarg}%
{\noexpandarg\StrSubstitute{#3}{UU}{x}\expandarg}%
}%
& \uncover<22->{ = } \uncover<handout:0| 22->{ \alert<handout:0| 22>{\StrMid{#9}{6}{6}}\infty.}
\end{align*}
\end{example}
}




%
%  An example of a limit calculation with factoring. 
%  
%  It looks as follows.  
%  
%  Find lim_{x \to #1} (#2, with x substituted for UU)/(#3, with x substituted for UU).
%  Plug in #1.
%  (#2, with (#1) substituted for UU)/(#3, with (#1) substituted for UU) = 0/0.
%  Zero over zero gives no information.
%  Factor: (#2, with x substituted for UU)/(#3, with x substituted for UU) = ((#4, with x substituted for UU) #6)/((#5, with x substituted for UU) #6)
%  = (#4, with x substituted for UU)/(#5, with x substituted for UU)
%  Plug in #1: = (#4, with (#1) substituted for UU)/(#5, with (#1) substituted for UU)
%  = #7
%  = #8
%
\newcommand{\limitfactor}[8]{%
\begin{example}[Limit with Factoring]%
\abovedisplayskip=0pt
\belowdisplayskip=0pt
\abovedisplayshortskip=0pt
\belowdisplayshortskip=0pt
\begin{align*}
\text{Find}\quad \lim_{x\to #1}
\frac%
{\noexpandarg\StrSubstitute{#2}{UU}{x}\expandarg}%
{\noexpandarg\StrSubstitute{#3}{UU}{x}\expandarg}%
& \\%
\uncover<2->{%
\text{Plug in $#1$:}\quad%
\frac%
{\alert<handout:0| 2-3>{\noexpandarg\StrSubstitute{#2}{UU}{(#1)}\expandarg}}%
{\alert<handout:0| 4-5>{\noexpandarg\StrSubstitute{#3}{UU}{(#1)}\expandarg}}%
}%
& \uncover<2->{%
= \frac%
{\uncover<3->{\alert<handout:0| 3>{0}}}%
{\uncover<5->{\alert<handout:0| 5>{0}}}%
}%
\uncover<6->{%
\intertext{Zero over zero is undefined, so we can't use direct substitution.}
}%
\uncover<7->{%
\text{Factor:}\quad%
\lim_{x\to #1} \frac%
{\alert<handout:0| 8-9>{\noexpandarg\StrSubstitute{#2}{UU}{x}\expandarg}}%
{\alert<handout:0| 10-11>{\noexpandarg\StrSubstitute{#3}{UU}{x}\expandarg}}%
}%
& \uncover<8->{%
 = \lim_{x\to #1} \frac%
{%
\uncover<handout:0| 9->{\alert<handout:0| 9>{%
(\noexpandarg\StrSubstitute{#4}{UU}{x}\expandarg)%
\alert<handout:0| 12>{#6}%
}}%
}{%
\uncover<handout:0| 11->{\alert<handout:0| 11>{%
(\noexpandarg\StrSubstitute{#5}{UU}{x}\expandarg)%
\alert<handout:0| 12>{#6}%
}}%
}%
}\\%
& \uncover<12->{%
 = \lim_{x\to #1} \frac%
{\uncover<handout:0| 12->{\noexpandarg\StrSubstitute{#4}{UU}{\alert<handout:0| 13>{x}}\expandarg}}%
{\uncover<handout:0| 12->{\noexpandarg\StrSubstitute{#5}{UU}{\alert<handout:0| 13>{x}}\expandarg}}%
}\\%
\uncover<13->{%
\text{Plug in $#1$:}\quad%
\lim_{x\to #1} \frac%
{\noexpandarg\StrSubstitute{#2}{UU}{x}\expandarg}%
{\noexpandarg\StrSubstitute{#3}{UU}{x}\expandarg}%
}%
& \uncover<13->{%
 = \frac%
{\uncover<handout:0| 13->{\noexpandarg\StrSubstitute{#4}{UU}{(\alert<handout:0| 13>{#1})}\expandarg}}%
{\uncover<handout:0| 13->{\noexpandarg\StrSubstitute{#5}{UU}{(\alert<handout:0| 13>{#1})}\expandarg}}%
}\\%
& \uncover<14->{%
= \uncover<handout:0| 14->{#7}%
}\\%
& \uncover<15->{%
= \uncover<handout:0| 14->{#8.}%
}%
\end{align*}
\end{example}
}




%
%  An example of a limit calculation with a conjugate radical. 
%  
%  It looks as follows.  
%  
%  Find lim_{x \to #1} (#2, with x substituted for UU)/(#3, with x substituted for UU).
%  Plug in #1.
%  (#2, with (#1) substituted for UU)/(#3, with (#1) substituted for UU) = 0/0.
%  Zero over zero gives no information.
%  Factor: (#2, with x substituted for UU)/(#3, with x substituted for UU) = ((#4, with x substituted for UU) #6)/((#5, with x substituted for UU) #6)
%  = (#4, with x substituted for UU)/(#5, with x substituted for UU)
%  Plug in #1: = (#4, with (#1) substituted for UU)/(#5, with (#1) substituted for UU)
%  = #7
%  = #8
%
\newcommand{\limitradical}[9]{%
\begin{example}[Limit with Conjugate Radical]%
\abovedisplayskip=0pt
\belowdisplayskip=0pt
\abovedisplayshortskip=0pt
\belowdisplayshortskip=0pt
\begin{align*}
& \text{Find}\quad \lim_{x\to #1}
\frac%
{\noexpandarg\StrSubstitute{#2}{UU}{x}\expandarg}%
{\noexpandarg\StrSubstitute{#3}{UU}{x}\expandarg}%
 \\%
\uncover<2->{%
& \text{Plug in $#1$:}\quad%
\frac%
{\alert<handout:0| 2-3>{\noexpandarg\StrSubstitute{#2}{UU}{(#1)}\expandarg}}%
{\alert<handout:0| 4-5>{\noexpandarg\StrSubstitute{#3}{UU}{(#1)}\expandarg}}%
}%
 \uncover<2->{%
= \frac%
{\uncover<3->{\alert<handout:0| 3>{0}}}%
{\uncover<5->{\alert<handout:0| 5>{0}}}%
}%
\uncover<6->{%
\intertext{Zero over zero gives no information.  Use a conjugate radical.}
}%
& \uncover<7->{%
\lim_{x\to #1} \frac%
{\noexpandarg\StrSubstitute{#2}{UU}{x}\expandarg}%
{\alert<handout:0| 7-8>{\noexpandarg\StrSubstitute{#3}{UU}{x}\expandarg}}%
\cdot %
\frac%
{\uncover<8->{\alert<8>{\noexpandarg\StrSubstitute{#4}{UU}{x}\expandarg}}}%
{\uncover<8->{\alert<8>{\noexpandarg\StrSubstitute{#4}{UU}{x}\expandarg}}}%
}\\%
& \uncover<9->{%
 = \lim_{x\to #1} \frac%
{(\noexpandarg\StrSubstitute{#2}{UU}{x}\expandarg)%
\left(\noexpandarg\StrSubstitute{#4}{UU}{x}\expandarg\right)}%
{#5}%
}\\%
& \uncover<10->{%
 = \lim_{x\to #1} \frac%
{(\alert<11-12>{\noexpandarg\StrSubstitute{#2}{UU}{x}\expandarg})%
\left(\noexpandarg\StrSubstitute{#4}{UU}{x}\expandarg\right)}%
{\alert<13-14>{#6}}%
}\\%
\uncover<11->{%
\text{Factor:}\quad%
}%
& \uncover<11->{%
 = \lim_{x\to #1} \frac%
{\uncover<12->{\alert<12>{(\noexpandarg\StrSubstitute{#7}{UU}{x}\expandarg)(x-#1)}}%
\left(\noexpandarg\StrSubstitute{#4}{UU}{x}\expandarg\right)}%
{\uncover<14->{\alert<14>{(\noexpandarg\StrSubstitute{#8}{UU}{x}\expandarg)(x-#1)}}}%
}\\%
& \uncover<15->{%
 = \lim_{x\to #1} \frac%
{(\noexpandarg\StrSubstitute{#7}{UU}{x}\expandarg)%
\left(\noexpandarg\StrSubstitute{#4}{UU}{x}\expandarg\right)}%
{\noexpandarg\StrSubstitute{#8}{UU}{x}\expandarg}%
}\\%
\uncover<16->{%
\text{Plug in $#1$:}\quad%
}%
& \uncover<16->{%
 = \frac%
{(\noexpandarg\StrSubstitute{#7}{UU}{(#1)}\expandarg)%
\left(\noexpandarg\StrSubstitute{#4}{UU}{(#1)}\expandarg\right)}%
{\noexpandarg\StrSubstitute{#8}{UU}{(#1)}\expandarg}%
}\\%
& \uncover<17->{%
#9.
}%
\end{align*}
\end{example}
}


%
%  An example of a limit calculation with direct substitution.  
%  
%  It looks as follows.  
%  
%  Find lim_{x \to #1} (#2, with x substituted for UU)/(#3, with x substituted for UU).
%  Plug in #1.
%  (#2, with (#1) substituted for UU)/(#3, with (#1) substituted for UU) = 0/0.
%  Zero over zero gives no information.
%  Factor: (#2, with x substituted for UU)/(#3, with x substituted for UU) = ((#4, with x substituted for UU) #6)/((#5, with x substituted for UU) #6)
%  = (#4, with x substituted for UU)/(#5, with x substituted for UU)
%  Plug in #1: = (#4, with (#1) substituted for UU)/(#5, with (#1) substituted for UU)
%  = #7
%  = #8
%
\newcommand{\limitsub}[7]{%
\begin{example}[%
\ifthenelse{\equal{#6}{0}}%
{Limit in Which Direct Substitution Doesn't Work}%
{Limit with Direct Substitution}%
]%
\abovedisplayskip=0pt
\belowdisplayskip=0pt
\abovedisplayshortskip=0pt
\belowdisplayshortskip=0pt
\begin{align*}
\text{Find}\quad \lim_{x\to #1}
\frac%
{\noexpandarg\StrSubstitute{#2}{UU}{x}\expandarg}%
{\noexpandarg\StrSubstitute{#3}{UU}{x}\expandarg}%
& \\%
\uncover<2->{%
\text{Plug in $#1$:}\quad%
\frac%
{\alert<handout:0| 2-3>{\noexpandarg\StrSubstitute{#2}{UU}{(#1)}\expandarg}}%
{\alert<handout:0| 4-5>{\noexpandarg\StrSubstitute{#3}{UU}{(#1)}\expandarg}}%
}%
& \uncover<2->{%
= \frac%
{\uncover<3->{\alert<handout:0| 3>{#4}}}%
{\uncover<5->{\alert<handout:0| 5>{#5}}}%
}\\%
\ifthenelse{\equal{#6}{0}}%
{ }%
{&}%
\uncover<6->{%
\ifthenelse{\equal{#6}{0}}%
{\intertext{Dividing by zero is undefined, so we can't use direct substitution.}}%
{ = #7.}%
}%
\ifthenelse{\equal{#6}{0}}%
{ }%
{ \text{Therefore}= #7.}%
\end{align*}
\end{example}
}



%
%  An example Newton's Method.  
%  
%  It looks as follows.  
%  
%  Starting with x_1 = #1, find the third approximation x_3 to the root of the equation #2.  
%  
%  f(x) = (#3, with x substituted for UU).
%  f'(x) = (#4, with x substituted for UU).
%  Newton's Method: x_{n+1} = x_n - f(x_n)/f'(x_n) = x_n - (#3, with x_n substituted for UU)/(#4, with x_n substituted for UU).
%  
%  x_2 = x_1 - (#3, with x_1 substituted for UU)/(#4, with x_1 substituted for UU)     x_3 = x_2 - (#3, with x_2 substituted for UU)/(#4, with x_2 substituted for UU)     
%   = (#1) - (#3, with (#1) substituted for UU)/(#4, with (#1) substituted for UU)     = (#5) - (#3, with (#5) substituted for UU)/(#4, with (#5) substituted for UU)
%  = #5.      = #6.
%
\newcommand{\newtonsmethod}[8]{%
\begin{example}[Newton's Method%
\ifthenelse{\equal{#8}{0}}%
{}%
{, #8}%
]%
\ifthenelse{\equal{#7}{0}}%
{%
Starting with $x_1 = #1$, find the third approximation $x_3$ to the root of the equation $#2$.  
}%
{#7}%
\abovedisplayskip=0pt
\belowdisplayskip=10pt
\abovedisplayshortskip=0pt
\belowdisplayshortskip=0pt
\begin{align*}
\uncover<2->{%
\alert<handout:0| 2-3,7>{f(x)}%
& \alert<handout:0| 2-3,7>{ = \uncover<3->{\noexpandarg \exploregroups \StrSubstitute{#3}{UU}{x}.\noexploregroups \expandarg}}%
}\\%
\uncover<4->{%
\alert<handout:0| 4-5,8>{f'(x)}%
& \alert<handout:0| 4-5,8>{ = \uncover<5->{\noexpandarg \exploregroups \StrSubstitute{#4}{UU}{x}.\noexploregroups \expandarg}}%
}\\%
\uncover<6->{%
\text{Newton's Method:}\quad %
x_{n+1} & = x_n - \frac{\alert<handout:0| 7>{f(x_n)}}{\alert<handout:0| 8>{f'(x_n)}}%
}
\uncover<7->{%
 = x_n - \frac%
{\alert<handout:0| 7>{\noexpandarg \exploregroups \StrSubstitute{#3}{UU}{x_n}\noexploregroups \expandarg}}%
{\alert<handout:0| 8>{\uncover<8->{\noexpandarg \exploregroups \StrSubstitute{#4}{UU}{x_n}\noexploregroups \expandarg}}}%
}
\end{align*}
\begin{align*}
\uncover<9->{%
x_2 %
}%
& \uncover<9->{%
 = \alert<handout:0| 10>{x_1} - \frac%
{\noexpandarg \exploregroups \StrSubstitute{#3}{UU}{\alert<handout:0| 10>{x_1}}\noexploregroups \expandarg}%
{\noexpandarg \exploregroups \StrSubstitute{#4}{UU}{\alert<handout:0| 10>{x_1}}\noexploregroups \expandarg}%
}%
& \uncover<12->{%
x_3 %
}%
& \uncover<12->{%
 = \alert<handout:0| 13>{x_2} - \frac%
{\noexpandarg \exploregroups \StrSubstitute{#3}{UU}{\alert<handout:0| 13>{x_2}}\noexploregroups \expandarg}%
{\noexpandarg \exploregroups \StrSubstitute{#4}{UU}{\alert<handout:0| 13>{x_2}}\noexploregroups \expandarg}%
}\\%
& \uncover<10->{%
 = \alert<handout:0| 10>{(#1)} - \frac%
{\noexpandarg \exploregroups \StrSubstitute{#3}{UU}{\alert<handout:0| 10>{(#1)}}\noexploregroups \expandarg}%
{\noexpandarg \exploregroups \StrSubstitute{#4}{UU}{\alert<handout:0| 10>{(#1)}}\noexploregroups \expandarg}%
}%
& % 
& \uncover<13->{%
 = \alert<handout:0| 13>{(#5)} - \frac%
{\noexpandarg \exploregroups \StrSubstitute{#3}{UU}{\alert<handout:0| 13>{(#5)}}\noexploregroups \expandarg}%
{\noexpandarg \exploregroups \StrSubstitute{#4}{UU}{\alert<handout:0| 13>{(#5)}}\noexploregroups \expandarg}%
}\\%
& \uncover<11->{%
 = #5.%
}%
& % 
& \uncover<14->{%
 = #6.
}%
\end{align*}
\end{example}
}


%
%  An example of a derivative using the Chain Rule twice, using dy/dx.
%  It looks as follows:
%  
%  Differentiate: y = #1.
%		  dy\dx  = d\dx(#1)
%  Chain Rule:     = (#2) (d/dx)(#3)
%  Chain Rule:     = (#2)(#4) d/dx(#5)
%  #7 [optional]    = (#2)(#3)(#6)
%                             = (#8)
%                             = (#9)    [optional]
%

\newcommand{\chainruletwice}[9]{%
\begin{example}[Using the Chain Rule twice]%
\abovedisplayskip=0pt
\belowdisplayskip=0pt
\abovedisplayshortskip=0pt
\belowdisplayshortskip=0pt
\begin{align*}
\text{Differentiate:}\quad y & = #1.\\%
\uncover<2->{\frac{\diff y}{\diff x} & = \alert<handout:0|3-5>{\frac{\diff}{\diff x}\left( #1\right)}}\\%
\uncover<4->{\text{Chain Rule:} \ \ \quad &= \alert<handout:0|4-5>{\left(\uncover<5-| handout:0>{#2} \right)\alert<handout:0|6-8>{\frac{\diff}{\diff x} \left(\uncover<4-| handout:0>{#3}\right)}}} \\%
\uncover<7->{\text{Chain Rule:} \ \ \quad &= \left(\uncover<7-| handout:0>{#2}\right) \alert<handout:0|7-8>{\left(\uncover<8-| handout:0>{#4}\right) \alert<handout:0|9-10>{\frac{\diff}{\diff x}\left( \uncover<7-| handout:0>{#5} \right)}}}\\%
\uncover<9->{\uncover<10->{\ifthenelse{\equal{#7}{}}{}{\text{#7 :} \ \ \quad}}& = \left(\uncover<9-| handout:0>{#2} \right) \left(\uncover<9-| handout:0>{#4}\right)\alert<handout:0|9-10>{\left( \uncover<10-| handout:0>{#6} \right) }} \\%
\uncover<11->{& = \uncover<11-| handout:0>{#8 \ifthenelse{\equal{#9}{}}{.}{\\}}}%
\ifthenelse{\equal{#9}{}}{}{\uncover<12->{& = \uncover<12-| handout:0>{#9.}}}
\end{align*}
\end{example}
}

\ProvidesPackage{pstricks-commands}
\usepackage{etex, ifthen}
\usepackage{auto-pst-pdf}
\usepackage{pst-plot}
\usepackage{pst-math}
%WARNING THE FOLLOWING PACKAGE IS BROKEN use only with EXTREME CAUTION
%\usepackage{pst-3dplot}

\makeatletter
\begingroup
\catcode `P=12  % digits and punct. catcode
\catcode `T=12  % digits and punct. catcode
\lowercase{%
\def\x{\def\rem@pt##1.##2PT{##1\ifnum##2>\z@.##2\fi}}}
\expandafter\endgroup\x%
\newcommand{\stripPoints}[1]{\expandafter\rem@pt\the#1}
\makeatother

\newcommand{\fcShiftX}{0}
\newcommand{\fcShiftY}{0}
\newcommand{\fcXLabel}{$x$}
\newcommand{\fcYLabel}{$y$}
\newcommand{\fcZLabel}{$z$}
\newcommand{\fcDelta}{0.5}
\newcommand{\fcZBufferNumXIntervals}{20}
\newcommand{\fcZBufferNumYIntervals}{20}
\newcommand{\fcStartXIId}{0}
\newcommand{\fcStartYIId}{0}
\newcommand{\fcIterationsX}{9\space}
\newcommand{\fcIterationsY}{9\space}
\newcommand{\fcIterationsU}{9\space}
\newcommand{\fcIterationsV}{9\space}
\newcommand{\fcScreenStyle}{z}
\newcommand{\fcLineColor}{black}
\newcommand{\fcLineWidth}{1\space}
\newcommand{\fcArrows}{}
\newcommand{\fcPlotPoints}{200}
\newcommand{\fcLineStyle}{0}
\newcommand{\fcDashLength}{2}
\newcommand{\fcDashesCode}{%
(\fcLineStyle) (dashed) eq %
{[\fcDashLength\space \fcDashLength] 0 setdash}%
{[] 0 setdash}%
ifelse\space %
}
\newcommand{\fcScreen}{[-1 1 -0.5] -1} %default projection plane. Renew this command to change projection plane.

\newcommand{\fcSet}[1]{\setkeys{fcGraphics}{#1}}

\makeatletter %needed for define@key command.
\define@key{pstricks,pst-plot}{xLabel}[]{}
\define@key{pstricks,pst-plot}{yLabel}[]{}
\define@key{pstricks,pst-plot}{zLabel}[]{}
\define@key{fcGraphics}{Delta}[\renewcommand{\fcDelta}{1}]{\renewcommand{\fcDelta}{#1}}
\define@key{fcGraphics}{shiftX}[\renewcommand{\fcShiftX}{0}]{\renewcommand{\fcShiftX}{#1}}
\define@key{fcGraphics}{shiftY}[\renewcommand{\fcShiftY}{0}]{\renewcommand{\fcShiftY}{#1}}
\define@key{fcGraphics}{startX}[\renewcommand{\fcStartXIId}{0}]{\renewcommand{\fcStartXIId}{#1}}
\define@key{fcGraphics}{startY}[\renewcommand{\fcStartYIId}{0}]{\renewcommand{\fcStartYIId}{#1}}
\define@key{fcGraphics}{iterationsU}[\renewcommand{\fcIterationsU}{9\space}]{\renewcommand{\fcIterationsU}{#1\space}}
\define@key{fcGraphics}{iterationsV}[\renewcommand{\fcIterationsU}{9\space}]{\renewcommand{\fcIterationsV}{#1\space}}
\define@key{fcGraphics}{iterationsX}[\renewcommand{\fcIterationsX}{9\space}]{\renewcommand{\fcIterationsX}{#1\space}}
\define@key{fcGraphics}{iterationsY}[\renewcommand{\fcIterationsY}{9\space}]{\renewcommand{\fcIterationsY}{#1\space}}
\define@key{fcGraphics}{screenStyle}[\renewcommand{\fcScreenStyle}{z}]{\renewcommand{\fcScreenStyle}{#1}}
\define@key{fcGraphics}{xLabel}[\renewcommand{\fcXLabel}{$x$}]{\renewcommand{\fcXLabel}{#1}}
\define@key{fcGraphics}{yLabel}[\renewcommand{\fcYLabel}{$y$}]{\renewcommand{\fcYLabel}{#1}}
\define@key{fcGraphics}{zLabel}[\renewcommand{\fcZLabel}{$z$}]{\renewcommand{\fcZLabel}{#1}}
\define@key{fcGraphics}{linecolor}[\renewcommand{\fcLineColor}{black}]{\renewcommand{\fcLineColor}{#1}}
\define@key{fcGraphics}{linestyle}[\renewcommand{\fcLineStyle}{0}]{\renewcommand{\fcLineStyle}{#1}}
\define@key{fcGraphics}{plotpoints}[\renewcommand{\fcPlotPoints}{200}]{\renewcommand{\fcPlotPoints}{#1}}
\define@key{fcGraphics}{dashes}[\renewcommand{\fcDashLength}{300}]{\renewcommand{\fcDashLength}{#1 \fcConvertPSXUnit\space}}
\define@key{fcGraphics}{arrows}[\renewcommand{\fcArrows}{}]{\renewcommand{\fcArrows}{#1}}
\makeatother %undoes \makeatletter.


\newcommand{\fcHollowDot}[2]{
\pscircle*[fillcolor=white, linecolor=red](#1, #2){0.07}
\pscircle*[fillcolor=white, linecolor=white](#1, #2){0.04}
}

\newcommand{\fcFullDot}[3][linecolor=red]{
\pscircle*[#1](! #2 #3){0.07}
}

\newcommand{\fcFullDotCode}{
\fcCoordsPStricksToPS [0.07 0] \fcCoordsPStricksToPS pop 0 360 arc 1 0 0 setrgbcolor fill stroke
}

\newcommand{\fcHollowDotBlue}[2]{
\pscircle*[fillcolor=white, linecolor=blue](#1, #2){0.07}
\pscircle*[fillcolor=white, linecolor=white](#1, #2){0.04}
}
\newcommand{\fcFullDotBlack}[2]{
\pscircle*[fillcolor=white, linecolor=black](#1, #2){0.07}
}
\newcommand{\fcFullDotBlue}[2]{
\pscircle*[fillcolor=white, linecolor=blue](#1, #2){0.07}
}
\newcommand{\fcXTickColored}[2]{\psline[linecolor=#1](#2, -0.1)(#2,0.1)}

\newcommand{\fcXTick}[1]{\psline(#1, -0.1)(#1,0.1)}
\newcommand{\fcYTick}[1]{\psline(-0.1, #1)(0.1, #1)}
\newcommand{\fcXYTick}[2]{\fcXTick{#1} \fcYTick{#2}}

\newcommand{\fcXTickWithLabel}[2]{\fcXTick{#1}\rput[t](#1,-0.2){#2}}
\newcommand{\fcYTickWithLabel}[2]{\fcYTick{#1}\rput[r](-0.2,#1){#2}}

\newcommand{\fcLabelNumberXaxis}[1]{\fcXTickWithLabel{#1}{#1}}
\newcommand{\fcLabelNumberYaxis}[1]{\fcYTickWithLabel{#1}{#1}}

\newcommand{\fcLabelNumberXYaxes}[2]{\fcLabelNumberXaxis{#1} \fcLabelNumberYaxis{#2} }

\newcommand{\fcLabelXOne}{\fcLabelNumberXaxis{1} }
\newcommand{\fcLabelYOne}{\fcLabelNumberYaxis{1} }

\newcommand{\fcLabelOnXaxis}[2]{\fcXTick{#1}\rput[t](#1,-0.2){#2}}
\newcommand{\fcLabelOnYaxis}[2]{\fcYTick{#1}\rput[r](-0.2, #1){#2}}

\newcommand{\fcLabels}[1][$x$]{%
  \def\ArgpsXAxisLabel{{#1}}%
  \fcLabelsRelay
}
\newcommand\fcLabelsRelay[3][$y$]{\rput[t](! #2 -0.1){\ArgpsXAxisLabel}\rput[r](! -0.1 #3){#1}}

\newcommand{\fcLabelsWithOnes}[2]{\psline(1, -0.1)(1,0.1) \rput[t](1, -0.2 ) { $1$} \psline(-0.1, 1)(0.1, 1) \rput[r](-0.2, 1 ) { $1$} \fcLabels{#1}{#2}}

\newcommand{\fcDefaultXLabel}{$x$}
\newcommand{\fcDefaultYLabel}{$y$}

\newcommand{\fcBoundingBox}[4]{%
\psframe*[linecolor=white](! #1\space #2)(! #3\space #4)%
\psline[linecolor=black!1](! #1 #2 )(! #1 #2 0.01 add)%
\psline[linecolor=black!1](! #3 #4 )(! #3 #4 0.01 add)%
}
\newcommand{\fcAxesStandardNoFrame}[4]{%
\psaxes[ticks=none, labels=none]{<->}(0,0)(#1,#2)(#3,#4)% \fcLabels[\fcDefaultXLabel][\fcDefaultYLabel]{#3}{#4}%
}%

\newcommand{\fcAxesStandard}[4]{%
\psframe*[linecolor=white](! #1\space #2)(! #3 \space 0.1 add #4 \space 0.1 add)%
\fcAxesStandardNoFrame{#1}{#2}{#3}{#4}%
}%
\newcommand{\fcColorTangent}{blue}
\newcommand{\fcColorGraph}{red}
\newcommand{\fcColorAreaUnderGraph}{cyan}
\newcommand{\fcColorNegativeAreaUnderGraph}{orange}

\newcommand{\fcMachine}[2]{
\pscustom*[linecolor=#2]{
\psline(1,1.1)(1,0.1)(1.5,0.1)(2, 0.6)(2.5, 0.6)(2.5, -0.6)(2, -0.6)(1.5,-0.1)(1,-0.1)(1,-1.1)(-1,-1.1)(-1,-0.1)(-1.5,-0.1)(-2, -0.6)(-2.5, -0.6)(-2.5, 0.6)(-2, 0.6)(-1.5,0.1)(-1,0.1)(-1,1.1)
}
\pscircle*[linecolor=white](0,0){0.3}
\rput(0,0){#1}
}

%command format
%first argument gives you formula for the direction field in
%postscript notation, for example x y add.
%second and third argument give the starting x,y coordinates
\newcommand{\fcDirectionFieldOneTangent}[6]{%
\pstVerb{%
3 dict begin%
/x #2 \space def%
/y #3 \space def%
/F #1 \space def%
}%
\psline[#6](! x F ATAN 57.295 mul cos #4 mul sub y F ATAN 57.295 mul sin #4 mul sub)(! x F ATAN 57.295 mul cos #4 mul add y F ATAN 57.295 mul sin #4 mul add)%
\pscircle*[linecolor=red!60](! x y){#5}%
\pstVerb{%
end%
}%
}

\newcommand{\fcDirectionFieldOneTangentDefault}[3]{%
\fcDirectionFieldOneTangent{#1}{#2}{#3}{0.3}{0.03}{linecolor=blue}%
}

%command format
%first argument gives you formula for the direction field in
%postscript notation, for example x y add.
%second and third argument give the starting x,y coordinates
%fourth coordinate gives the delta x=delta y
%fifth argument gives the number of iterations delta x
%sixth argument gives the number of iterations delta y
%seventh argument gives the length of the vector
%eighth  argument gives the circle radius
%ninth argument gives the arguments of the psline command
\newcommand{\fcDirectionFieldFull}[9]{%
\multido{\ra=#2+#4}{#5}{%
\multido{\rb=#3+#4}{#6}{%
\fcDirectionFieldOneTangent{#1}{\ra}{\rb}{#7}{#8}{#9}%
}%end multido
}%end multido
}%end newcommand

\newcommand{\fcDirectionFieldDefault}[5]{%
\fcDirectionFieldFull{#1}{#2}{#3}{#4}{#5}{#5}{0.2}{0.02}{linecolor=blue}%
}%
\newcommand{\fcDirectionFieldDefaultRange}[1]{%
\fcDirectionFieldFull{#1}{-4}{-4}{0.5}{21}{21}{0.2}{0.02}{linecolor=blue}%
}

\newcommand{\fcVectorProjectOntoVector}{%
\fcVectorNormalize dup 3 1 roll \fcVectorScalarVector \fcVectorTimesScalar%
} %

%fcAngleIIId Arguments:
%first optional: pstricks options
%second: vector describing arm of first angle
%third: vector describing arm of second angle
%fourth: radius of arc representing the angle
\newcommand{\fcAngleIIId}[4][]{%
\pstVerb{%
3 dict begin%
/firstV #2 \fcVectorNormalize def%
/orthonormalV #3 dup firstV  \fcVectorProjectOntoVector \fcVectorMinusVector \fcVectorNormalize def%
/theAngle firstV #3\space \fcVectorNormalize \fcVectorScalarVector arccos def%
}%
\parametricplot[#1]{0}{theAngle}{firstV t cos #4 mul \fcVectorTimesScalar orthonormalV t sin #4 mul \fcVectorTimesScalar \fcVectorPlusVector \fcCoordsIIIdToPStricks}%
\pstVerb{end}%
}

\makeatletter
\newcommand{\fcAngle}[5][linecolor=\fcColorGraph]{%
\ifPst@algebraic{%
\parametricplot[#1, algebraic=true]{#2}{#3}{#4*cos(t)| #4*sin(t)}%
\rput(! #2\space #3\space add 2 div 57.29578 mul cos #4\space 0.2 add mul #2\space #3\space add 2 div 57.29578 mul sin #4\space 0.2 add mul){#5}%
}%
\else%
\parametricplot[#1, algebraic=false]{#2}{#3}{t 57.29578 mul cos #4\space mul t 57.29578 mul sin #4\space mul}%
\rput(! #2\space #3\space add 2 div 57.29578 mul cos #4\space 0.2 add mul #2\space #3\space add 2 div 57.29578 mul sin #4\space 0.2 add mul){#5}%
\fi%
}
\makeatother

\newcommand{\fcDistance}{ \fcVectorMinusVector \fcVectorNorm\space}

\newcommand{\fcLengthIndicator}[5]{
\psline[arrows=<-, linecolor=red](! #1 #2)(! #1 0.58 mul #3 0.42 mul add #2 0.58 mul #4 0.42 mul add)
\psline[arrows=->, linecolor=red]{->}(! #1 0.42 mul #3 0.58 mul add #2 0.42 mul #4 0.58 mul add)(! #3 #4)
\rput(! #1 #3 add 0.5 mul #2 #4 add 0.5 mul){ #5}
}

\makeatletter
\newcommand{\fcDrawPolar}[4][linecolor=\fcColorGraph]{%
\ifPst@algebraic{%
\parametricplot[#1]{#2}{#3}{(#4) *cos(t) | (#4) * sin(t)}%
}%
\else%
\parametricplot[#1]{#2}{#3}{#4 t 57.29578 mul cos mul #4 t 57.29578 mul sin mul}%
\fi%
}
\makeatother

\newcommand{\fcPolarCurveEvaluateX}[2]{
1 dict begin /t #1 def #1 57.29578 mul cos #2 mul end
}

\newcommand{\fcPolarCurveEvaluateY}[2]{
1 dict begin /t #1 def #1 57.29578 mul sin #2 mul end
}

\newcommand{\fcPolarCurveEvaluateXY}[2]{
\fcPolarCurveEvaluateX{#1}{#2} \fcPolarCurveEvaluateY{#1}{#2}
}

\newcommand{\fcPolarWedge}[3]{%
\ifPst@algebraic{%
\rput(0,0){Set algebraic to FALSE}%
}%
\else%
\pstVerb{%
%/firstX 1 dict begin /t #1 def #1 57.29578 mul sin #2 mul end def%
/firstX \fcPolarCurveEvaluateX{#1}{#3} def%
/firstY \fcPolarCurveEvaluateY{#1}{#3} def%
/secondX \fcPolarCurveEvaluateX{#2}{#3} def%
/secondY \fcPolarCurveEvaluateY{#2}{#3} def%
}%
\pscustom[fillcolor=\fcColorAreaUnderGraph, fillstyle=solid, linecolor=blue]{%
\psline(0,0)(! \fcPolarCurveEvaluateXY{#1}{#3} )(! \fcPolarCurveEvaluateXY{#2}{#3})(0,0)%
}%
\fi%
}%

\newcommand{\fcPolarWedgeSequence}[4]{%
\multido{\ra=#1+#2}{#3}{%
\fcPolarWedge{\ra}{\ra\space #2 add}{#4}
}%
}

\newcommand{\fcRegularNgon}[3][linecolor=\fcColorGraph]{%
\multido{\ra=0+1}{#2}{%
\psline[#1](! \ra \space #2 div 360 mul cos #3 mul \ra \space #2 div 360 mul sin #3 mul)(! \ra \space 1 add #2 div 360 mul cos #3 mul \ra \space 1 add #2 div 360 mul sin #3 mul)%
}%end multido
}

\newcommand{\fcEvaluateT}[2]{%
1 dict begin /t #1 def #2 end
}

\newcommand{\fcPolylineAlongCurve}[5][linecolor=\fcColorGraph]{%
\multido{\ra=0+1}{#2}{%
\psline[#1](! \fcEvaluateT{\ra\space #2 div #3 mul 1 \ra \space #2 div sub #4 mul add}{#5})(! \fcEvaluateT{\ra\space 1 add #2 div #3 mul 1 \ra \space 1 add #2 div sub #4 mul add}{#5})%
\rput(! \fcEvaluateT{\ra\space #2 div #3 mul 1 \ra \space #2 div sub #4 mul add}{#5}){\fcFullDot{0}{0}}%
}%
\rput(! \fcEvaluateT{#3}{#5}){\fcFullDot{0}{0}}%
}

\newcommand{\fcPolylineAlongCurveWithLabels}[6][linecolor=\fcColorGraph]{%
\fcPolylineAlongCurve[#1]{#2}{#3}{#4}{#5}%
\multido{\ia=0+1}{#2}{%
\rput[b](! \fcEvaluateT{\ia\space #2 div #3 mul 1 \ia \space #2 div sub #4 mul add}{#5} 0.1 add){${#6}_{\ia}$}%
}%
\rput[b](! \fcEvaluateT{#3}{#5}){${#6}_{#2}$}%
}

\newcommand{\fcVectorNormalize}{ %
1 dict begin %
/theV exch def % theV is our vector
theV 1 theV \fcVectorNorm div \fcVectorTimesScalar %
end %
} %pushes elements of array onto the stack

\newcommand{\fcArrayToStack}{ %
aload pop
} %pushes elements of array onto the stack

\newcommand{\fcSpliceArrayOperationArray}{ %
5 dict begin %
/theOp exch def %
/secondV exch def %
/firstV exch def %
/counter 0 def %
/dimension firstV length def %
[dimension {firstV counter get secondV counter get theOp /counter counter 1 add def } repeat] %
end %
} %splices two arrays and operation, for example [a b] [c d] {op} -> [a c op b d op]

\newcommand{\fcSpliceArrayOperation}{ %
4 dict begin %
/theOp exch def %
/firstV exch def %
/counter 0 def %
/dimension firstV length def %
[ dimension {firstV counter get theOp /counter counter 1 add def } repeat ] %
end %
} %splices array with operation. [a b] {op} -> [a op b op]

\newcommand{\fcArrayOperation}{ %
4 dict begin %
/theOp exch def %
/firstV exch def %
/counter 0 def%
/dimension firstV length def %
dimension {firstV counter get /counter counter 1 add def} repeat %
dimension 1 sub {theOp} repeat %
end %
} %applies operation n-1 times to array. Example: [a b c] {op} -> a b c op op

\newcommand{\fcVectorScalarVector}{%
{mul} \fcSpliceArrayOperationArray {add}\fcArrayOperation
} %Scalar product two vectors

\newcommand{\fcVectorPlusVector}{%
{add} \fcSpliceArrayOperationArray %
} %Adds two vectors

\newcommand{\fcVectorMinusVector}{%
{sub} \fcSpliceArrayOperationArray %
} %Adds two vectors

\newcommand{\fcVectorTimesScalar}{ %
2 dict begin %
/theScalar exch def %
/theV exch def %
theV {theScalar mul} \fcSpliceArrayOperation %
end %
} %

\newcommand{\fcVectorTripleProduct}{%
\fcVectorCrossVector \fcVectorScalarVector\space %
}

\newcommand{\fcVectorCrossVector}{ %
8 dict begin %
/vectB exch def %
/vectA exch def %
vectA \fcArrayToStack %
/a3 exch def %The three coordinates of Vector a
/a2 exch def %
/a1 exch def %
vectB \fcArrayToStack %
/b3 exch def %The three coordinates of Vector b
/b2 exch def %
/b1 exch def %
[a2 b3 mul a3 b2 mul sub a3 b1 mul a1 b3 mul sub a1 b2 mul a2 b1 mul sub] %the cross product of a and b
end %
}

\newcommand{\fcVectorNorm}{%
dup \fcVectorScalarVector sqrt %
} %

\newcommand{\fcVectorNormSquared}{%
dup \fcVectorScalarVector %
} %

\newcommand{\fcProjectOntoScreen}{%
%(calling project onto plane with arguments:) == %
%dup == %
3 dict begin %
\fcScreen\space %
/theD exch def %
/theNormal exch def %
/theV exch def %
theV theNormal theD theV theNormal \fcVectorScalarVector sub theNormal \fcVectorNormSquared div \fcVectorTimesScalar \fcVectorPlusVector %
end %
} %Projection of point onto a plane. First argument is point, second argument is plane normal, third argument is the scalar product you need to have with the normal to be in the plane. Format: [1 2 3] [4 5 6] 7, corresponds to projecting the point (1,2,3) onto the plane 4x+5y+6z=7

\newcommand{\fcCoordsIIIdToPStricks}{%
5 dict begin %
/theV exch def %
/theVprojected theV \fcProjectOntoScreen [0 0 0] \fcProjectOntoScreen  \fcVectorMinusVector def%
/theNormalizedNormal \fcScreen\space pop \fcVectorNormalize def %
(\fcScreenStyle) (z) eq %
{ %
/theYUnitV [0 0 1] \fcProjectOntoScreen [0 0 0] \fcProjectOntoScreen \fcVectorMinusVector \fcVectorNormalize def %
/theXUnitV theNormalizedNormal theYUnitV \fcVectorCrossVector def %
} %
{ %
(\fcScreenStyle) (x) eq %
{
/theXUnitV [1 0 0] \fcProjectOntoScreen [0 0 0] \fcProjectOntoScreen \fcVectorMinusVector \fcVectorNormalize def %
/theYUnitV theXUnitV theNormalizedNormal \fcVectorCrossVector def%
}
{
/theYUnitV \fcScreenStyle \fcProjectOntoScreen [0 0 0] \fcProjectOntoScreen \fcVectorMinusVector \fcVectorNormalize def%
/theXUnitV theNormalizedNormal theYUnitV \fcVectorCrossVector def%
} ifelse%
}%
ifelse %
%(normalized normal: ) == theNormalizedNormal ==
%(y unit v) == theYUnitV ==
%(x unit v: ) == theXUnitV ==
theVprojected theXUnitV \fcVectorScalarVector theVprojected theYUnitV \fcVectorScalarVector
end %
}

\newcommand{\fcCoordsIIIdToPS}{%
[ exch \fcCoordsIIIdToPStricks ] \fcCoordsPStricksToPS
}

\newcommand{\fcBoxIIId}[5][]{%
\pstVerb{%
4 dict begin%
/visibleCorner #2 def%
/vectorOne #3 #2 \fcVectorMinusVector def%
/vectorTwo #4 #2 \fcVectorMinusVector def%
/vectorThree #5 #2 \fcVectorMinusVector def%
}%
\fcPolyLineIIId[#1]{visibleCorner dup vectorOne \fcVectorPlusVector dup vectorTwo \fcVectorPlusVector dup vectorOne \fcVectorMinusVector dup vectorTwo \fcVectorMinusVector visibleCorner}%
\fcPolyLineIIId[#1]{visibleCorner dup vectorOne \fcVectorPlusVector dup vectorThree \fcVectorPlusVector dup vectorOne \fcVectorMinusVector dup vectorThree \fcVectorMinusVector}%
\fcPolyLineIIId[#1]{visibleCorner vectorTwo \fcVectorPlusVector dup vectorThree \fcVectorPlusVector dup vectorTwo \fcVectorMinusVector}%
\fcPolyLineIIId[#1, linestyle=dashed]{visibleCorner vectorOne  vectorTwo vectorThree \fcVectorPlusVector \fcVectorPlusVector \fcVectorPlusVector dup vectorOne \fcVectorMinusVector}%
\fcPolyLineIIId[#1, linestyle=dashed]{visibleCorner vectorOne  vectorTwo vectorThree \fcVectorPlusVector \fcVectorPlusVector \fcVectorPlusVector dup vectorTwo \fcVectorMinusVector}%
\fcPolyLineIIId[#1, linestyle=dashed]{visibleCorner vectorOne  vectorTwo vectorThree \fcVectorPlusVector \fcVectorPlusVector \fcVectorPlusVector dup vectorThree \fcVectorMinusVector}%
\pstVerb{end}%
}

\newcommand{\fcBoxIIIdFilled}[5][]{%
\pscustom*[#1]{%
\fcPolyLineIIId{4 dict begin%
/visibleCorner #2 def%
/vectorOne #3 #2 \fcVectorMinusVector def%
/vectorTwo #4 #2 \fcVectorMinusVector def%
/vectorThree #5 #2 \fcVectorMinusVector def %
visibleCorner vectorOne \fcVectorPlusVector dup vectorTwo \fcVectorPlusVector dup vectorOne \fcVectorMinusVector dup vectorThree \fcVectorPlusVector dup vectorTwo \fcVectorMinusVector dup vectorOne \fcVectorPlusVector visibleCorner vectorOne \fcVectorPlusVector end %
}%
}%
}

\newcommand{\fcParallelogramIIId}[4][linecolor=cyan!30]{%
\pscustom*[#1]{%
\fcParallelogramHollowIIId{#2}{#3}{#4}%
}%
}

\newcommand{\fcParallelogramHollowIIId}[4][]{ %
\fcPolyLineIIId[#1]{3 dict begin /corner #2 def /vectorOne #3 #2 \fcVectorMinusVector def /vectorTwo #4 #2 \fcVectorMinusVector def corner dup vectorOne \fcVectorPlusVector dup vectorTwo \fcVectorPlusVector dup vectorOne \fcVectorMinusVector corner end
}%
}

\newcommand{\fcParallelogramHalfVisibleIIId}[4][]{%
\pstVerb{3 dict begin /corner #2 def /vectorOne #3 #2 \fcVectorMinusVector def /vectorTwo #4 #2 \fcVectorMinusVector def}%
\fcPolyLineIIId[#1]{corner vectorOne \fcVectorPlusVector corner dup vectorTwo \fcVectorPlusVector}%
\fcPolyLineIIId[#1,linestyle=dashed]{corner vectorOne \fcVectorPlusVector dup vectorTwo \fcVectorPlusVector dup vectorOne \fcVectorMinusVector}%
\pstVerb{end}%
}

\newcommand{\fcPolyLineIIId}[2][linecolor=black]{%
\listplot[#1]{ [#2] {\fcCoordsIIIdToPStricks} \fcSpliceArrayOperation \fcArrayToStack}%
}

\makeatletter %makes the @ symbol usable temporarily

\newcommand{\fcConvertPSXYUnit}{\fcShiftX\space sub \stripPoints{\psxunit} 72.27 div 8000 mul mul\space %
\pst@number\pst@dima\space %3 sub 
72.27 div 8000 mul sub  %
}%I have no clue why this works. Since the pstricks.tex code is a cesspool of filth, it is not possible for me to decode what it does, so this is the best I could do.
\newcommand{\fcConvertPSYUnit}{\fcShiftY\space sub \stripPoints{\psyunit} 72.27 div -8000 mul mul\space %
\pst@number\pst@dimb\space 72.27 div -8000 mul sub %
}%I have no clue why this works. Since the pstricks.tex code is a cesspool of filth, it is not possible for me to decode what it does, so this is the best I could do.
\makeatother

%\newcommand{\fcCoordsPStricksToPS}{\fcArrayToStack \fcConvertPSYUnit exch \fcConvertPSXUnit exch\space }
\makeatletter
\newcommand{\fcCoordsPStricksToPS}{\fcArrayToStack \tx@ScreenCoor\space }
\makeatother

\newcommand{\fcLine}[3][]{%
\pscustom{%
\code{%
1 setlinewidth newpath %
#2\space \fcCoordsPStricksToPS moveto %
#3\space \fcCoordsPStricksToPS lineto %
stroke %
}%
}%
}

\newcommand{\fcLineFormatCode}{\fcDashesCode \fcLineWidth\space setlinewidth }

\newcommand{\fcCurveCode}{%
%(calling fcCurveCode) == %
5 dict begin %
%newpath 0 0 moveto 1000 1000 lineto stroke
/theCurve exch def %
%theCurve == %
/tMin exch def%
/tMax exch def%
/Delta tMax tMin sub \fcPlotPoints \space 1 sub div def %
/t tMin def %
\fcLineFormatCode %
newpath %
theCurve \fcCoordsPStricksToPS moveto %
\fcPlotPoints\space 1 sub {/t t Delta add def theCurve \fcCoordsPStricksToPS lineto %
} repeat %
stroke %
end\space%
}

\newcommand{\fcCurve}[4][]{%
\setkeys{fcGraphics}{#1}%
\pstVerb{#2\space #3\space {#4} \space \fcCurveCode}%
}

\newcommand{\fcLineIIId}[3][linecolor=black]{%
\psline[#1](! #2 \space \fcCoordsIIIdToPStricks)(! #3 \space \fcCoordsIIIdToPStricks)%
}
\newcommand{\fcAxesIIIdFull}[4][linecolor=black, arrows=->]{%
\fcAxesIIId[#1]{#2}{#3}{#4}%
\fcLineIIId[#1]{[0 0 0]}{[#2\space -1 mul 0 0]}%
\fcLineIIId[#1]{[0 0 0]}{[0 #3\space -1 mul 0]}%
\fcLineIIId[#1]{[0 0 0]}{[0 0 #4\space -1 mul]}%
} %

\newcommand{\fcAxesIIId}[4][linecolor=black, arrows=->]{%
\setkeys{fcGraphics}{#1}%
\fcLineIIId[#1]{[0 0 0]}{[#2 0 0]}%
\rput(! [#2 0 0] \fcCoordsIIIdToPStricks){\fcXLabel}%
\fcLineIIId[#1]{[0 0 0]}{[0 #3 0]}%
\rput(! [0 #3 0] \fcCoordsIIIdToPStricks){\fcYLabel}%
\fcLineIIId[#1]{[0 0 0]}{[0 0 #4]}%
\rput(! [0 0 #4] \fcCoordsIIIdToPStricks){\fcZLabel}%
}

\newcommand{\fcDotIIId}[2][linecolor=\fcColorGraph]{%
\pscircle*[#1](! #2 \fcCoordsIIIdToPStricks){0.07} %
} %

\newcommand{\fcPutIIId}[3][]{ \rput[#1](! #2 \fcCoordsIIIdToPStricks) {#3}%
} %

\newcommand{\fcPaintCone}{ %
\fcArrayToStack %
15 dict begin %
/c exch def %
/b exch def %
/a exch def %
/z1 exch def %
/y1 exch def %
/x1 exch def %
/zmax exch def %
/zmin exch def %
}

\newcommand{\fcZBufferRowColumn}{ %
\fcCoordsIIIdToPStricks %
2 dict begin %
/rowIndex exch \space getZBufferYmin sub getZBufferYmax getZBufferYmin sub div \fcZBufferNumYIntervals\space mul floor cvi def %
/columnIndex exch getZBufferXmin sub getZBufferXmax getZBufferXmin sub div \fcZBufferNumXIntervals\space mul floor cvi def %
rowIndex \fcZBufferNumYIntervals\space ge {/rowIndex rowIndex 1 sub def}if %
columnIndex \fcZBufferNumXIntervals\space ge {/columnIndex columnIndex 1 sub def}if %
rowIndex \fcZBufferNumYIntervals\space ge {(ERROR: bad row index!!!) == rowIndex ==}if %
columnIndex \fcZBufferNumXIntervals\space ge {(ERROR: bad column index: ) == columnIndex == }if %
rowIndex 0 lt {/rowIndex rowIndex 1 add def}if %
columnIndex 0 lt {/columnIndex columnIndex 1 add def}if %
rowIndex 0 lt {(ERROR: bad row index!!!) == rowIndex ==}if %
columnIndex 0 lt {(ERROR: bad column index: ) == columnIndex ==  }if %
rowIndex columnIndex %
end %
}

\newcommand{\fcZBufferGetForeMostDepthAtPoint}{ %
5 dict begin %
/thePoint exch def %
thePoint \fcZBufferRowColumn %
/column exch def %
/row exch def %
theZBuffer row get column get %
dup (infty) eq {( <br>ERROR: z-buffer not computed properly, row, column: ) print row == column ==} if %
end %
}

\newcommand{\fcIsInForeGround}{ %
1 dict begin 
/thePoint exch def
thePoint \fcZBufferGetForeMostDepthAtPoint 
dup (infty) ne {
thePoint \fcScreen\space pop \fcVectorScalarVector 
sub -0.001 gt
} 
{ pop true
} ifelse
end
}

\newcommand{\fcZBufferAccountPoint}{ %
5 dict begin %
/thePoint exch def %
thePoint \fcZBufferRowColumn %
/currentColumn exch def %
/currentRow exch def %
/Zdepth theZBuffer currentRow get currentColumn get def %
/currentZdepth thePoint \fcScreen\space pop \fcVectorScalarVector def %
Zdepth (infty) eq {theZBuffer currentRow get currentColumn currentZdepth put }{ currentZdepth Zdepth lt {theZBuffer currentRow get currentColumn currentZdepth put} if %
}ifelse %
end %
}

\newcommand{\fcZBufferAccountBoundingBoxPoint}{ %
%Account bounding box:
\fcCoordsIIIdToPStricks % 
dup dup getZBufferYmin lt {setZBufferYmin}{pop}ifelse %
dup getZBufferYmax gt {setZBufferYmax}{pop}ifelse %
dup dup getZBufferXmin lt {setZBufferXmin}{pop}ifelse %
dup getZBufferXmax gt {setZBufferXmax}{pop}ifelse \space%
}

\newcommand{\fcZBufferSurface}{ %
(computing z buffer surface) == %
pstack
/theSurface exch def %
/vmax exch def %
/umax exch def %
/vmin exch def %
/umin exch def %
0 0 0 0
}

\newcommand{\fcZBufferEllipsoid}{ %
%(calling fcZBufferEllipsoid with input: ) == dup == %
\fcArrayToStack
6 dict begin
/c exch def %
/b exch def %
/a exch def %
/z1 exch def %
/y1 exch def %
/x1 exch def %
[a -1 mul x1 add b -1 mul y1 add c -1 mul z1 add a x1 add b y1 add c z1 add] %
end
}

\newcommand{\fcBoundingBoxCone}{ %
%(calling fcBoundingBoxCone with input: ) == dup ==
\fcArrayToStack %
9 dict begin %
/c exch def %
/b exch def %
/a exch def %
/z1 exch def %
/y1 exch def %
/x1 exch def %
/zmax exch abs def %
abs dup zmax gt {/zmax exch def}{pop}ifelse %
[a c div zmax sqrt div -1 mul b c div zmax sqrt div -1 mul zmax -1 mul  
a c div zmax sqrt div b c div zmax sqrt div zmax
] %
end %
}

\newcommand{\fcSegmentBoundingBox}{ %
\fcZBufferAccountBoundingBoxPoint %
\fcZBufferAccountBoundingBoxPoint %
}

\newcommand{\fcZBufferSegment}{ %
5 dict begin %
/secondPoint exch def %
/firstPoint exch def %
/numPointsToAccount %
[firstPoint \fcZBufferRowColumn ] %
[secondPoint\fcZBufferRowColumn ] %
\fcVectorMinusVector \fcVectorNorm 2 mul 2 add cvi def %
/Delta 1 numPointsToAccount 1 sub div def %
/t 0 def %
numPointsToAccount {secondPoint t \fcVectorTimesScalar firstPoint 1 t sub \fcVectorTimesScalar \fcVectorPlusVector \fcZBufferAccountPoint
/t t Delta add def %
}repeat %
end %
%(zbuffer after accounting: ) ==
%theZBuffer ==
}

\newcommand{\fcSegmentPaint}{ %
2 dict begin %
/firstPoint exch def %
/secondPoint exch def %                                                                                                                                                                                                                                                     
firstPoint \fcIsInForeGround secondPoint \fcIsInForeGround or %
{[] 0 setdash} %
{[1 1] 0 setdash} %
ifelse %
newpath %
firstPoint \fcCoordsIIIdToPS moveto %
secondPoint \fcCoordsIIIdToPS lineto %
stroke %
end %
}

\newcommand{\fcPaintZbuffForDebug}{ %
6 dict begin %
/DeltaX getZBufferXmax getZBufferXmin sub \fcZBufferNumXIntervals\space div def %
/DeltaY getZBufferYmax getZBufferYmin sub \fcZBufferNumYIntervals\space div def %
gsave %
0.1 setlinewidth %
/x getZBufferXmin def %
0.5 0.5 0.5 setrgbcolor %
\fcZBufferNumYIntervals {newpath [x getZBufferYmin] \fcCoordsPStricksToPS moveto [x getZBufferYmax] \fcCoordsPStricksToPS lineto stroke /x x DeltaX add def}repeat %
/y getZBufferYmin def %
\fcZBufferNumXIntervals { newpath [getZBufferXmin y] \fcCoordsPStricksToPS moveto [getZBufferXmax y] \fcCoordsPStricksToPS lineto stroke /y y DeltaY add def}repeat %
/y getZBufferYmin DeltaY 2 div add def %
/counterY 0 def %
\fcZBufferNumYIntervals { %
/x getZBufferXmin DeltaX 2 div add def %
/counterX 0 def %
\fcZBufferNumXIntervals { %
theZBuffer counterY get counterX get length 0 gt{ %
%[x y] \fcFullDotCode %
} if %
/x x DeltaX add def %
/counterX counterX 1 add def %
}repeat %
/y y DeltaY add def %
/counterY counterY 1 add def %
}repeat %
grestore %
end %
}

\newcommand{\fcStartIIIdScene}{%
\pstVerb{%
1 dict begin %
/theIIIdObjects [] def %
}%
}

\newcommand{\fcFinishIIIdScene}{%
\pscustom{%
\code{%
%print the objects we are about to paint:
theIIIdObjects length 0 gt { %
(about to process IIId scene given by: ) print %
theIIIdObjects == %
} if %
20 dict begin %
/ZBufferRectangle [0 0 0 0] def %
/setZBufferXmin {ZBufferRectangle exch 0 exch put} def %
/setZBufferYmin {ZBufferRectangle exch 1 exch put} def %
/setZBufferXmax {ZBufferRectangle exch 2 exch put} def %
/setZBufferYmax {ZBufferRectangle exch 3 exch put} def %
/getZBufferXmin {ZBufferRectangle 0 get} def %
/getZBufferYmin {ZBufferRectangle 1 get} def %
/getZBufferXmax {ZBufferRectangle 2 get} def %
/getZBufferYmax {ZBufferRectangle 3 get} def %
/theZBuffer [\fcZBufferNumYIntervals {[\fcZBufferNumXIntervals{[]} repeat]}repeat] def %
/processCurve {\fcZBufferBoundingBoxPolyline} def%
/processPatch {pop} def %
(computing bounding box IIId scene... ) print %
theIIIdObjects \fcArrayToStack theIIIdObjects length {\fcProcessObject } repeat %
%extend slightly the bounding box to take care of floating point errors at the 
%border 
getZBufferXmin 0.1 sub setZBufferXmin %
getZBufferXmax 0.1 add setZBufferXmax %
getZBufferYmin 0.1 sub setZBufferYmin %
getZBufferYmax 0.1 add setZBufferYmax %
(bounding box computed: ) == ZBufferRectangle == %
/processCurve {pop} def %
/processPatch {\fcZBufferPatch} def %
(computing z-buffer IIId scene... ) print %
theIIIdObjects \fcArrayToStack theIIIdObjects length {\fcProcessObject } repeat %
(z buffer computed) == 
theZBuffer ==
\fcPaintZbuffForDebug %
/processCurve {\fcPolylineInScene} def %
/processCurve {\fcPatchInScene} def %
(painting IIId scene... ) print %
theIIIdObjects \fcArrayToStack theIIIdObjects length {\fcProcessObject } repeat %
end %
}%
}%
\pstVerb{end}
}%

\newcommand{\fcProcessObject}{%
(processing object) == \fcArrayToStack dup == 
1 dict begin
4 1 roll setrgbcolor %
/HandlerNotFound true def %
HandlerNotFound{dup (surface) eq {pop \fcProcessSurface /HandlerNotFound false def} if} if %
HandlerNotFound{== (ERROR: OBJECT PAINTING HANDLER NOT FOUND)} if %
end %
}%

\newcommand{\fcProcessSurface}{%
20 dict begin %
(processing Surface: ) ==
pstack
/theSurface exch def %
/vMax exch def %
/uMax exch def %
/vMin exch def %
/uMin exch def %
/uIterations \fcIterationsU\space def %
/vIterations \fcIterationsV\space def %
/DeltaU uMax uMin sub uIterations div def %
/DeltaV vMax vMin sub vIterations div def %
/u uMin def %
/v {t} def %
uIterations 1 add { %
vMin vMax {theSurface} \fcCurveIIIdInSceneCode %
/u u DeltaU add def %
}repeat %
/v vMin def %
/u {t} def %
vIterations 1 add { %
uMin uMax {theSurface} \fcCurveIIIdInSceneCode %
/v v DeltaV add def %
}repeat %
/u uMin def %
uIterations { %
/v vMin def %
vIterations{ %
[ %start of patch data structure
theSurface %(x(u,v), y(u,v), z(u,v))
/oldU u def %
/u u DeltaU add def %
theSurface %(x(u+Delta,v), y(u+Delta,v), z(u+Delta,v))
/u oldU def %
/v v DeltaV add def %
theSurface %(x(u,v+Delta), y(u,v+Delta), z(u,v+Delta))
(patch) %
] %patch data structure complete
processPatch %
}repeat %
/u u DeltaU add def %
}repeat %
end %
}%

\newcommand{\fcPaintEllipsoid}{%
(painting ellipsoid: ) == %
dup == %
\fcArrayToStack %
15 dict begin %
/c exch def %
/b exch def %
/a exch def %
/z1 exch def %
/y1 exch def %
/x1 exch def %
/theSurfaceNormal {[v cos u cos a mul mul v sin u cos b mul mul u sin -1 mul c mul] [v sin -1 mul u sin a mul mul v cos u sin b mul mul 0] \fcVectorCrossVector} def %
0 0 180 360 {[v cos u sin a mul mul x1 add v sin u sin b mul mul y1 add u cos c mul z1 add]} \fcSurfaceIIIdCode %
end %
}

\newcommand{\fcGetCurrentColorCode}{
2 dict begin
/theColor {0 0 0} def %
/colorNotFound true def %
(\fcLineColor) (black) eq{/theColor {0 0 0} def /colorNotFound false def}if %
(\fcLineColor) (red) eq{/theColor {1 0 0} def /colorNotFound false def}if %
(\fcLineColor) (blue) eq{/theColor {0 0 1} def /colorNotFound false def}if %
(\fcLineColor) (green) eq{/theColor {0 1 0} def /colorNotFound false def}if %
colorNotFound{/theColor {\fcLineColor} def}if %
theColor %
end %
}

%Let the ellipsoid be given by the set of points  (x,y,z), for which (x-x1)^2/a^2+ (y-y1)/b^2 +(z-z1)^2/c^2=1, where x1, y1, z1, a, b, c are the parameters of the ellipsoid. We store the ellipsoid as the data a, b, c in the data structure [ [x1 y1 z1 a b c] red green blue (ellipsoid)]
\newcommand{\fcEllipsoid}[2][linecolor=red]{%
\setkeys{fcGraphics}{#1}%
\pstVerb{%
9 dict begin %
%default 
mark #2\space %
/c exch def %
/b exch def %
/a exch def %
/x1 0 def %
/y1 0 def %
/z1 0 def %
counttomark 0 gt{ %
/z1 exch def %
/y1 exch def %
/x1 exch def %
}if %
pop %remove the mark
[theIIIdObjects \fcArrayToStack [[x1 y1 z1 a b c] \fcGetCurrentColorCode (ellipsoid)] ] %
end %
/theIIIdObjects exch def %
}%
}

%Let the cone be given by the set of points  (x,y,z), for which (x-x1)^2/a^2+ (y-y1)/b^2 =(z-z1)^2/c^2, where x1, y1, z1, a, b, c are the parameters of the cone. We plot the cone from zmin to zmax We store the ellipsoid as the data a, b, c in the data structure [ [zmin zmax x1 y1 z1 a b c] red green blue (ellipsoid)]
\newcommand{\fcCone}[2][linecolor=red]{%
\setkeys{fcGraphics}{#1}%
\pstVerb{%
11 dict begin %
%default 
mark #2\space %
/c exch def %
/b exch def %
/a exch def %
/x1 0 def %
/y1 0 def %
/z1 0 def %
counttomark 2 gt{ %
/x1 exch def %
/y1 exch def %
/z1 exch def %
}if %
counttomark 1 gt{ %
/zmax exch def %
/zmin exch def %
}if %
pop %remove the mark
[theIIIdObjects \fcArrayToStack [[zmin zmax x1 y1 z1 a b c] \fcGetCurrentColorCode (cone)] ] %
end %
/theIIIdObjects exch def %
}%
}

%Format of surface: we store the surface in the format [umin vmin umax vmax [fx fy fz] red green blue (surface)]
\newcommand{\fcSurfaceInScene}[6][]{%
\setkeys{fcGraphics}{#1} %
\pstVerb{ %
[theIIIdObjects \fcArrayToStack [#2\space #3\space #4\space #5\space {#6} \fcGetCurrentColorCode (surface)] ] %
/theIIIdObjects exch def %
} %
}%

\newcommand{\fcCurveIIIdNoSceneCode}{%
15 dict begin%
/theCurve exch def%
/tMax exch def%
/tMin exch def%
/numPoints \fcPlotPoints\space def%
/Delta tMax tMin sub numPoints 1 sub div def%
/t tMin def %
\fcLineFormatCode %
newpath %
theCurve \fcCoordsIIIdToPS moveto %
numPoints 1 sub {/t t Delta add def theCurve \fcCoordsIIIdToPS lineto } repeat %
stroke %
end %
}%

\newcommand{\fcZBufferBoundingBoxPolyline}{ %
{\fcZBufferAccountBoundingBoxPoint} forall
}

\newcommand{\fcZBufferPolyline}{ %
{\fcZBufferAccountPoint} \fcSpliceArrayOperation pop
}

\newcommand{\fcAreEqual}{ %
1 dict begin
/areEqual
{5 dict begin 
/left exch def
/right exch def
left type  right type ne{false}
{ left type (arraytype) ne{ left right eq}
{ left length right length ne{false}
{ /counter 0 def %
true %
left length { left  counter get right counter get areEqual not{pop false exit }if /counter counter 1 add def }repeat
}ifelse
}ifelse
}ifelse
end
} def %
areEqual
end
}

\newcommand{\fcZBufferAccountPatchXY}{ %
(patch row col: ) print x == y ==
5 dict begin %
/currentArray theZBuffer y get x get def %
/counter 0 def %
true %
currentArray length {currentArray counter get thePatch \fcAreEqual {pop false exit}if /counter counter 1 add def} repeat
{ theZBuffer y get x [currentArray \fcArrayToStack thePatch ] put}if
end %
}

\newcommand{\fcZBufferPatch}{ %
15 dict begin %
/thePatch exch def %
(Processing patch: ) print 
thePatch ==
/secondPointBuffer [thePatch 2 get \fcZBufferRowColumn] def %
/firstPointBuffer  [thePatch 1 get \fcZBufferRowColumn] def %
/basePointBuffer   [thePatch 0 get \fcZBufferRowColumn] def %
(Index vector:) print
secondPointBuffer ==
basePointBuffer ==
firstPointBuffer ==
/firstDirection firstPointBuffer basePointBuffer \fcVectorMinusVector def %
/secondDirection secondPointBuffer basePointBuffer \fcVectorMinusVector def %
/iterationsFirst firstDirection \fcVectorNorm 2 mul 1 add round cvi def %
/iterationsSecond secondDirection \fcVectorNorm 2 mul 1 add round cvi def %
/s 0 def %
iterationsFirst{ %
/firstComponent firstDirection s iterationsFirst div \fcVectorTimesScalar basePointBuffer \fcVectorPlusVector def%
/t 0 def %
iterationsSecond{ %
secondDirection t iterationsSecond div \fcVectorTimesScalar %
firstComponent \fcVectorPlusVector \fcArrayToStack %
/y exch round cvi def %
/x exch round cvi def %
\fcZBufferAccountPatchXY %
/t t 1 add def %
}repeat
/s s 1 add def %
}repeat
end %
}

\newcommand{\fcPatchInScene}{ %
}

\newcommand{\fcPolylineInScene}{ %
15 dict begin %
(got to polyline in scene) ==
/thePoints exch def %
/numSegments thePoints length 1 sub def %
/currentStyle (none) def %
/styleNext (none) def %
/numSegmentsAccounted 0 def %
newpath %
{ numSegmentsAccounted numSegments ge {exit} if %
  thePoints numSegmentsAccounted get \fcZBufferRowColumn %
  /column exch def %
  /row exch def %
  thePoints numSegmentsAccounted get \fcIsInForeGround {/styleNext (solid) def}{/styleNext (dashed) def} ifelse
  /numSegmentsNext 1 def %
  { numSegmentsAccounted numSegmentsNext add numSegments ge {exit} if %
    thePoints numSegmentsAccounted numSegmentsNext add get %
    \fcZBufferRowColumn column ne exch row ne or {exit}if 
    thePoints numSegmentsAccounted numSegmentsNext add get %
    \fcIsInForeGround {/styleNext (solid) def} if %
    /numSegmentsNext numSegmentsNext 1 add def %
  }loop %
  styleNext currentStyle ne{ %
  (unequal styles) ==
  stroke  %
  styleNext (dashed) eq {[1 1]0 setdash }{[]0 setdash}ifelse %
  newpath %
  thePoints numSegmentsAccounted get \fcCoordsIIIdToPS moveto %
  }if %
  /currentStyle styleNext def %
  numSegmentsNext { %
    /numSegmentsAccounted numSegmentsAccounted 1 add def %  
    thePoints numSegmentsAccounted get %
    \fcCoordsIIIdToPS lineto %
  }repeat
}loop
stroke
end
}

\newcommand{\fcCurveIIIdInSceneCode}{%
20 dict begin %
/theCurve exch def %
/tmax exch def %
/tmin exch def %
/numPoints \fcPlotPoints\space dup == def %
/numIntervals numPoints 1 sub dup == def % 
/Delta tmax tmin sub numIntervals div def %
/t tmin def %
/outputArray [theCurve %
numIntervals { /t t Delta add def theCurve}repeat %
]def %
(got to curve processing) ==
outputArray processCurve %
end %
}

\newcommand{\fcCurveIIId}[4][linecolor=\fcColorGraph]{%
\parametricplot[#1]{#2}{#3}{#4 \fcCoordsIIIdToPStricks}%
}

\newcommand{\fcZeroVector}{[exch {0} repeat]}

\newcommand{\fcPerpendicularComputeHeel}[3]{%
\pstVerb{%
7 dict begin%
/thePoint #1 def%
/heelSize #3 def %
mark #2 %
counttomark 1 eq {%
/directionUnitVector exch \fcVectorNormalize def%
/basePoint thePoint length \fcZeroVector def%
}{%
/basePoint exch def%
/directionUnitVector exch basePoint \fcVectorMinusVector \fcVectorNormalize def%
} ifelse %
pop%
/heel directionUnitVector thePoint basePoint \fcVectorMinusVector directionUnitVector \fcVectorScalarVector \fcVectorTimesScalar basePoint \fcVectorPlusVector def%
%heel == %
/perpendicularUnitVector thePoint heel \fcVectorMinusVector \fcVectorNormalize def %
%perpendicularUnitVector == %
/polyLineInput {%
heel directionUnitVector heelSize \fcVectorTimesScalar \fcVectorMinusVector %
%dup ==
dup perpendicularUnitVector heelSize \fcVectorTimesScalar \fcVectorPlusVector %
heel perpendicularUnitVector heelSize \fcVectorTimesScalar \fcVectorPlusVector%
} def%
}%
}

\newcommand{\fcPerpendicular}[4][]{%
\fcPerpendicularComputeHeel{#2}{#3}{#4}%
\psline[#1](! thePoint \fcArrayToStack)(! heel \fcArrayToStack)%
\listplot[linecolor=red]{ [polyLineInput] {\fcArrayToStack} \fcSpliceArrayOperation \fcArrayToStack}%
\pstVerb{end}%
}

\newcommand{\fcPerpendicularIIId}[4][]{%
\fcPerpendicularComputeHeel{#2}{#3}{#4}%
\fcLineIIId[#1]{thePoint}{heel}%
\fcPolyLineIIId[linecolor=red]{polyLineInput}%
\pstVerb{end}%
}%

\newcommand{\fcPlotIIId}[7][]{%
\fcPlotIIIdXconst[#1]{#2}{#3}{#4}{#5}{#6}{#7}%
\fcPlotIIIdYconst[#1]{#2}{#3}{#4}{#5}{#6}{#7}%
}
\newcommand{\fcPlotIIIdXconst}[7][]{%
\setkeys{fcGraphics}{#2}%
\multido{\ra=0+1}{\fcIterationsX}{%
\pstVerb{%
3 dict begin %
/x \ra \space #3 mul \fcIterationsX \space \ra \space sub 1 sub  #5\space mul add \fcIterationsX\space 1 sub div def%
/ymin #4 def%
/ymax #6 def%
}%
\parametricplot[#1]{ymin}{ymax}{%
1 dict begin /y t def  [x y #7] \fcCoordsIIIdToPStricks end%
}%
\pstVerb{end}%
}%end multido
}

\newcommand{\fcPlotIIIdYconst}[7][]{%
\setkeys{fcGraphics}{#2}%
\multido{\ra=0+1}{\fcIterationsY}{%
\pstVerb{%
3 dict begin%
/y \ra \space #4 mul \fcIterationsY \space \ra \space sub 1 sub  #6\space mul add \fcIterationsY\space 1 sub div def%
/xmin #3 def%
/xmax #5 def%
}%
\parametricplot[#1]{xmin}{xmax}{%
1 dict begin /x t def  [x y #7] \fcCoordsIIIdToPStricks end%
}%
\pstVerb{end}%
}%end multido
}

\newcommand{\fcSurfaceIIIdUConst}[7][]{%
\setkeys{fcGraphics}{#2}%
\multido{\ra=0+1}{\fcIterationsX}{%
\pstVerb{%
3 dict begin%
/u \ra \space #3 mul \fcIterationsX \space \ra \space sub 1 sub  #5\space mul add \fcIterationsX\space 1 sub div def%
/vmin #4 def%
/vmax #6 def%
}%
\parametricplot[#1]{vmin}{vmax}{%
1 dict begin /v t def #7 \fcCoordsIIIdToPStricks end%
}%
\pstVerb{end}%
}%end multido
}

\newcommand{\fcSurfaceIIIdVConst}[7][]{%
\setkeys{fcGraphics}{#2}%
\multido{\ra=0+1}{\fcIterationsY}{%
\pstVerb{%
3 dict begin%
/v \ra \space #4 mul \fcIterationsY \space \ra \space sub 1 sub  #6\space mul add \fcIterationsY\space 1 sub div def%
/umin #3 def%
/umax #5 def%
}%
\parametricplot[#1]{umin}{umax}{%
1 dict begin /u t def #7 \fcCoordsIIIdToPStricks end%
}%
\pstVerb{end}%
}%end multido
}

\newcommand{\fcSurfaceDirectDraw}[7][]{%
\fcSurfaceIIIdUConst[#1]{#2}{#3}{#4}{#5}{#6}{#7}%
\fcSurfaceIIIdVConst[#1]{#2}{#3}{#4}{#5}{#6}{#7}%
}%

\newcommand{\fcVectorField}[3][linecolor=blue]{%
\setkeys{fcGraphics}{#2}%
\multido{\ra=\fcStartXIId+\fcDelta}{\fcIterationsX}{%
\multido{\rb=\fcStartYIId+\fcDelta}{\fcIterationsY}{%
\pstVerb{%
4 dict begin%
/x \ra\space def%
/y \rb\space def %
#3\space%
/vY exch def%
/vX exch def%
}%
\psline[#1](! x vX 2 div sub y vY 2 div sub)(! x vX 2 div add y vY 2 div add)%
\pscircle*[linecolor=red](! x y){0.02}%
\pstVerb{end}%
}%end multido
}%end multido
}%


\usepackage{pst-3dplot}


\usepackage[T1]{fontenc}
% Or whatever. Note that the encoding and the font should match. If T1
% does not look nice, try deleting the line with the fontenc.

\graphicspath{{../../modules/}}

\setbeamertemplate{navigation symbols}{}

\newcommand{\lect}[4]{
\ifnum#3=\currentLecture
  \date{#1}
  \lecture[#1]{#2}{#3}
#4
\else
%include nothing
\fi
}

\setbeamertemplate{footline}
{
  \leavevmode%
  \hbox{%
  \begin{beamercolorbox}[wd=.333333\paperwidth,ht=2.25ex,dp=1ex,center]{author in head/foot}%
    \usebeamerfont{author in head/foot}\insertshortauthor
  \end{beamercolorbox}%
  \begin{beamercolorbox}[wd=.333333\paperwidth,ht=2.25ex,dp=1ex,center]{title in head/foot}%
    \usebeamerfont{title in head/foot}\insertshorttitle
  \end{beamercolorbox}%
  \begin{beamercolorbox}[wd=.333333\paperwidth,ht=2.25ex,dp=1ex,center]{date in head/foot}%
    \usebeamerfont{date in head/foot}\insertshortdate{}
  \end{beamercolorbox}}%
  \vskip0pt%
}

% If you have a file called "university-logo-filename.xxx", where xxx
% is a graphic format that can be processed by latex or pdflatex,
% resp., then you can add a logo as follows:

%\pgfdeclareimage[height=0.8cm]{logo}{bluelogo}
%\logo{\pgfuseimage{logo}}

\begin{document}

\AtBeginLecture{%

\title[\insertlecture]{Math 130}
\subtitle{\insertlecture}
\author[FreeCalc \\ Math 130]{Greg Maloney \\~\\  Todor Milev}

%when people substitute for me:
%\author[FreeCalc \\ Math 140]{ Instructor: Shuang Cai }
\institute[UMass Boston]{University of Massachusetts Boston}
\date{\insertshortlecture}
\begin{frame}
  \titlepage
\end{frame}

\begin{frame}{Outline}
  \tableofcontents[pausesections]
\end{frame}
}%

\lect{\semester}{Lecture  1}{1}{
%DesiredLectureName: Cartesian_Coordinates_Vector_Addition_Scalar_Product
\fcLicense
\section{Cartesian coordinate system}
\begin{frame}
\frametitle{Rectangular/Cartesian Coordinates}
\begin{columns}
\column{0.3\textwidth}
\psset{xunit=2cm, yunit=2cm}
\begin{pspicture}(-1 ,-1)(1, 1)
\fcBoundingBox{-1}{-1}{1}{1}
\tiny
\psline[arrows=->](0,-1)(0,1)
\psline[arrows=->](-1,0)(1,0)
\rput[t](1, -0.1){$x$}
\rput[l](-0.1, 1){$y$}
\fcFullDot{0}{0}
\end{pspicture}


\column{0.7\textwidth}
\begin{itemize}
\item The Cartesian (rectangular) coordinate system is a way to represent points on the plane.
\item To introduce Cartesian coordinates, fix:
\begin{itemize}
\item<2-> a point $O$ (called the origin),
\item<3-> 2 pairwise perpendicular lines intersecting at the origin, called axes,
\item<4-> a direction in each of the coordinate axis.
\end{itemize}
\item<5-> The axes are labeled as $x$-axis and $y$-axis. 
\item<6-> The $x$ axis is drawn horizontal with direction pointing from left to right.
\item<7-> The $y$ axis is drawn vertical, pointing up.
\item<8-> The Cartesian coordinate system is named after Ren\'e Descartes (1596-1650) (Latinized name: Cartesius).
\end{itemize}

\vskip 3cm
\end{columns}
\end{frame}
\begin{frame}
\frametitle{Rectangular/Cartesian Coordinates}
\begin{columns}
\column{0.3\textwidth}
\psset{xunit=2cm, yunit=2cm}
\begin{pspicture}(-1 ,-1)(1, 1)%
\fcBoundingBox{-1}{-1}{1}{1}%
\tiny%
\psline[arrows=->](0,-1)(0,1)%
\psline[arrows=->](-1,0)(1,0)%
\rput[t](1, -0.1){$x$}%
\rput[l](-0.1, 1){$y$}%
%\fcFullDot{0}{0}
\fcFullDot{0.7}{0.5}%
\rput[b](0.7, 0.6){$P(x,y)$}%
\fcPerpendicular{[0.7 0.5]}{[0 1]}{0.07}%
\fcPerpendicular{[0.7 0.5]}{[1 0]}{0.07}%
\fcLengthIndicator[arrows=->]{0}{-0.1}{0.7}{-0.1}{$x$}%
\fcLengthIndicator[arrows=->]{-0.1}{0}{-0.1}{0.5}{$y$}%
\psline[linecolor=red](-0.05,0)(-0.15,0)
\psline[linecolor=red](-0.05,0.5)(-0.15,0.5)
\psline[linecolor=red](0,-0.05)(0,-0.15)
\psline[linecolor=red](0.7,-0.05)(0.7,-0.15)
\end{pspicture}
\column{0.7\textwidth}
\begin{itemize}
\item<1-> Let $P$ -point. We assign to it a pair of numbers $(x,y)$.
\item<2-> Assignment will be such that distinct points are assigned distinct pairs.
\item<3-> $Q=$ base of perpendicular from $P$ to $x$-axis.
\item<4-> Define $x$ as \alertNoH{5}{signed distance b-n $O$ and $Q$}.
\item<5-> Take distance with \alertNoH{5}{$+$ sign if $OQ$ points in direction of $x$-axis, $-$ sign else}.
\item<6-> To define $y$, do the same with the $y$ axis.
\item<7-> $(x,y)$ = Cartesian coordinates of $P$.
\item<8-> $x$ is called the $x$-coordinate of $P$, $y$- the $y$ coordinate.
\item<9-> $(x,y)$ = singed lengths of sides of the rectangle indicated in the picture.
\end{itemize}

\vfill
\end{columns}

\vskip 5cm

\end{frame}

\begin{frame}
\frametitle{Rectangular/Cartesian Coordinates}
\begin{columns}
\column{0.3\textwidth}
\psset{xunit=2cm, yunit=2cm}
\begin{pspicture}(-1 ,-1)(1.1, 1.1)%
\fcBoundingBox{-1}{-1}{1}{1}%
\tiny%
\psline[arrows=->](0,-1)(0,1)%
\psline[arrows=->](-1,0)(1,0)%
\rput[t](1, -0.1){$x$}%
\rput[l](-0.1, 1){$y$}%
%\fcFullDot{0}{0}
\rput(0.5, 0.5){Quadrant I}
\rput(-0.5, 0.5){Quadrant II}
\rput(-0.5, -0.5){Quadrant III}
\rput(0.5, -0.5){Quadrant IV}
\end{pspicture}

\begin{tabular}{|c|c|}\hline
Quadrant&$(x,y)$\\\hline
I & $(+,+)$\\\hline
II& $(-,+)$\\\hline
III& $(-,-)$\\\hline
IV& $(+,-)$\\\hline
\end{tabular}

\column{0.7\textwidth}
\begin{itemize}
\item The coordinate axes split the plane in 4 regions, called quadrants. 
\item The quadrants are labeled as indicated. 
\item For a point has coordinates $(x,y)$, $x\neq 0$, $y\neq 0$, the signs of $x$ and $y $ are determined by the quadrant that contains the point.

\end{itemize}

\vfill
\end{columns}

\vskip 5cm

\end{frame}

\begin{frame}
\begin{example}

\begin{columns}
\column{0.4\textwidth}
\psset{xunit=0.6cm, yunit=0.6cm}
\begin{pspicture}(-4, -4)(5,5)
\tiny
\fcBoundingBox{-4.2}{-4.2}{4.2}{4.2}
\psgrid[subgriddiv=0,griddots=10,gridlabels=0pt](-4,-4)(4,4)
\psaxes[labels=none, arrows=->](0,0)(-4,-4)(4,4)
\fcLabels{4}{4}
\uncover<3->{%
\fcFullDot{1}{2}
\rput[l](1,2){~$(1,2)$}
}
\uncover<7->{%
\fcFullDot{2}{-3}
\rput[r](2,-3){$(2,-3)~$}
}
\uncover<11->{
\fcFullDot{-3}{2}
\rput[b](-3,2.05){$(-3,2)$}
}
\uncover<15->{
\fcFullDot{-1}{-1}
\rput[tr](-1,-1){$(-1,-1)~$}
}
\end{pspicture}
\column{0.6\textwidth}
Plot the points and name the Quadrant that contains them
\begin{itemize}
\item \alertNoH{2-5}{$(1,2)$. } \uncover<4->{\alertNoH{4,5}{Quadrant} \fcAnswer{5}{I}} 
\item \alertNoH{6-9}{$(2,-3)$. } \uncover<8->{\alertNoH{8,9}{Quadrant} \fcAnswer{9}{IV}}
\item \alertNoH{10-13}{$(-3,2)$. } \uncover<12->{\alertNoH{12,13}{Quadrant} \fcAnswer{13}{II}}
\item \alertNoH{14-17}{$(-1,-1)$. } \uncover<16->{\alertNoH{16,17}{Quadrant} \fcAnswer{17}{III}}
\end{itemize}
\end{columns}

\end{example}
\end{frame}

\subsection{The Pythagorean Theorem, Euclidean Distance}
\begin{frame}
\begin{itemize}
\item A triangle is a right-angled triangle if two of its sides are perpendicular.
\item The two sides perpendicular to one another are called legs. 
\item The two legs form a right angle (90$^{\circ}$).
\item The side opposite to the right angle is called the hypothenuse. 
\end{itemize}

\begin{theorem}
Let $a,b$ be the lengths of the legs of a right-angled triangle and $c$ the length of its hypotenuse. Then
\[
a^2+b^2=c^2.
\]
\end{theorem}

\end{frame}
\begin{frame}
\begin{theorem}
Let $(x_1,y_1)$ and $(x_2, y_2)$ be two points in the plane. Then the distance $d$ between the two points is given by
\[
d=\sqrt{(x_1-x_2)^2-(y_1-y_2)^2}
\]
\end{theorem}

\end{frame}
\begin{frame}
\begin{example}
Find the distance between $(-2,3)$ and $(3,5)$.

$d=\sqrt{(x_1-x_2)^2+(y_1 -y_2)^2}= \sqrt{(-2-3)^2+ (3-5)^2} = \sqrt{(-5)^2+(-2)^2}= \sqrt{25+4}=\sqrt{29}\approx 5.385$.

\end{example}
\begin{example}
Find the distances between the indicated points.
\begin{tabular}{ccc}
$P$& $Q$& distance\\
$(2,3)$   & $(3,5)$ & $\textbf{?}$\\
$(-2,-3)$ & $(3,5)$ & $\textbf{?}$\\
$(-2,-3)$ & $(3,-5)$& $\textbf{?}$\\
$(-2, 3)$ & $(3,-5)$& $\textbf{?}$
\end{tabular}
\end{example}
\end{frame}
\begin{frame}
\begin{example}
Do the points $(1,2)$, $(2,3)$, $(4,-1)$ form a right-angled triangle?
\end{example}
\begin{example}
Do the points $(1,2)$, $(2,4)$, $(3,1)$ form a right-angled triangle?
\end{example}

\begin{example}
Do the indicated points form a right-angled triangle?
\begin{tabular}{cccc}
$(-1,-2)$& $(3,5)$& $(6,-6)$&\textbf{?}\\
$(1,2)$& $(3,5)$& $(6,6)$&\textbf{?}\\
$(0,0)$& $(2,3)$& $(3,-2)$&\textbf{?}\\
$(0,0)$& $(2,3)$& $(-2,3)$&\textbf{?}\\
\end{tabular}
\end{example}
\end{frame}
\subsection{Vectors}
\begin{frame}
\begin{columns}
\column{0.3\textwidth}
\psset{xunit=0.9cm, yunit=0.9cm}
\begin{pspicture}(-0.5,-0.5)(3.2,3.2)
\tiny
\fcAxesStandard{-0.5}{-0.5}{3}{3}
\fcLabels{3}{3}
\uncover<9-13>{%
\rput[lt](1, 0.45){$(x_1, y_1)$}
}
\uncover<10-13>{%
\rput[br](0.5, 1.55){$(x_2, y_2)$}
}
\uncover<10->{%
\fcFullDot{0}{0}
\rput[tr](0, 0){$O~~$}
}%
\uncover<12->{\fcFullDot{1.5}{2}}
\uncover<12,13>{\rput[b](1.5, 2.05){$(x_1+x_2, y_1+ y_2)$}}
\uncover<14,15>{
\psline[linecolor=blue](3,1.5)(-0.5, -0.25) 
\psline[linecolor=blue](3,2.75)(-0.5, 1) 
}
\uncover<15>{
\psline[linecolor=blue](!-0.5 3 div -0.5)(1, 3) 
\psline[linecolor=blue](! 2 3 div -0.5)(! 11 6 div 3) 
}
\uncover<16->{
\psline[arrows=->, linecolor=red, linewidth=1.5pt](1, 0.5)(1.5, 2)
\psline[arrows=->, linecolor=red, linewidth=1.5pt](0.5, 1.5)(1.5, 2)
\psline[arrows=->, linecolor=red, linewidth=1.5pt](0,0)(0.5, 1.5)
\psline[arrows=->, linecolor=red, linewidth=1.5pt](0,0)(1, 0.5)
}
\uncover<13->{
\rput[lt](1.5, 1.95){$P+Q$}
}
\uncover<9->{%
\fcFullDot{1}{0.5}
\rput[br](1, 0.55){$P$}
}%
\uncover<10->{%
\fcFullDot{0.5}{1.5}
\rput[tl](0.5, 1.45){$Q$}
}%
\end{pspicture}
\column{0.7\textwidth}
\begin{definition}[$+, \cdot$ in $\mathbb R^2$]
Let $\alertNoH{2,6}{\fcv{u}=(x_1, y_1)}$ and $\alertNoH{3}{\fcv{v}=(x_2, y_2)}$ be pairs of numbers and let $c$ be a number. \alertNoH{4,5,7,8}{Define}

\hfil \hfil $
\begin{array}{rcl}
\alertNoH{2}{\fcv{u}} +\alertNoH{3}{\fcv{v}} \uncover<2->{&=& \alertNoH{2}{(\alertNoH{4}{x_1}, \alertNoH{5}{ y_1})} \alertNoH{4,5}{+} \alertNoH{3}{( \alertNoH{4}{x_2}, \alertNoH{5}{y_2} )}} \uncover<4->{= (\alertNoH{4}{ x_1+x_2},\alertNoH{5}{ y_1+ y_2})} \\
c\cdot \alertNoH{6}{\fcv{u}}\uncover<6->{ &=&\alertNoH{7,8}{c \cdot} \alertNoH{6}{( \alertNoH{7}{ x_1}, \alertNoH{8}{y_1} )}} \uncover<7->{ =(\alertNoH{7}{cx_1},\alertNoH{8}{ cy_1} ).}
\end{array}
$
\end{definition}
\end{columns}
\begin{itemize}
\item<8-> Fix a Cartesian coordinate system in the plane.
\item<9-> Let $P,Q$ be points with respective coordinates $\alertNoH{9}{(x_1, y_1)}$ and $\alertNoH{10}{(x_2, y_2)}$; \alertNoH{11}{let $O$ be the origin}.
\item<12-> $(x_1, y_1)+(x_2, y_2)= \!(x_1+x_2, y_1+y_2)$ \uncover<13->{corresponds to a new point which we denote by \alertNoH{12}{$P+Q$}.}
\item<14-> One can show: the line through $O$ and $P$ is parallel to the line through $Q$ and $P+Q$.
\item<15-> One can show: the line through $O$ and $Q$ is parallel to the line through $P$ and $P+Q$.
\item<16-> The points $O$, $P$, $P+Q$ and $Q$ form a parallelogram. 
\end{itemize}


\vskip 10cm
\end{frame}


%\item<13-> $c(x_1, y_1)=(cx_1, cy_1)$ corresponds to a new point which we denote by $cP$.
%\item $\cdot, +$ are defined only relative to the fixed Cartesian coordinate system. If we change the coordinate system we change $+, \cdot$.

\begin{frame}
\begin{columns}
\column{0.3\textwidth}
\psset{xunit=0.9cm, yunit=0.9cm}
\begin{pspicture}(-0.5,-0.5)(3.2,3.2)
\tiny
\fcAxesStandard{-0.5}{-0.5}{3}{3}
\fcLabels{3}{3}
\fcFullDot{1}{0.5}
\uncover<4->{
\psline[linecolor=blue](-0.5, -0.25) (3,1.5)
}
\rput[b](1, 0.55){$P$}
\rput[t](1, 0.45){$(x_1, y_1)$}
\uncover<2->{
\fcFullDot{2}{1}
\rput[t](2, 0.95){$(cx_1, cy_1)$}
}
\uncover<3->{\rput[b](2, 1.05){$cP$}}
\fcFullDot{0}{0}
\rput[tr](0, 0){$O~~$}
\end{pspicture}
\column{0.7\textwidth}
\begin{definition}[$+, \cdot$ in $\mathbb R^2$]
Let $\fcv{u}=(x_1, y_1)$ and $\fcv{v}=(x_2, y_2)$ be pairs of numbers and let $c$ be a number. Define

\hfil \hfil $
\begin{array}{rcl}
\fcv{u} +\fcv{v} &=& (x_1,  y_1) + (x_2, y_2 ) = ( x_1+x_2, y_1+ y_2) \\
c\cdot \fcv{u}&=& c \cdot (  x_1, y_1 ) =(cx_1, cy_1 ).
\end{array}
$
\end{definition}
\end{columns}
\begin{itemize}
\item Fix a Cartesian coordinate system in the plane.
\item Let $P,Q$ be points with respective coordinates $(x_1, y_1)$ and $(x_2, y_2)$; let $O$ be the origin.
\item<2-> $c(x_1, y_1)=(cx_1, cy_1)$ \uncover<3->{corresponds to a new point which we denote by $cP$.}
\item<4-> One can show $O$, $P$ and $cP$ lie on the same line.
\end{itemize}



\vskip 10cm
\end{frame}






\begin{frame}
\begin{columns}
\column{0.3\textwidth}
\psset{xunit=0.9cm, yunit=0.9cm}
\begin{pspicture}(-0.5,-0.5)(3.2,3.2)
\tiny
\fcAxesStandard{-0.5}{-0.5}{3}{3}
\fcLabels{3}{3}
\psline[arrows=->, linecolor=red, linewidth=1.5pt](1, 0.5)(1.5, 2)
\psline[arrows=->, linecolor=red, linewidth=1.5pt](0.5, 1.5)(1.5, 2)
\psline[arrows=->, linecolor=red, linewidth=1.5pt](0,0)(0.5, 1.5)
\psline[arrows=->, linecolor=red, linewidth=1.5pt](0,0)(1, 0.5)
\fcFullDot{1}{0.5}
\rput[b](1, 0.55){$P$}
\rput[t](1, 0.45){$(x_1, y_1)$}
\fcFullDot{0.5}{1.5}
\rput[t](0.5, 1.45){$Q$}
\rput[b](0.5, 1.55){$(x_2, y_2)$}
\rput[b](1.5, 2.05){$(x_1+x_2, y_1+y_2)$}
\rput[t](1.5, 1.95){$P+Q$}
\end{pspicture}
\column{0.7\textwidth}
\begin{definition}[$+, \cdot$ in $\mathbb R^2$]
Let $\fcv{u}=(x_1, y_1)$ and $\fcv{v}=(x_2, y_2)$ be pairs of numbers and let $c$ be a number. Define

\hfil \hfil $
\begin{array}{rcl}
\fcv{u} +\fcv{v} &=& (x_1,  y_1) + (x_2, y_2 ) = ( x_1+x_2, y_1+ y_2) \\
c\cdot \fcv{u}&=& c \cdot (  x_1, y_1 ) =(cx_1, cy_1 ).
\end{array}
$
\end{definition}
\end{columns}
\begin{itemize}
\item Fix a Cartesian coordinate system in the plane.
\item<2-> The correspondence between points in the plane and pairs of numbers depends on the choice of Cartesian coordinate system.
\item<3-> If we change the coordinate system we change $+, \cdot$.
\item<4-> The points in the plane, equipped with the operations $+,\cdot$ form a mathematical object which we call a vector space. 
\end{itemize}
\vskip 10cm
\end{frame}










\begin{frame}
\begin{columns}
\column{0.3\textwidth}
\psset{xunit=0.9cm, yunit=0.9cm}
\begin{pspicture}(-0.5,-0.5)(3,3)
\tiny
\fcAxesStandard{-0.5}{-0.5}{3}{3}
\fcLabels{3}{3}
\psline[arrows=->, linecolor=blue, linewidth=1.5pt](1, 0.5)(1.5, 2)
%\psline[arrows=->, linecolor=blue, linewidth=1.5pt](0.5, 1.5)(1.5, 2)
%\psline[arrows=->, linecolor=red, linewidth=1.5pt](0,0)(0.5, 1.5)
\psline[arrows=->, linecolor=red, linewidth=1.5pt](0,0)(1, 0.5)
\fcFullDot{1}{0.5}
\fcFullDot{1.5}{2}
\rput[lb](1, 0.55){$~~P$}
\rput[lt](1, 0.45){$~~(x_1, y_1)$}
%\fcFullDot{0.5}{1.5}
%\rput[lb](0.5, 1.55){$(a, b)$}
\rput[lb](1.5, 2.05){$(x_1+a, y_1+b)$}
%\rput[lt](1.5, 1.95){$~~P+(a,b)$}
\end{pspicture}
\column{0.7\textwidth}
\begin{definition}[$+, \cdot$ in $\mathbb R^2$]
Let $\fcv{u}=(x_1, y_1)$ and $\fcv{v}=(x_2, y_2)$ be pairs of numbers and let $c$ be a number. Define

\hfil \hfil $
\begin{array}{rcl}
\fcv{u} +\fcv{v} &=& (x_1,  y_1) + (x_2, y_2 ) = ( x_1+x_2, y_1+ y_2) \\
c\cdot \fcv{u}&=& c \cdot (  x_1, y_1 ) =(cx_1, cy_1 ).
\end{array}
$
\end{definition}
\end{columns}

\begin{definition}[Translation]
Let $P$ with coordinates $(x,y)$ be a point and let $(a,b)$ be a pair of numbers. The point $P=(x,y)+(a,b)=(x+a,y+b)$ is called the translation (shift) of $P$ $a$ units right and $b$ units up.
\end{definition}


\begin{itemize}
\item<2-> We allow shifts by negative units.
\item<3-> Translation down by $b$ units we define to be translation up by $-b$ units.
\item<4-> Translation left by $a$ units we define to be translation right by $-a $ units.
\end{itemize}


\vskip 10cm
\end{frame}










\begin{frame}
\begin{example}
\begin{columns}
\column{0.3\textwidth}
\psset{xunit=0.5cm, yunit=0.5cm}
\begin{pspicture}(-3,-3)(4,3)
\tiny
\fcBoundingBox{-3.2}{-3.2}{4.2}{3.2}
\psgrid[subgriddiv=0,griddots=10,gridlabels=0pt](-3,-3)(3,3)
\fcAxesStandardNoFrame{-3}{-3}{3}{3}
\fcLabels{3}{3}
\uncover<2->{
\fcFullDot{-3}{2}
}
\uncover<7->{
\fcFullDot{1}{0}
}
\uncover<8->{
\psline[arrows=->, linecolor=red](-3, 2)(1,0)
}
\uncover<9->{
\psline[arrows=->, linecolor=red](0,0 )(-3, 2)
\psline[arrows=->, linecolor=red](0,0 )(4, -2)
\psline[arrows=->, linecolor=red](4,-2)(1,0)
}
\end{pspicture} 
\column{0.7\textwidth}
Translate  $\alertNoH{2}{(-3,2)}$  $4$ units left and $2$ units down.
\uncover<3->{
\[
(-3,2) +(\fcAnswer{4}{4}, \fcAnswer{4}{-2}) \uncover<5->{=(-3+4, 2+(-2))}\uncover<6->{= (\fcAnswerUncover{3}{7}{ 1}, \fcAnswerUncover{3}{7}{0})}.
\]
}
\end{columns}
\end{example}
\uncover<10->{
\begin{example}
Translate the point in the indicated way.

\begin{tabular}{ccc}
Point & Translation & result\\
$(2,3)$& $2$ units left $3$ units up& \textbf{?}\\
$(2,1)$& $2$ units left $-2$ units down& \textbf{?}\\
$(-2,1)$& $-1$ units right $2$ units down& \textbf{?}\\
$(-2,3)$& $-1$ units left $2$ units up& \textbf{?}\\
\end{tabular}
\end{example}
}

\end{frame}
\subsection{Segments, Midpoints}
\begin{frame}
\begin{observation}The segment connecting $P$ and $Q$ consists of all points of the form 
\[
t P+(1-t)Q,
\]
where $t$ runs over all numbers in the interval $[0,1]$.
\end{observation}
\begin{itemize}
\item Let $P$  have coordinates $(x_1, y_1)$ and $Q$ have coordinates $(x_2, y_2)$.

\item Then the segment between $P$ and $Q$ consists of the points with coordinates
\[
t(x_1, y_1)+(1-t)(x_2, y_2).
\]
\end{itemize}
\begin{observation}
The midpoint of the segment between $P$ and $Q$ is the point with $t=\frac{1}{2}$.
\[
\text{Midpoint}(P,Q)= \frac{1}{2}P+\frac{1}{2}Q=\left( \frac{x_1+x_2}{2}, \frac{y_1+y_2}{2} \right).
\]
\end{observation}
\end{frame}
\begin{frame}
\begin{example}
Let $P$ have coordinates $(x_1, y_1)$ and $Q$ have coordinates $(x_2, y_2)$. Let the midpoint of $P$ and $Q$ be $R$. Write the formula for the distance between $P$ and $Q$, and for the distance between $Q$ and $R$. Show the two formulas are equal.
\end{example}


\end{frame}
\begin{frame}
\begin{example}
Find the midpoint of the indicated pairs of points.
\begin{tabular}{ccc}
$P$ & $Q$ & midpoint\\
$(1,2)$& $(-1,-2)$&\textbf{?}\\
$(1,2)$& $(1,-2)$&\textbf{?}\\
$(-1,2)$& $(1,-2)$&\textbf{?}\\
$(-2,-3)$& $(3,2)$&\textbf{?}
\end{tabular}
\end{example}
\end{frame}
}% end lecture

\lect{\semester}{Lecture  2}{2}{
%DesiredLectureName: Graphing-Equations-Circle-Equation
\fcLicense
\section{Graph of an equation}
\begin{frame}
\begin{itemize}
\item Let $F(x,y)$ and $G(x,y)$ be arbitrary functions in the variables $x,y$. 
\item<2-> An equation in $x,y$ is an expression of the form 
\[
F(x,y)=G(x,y).
\]
\item<3-> An ordered pair of numbers $(a,b)$ is a solution of the equation if the number $F(a,b)$ equals the number $G(a,b)$.
\end{itemize}
\uncover<4->{
\begin{definition}
The graph of the equation $F(x,y)=G(x,y)$ is the set of all solutions $(a,b)$ of the equation. 
\end{definition}
}
\begin{itemize}
\item<5-> Two equations are equivalent if they have the same graphs (set of solutions).
\item<6-> If we set $H(x,y)=F(x,y)-G(x,y)$, we transform an arbitrary equation to an equivalent equation of the form:
\[
H(x,y)=0.
\]
\end{itemize}
\end{frame}
\begin{frame}
\begin{example}
Determine which  of the following points
\begin{itemize}
\item $\left(-3, -5\right)$
\item $(3,5)$
\end{itemize}
is a solution to the equation 

\hfil \hfil $\alertNoH{6,10}{7x-4y=-1}.$ 
\begin{itemize}
\item<2-> For $\alertNoH{3}{x=-3}$, $\alertNoH{4}{y=-5}$, we have: \uncover<3->{\[ \alertNoH{6}{7\alertNoH{3}{ x}-4\alertNoH{4}{y}} = 7(\alertNoH{3}{-3}) -4(\alertNoH{4}{-5}) \uncover<5->{=-21+20 \alertNoH{6}{=-1},}\]}
\uncover<6->{so $(-3,-5)$ is a solution.}
\item<7-> For $\alertNoH{8}{x=3}$, $\alertNoH{9}{y=5}$, we have: \uncover<8->{
\[
\left(7\alertNoH{8}{ x}-4\alertNoH{9}{y}\right) = 7(\alertNoH{8}{3}) -4(\alertNoH{9}{5}) \uncover<10->{=21-20=\alertNoH{10}{ 1\neq -1},}\]
}
\uncover<10->{so $(3,5)$ is \textbf{not} a solution.}
\end{itemize}

\end{example}

\end{frame}


\begin{frame}

\begin{example}
Determine which of the points
$\left(\frac{ \sqrt{3}}{2}, \frac{1}{2}\right)$,
$\left(\frac{ \sqrt{3}}{2}, \frac{\sqrt{2}}{4}\right)$
is a solution to the equation 

\hfil \hfil $\alertNoH{8,15}{x^2+2y^2=1}.$ 
\begin{itemize}
\item<2-> For $\alertNoH{3}{x=\frac{\sqrt{3}}{2}}$, $\alertNoH{4}{y=\frac{1 }{2}}$, we have: 

\uncover<3->{
\hfil \hfil$\displaystyle 
\alertNoH{8}{\alertNoH{3}{ x}^2+2\alertNoH{4}{y}^2} = \alertNoH{5}{\left(\alertNoH{3}{\frac{\sqrt{3}}{2}}\right)^2 } +2\alertNoH{6}{\left(\alertNoH{4}{\frac{1}{2}}\right)^2} \uncover<5->{ =\alertNoH{5}{\frac{3}{4}}+2\cdot \alertNoH{6}{\frac{1}{ 4}}} \uncover<7->{ = \frac{5}{4} \uncover<8->{\alertNoH{8}{\neq 1},}}
$
}

\uncover<8->{
\noindent so $\left(\frac{\sqrt{3}}{2}, \frac{1}{2}\right)$ is \textbf{not} a solution.
}
\item<9-> For $\alertNoH{10}{x=\frac{\sqrt{3}}{2}}$, $\alertNoH{11}{y=\frac{\sqrt{2}}{4}}$, we have: 

\uncover<10->{
\hfil \hfil $\displaystyle
\alertNoH{15}{\alertNoH{10}{ x^2}+2\alertNoH{11}{y} } = \alertNoH{12}{\left(\alertNoH{10}{\frac{\sqrt{3}}{2}}\right)^2} +2 \alertNoH{13}{\left(\alertNoH{11}{\frac{\sqrt{2}}{4}}\right)^2} \uncover<12->{=\alertNoH{12}{\frac{3}{4}}+ \fcCancel{14}{2}\cdot \alertNoH{13}{\frac{\fcCancel{14}{2}}{ 4^{\fcCancel{14}{2}}}} \uncover<14->{\alertNoH{15}{= 1},}}
$
}

\uncover<15->{\noindent so $\left(\frac{\sqrt{3}}{2}, \frac{\sqrt{2}}{4}\right)$ is a solution.}
\end{itemize}
\end{example}
\end{frame}

\begin{frame}
\begin{question}
Can we plot the graph of an arbitrary equation, $F(x,y)=G(x,y)$?
\end{question}
\begin{itemize}
\item<2-> For sufficiently well behaved equations the answer is yes.
\item<3-> We illustrate one computer algorithm for doing this.
\item<4-> The theory behind plotting arbitrary equations, even when they are well behaved, is well beyond the scope of the present course. 
\item<5-> In particular, while computer algorithms plot graphs of well-behaved equations relatively well, it is not clear why those algorithms work.
\item<6-> When, using algebra, we can express one variable in terms of the other, it is easy to produce the graph of the equation.
\end{itemize}
\end{frame}

\begin{frame}
\begin{question}
Can we plot the graph of an equation of the form $y=f(x)$ or $x=h(y)$ (for continuous $h,f$)? Yes.
\end{question}
\begin{columns}
\column{0.3\textwidth}
\psset{xunit=0.2cm, yunit=0.2cm}
\begin{pspicture}(-10,-10)(10.5,10.5)%
\tiny%
\newcommand{\theFun}{-1 3 div x mul 1 add\space}%
\newcommand{\xMax}{9\space}%
\newcommand{\xMin}{-9\space}%
\fcAxesStandard{-10}{-10}{10}{10}%
\newcommand{\numPoints}{4}%
\only<handout:1|13->{\renewcommand{\numPoints}{10}}%
\only<handout:0|14->{\renewcommand{\numPoints}{19}}%
\only<handout:2|15->{\renewcommand{\numPoints}{37}}%
\only<handout:3,4|16->{\renewcommand{\theFun}{1 9 div x x mul mul 0.5 x mul sub 4 sub\space }}
\only<handout:3|16->{\renewcommand{\numPoints}{4}}%
\only<handout:0|17->{\renewcommand{\numPoints}{10}}%
\only<handout:4|18->{\renewcommand{\numPoints}{37}}%
\only<handout:5,6|19->{\renewcommand{\theFun}{-1 9 div x x mul mul 0.5 x mul sub 6 add\space }}
\only<handout:5|19->{\renewcommand{\numPoints}{4}}%
\only<handout:0|20->{\renewcommand{\numPoints}{10}}%
\only<handout:6|21->{\renewcommand{\numPoints}{37}}%
\rput[l](-10, -5){\uncover<handout:1|7-12>{$\alertNoH{5}{n=3}$,} \uncover<handout:1|8-12>{$\alertNoH{6}{\Delta = \frac{9-(-9)}{3}=6}$,} \uncover<handout:1|9-12>{$\alertNoH{7}{x_j=-9+6j}$}}%
\uncover<5->{%
\psline(9, -0.4) (9, 0.4)%
\rput[t](9, -0.5){$9$}%
}%
\uncover<4->{%
\psline(-9, -0.4)(-9, 0.4)%
\rput[t](-9, -0.5){$-9$}%
}%
\pstVerb{40 dict begin %
/xMax \xMax def %
/xMin \xMin def %
/numPoints \numPoints\space def %
/Delta xMax xMin sub numPoints 1 sub div def %
}%
\multido{\na=0+1}{\numPoints}{%
\pstVerb{%
/counter \na \space def
/theX counter Delta mul xMin add def
/theXnext theX Delta add def
theXnext xMax gt{/theXnext theX def}if
/theY 1 dict begin /x theX def \theFun end def
/theYnext 1 dict begin /x theXnext def \theFun end def
}%
\uncover<handout:1,3,5|10-12>{%
\psline(! theX -0.4)(! theX 0.4)%
\rput[b](! theX 0.5){$x_{\na}$}%
}%
\uncover<handout:1,2|11-15>{\fcFullDot{theX}{theY}}%
\uncover<12->{\psline[linecolor=\fcColorGraph](! theX theY)(! theXnext theYnext)}%
}%
\pstVerb{end}%
\end{pspicture}
\uncover<3->{Demonstration of the algorithm for \only<handout:1,2|-15>{$\alertNoH{3}{y=-\frac{1}{3}x+1}$}
\only<handout:3,4|16-18>{$\alertNoH{16}{y=\frac{1}{9}x^2-\frac{1}{2}x+4}$}
\only<handout:5,6|19-21>{$\alertNoH{19}{y=-\frac{1}{9}x^2-\frac{1}{2}x+6}$}
} 

\uncover<4->{from $\alertNoH{4}{x=-9}$ to $\alertNoH{5}{x=9}$.} 
\column{0.7\textwidth}
\begin{itemize}
\item<2-> Suppose $y=f(x)$. 
\item<4-> To plot the graph from $\alertNoH{4}{x=a}$ to $\alertNoH{5}{x=b}$, select $n+1$ points $x_0, x_1, \dots, x_n$ on $[a,b]$. 
\begin{itemize}
\item<6-> Usually we choose the points to be evenly spaced with $x_0=a$ and $x_n=b$.
\item<7-> $n+1$ points split $[a,b]$ into $n$ intervals. 
\item<8-> Each interval has length $\alertNoH{6}{\Delta=\frac{b-a}{n}}$.
\item<9-> The formula for the $j^{th}$ point is then $\alertNoH{7}{x_j=a+\Delta\cdot j}$.
\end{itemize}
\item<11-> The points $(x_0, f(x_0)), \dots, (x_n, f(x_n))$ all lie on the graph. 
\item<12-> Connect them with straight lines. 
\item<13-> Repeat for increasing number of segments $n$.
\end{itemize}
\end{columns}
\end{frame}
\youWillNotBeTested
\begin{frame}
\begin{emptyTheorem}[Elementary Computer algorithm for sketching graphs]
\alertNoH{1}{\alertNoH{14}{Let $H$-continuous}; is there simple algorithm to sketch $\alertNoH{5}{H(x,y)=0}$?} \uncover<2->{\alertNoH{2}{Yes.}}
\end{emptyTheorem}
\begin{columns}
\column{0.3\textwidth}
\only<handout:1-6|-27>{%
\psset{xunit=0.8cm, yunit=0.8cm}
\begin{pspicture}(-1,-1)(1,1)
\tiny
\fcBoundingBox{-1.8}{-1.8}{3}{3}
\newcommand{\theFun}{x x mul 2 y y mul mul add 1 sub}
\fcAxesStandard{-1.6}{-1.6}{1.6}{1.6}
\uncover<7-18>{\fcImplicitIId[linestyle=dashed, dashes={[1 1] 0}, showGridImplicitIId=true]{-1.6}{-1.6}{2}{2}{1.6}{1.6}{-1} }
\uncover<handout:1|19>{\fcImplicitIId[linestyle=solid, linecolor=red, showGridImplicitIId=true]{-1.6}{-1.6}{2}{2}{1.6}{1.6}{\theFun} }
\uncover<handout:2|20>{\fcImplicitIId[linestyle=solid, linecolor=red, showGridImplicitIId=true]{-1.6}{-1.6}{4}{4}{0.8}{0.8}{\theFun} }
\uncover<handout:3|21>{\fcImplicitIId[linestyle=solid, linecolor=red, showGridImplicitIId=true]{-1.6}{-1.6}{8}{8}{0.4}{0.4}{\theFun} }
\uncover<handout:4|22>{\fcImplicitIId[linestyle=solid, linecolor=red, showGridImplicitIId=false]{-1.6}{-1.6}{16}{16}{0.2}{0.2}{\theFun} }
\uncover<handout:0|23>{\fcImplicitIId[linestyle=solid, linecolor=red, showGridImplicitIId=false]{-1.6}{-1.6}{32}{32}{0.1}{0.1}{\theFun} }
\uncover<handout:0|24>{\fcImplicitIId[linestyle=solid, linecolor=red, showGridImplicitIId=false]{-1.6}{-1.6}{64}{64}{0.05}{0.05}{\theFun} }
\uncover<handout:0|25>{\fcImplicitIId[linestyle=solid, linecolor=red, showGridImplicitIId=false]{-1.6}{-1.6}{128}{128}{0.025}{0.025}{\theFun} }
\uncover<handout:5|26>{\fcImplicitIId[linestyle=solid, linecolor=red, showGridImplicitIId=false]{-1.6}{-1.6}{400}{400}{3.2 400 div }{3.2 400 div}{\theFun} }
\uncover<handout:6|27>{\fcImplicitIId[linestyle=solid, linecolor=red, showGridImplicitIId=false]{-1.6}{-1.6}{1000}{1000}{3.2 1000 div }{3.2 1000 div}{\theFun} }
\uncover<handout:0|6>{
\psline[linestyle=dotted](-1.6,-1.6)(1.6,-1.6)(1.6,1.6)(-1.6,1.6)(-1.6,-1.6)
}
\uncover<9-18>{
\fcFullDot{1.6}{1.6}
\fcFullDot{0}{0}
\rput[lt](1.6,1.5){$\begin{array}{l}\alertNoH{9}{P(x_P, y_P)}\\\alertNoH{9}{=(1.6, 1.6)}\end{array} $}
\rput[lt](0,-0.1){$\begin{array}{l}\alertNoH{10}{ Q(x_Q, y_Q)}\\ \alertNoH{10}{=(0,0)}\end{array}$}
}
\uncover<11-18>{\rput[lb](1.6,1.65){$\begin{array}{l}\alertNoH{11,13}{H(1.6,1.6)}\\ \alertNoH{11}{=6.68}\uncover<13->{\alertNoH{13}{ >0}}\end{array} $}}

\uncover<12-18>{\rput[br](0,0){$\begin{array}{l}\alertNoH{12,13}{H(0,0)} \\\alertNoH{12}{=-1}\uncover<13->{\alertNoH{13}{<0}}\end{array}$}}
\uncover<handout:0|14>{\psline[linecolor=red, linewidth=1.5pt](0,0)(1.6,1.6)}
\uncover<16-19>{\fcFullDot{0.8}{0.8}}
\uncover<17-19>{\fcFullDot{0.8}{0}}
\uncover<18>{\psline[linecolor=red](0.8, 0)(0.8, 0.8)}
\uncover<handout:1|19>{
\fcFullDot{0}{-0.8}
\fcFullDot{-0.8}{-0.8}
\fcFullDot{-0.8}{0}
\fcFullDot{0}{0.8}
}
\end{pspicture}
\uncover<3-27>{
We illustrate the algorithm for:
$
\begin{array}{@{}r@{}c@{}l@{}l@{}|l}
\uncover<3->{x^{2}+2y^2&\alertNoH{0}{=}&\alertNoH{4}{1}}\\
\uncover<4->{\alertNoH{5}{x^{2}+2y^2\alertNoH{4}{-1}}&\alertNoH{0}{=}&0}\\
\uncover<5->{\text{Set }\alertNoH{5,11,12}{H(x,y)}&\alertNoH{5,11,12}{=}&\alertNoH{5,11,12}{x^{2}+2y^{2}-1}}
\end{array}
$
}}%only

\only<handout:7|28-39>{
\psset{xunit=0.2cm, yunit=0.2cm}
\begin{pspicture}(-1,-1)(1,1)
\tiny
\fcBoundingBox{-7.2}{-7.2}{12}{12}
\newcommand{\theFun}{ y y mul y y mul 3 sub mul x x mul x x mul 5 sub mul sub}
\fcAxesStandard{-6}{-6}{6}{6}
\uncover<handout:0|28,29>{
\fcImplicitIId[linestyle=solid, linecolor=red, showGridImplicitIId=true]{-6}{-6}{4}{4}{3}{3}{-1} 
}
\uncover<handout:0|30>{
\fcImplicitIId[linestyle=solid, linecolor=red, showGridImplicitIId=true]{-6}{-6}{4}{4}{3}{3}{\theFun} 
}
\uncover<handout:0|31>{
\fcImplicitIId[linestyle=solid, linecolor=red, showGridImplicitIId=true]{-6}{-6}{6}{6}{2}{2}{\theFun} 
}
\uncover<handout:0|32>{
\fcImplicitIId[linestyle=solid, linecolor=red, showGridImplicitIId=true]{-6}{-6}{8}{8}{1.5}{1.5}{\theFun} 
}
\uncover<handout:0|33>{
\fcImplicitIId[linestyle=solid, linecolor=red, showGridImplicitIId=true]{-6}{-6}{12}{12}{1}{1}{\theFun} 
}
\uncover<handout:0|34>{
\fcImplicitIId[linestyle=solid, linecolor=red, showGridImplicitIId=true]{-6}{-6}{24}{24}{0.5}{0.5}{\theFun} 
}
\uncover<handout:0|35>{
\fcImplicitIId[linestyle=solid, linecolor=red, showGridImplicitIId=false]{-6}{-6}{48}{48}{0.25}{0.25}{\theFun} 
}
\uncover<handout:0|36>{
\fcImplicitIId[linestyle=solid, linecolor=red, showGridImplicitIId=false]{-6}{-6}{100}{100}{0.12}{0.12}{\theFun} 
}
\uncover<handout:0|37>{
\fcImplicitIId[linestyle=solid, linecolor=red, showGridImplicitIId=false]{-6}{-6}{200}{200}{0.06}{0.06}{\theFun} 
}
\uncover<handout:0|38>{
\fcImplicitIId[linestyle=solid, linecolor=red, showGridImplicitIId=false]{-6}{-6}{400}{400}{0.03}{0.03}{\theFun} 
}
\uncover<handout:7|39>{
\fcImplicitIId[linestyle=solid, linecolor=red, showGridImplicitIId=false]{-6}{-6}{800}{800}{0.015}{0.015}{\theFun} 
}
\end{pspicture}
Illustrate the algorithm for:
$\begin{array}{@{}r@{}c@{}l@{}l@{}|l}
\alertNoH{28}{y^{2}(y^{2}-3)}&\alertNoH{28}{=}&\alertNoH{28}{x^2(x^2-5)}\\
\alertNoH{29}{H(x,y)}&\alertNoH{29}{=}&\alertNoH{29}{y^{2}(y^{2}-3)}\\
&&\alertNoH{29}{-x^2(x^2-5)}
\end{array}
$
}
\column{0.7\textwidth}
\begin{itemize}
\item<6-> Elementary algorithm: fix  large rectangle. 
\item<7-> Split the grid in triangular mesh. One strategy to do that is shown.
\item<8-> \alertNoH{19}{For each triangle:}
\begin{itemize}
\item<9-> Fix two corners $\alertNoH{9}{P(x_P, y_P)}$ and $\alertNoH{10}{Q(x_Q, y_Q)}$. 
\item<11-> If $\alertNoH{11}{H(x_P, y_P)}$ and $\alertNoH{12}{H(x_Q, y_Q)}$ \alertNoH{13}{have different sign} \alertNoH{14}{then $H$ must become zero somewhere on the segment between $P$ and $Q$}. 
\item<15-> Select a point between $P$ and $Q$ and ``guess'' that $H$ is zero there. 
\begin{itemize}
\item<16-> In our implementation, we select the midpoint (i.e., $\frac{1}{2}P+\frac{1}{2}Q$).
\item<17-> Connect the selected pts. for each triangle.
\end{itemize}
\item<20-> Repeat for ever finer grid.
\end{itemize}

\end{itemize}
\end{columns}

\end{frame}
\begin{frame}
\begin{definition}
Points on the graph of $ F(x,y)=G(x,y)$ for which 
\begin{itemize}
\item $x=0$ are called $y$-intercepts (those lie on the $y$ axis)
\item<2-> $y=0$ are called $x$-intercept (those lie on the $x$ axis). 
\end{itemize}
\end{definition}
\begin{columns}
\column{0.2\textwidth}

\begin{pspicture}(-1,-1)(1,1)
\tiny
\fcAxesStandard{-1.2}{-1}{1.4}{1.4}
\fcLabels{1.4}{1.4}
\parametricplot[linecolor=\fcColorGraph]{0}{360}{t sin t 30 add cos mul t cos t 30 sub cos mul 1.5 mul 0.5 sub}
\fcFullDot{0}{30 cos 1.5 mul 0.5 sub}
\fcFullDot{0}{60 cos 30 cos mul 1.5 mul 0.5 sub}
\uncover<2->{\fcFullDot{2 -0.99702 180 3.141592654 div mul  mul  dup sin exch 30 add cos mul}{0}
\fcFullDot{2 -0.31198 180 3.141592654 div mul  mul  dup sin exch 30 add cos mul}{0}}
\end{pspicture}
\column{0.8\textwidth}
\begin{itemize}
\item<3-> To find the $x$ intercepts set $y=0$ and solve for $x$.
\item<4-> To find the $y$ intercepts set $x=0$ and solve for $y$.
\end{itemize}
\end{columns}
\end{frame}
\begin{frame}
\begin{example}
\begin{columns}
\column{0.25\textwidth}
\begin{pspicture}(-2,-2)
\fcAxesStandard{-1.3}{-1.3}{1.3}{1.3}
\psplot[linecolor=\fcColorGraph]{-1.3}{1.3}{x x x mul mul x sub}
\uncover<2->{
\fcFullDot{-1}{0}
\fcFullDot{1}{0}
\fcFullDot{0}{0}
}
\end{pspicture}
\column{0.75\textwidth}
Find the $x$ and $y$ intercepts of the graph of the equation 
$
y=x^3-x.
$

\uncover<2->{To find the $y$ intercept, set $x=0$ to get $y=0$. To find the $x$ intercepts, set $y=0$ and solve 
$\begin{array}{rcl}
x^3-x&=&0\\
x(x^2-1)&=&0\\
x(x-1)(x+1)&=&0\\
x=0 \text{ or } x=1 && \text{ or }x=-1.
\end{array}
$
The $x$-intercepts are: $(-1,0), (0,0), (1,0)$, the $y$-intercept is $(0,0)$.
}
\end{columns}

\end{example}
\end{frame}
\begin{frame}
\begin{example}
\begin{columns}
\column{0.3\textwidth}
\psset{xunit=0.7cm, yunit=0.7cm}
\begin{pspicture}(-2,-2)
\fcAxesStandard{-4}{-2.5}{1}{2.5}
\parametricplot[linecolor=\fcColorGraph]{0}{360}{t cos 2 mul -1.5 add t sin 2 mul }
\uncover<2->{
\fcFullDot{0}{7 4 div sqrt}
\fcFullDot{0}{7 4 div sqrt -1  mul}
\fcFullDot{0.5}{0}
\fcFullDot{-7 2 div}{0}
}
\end{pspicture}
\uncover<2->{Answer: the $y$-intercepts are: $\left(0,\sqrt{\frac{7}{4}} \right)$, $\left(0,-\sqrt{\frac{7}{4}} \right)$; the $x$ intercepts are:  $\left(\frac{1}{2},0)\right)$ and $(-\frac{7}{2},0)$}

\column{0.7\textwidth}
Find the $x$ and $y$ intercepts of the graph of the equation 
$
x^{2}+3x+ y^{2}= \frac{7}{4}.
$

\uncover<2->{To find the $y$ intercept, set $x=0$ and solve:

$
\displaystyle y^{2}=\displaystyle \frac{7}{4}\Rightarrow
\displaystyle y = \displaystyle \pm \sqrt{\frac{7}{4}}
$



To find the $x$ intercepts, set $y=0$ and solve:

$\begin{array}{rcl}
\displaystyle x^{2}+3x&=&\displaystyle \frac{7}{4}\\
\displaystyle x^2+3x-\frac{7}{4}&=&0\\
\displaystyle 4x^2+12x-7&=&0\\
\displaystyle (2x-1)(2x+7)&=&0\\
2x-1=0 &\text{or}& 2x+7=0\\
\displaystyle x=\frac{1}{2}&\text{or}&\displaystyle x=-\frac{7}{2}
\end{array}
$
}
\end{columns}

\end{example}
\end{frame}
\begin{frame}
\begin{itemize}
\item A graph is symmetric with respect to the $x$ axis for each $(x,y)$ lying on the graph $(x,-y)$ also lies on the graph.
\item A graph is symmetric with respect to the $y$ axis for each $(x,y)$ lying on the graph $(-x,y)$ also lies on the graph.
\item A graph is symmetric with respect to the origin if for each $(x,y)$ lying on the graph $(-x,-y)$ also lies on the graph.


\end{itemize}
\end{frame}
\begin{frame}
\begin{columns}
\column{0.3\textwidth}
\begin{pspicture}(-1, -1)(2.5,2.5)%
\tiny
\fcAxesStandard{-0.5}{-0.5}{2.5}{2.5}%
\parametricplot[linecolor=\fcColorGraph]{0}{360}{t cos 1.2 add  t sin 1.4 add}%
\fcFullDot{1.2}{1.4}%
\rput[l](1.2, 1.4){$~(h,k)$}%
\psline[linecolor=blue](1.2, 1.4)(! 1.2 2 sqrt 2 div sub 2 sqrt -2 div  1.4 add)
\fcFullDot{1.2 2 sqrt 2 div sub}{2 sqrt -2 div  1.4 add}
\rput[t](0.8, 0.8){$r$}
\end{pspicture}
\column{0.7\textwidth}
\begin{observation}
The graph of the equation 
\[
(x-h)^2+(y-k)^2=r^2
\]
is a circle with radius $r$ and center $(h,k)$.
\end{observation}
\end{columns}
\end{frame}
\begin{frame}
\begin{example}
Show that the graph of the given equation is a circle. Find the center and radius of the circle.

\begin{itemize}
\item $x^2+2x+y^2=1$.
\item $x^2+x+2y^2+y=1+y^2$.
\item $x^2=3x-y^2-2y$.
\item $3x^2+y=-3y^2$.
\item $2x^2+y=\frac{1}{2}x-2y^2$.
\end{itemize}
\end{example}
\end{frame}
\begin{frame}

\begin{example}
Find an equation of a circle with center  $(2,3)$ and passing through the point $(-1,1)$.
\end{example}

\end{frame}

}% end lecture

\lect{\semester}{Lecture  3}{3}{
%DesiredLectureName: Lines
\fcLicense
\begin{frame}
\frametitle{$\mathbb R^2$}
\begin{definition}
The set of ordered pairs of real numbers is denoted by $\mathbb R^2$.
\end{definition}
\end{frame}
\begin{frame}
\begin{definition}
\begin{itemize}
\item An equation of the form $ax+by+c=0$, where $a,b,c$ are constants such that $a$ and $b$ are not simultaneously zero, is called a linear equation
\item<2->\alertNoH{0}{A set of pairs of numbers $(x,y)$ is called a line in $\mathbb R^2$ if it is the set of solutions to some linear equation.}
\item<3->  \alertNoH{6}{\only<handout:0|7->{\color{gray!30}} A set of points in the plane will be called a line if it is the graph of some linear equation.}
\end{itemize}
\end{definition}

\begin{itemize}
\item<4-> To introduce the Cartesian coordinate system we used informal, intuitive notions of point, lines and the plane.
\item<5-> We could (and often do in more advanced subjects) remove this informality by \emph{defining} $\mathbb R^2$ to be the Euclidean plane.
\end{itemize}
\end{frame}
\begin{frame}

\begin{definition}[non-vertical line, slope-intercept form]
A line that is the graph of an equation of the form 

\hfil \hfil $
y=mx+b
$

is called a \emph{non-vertical} line.
\begin{itemize}
\item<2-> The equation above is called the slope-intercept form of the (non-vertical) line. 
\item<3-> The number $m$ is called the slope of the line.
\item<4-> The number $b$ is the $y$ intercept of the line.
\end{itemize}
\end{definition}
\begin{center}
\psset{xunit=0.9cm, yunit=0.9cm}
\begin{pspicture}(-1,-1)%
\tiny%
\fcAxesStandard{-0.5}{-0.5}{3}{2.8}%
\psline[linecolor=\fcColorGraph](-0.5, 0.5)(3,2.5)%
\fcLabels{3}{2.8}%
\end{pspicture}

\end{center}
\end{frame}
\begin{frame}\frametitle{Geometric Interpretation of Slope}

\begin{definition}[non-vertical line, slope-intercept form]
$
y=mx+b
$, $m$ - is called slope, $b$ is called  $y$-intercept.

\end{definition}
\vskip -0.15cm
\begin{columns}
\column{0.25\textwidth}
\psset{xunit=0.9cm, yunit=0.9cm}
\begin{pspicture}(-0.5,-0.5)(3.3,3)%
\tiny%
\fcAxesStandard{-0.5}{-0.5}{3}{2.8}%
\psline[linecolor=\fcColorGraph](-0.5, 0.5)(3,2.5)%
\fcLabels{3}{2.8}%
\uncover<3->{%
\fcPerpendicular{[2.65 2.3]}{[1 0.9] [-0.5 0.9]}{0.1}%
\psline(0.2, 0.9)(2.65, 0.9)%
}%
\uncover<2->{%
\fcFullDot{0.2}{0.9}%
\fcFullDot{2.65}{2.3}%
\rput[r](2.65, 2.4){$(x_2,y_2)$}%
\rput[tl](0.1,0.8 ){$(x_1,y_1)$}%
}%
\uncover<5->{\rput[l](2.7, 1.6 ){rise}}%
\uncover<3->{\rput[t](1.4, 0.8 ){run}}%
\end{pspicture}
\uncover<6->{
\psset{xunit=0.9cm, yunit=0.9cm}
\begin{pspicture}(-0.5,-0.5)(3.3,3)%
\tiny%
\fcAxesStandard{-0.5}{-0.5}{3}{2.8}%
\psline[linecolor=\fcColorGraph](-0.5, 2.47)(!3  0.75)%
\fcLabels{3}{2.8}%
\fcPerpendicular{[2.5 1]}{  [2 2.05] [-0.5 2.05]}{0.1}%
\psline(0.4, 2.05)(2.5, 2.05)%
\fcFullDot{0.4}{2.05}%
\fcFullDot{2.5}{1}%
\rput[t](2.5, 0.9){$(x_2,y_2)$}%
\rput[bl](0.1,2.1 ){$(x_1,y_1)$}%
\rput[r](2.45, 1.4 ){rise<0}%
\rput[b](1.4, 2.1 ){run}%
\rput[lt](0.2,-0.1){negative rise allowed}%
\end{pspicture}
}%
\column{0.75\textwidth}
\begin{itemize}
\item<2-> Fix pts. $(x_1, y_1)$, $(x_{2}, y_2)$ on the line with \alertNoH{4}{$x_2>x_1$}.
\item<3-> \alertNoH{15}{Call $x_2-x_1$ the run} of the line between the points. \uncover<4->{\alertNoH{4}{The run is assumed positive.}}
\item<5-> \alertNoH{14}{Call $y_{2}-y_1 $ the rise} of the line between the two points. \uncover<6->{Negative rise is allowed.}

$
\begin{array}{rrcl}
\uncover<7->{&y_2&=&m x_2+b\\
\cline{1-1}&y_1&=&mx_1+b\\\hline}
\uncover<8->{&y_2-y_1&=&\alertNoH{10}{m}x_2+\fcCancel{9}{b}-\alertNoH{10}{m}x_1-\fcCancel{9}{b}}\\
\uncover<9->{& \alertNoH{12}{y_2-y_1}& =&\alertNoH{10, 11}{m}( \alertNoH{13}{ x_2-x_1})}\\
\uncover<11->{&\alertNoH{11}{m}&=&\displaystyle \frac{\alertNoH{12, 14}{y_2-y_1}}{\alertNoH{13,15}{x_2-x_1}}}\\
\uncover<14->{&m&=&\displaystyle \frac{\alertNoH{14}{ \text{rise} }}{\alertNoH{15}{ \text{run}}}}
\end{array}
$

\end{itemize}

\end{columns}
\end{frame}
\begin{frame}

\begin{definition}[Vertical line]
A line of the form $x=a$ is called a vertical line.
\end{definition}
\begin{columns}
\column{0.25\textwidth}
\begin{pspicture}(-0.6,-0.6)(2.1,2.1)
\tiny
\fcAxesStandard{-0.5}{-0.5}{2}{2}
\psline[linecolor=\fcColorGraph](1, -0.5)(1,2)
\rput[lt](1.05,-0.05){$a$}
\rput[lt](1.05,1){$(a,y)$}
\fcFullDot{1}{1}
\end{pspicture}
\column{0.75\textwidth}

\begin{itemize}
\item<2-> $y$ does not participate directly in the equation $x=a$.
\item<3-> Therefore the equation cannot be rewritten in slope-intercept form ($y=\textbf{?}x+\textbf{?}$).
\item<4-> Consequently the notion of a slope is not undefined for vertical lines. 
\end{itemize}
\end{columns}
\end{frame}
\begin{frame}
\frametitle{Plotting Lines from line equation}
To plot a line from its equation $ax+by=c$ do the following.
\begin{itemize}
\item<2-> If $b=0 $, the line is vertical through $x=\frac{c}{a}$. \uncover<3->{Suppose $b\neq 0$.}
\begin{itemize}
\item<4-> Plug in arbitrary number for $x$ and find $y$ from $y=\frac{c-ax}{b}$.
\item<5-> Use same procedure to find a second point on the line.
\item<6-> If $a\neq 0$: can also plug in values for $y$ to find $x$.
\item<7-> Draw a line between the two dots. 
\end{itemize}
\end{itemize}
\uncover<7->{
\vskip -0.2cm
\begin{example}
\vskip -0.1cm
\begin{columns}
\column{0.4\textwidth}
\psset{xunit=0.46cm, yunit=0.46cm}
\begin{pspicture}(-4, -4)(5,5)
\tiny
\fcBoundingBox{-4.2}{-4.2}{4.2}{4.2}
\psgrid[subgriddiv=0,griddots=10,gridlabels=0pt](-4,-4)(4,4)
\psaxes[labels=none, arrows=->](0,0)(-4,-4)(4,4)
\fcLabels{4}{4}
\uncover<9-10>{
\fcFullDot{1}{0}
\fcFullDot{0}{1}
}
\uncover<10>{\psline[linecolor=\fcColorGraph](-3,4)(4,-3)}
\uncover<11->{\psline[linecolor=gray](-3,4)(4,-3)}
%
\uncover<12-13>{
\fcFullDot{0}{-3}
\fcFullDot{1.5}{0}
}
\uncover<13>{\psline[linecolor=\fcColorGraph](3.5,4)(-0.5,-4)}
\uncover<14->{\psline[linecolor=gray](3.5,4)(-0.5,-4)}
%
\uncover<15-16>{
\fcFullDot{0}{2}
\fcFullDot{1}{2}
}
\uncover<16>{\psline[linecolor=\fcColorGraph](-4,2)(4,2)}
\uncover<17->{\psline[linecolor=gray](-4,2)(4,2)}
%
\uncover<18-19>{
\fcFullDot{-1}{0}
\fcFullDot{-1}{1}
}
\uncover<19>{\psline[linecolor=\fcColorGraph](-1,-4)(-1,4)}
%\uncover<19>{\psline[linecolor=gray](-3,4)(4,-3)}
\end{pspicture}
\column{0.6\textwidth}
Plot the line with the given equation.
\begin{tabular}{ccc}
equation& pt.  & another pt.\\
$\alertNoH{8-10}{x+y=1}$ & $\fcAnswer{9}{(1,0)}$ &$\fcAnswer{9}{(0,1)}$ \\
$\alertNoH{11-13}{2x-y=3}$& $\fcAnswer{12}{(0,-3)}$ &$\fcAnswer{12}{\left(\frac{3}{2},0\right)}$\\
$\alertNoH{14-16}{y=2}$   & $\fcAnswer{15}{(0,2)}$ &$\fcAnswer{15}{(1,2)}$\\
$\alertNoH{17-19}{x=-1}$  & $\fcAnswer{18}{(-1,0)}$ &$\fcAnswer{18}{(-1,1)}$
\end{tabular}
Other points can be used as well.
\end{columns}
}
\vskip -0.2cm
\end{example}
\end{frame}
\begin{frame}
\begin{example}
\newcommand{\xone}{\makebox[\widthof{$x_1$}]{1}}
\newcommand{\xtwo}{\makebox[\widthof{$x_2$}]{2}}
\newcommand{\yone}{\makebox[\widthof{$y_1$}]{3}}
\newcommand{\ytwo}{\makebox[\widthof{$y_2$}]{6}}
Find an equation of the line passing through $(\alertNoH{13}{\xone},\alertNoH{13}{\yone})$ and $(\alertNoH{13}{\xtwo},\alertNoH{13}{\ytwo})$.
\[
\fcAnswerUncover{3}{9}{(\alertNoH{13}{\xtwo}-\alertNoH{13}{\xone})} \uncover<3->{(\alertNoH{12,5}{y-\alertNoH{13}{\yone}})=}
\fcAnswerUncover{3}{10}{(\alertNoH{13}{\ytwo}-\alertNoH{13}{\yone})} \uncover<3->{(\alertNoH{12,4}{x-\alertNoH{13}{\xone}})} \]
\begin{itemize}

\item<2-> It suffices to manufacture a linear equation such that when we \alertNoH{3}{plug in} \alertNoH{3}{$ (\alertNoH{13}{\xone}, \alertNoH{13}{\yone} )$} and \alertNoH{6,7}{$ (\alertNoH{13}{\xtwo}, \alertNoH{13}{\ytwo} )$} we get an identity.
\item<3-> A (very simple) equation satisfied by $\alertNoH{3, 4}{x =\alertNoH{13}{\xone}}$, $\alertNoH{3, 5}{y=\alertNoH{13}{\yone}}$ is:

\hfil \hfil$
\alertNoH{5}{\alertNoH{7}{y}-\alertNoH{13}{\yone}}=\alertNoH{4}{\alertNoH{6}{x}-\alertNoH{13}{\xone}}.
$

\uncover<4->{This is so because both sides \alertNoH{4,5}{become zero} when $\alertNoH{4}{x=\alertNoH{13}{\xone}}$, $\alertNoH{5}{y=\alertNoH{13}{\yone}}$.}
\item<6-> If we plug in $\alertNoH{6}{x=\alertNoH{13}{\xtwo}}$ and $\alertNoH{7}{y=\alertNoH{13}{\ytwo}}$ in the above \alertNoH{8}{we don't get an identity}\uncover<9->{, but that can be easily fixed:}

\hfil \hfil $
\uncover<9->{\alertNoH{9}{(\alertNoH{13}{\xtwo}-\alertNoH{13}{\xone})}} \alertNoH{10}{(\alertNoH{7,13}{\ytwo}-\alertNoH{13}{\yone})}
\only<handout:0|6-9>{\alertNoH{8}{\neq}}  
\only<10->{=} 
\uncover<10->{\alertNoH{10}{(\alertNoH{13}{\ytwo}-\alertNoH{13}{\yone})}} \alertNoH{9}{(\alertNoH{6,13}{\xtwo}-\alertNoH{13}{\xone})} 
$
\item<11-> Perhaps the last modification caused $x=\alertNoH{13}{\xone}$, $y=\alertNoH{13}{\yone}$ to no longer be solutions? \uncover<12->{No - both sides are still zero when $\alertNoH{12}{x=\alertNoH{13}{\xone}}$, $\alertNoH{12}{y=\alertNoH{13}{\yone}}$.}
\end{itemize}
\end{example}
\end{frame}

\begin{frame}
\begin{example}
Find an equation of the line passing through $(\alertNoH{1}{x_1},\alertNoH{1}{y_1})$ and $(\alertNoH{1}{x_2}, \alertNoH{1}{y_2})$.
\[
\alertNoH{2}{(\alertNoH{1}{x_2}-\alertNoH{1}{x_1})(y-\alertNoH{1}{y_1})=(\alertNoH{1}{y_2}-\alertNoH{1}{y_1})(x-\alertNoH{1}{x_1})}
\]
\begin{itemize}

\item It suffices to manufacture a linear equation such that when we plug in $(\alertNoH{1}{x_1},\alertNoH{1}{y_1})$ and $(\alertNoH{1}{x_2},\alertNoH{1}{y_2})$ we get an identity.
\item A (very simple) equation satisfied by $x=\alertNoH{1}{x_1}$, $ y=\alertNoH{1}{y_2}$ is:

\hfil \hfil$
y-\alertNoH{1}{y_2}=x-\alertNoH{1}{x_1}.
$

This is so because both sides become zero when $x=\alertNoH{1}{x_1}$, $y=\alertNoH{1}{y_1}$.
\item If we plug in $x=\alertNoH{1}{x_2}$ and $y=\alertNoH{1}{y_2}$ in the above we don't get an identity (necessarily), but that can be easily fixed:

\hfil \hfil $
(\alertNoH{1}{x_2}-\alertNoH{1}{x_1} )(\alertNoH{1}{y_2}- \alertNoH{1}{ y_1 } )= ( \alertNoH{1}{y_2}-\alertNoH{1}{y_1}) (\alertNoH{1}{x_2} - \alertNoH{1}{ x_1}) 
$
\item Perhaps the last modification caused $x=\alertNoH{1}{x_1}$, $y=\alertNoH{1}{y_1}$ to no longer be solutions? No - both sides are still zero when $x=\alertNoH{1}{x_1}$, $y=\alertNoH{1}{y_1}$.
\end{itemize}
\uncover<2>{}
\end{example}
\end{frame}
\begin{frame}
\begin{example}
Find an equation of a line passing though the indicated pairs of points.
\begin{itemize}
\item $(1,2)$ and $(2,-1)$.
\item $(1,1)$ and $(2,-2)$.
\item $(0,1)$ and $(1,0)$.
\item $(3,5)$ and $(7,-11)$.
\end{itemize}
\end{example}
\end{frame}

\begin{frame}
To find the intersection of two lines  (if they do intersect) with equations $a_1x+b_1y+c_1=0$ and $a_2x+b_2y+c_2=0$ we need to solve the system of equations

\[
\left|\begin{array}{rcl}
a_1x+b_1y+c_1&=&0\\
a_2x+b_2y+c_2&=&0
\end{array}\right.
\]
\end{frame}
\begin{frame}
\begin{example}
Find the intersection of the following lines.
\begin{enumerate}
\item $x-y=3$ and $x+2y=10$.
\item $3x-y=3$ and $x=1-3y$.
\item Line $x=3$ and $x=1-2y$
\item Line through $(2,0)$ and $(1,2)$ and line through $(3,7 )$ and $(2,5)$.
\item Line through $(3,-1)$ and $(-1, 3)$ and line through $(1,1)$ and $(2,3)$.
\end{enumerate}

\end{example}

\end{frame}
%\input{../../modules/lines-2d/parallel-lines-and-slope}

}% end lecture

\end{document}
