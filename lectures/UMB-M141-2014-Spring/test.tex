\documentclass%
%[handout]
{beamer}
% % % % % % % %
% % % % % % % %
% % % % % % % %
%IMPORTANT
%compiles with 
%pdflatex -shell-escape 
%IMPORTANT
% % % % % % % %
% % % % % % % %
% % % % % % % %
\mode<presentation>
{
\useinnertheme{rounded}
\useoutertheme{infolines}
\usecolortheme{orchid}
\usecolortheme{whale}
}

\usepackage[english]{babel}
\usepackage[latin1]{inputenc}
\usepackage[all,cmtip]{xy}
\usepackage{times}
\usepackage[T1]{fontenc}
\usepackage{../example-templates}
\usepackage{../pstricks-commands}

\usepackage{auto-pst-pdf}
\usepackage{pst-plot}
%\usepackage{pstricks-add} 

% Or whatever. Note that the encoding and the font should match. If T1
% does not look nice, try deleting the line with the fontenc.


\graphicspath{{../../modules/}}

\newtheoremstyle{partialproof}{3pt}{3pt}{}{}{}{.}{.5em}{}
\theoremstyle{partialproof} \newtheorem{partialproof}[theorem]{Proof.}
%\DeclareMathOperator{\diff}{d}
\setbeamertemplate{navigation symbols}{}

\includeonlylecture{1}

\newcommand{\lect}[3]{
  \date{#1}
  \lecture[#1]{#2}{#3}
}

\setbeamertemplate{footline}
{
  \leavevmode%
  \hbox{%
  \begin{beamercolorbox}[wd=.333333\paperwidth,ht=2.25ex,dp=1ex,center]{author in head/foot}%
    \usebeamerfont{author in head/foot}\insertshortauthor
  \end{beamercolorbox}%
  \begin{beamercolorbox}[wd=.333333\paperwidth,ht=2.25ex,dp=1ex,center]{title in head/foot}%
    \usebeamerfont{title in head/foot}\insertshorttitle
  \end{beamercolorbox}%
  \begin{beamercolorbox}[wd=.333333\paperwidth,ht=2.25ex,dp=1ex,center]{date in head/foot}%
    \usebeamerfont{date in head/foot}\insertshortdate{}
  \end{beamercolorbox}}%
  \vskip0pt%
}

% If you have a file called "university-logo-filename.xxx", where xxx
% is a graphic format that can be processed by latex or pdflatex,
% resp., then you can add a logo as follows:

%\pgfdeclareimage[height=0.8cm]{logo}{bluelogo}
%\logo{\pgfuseimage{logo}}
\renewcommand{\Arcsin}{\arcsin}
\renewcommand{\Arccos}{\arccos}
\renewcommand{\Arccot}{\text{arccot}}
\renewcommand{\Arctan}{\arctan}


\begin{document}

\AtBeginLecture{%

\title[\insertlecture]{FreeCalc}
\subtitle{\insertlecture}
\author[FreeCalc]{}
\institute[UMass Boston]{University of Massachusetts Boston}
\date{\insertshortlecture}
\begin{frame}
  \titlepage
\end{frame}
}%

% begin lecture
\lect{\today}{Sample}{1}
\begin{frame}
The integral $\int \sec \theta \diff \theta$ appears often in practice. A somewhat quicker alternative method for doing this particular integral will be shown later. Here we solve that integral using the standard method.
\begin{example}
Set $\theta=2\arctan t$, $\cos \theta = \frac{1-\tan^2(\frac{\theta}{2}) }{1+\tan^2\left(\frac{\theta}{2}\right)}=\frac{1-t^2}{1+t^2}$, $\diff \theta = 2\frac{1}{1+t^2}\diff t$.
\[
\begin{array}{rclll}
\int \sec \theta\diff\theta &=& \int \frac{1}{\cos \theta}\diff \theta = \int \frac{1}{\frac{1-t^2}{1+t^2}} \frac{2}{1+t^2} \diff t= \int \frac{2}{1-t^2}\diff t &&\textbf{split to partial fractions }\\
&=&\int \left(\frac{1}{1-t}+\frac{1}{1+t}\right)\diff t=- \ln |1-t|+\ln |1+t| +C=\ln |\frac{1-t}{1+t}|+C\\
&=& \ln |\frac{1-\tan \left(\frac{\theta}{2}\right)}{1+\tan \left(\frac{\theta}{2}\right)}|+C = \ln |\tan \theta +\sec \theta| +C.
\end{array}
\]
\end{example}

\end{frame}
%% begin module trig-integrals-tan-sec
\begin{frame}
\frametitle{Strategy for Evaluating $\int \tan^m x \sec^n x \diff x$}
\only<handout:1| -1>{%
\begin{enumerate}
\item  If the power of secant is even ($n = 2k$), save a factor of $\sec^2 x$ and use $\sec^2 x = 1 + \tan^2 x$ to express the remaining factors in terms of tangent:
\begin{eqnarray*}
\int \tan^m x \sec^{2k} x \diff x & = & \int \tan^m x (\sec^2 x)^{k-1} \sec^2 x \diff x\\
& = & \int \tan^m x (1 + \tan^2 x)^{k-1}\sec^2 x \diff x
\end{eqnarray*}
Then substitute $u = \tan x$.
\end{enumerate}
}%
\only<handout:2| 2>{%
\begin{enumerate}
\setcounter{enumi}{1}
\item  If the power of tangent is odd ($m = 2k+1$), save one factor of $\sec x \tan x$ and use $\tan^2 x =  \sec^2 x - 1$ to express the remaining factors in terms of secant:
\begin{eqnarray*}
\int \tan^{2k+1} x \sec^{n} x \diff x & = & \int (\tan^2 x)^k \sec^{n-1} x \sec x\tan x \diff x\\
& = & \int (\sec^2 x - 1)^k\sec^{n-1} x\sec x\tan x \diff x
\end{eqnarray*}
Then substitute $u = \sec x$.
\end{enumerate}
}%
\only<handout:3| 3->{%
For other cases, more creativity is required.  We may need to use identities or integration by parts.  We will also need to know the indefinite integrals of tangent and secant.

\begin{columns}[t]
\column{.5\textwidth}
\abovedisplayskip=0pt
\belowdisplayskip=0pt
\begin{eqnarray*}
& & \int \tan x \diff x\\
& \uncover<4->{ = } & %
\uncover<4->{\int \frac{\sin x}{\cos x}\diff x}\\
& & \uncover<5->{\textrm{Let $u = \cos x$, so }}\\
& & \uncover<5->{\diff u = -\sin x \diff x}\\
& \uncover<6->{ = } & %
\uncover<6->{-\int \frac{\diff u}{u}}\\
& \uncover<7->{ = } & %
\uncover<7->{-\ln |u| + C}\\
& \uncover<8->{ = } & %
\uncover<8->{-\ln |\cos x| + C}\\
& \uncover<9->{ = } & %
\uncover<9->{\ln |\sec x| + C}\\
\end{eqnarray*}
\column{.5\textwidth}
\abovedisplayskip=0pt
\belowdisplayskip=0pt
\begin{eqnarray*}
& & \int \sec x \diff x\\
& \uncover<10->{ = } & %
\uncover<10->{\int \sec x\frac{\sec x + \tan x}{\sec x + \tan x}\diff x}\\
& \uncover<11->{ = } & %
\uncover<11->{\int \frac{\sec^2 x + \sec x \tan x}{\sec x + \tan x}\diff x}\\
& & \uncover<12->{\textrm{Let $u = \sec x + \tan x$:}}\\
& \uncover<13->{ = } & %
\uncover<13->{\int \frac{\diff u}{u}}\\
& \uncover<14->{ = } & %
\uncover<14->{\ln |u| + C}\\
& \uncover<15->{ = } & %
\uncover<15->{\ln |\sec x + \tan x| + C}\\
\end{eqnarray*}
\end{columns}
}%
\end{frame}
% end module trig-integrals-tan-sec


%\section{Integrals of form $\int R(x,\sqrt{ax^2+bx+c}) \diff x$, $R$ - rational function}
%%begin module Euler-substitution-intro
\begin{frame}
\frametitle{Integrals of form $\int R(x,\sqrt{ax^2+bx+c}) \diff x$, $R$ - rational function}
Let $R(x,y)$ be an arbitrary rational expression in two variables (quotient of polynomials in two variables).
\begin{question}
Can we integrate $\alert<10>{\displaystyle\int R(x,\sqrt{ax^2+bx+c})\diff x}$?
\end{question}
\begin{itemize}
\item<2-> Yes. We will learn how in what follows.
\item<3-> The algorithm for integration is roughly:
\begin{itemize}
\item<4-> Use linear substitution to transform to one of three integrals: 
\uncover<5->{$\int R(x, \sqrt{-x^2+1})\diff x$, } \uncover<6->{ $\int R(x, \sqrt{x^2+1})\diff x$, } \uncover<7->{$\int R(x, \sqrt{x^2-1}) \diff x$.}
\item<8-> Use Euler substitution to transform to rational function integral (no radicals).
\item<9-> Solve as previously studied.
\end{itemize}
\item<10,11-> We motivate why we need \alert<10>{such integrals later}; we promise they allow to compute ellipse area and the volume of a ball.
\end{itemize}
\end{frame}
%end module Euler-substitution-intro
%\subsection{Euler substitution}
%%begin module Euler substitution
\begin{frame}
\frametitle{Euler substitution}
\begin{itemize}
\item Using linear substitutions, radicals of form  $\sqrt{ay^2+by+c})$, $a\neq 0$, $b^2-4ac\neq 0$ can be transformed to (multiple of):
\begin{itemize}
\item $\sqrt{x^2+1}$ 
\item $\sqrt{-x^2+1}$
\item $\sqrt{x^2-1}$.
\end{itemize}
\item We already studied how to do that using completing the square. 
\end{itemize}
\end{frame}
\begin{frame}
Recall that a (real) linear substitution is a substitution of the form $u=px+q$, $p,q$- (real) constants.
\begin{example}
Use a linear substitution to transform $\sqrt{x^2+x+1}$ to (a multiple of) an expression of the form $\sqrt{u^2+1}$. 

\[
\begin{array}{rcl}
\sqrt{x^2+x+1}&=&\sqrt{ x^2+2\frac{1}{2}x +\frac{1}{4}\textbf{?}-\frac{1}{4}\textbf{?} +1} \\
&=& \sqrt{ \left(x+\frac{1}{2}\textbf{?} \right)^2-\textbf{?} }\\
&=&\sqrt{\frac{3}{4}\left( \frac{4}{3} \left(x+\frac{1}{2}\right)^2+1 \right)}\\
&=&\frac{\sqrt{3}}{2}\sqrt{\left(\frac{2}{\sqrt{3}}\left( x+\frac{1}{2}\right)\right)^2+1}\\
&=& \frac{\sqrt{3}}{2} \sqrt{u^2+1},
\end{array}
\]
where $u=\frac{2}{\sqrt{3}}\left( x+\frac{1}{2}\right) \textbf{?}=\frac{2\sqrt{3}}{3}x+\frac{\sqrt{3}}{3}$.
\end{example}
\vspace{5cm}
\end{frame}
\begin{frame}
Recall that a (real) linear substitution is a substitution of the form $u=px+q$, $p,q$- (real) constants.
\begin{example}
Use a linear substitution to transform $\sqrt{-2x^2+x+1}$ to (a multiple of) an expression of the form $\sqrt{-u^2+1}$. 

\[
\begin{array}{rcl}
\sqrt{-2x^2+x+1}&=&\sqrt{ -2\left(x^2-\frac{1}{2}x -\frac{1}{2}\right) } \\
&=& \sqrt{ -2\left(x^2-2\frac{1}{4}x +\frac{1}{16}-\frac{1}{16}-\frac{1}{2}\right) }\\
&=&\sqrt{-2\left(\left(x-\frac{1}{16}\right)^2-\frac{9}{16} \right)}\\
&=&\sqrt{\frac{9}{8}\left(-\frac{16}{9}\left(x-\frac{1}{16}\right)^2+1 \right)}\\
&=&\frac{3}{\sqrt{8}}\sqrt{-\left(\frac{4}{3}\left(x-\frac{1}{16}\right)\right)^2+1 }\\
&=&\frac{ 3}{\sqrt{8}} \sqrt{-u^2+1}
\end{array}
\]
where $u=\frac{4}{3}\left(x-\frac{1}{16}\right)  \textbf{?}=\frac{4}{3}x-\frac{1}{12}$
\end{example}
\end{frame}
%end module Euler substitution.
%\input{../../modules/Euler-substitution/Euler-substitution-case-1}
%%begin module Euler-substitution-case-2
\begin{frame}
\frametitle{Euler substitution to handle $\sqrt{-x^2+1} $}
\begin{itemize}
\item<1-> Suppose we want to integrate 
\[
\int R(x, \sqrt{-x^2+1})\diff x\quad .
\]
\only<2-15>{
\item<2-> Substitute $\alert<9,15>{\sqrt{-x^2+1}}=(1-\alert<11>{x})t  \uncover<11->{ =\left( 1 -\alert<11>{ \left(1-\frac{2}{ t^2+1} \right)} \right)t =\alert<15>{\frac{2t}{t^2+1}}}$. 
\item<3-> How is this a substitution of $x$ via t?
\uncover<4->{
\[
\begin{array}{rcll|l}
\sqrt{-x^2+1}&=&(1-x)t&&\text{square}\\
\uncover<5->{(1-x)(1+x)&=&(1-x)^2t^2&&\text{divide by }1-x}\\
\uncover<6->{1+x&=&(1-x)t^2}\\
\uncover<7->{xt^2+x &=&t^2-1}\\
\uncover<8->{x(t^2+1) &=&t^2-1 &&\text{divide by }t^2+1}\\
\uncover<9->{\alert<11,12,15>{x}&\alert<11,15>{=}& \frac{  t^2\uncover<10->{+1-1}-1}{t^2+1} \uncover<10->{ = \alert<11,12,15>{1 -\frac{2}{ t^2+1} }\quad .}}
\end{array}
\] 
\item<12-> 
\[
\alert<12,15>{\diff x =} \alert<12,13,14>{ \diff \left(1-\frac{2}{t^2+1}\right)} \uncover<13->{\alert<13,14,15>{=}}\only<13>{\alert<13>{\textbf{?}}} \uncover<14->{\alert<14,15>{ \frac{4t}{(1+t^2)^2}dt} \quad .}
 {~~~~~~~~~~~~~~~~~~~~~~~~~~~~~~~~~~~~~~~~~~~~~~~~~~~~}
\] 
}
}
\end{itemize}
\uncover<16->{
\begin{definition}
The Euler substitution for $\sqrt{-x^2+1}$ is the substitution given by:
\[
\begin{array}{rcl}
\displaystyle \alert<16>{\sqrt{-x^2+1}}&\alert<16>{=}& \displaystyle  \alert<16>{\frac{2t}{t^2+1}}\\
\displaystyle  \alert<16>{x}&\alert<16>{=}&\displaystyle  \alert<16>{1-\frac{2}{t^2+1}}\\
\displaystyle \alert<16>{\diff x }&\alert<16>{=} & \displaystyle  \alert<16>{\frac{4t}{(t^2+1)^2}dt }\quad .
\end{array}
\] 
\end{definition}
}
\vspace{5cm}
\end{frame}
%end module Euler-substitution-case-2
%%begin module Euler-substitution-case-3
\begin{frame}

\frametitle{Euler substitution to handle $\sqrt{x^2-1} $}

\begin{itemize}
\item<1-> Suppose we want to integrate 
\[
\int R(x, \sqrt{x^2-1})\diff x\quad .
\]
\item<2-> Substitute $\alert<12>{\sqrt{x^2-1}=(x-1)t}$. 
\item<3-> How is this a substitution of $x$ via $t$? 
\[
\begin{array}{rcll|l}
\uncover<4->{\sqrt{x^2-1}&=&(x-1)^2t^2}&&\text{square both sides}\\
\uncover<5->{(x-1)(x+1)  &=&(x-1)^2t^2}\\
\uncover<6->{x+1 &=&(x-1)t^2}\\
\uncover<7->{x(1-t^2) &=&-1-t^2}\\
\uncover<8->{\alert<9,12>{x}&\alert<9,12>{=}&\displaystyle \frac{-1-t^2}{1-t^2}=\alert<9,12>{1+\frac{2}{t^2-1}}}
\end{array}
\]
\item<9->
\[
\alert<9>{ \alert<12>{\diff x=} \alert<10,11>{\diff \left(1+\frac{2}{t^2-1} \right)}} \uncover<10->{\alert<10,11>{=}} \only<10>{\alert<10>{\textbf{?}}} \uncover<11->{\alert<11,12>{\frac{-4t}{(t^2-1)^2}\diff t} .} {~~~~~~~~~~~~~~~~~~~~~~~~~~~~~~~~~~~~~~~~~~~~~~~~}
\] 
\end{itemize}
\uncover<13->{
\begin{definition}
\[
\begin{array}{rcl}
\diff x&=&\frac{-4t}{(t^2-1)^2}dt\\
\sqrt{-x^2+1}&=&\frac{2t}{t^2-1} \\
x&=&1+\frac{2}{t^2-1}\quad .
\end{array}
\] 
\end{definition}
}
\end{frame}
%end module Euler-substitution-case-3
%\input{../../modules/Euler-substitution/Euler-substitution-theorem}
%\section{Integrals of form $\int R(\cos x,\sin x) \diff x$, $R$ - rational function}
%%begin module trig-integrals-rationalizing-substitution
\begin{frame}
\frametitle{Integrals of the form $\int R(\cos \theta,\sin \theta) \diff \theta$, $R$}

Let $R$ be an arbitrary rational function in two variables (quotient of polynomials in two variables).
\begin{question}
Can we integrate $\int R(\cos \theta, \sin \theta)\diff \theta$?
\end{question}
\begin{itemize}
\item<2-> Yes. We will learn how in what follows.
\item<3-> The algorithm for integration is roughly:
\begin{itemize}
\item<4-> Apply the substitution $\theta=2\Arctan t$ to transform to integral of rational function.
\item<5-> Solve as previously studied.
\end{itemize}
\end{itemize}
\end{frame}

\begin{frame}
\frametitle{The rationalizing substitution $\theta= 2\Arctan t$}
\uncover<13->{
\noindent 
Let $R$- rational function in two variables. 
$\int R(\alert<14,21>{\cos \theta,\sin \theta} ) \alert<22>{\diff \theta} $
can be integrated via the  substitution $\alert<14,15,18,23, 26>{ \theta=2\arctan t} $.
\uncover<14->{ How does this transform \alert<14,21>{$\sin \theta$, $\cos\theta$}? }\uncover<22->{How does this transform $\alert<22>{\diff \theta} $?} \uncover<26->{\alert<26>{How is $t$ expressed via $\theta$?}}
\[
\begin{array}{rcl}
\uncover<14->{ \alert<14,21>{\sin\alert<15>{\theta}}} &\uncover<14->{=} &\displaystyle \uncover<15->{ \alert<16>{\sin (\alert<15>{ 2\Arctan t} )}} \uncover<16->{ \alert<16>{= \frac{2 \alert<17>{\tan\left( \Arctan t\right)} }{1 + {\alert<17>{\tan}}^2 \alert<17>{ \left(\Arctan t \right)}} }} \uncover<17->{\alert<21>{ = \frac{2\alert<17>{ t}}{1+ {\alert<17>{t }}^2}}}\\
\uncover<14->{\alert<14,21>{\cos \alert<18>{\theta}}  } &\uncover<14->{=} &\displaystyle \alert<19>{ \uncover<18-> {\alert<19>{ \cos (\alert<18>{2\Arctan t}) }} } \uncover<19->{ \alert<19>{= \frac{1-{\alert<20>{\tan} }^2 \alert<20>{ (\Arctan t)}}{1+ {\alert<20>{\tan}}^2 \alert<20>{ (\Arctan t) }}}} \uncover<20->{  \alert<21>{= \frac{1- {\alert<20>{ t}}^2 }{1 +{\alert<20>{t} }^2}}} \\
\only<22->{
\uncover<22->{
\alert<22,23>{\diff \theta}}&\uncover<23->{\alert<23>{=}}& \displaystyle \uncover<23->{ \alert<23,24,25>{2 \diff \left(\Arctan t\right)}}\uncover<24->{ \alert<24,25>{=  \uncover<25->{ \alert<25>{\frac{1}{ 1+t^2}}} \uncover<24->{ \uncover<24>{ \textbf{?}}} \diff t}}\\
\uncover<26->{\alert<26,27>{t}&\alert<26,27>{=}&}\displaystyle \uncover<27->{\alert<27>{\tan \left(\frac{\theta}{2}\right)} }
}
\end{array}
\]
}

\only<1-20>{
Recall the expression of $\sin (2z), \cos (2z)$ via $\tan z$:
\[
\begin{array}{rcl}
\uncover<1->{\alert<1,2,16>{\sin \left(2z\right)}} &\uncover<1->{\alert<1>{=}}&\displaystyle  \uncover<2->{ \alert<2>{2\sin z\cos z}} \uncover<3->{=\frac{2 \alert<5>{\sin z\cos z} \uncover<4->{\alert<4>{ \frac{1}{\alert<5>{ \cos^2z}}}}}{\alert<3,6>{( \cos^2z +\sin^2z) }\uncover<4->{\alert<4,6>{\frac{1}{\cos^2z}}}}} \uncover<5->{\alert<16>{= \frac{2\alert<5>{\tan z} }{ \alert<6>{ 1+ \tan^2z}} } \quad .}\\
\uncover<1->{\alert<7,8,19>{\cos (2z)} }& \uncover<7->{ \alert<7,8>{= }}&\displaystyle\uncover<8->{\alert<8>{ \cos^2z-\sin^2z}} \uncover<9->{= \frac{ \alert<11>{ \left(\cos^2 z-\sin^2 z\right) \uncover<10->{\alert<10>{ \frac{1}{ \cos^2z} }}}}{\alert<12>{ \alert<9>{\left(\cos^2z +\sin^2 z\right)} \uncover<10->{\alert<10>{ \frac{1}{ \cos^2z }}} }}} \uncover<11->{\alert<19>{ =\frac{\alert<11>{ 1-\tan^2 z} }{\alert<12>{1+\tan^2z}}} ~ .}
\end{array}
\]
}



\uncover<28->{ 
\begin{theorem}
The substitution given above  transforms $ \int R(\cos \theta, \sin\theta)\diff \theta$ to an integral of a rational function of $t$.
\end{theorem}
}
\vspace{10cm}
\end{frame}
%end module trig-integrals-rationalizing-substitution
%%begin module trig-integrals-rationalizing-substitution-ex1
\begin{frame}
\begin{example}
\uncover<2->{Let $\alertNoH{2,25}{\theta=2\arctan t}$, \alertNoH{4}{$\cos \theta=\frac{1-t^2}{1+t^2}$}, \alertNoH{3}{$\sin \theta=\frac{2t}{1+t^2}$}}\uncover<22->{, $\alertNoH{22,24}{z= \frac{3}{\sqrt{5}} \left(t + \frac{1}{3} \right)}$.}
\[
\begin{array}{rcl}
\displaystyle \int \frac{\alertNoH{2}{ \diff \theta} }{ 2\alertNoH{3}{\sin \theta} -\alertNoH{4}{ \cos \theta} +5}
\only<handout:1|1-16>{
\uncover<2->{&=& \displaystyle \int \frac{\alertNoH{2}{ 2\diff t} }{\alertNoH{2}{\alertNoH{6,7,9}{(\alertNoH{8}{1}+\alertNoH{5}{t^2})}} \left(\alertNoH{7}{ 2 \alertNoH{3}{\frac{2 t}{ t^2+1}} } \alertNoH{6,9}{-} \alertNoH{4}{ \frac{(\alertNoH{9}{ 1} \alertNoH{6}{- t^2}) }{\alertNoH{6,9}{1+t^2}}}+\alertNoH{5,8}{5}\right)}} \\
\uncover<5->{ &=&\displaystyle \int \frac{\alertNoH{10}{2} \diff t}{ \alertNoH{5,6}{ \alertNoH{10}{6}t^2} +\alertNoH{7}{ \alertNoH{10}{4} t} +\alertNoH{8,9,10} {4}}}\\
\uncover<10->{&=&\displaystyle  \int \frac{\diff t}{\alertNoH{10,11}{3}t^2+\alertNoH{10}{2}t+\alertNoH{10}{2}}}\\
\uncover<11->{ \uncover<12>{\alertNoH{12}{\text{(complete square)}}} &=&\displaystyle \int \frac{\diff t}{ \alertNoH{11}{3}\left(\alertNoH{13}{ t^2+ 2t\frac{ 1}{\alertNoH{11}{3}}} \uncover<12->{\alertNoH{12}{\alertNoH{13}{+ \frac{1}{9}} \alertNoH{14}{-\frac{1}{9}}}} \alertNoH{14}{+ \frac{ 2}{ \alertNoH{11}{3}}} \right)}} \\
\uncover<13->{ &=& \displaystyle \frac{1}{3}\int\frac{\diff t}{\alertNoH{13}{ \left(t+\frac{1}{3}\right)^2} + \alertNoH{14,15}{ \frac{ 5}{9}}}} \\
}
\uncover<15->{&\alertNoH{16,17}{=}&\alertNoH{16,17}{\displaystyle \alertNoH{18}{\frac{1}{3}} \int \frac{\diff t }{ \alertNoH{15,18}{\frac{5}{9}} \left( \alertNoH{15,19}{\frac{9}{5}} \left(t+ \frac{1}{3} \right)^2 +\alertNoH{15}{1} \right)}}} {~~~~~~~~~~~~~~~~~~~~~~~~~~~~~~~} {~~~~~~~~~~~~~~~}\\
\only<handout:2|17->{
\uncover<18->{&=&\displaystyle \alertNoH{18}{\frac{3}{5}}\int \frac{\uncover<20->{\alertNoH{20}{ \frac{\sqrt{5}}{3}}} \diff \left( \alertNoH{22}{ \uncover<20->{\alertNoH{20}{\frac{3}{ \sqrt{5}}}}\left( t \uncover<21->{\alertNoH{21}{+\frac{1}{3}}}\right)} \right) }{\left(\left(\alertNoH{22}{ \alertNoH{19}{ \frac{3}{ \sqrt{5}}} \left( t+\frac{1}{3}\right)}\right)^{\alertNoH{19}{2} }+1 \right)} }\\
\uncover<22->{ &=&\displaystyle \frac{\sqrt{5}}{5} \alertNoH{23}{\int \frac{\diff \alertNoH{22}{ z}}{{\alertNoH{22}{ z}}^2+1}}}\\
\uncover<23->{&=& \displaystyle  \frac{\sqrt{5}}{5} \alertNoH{23}{\Arctan \alertNoH{24}{z}} +C} \\
\uncover<24->{ &=&\displaystyle \frac{ \sqrt{5}}{5}\Arctan \left(\alertNoH{24}{ \frac{3}{ \sqrt{5}} \left(\alertNoH{25}{ t}+\frac{1}{3} \right)} \right)+C}\\
\uncover<25->{&=&\displaystyle \frac{ \sqrt{5}}{5}\Arctan \left( \frac{3}{ \sqrt{5}} \left(\alertNoH{25}{ \tan \left(\frac{\theta}{2} \right)}+\frac{1}{3} \right) \right)+C
}
}
\end{array}
\]

\end{example}
\vspace{5cm}

\end{frame}
%end module trig-integrals-rationalizing-substitution-ex1

%\input{../../modules/trig-substitution/quadratic-radicals-integrals-intro}
%% begin module trig-substitutions-intro
\begin{frame}
\frametitle{Trigonometric Substitution}
\begin{itemize}
\item  To find the area of a circle or ellipse, one needs to compute $\int \sqrt{a^2 - x^2} \diff x$.
\item<2->  For $\int x\sqrt{a^2 - x^2}\diff x$, the substitution $u = a^2 - x^2$ would work.
\item<3->  For $\int \sqrt{a^2 - x^2}\diff x$, we need a more elaborate substitution.
\item<4-| alert@6>  Instead, substitute $x = a\sin \theta$.
\end{itemize}
\[
\uncover<5->{%
\sqrt{a^2-\alert<handout:0| 6>{x^2}} = %
}%
\uncover<6->{%
\sqrt{a^2-\alert<handout:0| 6>{a^2\sin^2 \theta}} = %
}%
\uncover<7->{%
\sqrt{a^2(1 - \sin^2 \theta )} = %
}%
\uncover<8->{%
\sqrt{a^2\cos^2 \theta} = a|\cos \theta |.%
}%
\]
\begin{itemize}
\item<9->  With $u = a^2 - x^2$, the new variable is a function of the old one.
\item<10->  With $x = a\sin \theta$, the old variable is a function of the new one.
%Greg: the below remarks seem redundant to me. Students know how to compute \diff x, I'd think they'd be more confused than englightened by ``substitutions in reverse''.
%\item<11->  To make a substitution of the form $x = g(t)$, use the substitution rule in reverse.
%\item<12->  We call this inverse substitution.
\end{itemize}
\end{frame}
% end module trig-substitutions-intro

%%begin module trig-integrals-rationalizing-substitution-ex1
\begin{frame}
\begin{example}
\uncover<2->{Let $\alertNoH{2,25}{\theta=2\arctan t}$, \alertNoH{4}{$\cos \theta=\frac{1-t^2}{1+t^2}$}, \alertNoH{3}{$\sin \theta=\frac{2t}{1+t^2}$}}\uncover<22->{, $\alertNoH{22,24}{z= \frac{3}{\sqrt{5}} \left(t + \frac{1}{3} \right)}$.}
\[
\begin{array}{rcl}
\displaystyle \int \frac{\alertNoH{2}{ \diff \theta} }{ 2\alertNoH{3}{\sin \theta} -\alertNoH{4}{ \cos \theta} +5}
\only<handout:1|1-16>{
\uncover<2->{&=& \displaystyle \int \frac{\alertNoH{2}{ 2\diff t} }{\alertNoH{2}{\alertNoH{6,7,9}{(\alertNoH{8}{1}+\alertNoH{5}{t^2})}} \left(\alertNoH{7}{ 2 \alertNoH{3}{\frac{2 t}{ t^2+1}} } \alertNoH{6,9}{-} \alertNoH{4}{ \frac{(\alertNoH{9}{ 1} \alertNoH{6}{- t^2}) }{\alertNoH{6,9}{1+t^2}}}+\alertNoH{5,8}{5}\right)}} \\
\uncover<5->{ &=&\displaystyle \int \frac{\alertNoH{10}{2} \diff t}{ \alertNoH{5,6}{ \alertNoH{10}{6}t^2} +\alertNoH{7}{ \alertNoH{10}{4} t} +\alertNoH{8,9,10} {4}}}\\
\uncover<10->{&=&\displaystyle  \int \frac{\diff t}{\alertNoH{10,11}{3}t^2+\alertNoH{10}{2}t+\alertNoH{10}{2}}}\\
\uncover<11->{ \uncover<12>{\alertNoH{12}{\text{(complete square)}}} &=&\displaystyle \int \frac{\diff t}{ \alertNoH{11}{3}\left(\alertNoH{13}{ t^2+ 2t\frac{ 1}{\alertNoH{11}{3}}} \uncover<12->{\alertNoH{12}{\alertNoH{13}{+ \frac{1}{9}} \alertNoH{14}{-\frac{1}{9}}}} \alertNoH{14}{+ \frac{ 2}{ \alertNoH{11}{3}}} \right)}} \\
\uncover<13->{ &=& \displaystyle \frac{1}{3}\int\frac{\diff t}{\alertNoH{13}{ \left(t+\frac{1}{3}\right)^2} + \alertNoH{14,15}{ \frac{ 5}{9}}}} \\
}
\uncover<15->{&\alertNoH{16,17}{=}&\alertNoH{16,17}{\displaystyle \alertNoH{18}{\frac{1}{3}} \int \frac{\diff t }{ \alertNoH{15,18}{\frac{5}{9}} \left( \alertNoH{15,19}{\frac{9}{5}} \left(t+ \frac{1}{3} \right)^2 +\alertNoH{15}{1} \right)}}} {~~~~~~~~~~~~~~~~~~~~~~~~~~~~~~~} {~~~~~~~~~~~~~~~}\\
\only<handout:2|17->{
\uncover<18->{&=&\displaystyle \alertNoH{18}{\frac{3}{5}}\int \frac{\uncover<20->{\alertNoH{20}{ \frac{\sqrt{5}}{3}}} \diff \left( \alertNoH{22}{ \uncover<20->{\alertNoH{20}{\frac{3}{ \sqrt{5}}}}\left( t \uncover<21->{\alertNoH{21}{+\frac{1}{3}}}\right)} \right) }{\left(\left(\alertNoH{22}{ \alertNoH{19}{ \frac{3}{ \sqrt{5}}} \left( t+\frac{1}{3}\right)}\right)^{\alertNoH{19}{2} }+1 \right)} }\\
\uncover<22->{ &=&\displaystyle \frac{\sqrt{5}}{5} \alertNoH{23}{\int \frac{\diff \alertNoH{22}{ z}}{{\alertNoH{22}{ z}}^2+1}}}\\
\uncover<23->{&=& \displaystyle  \frac{\sqrt{5}}{5} \alertNoH{23}{\Arctan \alertNoH{24}{z}} +C} \\
\uncover<24->{ &=&\displaystyle \frac{ \sqrt{5}}{5}\Arctan \left(\alertNoH{24}{ \frac{3}{ \sqrt{5}} \left(\alertNoH{25}{ t}+\frac{1}{3} \right)} \right)+C}\\
\uncover<25->{&=&\displaystyle \frac{ \sqrt{5}}{5}\Arctan \left( \frac{3}{ \sqrt{5}} \left(\alertNoH{25}{ \tan \left(\frac{\theta}{2} \right)}+\frac{1}{3} \right) \right)+C
}
}
\end{array}
\]

\end{example}
\vspace{5cm}

\end{frame}
%end module trig-integrals-rationalizing-substitution-ex1


%%begin module quadratic-radicals-linear-substitution-preparation-ex1
\begin{frame}
\frametitle{Linear substitutions to simplify radicals $\sqrt{ay^2+by+c}$}
\begin{itemize}
\item Using linear substitutions, radicals of form  $\sqrt{ay^2+by+c}$, $a\neq 0$, $b^2-4ac\neq 0$ can be transformed to (multiple of):
\begin{itemize}
\item $\sqrt{x^2+1}$ 
\item $\sqrt{-x^2+1}$
\item $\sqrt{x^2-1}$.
\end{itemize}
\item We already studied how to do that using completing the square when dealing with rational functions. 
\end{itemize}
\end{frame}
\begin{frame}
Recall: linear substitution is subst. of the form $u=px+q$.
\begin{example}
Use linear substitution to transform $\sqrt{x^2+x+1}$ to multiple of $\sqrt{u^2+1}$. 

\noindent $
\begin{array}{rcl}
\sqrt{x^2+x+1}&=&\displaystyle \uncover<2->{ \sqrt{ x^2+2\frac{1}{2}x + \uncover<3->{ \alert<3>{ \frac{1}{4} } } \uncover<2>{ \alert<2>{ \textbf{?}}} \uncover<2->{ \alert<2,3>{-} } \uncover<3->{ \alert<3>{ \frac{1}{4}}} \uncover<2>{\alert<2>{\textbf{?}}} +1}} \\
\uncover<4->{&=&\displaystyle \sqrt{ {\left(x+\uncover<5->{\alert<5>{\frac{1}{2}}} \uncover<4>{ \alert<4>{ \textbf{?}}} \right)}^2 + \uncover<4>{\alert<4>{\textbf{?} }} \uncover<5->{ \alert<5,6>{ \frac{3}{4}}} }} \\
\uncover<6->{&=&\displaystyle \sqrt{ \alert<6,7>{ \frac{3}{4}}\left( \alert<6,8>{\frac{4}{3}} \left(x+\frac{1}{2}\right)^{\alert<8>{2}} +\alert<6>{ 1} \right)}}\\
\uncover<7->{&=&\displaystyle \alert<7>{\frac{\sqrt{3}}{2}} \sqrt{\left(  \alert<9>{\alert<8>{\frac{2}{\sqrt{3}}} \left( x+ \frac{1}{2} \right)}\right)^{\alert<8>{2}}+1}}\\
\uncover<9->{ &=&\displaystyle \frac{\sqrt{3}}{2} \sqrt{ {\alert<9>{u}}^2+1},}
\end{array}
$

\noindent \uncover<9->{ where $\displaystyle \alert<9>{u= \frac{2}{\sqrt{3}}\left( x+\frac{1}{2}\right)}  =\frac{2\sqrt{3}}{3}x +\frac{\sqrt{3}}{3} $.}
\end{example}
\vspace{5cm}
\end{frame}
%end module quadratic-radicals-linear-substitution-preparation-ex1
%%begin module quadratic-radicals-linear-substitution-preparation-ex2
\begin{frame}
Recall: linear substitution is subst. of the form $u=px+q$.
\begin{example}
Use linear subst. to transform $\sqrt{-2x^2+x+1}$ to multiple of $\sqrt{-u^2+1}$.

\noindent
$
\begin{array}{rcl}
\sqrt{\alertNoH{2}{ -2}x^2+x+\alertNoH{2}{1}}&=& \uncover<2->{ \sqrt{ \alertNoH{2}{-2} \left(x^2\alertNoH{2}{- \alertNoH{3 }{ \frac{ 1}{2}}} x \alertNoH{2}{-\frac{1}{2}}\right) }} \\
\uncover<3->{ &=& \sqrt{ -2 \left( \alertNoH{6}{ x^2- \alertNoH{3}{2\frac{1}{4}}x  + \fcAnswer{5}{ \frac{1}{16}}} \alertNoH{7}{-}  \fcAnswer{5}{\alertNoH{7}{ \frac{1}{16}}} \alertNoH{7}{-\frac{1}{2}}\right) }}\\
\uncover<6->{&=&\sqrt{\alertNoH{8,9}{-2} \left(\alertNoH{6}{ \left(x-\frac{1}{4}\right)^2} \alertNoH{7,8}{-\frac{9}{16}} \right)}} \\
\uncover<8->{&=&\sqrt{ \alertNoH{8,9,10}{ \frac{9}{8}}\left( \alertNoH{9}{-\alertNoH{11}{\frac{16}{9}} } \left(x- \frac{1}{4} \right)^{\alertNoH{11}{2}}+ \alertNoH{8}{1} \right)}}\\
\uncover<10->{&=&\alertNoH{10}{ \frac{3}{\sqrt{8}}} \sqrt{- \left(\alertNoH{12}{ \alertNoH{11}{\frac{4}{3}} \left(x-\frac{1}{4}\right)}\right)^{\alertNoH{11}{2}}+1 }}\\
\uncover<12->{&=&\frac{ 3}{\sqrt{8}} \sqrt{-{\alertNoH{12}{ u}}^2+1},}
\end{array}
$

\noindent \uncover<12->{where $\alertNoH{12}{u=\frac{4}{3}\left(x-\frac{1}{4}\right) } =\frac{4}{3}x-\frac{1}{3}$.}
\end{example}
\end{frame}
%end module quadratic-radicals-linear-substitution-preparation-ex2

%%begin module area-under-hyperbola-ex1


\begin{frame}
\begin{example}
Find the area locked b-n the hyperbolas $\alert<2,3>{ y=\pm \sqrt{ x^2+1}}$ and $x=\pm 2\sqrt{ 2}$.
\begin{columns}
\column{.5\textwidth}
\psset{xunit=0.7cm, yunit=0.7cm}
\begin{pspicture}(-3.328427, -3)(3.328427,3)
\psframe*[linecolor=white](-3.328427,-3)(3.328427,3)
\tiny
\uncover<31->{
\pscustom*[linecolor=\fcColorAreaUnderGraph]{
\psplot[linecolor=\fcColorGraph, plotpoints = 1000 ] {-2.828427} {2.828427}{1 x 2 exp add 0.5 exp }
\psline[linecolor=\fcColorGraph](2.828427,-3)(2.828427,3)
\psplot[linecolor=\fcColorGraph, plotpoints=1000] { 2.828427 } {-2.828427}{1 x 2 exp add 0.5 exp -1 mul }
\psline[linecolor=\fcColorGraph](-2.828427,-3)(-2.828427,3)
}
}
\uncover<1-26,28->{
\psaxes[arrows=<->,ticks=none, labels=none](0,0)(-3,-3)(3,3)
}
\psline[linecolor=red!1](3.301,2)(3.302,2)
\psline[linecolor=red!1](-3.301,2)(-3.302,2)

%Function formula: - (x^{2}+1)^{1/2}
\psplot[linecolor=\fcColorGraph, plotpoints=1000]{-2.828427}{2.828427}{1 x 2 exp add 0.5 exp -1 mul }
\uncover<3-4>{\rput[tl](-2.2, -2.4){ \alert<3>{ $y= - \sqrt{ x^2 +1 }$}}}

%Function formula: (x^{2}+1)^{1/2}
\psplot[linecolor=\fcColorGraph, plotpoints=1000]{-2.828427}{ 2.828427 }{1 x 2 exp add 0.5 exp }
\uncover<2-4>{\rput[bl](-2.1, 2.4){\alert<2>{ $y=\sqrt{ x^2 +1} $}}}

\uncover<29->{
\psline[linecolor=\fcColorGraph](-2.828427,3)(-2.828427,-3)
}
\uncover<30->{
\psline[linecolor=\fcColorGraph](2.828427,3)(2.828427,-3)
}
\uncover<25-27>{
\psline{<->}(-2.9,2.9)(2.9,-2.9)
\rput[t](-2.1, 1.7){$\begin{array}{l} \alert<25>{v=0} \\\uncover<1-26>{\alert<25>{y+x=0}} \end{array}$}
}
\uncover<15-27>{
\psline{<->}(-2.9,-2.9)(2.9,2.9)
\rput[b](-2.1, -1.9){$\begin{array}{l} \uncover<1-26>{ \alert<15>{ y-x=0 }}\\\uncover<16->{\alert<16>{u=0}} \end{array}$}
}
\uncover<17-26>{
\fcFullDot{1.4}{1.4}
\rput[l]( 1.6, 1.4){$(\frac{y+x}{2},\frac{y+x}{2})$}
}
\uncover<14-26>{
\fcFullDot{0.6}{2.2}
\rput[lb](0.65, 2.2){$(x,y)$}
}
\uncover<26>{
\psline(0.6,2.2)(-0.8,0.8)
\psline(-0.7, 0.9)(-0.6, 0.8)(-0.7, 0.7)
\rput[rb](-0.3, 1.3){\alert<26>{$v$}}
}
\uncover<18-26>{
\psline(0.6,2.2)(1.4, 1.4)
\psline(1.3, 1.5)(1.2,1.4)(1.3, 1.3)
}
\uncover<23-26>{
\rput[tr](0.95, 1.8){\alert<23>{$u$}}
}
\uncover<14-26>{
\fcFullDot{2.2}{0.6}
\rput[lt]( 2.2, 0.65){$(y,x)$}
}
\end{pspicture}

\vbox to 3.0cm {
\uncover<18->{\alert<18>{
\uncover<22->{\alert<22>{Signed}} distance b-n $(x,y)$ and line $u=0$ equals}}
\only<1-23>{
$\uncover<19->{\uncover<22->{\alert<22>{\pm}} \alert<19>{ \sqrt{ \alert<20>{ \left(x-\frac{(x+y)}{2} \right)^2+ \left( y- \frac{(x+y )}{2} \right)^2}}}}
$
$\uncover<20->{=\uncover<22->{\alert<22>{\pm}} \sqrt{ \alert<20>{ \frac{1}{2}(y-x)^2 }}} \uncover<21->{= \alert<21>{ \uncover<1-21>{\pm} \alert<23>{ \frac{\sqrt{2 }}{ 2 } ( y-x)}}} \uncover<23>{ \alert<23>{=}}$
} %only<1-23>
\uncover<23->{ \alert<23,24>{$u $}.}
\only<24->{\uncover<25->{
Similarly compute that \alert<26>{signed distance b-n $(x,y)$ and the \alert<25>{line $v=0$} equals $v$}.
\uncover<27->{$\Rightarrow$ $y^2-x^2=1$ is the \alert<27>{ hyperbola $v=\frac{1/2}{v}$} in the $(u,v)$-plane.}
}}

\vfil
} %vbox

\column {.5\textwidth}
\only<1-27>{
\uncover<4->{We studied $\alert<27>{v=\frac{1/2}{u}}$ is called a hyperbola:}\uncover<3->{ why do we call $y= \sqrt{ x^2 +1}$ hyperbola?} \uncover<5->{Compute:}
\[
\begin{array}{rcl}
\uncover<5->{\sqrt{x^2+1} &=& y}\\
\uncover<6->{ x^2+1 &=& y^2}\\
\uncover<7->{y^2-x^2&=&1}\\
\uncover<8->{\uncover<9>{\alert<9>{\frac{1}{2}}} \uncover<10->{\alert<10,11>{\frac{\sqrt{2}}{2}}} \alert<11>{(y-x)} \uncover<10->{\alert<10,12>{\frac{\sqrt{2}}{2}}} \alert<12>{(y+x)}&=&\uncover<9->{\alert<9>{\frac{1}{2}}} \uncover<8>{1}}\\
\uncover<11->{\alert<11>{u}\alert<12>{v}&=& \frac{1}{2}}\\
\uncover<13->{\alert<27>{v}&\alert<27>{=}& \alert<27>{\frac{1/2}{u}},}
\end{array}
\]
\uncover<11->{where $\begin{array}{|l}
\alert<11,16,23>{u=\frac{\sqrt{2}}{2} \left(y-x\right)}\\
\alert<12,25>{v=\frac{\sqrt{2}}{2}\left(y+x\right)}
\end{array}$. } \uncover<14->{Consider an arbitrary point $(x,y)$.}
} %only<1-27>
\only<28->{
The area in question is:
$
\begin{array}{l}
\displaystyle\phantom{=} \int \limits^{{{\uncover<28,29>{\alert<29>{ \textbf{?}}}\uncover<30->{\alert<30>{ 2\sqrt{2}}}}}}_{\uncover<28>{\alert<28>{\textbf{?}}}\uncover<29->{ -2\sqrt{2}}} 2\sqrt{x^2+1}\diff x \\
\displaystyle \uncover<32->{= \uncover<33->{\alert<33>{2}} \left[x\sqrt{x^2+1} \vphantom{\ln \left(\sqrt{x^2+1}+x\right) }\right.}\\
\displaystyle \uncover<32->{\left. \ln \left(\sqrt{x^2+1}+x\right)\right]^{2\sqrt{2}}_{\only<33->{\alert<33>{0}} \uncover<1-32>{-2\sqrt{2}}}}\\
\uncover<34->{=2\left(2\sqrt{2} \sqrt{(2\sqrt{2})^2+1}\right.} \\
\uncover<34->{\left.+ \ln \left(\sqrt{(2\sqrt{2})^2+1}+2\sqrt{2} \right) \right)}\\
\uncover<35->{=12\sqrt{2} +2\ln \left(3+2\sqrt{2}\right )}\\
\uncover<36->{\approx 20.496}
\end{array}
$
}
\end{columns}

\end{example}

\end{frame}

%end module area-under-hyperbola-ex1

%% begin module trig-substitutions-ex2
\begin{frame}
\begin{example}
%\begin{columns}[c]
%\column{.4\textwidth}
Find the area enclosed by the ellipse $\frac{x^2}{a^2} + \frac{y^2}{b^2} = 1$, $a,b>0$.
\begin{columns}
\column{0.37\textwidth}
\psset{xunit=1cm, yunit=1cm}
\begin{pspicture}(-2.35, -1.6)(2.35,1.7)
\tiny
\psframe*[linecolor=white](-2.35, -1.6)(2.35,1.7)
\uncover<8>{ \pscustom*[linecolor=\fcColorAreaUnderGraph]{
\psplot[linecolor=\fcColorGraph, plotpoints=1000]{-2.000000}{2.000000}{1 x 2 exp -0.25 mul add sqrt -1 mul }
\psplot[linecolor=\fcColorGraph, plotpoints=1000]{-2.000000}{2.000000}{1 x 2 exp -0.25 mul add sqrt }
}
}
\uncover<9-11>{ \pscustom*[linecolor=\fcColorAreaUnderGraph]{
\psplot[linecolor=\fcColorGraph, plotpoints=1000]{-2.000000}{2.000000}{1 x 2 exp -0.25 mul add sqrt }
\psline(2,0)(-2,0)
}
}
\uncover<12->{ \pscustom*[linecolor=\fcColorAreaUnderGraph]{
\psplot[linecolor=\fcColorGraph, plotpoints=1000]{0}{2.000000}{1 x 2 exp -0.25 mul add sqrt }
\psline(2,0)(0,0)
}
}

\fcAxesStandardNoFrame{-2.2}{-1.6}{2.2}{1.6}
\rput[b](1,1){\uncover<6->{\alertNoH{6}{$y=b\sqrt{1-\frac{x^2}{a^2}} $}}}
\rput[t](1,-1){\uncover<7->{\alertNoH{7}{$y=-b\sqrt{1-\frac{x^2}{a^2}} $}}}

\uncover<11->{
\fcFullDot{2}{0}
\rput[rb](1.9,0.1){$(a,0)$}
\fcFullDot{-2}{0}
\rput[lb](-1.9,0.1){$(-a,0)$}
}

%Function formula: - \sqrt{-1/4 x^{2}+1}
\uncover<7->{ \psplot[linecolor=\fcColorGraph, plotpoints=1000]{-2.000000}{2.000000}{1 x 2 exp -0.25 mul add sqrt -1 mul } }
%Function formula: \sqrt{-1/4 x^{2}+1}
\uncover<6->{
\psplot[linecolor=\fcColorGraph, plotpoints=1000]{-2.000000}{2.000000}{1 x 2 exp -0.25 mul add sqrt }
}
\end{pspicture}
\vbox to 4cm {
\uncover<8->{The area in question is }
$\begin{array}{l}
\displaystyle \uncover<9->{\phantom{=}\phantom{4} \int\limits^{ { \fcAnswerUncover{9}{11}{a} } }_{{ \fcAnswerUncover{9}{11}{ -a}} } \alertNoH{9}{2} b\sqrt{ 1- \frac{ x^2}{ a^2} }\diff x} \\
\displaystyle\uncover<12->{=\alertNoH{12}{4}\int \limits_{{ \alertNoH{12}{0}}}^{a}  b\sqrt{1-\frac{x^2}{a^2}}\diff x} \uncover<12->{.}
\end{array}
$

\vfil
} %vbox
\column{0.63\textwidth}
\uncover<2->{Express $y$ via $x$:}
$
\begin{array}{rcl}
\displaystyle \uncover<2->{\alertNoH{3}{\frac{x^2}{a^2}} + \frac{y^2}{b^2} &=& \displaystyle 1} \\
\displaystyle \uncover<3->{\frac{y^2}{\alertNoH{4}{ b^2} }&=&\displaystyle 1 \alertNoH{3}{-\frac{x^2}{a^2}}}\\
\displaystyle \uncover<4->{y^2&=&\displaystyle \alertNoH{4}{b^2}\left(1-\frac{x^2}{a^2}\right)}\\
\displaystyle \uncover<5->{y&=&\displaystyle \pm b\sqrt{1-\frac{x^2}{a^2}}}
\end{array}
$
\end{columns}
\end{example}

\vspace{10cm}
\end{frame}

\begin{frame}
\begin{example} 
%\begin{columns}[c]
%\column{.4\textwidth}
Find the area enclosed by the ellipse $\frac{x^2}{a^2} + \frac{y^2}{b^2} = 1$, $a,b>0$.
\begin{columns}
\column{0.37\textwidth}
\psset{xunit=1cm, yunit=1cm}
\begin{pspicture}(-2.35, -1.6)(2.35,1.7)
\tiny
\psframe*[linecolor=white](-2.35, -1.6)(2.35,1.7)
\pscustom*[linecolor=\fcColorAreaUnderGraph]{
\psplot[linecolor=\fcColorGraph, plotpoints=1000]{-2.000000}{2.000000}{1 x 2 exp -0.25 mul add sqrt -1 mul }
\psplot[linecolor=\fcColorGraph, plotpoints=1000]{-2.000000}{2.000000}{1 x 2 exp -0.25 mul add sqrt }
}
\fcAxesStandardNoFrame{-2.2}{-1.6}{2.2}{1.6}
\rput[b](1,1){\uncover<1->{$y=b\sqrt{1-\frac{x^2}{a^2}}$}}
\rput[t](1,-1){$y=-b\sqrt{1-\frac{x^2}{a^2}} $}

%Function formula: - \sqrt{-1/4 x^{2}+1}
\psplot[linecolor=\fcColorGraph, plotpoints=1000]{-2.000000}{2.000000}{1 x 2 exp -0.25 mul add sqrt -1 mul }
%Function formula: \sqrt{-1/4 x^{2}+1}
\psplot[linecolor=\fcColorGraph, plotpoints=1000]{-2.000000}{2.000000}{1 x 2 exp -0.25 mul add sqrt }
\end{pspicture}
\vbox to 4cm {
The area in question is
$\begin{array}{l}
\displaystyle \phantom{=}\phantom{4} \int \limits_{ \vphantom{ \textbf{?}}-a}^{ \vphantom{\textbf{?}}a}2b\sqrt{ 1-\frac{x^2}{ a^2} }\diff x\\
\uncover<1->{=} \displaystyle 4\alertNoH{15}{ \int \limits_{ 0}^{ a}}  b \alertNoH{15}{ \sqrt{ 1-\frac{ x^2}{ a^2}} \diff x} \\
\uncover<15->{=4b\alertNoH{15}{ \frac{a\pi}{4}}}\uncover<16->{=\pi a b\quad .}
\end{array}
$

\vfil
}
\column{0.63\textwidth}
\uncover<2->{Trig subst.: set $x= a\sin \theta $, $\alertNoH{ 5}{ \theta\in \left(0,\frac{\pi}{2}\right)} $.} \uncover<3->{Compute: $\alertNoH{6}{ \sqrt{1-\frac{ { \alertNoH{3 }{x}}^2}{a^2}}} =\sqrt{1 -\frac{\alertNoH{4}{ {\alertNoH{3}{a}}^2} {\alertNoH{3}{\sin}}^2\alertNoH{3}{ \theta} }{\alertNoH{4}{ a^2}} } \uncover<4->{ =\sqrt{1- \sin^2\theta}} \uncover<5->{\alertNoH{5,6}{=\cos\theta}}$.} \uncover<7->{When $x=0$, $\theta=0$ and when $x=a$, $\theta=\frac{\pi}{2}$.}

\noindent $
\begin{array}{rcl}
\displaystyle
\alertNoH{15}{ \int_{0}^{a}\sqrt{1-\frac{x^2}{a^2}}\diff x} &\uncover<6->{\alertNoH{6}{=}}& \displaystyle \uncover<6->{ \int_{{\uncover<7->{\alertNoH{7}{0}}}}^{\uncover<7->{{\alertNoH{7}{ \frac{\pi}2}}}} \alertNoH{6}{\cos \theta} ~ \alertNoH{9}{ \diff \left(\alertNoH{8}{a} \sin \theta\right)} }\\
&\uncover<8->{=}& \displaystyle \uncover<8->{ \alertNoH{8}{a} \int_{0}^{\frac{\pi}2} \alertNoH{10}{ {\alertNoH{9}{\cos}}^2 \alertNoH{9}{\theta}} \alertNoH{9}{ \diff \theta} } \\
&\uncover<10->{=}&\displaystyle \uncover<10->{ a\int_{ 0}^{ \frac{\pi}2} \alertNoH{10}{\frac{ \alertNoH{11}{\cos(2\theta)} + \alertNoH{12}{1}}{ \alertNoH{ 11, 12}{ 2} } } \diff \theta}\\
&\uncover<11->{=}& \uncover<11->{ a \left[  \alertNoH{11}{ \frac{ \sin (2\theta)}{4}} +\alertNoH{12}{ \frac{\theta}{2}} \right]_{\theta=0}^{\theta=\frac{\pi}{2}}} \\
&\uncover<13->{=}&\uncover<13->{ a\left(0+\frac{\pi}{4}-(0+ 0) \right) }\\
\uncover<14->{&\alertNoH{15}{=}& \alertNoH{15}{\frac{a\pi }{4}}}
\end{array}
$
\end{columns}
\end{example}

\vspace{10cm}
\end{frame}
% end module trig-substitutions-ex2

%%begin module area-under-hyperbola-ex1


\begin{frame}
\begin{example}
Find the area locked b-n the hyperbolas $\alert<2,3>{ y=\pm \sqrt{ x^2+1}}$ and $x=\pm 2\sqrt{ 2}$.
\begin{columns}
\column{.5\textwidth}
\psset{xunit=0.7cm, yunit=0.7cm}
\begin{pspicture}(-3.328427, -3)(3.328427,3)
\psframe*[linecolor=white](-3.328427,-3)(3.328427,3)
\tiny
\uncover<31->{
\pscustom*[linecolor=\fcColorAreaUnderGraph]{
\psplot[linecolor=\fcColorGraph, plotpoints = 1000 ] {-2.828427} {2.828427}{1 x 2 exp add 0.5 exp }
\psline[linecolor=\fcColorGraph](2.828427,-3)(2.828427,3)
\psplot[linecolor=\fcColorGraph, plotpoints=1000] { 2.828427 } {-2.828427}{1 x 2 exp add 0.5 exp -1 mul }
\psline[linecolor=\fcColorGraph](-2.828427,-3)(-2.828427,3)
}
}
\uncover<1-26,28->{
\psaxes[arrows=<->,ticks=none, labels=none](0,0)(-3,-3)(3,3)
}
\psline[linecolor=red!1](3.301,2)(3.302,2)
\psline[linecolor=red!1](-3.301,2)(-3.302,2)

%Function formula: - (x^{2}+1)^{1/2}
\psplot[linecolor=\fcColorGraph, plotpoints=1000]{-2.828427}{2.828427}{1 x 2 exp add 0.5 exp -1 mul }
\uncover<3-4>{\rput[tl](-2.2, -2.4){ \alert<3>{ $y= - \sqrt{ x^2 +1 }$}}}

%Function formula: (x^{2}+1)^{1/2}
\psplot[linecolor=\fcColorGraph, plotpoints=1000]{-2.828427}{ 2.828427 }{1 x 2 exp add 0.5 exp }
\uncover<2-4>{\rput[bl](-2.1, 2.4){\alert<2>{ $y=\sqrt{ x^2 +1} $}}}

\uncover<29->{
\psline[linecolor=\fcColorGraph](-2.828427,3)(-2.828427,-3)
}
\uncover<30->{
\psline[linecolor=\fcColorGraph](2.828427,3)(2.828427,-3)
}
\uncover<25-27>{
\psline{<->}(-2.9,2.9)(2.9,-2.9)
\rput[t](-2.1, 1.7){$\begin{array}{l} \alert<25>{v=0} \\\uncover<1-26>{\alert<25>{y+x=0}} \end{array}$}
}
\uncover<15-27>{
\psline{<->}(-2.9,-2.9)(2.9,2.9)
\rput[b](-2.1, -1.9){$\begin{array}{l} \uncover<1-26>{ \alert<15>{ y-x=0 }}\\\uncover<16->{\alert<16>{u=0}} \end{array}$}
}
\uncover<17-26>{
\fcFullDot{1.4}{1.4}
\rput[l]( 1.6, 1.4){$(\frac{y+x}{2},\frac{y+x}{2})$}
}
\uncover<14-26>{
\fcFullDot{0.6}{2.2}
\rput[lb](0.65, 2.2){$(x,y)$}
}
\uncover<26>{
\psline(0.6,2.2)(-0.8,0.8)
\psline(-0.7, 0.9)(-0.6, 0.8)(-0.7, 0.7)
\rput[rb](-0.3, 1.3){\alert<26>{$v$}}
}
\uncover<18-26>{
\psline(0.6,2.2)(1.4, 1.4)
\psline(1.3, 1.5)(1.2,1.4)(1.3, 1.3)
}
\uncover<23-26>{
\rput[tr](0.95, 1.8){\alert<23>{$u$}}
}
\uncover<14-26>{
\fcFullDot{2.2}{0.6}
\rput[lt]( 2.2, 0.65){$(y,x)$}
}
\end{pspicture}

\vbox to 3.0cm {
\uncover<18->{\alert<18>{
\uncover<22->{\alert<22>{Signed}} distance b-n $(x,y)$ and line $u=0$ equals}}
\only<1-23>{
$\uncover<19->{\uncover<22->{\alert<22>{\pm}} \alert<19>{ \sqrt{ \alert<20>{ \left(x-\frac{(x+y)}{2} \right)^2+ \left( y- \frac{(x+y )}{2} \right)^2}}}}
$
$\uncover<20->{=\uncover<22->{\alert<22>{\pm}} \sqrt{ \alert<20>{ \frac{1}{2}(y-x)^2 }}} \uncover<21->{= \alert<21>{ \uncover<1-21>{\pm} \alert<23>{ \frac{\sqrt{2 }}{ 2 } ( y-x)}}} \uncover<23>{ \alert<23>{=}}$
} %only<1-23>
\uncover<23->{ \alert<23,24>{$u $}.}
\only<24->{\uncover<25->{
Similarly compute that \alert<26>{signed distance b-n $(x,y)$ and the \alert<25>{line $v=0$} equals $v$}.
\uncover<27->{$\Rightarrow$ $y^2-x^2=1$ is the \alert<27>{ hyperbola $v=\frac{1/2}{v}$} in the $(u,v)$-plane.}
}}

\vfil
} %vbox

\column {.5\textwidth}
\only<1-27>{
\uncover<4->{We studied $\alert<27>{v=\frac{1/2}{u}}$ is called a hyperbola:}\uncover<3->{ why do we call $y= \sqrt{ x^2 +1}$ hyperbola?} \uncover<5->{Compute:}
\[
\begin{array}{rcl}
\uncover<5->{\sqrt{x^2+1} &=& y}\\
\uncover<6->{ x^2+1 &=& y^2}\\
\uncover<7->{y^2-x^2&=&1}\\
\uncover<8->{\uncover<9>{\alert<9>{\frac{1}{2}}} \uncover<10->{\alert<10,11>{\frac{\sqrt{2}}{2}}} \alert<11>{(y-x)} \uncover<10->{\alert<10,12>{\frac{\sqrt{2}}{2}}} \alert<12>{(y+x)}&=&\uncover<9->{\alert<9>{\frac{1}{2}}} \uncover<8>{1}}\\
\uncover<11->{\alert<11>{u}\alert<12>{v}&=& \frac{1}{2}}\\
\uncover<13->{\alert<27>{v}&\alert<27>{=}& \alert<27>{\frac{1/2}{u}},}
\end{array}
\]
\uncover<11->{where $\begin{array}{|l}
\alert<11,16,23>{u=\frac{\sqrt{2}}{2} \left(y-x\right)}\\
\alert<12,25>{v=\frac{\sqrt{2}}{2}\left(y+x\right)}
\end{array}$. } \uncover<14->{Consider an arbitrary point $(x,y)$.}
} %only<1-27>
\only<28->{
The area in question is:
$
\begin{array}{l}
\displaystyle\phantom{=} \int \limits^{{{\uncover<28,29>{\alert<29>{ \textbf{?}}}\uncover<30->{\alert<30>{ 2\sqrt{2}}}}}}_{\uncover<28>{\alert<28>{\textbf{?}}}\uncover<29->{ -2\sqrt{2}}} 2\sqrt{x^2+1}\diff x \\
\displaystyle \uncover<32->{= \uncover<33->{\alert<33>{2}} \left[x\sqrt{x^2+1} \vphantom{\ln \left(\sqrt{x^2+1}+x\right) }\right.}\\
\displaystyle \uncover<32->{\left. \ln \left(\sqrt{x^2+1}+x\right)\right]^{2\sqrt{2}}_{\only<33->{\alert<33>{0}} \uncover<1-32>{-2\sqrt{2}}}}\\
\uncover<34->{=2\left(2\sqrt{2} \sqrt{(2\sqrt{2})^2+1}\right.} \\
\uncover<34->{\left.+ \ln \left(\sqrt{(2\sqrt{2})^2+1}+2\sqrt{2} \right) \right)}\\
\uncover<35->{=12\sqrt{2} +2\ln \left(3+2\sqrt{2}\right )}\\
\uncover<36->{\approx 20.496}
\end{array}
$
}
\end{columns}

\end{example}

\end{frame}

%end module area-under-hyperbola-ex1

%%begin module trig-substitutions-euler-substitutions-table

\begin{frame}
\begin{itemize}
\item Let $R$ be a rational function in two variables.
\item<2-> So far, with linear transformations we converted all integrals of the form $\displaystyle\int R(x, \sqrt{ax^2+bx+c})\diff x$ to one of the three forms:

\alertNoH{4,9}{$\int R(x, \sqrt{x^2+1})\diff x$}, \alertNoH{5,10}{$\int R(x, \sqrt{-x^2+1})\diff x$} , \alertNoH{6,11}{$\int R(x, \sqrt{x^2-1}) \diff x$}.
\item<3-> Each of the above integrals can be transformed to a rational trigonometric integral using 3 pairs of substitutions:

\alertNoH{4,9}{$x=\tan\theta $, $x=\cot \theta$;}
\alertNoH{5,10}{$x=\sin\theta $, $x=\cos \theta$;}
\alertNoH{6,11}{$x=\csc\theta $, $x=\sec \theta$.}
\item<7-> We studied that trigonometric integrals are converted to rational function integrals via $\theta=2\arctan t$.
\item<8-> The resulting 3 pairs of substitutions are called Euler substitutions:
\alertNoH{9,13}{$x=\tan (2\arctan t) $, $x=\cot (2\arctan t)$;}
\alertNoH{10,13}{$x=\sin(2\arctan t) $, $x=\cos (2\arctan t)$;}
\alertNoH{11,13}{$x=\csc(2\arctan t) $, $x=\sec (2\arctan t)$.}
\item<12-> The Euler substitutions directly transform the integral to a rational function integral.
\item<13-> We will demonstrate that the Euler substitutions are \alertNoH{13}{rational}.
\end{itemize}

\end{frame}

\begin{frame}
\frametitle{Trigonometric substitution and Euler substitution}
{\tabcolsep=0.11cm
\noindent\begin{tabular}{|l|l|l|r|}
\hline
Expression & Substitution& Variable range & Relevant identity\\\hline
\multirow{2}{*}{$\sqrt{x^2+1}$} & $x = \tan \theta$ &  $ \theta\in \left(-\frac{\pi}{2} , \frac{\pi}{2}\right)$ & $1 + \tan^2 \theta = \sec^2 \theta$\\
&$x=\cot \theta$ &$ \theta\in (0, \pi) $ & $1+\cot^2\theta =\csc^2\theta $ \\ \hline
\multirow{2}{*}{ $\sqrt{-x^2+1 }$} & $x = \sin \theta$ &  $ \theta\in \left[ -\frac{\pi}{2} ,\frac{\pi}{2}\right]$ & $1 - \sin^2 \theta = \cos^2 \theta$\\
& $x = \cos \theta$ & $\theta\in (0,\pi)$& $1-\cos^2\theta=\cos^2\theta$ \\\hline
\multirow{2}{*}{$\sqrt{x^2-1}$} &$x=\csc \theta$ &$\theta\in \left[0, \frac{\pi}{2} \right) \cup \left[ \pi, \frac{3\pi}{2}\right)$ &  $\csc^2\theta-1=\cot^2\theta $ \\
&$x = \sec \theta$ &
$\theta\in \left[0, \frac{\pi}{2}\right)\cup \left[\pi, \frac{ 3 \pi}{2}\right)$
& $\sec^2\theta - 1 = \tan^2\theta$
\\
\hline
\multicolumn{4}{c}{Euler substitution by applying in addition $\theta=2\arctan t$}\\
\hline
\multirow{2}{*}{$\sqrt{x^2+1}$} & $ x =\frac{2t}{1-t^2}$ & $-1< t< 1$ & (?) \\
&$ x=\frac{1}{2} \left(\frac{1}{t}-t\right)$ & $0<t $ &  (?)\\ \hline
\multirow{2}{*}{ $\sqrt{-x^2+1 }$} & $x=\frac{2t}{1+t^2} $ & $-1\leq t\leq 1 $ & (?)\\
& $x =\frac{1-t^2}{1+t^2} $ & $0<t$&  (?)\\\hline
\multirow{2}{*}{ $\sqrt{x^2-1}$} & $x=\frac{1}{2}\left(\frac{ 1}{t}+t\right)$ & $t\in (-\infty, -1)\cup [0,1)$&(?)\\
& $x =\frac{1+t^2}{1-t^2} $ & $t \in (-\infty,-1)\cup [0,1)$ & (?)\\\hline
\end{tabular}
}
\end{frame}
%end module trig-substitutions-euler-substitutions-table


%%begin module trig-substitution-case-1-cot
\begin{frame}
\frametitle{Trigonometric substitution $x=\cot \theta$  for $\sqrt{ x^2+1}$}
The trigonometric substitution $ \alertNoH{2,13}{x=\cot \theta}$, $\theta\in \left(0 , \pi\right) $ for $\sqrt{x^2+1}$:
\[
\begin{array}{rclll}
\displaystyle  \alertNoH{8,9,12}{\sqrt{\alertNoH{2}{x}^2+1}}
\vphantom{\frac{1}{\sin \theta}}
\only<handout:1|1-8>{
&=&\displaystyle \uncover<2->{ \sqrt{\alertNoH{3}{\alertNoH{2}{\cot}^2 \alertNoH{2}{ \theta} } +1} }\\
\uncover<3->{&=&\displaystyle \sqrt{\alertNoH{4}{\alertNoH{3}{\frac{\cos^2 \theta }{ \sin^2 \theta}} + 1} }} \\
\uncover<4->{&=&\displaystyle \sqrt{ \alertNoH{4}{\frac{\alertNoH{5}{ \cos^2 \theta+ \sin^2\theta}}{ \sin^2 \theta}}}} \\
\uncover<5->{&=& \displaystyle  \sqrt{\frac{\alertNoH{5}{ 1} }{ \sin^2 \theta}}=\frac{1}{\alertNoH{6}{\sqrt{\sin^2\theta}}} } \uncover<6->{ &&
\begin{array}{|l}\displaystyle \text{when }\theta\in \left(0 , \pi\right) \text{ we have }\\ ~ \sin \theta \geq 0\text{ and so } \\ \alertNoH{6}{ \sqrt{\sin^2 \theta}=\sin\theta}  \end{array} }
\\
} %only<1-8>
\uncover<6->{&\alertNoH{8,9,12}{ =}& \displaystyle  \alertNoH{8,9,12}{\frac{1}{\alertNoH{6}{ \sin \theta} }}} \uncover<7->{\alertNoH{8,9,12}{= \csc \theta \quad . }} && {~~~~~~~~~~~~~~~~~~} {~~~~~~~~~~~~~~~~~~~~} {~~~~~~~~~~~~~~~~~~~} %white space flushes formulas to the left
\end{array}
\]
\uncover<handout:2|10->{
The differential $\diff x$ can be expressed via $\diff \theta$ from $x=\cot \theta$. \uncover<11->{
To summarize:
\begin{definition}The trigonometric substitution $\alertNoH{13}{ x=\cot \theta }$, $\theta\in (0,\pi)$ for $\sqrt{x^2+1} $ is given by:
\[
\begin{array}{rcl}
x &=&\displaystyle \cot \theta \\
\alertNoH{12}{\sqrt{x^2+1}}&\alertNoH{12}{=}&\displaystyle \alertNoH{12}{\frac{1}{\sin \theta}=\csc \theta}\\
\alertNoH{14,15}{\diff x} &\alertNoH{14,15}{=}&\displaystyle \uncover<15->{\alertNoH{15}{ -\frac{\diff \theta}{\sin^2\theta} = - \csc^2 \theta} } \uncover<1-14>{\alertNoH{14}{\textbf{?}}} \alertNoH{14,15}{\diff \theta}\\
\alertNoH{13}{\theta}& \alertNoH{13}{=}& \alertNoH{13}{\Arccot x}\quad .
\end{array}
\]
\end{definition}
}
}

\vspace{10cm}
\end{frame}
%end module trig-substitution-case-1-cot

%%begin module Euler-substitution-case-1-cot
\begin{frame}
\frametitle{Trigonometric substitution $x=\cot \theta$  for $\sqrt{x^2+1}$}
The trigonometric substitution $x=\cot \theta$ is given by 
\[
\begin{array}{rcll|l}
\displaystyle \sqrt{ x^2+1}&=&\displaystyle \sqrt{\cot^2 \theta+1}\\
&=&\displaystyle \sqrt{\frac{\cos^2\theta}{ \sin^2 \theta} + 1}\\
&=&\displaystyle \sqrt{ \frac{ \sin^2\theta+\cos^2 \theta}{ \sin^2 \theta}} \\
&=& \displaystyle  \sqrt{\frac{1}{\sin^2\theta}} && \begin{array}{l}\displaystyle \text{when }\theta\in \left(0 , \pi\right) \text{ we have }\\ ~ \sin \theta \geq 0\text{ and so } \sqrt{\sin^2 \theta}=\sin\theta  \end{array}\\
&=&\displaystyle  \frac{1}{\sin \theta}= \csc \theta\quad .
\end{array}
\]
The differential $\diff x$ can be expressed via $\diff \theta$ from $x=\cot \theta$. The substitution $x=\cot \theta$ can be now summarized as:
\[
\begin{array}{rcl}
x&=&\displaystyle \cot \theta\\
\sqrt{x^2+1}&=&\displaystyle \frac{1}{\sin \theta}=\csc \theta\\
\diff x&=&\displaystyle  -\frac{\diff \theta}{\sin^2\theta} = - \csc^2 \theta \diff \theta\\
%\theta& =& \Arccot x\quad .
\end{array}
\]
\end{frame}

%end module Euler-substitution-case-1-cot
%\subsubsection{Trigonometric substitution $x=\cos \theta$, $\theta\in\left[0, \pi\right] $}
The trigonometric substitution $x=\cos \theta$ is given by 
\[
\begin{array}{rcll|l}
\displaystyle \sqrt{-x^2+1}&=&\displaystyle \sqrt{1-\cos^2\theta}\\
&=&\displaystyle \sqrt{\sin^2\theta} &&\begin{array}{l} \text{when }\theta\in\left[-\frac{\pi}{2}, \frac{\pi}{2} \right] \text{we have}\\
\sin \theta\geq 0 \text{ and so } \sqrt{\sin^2\theta}=\sin \theta
\end{array} \\
&=&\displaystyle \sin \theta\quad .
\end{array}
\]
The differential $\diff x$ can be expressed via $\diff \theta$ from $x=\cos \theta$. The substitution $x=\cos \theta $ can be now summarized as:
\begin{equation*}
\begin{array}{rcl}
x&=&\cos \theta\\
\sqrt{-x^2+1}&=&\cos \theta\\
\diff x&=& -\sin \theta \diff \theta\\
\theta&=&\arccos x \quad .
\end{array}
\end{equation*}
%\begin{frame}
\frametitle{Euler subst. for $\sqrt{-x^2+1}$ corresponding to $x=\cos \theta$ }
\begin{itemize}
\item $\alert<4>{ x =\cos \theta}$ transforms $\diff x, x,\sqrt{-x^2+1}$ to trig form.
\item $\alert<5>{\theta=2\Arctan t}$, \uncover<3->{ $t>0$} transforms $\diff \theta, \cos\theta,\sin \theta$ to rational form.
\end{itemize}
\uncover<2->{\alert<2>{What if we compose the above?}} \uncover<3->{\alert<3>{We get the Euler substitution:}}
\only<1-37>{ %
\[
\begin{array}{rclll}
\uncover<3->{\alert<4,13,22,33,37>{x}}&\uncover<3->{\alert<4,13,22,33,37>{=}}&\displaystyle \vphantom{\frac{1- t^2}{ 1+ t^2} } 
\only<1-10>{
\displaystyle \uncover<4->{\alert<4>{ \cos \alert<5>{\theta}} } \\
\uncover<5->{ &=&\displaystyle \alert<8>{ \cos (\alert<5>{2\arctan t})}} \uncover<6->{&&\begin{array}{|l}
\displaystyle \displaystyle \alert<6,7,8>{\cos (2z) =}\uncover<7->{\alert<7,8>{ \frac{ 1-\tan^2 z }{1+ \tan^2 z}}}
\end{array}}
\\
\uncover<8->{ &=&\displaystyle \alert<8>{ \frac{1- {\alert<9>{\tan}}^2 ( \alert<9>{\Arctan t})}{1+{\alert<9>{\tan}}^2(\alert<9>{\Arctan t})} } } \\
\uncover<9->{&=&}
} %only<1-10>
\uncover<9->{\displaystyle \alert<10,11,13,22,33,37>{ \frac{\alert<22>{ 1- { \alert<9>{t}}^2}}{\alert<23>{ 1+ { \alert<9>{t}}^2}} }  &&{~~~~~~~~~~~~~~~~~~~~~~~~~~~~~~~~~~~~~~~~~~~~~~~~~~~~~~~~~~~~~~~~~~~~~~~~~~~~~~~~~} %whitespace flushes formulas left
} 
\uncover<12->{ \\\hline}
\uncover<12->{\alert<37>{ \sqrt{- {\alert<13>{x}}^2+1 }}}&\uncover<12->{\alert<37>{=}} &\displaystyle 
\only<1-20>{
\uncover<13->{ \sqrt{\alert<14>{1} - \left(\alert<13>{ \frac{1-t^2}{ 1+ t^2 }} \right)^2}}\\
\uncover<14->{&=&\displaystyle \sqrt{\frac{ \alert<15>{ \alert<14,17>{ (1+ t^2)^2} -(1-t^2)^2} }{ \alert<14>{(1+t^2)^2}} }}&&
\uncover<15->{\begin{array}{|l}
\alert<15,16>{(1+t^2)^2-(1-t^2)^2=}\only<15>{\alert<15>{\textbf{?}}} \uncover<16->{\alert<16>{4t^2}}}
\end{array} 
\\
\uncover<17->{&=&\displaystyle  \sqrt{\frac{ \alert<17>{4t^2} }{ (1+t^2)^2}}} \uncover<18->{&& 
\begin{array}{|l}
\displaystyle \alert<18>{ \alert<19>{\sqrt{4t^2}=2t} \text{ because } t>0} }
\end{array}
\\
\uncover<19->{&=&}
} %only<1-20>
\uncover<19->{ \displaystyle \alert<20,21,37>{ \frac{{2t} }{1+t^2}} \vphantom{ \sqrt{1 - \left( \frac{1-t^2}{ 1+ t^2 } \right)^2}} }\uncover<22->{\\\hline }
\only<1-30>{
\uncover<22->{ \displaystyle \alert<23>{ (\alert<25>{1}+ \alert<24>{t^2})} \alert<22,24,25>{ x} &=&\displaystyle  \alert<22>{\alert<25>{1} -\alert<24>{t^2}}}\\
\uncover<24->{ \displaystyle \alert<24>{ t^2(\alert<26>{x+1})}&=&\displaystyle \alert<25>{ 1-x}}\\
\uncover<26->{ \displaystyle t^2&=&\displaystyle \frac{1-x}{\alert<26>{1+x}}}\\
}
\uncover<27->{ \displaystyle \alert<30,31,37>{t}} &\alert<30,31,37>{\uncover<27->{=}}&\displaystyle \only<27-30>{\uncover<27->{ \frac{ \alert<29>{ \sqrt{1 -x }}}{\sqrt{ 1+x}}} \uncover<28->{ \frac{ \alert<29>{  \sqrt{1+x}} }{\sqrt{1+x}} }}\only<29-30>{=} \uncover<29->{ \alert<30,31,37>{ \frac{ \alert<29>{ \sqrt{-x^2+1} }}{x+1}}} && 
\uncover<27-30>{
\begin{array}{l}\text{here we use } t>0
\end{array}
} 
\uncover<31->{ \\\hline}
\uncover<32->{\alert<37>{ \diff \alert<33>{x}}} &\uncover<33->{=} & \displaystyle \vphantom{\diff \left( \frac{ 1- t^2}{ 1+ t^2}\right)} \only<33->{\diff \left( \alert<33>{\frac{ 1- t^2}{ 1+ t^2}}\right)} 
\only<34->{=\diff\left(\frac{2-\alert<35>{ (1+t^2)} }{ \alert<35>{ 1 + t^2}} \right)}\\
&\uncover<35->{=}&\only<35->{\displaystyle \diff\left(\frac{2}{1+ t^2} \alert<35>{-1} \right)}
\only<36->{\alert<37>{=-\frac{4t}{(1+t^2)^2}\diff t}}
\end{array}
\]
} %only<1-37>
\uncover<38->{
\begin{definition}
The Euler substitution for $\sqrt{x^2+1}$ corresponding to $x=\cos \theta$ is given by:
\[
\alert<38>{
\begin{array}{rcl}
x&=&\displaystyle \frac{1-t^2}{1+t^2}, \quad \quad t>0\\
\sqrt{-x^2+1}&=&\displaystyle \frac{2t}{1+t^2}  \\
\diff x&=&\displaystyle  -\frac{4 t}{(t^{2}+1)^{2}} \diff t\\
t&=&\displaystyle \frac{\sqrt{-x^2+1}}{x+1} \quad .
\end{array}
}
\]
\end{definition}

\vspace{7cm}
}
\end{frame}

%% begin module trig-substitution-case-3-sec

\begin{frame}
\frametitle{Trigonometric substitution $x=\sec \theta$ for $\sqrt{ x^2-1}$ }
The trigonometric substitution $ \alertNoH{10,12}{x=\sec \theta}$, $\theta \in \left[0, \frac{\pi}{2}\right)\cup \left[\pi, \frac{3\pi}{ 2} \right) $:
\[
\begin{array}{rclll}
\displaystyle \alertNoH{11}{\sqrt{x^2-1}}&\alertNoH{11}{=} & \displaystyle
\only<handout:1|1-8>
{\uncover<2->{ \sqrt{\sec^2\theta-1}}\\
\uncover<3->{&=& \displaystyle \sqrt{\frac{1}{\cos^2\theta}-1}}\\
\uncover<4->{&=&\displaystyle \sqrt{\frac{\sin^2\theta}{\cos^2\theta}}} \\
\uncover<5->{&=&\displaystyle \sqrt{ \tan^2 \theta}} &&\uncover<6->{ \begin{array}{|l} \text{when }\theta\in \theta \in \left[0, \frac{ \pi}{2 }\right)\cup \left[\pi, \frac{3\pi}{2}\right) \text{we have}\\
\tan \theta\geq 0 \text{ and so } \sqrt{\tan^2\theta}=\tan \theta
\end{array}} \\
\uncover<7->{&=&}
}
\uncover<7->{\displaystyle \alertNoH{8,9,11}{\tan \theta} \vphantom{\sqrt{\sec^2\theta-1}}\quad .&&{~~~~~~~~~~~~~~~~~~~~~~~~~~~~~~~~~~~~~~~~~~~~~~~~~~~~~~~~~~~~~~~~~~~~~~~~~~~~~~~~~~~~~~~~~~~~~~~~~~~~~~} }
\end{array}
\]
\uncover<handout:2|10->{
\begin{definition}The trigonometric substitution $\alertNoH{10,12}{ x=\sec \theta }$, $\theta\in (0,\pi)$ for $\sqrt{x^2+1} $ is given by:
\[
\begin{array}{rcl}
\displaystyle \alertNoH{10,13}{ x}&\alertNoH{10,13}{=}& \displaystyle \alertNoH{10,13}{\sec\theta= \frac{1}{\cos \theta} } \quad \quad \theta \in \left[0, \frac{\vphantom{3} \pi}{2}\right)\cup \left[\pi, \frac{ 3 \pi}{ 2} \right)\\
\displaystyle \alertNoH{11}{ \sqrt{x^2-1}}&\alertNoH{11}{ =}& \displaystyle \alertNoH{11}{ \tan \theta}\\
\displaystyle \alertNoH{13,14}{\diff x}& \alertNoH{13,14}{=} & \displaystyle \fcAnswerUncover{10}{14}{ \frac{\sin\theta}{ \cos^2\theta} \diff \theta= \sec\theta\tan\theta }  \alertNoH{14}{ \diff \theta} \\
\displaystyle \alertNoH{12}{\theta}&\alertNoH{12}{=} &\alertNoH{12}{ \Arcsec x} \quad .
\end{array}
\]
\end{definition}
}

\vspace{20cm}
\end{frame}

% end module trig-substitution-case-3-sec
%\begin{frame}
\frametitle{Euler substitution $x=\sec \theta$, $\theta = 2\arctan t$}
\begin{itemize}
\item $\alert<4>{ x =\sec \theta}$ transforms $\diff x, x,\sqrt{x^2-1}$ to trig form.
\item $\alert<5>{\theta=2\Arctan t}$, \uncover<3->{ $ t\in (-\infty, -1) \cup \left[0, 1 \right) $ } rationalizes $\diff \theta, \cos\theta,\sin \theta$.
\end{itemize}
\uncover<2->{\alert<2>{What if we compose the above?}} \uncover<3->{\alert<3>{We get the Euler substitution:}}
\only<1-33>{
\[
\begin{array}{rclll}
\phantom{ ( 1- t^2 )} %phantom needed to aling table well.
\uncover<3->{\alert<4,15,22,30,33>{x}}&\uncover<3->{\alert<4,15,22,33>{=}}& \displaystyle  
\only<1-12>{
\uncover<4->{ \alert<4>{\sec \theta} =\frac{1}{\cos \alert<5>{ \theta} }} \\
\uncover<5->{&=& \displaystyle \alert<8>{ \frac{1} {\cos(\alert<5>{ 2\arctan t} )}} } &&
\uncover<6->{
\begin{array}{|l} \displaystyle \alert<8>{ \alert<6,7>{\cos (2z) =} \uncover<7->{\alert<7>{\frac{ 1- \tan^2 z}{1+ \tan^2 z }}}}
\end{array}
}
\\
\uncover<8->{&=& \displaystyle \alert<8>{  \frac{1+ {\alert<9>{\tan}}^2 (\alert<9>{\Arctan t })}{ 1- {\alert<9>{\tan}}^2 (\alert<9>{\Arctan t}) } }}
\\
\uncover<9->{&=&\displaystyle \frac{1+ {\alert<9>{t}}^2 }{1 - { \alert<9>{ t}}^2}}\uncover<10->{=\frac{2\alert<11>{- (1 - t^2 )} }{\alert<11>{1 - t^2} }}  \\
\uncover<11->{&=&} 
} %only<1-12>
\uncover<11->{\displaystyle \alert<12,13,15,22,30,33>{ \alert<11>{-1}+\frac{2}{1-t^2}} }
&&{~~~~~~~~~~~~~~~~~~~~~~~~~~~~~~~~~~~~~~~~~~~~~~~~~~~~~~~~~~~~~~~~~~~~~~~~~~~~~~~~~~~~~~~~~~~~~~~~~~~~~~~~~~~~~~~~~~~~~~~~~~~~~}  %whitespace flushes formulas left

\uncover<13->{\\\hline }
\uncover<14->{\alert<20,33>{ \sqrt{{\alert<15>{x}}^2-1 } }} &\uncover<14->{\alert<20,33>{ =} } &\displaystyle 
\only<1-20>{
\uncover<15->{\sqrt{ \left(\alert<15>{ \frac{1 +t^2 }{\alert<16>{ 1- t^2 }}} \right)^2 -\alert<16>{1}} }\\
\uncover<16->{&=& \displaystyle \sqrt{\frac{ \alert<19>{(1+t^2)^2 - \alert<16>{ (1-t^2)^2}}}{ \alert<16>{(1-t^2)^2} } }} &&
\uncover<17->{
\begin{array}{|l}
\alert<17,18,19>{(1+t^2)^2-(1-t^2)^2= } \only<17>{ \alert<17>{ \textbf{?}}} \uncover<18->{\alert<18,19>{4t^2}}
\end{array}
}
\\
\uncover<19->{ &=& \displaystyle \alert<20>{\sqrt{\frac{ \alert<19>{ 4t^2} } { (1-t^2)^2}}} }&& 
\uncover<20->{ \alert<20>{
\begin{array}{|l} \displaystyle t , 1-t^2\text{ have same sign}\\ \text{when } t\in (-\infty, -1) \cup \left[0, 1 \right)
\end{array}
}
}
\\
\uncover<20>{&=&}
} %only<1-20>
\uncover<20->{ \displaystyle \alert<20,21,33>{ \frac{2t}{1- t^2}} \vphantom{\sqrt{ \left( \frac{1 +t^2 }{ 1- t^2 } \right)^2 -1} }  &&
{~~~~~~~~~~~~~~~~~~~~~~~~~~~~~~~~~~~~~~~~~~~~~~~~~~~~~~~~~~~~~~~~~~~~~~~~~~~~~~~~~~~~~~~~~~~~~~~~~~~~~~~~~~~~~~~~~~~~~~~~~~~~~}  %whitespace flushes formulas left
}
\uncover<21->{\\\hline}
\only<1-27>{
\uncover<22->{\displaystyle \alert<22>{x} &\alert<22>{=} &\displaystyle  \alert<22>{\frac{1+t^2}{ \alert<23>{1- t^2}}}} \\
\uncover<23->{\displaystyle (\alert<23>{ \alert<25>{1} \alert<24>{- t^2}} )\alert<24,25>{x}&=&\displaystyle \alert<25>{ 1}+\alert<24>{ t^2}} \\
\uncover<24->{\displaystyle \alert<24>{\alert<26>{ (1+ x)} t^2} &=&\displaystyle \alert<25>{ x-1} \\
\uncover<26->{ \displaystyle t^2&=&\displaystyle  \frac{x-1 }{\alert<26>{ x+1} }} } \\
}
\uncover<27->{
\displaystyle  
\alert<27,28,33>{t} &\alert<27,28,33>{ =} & \displaystyle \alert<27,28,33>{ \pm \sqrt{\frac{x-1}{x+1}}}
}
\uncover<28->{\\\hline}
\uncover<29->{
\alert<33>{ \diff x} &=& \displaystyle \uncover<30->{ \alert<31,32>{ \diff \left(\alert<30>{ -1+\frac{2}{1-t^2}} \right)}}\\
\uncover<31->{&\alert<31,32,33>{=}& \displaystyle \alert<33>{ \alert<32>{\uncover<32->{ \frac{4 t}{(1- t^{2})^{2}}}} \uncover<31>{\alert<31>{\textbf{?}}} \diff t}
}
}
\end{array}
\]
} %only<1-33>
\uncover<34->{
\begin{definition}
The Euler substitution for $\sqrt{x^2-1}$ corresponding to $x=\sec \theta$ is given by:
\[
\begin{array}{rcl}
\alert<34>{
x} &\alert<34>{=}&\displaystyle \alert<34>{ \frac{1+t^2}{1-t^2}, \quad \quad \quad t\in (-\infty, -1) \cup \left[0, 1 \right)} \\
\alert<34>{\sqrt{x^2-1}}&\alert<34>{=}&\displaystyle \alert<34>{\frac{2t}{1-t^2}}  \\
\alert<34>{\diff x}&\alert<34>{=}&\displaystyle  \alert<34>{\frac{4 t}{(1- t^{2})^{2}} \diff t}\\
\alert<34>{t}&\alert<34>{=}&\displaystyle \alert<34>{\pm \frac{ \sqrt{x^2-1}}{x+1} }\quad .
\end{array}
\]
\end{definition}
}

\vspace{8cm}
\end{frame}
%% begin module inverse-notation-warning
\begin{frame}
\alert<1->{WARNING:}

Do not mistake the $-1$ in $f^{-1}(x)$ for an exponent.
\[
f^{-1}(x) \ \text{does not mean } \ \frac{1}{f(x)} .
\]

If you want to write $\frac{1}{f(x)}$ using exponents, you can write $(f(x))^{-1}$.
\begin{itemize}
\item<2->  $f^{-1}(x)$ is the compositional inverse of $f$.
\item<3->  $\frac{1}{f(x)}$ is the multiplicative inverse of $f$.
\end{itemize}
\end{frame}
% end module inverse-notation-warning

%% begin module parametric-intro
\begin{frame}
\frametitle{(11.1)  Curves Defined by Parametric Equations}
\begin{columns}[c]
\column{.4\textwidth}
\ \only<handout:0| -1>{%
\includegraphics[height=7cm]{parametric-curves/pictures/11-01-parametrica.pdf}%
}%
\only<handout:0| 2>{%
\includegraphics[height=7cm]{parametric-curves/pictures/11-01-parametricb.pdf}%
}%
\only<handout:0| 3>{%
\includegraphics[height=7cm]{parametric-curves/pictures/11-01-parametricc.pdf}%
}%
\only<handout:0| 4>{%
\includegraphics[height=7cm]{parametric-curves/pictures/11-01-parametricd.pdf}%
}%
\only<handout:0| 5-7>{%
\includegraphics[height=7cm]{parametric-curves/pictures/11-01-parametrice.pdf}%
}%
\only<8->{%
\includegraphics[height=7cm]{parametric-curves/pictures/11-01-parametricf.pdf}%
}%
\column{.6\textwidth}
\begin{itemize}
\item  Suppose a particle moves along the curve in the picture.
\item<6->  We can't write the curve as $y = f(x)$ because it fails the vertical line test.
\item<7->  But the $x$-coordinate and $y$-coordinate of the particle are functions of the time $t$.
\item<8->  We can write $x = f(t)$ and $y = g(t)$.
\item<9->  These are called parametric equations, and the curve is called a parametric curve.
\end{itemize}
\end{columns}
\end{frame}
% end module parametric-intro

%%begin module parametric-curve-definition
\begin{frame}
%Let $[a,b]$ be an interval, and let $f_1, \dots, f_n$ be functions on the interval $[a,b]$. 
\begin{definition}[Curve in $n$-dimensional space]
We define an arbitrary $n$-tuple of functions $f_1,\dots, f_n$ on $[a,b]$ to be a \emph{parametric curve} (or simply \emph{curve}). If $\gamma$ is a curve, we write $\gamma$ as:
\[
\gamma:\left| 
\begin{array}{rcl}
x_1&=&f_1(t)\\
x_2&=&f_2(t)\\
&\vdots & \\
x_n&=&f_n(t)
\end{array} \right., t\in [a,b]\quad 
\]
where $x_1,\dots, x_n$ are the labels of the $n$-dimensional coordinate system.
\end{definition}
Curves in 2- and 3-dimensional space will be of special interest:
\begin{columns}
\column{0.5\textwidth}
A curve in dimension 2 is given by:
\[
\gamma:\left| 
\begin{array}{rcl}
x&=&f(t)\\
y&=&g(t)\\
\end{array} \right., t\in [a,b]\quad .
\]

\column{0.5\textwidth}
A curve in dimension 3 is given by:
\[
\gamma:\left| 
\begin{array}{rcl}
x&=&f(t)\\
y&=&g(t)\\
z&=&h(t)\\
\end{array} \right., t\in [a,b]\quad .
\]

\end{columns}

\end{frame}
%end module parametric-curve-definition
%%begin module curve-image-definition-intro
\begin{frame}
Consider the two parametric curves:
\begin{columns}
\column{0.5\textwidth}
\[
\gamma_1:
\left|
\begin{array}{rcl}
x&=&t^2\\
y&=&t^2\\
\end{array} \right., t\in [0,1]\quad
\]
\begin{center}
\psset{xunit=2cm, yunit=2cm}
\begin{pspicture}(-0.5, -0.5)(1.4,1.4)
\psframe*[linecolor=white](-0.5, -0.5)(1.400000,1.4)
\tiny
\fcAxesStandard{-0.200000}{-0.2}{1.2}{1.2}
\uncover<8->{
\psline[linecolor=\fcColorGraph](0,0)(1,1)
}
\uncover<2->{
\fcFullDot{0}{0}
}
\uncover<3->{
\fcFullDot{0.04}{0.04}
}
\uncover<4->{
\fcFullDot{0.16}{0.16}
}
\uncover<5->{
\fcFullDot{0.36}{0.36}
}
\uncover<6->{
\fcFullDot{0.64}{0.64}
}
\uncover<7->{
\fcFullDot{1}{1}
}
\end{pspicture}
\end{center}

\column{0.5\textwidth}
\[
\gamma_2:
\left|
\begin{array}{rcl}
x&=&t\\
y&=&t\\
\end{array} \right., t\in [0,1]\quad
\]
\begin{center}
\psset{xunit=2cm, yunit=2cm}
\begin{pspicture}(-0.5000000, -0.5)(1.400000,1.4)
\psframe*[linecolor=white](-0.5000000, -0.5)(1.400000,1.4)
\tiny
\fcAxesStandard{-0.200000}{-0.2}{1.2}{1.2}
\uncover<8->{
\psline[linecolor=\fcColorGraph ](0,0)(1,1)
}
\uncover<2->{
\fcFullDot{0}{0}
}
\uncover<3->{
\fcFullDot{0.2}{0.2}
}
\uncover<4->{
\fcFullDot{0.4}{0.4}
}
\uncover<5->{
\fcFullDot{0.6}{0.6}
}
\uncover<6->{
\fcFullDot{0.8}{0.8}
}
\uncover<7->{
\fcFullDot{1}{1}
}
\end{pspicture}
\end{center}
\end{columns}
\uncover<2->{Plug in} \uncover<2->{\alert<2>{$ t=0 $}}\uncover<3->{, \alert<3>{$t=0.2$}}\uncover<4->{, \alert<4>{$t = 0.4 $}}\uncover<5->{, \alert<5>{$t = 0.6$}}\uncover<6->{, \alert<6>{$t=0.8$}}\uncover<7->{, \alert<7>{$t = 1$}.}
\uncover<9->{
\begin{question}
Are the above curves different?
\end{question}
}
\uncover<10->{
To answer this question we need a definition.
}
\end{frame}
%end module curve-image-definition-intro

%
\begin{frame}
Recall a parametric curve $\gamma$  was defined as the data
\[
\gamma:
\left| 
\begin{array}{rcl}
x_1&=&f_1(t)\\
x_2&=&f_2(t)\\
&\vdots & \\
x_n&=&f_n(t)
\end{array} \right., t\in [a,b]\quad 
\]
\begin{definition}
A \emph{curve image} (or simply a curve) is any set of points that arises by traversing some \alert<2>{continuous} curve. In other words, a curve image is any set that can be written in the form
\[
\left\{(f_1(t),\dots, f_n(t))~|~ t\in [a,b]\right\}\quad ,
\]
for some \alert<2>{continuous} functions $f_1, \dots, f_n$.
\end{definition}
\only<2>{If we don't require that the functions be continuous, else every set of points will be a curve and the definition would be pointless.}

\uncover<3->{Informally, a curve image ``remembers'' only the points lying on the curve but forgets the ``speed'' with which each point was visited and ``how many times'' each point was visited.
}
\end{frame}
%%begin module parametric-curve-vs-curve-image-terminology
\begin{frame}[t]
\begin{columns}
\column{0.5\textwidth}
\begin{center}
\psset{xunit=1.3cm, yunit=1.3cm}
\begin{pspicture}(-0.2, -0.2)(1.2,1.2) 
\tiny 
\psaxesStandard{-0.2}{-0.2}{1.2}{1.2}
\psline[linecolor=\psColorGraph](0,0)(1,1)
\rput[l](0.4,0.3){$C_1:
\left| 
\begin{array}{rcl}
x&=&t^2\\
y&=&t^2\\
\end{array} \right., t\in [0,1]
$}
\end{pspicture} 
\end{center}

\column{0.5\textwidth}
\begin{center}
\psset{xunit=1.3cm, yunit=1.3cm}
\begin{pspicture}(-0.4, -0.4)(1.4,1.4) 
\tiny 
\psaxesStandard{-0.200000}{-0.2}{1.2}{1.2}
\psline[linecolor=\psColorGraph ](0,0)(1,1)
\rput[l](0.4,0.3){$C_2:
\left| 
\begin{array}{rcl}
x&=&t\\
y&=&t\\
\end{array} \right., t\in [0,1]$
}
\end{pspicture}
\end{center}
\end{columns}
\begin{question}
$\begin{array}{l|l}
\only<1-3>{\text{ Are the above curves different?}}
\only<4->{\alert<4>{\xcancel{\text{ Are the above curves different?}}}} &\begin{array}{l} \uncover<2->{\alert<2>{\text{Are the above parametric curves} }
\\
\alert<2>{\text{different? Yes.}}}
\\
\uncover<3->{\alert<3>{\text{Are the above curve images}}\\ 
\alert<3>{\text{ different? No.}}}
\end{array}
\end{array}
$
\end{question}
\begin{itemize}
\only<1-4>{
\item<2-> As parametric curves, $C_1$ and $C_2$ are different: $C_1, C_2$ are given by different functions.
\item<3-> As curve images, $C_1,C_2$ coincide.
\item<4-> The original question is incorrectly posed: the word ``curve'' does not have a mathematical definition without the words ``parametric'' or ``image'' attached to it.
}
\item<5-> Nonetheless we sometimes use the word ``curve'' \alert<5>{informally}, without specifying ``parametric curve'' or ``curve image''. 
\item<6-> In this case, whether we mean  ``parametric curve'' or ``curve image'' should be clear from the context. \uncover<7->{\alert<7>{If not, we are using mathematical language incorrectly.}}
\end{itemize}

\vspace{5cm}

\end{frame}



%end module parametric-curve-vs-curve-image-terminology

%% begin module arc-length-intro
\begin{frame}
\frametitle{Arc Length}
\begin{center}

%\psset{xunit=1cm, yunit=1cm}
%\begin{pspicture}(-1.500000, -5)(1.500000,5) 
%\psframe*[linecolor=white](-1.500000,-5)(1.500000,5) 
%\tiny 
%\psaxesStandard{-1.000000}{-4.5}{1.000000}{4.5}
%\psplot[linecolor=\psColorGraph, plotpoints=1000]{-1.000000}{1.000000}{1 x 2 exp -1 mul add sqrt -1 mul }
%Function formula: \sqrt{- x^{2}+1} 
%\psplot[linecolor=\psColorGraph, plotpoints=1000]{-1.000000}{1.000000}{1 x 2 exp -1 mul add sqrt }
%\end{pspicture} 


\ \only<-2>{%
\includegraphics[height=4cm]{arc-length/pictures/09-01-circlea.pdf}%
}%
\only<handout:0| 3>{%
\includegraphics[height=4cm]{arc-length/pictures/09-01-circleb.pdf}%
}%
\only<handout:0| 4>{%
\includegraphics[height=4cm]{arc-length/pictures/09-01-circlec.pdf}%
}%
\only<handout:0| 5>{%
\includegraphics[height=4cm]{arc-length/pictures/09-01-circled.pdf}%
}%
\only<handout:0| 6>{%
\includegraphics[height=4cm]{arc-length/pictures/09-01-circlee.pdf}%
}%
\only<handout:0| 7->{%
\includegraphics[height=4cm]{arc-length/pictures/09-01-circlef.pdf}%
}%
\end{center}
\begin{itemize}
\item  What do we mean by the length of a curve?
\item<2->  The length of a polygon is easy to compute: add up the length of the line segments that form the polygon.
\item<3->  If the curve is a circle, approximate it by a polygon.
\item<4->  Then take the limit as the number of segments of the polygon goes to $\infty$.
\end{itemize}
\end{frame}
% end module arc-length-intro

%%begin module arc-length-derivation-parametric
\begin{frame}
Let $\gamma $ be the curve
$ \gamma: \left|
\begin{array}{rcl}
x=x(t)\\
y=y(t)
\end{array}, t\in [a,b]
\right.$

\begin{itemize}
\item<2->  Divide $[a,b]$ into $n$ subintervals with endpoints $t_0, t_1, \ldots , t_n$ and equal width $\Delta t$.
\item<3->  The points $P_i = (x(t_i), y(t_i))$ lie on the curve $\gamma$. The lengths of the segments with endpoints with consecutive indices from $P_0, P_1, \ldots , P_n$ approximate the length of the curve $\gamma$.
\item<4->  The length $L$ of the curve $\gamma$ is the limit of the lengths of these segments as $n\rightarrow \infty$.
\end{itemize}
\begin{columns}[c]
\column{.6\textwidth}
%\begin{center}breaks on Ubuntu
\psset{xunit=1.4cm, yunit=1.4cm}
\begin{pspicture}(-0.4,-0.3)(2.3,1.994800)
\tiny
\newcommand{\theCurve}{t 13 t -50 mul add t 2 exp 70 mul add t 3 exp -40 mul add t 4 exp 8 mul add}
\newcommand{\lowBound}{0.4}
\newcommand{\highBound}{2}
\fcAxesStandard{-0.3}{-0.3}{2.146803}{1.994800}
%\theCurve is defined in the beginning of this module, has limited scope
\parametricplot{\lowBound}{\highBound}{\theCurve}
\uncover<handout:0|2-4>{%
\fcPolylineAlongCurveWithLabels[linecolor=\fcColorGraph]{3}{\lowBound}{\highBound}{\theCurve}{P}%
}
\uncover<handout:0|5>{%
\fcPolylineAlongCurveWithLabels[linecolor=\fcColorGraph]{4}{\lowBound}{\highBound}{\theCurve}{P}%
}
\uncover<6>{%
\fcPolylineAlongCurveWithLabels[linecolor=\fcColorGraph]{5}{\lowBound}{\highBound}{\theCurve}{P}%
}
\uncover<handout:0|7>{%
\fcPolylineAlongCurveWithLabels[linecolor=\fcColorGraph]{6}{\lowBound}{\highBound}{\theCurve}{P}%
}
\uncover<handout:0|8>{%
\fcPolylineAlongCurveWithLabels[linecolor=\fcColorGraph]{10}{\lowBound}{\highBound}{\theCurve}{P}%
}
\uncover<handout:0|9->{%0
\fcPolylineAlongCurveWithLabels[linecolor=\fcColorGraph]{14}{\lowBound}{\highBound}{\theCurve}{P}%
}
\end{pspicture}
%\end{center}
\column{.4\textwidth}
\uncover<10->{%
\[
L = \lim\limits_{n\rightarrow \infty} \sum\limits_{i=1}^n |P_{i-1}P_i|
\]
}%
\end{columns}
\end{frame}



\begin{frame}
Let $\gamma $ be the curve
$ \gamma: \left|
\begin{array}{rcl}
x=x(t)\\
y=y(t)
\end{array}, t\in [a,b]
\right.$

$\begin{array}{rcl}
\uncover<1->{%
L = \lim\limits_{n\rightarrow \infty} \sum\limits_{i = 1}^n \alertNoH{ 10}{|P_{i-1}P_i|}%
}%
& \uncover<10->{ = } &%
\uncover<10->{%
\alertNoH{ 12}{\lim\limits_{n\rightarrow\infty} \sum\limits_{i=1}^n} \alertNoH{ 10}{\sqrt{\alertNoH{13}{ (x'(s_i))^2}+\alertNoH{ 13}{(y'(r_i))^2}}\ \alertNoH{ 14}{\Delta t}}%
}\\%
& \uncover<11->{ = } &%
\uncover<11->{%
\alertNoH{ 12}{\int_a^b} \sqrt{ \alertNoH{ 13}{(x'(t))^2} +\alertNoH{ 13}{(y'(t))^2}} \ \alertNoH{ 14}{\diff t}%
}%
\end{array}
$
\begin{itemize}
\item If $f$ has continuous derivative, we can compute the above limit.
\item<2-> Let
$\left|\begin{array}{r@{~}c@{~}l}
x_i &=& x(t_i)\\
y_i &=& y(t_i)
\end{array}\right.$,
and
$\left|\begin{array}{rcl}
\Delta x &=& x_i - x_{i-1} = x(t_i) - x(t_{i-1})\\
\Delta y &=& y_i - y_{i-1} = y(t_i) - y(t_{i-1})
\end{array}\right. $.
\item<3-| alert@6> Then $|P_iP_{i-1}| = \sqrt{(\Delta x)^2 + (\Delta y)^2}$.
\item<4-> Mean Value Theorem: there exist numbers $s_i$ and $r_i$ between $t_{i-1}$ and $t_i$ such that $\alertNoH{ 5}{x(t_i) - x(t_{i-1}) = x'(s_i )(t_i- t_{i-1})}$  and $\alertNoH{ 5}{y(t_i) - y(t_{i-1}) = y'(r_i)( t_i-t_{i-1})}$.
\item<5-| alert@5,7> $\Delta x = x'(s_i)\Delta t$, $\Delta y = y'(r_i)\Delta t$.
\end{itemize}
$\begin{array}{rcl}
\uncover<6->{%
\alertNoH{ 10}{|P_{i-1}P_i|}%
}%
& \uncover<6->{\alertNoH{ 10}{ = }} &%
\uncover<6->{%
\sqrt{(\alertNoH{ 7}{\Delta x})^2 + (\alertNoH{ 7}{\Delta y})^2}%
}  \uncover<7->{ = } \uncover<7->{%
\sqrt{ (\alertNoH{ 7}{x'(s_i)\Delta t})^2 + (\alertNoH{ 7}{y'(r_i)\Delta t})^2}%
}\\%
& \uncover<8->{ = } &%
\uncover<8->{%
\sqrt{(x'(s_i))^2 + (y'(r_i))^2}\sqrt{(\Delta t)^2}%
}  \uncover<9->{ = } \uncover<9->{%
\alertNoH{ 10}{\sqrt{(x'(s_i))^2 + (y'(r_i))^2}\ \Delta t }%
}\\%
\end{array}
$
\end{frame}
%end module arc-length-derivation-parametric

%% begin module arc-length-derivation
\begin{frame}
\begin{itemize}
\item  What about $y = f(x)$ for a continuous function $f$ on $[a, b]$?
\item<2->  Divide $[a,b]$ into $n$ subintervals with endpoints $x_0, x_1, \ldots , x_n$ and equal width $\Delta x$.
\item<3->  The points $P_i = (x_i, f(x_i))$ lie on $y = f(x)$, and the segments with vertices $P_0, P_1, \ldots , P_n$ are an approximation to $y = f(x)$.
\item<4->  The length $L$ of the curve $y = f(x)$ is the limit of the lengths of these segments as $n\rightarrow \infty$.
\end{itemize}
\begin{columns}[c]
\column{.6\textwidth}
\begin{center}
\ \only<handout:0| -1>{%
\includegraphics[height=4.5cm]{arc-length/pictures/09-01-arclengtha.pdf}%
}%
\only<handout:0| 2-4>{%
\includegraphics[height=4.5cm]{arc-length/pictures/09-01-arclengthb.pdf}%
}%
\only<handout:0| 5>{%
\includegraphics[height=4.5cm]{arc-length/pictures/09-01-arclengthc.pdf}%
}%
\only<handout:0| 6>{%
\includegraphics[height=4.5cm]{arc-length/pictures/09-01-arclengthd.pdf}%
}%
\only<handout:0| 7>{%
\includegraphics[height=4.5cm]{arc-length/pictures/09-01-arclengthe.pdf}%
}%
\only<8>{%
\includegraphics[height=4.5cm]{arc-length/pictures/09-01-arclengthf.pdf}%
}%
\only<handout:0| 9->{%
\includegraphics[height=4.5cm]{arc-length/pictures/09-01-arclengthg.pdf}%
}%
%\only<10->{%
%\includegraphics[height=4.5cm]{arc-length/pictures/09-01-arclengthh.pdf}%
%}%
\end{center}
\column{.4\textwidth}
\uncover<10->{%
\[
L = \lim_{n\rightarrow \infty} \sum_{i=1}^n |P_{i-1}P_i|
\]
}%
\end{columns}
\end{frame}



\begin{frame}
\begin{eqnarray*}
\uncover<1->{%
L = \lim_{n\rightarrow \infty} \sum_{i = 1}^n \alertNoH{ 10}{|P_{i-1}P_i|}%
}%
& \uncover<10->{ = } &%
\uncover<10->{%
\alertNoH{ 12}{\lim_{n\rightarrow\infty} \sum_{i=1}^n} \alertNoH{ 10}{\sqrt{1+\alertNoH{ 13}{(f'(x_i^*))^2}}\ \alertNoH{ 14}{\Delta x}}%
}\\%
& \uncover<11->{ = } &%
\uncover<11->{%
\alertNoH{ 12}{\int_a^b} \sqrt{1+\alertNoH{ 13}{(f'(x))^2}} \ \alertNoH{ 14}{\diff x}%
}%
\end{eqnarray*}
\begin{itemize}
\item  This formula is not useful for computational purposes.
\item  We can find a better formula when $f$ has a continuous derivative.
\item<2->  Let $y_i = f(x_i)$, and $\Delta y = y_i - y_{i-1} = f(x_i) - f(x_{i-1})$.
\item<3-| alert@6>  Then $|P_iP_{i-1}| = \sqrt{(x_i-x_{i-1})^2+(y_i-y_{i-1})^2} = \sqrt{(\Delta x)^2 + (\Delta y)^2}$.
\item<4->  Mean Value Theorem: there exists $x_i^*$ between $x_{i-1}$ and $x_i$ such that $\alertNoH{ 5}{f(x_i) - f(x_{i-1}) = f'(x_i^*)(x_i-x_{i-1})}$.
\item<5-| alert@5,7>  $\Delta y = f'(x_i^*)\Delta x$.
\end{itemize}
\begin{eqnarray*}
\uncover<6->{%
\alertNoH{ 10}{|P_{i-1}P_i|}%
}%
& \uncover<6->{\alertNoH{ 10}{ = }} &%
\uncover<6->{%
\sqrt{(\Delta x)^2 + (\alertNoH{ 7}{\Delta y})^2}%
}  \uncover<7->{ = } \uncover<7->{%
\sqrt{(\Delta x)^2 + (\alertNoH{ 7}{f'(x_i^*)\Delta x})^2}%
}\\%
& \uncover<8->{ = } &%
\uncover<8->{%
\sqrt{1 + (f'(x_i^*))^2}\sqrt{(\Delta x)^2}%
}  \uncover<9->{ = } \uncover<9->{%
\alertNoH{ 10}{\sqrt{1 + (f'(x_i^*))^2}\ \Delta x}%
}\\%
\end{eqnarray*}
\end{frame}
% end module arc-length-derivation

%% begin module arc-length-def
\begin{frame}
\frametitle{The Arc Length Formula}
Let $\gamma:\left|\begin{array}{rcl} x&=&x(t)\\ y&=&y(t)\end{array}  \right., t\in [a,b]$.


\begin{definition}
Suppose $x'(t)$ and $y'(t)$ (exist and) are continuous on $[a,b]$. Then the length of the curve $\gamma$ is defined as 
\[
\begin{array}{rclll}
\displaystyle L(\gamma) &=&\displaystyle  \int_a^b \sqrt{(x'(t))^2 +(y'(t))^2} ~ \diff x\\
\uncover<2->{&=& \displaystyle \int_a^b \sqrt{\left(\frac{\diff x}{\diff t}\right)^2 + \left( \frac{\diff y}{\diff t}\right)^2} ~ \diff t &&\text{in Leibniz notation .}}
\end{array}
\]

\end{definition}
\end{frame}
% end module arc-length-def

%%begin module graphs-of-functions-as-curves
\begin{frame}
\frametitle{Graphs of functions as curve images}
\begin{itemize}
\item Consider a graph of a function given by 
\[
y=f(x)
\]
\item<2-> Write $x=t$. Then $y=f(x)=f(t)$, so we get the system 
\[
\gamma: \left|\begin{array}{rcl}
y&=&f(t)\\
x&=&t
\end{array}\right., t\in [a,b]
\]
\end{itemize}
\uncover<3->{
\begin{observation}
The graph of an arbitrary function can be written as the image of a curve $\gamma$ using the above transformation.
\end{observation}
}
\end{frame}

%end module graphs-of-functions-as-curves

%%begin module arc-length-function-graph-from-parametric-curve-length

\begin{frame}
\frametitle{Arc length of graph of a function}
\begin{question}
What is the length of the graph of the curve given by the graph of $y=f(x)$?
\end{question}
\begin{itemize}
\item<2-> The graph of $y=f(x)$ is written as a curve as 
\[
\gamma:\left|
\begin{array}{rcl}
\alertNoH{7}{x}&\alertNoH{7}{=}&\alertNoH{7}{t}\\
y&\alertNoH{0}{=}&f(t) 
\end{array}\right.,t\in [a,b]\quad .
\]
\item<3-> In other words, the question asks what is the length $L(\gamma)$ of $\gamma$. \uncover<4->{That is a straightforward computation: 
$
\begin{array}{rcl}
\uncover<4->{L(\gamma)}&\uncover<4->{=}&
\displaystyle \int \sqrt{( \alertNoH{6,7}{x'(t)})^2 +(\alertNoH{5}{y'(t)})^2 } \diff t= \uncover<5->{\int \sqrt{ \fcAnswerUncover{5}{7}{1} +(\alertNoH{5}{f'(t)})^2 } \diff t }
\end{array}
$
}
\end{itemize}

\end{frame}
%end module arc-length-function-graph-from-parametric-curve-length
%%begin module arc-length-does-not-depend-on-parametrization-when-one-to-one

%end module arc-length-does-not-depend-on-parametrization-when-one-to-one
%% begin module arc-length-def
\begin{frame}
\frametitle{The Arc Length Formula}
Let $\gamma:\left|\begin{array}{rcl} x&=&x(t)\\ y&=&y(t)\end{array}  \right., t\in [a,b]$.


\begin{definition}
Suppose $x'(t)$ and $y'(t)$ (exist and) are continuous on $[a,b]$. Then the length of the curve $\gamma$ is defined as 
\[
\begin{array}{rclll}
\displaystyle L(\gamma) &=&\displaystyle  \int_a^b \sqrt{(x'(t))^2 +(y'(t))^2} ~ \diff x\\
\uncover<2->{&=& \displaystyle \int_a^b \sqrt{\left(\frac{\diff x}{\diff t}\right)^2 + \left( \frac{\diff y}{\diff t}\right)^2} ~ \diff t &&\text{in Leibniz notation .}}
\end{array}
\]

\end{definition}
\end{frame}
% end module arc-length-def

%% begin module arc-length-def
\begin{frame}
\frametitle{The Arc Length Formula}

\begin{definition}
Suppose $f'$ exists and is continuous on $[a,b]$. Then the length of the curve $y = f(x)$, $a\leq x \leq b$, is
\[
\begin{array}{rclll}
\displaystyle L& =&\displaystyle  \int\limits_a^b \sqrt{1 +(f'(x))^2} \ \diff x\\
 &=&\displaystyle  \int\limits_a^b \sqrt{1 + \left( \frac{\diff y}{\diff x}\right)^2} \ \diff x&&\text{(in Leibniz notation)\quad .}
\end{array}
\]
\end{definition}

\end{frame}
% end module arc-length-def

%%begin module parametric-curve-definition
\begin{frame}
%Let $[a,b]$ be an interval, and let $f_1, \dots, f_n$ be functions on the interval $[a,b]$. 
\begin{definition}[Curve in $n$-dimensional space]
We define an arbitrary $n$-tuple of functions $f_1,\dots, f_n$ on $[a,b]$ to be a \emph{parametric curve} (or simply \emph{curve}). If $\gamma$ is a curve, we write $\gamma$ as:
\[
\gamma:\left| 
\begin{array}{rcl}
x_1&=&f_1(t)\\
x_2&=&f_2(t)\\
&\vdots & \\
x_n&=&f_n(t)
\end{array} \right., t\in [a,b]\quad 
\]
where $x_1,\dots, x_n$ are the labels of the $n$-dimensional coordinate system.
\end{definition}
Curves in 2- and 3-dimensional space will be of special interest:
\begin{columns}
\column{0.5\textwidth}
A curve in dimension 2 is given by:
\[
\gamma:\left| 
\begin{array}{rcl}
x&=&f(t)\\
y&=&g(t)\\
\end{array} \right., t\in [a,b]\quad .
\]

\column{0.5\textwidth}
A curve in dimension 3 is given by:
\[
\gamma:\left| 
\begin{array}{rcl}
x&=&f(t)\\
y&=&g(t)\\
z&=&h(t)\\
\end{array} \right., t\in [a,b]\quad .
\]

\end{columns}

\end{frame}
%end module parametric-curve-definition
%%begin module curve-image-definition-intro
\begin{frame}
Consider the two parametric curves:
\begin{columns}
\column{0.5\textwidth}
\[
\gamma_1:
\left|
\begin{array}{rcl}
x&=&t^2\\
y&=&t^2\\
\end{array} \right., t\in [0,1]\quad
\]
\begin{center}
\psset{xunit=2cm, yunit=2cm}
\begin{pspicture}(-0.5, -0.5)(1.4,1.4)
\psframe*[linecolor=white](-0.5, -0.5)(1.400000,1.4)
\tiny
\fcAxesStandard{-0.200000}{-0.2}{1.2}{1.2}
\uncover<8->{
\psline[linecolor=\fcColorGraph](0,0)(1,1)
}
\uncover<2->{
\fcFullDot{0}{0}
}
\uncover<3->{
\fcFullDot{0.04}{0.04}
}
\uncover<4->{
\fcFullDot{0.16}{0.16}
}
\uncover<5->{
\fcFullDot{0.36}{0.36}
}
\uncover<6->{
\fcFullDot{0.64}{0.64}
}
\uncover<7->{
\fcFullDot{1}{1}
}
\end{pspicture}
\end{center}

\column{0.5\textwidth}
\[
\gamma_2:
\left|
\begin{array}{rcl}
x&=&t\\
y&=&t\\
\end{array} \right., t\in [0,1]\quad
\]
\begin{center}
\psset{xunit=2cm, yunit=2cm}
\begin{pspicture}(-0.5000000, -0.5)(1.400000,1.4)
\psframe*[linecolor=white](-0.5000000, -0.5)(1.400000,1.4)
\tiny
\fcAxesStandard{-0.200000}{-0.2}{1.2}{1.2}
\uncover<8->{
\psline[linecolor=\fcColorGraph ](0,0)(1,1)
}
\uncover<2->{
\fcFullDot{0}{0}
}
\uncover<3->{
\fcFullDot{0.2}{0.2}
}
\uncover<4->{
\fcFullDot{0.4}{0.4}
}
\uncover<5->{
\fcFullDot{0.6}{0.6}
}
\uncover<6->{
\fcFullDot{0.8}{0.8}
}
\uncover<7->{
\fcFullDot{1}{1}
}
\end{pspicture}
\end{center}
\end{columns}
\uncover<2->{Plug in} \uncover<2->{\alert<2>{$ t=0 $}}\uncover<3->{, \alert<3>{$t=0.2$}}\uncover<4->{, \alert<4>{$t = 0.4 $}}\uncover<5->{, \alert<5>{$t = 0.6$}}\uncover<6->{, \alert<6>{$t=0.8$}}\uncover<7->{, \alert<7>{$t = 1$}.}
\uncover<9->{
\begin{question}
Are the above curves different?
\end{question}
}
\uncover<10->{
To answer this question we need a definition.
}
\end{frame}
%end module curve-image-definition-intro

%
\begin{frame}
Recall a parametric curve $\gamma$  was defined as the data
\[
\gamma:
\left| 
\begin{array}{rcl}
x_1&=&f_1(t)\\
x_2&=&f_2(t)\\
&\vdots & \\
x_n&=&f_n(t)
\end{array} \right., t\in [a,b]\quad 
\]
\begin{definition}
A \emph{curve image} (or simply a curve) is any set of points that arises by traversing some \alert<2>{continuous} curve. In other words, a curve image is any set that can be written in the form
\[
\left\{(f_1(t),\dots, f_n(t))~|~ t\in [a,b]\right\}\quad ,
\]
for some \alert<2>{continuous} functions $f_1, \dots, f_n$.
\end{definition}
\only<2>{If we don't require that the functions be continuous, else every set of points will be a curve and the definition would be pointless.}

\uncover<3->{Informally, a curve image ``remembers'' only the points lying on the curve but forgets the ``speed'' with which each point was visited and ``how many times'' each point was visited.
}
\end{frame}
%%begin module parametric-curve-vs-curve-image-terminology
\begin{frame}[t]
\begin{columns}
\column{0.5\textwidth}
\begin{center}
\psset{xunit=1.3cm, yunit=1.3cm}
\begin{pspicture}(-0.2, -0.2)(1.2,1.2) 
\tiny 
\psaxesStandard{-0.2}{-0.2}{1.2}{1.2}
\psline[linecolor=\psColorGraph](0,0)(1,1)
\rput[l](0.4,0.3){$C_1:
\left| 
\begin{array}{rcl}
x&=&t^2\\
y&=&t^2\\
\end{array} \right., t\in [0,1]
$}
\end{pspicture} 
\end{center}

\column{0.5\textwidth}
\begin{center}
\psset{xunit=1.3cm, yunit=1.3cm}
\begin{pspicture}(-0.4, -0.4)(1.4,1.4) 
\tiny 
\psaxesStandard{-0.200000}{-0.2}{1.2}{1.2}
\psline[linecolor=\psColorGraph ](0,0)(1,1)
\rput[l](0.4,0.3){$C_2:
\left| 
\begin{array}{rcl}
x&=&t\\
y&=&t\\
\end{array} \right., t\in [0,1]$
}
\end{pspicture}
\end{center}
\end{columns}
\begin{question}
$\begin{array}{l|l}
\only<1-3>{\text{ Are the above curves different?}}
\only<4->{\alert<4>{\xcancel{\text{ Are the above curves different?}}}} &\begin{array}{l} \uncover<2->{\alert<2>{\text{Are the above parametric curves} }
\\
\alert<2>{\text{different? Yes.}}}
\\
\uncover<3->{\alert<3>{\text{Are the above curve images}}\\ 
\alert<3>{\text{ different? No.}}}
\end{array}
\end{array}
$
\end{question}
\begin{itemize}
\only<1-4>{
\item<2-> As parametric curves, $C_1$ and $C_2$ are different: $C_1, C_2$ are given by different functions.
\item<3-> As curve images, $C_1,C_2$ coincide.
\item<4-> The original question is incorrectly posed: the word ``curve'' does not have a mathematical definition without the words ``parametric'' or ``image'' attached to it.
}
\item<5-> Nonetheless we sometimes use the word ``curve'' \alert<5>{informally}, without specifying ``parametric curve'' or ``curve image''. 
\item<6-> In this case, whether we mean  ``parametric curve'' or ``curve image'' should be clear from the context. \uncover<7->{\alert<7>{If not, we are using mathematical language incorrectly.}}
\end{itemize}

\vspace{5cm}

\end{frame}



%end module parametric-curve-vs-curve-image-terminology
%%begin module graphs-of-functions-as-curves
\begin{frame}
\frametitle{Graphs of functions as curve images}
\begin{itemize}
\item Consider a graph of a function given by 
\[
y=f(x)
\]
\item<2-> Write $x=t$. Then $y=f(x)=f(t)$, so we get the system 
\[
\gamma: \left|\begin{array}{rcl}
y&=&f(t)\\
x&=&t
\end{array}\right., t\in [a,b]
\]
\end{itemize}
\uncover<3->{
\begin{observation}
The graph of an arbitrary function can be written as the image of a curve $\gamma$ using the above transformation.
\end{observation}
}
\end{frame}

%end module graphs-of-functions-as-curves

%% begin module parametric-tangents-intro-version2
\begin{frame}
\frametitle{Tangents}
Let $C $ be the curve $C:\left|\begin{array}{rcl}x&=&f(t)\\y&=&g(t)\end{array} \right., t\in [a,b]$.
\begin{definition}
Suppose \alertNoH{4,5}{ $f'(t)$ and $g'(t)$ are not simultaneously equal to $0$.}
\begin{itemize}
\item  We define $(f'(t), g'(t))$ to be the \emph{tangent vector} to $C$ at $t$.
\item<2->  We define the line passing through $(f(t), g(t))$ with direction vector equal to the tangent vector to be \emph{tangent line} to $C$ at $t$. In other words, the tangent line has equation
\[
(x-f(t))g'(t) =(y-g(t))f'(t)\quad .
\]
\item<3->  We say that the tangent to $C$ at $t$ is vertical if $f'(t)=0$ (\alertNoH{4}{and therefore $g'(t)\neq 0$}).
\end{itemize}
\end{definition}
\uncover<5->{\alertNoH{5-}{Note.} When $f'(t)=g'(t)=0$, for curves $C$ with additional properties, natural definition(s) of tangent(s) do exist but are beyond Calc II.}
\end{frame}


%% begin module parametric-tangents-ex1
\begin{frame}[t]
\begin{example}
A curve $C$ is defined by $x = t^2, y = t^3 - 3t$.
\begin{enumerate}
\item  Show $C$ has two tangents at $(x,y)=(3,0)$ and find their slopes.
\item  Find the points on $C$ where the tangents are horizontal or vertical.
\item  Find two intervals where we can write $y$ as a function of $x$.
\item  Determine concavity intervals of the functions found in item 3.
\end{enumerate}
\end{example}
\end{frame}




\begin{frame}[t]
\begin{example}
A curve $C$ is defined by \alert<handout:0| 2>{$x = t^2, y = t^3 - 3t$}.
\begin{enumerate}
\item  Show $C$ has two tangents at $(x,y)=(3,0)$ and find their slopes.
%\item  Find the points on $C$ where the tangents are horizontal or vertical.
%\item  Determine where the curve is concave up or down.
\end{enumerate}
\begin{itemize}
\item<2-| alert@3-4>  $3 = \alert<handout:0| 2>{x = t^2}$ \ if \ $t = $ \uncover<4->{$\pm \sqrt{3}$.}
\item<2-| alert@5-6>  $0 = \alert<handout:0| 2>{y = t^3 - 3t} = t(t^2-3)$\  if \ $t = $ \uncover<6->{$0$\  or\  $\pm \sqrt{3}$.}
\item<7->  Therefore the point $(3,0)$ is traversed when $t$ equals $\sqrt{3}$ or $-\sqrt{3}$.
\end{itemize}
\abovedisplayskip=0pt
\belowdisplayskip=0pt
\[
\uncover<8->{%
\frac{\diff y}{\diff x} = \frac{\alert<handout:0| 9-10>{\diff y / \diff t}}{\alert<handout:0| 11-12>{\diff x / \diff t}} %
}%
\uncover<9->{%
 = \frac{\alert<handout:0| 10>{\uncover<10->{3t^2-3 }}}{\alert<handout:0| 12>{\uncover<12->{2t }}}%
}%
\]
\uncover<13->{%
Plug in $t = \pm \sqrt{3}$:
}%
\abovedisplayskip=0pt
\belowdisplayskip=0pt
\[
\uncover<13->{%
\left. \frac{\diff y}{\diff x} \right|_{t = \pm \sqrt{3}} = \frac{3(\pm \sqrt{3})^2 - 3}{2(\pm \sqrt{3})} = %
}%
\uncover<14->{%
\pm \frac{6}{2\sqrt{3}} = \pm \sqrt{3}%
}%
\]
\uncover<15->{%
Therefore the tangents at $(3,0)$ have slopes $\pm \sqrt{3}$.
}%
\end{example}
\end{frame}



\begin{frame}[t]
\begin{example}
A curve $C$ is defined by $x = t^2, y = t^3 - 3t$.
\begin{enumerate}
\setcounter{enumi}{1}
%\item  Show that $C$ has two tangents at $(3,0)$ and find their slopes.
\item  Find the points on $C$ where the tangents are horizontal or vertical.
%\item  Determine where the curve is concave up or down.
\end{enumerate}
\begin{columns}[t]
\column{.5\textwidth}
Horizontal tangent:
\abovedisplayskip=0pt
\belowdisplayskip=0pt
\begin{eqnarray*}
\frac{\diff y}{\diff t} & = & 0\\
\uncover<2->{%
3t^2 - 3%
}%
& \uncover<2->{ = } & %
\uncover<2->{%
0
}\\%
\uncover<3->{%
3(t^2 - 1)%
}%
& \uncover<3->{ = } & %
\uncover<3->{%
0
}\\%
\uncover<4->{%
t%
}%
& \uncover<4->{ = } & %
\uncover<4->{%
\pm 1%
}%
\end{eqnarray*}
\uncover<5->{$\frac{\diff x}{\diff t} \neq 0$ when $t = \pm 1$, so there are horizontal tangents when $t = \pm 1$.}

\uncover<6->{%
The points are $(1, 2)$ and $(1, -2)$.
}%
\column{.5\textwidth}
Vertical tangent:
\abovedisplayskip=0pt
\belowdisplayskip=0pt
\begin{eqnarray*}
\frac{\diff x}{\diff t} & = & 0\\
\uncover<7->{%
2t%
}%
& \uncover<7->{ = } & %
\uncover<7->{%
0%
}\\%
\uncover<8->{%
t%
}%
& \uncover<8->{ = } & %
\uncover<8->{%
0%
}%
\end{eqnarray*}
\uncover<9->{$\frac{\diff y}{\diff t} \neq 0$ when $t =  0$, so there is a vertical tangent when $t = 0$.}

\uncover<10->{%
The points is $(0,0)$.
}%
\end{columns}
\end{example}
\end{frame}



\begin{frame}[t]
\begin{example} %[Example 1, p. 667]
A curve $C$ is defined by $x = t^2, y = t^3 - 3t$.
\begin{enumerate}
\setcounter{enumi}{2}
%\item  Show that $C$ has two tangents at $(3,0)$ and find their slopes.
%\item  Find the points on $C$ where the tangents are horizontal or vertical.
\item  Find two intervals where we can write $y$ as a function of $x$.
\item  Determine the concavity intervals of the functions found in item 3.
\end{enumerate}
\uncover<2->{Find the second derivative:}%
\begin{eqnarray*}
\uncover<2->{%
\frac{\diff^2 y}{\diff x^2}%
}%
& \uncover<2->{ = } &%
\uncover<2->{%
\frac{\frac{\diff}{\diff t}\left( \alert<handout:0| 3-4>{\frac{\diff y}{\diff x}}\right)}{\alert<handout:0| 5-6>{\frac{\diff x}{\diff t}}}%
}  \uncover<3->{ = }  \uncover<3->{%
\frac{\frac{\diff}{\diff t}\left( \alert<handout:0| 3-4,7>{\uncover<4->{\frac{3t^2-3}{2t}}}\right)}{\alert<handout:0| 5-6>{\uncover<6->{2t}}}%
}\\%
& \uncover<7->{ = } &%
\uncover<7->{%
\frac{\alert<handout:0| 8-9>{\frac{\diff}{\diff t}\left( \alert<handout:0| 7>{\frac{3}{2}\left( t - \frac{1}{t}\right)}\right)}}{2t}%
}  \uncover<8->{ = }  \uncover<8->{%
\frac{\alert<handout:0| 8-9>{\uncover<9->{\frac{3}{2} + \frac{3}{2t^2} }}}{2t}%
}\\%
& \uncover<10->{ = } &%
\uncover<10->{%
\frac{\frac{3t^2 + 3}{2t^2}}{2t}%
}  \uncover<11->{ = } \uncover<11->{%
\frac{3(t^2 + 1)}{4t^3}%
}%
\end{eqnarray*}
\uncover<12->{%
Therefore $y$ as a function of $x(t)$ is concave up when $t > 0$ and concave down when $t < 0$.
}%
\end{example}
\end{frame}
% end module parametric-tangents-ex1

%% begin module cycloid-def
\begin{frame}
\frametitle{The Cycloid}

\psset{xunit=0.8cm, yunit=0.8cm}
\begin{pspicture}(-1.499950, -5)(14.066321,5) 
\psframe*[linecolor=white](-1.499950,-5)(14.066321,5) 
\tiny 

%circles generated by calculator commands:
%precision:=0.99995;f{}{{t}}:=plot2D(\sqrt{1-(x-t)^2}+1, t-precision, t+precision )+plot2D(-\sqrt{1-(x-t)^2}+1, t-precision, t+precision );f{}(0)+f{}(0.5\pi)+f{}(\pi)+f{}(1.5\pi)+f{}(2\pi)+f{}(2.5\pi)+f{}(3\pi)+f{}(3.5\pi)+f{}(4\pi)

\psaxesStandard{-1.1}{-0.5}{13.566321}{2.3}

%calcululator commands: f{}{{t}}:=(DoubleValue (t\pi-sin (t\pi) ), 1-cos (\pi t));(f{}0, f{}0.5, f{}1, f{}1.5, f{}2, f{}2.5, f{}3, f{}3.5, f{}4)
%generate the following points: 
%(0.000000, 0.000000), (0.570796, 1.000000), (3.141593, 2), (5.712389, 1.000000), (6.283185, 0.000000), (6.853982, 1.000000), (9.424778, 2.000000), (11.995574, 1.000000), (12.566371, 0.000000)

\uncover<1>{
%Function formula: (- x^{2}+1)^{1/2}+1 
\psplot[linecolor=\psColorTangent, plotpoints=1000]{-0.999990}{0.999990}{1.0000000 1.0000000 x 2.0000000 exp -1.0000000 mul add 0.5000000 exp add }
%Function formula: - (- x^{2}+1)^{1/2}+1 
\psplot[linecolor=\psColorTangent, plotpoints=1000]{-0.999990}{0.999990}{1.0000000 1.0000000 x 2.0000000 exp -1.0000000 mul add 0.5000000 exp -1.0000000 mul add }
\psFullDot{0}{0}
\rput[bl](0.1, 0.1){$P$}
}

\uncover<2>{
%Function formula: (- (x-1/2 \pi)^{2}+1)^{1/2}+1 
\psplot[linecolor=\psColorTangent, plotpoints=1000]{0.570806}{2.570786}{1.0000000 1.0000000 3.141592654 -0.5000000 mul x add 2.0000000 exp -1.0000000 mul add 0.5000000 exp add }
%Function formula: - (- (x-1/2 \pi)^{2}+1)^{1/2}+1 
\psplot[linecolor=\psColorTangent, plotpoints=1000]{0.570806}{2.570786}{1.0000000 1.0000000 3.141592654 -0.5000000 mul x add 2.0000000 exp -1.0000000 mul add 0.5000000 exp -1.0000000 mul add }
\psFullDot{0.570796}{1}
\rput[l](0.670796, 1.000000){$P$}
}

\uncover<3>{
%Function formula: (- (x- \pi)^{2}+1)^{1/2}+1 
\psplot[linecolor=\psColorTangent, plotpoints=1000]{2.141603}{4.141583}{1.0000000 1.0000000 3.141592654 -1.0000000 mul x add 2.0000000 exp -1.0000000 mul add 0.5000000 exp add }
%Function formula: - (- (x- \pi)^{2}+1)^{1/2}+1 
\psplot[linecolor=\psColorTangent, plotpoints=1000]{2.141603}{4.141583}{1.0000000 1.0000000 3.141592654 -1.0000000 mul x add 2.0000000 exp -1.0000000 mul add 0.5000000 exp -1.0000000 mul add }
\psFullDot{3.141593}{2}
\rput[t](3.141593, 1.9){$P$}
}

\uncover<4>{
%Function formula: (- (x-3/2 \pi)^{2}+1)^{1/2}+1 
\psplot[linecolor=\psColorTangent, plotpoints=1000]{3.712399}{5.712379}{1.0000000 1.0000000 3.141592654 -1.5000000 mul x add 2.0000000 exp -1.0000000 mul add 0.5000000 exp add }
%Function formula: - (- (x-3/2 \pi)^{2}+1)^{1/2}+1 
\psplot[linecolor=\psColorTangent, plotpoints=1000]{3.712399}{5.712379}{1.0000000 1.0000000 3.141592654 -1.5000000 mul x add 2.0000000 exp -1.0000000 mul add 0.5000000 exp -1.0000000 mul add }
\psFullDot{5.712389}{1}
\rput[r](5.612389, 1){$P$}
}

\uncover<5>{
%Function formula: - (- (x-2 \pi)^{2}+1)^{1/2}+1 
\psplot[linecolor=\psColorTangent, plotpoints=1000]{5.283195}{7.283175}{1.0000000 1.0000000 3.141592654 -2.0000000 mul x add 2.0000000 exp -1.0000000 mul add 0.5000000 exp -1.0000000 mul add }
%Function formula: (- (x-2 \pi)^{2}+1)^{1/2}+1 
\psplot[linecolor=\psColorTangent, plotpoints=1000]{5.283195}{7.283175}{1.0000000 1.0000000 3.141592654 -2.0000000 mul x add 2.0000000 exp -1.0000000 mul add 0.5000000 exp add }
\psFullDot{6.283185}{0}
\rput[lb](6.283185, 0.1){$P$}
}

\uncover<6>{
%Function formula: (- (x-5/2 \pi)^{2}+1)^{1/2}+1 
\psplot[linecolor=\psColorTangent, plotpoints=1000]{6.853992}{8.853972}{1.0000000 1.0000000 3.141592654 -2.5000000 mul x add 2.0000000 exp -1.0000000 mul add 0.5000000 exp add }
%Function formula: - (- (x-5/2 \pi)^{2}+1)^{1/2}+1 
\psplot[linecolor=\psColorTangent, plotpoints=1000]{6.853992}{8.853972}{1.0000000 1.0000000 3.141592654 -2.5000000 mul x add 2.0000000 exp -1.0000000 mul add 0.5000000 exp -1.0000000 mul add }
\psFullDot{6.853982}{1}
\rput[l](6.953982, 1.000000){$P$}
}

\uncover<7>{
%Function formula: (- (x-3 \pi)^{2}+1)^{1/2}+1 
\psplot[linecolor=\psColorTangent, plotpoints=1000]{8.424788}{10.424768}{1.0000000 1.0000000 3.141592654 -3.0000000 mul x add 2.0000000 exp -1.0000000 mul add 0.5000000 exp add }
%Function formula: - (- (x-3 \pi)^{2}+1)^{1/2}+1 
\psplot[linecolor=\psColorTangent, plotpoints=1000]{8.424788}{10.424768}{1.0000000 1.0000000 3.141592654 -3.0000000 mul x add 2.0000000 exp -1.0000000 mul add 0.5000000 exp -1.0000000 mul add }
\psFullDot{9.424778}{2}
\rput[t](9.424778, 1.9){$P$}
}

\uncover<8>{
%Function formula: - (- (x-7/2 \pi)^{2}+1)^{1/2}+1 
\psplot[linecolor=\psColorTangent, plotpoints=1000]{9.995584}{11.995564}{1.0000000 1.0000000 3.141592654 -3.5000000 mul x add 2.0000000 exp -1.0000000 mul add 0.5000000 exp -1.0000000 mul add }
%Function formula: (- (x-7/2 \pi)^{2}+1)^{1/2}+1 
\psplot[linecolor=\psColorTangent, plotpoints=1000]{9.995584}{11.995564}{1.0000000 1.0000000 3.141592654 -3.5000000 mul x add 2.0000000 exp -1.0000000 mul add 0.5000000 exp add }
\psFullDot{11.995574}{1}
\rput[r](11.895574, 1.000000){$P$}
}

\uncover<9>{
%Function formula: (- (x-4 \pi)^{2}+1)^{1/2}+1 
\psplot[linecolor=\psColorTangent, plotpoints=1000]{11.566381}{13.566361}{1.0000000 1.0000000 3.141592654 -4.0000000 mul x add 2.0000000 exp -1.0000000 mul add 0.5000000 exp add }
%Function formula: - (- (x-4 \pi)^{2}+1)^{1/2}+1 
\psplot[linecolor=\psColorTangent, plotpoints=1000]{11.566381}{13.566361}{1.0000000 1.0000000 3.141592654 -4.0000000 mul x add 2.0000000 exp -1.0000000 mul add 0.5000000 exp -1.0000000 mul add }
\psFullDot{12.566371}{0}
\rput[lb](12.566371, 0.1){$P$}
}

%Calculator input:f{}{{p}}:=plotCurve(t+\cos(- t+3\pi/2), 1+\sin(3\pi/2- t),p, p+\pi/4)+ plotCurve(t+\cos(- t+3\pi/2), 1+\sin(3\pi/2- t), p+\pi/4, p+\pi/2); f{}0+f{}(0.5\pi)+ f{}(\pi)+f{}(1.5\pi)+f{}(2\pi)+f{}(2.5\pi)+f{}(3\pi)+f{}(3.5\pi)+f{}(4\pi)


\uncover<2->{
%Calculator input: plotCurve{}(\cos{}(- t+3/2 \pi)+t, \sin{}(- t+3/2 \pi)+1, 0, 1/4 \pi)
\parametricplot[linecolor=\psColorGraph, arrows=->, plotpoints=1000]{0}{0.785398}{t 3.141592654 1.5000000 mul t -1.0000000 mul add 57.29578 mul cos add 1.0000000 3.141592654 1.5000000 mul t -1.0000000 mul add 57.29578 mul sin add }
%Calculator input: plotCurve{}(\cos{}(- t+3/2 \pi)+t, \sin{}(- t+3/2 \pi)+1, 1/4 \pi, 1/2 \pi)
\parametricplot[linecolor=\psColorGraph, plotpoints=1000]{ 0.785398}{1.5708}{t 3.141592654 1.5000000 mul t -1.0000000 mul add 57.29578 mul cos add 1.0000000 3.141592654 1.5000000 mul t -1.0000000 mul add 57.29578 mul sin add }
}
\uncover<3->{
%Calculator input: plotCurve{}(\cos{}(- t+3/2 \pi)+t, \sin{}(- t+3/2 \pi)+1, 1/2 \pi, 3/4 \pi)
\parametricplot[linecolor=\psColorGraph, arrows=->, plotpoints=1000]{1.5708}{2.35619}{t 3.141592654 1.5000000 mul t -1.0000000 mul add 57.29578 mul cos add 1.0000000 3.141592654 1.5000000 mul t -1.0000000 mul add 57.29578 mul sin add }
%Calculator input: plotCurve{}(\cos{}(- t+3/2 \pi)+t, \sin{}(- t+3/2 \pi)+1, 3/4 \pi, \pi)
\parametricplot[linecolor=\psColorGraph, plotpoints=1000]{2.35619}{3.14159}{t 3.141592654 1.5000000 mul t -1.0000000 mul add 57.29578 mul cos add 1.0000000 3.141592654 1.5000000 mul t -1.0000000 mul add 57.29578 mul sin add }
}
\uncover<4->{
%Calculator input: plotCurve{}(\cos{}(- t+3/2 \pi)+t, \sin{}(- t+3/2 \pi)+1, \pi, 5/4 \pi)
\parametricplot[linecolor=\psColorGraph, arrows=->, plotpoints=1000]{3.14159}{3.92699}{t 3.141592654 1.5000000 mul t -1.0000000 mul add 57.29578 mul cos add 1.0000000 3.141592654 1.5000000 mul t -1.0000000 mul add 57.29578 mul sin add }
%Calculator input: plotCurve{}(\cos{}(- t+3/2 \pi)+t, \sin{}(- t+3/2 \pi)+1, 5/4 \pi, 3/2 \pi)
\parametricplot[linecolor=\psColorGraph, plotpoints=1000]{3.92699}{4.71239}{t 3.141592654 1.5000000 mul t -1.0000000 mul add 57.29578 mul cos add 1.0000000 3.141592654 1.5000000 mul t -1.0000000 mul add 57.29578 mul sin add }
}
\uncover<5->{
%Calculator input: plotCurve{}(\cos{}(- t+3/2 \pi)+t, \sin{}(- t+3/2 \pi)+1, 3/2 \pi, 7/4 \pi)
\parametricplot[linecolor=\psColorGraph, arrows=->, plotpoints=1000]{4.71239}{5.49779}{t 3.141592654 1.5000000 mul t -1.0000000 mul add 57.29578 mul cos add 1.0000000 3.141592654 1.5000000 mul t -1.0000000 mul add 57.29578 mul sin add }
%Calculator input: plotCurve{}(\cos{}(- t+3/2 \pi)+t, \sin{}(- t+3/2 \pi)+1, 7/4 \pi, 2 \pi)
\parametricplot[linecolor=\psColorGraph, plotpoints=1000]{5.49779}{6.28319}{t 3.141592654 1.5000000 mul t -1.0000000 mul add 57.29578 mul cos add 1.0000000 3.141592654 1.5000000 mul t -1.0000000 mul add 57.29578 mul sin add }
}
\uncover<6->{
%Calculator input: plotCurve{}(\cos{}(- t+3/2 \pi)+t, \sin{}(- t+3/2 \pi)+1, 2 \pi, 9/4 \pi)
\parametricplot[linecolor=\psColorGraph, arrows=->, plotpoints=1000]{6.28319}{7.06858}{t 3.141592654 1.5000000 mul t -1.0000000 mul add 57.29578 mul cos add 1.0000000 3.141592654 1.5000000 mul t -1.0000000 mul add 57.29578 mul sin add }
%Calculator input: plotCurve{}(\cos{}(- t+3/2 \pi)+t, \sin{}(- t+3/2 \pi)+1, 9/4 \pi, 5/2 \pi)
\parametricplot[linecolor=\psColorGraph, plotpoints=1000]{7.06858}{7.85398}{t 3.141592654 1.5000000 mul t -1.0000000 mul add 57.29578 mul cos add 1.0000000 3.141592654 1.5000000 mul t -1.0000000 mul add 57.29578 mul sin add }
}
\uncover<7->{
%Calculator input: plotCurve{}(\cos{}(- t+3/2 \pi)+t, \sin{}(- t+3/2 \pi)+1, 5/2 \pi, 11/4 \pi)
\parametricplot[linecolor=\psColorGraph, arrows=->, plotpoints=1000]{7.85398}{8.63938}{t 3.141592654 1.5000000 mul t -1.0000000 mul add 57.29578 mul cos add 1.0000000 3.141592654 1.5000000 mul t -1.0000000 mul add 57.29578 mul sin add }
%Calculator input: plotCurve{}(\cos{}(- t+3/2 \pi)+t, \sin{}(- t+3/2 \pi)+1, 11/4 \pi, 3 \pi)
\parametricplot[linecolor=\psColorGraph, plotpoints=1000]{8.63938}{9.42478}{t 3.141592654 1.5000000 mul t -1.0000000 mul add 57.29578 mul cos add 1.0000000 3.141592654 1.5000000 mul t -1.0000000 mul add 57.29578 mul sin add }
}
\uncover<8->{
%Calculator input: plotCurve{}(\cos{}(- t+3/2 \pi)+t, \sin{}(- t+3/2 \pi)+1, 3 \pi, 13/4 \pi)
\parametricplot[linecolor=\psColorGraph, arrows=->, plotpoints=1000]{9.42478}{10.2102}{t 3.141592654 1.5000000 mul t -1.0000000 mul add 57.29578 mul cos add 1.0000000 3.141592654 1.5000000 mul t -1.0000000 mul add 57.29578 mul sin add }
%Calculator input: plotCurve{}(\cos{}(- t+3/2 \pi)+t, \sin{}(- t+3/2 \pi)+1, 13/4 \pi, 7/2 \pi)
\parametricplot[linecolor=\psColorGraph, plotpoints=1000]{10.2102}{10.9956}{t 3.141592654 1.5000000 mul t -1.0000000 mul add 57.29578 mul cos add 1.0000000 3.141592654 1.5000000 mul t -1.0000000 mul add 57.29578 mul sin add }
}
\uncover<9->{
%Calculator input: plotCurve{}(\cos{}(- t+3/2 \pi)+t, \sin{}(- t+3/2 \pi)+1, 7/2 \pi, 15/4 \pi)
\parametricplot[linecolor=\psColorGraph, arrows=->, plotpoints=1000]{10.9956}{11.781}{t 3.141592654 1.5000000 mul t -1.0000000 mul add 57.29578 mul cos add 1.0000000 3.141592654 1.5000000 mul t -1.0000000 mul add 57.29578 mul sin add }
%Calculator input: plotCurve{}(\cos{}(- t+3/2 \pi)+t, \sin{}(- t+3/2 \pi)+1, 15/4 \pi, 4 \pi)
\parametricplot[linecolor=\psColorGraph, plotpoints=1000]{11.781}{12.5664}{t 3.141592654 1.5000000 mul t -1.0000000 mul add 57.29578 mul cos add 1.0000000 3.141592654 1.5000000 mul t -1.0000000 mul add 57.29578 mul sin add }
}
\uncover<10->{
%Calculator input: plotCurve{}(\cos{}(- t+3/2 \pi)+t, \sin{}(- t+3/2 \pi)+1, 4 \pi, 17/4 \pi)
\parametricplot[linecolor=\psColorGraph, arrows=->, plotpoints=1000]{12.5664}{13.3518}{t 3.141592654 1.5000000 mul t -1.0000000 mul add 57.29578 mul cos add 1.0000000 3.141592654 1.5000000 mul t -1.0000000 mul add 57.29578 mul sin add }
%Calculator input: plotCurve{}(\cos{}(- t+3/2 \pi)+t, \sin{}(- t+3/2 \pi)+1, 17/4 \pi, 9/2 \pi)
\parametricplot[linecolor=\psColorGraph, plotpoints=1000]{13.3518}{14.1372}{t 3.141592654 1.5000000 mul t -1.0000000 mul add 57.29578 mul cos add 1.0000000 3.141592654 1.5000000 mul t -1.0000000 mul add 57.29578 mul sin add }
}
\end{pspicture} 


%\ \only<handout:0| 1>{%
%\includegraphics[width=12cm]{parametric-curves/pictures/11-01-cycloida.pdf}%
%}%
%\only<handout:0| 2>{%
%\includegraphics[width=12cm]{parametric-curves/pictures/11-01-cycloidb.pdf}%
%}%
%\only<handout:0| 3>{%
%\includegraphics[width=12cm]{parametric-curves/pictures/11-01-cycloidc.pdf}%
%}%
%\only<handout:0| 4>{%
%\includegraphics[width=12cm]{parametric-curves/pictures/11-01-cycloidd.pdf}%
%}%
%\only<handout:0| 5>{%
%\includegraphics[width=12cm]{parametric-curves/pictures/11-01-cycloide.pdf}%
%}%
%\only<handout:0| 6>{%
%\includegraphics[width=12cm]{parametric-curves/pictures/11-01-cycloidf.pdf}%
%}%
%\only<handout:0| 7>{%
%\includegraphics[width=12cm]{parametric-curves/pictures/11-01-cycloidg.pdf}%
%}%
%\only<8>{%
%\includegraphics[width=12cm]{parametric-curves/pictures/11-01-cycloidh.pdf}%
%}%
%\only<handout:0| 9>{%
%\includegraphics[width=12cm]{parametric-curves/pictures/11-01-cycloidi.pdf}%
%}%
%\only<handout:0| 10->{%
%\includegraphics[width=12cm]{parametric-curves/pictures/11-01-cycloidj.pdf}%
%}%
\begin{definition}[Cycloid]
The curve traced out by a point $P$ on the circumference of a circle as the circle rolls along a straight line is called a cycloid. \uncover<10>{}
\end{definition}
\end{frame}
% end module cycloid-def

%% begin module cycloid-equations-ex7
\begin{frame}
\begin{example} %[Example 7, p. 660]
Find parametric equations of a cycloid made using a circle with radius $r$ that rolls along the $x$-axis such that $P$ hits the origin.
\begin{columns}[c]
\column{.4\textwidth}

\psset{xunit=1.6cm, yunit=1.6cm}
\begin{pspicture}( -0.6, -0.6)(2.65,2.3)
\psframe*[linecolor=white](-0.6, -0.6)(2.65,2.3)
\tiny%
\psaxes[arrows=<->, ticks=none, labels=none ](0,0)( -0.500000, -0.5)(2.55,2.2)%
\parametricplot[linecolor=\fcColorTangent, plotpoints=1000, algebraic=true]{0}{6.283185307}{cos(t)+1.256637061|sin(t)+1}%
\uncover<2->{%
\psplot[linecolor=green, plotpoints=300]{0.305581} {1.256637061} {1 1 -1.25664 x add 2 exp -1 mul add 0.5 exp -1 mul add }%
}%
%Calculator input: plotCurve{}(- \sin{}t+t, - \cos{}t+1, 0, \pi)
\parametricplot[linecolor=\fcColorGraph, plotpoints=1000] { 0}{2.8}{t t 57.29578 mul sin -1 mul add 1 t 57.29578 mul cos -1 mul add }

\fcFullDot{0.305581}{0.690983}
\rput[r](0.2, 0.69){$P$}
\fcFullDot{1.256637061}{1}
\rput[bl](1.286637061,1.05){\uncover<5->{\alertNoH{5}{$ C=(r\theta,r)$}}}

\uncover<6->{\psline[linestyle=dashed](0.305581,0)(0.305581,0.690983)
\rput[b](0.15, 0.05){\alertNoH{6}{$x$}}
\rput[l](0.32, 0.35){\alertNoH{6}{$y$}}
}

\psline(0.305581,0.690983)(1.256637061,1)
\rput[b](0.75, 0.85){$r$}

\uncover<7->{
\psline[linestyle=dashed](0.305581,0.690983)(1.256637061,0.690983)
\psline(1.156637061,0.690983)(1.156637061,0.590983)(1.256637061,0.590983)
\rput[l](1.306637061,0.690983){$Q$}
}

\psline(1.256637061,0.1)(1.356637061,0.1)(1.356637061,0)
\rput[lt](1.256637061,-0.1){$T$}
\psline(1.256637061,1)(1.256637061,0)

\uncover<3->{\psline[linecolor=green](0,0)(1.256637061, 0)}

\uncover<4->{
\psline{<-}(0,-0.2)(0.5, -0.2)
\psline{->}(0.72,-0.2)(1.256637061, -0.2)
\rput(0.62, -0.2){\alertNoH{4}{$r\theta$}}
}
\rput(1.2, 0.9){\uncover<2->{\alertNoH{2}{$\theta$}}}
\rput (-0.2, -0.2){$O$}
\end{pspicture}

%\ \only<handout:0| -2>{%
%\includegraphics[width=5cm]{parametric-curves/pictures/11-01-cycloideqa.pdf}%
%}%
%\only<handout:0| 3>{%
%\includegraphics[width=5cm]{parametric-curves/pictures/11-01-cycloideqb.pdf}%
%}%
%\only<handout:0| 4>{%
%\includegraphics[width=5cm]{parametric-curves/pictures/11-01-cycloideqc.pdf}%
%}%
%\only<handout:0| 5>{%
%\includegraphics[width=5cm]{parametric-curves/pictures/11-01-cycloideqd.pdf}%
%}%
%\only<handout:0| 6>{%
%\includegraphics[width=5cm]{parametric-curves/pictures/11-01-cycloideqe.pdf}%
%}%
%\only<7->{%
%\includegraphics[width=5cm]{parametric-curves/pictures/11-01-cycloideqf.pdf}%
%}%
\column{.6\textwidth}
\begin{itemize}
\item<2->  We choose our parameter to be \alertNoH{2}{$\theta$}, the angle of rotation of the circle.
\item<3->  How far has the circle moved if it has rolled through $\theta$ radians?
\abovedisplayskip=0pt
\belowdisplayskip=0pt
\[
\uncover<3->{%
{|OT|} = \alertNoH{ 4}{{ \textrm{arc} PT} }%
}%
\uncover<4->{%
\alertNoH{ 4}{ = r\theta}%
}%
\]
\item<5->  Then the center is $\alertNoH{5}{C = (r\theta , r)}$.
\item<6->  Let the coordinates of $P$ be $(x,y)$.
\end{itemize}
\[
\begin{array}{cccccc}
\uncover<6->{%
\alertNoH{ 8}{x}%
}&%
\uncover<6->{%
\alertNoH{ 8}{=}%
}&%
\uncover<8->{%
\alertNoH{ 8}{\alertNoH{ 9-10}{|OT|} - \alertNoH{ 11-12}{|PQ|}}%
}&%
\uncover<9->{%
=%
}&%
\uncover<10->{%
\alertNoH{ 10}{r\theta}%
}%
\uncover<9->{-}%
\uncover<12->{%
\alertNoH{ 12}{r\sin \theta}%
}\\%

\uncover<6->{%
\alertNoH{ 13}{y}%
}&%
\uncover<6->{%
\alertNoH{ 13}{=}%
}&%
\uncover<13->{%
\alertNoH{ 13}{\alertNoH{ 14-15}{|CT|} - \alertNoH{ 16-17}{|CQ|}}%
}&%
\uncover<14->{%
=%
}&%
\uncover<15->{%
\alertNoH{ 15}{r}%
}%
\uncover<14->{-}%
\uncover<17->{%
\alertNoH{ 17}{r\cos \theta}%
}\\%
\end{array}
\]
\end{columns}
\uncover<18->{%
Therefore the equations are
\abovedisplayskip=0pt
\belowdisplayskip=0pt
\[
x = r(\theta - \sin \theta ),\qquad y = r(1-\cos \theta ),\qquad \theta \in \mathbb{R}
\]
}%
\end{example}
\end{frame}
% end module cycloid-equations-ex7

%% begin module polar-intro
\begin{frame}
\frametitle{Polar Coordinates}
\begin{itemize}
\item<1->  The polar coordinates system is an alternative to the Cartesian coordinates system.
%\item  Instead of specifying horizontal distance and vertical distance, we specify angle and distance from the origin.
\item<2->  Choose a point in the plane called $O$ (the origin).
\item<3->  Draw a ray starting at $O$ called the polar axis.  This ray is usually drawn horizontally to the right.
\end{itemize}
\begin{columns}[c]
\column{.5\textwidth}
\psset{xunit=5cm, yunit=5cm}
\begin{pspicture}(-1.000000, -5)(5.500000,5) 
\psframe*[linecolor=white](-1.000000,-5)(5.500000,5) 
\tiny 
%force a boudning box:
\psline[linecolor=red!1](-0.1, -0.1)(-0.21,0.2)
\psline[linecolor=red!1](1.1, 0.6)(1.1,0.61)

\uncover<5->{
%Calculator input: plotCurve{}(1/5 \cos{}t, 1/5 \sin{}t, 0, 1/6 \pi)
\parametricplot[arrows=->, linecolor=\psColorGraph, plotpoints=1000] {0}{0.523599}{t 57.29578 mul cos 0.2 mul t 57.29578 mul sin 0.2 mul }
\psline[linecolor=blue](0,0)(0.866025404, 0.5)
\rput(0.22, 0.06){$\alert<5>{\theta}$}
}

\uncover<4->{
\psFullDotBlue{0.866025404}{0.5}
\rput[l](0.88, 0.5){$P\uncover<7->{\alert<7>{(r,\theta)}}$}
}

\uncover<2->{
\psFullDotBlue{0}{0}
\rput(-0.1, 0){$O$}
}
\uncover<6->{
\rput(0.4, 0.3){$\alert<6>{r}$}
}
\uncover<3->{
\psline{->}(0,0)(1,0)
\rput (0.5, -0.05){polar axis}
}
\end{pspicture} 

%\includegraphics[height=4cm]{polar-curves/pictures/11-03-polar.pdf}%
\column{.5\textwidth}
\begin{itemize}
\item<4->  Let $P$ be a point in the plane.
\item<5->  Let $\theta$ denote the angle between the polar axis and the line $OP$.
\item<6->  Let $r$ denote the length of the segment $OP$.
\item<7->  Then $P$ is represented by the ordered pair $(r, \theta )$.
\end{itemize}
\end{columns}
\end{frame}
% end module polar-intro


%%begin module area-under-hyperbola-ex1


\begin{frame}
\begin{example}
Find the area locked b-n the hyperbolas $\alert<2,3>{ y=\pm \sqrt{ x^2+1}}$ and $x=\pm 2\sqrt{ 2}$.
\begin{columns}
\column{.5\textwidth}
\psset{xunit=0.7cm, yunit=0.7cm}
\begin{pspicture}(-3.328427, -3)(3.328427,3)
\psframe*[linecolor=white](-3.328427,-3)(3.328427,3)
\tiny
\uncover<31->{
\pscustom*[linecolor=\fcColorAreaUnderGraph]{
\psplot[linecolor=\fcColorGraph, plotpoints = 1000 ] {-2.828427} {2.828427}{1 x 2 exp add 0.5 exp }
\psline[linecolor=\fcColorGraph](2.828427,-3)(2.828427,3)
\psplot[linecolor=\fcColorGraph, plotpoints=1000] { 2.828427 } {-2.828427}{1 x 2 exp add 0.5 exp -1 mul }
\psline[linecolor=\fcColorGraph](-2.828427,-3)(-2.828427,3)
}
}
\uncover<1-26,28->{
\psaxes[arrows=<->,ticks=none, labels=none](0,0)(-3,-3)(3,3)
}
\psline[linecolor=red!1](3.301,2)(3.302,2)
\psline[linecolor=red!1](-3.301,2)(-3.302,2)

%Function formula: - (x^{2}+1)^{1/2}
\psplot[linecolor=\fcColorGraph, plotpoints=1000]{-2.828427}{2.828427}{1 x 2 exp add 0.5 exp -1 mul }
\uncover<3-4>{\rput[tl](-2.2, -2.4){ \alert<3>{ $y= - \sqrt{ x^2 +1 }$}}}

%Function formula: (x^{2}+1)^{1/2}
\psplot[linecolor=\fcColorGraph, plotpoints=1000]{-2.828427}{ 2.828427 }{1 x 2 exp add 0.5 exp }
\uncover<2-4>{\rput[bl](-2.1, 2.4){\alert<2>{ $y=\sqrt{ x^2 +1} $}}}

\uncover<29->{
\psline[linecolor=\fcColorGraph](-2.828427,3)(-2.828427,-3)
}
\uncover<30->{
\psline[linecolor=\fcColorGraph](2.828427,3)(2.828427,-3)
}
\uncover<25-27>{
\psline{<->}(-2.9,2.9)(2.9,-2.9)
\rput[t](-2.1, 1.7){$\begin{array}{l} \alert<25>{v=0} \\\uncover<1-26>{\alert<25>{y+x=0}} \end{array}$}
}
\uncover<15-27>{
\psline{<->}(-2.9,-2.9)(2.9,2.9)
\rput[b](-2.1, -1.9){$\begin{array}{l} \uncover<1-26>{ \alert<15>{ y-x=0 }}\\\uncover<16->{\alert<16>{u=0}} \end{array}$}
}
\uncover<17-26>{
\fcFullDot{1.4}{1.4}
\rput[l]( 1.6, 1.4){$(\frac{y+x}{2},\frac{y+x}{2})$}
}
\uncover<14-26>{
\fcFullDot{0.6}{2.2}
\rput[lb](0.65, 2.2){$(x,y)$}
}
\uncover<26>{
\psline(0.6,2.2)(-0.8,0.8)
\psline(-0.7, 0.9)(-0.6, 0.8)(-0.7, 0.7)
\rput[rb](-0.3, 1.3){\alert<26>{$v$}}
}
\uncover<18-26>{
\psline(0.6,2.2)(1.4, 1.4)
\psline(1.3, 1.5)(1.2,1.4)(1.3, 1.3)
}
\uncover<23-26>{
\rput[tr](0.95, 1.8){\alert<23>{$u$}}
}
\uncover<14-26>{
\fcFullDot{2.2}{0.6}
\rput[lt]( 2.2, 0.65){$(y,x)$}
}
\end{pspicture}

\vbox to 3.0cm {
\uncover<18->{\alert<18>{
\uncover<22->{\alert<22>{Signed}} distance b-n $(x,y)$ and line $u=0$ equals}}
\only<1-23>{
$\uncover<19->{\uncover<22->{\alert<22>{\pm}} \alert<19>{ \sqrt{ \alert<20>{ \left(x-\frac{(x+y)}{2} \right)^2+ \left( y- \frac{(x+y )}{2} \right)^2}}}}
$
$\uncover<20->{=\uncover<22->{\alert<22>{\pm}} \sqrt{ \alert<20>{ \frac{1}{2}(y-x)^2 }}} \uncover<21->{= \alert<21>{ \uncover<1-21>{\pm} \alert<23>{ \frac{\sqrt{2 }}{ 2 } ( y-x)}}} \uncover<23>{ \alert<23>{=}}$
} %only<1-23>
\uncover<23->{ \alert<23,24>{$u $}.}
\only<24->{\uncover<25->{
Similarly compute that \alert<26>{signed distance b-n $(x,y)$ and the \alert<25>{line $v=0$} equals $v$}.
\uncover<27->{$\Rightarrow$ $y^2-x^2=1$ is the \alert<27>{ hyperbola $v=\frac{1/2}{v}$} in the $(u,v)$-plane.}
}}

\vfil
} %vbox

\column {.5\textwidth}
\only<1-27>{
\uncover<4->{We studied $\alert<27>{v=\frac{1/2}{u}}$ is called a hyperbola:}\uncover<3->{ why do we call $y= \sqrt{ x^2 +1}$ hyperbola?} \uncover<5->{Compute:}
\[
\begin{array}{rcl}
\uncover<5->{\sqrt{x^2+1} &=& y}\\
\uncover<6->{ x^2+1 &=& y^2}\\
\uncover<7->{y^2-x^2&=&1}\\
\uncover<8->{\uncover<9>{\alert<9>{\frac{1}{2}}} \uncover<10->{\alert<10,11>{\frac{\sqrt{2}}{2}}} \alert<11>{(y-x)} \uncover<10->{\alert<10,12>{\frac{\sqrt{2}}{2}}} \alert<12>{(y+x)}&=&\uncover<9->{\alert<9>{\frac{1}{2}}} \uncover<8>{1}}\\
\uncover<11->{\alert<11>{u}\alert<12>{v}&=& \frac{1}{2}}\\
\uncover<13->{\alert<27>{v}&\alert<27>{=}& \alert<27>{\frac{1/2}{u}},}
\end{array}
\]
\uncover<11->{where $\begin{array}{|l}
\alert<11,16,23>{u=\frac{\sqrt{2}}{2} \left(y-x\right)}\\
\alert<12,25>{v=\frac{\sqrt{2}}{2}\left(y+x\right)}
\end{array}$. } \uncover<14->{Consider an arbitrary point $(x,y)$.}
} %only<1-27>
\only<28->{
The area in question is:
$
\begin{array}{l}
\displaystyle\phantom{=} \int \limits^{{{\uncover<28,29>{\alert<29>{ \textbf{?}}}\uncover<30->{\alert<30>{ 2\sqrt{2}}}}}}_{\uncover<28>{\alert<28>{\textbf{?}}}\uncover<29->{ -2\sqrt{2}}} 2\sqrt{x^2+1}\diff x \\
\displaystyle \uncover<32->{= \uncover<33->{\alert<33>{2}} \left[x\sqrt{x^2+1} \vphantom{\ln \left(\sqrt{x^2+1}+x\right) }\right.}\\
\displaystyle \uncover<32->{\left. \ln \left(\sqrt{x^2+1}+x\right)\right]^{2\sqrt{2}}_{\only<33->{\alert<33>{0}} \uncover<1-32>{-2\sqrt{2}}}}\\
\uncover<34->{=2\left(2\sqrt{2} \sqrt{(2\sqrt{2})^2+1}\right.} \\
\uncover<34->{\left.+ \ln \left(\sqrt{(2\sqrt{2})^2+1}+2\sqrt{2} \right) \right)}\\
\uncover<35->{=12\sqrt{2} +2\ln \left(3+2\sqrt{2}\right )}\\
\uncover<36->{\approx 20.496}
\end{array}
$
}
\end{columns}

\end{example}

\end{frame}

%end module area-under-hyperbola-ex1


 


\end{document}
