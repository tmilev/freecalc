% begin module antiderivatives-ex
\begin{frame}
\begin{example}
\begin{itemize}
\item  Let $f(x) = x^2$.
\item<2-> Use the Power Rule to find an antiderivative of $f$:
\item<2-> If \alert<handout:0| 3-4>{$F(x) = \uncover<4->{\frac{1}{3}x^3}$}, then $F'(x) = x^2 = f(x)$.
\item<5-> Is this the only one?
\item<6-> No.  If $G(x) = \frac{1}{3}x^3 + 1$, then $G'(x) = x^2 = f(x)$.
\item<7-> $\frac{1}{3}x^3 + 2$ will also work.
\item<8-> Any function of the form $H(x) = \frac{1}{3}x^3 + C$, where $C$ is a constant, is an antiderivative of $f$.
\end{itemize}
\end{example}
\end{frame}
% end module antiderivatives-ex
