% begin module arc-length-half
\begin{frame}
The previous example has a property that is common to many arc length questions.

In particular, $y'$ has the form $a - b$, where $2ab = \frac{1}{2}$.
\end{frame}
% end module arc-length-half



Sometimes an arc length problem can be simplified significantly if the function $y = f(x)$ has a certain form.  In particular, if $y' = a - b$, where $2ab = \frac{1}{2}$, then a shortcut is possible.  Consider the following example:

\textbf{Example:}  Find the arc length of $\displaystyle y = \frac{x^7}{14} + \frac{1}{10x^5}$ between $x = 1$ and $x = 2$.

The arc length formula is $L = \int_1^2 \sqrt{1 + (y')^2}\diff x$, so we need to know $(y')^2$.
\begin{eqnarray*}
y' & = & 7\frac{x^6}{14} + (-5)\frac{1}{10x^6}\\
& = & \frac{1}{2} x^6 - \frac{1}{2}x^{-6}
\end{eqnarray*}

Let $a = \frac{1}{2}x^6$ and $b = \frac{1}{2}x^{-6}$.  Then $2ab = 2\left( \frac{1}{2}x^6\right) \left(\frac{1}{2}x^{-6}\right) = \frac{1}{2}$, and, when we square $y'$, this is what we get:
\[
\begin{array}{rcl@{}c@{}rcc@{}c@{}c@{}c@{}ccc@{}c@{}c@{}c@{}c@{}c}
(y')^2 & = & %
\left( \frac{1}{2}x^6 \right. & - & \left. \frac{1}{2}x^{-6}\right)^2 %
 & = & \left( \frac{1}{2}x^6 \right)^2 & - & 2  \left(\frac{1}{2}x^6\right)  \left(\frac{1}{2}x^{-6}\right) & + & \left( \frac{1}{2}x^{-6}\right)^2 %
 & = & \frac{1}{4}x^{12} & - & \frac{1}{2} & + & \frac{1}{4}x^{-12} \\%
 & & & & & & & & & & & & & & & & \\
 &  & %
\left( a \right. & - & \left. b \right)^2 %
 & = & a^2 & - & 2  a  b & + & b^2 %
 & = & a^{2} & - & \frac{1}{2} & + & b^{2} %
\end{array}
\]

The simplification comes after we substitute this into the arc length formula.  Notice that, if $2ab = \frac{1}{2}$, then
\[
\begin{array}{rcc@{\ }c@{\ }c@{\ }c@{\ }ccc@{\ }c@{\ }c@{\ }c@{\ }c}
(a-b)^2 & = & a^2 & - & 2ab & + & b^2 
 & = & a^2 & - & \frac{1}{2} & + & b^2 \\
(a+b)^2 & = & a^2 & + & 2ab & + & b^2 
 & = & a^2 & + & \frac{1}{2} & + & b^2 
\end{array}
\]
and then
\begin{eqnarray*}
1 + (a-b)^2 & = & 1 + \left( a^2 - \frac{1}{2} + b^2\right)\\
& = & a^2 + \frac{1}{2} + b^2 \\
& = & (a+b)^2.
\end{eqnarray*}

Therefore, in the example,
\begin{eqnarray*}
1 + (y')^2 & = & 1 + \left( \frac{1}{2}x^6 - \frac{1}{2}x^{-6}\right)^2\\
 & = & \left( \frac{1}{2}x^6 + \frac{1}{2}x^{-6}\right)^2.
\end{eqnarray*}

Apply this to the arc length formula:
\begin{eqnarray*}
L & = & \int_1^2 \sqrt{1 + (y')^2}\diff x\\
& = & \int_1^2 \sqrt{1 + \left( \frac{1}{2}x^6 - \frac{1}{2}x^{-6}\right)^2}\diff x\\
& = & \int_1^2 \sqrt{\left(\frac{1}{2}x^6 + \frac{1}{2}x^{-6}\right)^2}\diff x\\
& = & \int_1^2 \left( \frac{1}{2}x^6 + \frac{1}{2}x^{-6}\right)\diff x\\
& = & \left[ \frac{1}{2}\cdot \frac{1}{7}x^7 + \frac{1}{2} \cdot \frac{1}{-5}x^{-5}\right]_1^2\\
& = & \left[ \frac{1}{14}x^7 - \frac{1}{10}x^{-5}\right]_1^2\\
& = & \left( \frac{128}{14} - \frac{1}{10}\cdot \frac{1}{32}\right) - \left( \frac{1}{14} - \frac{1}{10}\right) \\
& = & \frac{20537}{2240}.
\end{eqnarray*}

There are several examples like this in the textbook and on WebWork.

In chapter 9.1 of the textbook, exercises 9, 10, and 22 are all of this type.

In the WebWork set \texttt{09-arc-length-S3}, question 4 is of this type.


\end{document}
