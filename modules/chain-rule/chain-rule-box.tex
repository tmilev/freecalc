% begin module chain-rule-box
\begin{frame}
\frametitle{How to Think About the Chain Rule}
\begin{itemize}
\item  Always ask ``what is the last operation you do?''
\item  Everything else gets put in a box $\square$, and you apply that operation to $\square$.
\item  To take the derivative: 
\begin{enumerate}
\item  Take the derivative of the last operation.
\item  Evaluate it on the box.
\item  Multiply by the derivative of the box.
\end{enumerate}
\end{itemize}
\uncover<2->{%
\begin{example}
Differentiate $y = \tan (x^3)$.
\begin{itemize}
\item<3-| alert@4-5>  Last operation:  \uncover<5->{tangent.}
\item<3-| alert@6-7,10>  $\square = \uncover<7->{x^3}$\uncover<7->{.}
\end{itemize}
\[
\uncover<8->{%
y' = \left(\tan \square \right)'%
}%
\uncover<9->{%
 = \left( \sec^2 \alert<handout:0| 10>{\square} \right) \alert<handout:0| 10>{\square}'%
}%
\uncover<10->{%
 = \left( \sec^2 ( \alert<handout:0| 10>{x^3} )\right) \alert<handout:0| 10-11>{(x^3)}\alert<handout:0| 11>{'}%
}%
\uncover<11->{%
 = \alert<handout:0| 11>{3x^2} \sec^2 \left( x^3 \right) 
}%
\]
\end{example}
}%
\end{frame}
% end module chain-rule-box
