\solution{\ref{problemd/dx(sqrt(3x^2-x+2))}

$\begin{array}{rcl}
\displaystyle \frac{\diff}{\diff x}\left(\sqrt{3 x^{2}-x+2}\right)&=&\displaystyle \frac{(3 x^{2}-x+2)'}{2\sqrt{3 x^{2}-x+2}}=\frac{6x-1}{2\sqrt{3 x^{2}-x+2}}.
\end{array}
$
}
\solution{\ref{problemDifferentialtexDivsqrt(1+2divx^2)}
\[
\begin{array}{rclr|r}
\displaystyle\left(\frac{x }{\sqrt{1+\frac{2}{x^2}}}\right)'&=&\displaystyle\frac{\sqrt{1+\frac{2}{x^2}}- x\left(\sqrt{1+\frac{2}{x^2}}\right)'}{1+\frac{2}{x^2}} =\frac{\sqrt{1+\frac{2}{x^2}}-  x\frac{\frac12}{\sqrt{1 +\frac{ 2}{ x^2 }}}  \left(\frac{2}{x^2}\right)'}{1+\frac{2}{x^2}}\\
&=& \displaystyle\frac{\sqrt{1+\frac{2}{x^2}}+  \frac{2}{x^2\sqrt{1 +\frac{ 2}{ x^2 }}} }{1+\frac{2}{x^2}} = \frac{x^2\left(1+\frac{2}{x^2}\right)+  2 }{x^2\left(1+\frac{2}{x^2}\right)^{\frac32}}= \frac{x^2+4}{x^2\left(1+\frac{2}{x^2}\right)^{\frac32}}
\end{array}
\]
Please note that this problem can be solved also by applying the transformation 
\[
\displaystyle  \frac{x}{\sqrt{1+\frac{2}{x^2}}}= \frac{x}{\sqrt{\frac{x^2+2}{x^2}}} =\frac{x}{\frac{1}{\pm x}\sqrt{x^2+2}} = \frac{\pm x^2}{\sqrt{x^2+2}}
\]
before differentiating, however one must not forget the $\pm $ sign arising from $\sqrt{x^2}=\pm x$. Our original approach resulted in more algebra, but did not have the disadvantage of dealing with the $\pm$ sign.
}

\solution{\ref{problemd/dxsqrt(1-sqrt(x))}

\[
\begin{array}{rcll|l}
\displaystyle \frac{\diff }{\diff x}\left(\sqrt{1-\sqrt{x}}\right)&= &\displaystyle \frac{\diff }{\diff x}\left(\left(1-x^{\frac{1}{2}} \right)^{ \frac{ 1}{2}}\right)&&\text{chain rule}\\
&=&\displaystyle \frac{1}{2}\left(1-x^{\frac{1}{ 2}}\right)^{- \frac{1}{ 2}}\frac{\diff }{\diff x}\left(1-x^{\frac{1}{2}}\right)\\
&=&\displaystyle -\frac{1}{4}x^{-\frac{1}{2}} \left(1-x^{\frac{1}{ 2}}\right)^{- \frac{1}{ 2}}
\end{array}
\]
}

\solution{ \ref{problemd/dx((cosx)^(1/2))}%
\begin{align*}
\text{Let } \quad u & = \cos x. \\
\text{Then } \quad y & = u^{\frac{1}{2}}. \\
\text{Chain Rule: } \quad \frac{\diff y}{\diff x} & = \frac{\diff y}{\diff u}\frac{\diff u}{\diff x} \\
 & = \left(\frac{1}{2}u^{-\frac{1}{2}}\right) (-\sin x) \\
 & = -\frac{1}{2} \sin x (\cos x)^{-\frac{1}{2}}.
\end{align*}
}%

\solution{\ref{problemd/dx((1+cosx)^2)} %
\begin{align*}
\text{Let } \quad u & = 1+\cos x. \\
\text{Then } \quad y & = u^{2}. \\
\text{Chain Rule: } \quad \frac{\diff y}{\diff x} & = \frac{\diff y}{\diff u}\frac{\diff u}{\diff x} \\
 & = (2u) (-\sin x) \\
 & = -2 \sin x (1+\cos x) \\
 & = -2\sin x -2 \sin x \cos x \\
 & = -2\sin x -\sin (2x). \quad \text{(This last step is optional.)}
\end{align*}
}%

\solution{\ref{problemd/dx(sin(sqrt(x)))} %
\begin{align*}
\text{Let } \quad u & = \sqrt{x}. \\
\text{Then } \quad y & = \sin u. \\
\text{Chain Rule: } \quad \frac{\diff y}{\diff x} & = \frac{\diff y}{\diff u}\frac{\diff u}{\diff x} \\
 & = (\cos u) \left(\frac{1}{2}u^{-\frac{1}{2}}\right) \\
 & = \frac{\cos\left(\sqrt{x}\right) }{2\sqrt{x}}.
\end{align*}
}%

\solution{\ref{problemd/dx(e^(-1/x))}
\[
\begin{array}{rcll|l}
\frac{\diff }{\diff x}\left(e^{-\frac{1}{x}}\right)&=&e^{-\frac{1}{x}} \frac{\diff }{\diff x}\left(-\frac{1}{x}\right)&&\text{chain rule}\\
&=&\displaystyle -e^{-\frac{1}{x}} \frac{\diff }{\diff x} \left(x^{-1}\right)\\
&=&\displaystyle x^{-2}e^{-\frac{1}{x}}\\
&=&\displaystyle \frac{e^{-\frac{1}{x}}}{x^2}
\end{array}
\]
}

\solution{\ref{problemd/dx(sqrt(sec(4x)))} %
\begin{align*}
\text{Chain Rule: } \quad \frac{\diff y}{\diff x} & = \left( \frac{1}{2}(\sec (4x))^{-\frac{1}{2}} \right) \frac{\diff}{\diff x}(\sec (4x)) \\
\text{Chain Rule: } \quad \frac{\diff y}{\diff x} & = \left( \frac{1}{2\sqrt{\sec (4x)}} \right) (\sec (4x) \tan (4x))\frac{\diff}{\diff x}(4x) \\
 & = \left( \frac{1}{2\sqrt{\sec (4x)}}\right) (\sec (4x) \tan (4x))(4) \\
 & =  \frac{2\sec (4x)\tan (4x)}{\sqrt{\sec (4x)}} \\
\intertext{There are many ways to simplify this answer, including both of the following.}
 & =  2\sqrt{\sec (4x)}\tan (4x). \\
 & =  2(\sec (4x))^{\frac{3}{2}}\sin (4x). 
\end{align*}
}%

\solution{\ref{problemd/dx(x^2tan(5x))} %
\begin{align*}
\text{Product Rule: } \quad \frac{\diff y}{\diff x} & = (x^2)\frac{\diff}{\diff x}(\tan (5x)) + (\tan (5x))\frac{\diff}{\diff x}(x^2) \\
\intertext{Use the Chain Rule to differentiate $\tan (5x)$ in the first term.}
\frac{\diff y}{\diff x} & = (x^2)(-5\sec^2 (5x) + (\tan (5x))(2x) \\
 & = 2x\tan (5x) - 5x^2\sec^2 (5x).
\end{align*}
}%


\solution{\ref{problemd/dx((1+sin(x^2))/(1+cos(x^2)))} %
\begin{align*}
\text{Quotient Rule: } \quad \frac{\diff y}{\diff x} & = \frac{\left(1+ \cos \left( x^2 \right)\right)\frac{\diff}{\diff y}(1+\sin \left(x^2\right) ) - (1+\sin \left(x^2\right))\frac{\diff}{\diff x}(1+\cos \left(x^2\right))}{(1+\cos \left(x^2\right))^2} \\
\intertext{By the Chain Rule, $\frac{\diff}{\diff x}(1+\cos \left(x^2\right)) = -2x\sin \left(x^2\right)$ and $\frac{\diff}{\diff x}(1+\sin \left(x^2\right)) = 2x\cos \left(x^2\right)$.}
\frac{\diff y}{\diff x} & = \frac{(1+\cos \left(x^2\right))(2x\cos \left(x^2\right)) - (1+\sin \left(x^2\right))(-2x\sin \left(x^2\right))}{(1+\cos \left(x^2\right))^2} \\
 & = \frac{2x\cos \left(x^2\right) + 2x\cos^2 \left(x^2\right) + 2x\sin \left(x^2\right) + 2x\sin^2 \left(x^2\right)}{(1+\cos \left(x^2\right))^2} \\
 & = \frac{2x(\cos^2 \left(x^2\right) + \sin^2 \left(x^2\right)) + 2x(\cos \left(x^2\right) + \sin \left(x^2\right))}{(1+\cos \left(x^2\right))^2} \\
\intertext{By the Pythagorean Identity, $\cos^2 \left(x^2\right) + \sin^2 \left(x^2\right) = 1$.}
\frac{\diff y}{\diff x} & = \frac{2x + 2x(\cos \left(x^2\right) + \sin \left(x^2\right))}{(1+\cos \left(x^2\right))^2} \\
 & = \frac{2x(1 + \cos \left(x^2\right) + \sin \left(x^2\right))}{(1+\cos \left(x^2\right))^2}.
\end{align*}
}%

