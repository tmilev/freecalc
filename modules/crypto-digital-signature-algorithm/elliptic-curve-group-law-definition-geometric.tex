\begin{frame}
\begin{columns}
\column{0.25\textwidth}
\vskip -0.1cm
\psset{xunit=0.4cm, yunit=0.4cm}
\begin{pspicture}(-2,-4.4)(4.4,4.45)%
\tiny%
\newcommand{\tangentWidth}{1pt}%
\newcommand{\tangentColor}{black}%
\fcAxesStandard{-2}{-4.4}{4.4}{4.4}%
\pstVerb{%
40 dict begin /stickingCoeffOne 1.2 def /stickingCoeffTwo -0.2 def  /coeffA 0.2 def /coeffB 1.2 def /theFun {x x x mul mul x coeffA mul coeffB add add sqrt} def%
}%
\pstVerb{%
/xOne -1 def %
/xTwo 0 def %
}%
\onlyNoH{6-14}{\pstVerb{%
/xOne -0.9 def %
/xTwo -0.1 def %
/stickingCoeffOne 1.4 def 
/stickingCoeffTwo -0.4 def
}}%
\onlyNoH{7}{\pstVerb{%
/xOne -0.8 def %
/xTwo -0.2 def %
}}%
\onlyNoH{8}{\pstVerb{%
/xOne -0.7 def %
/xTwo -0.3 def %
}}%
\onlyNoH{9}{\pstVerb{%
/xOne -0.6 def %
/xTwo -0.4 def %
}}%
\onlyNoH{10-14}{\pstVerb{%
/xOne -0.49 def %
/xTwo -0.51 def %
}}%
\pstVerb{%
%Input: xA yA xB yB. Let (xA, yA) * (xB, yB) = (xC, yC). 
%This function outputs xC, and the next function outputs yC.
/xNew {
20 dict begin /a coeffA def /b coeffB def
/yB exch def /xB exch def /yA exch def /xA exch def 
b 2 1 div mul xA a mul add xB a mul add yB yA mul -2 1 div mul add xB 2 1 div exp xA mul add xB xA 2 1 div exp mul add xB xA mul -2 1 div mul xA 2 1 div exp add xB 2 1 div exp add div
end
} def %
/yNew {
20 dict begin /a coeffA def /b coeffB def
/yB exch def /xB exch def /yA exch def /xA exch def 
yA b mul 2 1 div mul yB b mul -4 1 div mul add b yA mul 2 1 div mul add yA xA mul a mul add yB xA mul a mul -3 1 div mul add yA xB mul a mul add yB xB mul a mul -1 1 div mul add xB a mul yA mul 2 1 div mul add xB 3 1 div exp yA mul 2 1 div mul add yB xA 3 1 div exp mul -1 1 div mul add yA xB 3 1 div exp mul -1 1 div mul add yA xB 2 1 div exp mul xA mul 3 1 div mul add yB xB mul xA 2 1 div exp mul -3 1 div mul add xA 3 1 div exp -1 1 div mul xB 3 1 div exp add xB 2 1 div exp xA mul -3 1 div mul add xB xA 2 1 div exp mul 3 1 div mul add div
end
} def %
/yOne 1 dict begin /x xOne def theFun end def%
/yTwo 1 dict begin /x xTwo def theFun end def%
}%
\psplot[linecolor=\fcColorGraph, plotpoints = 300]{-1}{2.5}{theFun}%
\psplot[linecolor=\fcColorGraph, plotpoints = 300]{-1}{2.5}{theFun -1 mul}%
\onlyNoH{12,13,14}{\pstVerb{%
/yTwo yOne -1 mul def%
/xTwo xOne def%
}}%
\onlyNoH{11,12}{%
\renewcommand{\tangentWidth}{2pt}%
\renewcommand{\tangentColor}{blue}%
}%
\only<1-11,15->{%
\pstVerb{%
/xThree xOne yOne xTwo yTwo xNew def%
/yThree xOne yOne xTwo yTwo yNew def%
}%
\uncover<2->{%
\psline[linewidth=\tangentWidth, linecolor=\tangentColor](!%
xOne stickingCoeffOne mul xThree stickingCoeffTwo mul add %
yOne stickingCoeffOne mul yThree -1 mul stickingCoeffTwo mul add %
)(!%
xOne stickingCoeffTwo mul xThree stickingCoeffOne mul add %
yOne stickingCoeffTwo mul yThree -1 mul stickingCoeffOne mul add %
)%
\rput[tl](! xThree yThree -1 mul){$C'$}%
}%
\fcFullDot[linecolor=blue]{xThree}{yThree -1 mul}%
\uncover<4->{%
\fcFullDot{xThree}{yThree}%
\fcPerpendicular[linestyle=dotted]{[xThree yThree -1 mul]}{[1 0]}{0.2}%
\psline[linestyle=dotted](! xThree yThree)(! xThree 0)%
\rput[bl](! xThree yThree){$C$}%
}%
}%
\onlyNoH{12,13,14}{%
\psline[linewidth=\tangentWidth, linecolor=\tangentColor](! xOne -4.4)(! xOne 4.4)%
\rput[lt](!xOne 4.4){$~~~~~\mathbf 1 =$ ``point at $\infty$''}
}%
\fcFullDot{xOne}{yOne}%
\fcFullDot{xTwo}{yTwo}%
\rput[tl](! xOne yOne){$A$}%
\rput[br](! xTwo yTwo){$\uncover<1-10,12->{B}$}%
\pstVerb{end}%
\end{pspicture}

	\column{0.75\textwidth}
	\begin{definition}[Elliptic curve group law]
		\begin{itemize}
			\item<4-> If line through $A,B$ non-vertical, define $A\cdot B = C$.
			\item<5-> Define $A\cdot A$ similarly \uncover<11->{\alertNoH{11}{but use the tangent through $A$ in place of the line through $A, B$}.}
			\item<12-> \alertNoH{12}{If line through $A,B$ vertical}\uncover<13->{, \alertNoH{13}{define $A\cdot B= \mathbf 1$}.}
			\item<14-> Define $\mathbf 1\cdot A = A\cdot \mathbf 1 = A$ for all $ A$.
		\end{itemize}
	\end{definition}
\end{columns}

	Let $A = (x_A, y_A)$, $B=(x_B, y_B)$ - points on non-degenerate elliptic curve: 

\hfil \hfil $y^2 = x^3 + ax +b$. 
\begin{itemize}
	\item<2-> Let $C'$ be intersection of line through $A,B$ with the elliptic curve.
	\item<3-> Unless the line through $A,B$ is vertical, such $C'$ exists.
	\item<4-> Let $C$ be the reflection of $C'$ across the $x$ axis. 
\end{itemize}
\uncover<15->{\alert<15->{WARNING.} Many authors use $ +$ in place of $\cdot $ and $\mathbf 0$ in place of $\mathbf 1$.}

\vskip 10cm
\end{frame}