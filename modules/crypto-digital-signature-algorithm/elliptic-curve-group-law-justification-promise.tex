\begin{frame}[fragile]
\begin{columns}
	\column{0.25\textwidth}
	\vskip -0.1cm
	\psset{xunit=0.4cm, yunit=0.4cm}
	\begin{pspicture}(-2,-4.4)(4.4,4.45)%
	\tiny%
	\newcommand{\tangentWidth}{1pt}%
	\newcommand{\tangentColor}{black}%
	\fcAxesStandard{-2}{-4.4}{4.4}{4.4}%
	\pstVerb{%
		40 dict begin /stickingCoeffOne 1.2 def /stickingCoeffTwo -0.2 def  /coeffA 0.2 def /coeffB 1.2 def /theFun {x x x mul mul x coeffA mul coeffB add add sqrt} def%
	}%
	\pstVerb{%
		/xOne -1 def %
		/xTwo 0 def %
	}%
%	\onlyNoH{9}{\pstVerb{%
%			/xOne -0.6 def %
%			/xTwo -0.4 def %
%	}}%
	\pstVerb{%
		%Input: xA yA xB yB. Let (xA, yA) * (xB, yB) = (xC, yC). 
		%This function outputs xC, and the next function outputs yC.
		/xNew {
			20 dict begin /a coeffA def /b coeffB def
			/yB exch def /xB exch def /yA exch def /xA exch def 
			b 2 1 div mul xA a mul add xB a mul add yB yA mul -2 1 div mul add xB 2 1 div exp xA mul add xB xA 2 1 div exp mul add xB xA mul -2 1 div mul xA 2 1 div exp add xB 2 1 div exp add div
			end
		} def %
		/yNew {
			20 dict begin /a coeffA def /b coeffB def
			/yB exch def /xB exch def /yA exch def /xA exch def 
			yA b mul 2 1 div mul yB b mul -4 1 div mul add b yA mul 2 1 div mul add yA xA mul a mul add yB xA mul a mul -3 1 div mul add yA xB mul a mul add yB xB mul a mul -1 1 div mul add xB a mul yA mul 2 1 div mul add xB 3 1 div exp yA mul 2 1 div mul add yB xA 3 1 div exp mul -1 1 div mul add yA xB 3 1 div exp mul -1 1 div mul add yA xB 2 1 div exp mul xA mul 3 1 div mul add yB xB mul xA 2 1 div exp mul -3 1 div mul add xA 3 1 div exp -1 1 div mul xB 3 1 div exp add xB 2 1 div exp xA mul -3 1 div mul add xB xA 2 1 div exp mul 3 1 div mul add div
			end
		} def %
		/yOne 1 dict begin /x xOne def theFun end def%
		/yTwo 1 dict begin /x xTwo def theFun end def%
	}%
	\psplot[linecolor=\fcColorGraph, plotpoints = 300]{-1}{2.5}{theFun}%
	\psplot[linecolor=\fcColorGraph, plotpoints = 300]{-1}{2.5}{theFun -1 mul}%
		\pstVerb{%
			/xThree xOne yOne xTwo yTwo xNew def%
			/yThree xOne yOne xTwo yTwo yNew def%
		}%
			\psline[linewidth=\tangentWidth, linecolor=\tangentColor](!%
			xOne stickingCoeffOne mul xThree stickingCoeffTwo mul add %
			yOne stickingCoeffOne mul yThree -1 mul stickingCoeffTwo mul add %
			)(!%
			xOne stickingCoeffTwo mul xThree stickingCoeffOne mul add %
			yOne stickingCoeffTwo mul yThree -1 mul stickingCoeffOne mul add %
			)%
			\rput[tl](! xThree yThree -1 mul){$C'$}%
		%
		\fcFullDot[linecolor=blue]{xThree}{yThree -1 mul}%
			\fcFullDot{xThree}{yThree}%
			\fcPerpendicular[linestyle=dotted]{[xThree yThree -1 mul]}{[1 0]}{0.2}%
			\psline[linestyle=dotted](! xThree yThree)(! xThree 0)%
			\rput[bl](! xThree yThree){$C$}%
		%
	%
	\fcFullDot{xOne}{yOne}%
	\fcFullDot{xTwo}{yTwo}%
	\rput[tl](! xOne yOne){$A$}%
	\rput[br](! xTwo yTwo){$B$}%
	\pstVerb{end}%
	\end{pspicture}
	
	\column{0.75\textwidth}
	\begin{definition}[Elliptic curve group law]
		\begin{itemize}
			\item If line through $A,B$ non-vertical, define $A\cdot B = C$.
			\item Define $A\cdot A$ similarly but use the tangent through $A$ in place of the line through $A, B$.
			\item If line through $A,B$ vertical, define $A\cdot B= \mathbf 1$.
			\item Define $\mathbf 1\cdot A = A\cdot \mathbf 1 = A$ for all $ A$.
		\end{itemize}
	\end{definition}
\end{columns}
\begin{itemize}
	\item $\cdot$ turns the points on the curve into a group.
	\item The reason for this
\end{itemize}



\vskip 10cm

\end{frame}