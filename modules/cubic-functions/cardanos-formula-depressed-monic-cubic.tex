\begin{frame}
\frametitle{Cardano's formula}
\begin{theorem}[Cardano's formula for depressed monic cubic]
The equation 
\[
x^3+px+q=0
\]
has its roots given by:

\[
\begin{array}{rcl}
x_1 &=&\displaystyle  ~~~ \sqrt[3]{-\frac{q}{2}-\sqrt{\frac{q^2}{4}-\frac{p^3}{27}}}+~~~\sqrt[3]{-\frac{q}{2}+ \sqrt{\frac{q^2}{4}-\frac{p^3}{27}}}\\
x_2 &=&\displaystyle \omega  \sqrt[3]{-\frac{q}{2}-\sqrt{\frac{q^2}{4}-\frac{p^3}{27}}}+\omega^2 \sqrt[3]{-\frac{q}{2} + \sqrt{\frac{q^2}{4}-\frac{p^3}{27}}}\\
x_3 &=&\displaystyle  \omega^2 \sqrt[3]{-\frac{q}{2}-\sqrt{\frac{q^2}{4}-\frac{p^3}{27}}}+\omega \sqrt[3]{-\frac{q}{2}+\sqrt{\frac{q^2}{4}-\frac{p^3}{27}}}\\
\end{array}
\]
where $\omega =\displaystyle \frac{-1+i\sqrt{3}}{2} = \cos 120^\circ + i \sin 120^\circ$  (the third root of unity).
\end{theorem}	
\end{frame}