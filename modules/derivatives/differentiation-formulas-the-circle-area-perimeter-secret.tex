% begin module differentiation-formulas-ex1
\begin{frame}
\alert<1>{You will not be tested on the material in the following slide.}
\end{frame}
\begin{frame}
\frametitle{Derivative ball volume =surface area}
The relationship between surface area and volume of a ball.

\footnotesize
\begin{tabular}{|p{0.6cm}p{2cm}p{1cm}p{1cm}p{1cm}p{1cm}p{2.8cm}|}\hline
\alert<0>{Di-men-sion} & \alert<2,14,26>{Pts. at distance $\leq r$ from origin} &  \alert<4,16,28>{Inside measure } & \alert<6, 18,30>{Measure f-la} & \alert<8,20,32>{Boundary name} & \alert<10,22,34>{Boundary measure} & \alert<12,24,36>{Derivative of inside measure }\\\hline
%
\alert<2>{3} & 
\uncover<3->{
%This is a picture of sphere. To use you need to edit the /trunk/pstricks-commands.sty to include the package \usepackage{pst-3dplot}
%The package is disabled as it slows down compilation way too much, enable only to compile this current slide.
%\begin{pspicture}(-4,-2.25)(2,4.25)
%\pstThreeDSphere[SegmentColor=\psColorAreaUnderGraph](1,-1,2){0.5}
%\end{pspicture}
\alert<3>{ball} 
} & \uncover<5->{\alert<5>{volume}} &  \uncover<7->{\alert<7, 12>{$\frac {4}{3}\pi r^3$}} & 
\uncover<9->{
%This is a picture of sphere. To use you need to edit the /trunk/pstricks-commands.sty to include the package \usepackage{pst-3dplot}
%The package is disabled as it slows down compilation way too much, enable only to compile this current slide.
%\begin{pspicture}(-4,-2.25)(2,4.25)
%\pstThreeDSphere[SegmentColor={[cmyk]{0,0,0,0}}](1,-1,2){0.5}
%\end{pspicture}
\alert<9>{sphere} 
} 
& \uncover<11->{\alert<11, 13>{$4\pi r^2$}} & \uncover<12->{$\displaystyle\alert<12>{\frac{d}{dr}\left(\displaystyle\frac {4}{3}\pi r^3\right)=}\uncover<13->{\alert<13>{4\pi r^2}}$} \\\hline
%
\alert<14>{2} & 
\uncover<15->{
\begin{pspicture}(-1,-1)(1,1)
\pscircle*[linecolor=\psColorAreaUnderGraph](0,0){0.3}
\pscircle(0,0){0.3}
\psline(0,0)(0.3,0)
\tiny
\rput[b](0.15,0.05){$r$}
\end{pspicture} ~~~~~~
\alert<15>{disk, circle}
} 
& \uncover<17->{\alert<17>{circle area}} & \uncover<19->{\alert<19,24>{$\pi r^2$}} & 
\uncover<21->{
\begin{pspicture}(-1,-1)(1,1)
\pscircle(0,0){0.3}
\end{pspicture}
\alert<21>{circum-ference}
} 
& \uncover<23->{\alert<23,25>{$2\pi r$}} & \uncover<24->{$\displaystyle{\alert<24>{\frac{d}{dr}\left(\pi r^2\right)=}} \uncover<25->{\alert<25>{2\pi r}}$} \\\hline
%
\alert<26>{1} & 
\uncover<27->{
\begin{pspicture}(-1,-1)(1,1)
\psline[linecolor=\psColorAreaUnderGraph](-0.3,0)(0.3,0)
\tiny
\rput[b](0.15,0.05){$r$}
\psXTick{0}
\psFullDotBlack{0.3}{0}
\psFullDotBlack{-0.3}{0}
\end{pspicture}
\alert<27>{interval}
} 
& \uncover<29->{\alert<29>{length}} & \uncover<31->{\alert<31>{$2r$}} & 
\uncover<33->{
\begin{pspicture}(-1,-1)(1,1)
\tiny
\psFullDotBlack{0.3}{0}
\psFullDotBlack{-0.3}{0}
\end{pspicture}
\alert<33>{endpts}
} 
& \uncover<35->{\alert<35,37>{$2$}} &\uncover<36->{$\displaystyle\alert<36>{\frac{d}{dr}(2r)=} \uncover<37->{\alert<37>{2}}$} \\
\hline
\end{tabular}
\end{frame}
% end module differentiation-formulas-ex1
