\solution{\ref{problemDFQseparable-mixing-problem-1}. 
Let 
\[
y(t)=\text{salt in the tank after } t \text{ minutes (in kg)}\quad .
\] 
We are given $y(0)= 30$kg, the initial amount of salt. We are looking to find $y(45)$, the amount of salt after $45$ minutes. We have that 
\[
\frac{\diff y}{\diff t}= \text{(rate in)} - \text{(rate~out)} \quad .
\]
The rate of salt entering the tank is constant: 
\[
\text{(rate in)}=0.05 kg/L \cdot 10 L/min= 0.5 kg/min\quad .
\] 
As the solution is thoroughly mixed, at any time the concentration of salt in the tank is 
\[
\displaystyle \frac{y}{ 10000} kg/L.
\] 
Therefore the rate of salt going out of the tank is 
\[
\text{(rate out)}=\frac{y}{10000} kg/L * 10 L/min = \frac{y}{1000} kg/min\quad .
\] 
Therefore the differential equation for the amount of salt in the tank is
\[
\frac{\diff y}{\diff t}= \underbrace{ 0.5}_{\text{(rate  in)}}- \underbrace{ \frac{y}{1000} }_{\text{(rate out)}}\quad .
\]
There are two variants for remainder of the solution. Variant I uses indefinite integration and is slightly informal, but is easier to learn and remember. Variant II is rigorous, but more challenging understand and write up. Both solutions are acceptable for full credit in a Calculus exam. Variant I is recommended when taking exams and Variant II is recommended when writing scientific texts.

\textbf{Variant I} 
\[
{\renewcommand{\arraystretch}{1.5}
\begin{array}{rcll|l}
\displaystyle\frac{\diff y}{\diff t}&=&\displaystyle 0.5-  \frac{y}{1000} \quad .\\
\displaystyle\frac{\diff y}{\diff t}&=&\displaystyle   \frac{500-y}{1000} \quad .\\
\displaystyle \frac{1000}{500- y} \frac{\diff y}{\diff t}&=& \displaystyle 1 &&\text{Use indefinite integration}\\
\displaystyle \int \frac{1000}{500- y} \underbrace{ \frac{\diff y}{\diff t} \diff t}_{\diff y} &=& \displaystyle \int \diff t \\
\displaystyle \int \frac{1000}{500-y}\diff y &=& t+C\\
\displaystyle -1000 \int  \frac{1}{500-y}\diff(500-y)&=& t+C \\
\displaystyle  -1000 \ln |500-y| &=& t+C &&
\begin{array}{l}
\text{The constant from} \\\text{the second integral} \\\text{is accounted by the constant }C
\end{array}
\\
\displaystyle\ln |500-y|&=&\displaystyle -\frac{t+C}{1000} \\
\displaystyle|500-y| &=&\displaystyle e^{-\frac{t+C}{1000}}&& 
\begin{array}{l}
\text{Since }500-y(0)= 500-30=470 >0\\
\text{we can drop the absolute values}
\end{array}\\
\displaystyle 500-y &=&\displaystyle e^{-\frac{t+C}{1000}}\\
\displaystyle y&=& \displaystyle 500-e^{-\frac{t+C}{1000}} &&\text{Set } D= e^{-\frac{C}{1000}}\\
\displaystyle y&=&\displaystyle  500-De^{-\frac{t}{1000}}
\quad .
\end{array}
}
\]
To find the constant $D$, we observe that 
\[
\begin{array}{rcl}
30&=&y(0)= 500 - De^{-\frac{0}{1000}}= 500-D\\
D&=&470\quad .
\end{array}
\]
Therefore 
\[
\displaystyle y(t)= 500- 470 e^{-\frac{t}{1000}}\quad ,
\]
and the final answer is
\[
\displaystyle y(45)=500-470e^{-\frac{45}{1000}}\approx 50.68 
\]
with measurement unit $kg$.

\textbf{Variant II.} 
To find $y(45)$, we integrate from $t=0$ to $t=45$:
\[
{\renewcommand{\arraystretch}{1.5}
\begin{array}{rcll|l}
\displaystyle \int\limits_{t=0}^{45} \frac{1000}{500- y} \underbrace{ \frac{\diff y}{\diff t} \diff t}_{\diff (y(t))} &=& \displaystyle \int\limits_{t=0}^{45} \diff t \\
\displaystyle \int\limits_{t=0}^{t=45} \frac{1000}{500-y(t)}\diff (y(t))&=& 45&&\text{set }z=y(t) \\
\displaystyle -1000 \int \limits_{z=y(0)=30}^{z=y(45)} \frac{1}{500-z}\diff(500-z)&=& 45 \\
\displaystyle \left. -1000 \ln |500-y| \right]_{y(0)=30}^{y(45)}&=& 45 \\
\displaystyle -1000 \left( \ln |500-y(45)|\right. \\
\left. -\ln |500- 30|  \right) &=& 45 \\
\displaystyle \ln \left| \frac{470}{500-y(45)} \right|  &=& \displaystyle \frac{45}{1000}
\\
\displaystyle \ln \left( \frac{470}{500-y(45)} \right)  &=& \displaystyle \frac{45}{1000} &&\text{see below}
\\
\displaystyle \frac{470}{500-y(45)}&=&\displaystyle e^{\frac{45}{1000}}\\
\displaystyle 500-y(45)&=&\displaystyle  470e^{-\frac{9}{200}}\\
\displaystyle y(45)&=&\displaystyle 500-470e^{-\frac{9}{200}}\\
&\approx& 500-470\cdot 0.955997 \\
&\approx& 50.681184 \quad ,
\end{array}
}
\]
where we have used that $\displaystyle \frac{470}{500-y(t)}>0 $. The fact that $\displaystyle \frac{470}{500-y(t)}>0 $ can be seen as follows. As $500-y(0)=470>0$ and $y(t)$ is continuous, in order to have $500-y(t)<0$ there must exist some $x_1$ for which $y(x_1)=500$. However this is impossible since $\displaystyle x=\ln \left|\frac{470}{500-y(x)}\right|  $. 

As the unit of measurement is $kg$, the final answer to the problem is $\approx 50.68 kg$ salt.
}