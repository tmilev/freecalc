\solution{\noindent \ref{problemDFQseparable-yprime=ysquared-1-part1}.
There are two variants for solving this problem. The first variant uses indefinite integration and is slightly informal, but easier to apply and remember. The second variant is more rigorous but more difficult to write up. Both solutions are acceptable for full credit in a Calculus exam. Variant I is recommended when taking exams and Variant II is recommended when writing scientific texts.

\textbf{Variant I}

\renewcommand{\arraystretch}{2}
\[
\begin{array}{rcll|l}
\displaystyle \frac{ \diff y}{ \diff x}&=&y^2-1 && \text{Suppose } y^2-1\neq 0 \\
\displaystyle \frac{\frac{ \diff y}{ \diff x}}{y^2-1}&=&1 \\
\displaystyle\int\frac{1}{ y^2- 1} \underbrace{ \frac{\diff y}{\diff x}\diff x}_{=\diff y } &=& \displaystyle\int \limits \diff x \\
\displaystyle\int \frac{\diff y}{ y^2-1}& =& \displaystyle x +C \\
\displaystyle\int \left(\frac{\frac{1}{2} }{y-1}- \frac{\frac{1 }{2}}{ y+1}\right)\diff y&=& x +C \\
\displaystyle \frac{1}2 \ln \left|\frac{y-1}{y+1}\right| &=& x+C\\
\displaystyle \ln \left|\frac{y-1}{y+1}\right|&=& 2 x + 2C\\
\displaystyle\left|\frac{y-1}{y+1}\right|&=& e^{ 2x +2C} \\
\displaystyle\frac{y-1}{y+1}&=& \pm e^{ 2x +2C} \\
\displaystyle y-1&=&\displaystyle \pm e^{2x+2C} (y+1)\\
\displaystyle y(1\mp e^{2x+2C})&=&\displaystyle 1\pm e^{2x+2C} \\
\displaystyle y&=&\displaystyle \frac{1\pm e^{2x+2C}}{1\mp e^{2x+2C}}  \\
\displaystyle y&=&\displaystyle \frac{1\pm e^{2C} e^{2x}}{1\mp e^{2C}e^{2x}}&&\text{Set }D=\pm e^{2C}\\
\displaystyle y&=&\displaystyle  \frac{1+D e^{2x}}{1- De^{2x}}\quad .
\end{array}
\]
The above solution works on condition that $y^2-1\neq 0$. So the only case not covered is that of $y^2-1=0$, which yields the two solutions $y=\pm 1$.

Our final answer is
\[
y(x)= \frac{1+De^{2x}}{1-De^{2x}} \quad \text{ or }\quad y(x)=-1,
\]
where $D$ is an arbitrary real number. Notice that in the above answer, by allowing $D=0$, we have covered the case $y(x)=1 $. Finally, we note that if we let $D\to \infty$, the solution $y(x) = \frac{1+De^{2x }}{ 1- De^{2x}}  $ tends to the solution $y(x)=-1$ (here we fix a value of $x$ before we let $D\to \infty$).


\textbf{Variant II}

\noindent Case 1. Suppose there exists a number $x_0$ such that $(y(x_0) )^2 - 1\neq 0$. Since $y$ is a differentiable function of $x$, it is also continuous. Therefore for some $t$ sufficiently close to $x_0$, all numbers $x$ in the interval between $t$ and $x_0$ satisfy $ y(x)^2-1\neq 0$.
\[
\begin{array}{rcll|l}
\displaystyle \frac{\frac{ \diff y}{ \diff x}}{y^2-1}&=&1 \\
\displaystyle\int\limits_{x=x_0}^{x=t} \frac{1}{ y^2- 1} \underbrace{ \frac{\diff y}{\diff x}\diff x}_{=\diff (y(x)) }&=&\displaystyle\int\limits_{x=x_0}^{x=t}\diff x &&\text{can integrate as }  y(x)^2-1\neq 0\\
\displaystyle\int\limits_{t=x_0 }^{x=t} \frac{\diff (y(x))}{ (y(x))^2-1}& =& \displaystyle\left.x \right|_{ x=x_0}^{x=t} &&\text{set } z=y(x)\\
\displaystyle\int\limits_{z=y(x_0) }^{z=y(t)} \frac{\diff z}{ z^2-1}& =& \displaystyle t-x_0 \\
\displaystyle\int\limits_{z=y(x_0)}^{z=y(t)} \left(\frac{\frac12 }{z-1}- \frac{\frac12}{z+1}\right)\diff z&=& t-x_0
\\
\displaystyle\left .\frac{1}2 \ln \left|\frac{z-1}{z+1}\right|\right]_{z=y(x_0)}^{z=y(t)}&=& t-x_0 && \text{Set } C=2x_0-  \ln \left|\frac{y(x_0)-1}{ y(x_0)+ 1} \right|\\
\displaystyle \ln \left|\frac{y(t)-1}{y(t)+1}\right|&=& 2t - C&&\text{relabel dummy variable } t \text { to } x \\
\displaystyle
\ln \left|\frac{y(x)-1}{y(x)+1}\right|&=& 2x - C
\end{array}
\]
Set
\[
D=e^{-C}\quad .
\]
By the assumption of our case, $ (y(x_0))^2-1\neq 0$, so there are two remaining cases: $ (y(x_0))^2-1>0$ and $ (y(x_0))^2-1<0$.

\noindent Case 1.1. Suppose $\displaystyle \frac{y(x_0)-1}{ y(x_0)+1}>0$. As the function $y(x)$ is differentiable, it is also continuous. Therefore $\displaystyle \frac{y(x)-1}{y(x)+1}>0$ for all $x$ near $x_0$. Then we can remove the absolute values in the equality above to get that for all $x$ close to $x_0$ we have that
\[
\begin{array}{rcll|l}
\displaystyle \ln \left(\frac{y(x)-1}{y(x)+1}\right)&=& 2x - C&&\text{exponentiate, recall }D=e^{-C}\\
\displaystyle \frac{y(x)-1}{y(x)+1}&=& D e^{2x}\\
\displaystyle y(x)-1&=&\displaystyle  De^{2x}(y(x)+1)\\
\displaystyle y(x)\left(1- De^{2x}\right)&=&\displaystyle  De^{2x}+1\\
\displaystyle y(x)&=&\displaystyle  \frac{ 1+De^{2x}}{1- De^{2x}}\quad .\\
\end{array}
\]
The solution $y(x)$ given above satisfies $\displaystyle \frac{y(x)-1}{y(x)+1}= De^{2x}$ for all $x$. As $D>0$, this implies that $\displaystyle \frac{y(x)-1}{ y(x)+1}>0$. Therefore the considerations above are valid for all $x$, rather than only for those $x$ near $x_0$. Therefore our first case yields the solution
\[
y(x)=\frac{ 1+De^{2x}}{1- De^{2x}}\quad .
\]

\noindent Case 1.2. Suppose  $\displaystyle \frac{y(x_0) -1}{y(x_0) +1} <0$. Then for all $x$ near $x_0$ we get $\displaystyle \ln \left| \frac{y(x) -1}{y(x) +1}\right|= \ln \left( \frac{ 1- y(x) }{ y( x) +1}\right)$ and, similarly to Case 1, we get
\[
\begin{array}{rcl}
\displaystyle \frac{1-y(x)}{y(x)+1}&=& D e^{2x}\\
1-y(x)&=& De^{2x}(y(x)+1)\\
y(x)\left(1+ De^{2x}\right)&=& 1-De^{2x}\\
y(x)&=&\displaystyle \frac{1- De^{2x}}{1+ De^{2x}}\quad .
\end{array}
\]
Since $D$ is a positive constant, we conclude in a fashion analogous to Case 1 that $y(x)<0$ for all $ x$.

Case 2.  Suppose $\displaystyle  (y(x_0))^2-1=0 $.  Then $y(x_0)=\pm 1$. Clearly the constant functions $y(x)= \pm 1$ are two solutions: if we can plug back $y=\pm 1$ in the original equation we get that $\frac{\diff y}{\diff x}= 0$ and $y$ is a constant function of $x$. From the preceding two cases we know that if $\frac{y(x) -1}{y(x) +1}$ is defined and not equal to zero for some value of $x$, then $\frac{y(x)-1}{y(x)+1}$ is defined and not equal to zero for all values of $x$. Therefore the present case yields only two solutions, the constant functions $y(x)=\pm 1$.

Our final answer is
\[
y(x)= \frac{1+De^{2x}}{1-De^{2x}} \quad \text{ or }\quad y(x)=-1,
\]
where $D$ is an arbitrary real number. Notice that in the above answer, we have combined Cases 1.1, 1.2 and the case $y(x)=1 $: by allowing $D$ to be negative we included Case 1.2 and by allowing $D $ to be zero we included the case $y(x)=1$. Finally, we note that if we let $D\to \infty$, the solution $y(x) = \frac{1+De^{2x }}{ 1- De^{2x}}  $ tends to the solution $y(x)=-1$ (for all values of $x$).


\textbf{Solution plots.}

We may plot solutions for a few values of $D$ as follows. We overlay the solutions on top of the direction field of the differential equation. The picture tells us a lot about the properties of the solutions of the differential equations.

\optionalDisplay{
\begin{pspicture}(-6,-6)(6,6)
\newcommand{\Dconst}{1}
\psplot[linecolor=green]{-4}{4}{1 \Dconst\space 2.718281828 2 x mul exp mul sub 1 \Dconst\space 2.718281828 2 x mul exp mul add div}
\renewcommand{\Dconst}{0.25}
\psplot[linecolor=green]{-4}{4}{1 \Dconst\space 2.718281828 2 x mul exp mul sub 1 \Dconst\space 2.718281828 2 x mul exp mul add div}
\renewcommand{\Dconst}{4}
\psplot[linecolor=green]{-4}{4}{1 \Dconst\space 2.718281828 2 x mul exp mul sub 1 \Dconst\space 2.718281828 2 x mul exp mul add div}
\rput[l](5,2 ){$\frac{1- \frac{1}4 e^{2x}}{1+\frac 14 e^{2x}}$ }
\rput[l](5,0.5 ){$\frac{1- e^{2x}}{1+e^{2x}}$ }
\rput[l](5,-2 ){$\frac{1- 4e^{2x}}{1+4e^{2x}}$ }
\psline[arrows=->, linestyle=dotted](5,2)(0,0.6)
\psline[arrows=->, linestyle=dotted](5,0.5)(0,0)
\psline[arrows=->, linestyle=dotted](5,-2)(0,-0.6)

\renewcommand{\Dconst}{1}
\psplot[linecolor=green]{-4}{-0.17}{1 \Dconst\space 2.718281828 2 x mul exp mul add 1 \Dconst\space 2.718281828 2 x mul exp mul sub  div}
\psplot[linecolor=green]{0.17}{4}{1 \Dconst\space 2.718281828 2 x mul exp mul add 1 \Dconst\space 2.718281828 2 x mul exp mul sub  div}
\renewcommand{\Dconst}{4}

\psplot[linecolor=green]{-4}{-0.863147181}{1 \Dconst\space 2.718281828 2 x mul exp mul add 1 \Dconst\space 2.718281828 2 x mul exp mul sub  div}
\psplot[linecolor=green]{-0.523147181}{4}{1 \Dconst\space 2.718281828 2 x mul exp mul add 1 \Dconst\space 2.718281828 2 x mul exp mul sub  div}

\renewcommand{\Dconst}{0.25}
\psplot[linecolor=green]{-4}{0.523147181}{1 \Dconst\space 2.718281828 2 x mul exp mul add 1 \Dconst\space 2.718281828 2 x mul exp mul sub  div}
\psplot[linecolor=green]{0.863147181}{4}{1 \Dconst\space 2.718281828 2 x mul exp mul add 1 \Dconst\space 2.718281828 2 x mul exp mul sub  div}
\rput[r](-5,0.5 ){$\frac{1+\frac 14 e^{2x}}{1- \frac{1}4 e^{2x}}$ }
\rput[r](-5,2 ){$\frac{1+e^{2x}}{1- e^{2x}}$ }
\rput[r](-5,-2 ){$\frac{1+4e^{2x}}{1- 4e^{2x}}$ }
\psline[arrows=->, linestyle=dotted](-5,0.5)(0,1.6667)
\psline[arrows=->, linestyle=dotted](-5,0.5)(1,-3.360539267)
\psline[arrows=->, linestyle=dotted](-5,2)(-0.2,5.066489563)
\psline[arrows=->, linestyle=dotted](-5,2)(0.2,-5.066489563)
\psline[arrows=->, linestyle=dotted](-5,-2)(0,-1.6667)
\psline[arrows=->, linestyle=dotted](-5,-2)(-1,3.360539267)
\psaxes[arrows=<->](0,0)(-4.5,-4.5)(4.5,4.5)

\rput (5,5){The direction field  $\frac{\diff y}{\diff x}=y^2-1$}

\fcDirectionFieldDefault{y y mul 1 sub}{-4}{-4}{0.5}{17}
\end{pspicture}
} %optionalDisplay

\noindent \ref{problemDFQseparable-yprime=ysquared-1-part2}.
From the computer generated picture above, we may visually estimate that $y(x)=\frac{1-4 e^{2x} }{1+4 e^{2x} }$ intersects the $x$-axis at $\left(0, -\frac {3}{ 5}\right)$. Furthermore, we may check directly that for
\[
y(x)=\frac{1-4 e^{2x} }{1+4 e^{2x} }
\]
we have $y(0)= \frac{1-4}{1+5}=  -\frac{3}{5}$ and that is a solution to our problem (this however does not prove the solution is unique).

Alternatively, let us give an algebraic solution. As we are given that $y(0)=-\frac35$ and so
\[
\begin{array}{rcl}
\displaystyle -\frac{3}{5}&=&\displaystyle y(0)= \frac{1-De^{2\cdot 0}}{1+ De^{2\cdot 0}}= \frac{1-D}{1+D}\\
\displaystyle -\frac{3}{5} (1+D)&=&1-D\\
\displaystyle \frac{2}{5} D&=&\displaystyle \frac{8}{5}\\
D&=&4\quad ,
\end{array}
\]
which is our final answer.
}

\solution{\ref{problemy'=y^2(1+x),y(0)=3}. 

This is a concise solution written up in a form suitable for exam taking.
\[\begin{array}{rcl}
\displaystyle \frac{\diff y}{\diff x}&=&\displaystyle y^2(1+x)\\
\displaystyle \frac{\diff y}{y^2} &=&\displaystyle  (1+x) \diff x\\
\displaystyle \int \frac{\diff y}{y^2} &=&\displaystyle \int (1+x) \diff x\\
\displaystyle -\frac{1}{y} &=&\displaystyle  x + \frac{x^2}{2} + C\\
\displaystyle -\frac{1}{3}& =&\displaystyle  0 + 0 + C\\
\displaystyle y &=&\displaystyle  -\frac{1}{\frac{x^2}{2}+x  - \frac{1}{3}} = -\frac{3}{3x^2+6x-2}\quad .
\end{array}
\]
}

\solution{\ref{problemDFQseparabley'=xtany_initial_condition1} and 
\ref{problemDFQseparabley'=xtany_initial_condition2}
\[\begin{array}{rcll|l}
\displaystyle y'&=&\displaystyle x\tan y\\
\displaystyle\frac{y'}{\tan y}&=&\displaystyle x\\
\displaystyle\frac{(\cos y) y'}{\sin y}&=&\displaystyle x &&\text{Integrate from }0\\
\displaystyle \int\limits_{t=0}^{t=x} \frac{\cos(y(t))}{\sin (y(t))} (y' \diff t)&=&\displaystyle  \int\limits_{t=0}^xt \diff t \\
\displaystyle \int\limits_{t=0}^{t=x} \frac{\cos(y(t))}{\sin (y(t))}\diff (y(t))&=&\displaystyle  \frac{x^2}{2} &&\text{Set }z=y(t)\\
\displaystyle \int \limits_{z=y(0)}^{z=y(x)} \frac{\cos z}{\sin z} \diff z&=&\displaystyle  \frac{x^2}{2}\\
\displaystyle \int \limits_{z=y(0)}^{z=y(x)} \frac{\diff (\sin z)}{\sin z} &=&\displaystyle  \frac{x^2}{2}\\
\displaystyle \left[\ln | \sin z|\right]_{y(0)}^{y} & = & \displaystyle  \frac{x^2}{2}\\
\displaystyle \ln |\sin y|- \ln |\sin (y(0))|&=& \displaystyle  \frac{x^2}{2} \\
\displaystyle \ln |\sin y|&=& \displaystyle  \frac{x^2}{2}+\ln |\sin (y(0))| \\
|\sin y|&=&\displaystyle  e^{\frac{x^2}{2}+\ln |\sin (y(0))|}\\ |\sin y|&=&\displaystyle \doublebrace{e^{\frac{x^2}{2}+\ln \left|\sin \left(\Arcsin \left(\frac{1}{e}\right) \right)\right|}}{\text{for problem \ref{problemDFQseparabley'=xtany_initial_condition1}}}{e^{\frac{x^2}{2}+\ln \left|\sin \left(\pi+ \Arcsin \left(\frac{1}{e}\right) \right)\right|} }{\text{for problem \ref{problemDFQseparabley'=xtany_initial_condition2}}} \\
|\sin y|&=&\displaystyle  e^{\frac{x^2}{2}+ \ln \left(\frac{1}{ e}\right)}\\
\displaystyle |\sin y|&=&\displaystyle e^{\frac{x^2}{2}-1} &&\begin{array}{l} y(0)>0 \text{ for both problems}\\
\text{therefore }  \sin y(0) > 0\end{array}\\
\displaystyle \sin y&=&e^{\frac{x^2}{2}-1} \quad.
\end{array}
\]
From the elementary properties of the trigonometric functions, we know that  $\sin y=\sin \alpha$ implies that either
\begin{itemize}
\item $y=\alpha +2k\pi$, where $k$ is an arbitrary integer or
\item $y=(2k+1)\pi-\alpha$, where $k$ is an arbitrary integer.
\end{itemize}
In other words, if we are given $\sin y$, we know $y$ up to a choice of sign and a choice of an integer $k$. For our problem, this means that 

\[
y=\left\{\begin{array}{ll} 2 k \pi+\Arcsin\left( e^{\frac{x^2}{2}-1} \right) &{k -\text{integer}} \\{\text{or}}\\{(2k+1)\pi- \Arcsin\left( e^{\frac{x^2}{2}-1} \right)}&{k-\text{integer}}\end{array}\right.
\]

For problem \ref{problemDFQseparabley'=xtany_initial_condition1}, 
the only choice for $k$ and sign which fits the initial condition $y(0)= \Arcsin\left(\frac{1}{e}\right)$ is
\[
y=\Arcsin\left(e^{\frac{x^2}{2}-1} \right)\quad ,
\]
which is our final answer. 

For problem \ref{problemDFQseparabley'=xtany_initial_condition2}, 
the only choice for $k$ and sign which fits the initial condition $y(0)=\pi+ \Arcsin\left(-\frac{1}{e}\right)=\pi- \Arcsin \left( \frac{1}{e}\right) $ is
\[
y=\pi- \Arcsin\left(e^{\frac{x^2}{2}-1} \right)\quad, 
\]
which is our final answer.
}
