\begin{frame}
\begin{definition}[Division, exact]
To \alertNoH{1}{divide} a number \alertNoH{2}{$p$ (\textbf{dividend})} by a number \alertNoH{3}{$d$ (\textbf{divisor})} means to find a number \alertNoH{4}{$q$ (\textbf{quotient})} so that 
\[
\alertNoH{5}{ \alertNoH{4,8}{q} \cdot \alertNoH{3,11}{d} =\alertNoH{2,10}{ p}}
\]
\end{definition}

\uncover<6->{
\begin{example}
\alertNoH{6,7}{Divide $5$ by $3$. }
\[
\alertNoH{6,7}{\alertNoH{8}{\text{answer}}=}\fcAnswer{7}{ \alertNoH{9}{ \frac{\alertNoH{10}{5}}{\alertNoH{11}{3}}} }
\]
\end{example}
}
\uncover<8->{
\begin{observation}
\alertNoH{8}{The quotient of two integers} equals \alertNoH{9}{the fraction} formed by putting the \alertNoH{10}{dividend as the numerator} and the \alertNoH{11}{divisor as the denominator}. 
\[
\alertNoH{8}{q} =\alertNoH{9}{\frac{\alertNoH{10}{p}}{\alertNoH{11}{d}}}
\]

\end{observation}
}

\end{frame}