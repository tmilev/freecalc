\begin{frame}
\vskip -0.1cm
\begin{example}[Divisor 1-digit, quotient 1-digit]
Divide $7$ by $2$ with remainder. To solve, we need to answer: what is the largest integer which, when multiplied by $2$, stays smaller than $7$? Try:

\hfil\hfil $
\begin{array}{rcl}
0\cdot 2&=& 0\\
1\cdot 2&=& 2\\
2\cdot 2&=& 4\\
3\cdot 2&=& 6\\
4\cdot 2&=& 8>7
\end{array}
$

$\Rightarrow$ $7= 3\cdot 2 +x$. Solve: $x=7-3\cdot 2= 1$. Therefore $7= 3\cdot 2 +1$.

\end{example}


\vskip -0.1cm
\begin{observation}[Question to answer when dividing with remainder]

What is the largest integer which, when multiplied by $d$, remains smaller than $p$?
\end{observation}

\vskip -0.1cm
\begin{itemize}
\item To answer this question, we guess quickly as shown above.
\item Later on we learn to divide large numbers without guessing.
\item However, we still need the guessing approach as a building block of the complete division algorithm.
\end{itemize}
\end{frame}