\begin{frame}
\vskip -0.06cm
\begin{itemize}
\item Recall exact division: 
$
\begin{array}{rcl}
p'&=&q'\cdot d'\\
q'&=&\frac{p'}{d'}
\end{array}
$
\item Quotient may fail to reduce to an integer.
\item What if we want an integer quotient?
\end{itemize}
\vskip -0.05cm
\begin{definition}[Integer division with remainder]
To \textbf{divide} an integer $p >0$ by an integer $d > 0$  \textbf{with remainder} $r \geq 0$ means to find the largest integer $q \geq 0$ and the smallest $0\leq r$ so that:

\vskip -0.2cm
\[
p = q\cdot d+ r
\]
\vskip -0.01cm

$p$ is called the \textbf{dividend}, $d$ is called the \textbf{divisor}, $q$ is called the \textbf{quotient} and $r$ is called the \textbf{remainder}.
\end{definition}
\vskip -0.06cm
\begin{example}
Divide $7 $ by $3$ with remainder. $7 = 2\cdot 3 +1 $.
\end{example}
\vskip -0.06cm
\begin{itemize}
\item Differences between exact division integer division.
\begin{itemize}
\item Integer division quotient is integer, exact division quotient is fraction.
\item Exact division: no notion of remainder.
\end{itemize}
\end{itemize}


\end{frame}