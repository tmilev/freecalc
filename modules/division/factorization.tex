\begin{frame}
\begin{definition}[Factor positive integer completely]
To factor a positive integer completely means to write it as a product of prime factors.
\[
x= p_1\cdot p_2 \cdots p_n
\]
\end{definition}
\begin{itemize}
\item<2-> It is best practice to sort (order) the prime factors. Most frequently used order: smaller factors come first.
\uncover<3->{
\begin{lemma}[Unique prime factorization]
Up to shuffling prime factors, there is only one way to factor a number.
\end{lemma}
}
\item<4-> Consequence: two numbers are equal if and only if their sorted prime factorizations are equal.
\item<5-> When factoring, we may or may not use exponent notation:
\[
\begin{array}{rcl}
\alertNoH{5,6}{16} &\alertNoH{5,6}{=}& \fcAnswer{6}{ \alertNoH{7}{4} \cdot\alertNoH{8}{ 4} } \uncover<6->{= \onlyNoH{4-8}{\color{white}} \underbrace{\color{black} \alertNoH{9}{ \alertNoH{7}{\alertNoH{10}{2}\cdot \alertNoH{10}{2}}\cdot \alertNoH{8}{\alertNoH{10}{2}\cdot \alertNoH{10}{2}}}}_{ \uncover<9->{ \alertNoH{9}{\alertNoH{11}{4} \text{ \alertNoH{10}{copies}}}} }\uncover<9->{ =\alertNoH{9}{ {\alertNoH{10}{2}} ^{\alertNoH{11}{4}}}} }\\
\alertNoH{12,13}{36} &\alertNoH{12,13}{=}& \fcAnswer{13}{\alertNoH{14}{ 4} \cdot\alertNoH{15}{ 9} = \alertNoH{16}{ \only<1-15>{\color{white}} \underbrace{\color{black} \alertNoH{14,16}{2\cdot 2} }_{2\text{ copies}}} \cdot \alertNoH{17}{\only<1-15>{\color{white}} \underbrace{\color{black} \alertNoH{15,17}{3\cdot 3}}_{2\text{ copies}}}} \uncover<16->{ = \alertNoH{16}{2^2} \cdot \alertNoH{17}{ 3^2} }
\end{array}
\] 

\end{itemize}

\end{frame}