\begin{frame}
\begin{example}[Reading fractions]
\alertNoH{1} {Read the fraction.} Honor your English dialect naming convention, if different from the one given here.
\begin{itemize}
\item $\alertNoH{1,2}{\frac{1}{3}}$ \fcAnswer{2}{\alertNoH{14}{one third} (can also say ``a third'').}
\item $\alertNoH{3,4}{\frac{1}{2}}$ \fcAnswer{4}{\alertNoH{14}{one half}.}
\item $\alertNoH{5,6}{\frac{1}{4}}$ \fcAnswer{6}{\alertNoH{14}{one quarter}; \alertNoH{13}{one fourth}.}
\item $\alertNoH{7,8}{\frac{3}{4}}$ \fcAnswer{8}{\alertNoH{14}{three quarters}, \alertNoH{13}{three fourths}.}
\item $\alertNoH{9,10}{\frac{6}{7}}$ \fcAnswer{10}{\alertNoH{13}{six sevenths}.}
\item $\alertNoH{11,12}{\frac{11}{6}}$ \fcAnswer{12}{\alertNoH{13}{eleven sixths}.}

\end{itemize}
\end{example}


\begin{itemize}
\item<13-> \alertNoH{14}{Most} frequently, a fraction $\frac{a}{b}$ is read along the template ``\alertNoH{13}{$a$ $b$-th(s)}''.
\item<15-> Most important exceptions: $\displaystyle \frac{1}{2}$ is read as ``one half'' (or just ``half''). $\displaystyle \frac{1}{4}$ is read as both ``one fourth'' and as ''one quarter''. 
\end{itemize}

\end{frame}