\begin{frame}
\begin{itemize}
\item We introduce multiple-digit divisors. 
\item The multi-digit division algorithm is similar to the one-digit divisor, with some differences.
\item The multi-digit division uses the \alertNoH{2}{leading digit of the divisor plus one}, where as the single-digit division uses leading digit as is.
\item An extra step to collect the quotient digits is needed.
\item In the division step, there are three major cases:
\begin{enumerate}
\item Leading digit of remaining dividend is larger than leading digit of divisor.
\item Leading digits of remaining dividend equals the divisor leading digit and the start of the dividend is larger than the divisor.
\item The remaining case: either: 
\begin{itemize}
\item leading digit of remaining dividend equals the divisor leading digit, but the dividend start is smaller than the divisor 
\item leading digit of remaining dividend is smaller than the divisor leading digit.
\end{itemize}
\end{enumerate}
\end{itemize}
\end{frame}