\begin{frame}
\begin{definition}
An integer \alertNoH{7}{greater than one} is prime if it cannot be factored properly.
\end{definition}
\begin{itemize}
\item<2-> \alertNoH{2,7}{$0$ and $1$ are declared not prime by definition}.
\item<3-> Whether \alertNoH{3}{negatives can be prime} is usually left \alertNoH{3}{undefined}. 
\begin{itemize}
\item<4-> Both options could be made to make sense. 
\item<5-> To avoid confusion: avoid question of prime negative numbers.
\end{itemize} 
\end{itemize}
\begin{example}
\begin{tabular}{rcl|}
&Prime?&Full fact.\\
$\alertNoH{6 ,7 }{1}$&\fcAnswer{7 }{no }& \uncover<7->{-}\\
$\alertNoH{8 ,9 }{2}$&\fcAnswer{9 }{yes}&$\uncoverAlert{9 }{2}$\\
$\alertNoH{10,11}{3}$&\fcAnswer{11}{yes} &$\uncoverAlert{11}{3}$\\
$\alertNoH{12,13}{4}$&\fcAnswer{13}{no }&$\uncoverAlert{13}{2\cdot 2}$\\
$\alertNoH{14,15}{5}$&\fcAnswer{15}{yes} &$\uncoverAlert{15}{5}$\\
$\alertNoH{16,17}{6}$&\fcAnswer{17}{no }&$\uncoverAlert{17}{2\cdot 3}$\\
$\alertNoH{18,19}{7}$&\fcAnswer{19}{yes}&$\uncoverAlert{19}{7}$\\
$\alertNoH{20,21}{8}$&\fcAnswer{21}{no }&$\uncoverAlert{21}{2\cdot 2\cdot 2}$\\
\end{tabular}
\begin{tabular}{|rcl}
&Prime?&Full factorization\\
$\alertNoH{22,23}{9 }$&\fcAnswer{23}{no }&$\uncoverAlert{23}{3\cdot 3}$\\
$\alertNoH{24,25}{10}$&\fcAnswer{25}{no }&$\uncoverAlert{25}{2\cdot 5}$\\
$\alertNoH{26,27}{11}$&\fcAnswer{27}{yes}&$\uncoverAlert{27}{11}$\\
$\alertNoH{28,29}{12}$&\fcAnswer{29}{no }&$\uncoverAlert{29}{2\cdot 2\cdot 3}$\\
$\alertNoH{30,31}{13}$&\fcAnswer{31}{yes}&$\uncoverAlert{31}{13}$\\
$\alertNoH{32,33}{14}$&\fcAnswer{33}{no }&$\uncoverAlert{33}{2\cdot 7}$\\
$\alertNoH{34,35}{15}$&\fcAnswer{35}{no }&$\uncoverAlert{35}{3\cdot 5}$\\
$\alertNoH{36,37}{16}$&\fcAnswer{37}{no }&$\uncoverAlert{37}{2\cdot 2\cdot 2\cdot 2}$\\
\end{tabular}
\end{example}
\end{frame}
