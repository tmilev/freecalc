\begin{frame}
\begin{definition}[Group, mathematics]
A group $\mathcal G$ is a set equipped with operation $\cdot$ with $a\cdot b \in \mathcal G$ so that:
\begin{itemize}
\item  $(a\cdot b)\cdot c = a\cdot (b\cdot c)$ for every $a,b,c\in \mathcal G$. \hfill (Associativity)
\item There exists $e\in \mathcal G$ with $e\cdot a = a \cdot e = a$ for every $a\in \mathcal G$. \hfill (Identity)
\item For every $a$ exists $b\in \mathcal G$ s.t. $a\cdot b = e$. Write $b =  a^{\cdot -1}$. \hfill (Inverse)

\end{itemize}
\end{definition}
\begin{definition}[Abelian (commutative) group]
	The group is called abelian (commutative) if in addition: 
\begin{itemize}
\item $a\cdot b = b\cdot a$ for all $a,b\in \mathcal G$.
\end{itemize}
\end{definition}

\end{frame}

\begin{frame}
\begin{itemize}
\item $(a\cdot b)\cdot c= a\cdot(b\cdot c)$.
\item Exists $e$ s.t. $e\cdot a =a\cdot e = a$.
\item Given $a$ exists $b$ s.t. $a\cdot b = e$.
\item $a\cdot b = b\cdot a$. 
\end{itemize}

\begin{example}
Take $G = \mathbb Z$, define $a \cdot b =a + b$. 
\begin{itemize}
\item $(a+b)+c = a+ (b+c)$.
\item Set $e=0 $. Then $0+a= a+ 0 = a$.
\item For every $a$, take $b=-a$. Then $a + b = a+ (-a) = 0=e$.
\item $a+b=b+a$ for all $a,b$.
\end{itemize}
\end{example}

\end{frame}