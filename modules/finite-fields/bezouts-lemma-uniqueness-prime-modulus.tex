\begin{frame}
	\vskip -0.2cm
	\footnotesize
\begin{lemma}[B\'ezout's lemma]
Let $a,b\in \mathbb  K[x]$ be non-constant monic polynomials with greatest common divisor $d$ with $\mathbb K$ - field (can be $\mathbb Q, \mathbb R, \mathbb Z_p$). $\Rightarrow$ there exist monic polynomials $r,s\in  \mathbb K[x]$ such that 
\[
ra+sb=d
\]
Furthermore $\deg r < \deg b$ and $\deg s < \deg a$.
\end{lemma}
\begin{proof}
\tiny 
\only<1|handout:1>{
Carry out the consecutive divisions with remainder


\hfil\hfil$
\begin{array}{rcll|l|lcl}
a&=& q_1b &+c_3 &\deg c_3 < \deg b &c_{n+1}|b, c_{n+1}|c_3&\Rightarrow& c_{n+1} | a\\
b&=& q_2c_3 &+c_4& \deg c_4 < \deg c_3& c_{n+1}|c_4, c_{n+1}|c_3&\Rightarrow& c_{n+1} | b \\
c_3&=&q_3 c_4&+c_5& \deg c_5 < \deg c_4& c_{n+1}|c_4, c_{n+1}|c_3&\Rightarrow& c_{n+1} | c_3 \\
&\vdots&&&&\vdots \\
c_{n-1}&=&q_{n-1} c_{n}&+c_{n+1}  & \deg c_{n+1} < \deg c_n& c_{n+1}|c_n, c_{n+1}|c_{n+1}&\Rightarrow& c_{n+1}|c_{n-1}\\
c_n&=&q_n c_{n+1}&+0  & \deg c_k \text{ decreases so we must reach }0& c_{n+1}|c_n
\end{array}
$

By reading the equations above in reverse order, we see that $c_{n+1}|c_n, c_{n+1}|c_{n-1}, \dots$ and so $c_{n+1}|a$ and $c_{n+1}|b$, i.e., $c_{n+1}$ is a common divisor of $a,b$. Transfer each term $q_{k-1}c_k $ to the left hand side and solve for $c_{n+1}$ as shown below:

\hfil\hfil$
\begin{array}{rrcl|lcr}
a&- q_1b &=&c_3 &c_3 \text{ is of the form } \bullet \cdot a+ \bullet \cdot b \\
b&- q_2c_3 &=&c_4& c_4\text{ is of the form } \bullet \cdot b+ \bullet \cdot c_3&\Rightarrow &c_4 \text{ is of the form } \bullet \cdot a+ \bullet \cdot b\\
c_3&-q_3 c_4&=&c_5&   c_5\text{ is of the form } \bullet \cdot c_3+ \bullet \cdot c_4&\Rightarrow& c_5 \text{ is of the form } \bullet \cdot a+ \bullet \cdot b\\
&&\vdots&&\vdots \\
c_{n-1}&-q_{n-1} c_{n}&=&c_{n+1}  & c_{n+1}\text{ is of the form } \bullet \cdot c_{n-1}+ \bullet \cdot c_{n}&\Rightarrow& c_{n+1} \text{ is of the form } \bullet \cdot a+ \bullet \cdot b
\end{array}
$

Thus $c_{n+1}$ is a linear combination $\bullet \cdot a+ \bullet \cdot b$. So every common divisor of $a,b$ is a divisor of $c_{n+1}$. $\Rightarrow$ $c_{n+1}=d$.
}
\only<2-|handout:2>{
	
After possible swap of $a$ and $b$, assume $\deg a \geq  \deg b$.
	
\hfil\hfil$
\begin{array}{rclcl|ll}
a&-& q_1b &=&c_3 &\deg q_1 = \deg a- \deg b& \deg c_3 <\deg b  \\
b&-& q_2c_3 &=&c_4&  \deg q_2 = \deg b - \deg c_3& \deg c_4 < \deg c_3\\
c_3&-&q_3 c_4&=&c_5&   \deg q_3 = \deg c_3-\deg c_4& \deg c_5 < \deg c_4\\
&&&\vdots&&\vdots \\
c_{n-2}&-&q_{n-2} c_{n-1}&=&c_{n} &\deg q_{n-2} = \deg c_{n-2}-\deg c_{n-1}& \deg c_{n}<\deg c_{n-1}\\
c_{n-1}&-&q_{n-1} c_{n}&=&c_{n+1}=d  &\deg q_{n-1} = \deg c_{n-1}-\deg c_n& \deg c_{n+1}<\deg c_n\\
\end{array}
$

Expand $d$ as a combination of $a$ and $b$:

\hfil \hfil
$\begin{array}{rclcl}
d&=&c_{n-1}&-&q_{n-1}c_n \\
&=& c_{n-1}&-& q_{n-1}\left( c_{n-2}-q_{n-2}c_{n-1} \right)\\
&=& \left(c_{n-3}-q_{n-3}c_{n-2}\right) &-& \alert<handout:2>{ q_{n-1}} \left( c_{n-2}- \alert<handout:2>{q_{n-2}}\left(c_{n-3}-\alert<handout:2>{q_{n-3}c_{n-2}} \right) \right)\\
&\vdots&
\end{array}
$

$\Rightarrow$ Highest order summand involving $b$ in full expansion of $d$ is: 

\hfil \hfil $q_{n-1}\cdot q_{n-2}\cdots q_{1}\cdot b$

with degree 

\hfil \hfil$
\deg \left(q_{n-1} \cdots q_1\right) = (\deg a-\deg b) + \left(\deg b-\deg c_3\right) + \dots + \left( \deg c_{n-1}-\deg c_n\right) = \deg a-\deg c_n<\deg q
$

Similarly, the highest order summand involving $a$ is $q_{n-1}\cdot q_{n-2}\cdots q_2\cdot a $, with coefficient of degree less than $b$.
	
}

\end{proof}

\vskip 5cm
\end{frame}