\begin{frame}
\frametitle{$=$ vs $\equiv$}

\begin{itemize}
\item $\equiv$ stands for equivalence $\mod n$. We write $8\equiv 2 \mod 3 $.
\item We also have $(8\mod 3)= (2\mod 3)$: no need for $\equiv$ here.
\item Writing $\mod 3$ everywhere is cumbersome.
\item Convention: drop $\mod 3$, except for the last line:
\[
\begin{array}{c}
2+2+2+2 = 8=2 \mod 3\\
\text{instead of}\\

((2+2+2+2) (\mod 3))=(8 \mod 3) = (2\mod 3) \\
\end{array}
\]
\item We can write:
\[
2+2+2+2=8\equiv 2 \mod 3
\]
to stress the fact that $8\neq 2$ over $\mathbb Z$.
\item With variables, we can't always tell $=$ from $\equiv$:
\[
2+2+2+2=8 \mod p
\] 
\item Informally, we allow ourselves to write $8\in \mathbb Z_3$, where it's implied $8=2\mod 3$ are two different ways of writing $2\in \mathbb Z_3$.
\end{itemize}
\end{frame}