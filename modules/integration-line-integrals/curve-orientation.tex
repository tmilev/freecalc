\begin{frame}
\frametitle{Curve orientation}
\vskip -0.1cm
\begin{columns}
\column{0.11\textwidth}
\begin{pspicture}(-0.5,-0.5)(1,1.5)
\fcBoundingBox{-0.5}{-0.4}{1}{1.5}
\pstVerb{/theCurve {t cos 2 t mul cos mul t sin 2 t mul cos mul t 50 div dup mul add t 90 div 3 exp sub} def
/tmin -35 def
/tmax 90 def
}
\uncover<1-14>{
\parametricplot[linecolor=\fcColorGraph]{tmin}{tmax}{theCurve}
\parametricplot[arrows=->,linecolor=\fcColorGraph]{tmin}{40}{theCurve}
}
\uncover<15->{
\parametricplot[linewidth=2pt, linecolor=\fcColorGraph]{tmin}{tmax}{theCurve}
\parametricplot[linewidth=2pt,  arrows=->, linecolor= \fcColorGraph]{ tmax}{ 30}{ theCurve}
}
\uncover<3->{
\fcFullDot{/t tmin def theCurve pop}{/t tmin def theCurve exch pop}
\fcFullDot{/t tmax def theCurve pop}{/t tmax def theCurve exch pop}
}
\rput(0.5,1){$\alert<15>{\uncover<15->{-}C}$} 
\end{pspicture}

\column{0.89\textwidth}
\begin{itemize}
%\item Let $C$ be a curve with one-to-one parametrization $\fcv r(t), t\in [a,b]$. 
%\item Then $C$ starts at $A=\fcv r(a)$ and ends at $B=\fcv r(b)$.
%\item We say that the curve is oriented from $A$ to $B$. 
\item Let $C$ be curve image (not equipped with parametrization). 
\item<2-> Suppose $C$ can be equipped with some one-to-one continuous parametrization of the form $\fcv r(t)$, $t\in [a,b]$ so that \alert<3>{$A=\fcv r(a)$ (starting point),  $B=\fcv r(b)$ (endpoint), $A\neq B$}.
\end{itemize}
\end{columns}
\uncover<4->{
\begin{definition}[Curve orientation, endpoints are distinct]
\begin{itemize}
\item<4-> We say the parametrization $\fcv r$ \emph{orients} the curve $C$.
\item<5-> We say that two one-to-one parametrizations of $C$ have \emph{the same orientation} if they determine the same starting and endpoints. 
\item<6-> To orient a curve image $C$ means to specify which of the two endpoints is a starting point and which - endpoint. 
\end{itemize}
\end{definition}
}
\begin{itemize}
\only<1-9>{
\item<7-> Alternatively, to orient a curve means to specify the order in which its points are traversed (``direction of flow'').
\item<8-> The definition can be extended to when the parametr. is not one-to-one (allowing $A=B$). \uncover<9->{Requires 1-dimensional manifolds.}
}
\only<10->{
\item<10-> Consider the curve parametrized by $\alert<11,12,15>{\fcv s(t)= \fcv r(b (1-t)+ t a) }$, $t\in [0,1]$. \uncover<11->{Then $\alert<11>{\fcv s(0)=\fcv r(b)}$ and $\alert<12>{\fcv s(1)=\fcv r(a)}$,} \uncover<13->{so the curve image coincides with that of $C$ but has opposite orientation.} \uncover<14->{We write \alert<15>{$-C$} for the curve with image as $C$ but opposite orientation.}
}
\end{itemize}

\vskip 10cm

\end{frame}