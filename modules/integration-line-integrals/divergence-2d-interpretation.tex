\begin{frame}
\small
  \frametitle{Interpretation of Divergence}

$$\oint_{\partial D} \fcv{F} \cdot \fcv{dn} = \iint_{D} \divg \fcv{F} \, dA$$

  \begin{itemize}
    \item If $\divg \fcv{F}(p)>0$:
    \begin{itemize}
      \item \pause for small enough regions around $p$, $\fcv{F}$ carries more outside than it brings inside through the boundary;
      \item \pause $p$ acts as a source.
    \end{itemize}
    \item If $\divg \fcv{F}(p)<0$:
    \begin{itemize}
      \item \pause for small enough regions around $p$, $\fcv{F}$ carries less outside than it brings inside through the boundary;
      \item \pause $p$ acts as a sink
    \end{itemize}
    \item If the divergence is identically zero on a region $D$:
    \begin{itemize}
      \item \pause the outward flux through any closed curve is zero;
      \item \pause the amount pushed inside in some places is equal to the amount pushed outside somewhere else;
      \item The field $\fcv{F}$ is \emph{incompressible} or \emph{solenoidal}.
    \end{itemize}
  \end{itemize}
\end{frame}