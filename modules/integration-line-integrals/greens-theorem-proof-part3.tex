\begin{frame}
\begin{columns}
\column{0.3\textwidth}
\psset{xunit=0.5cm, yunit=0.5cm}
\begin{pspicture}(-1,-1)(2,1)%
\tiny%
\fcAxesStandard{-0.5}{-0.5}{3.2}{2.7}%
\pstVerb{
/theCurve {
t -90 gt t 90 le and {t cos 2 add t sin 1.5 add  }ifelse
t 90 gt t 270 le {-t  cos 2 add t sin 1.5 add  }if
} def
/tmin -360 def
/tmax 360 3 mul def
}
\pscustom*[linecolor=\fcColorAreaUnderGraph]{
\parametricplot{tmin}{tmax}{theCurve}
}
\end{pspicture}
\column{0.7\textwidth}
\vskip -0.2cm
\begin{theorem}[Green]
$\displaystyle  \oint_{\partial D}\left( P \diff x + Q \diff y\right) =  \iint_D \left( \frac{ \partial Q}{\partial x} - \frac{ \partial P}{ \partial y} \right) \diff x \diff y \; .$
\end{theorem}
\end{columns}
\begin{proof} [When $D$= representable by curv. trapezoids in both directions]
So far, we demonstrated that  
$
\begin{array}{rcl|l}
\oint_{\partial D} P \diff x &=&\iint_D\left(- \frac{ \partial P}{ \partial y} \right) \diff x \diff y&  \text{curv. trapezoids vert. bases}\\
\oint_{\partial D} Q \diff y&=& \iint_D  \frac{ \partial Q}{\partial x} \diff x \diff y &\text{curv. trapezoids horiz. bases}
\end{array}
$.

Suppose $D$ can be represented as a union of curvilinear trapezoids with vertical bases, pairwise intersecting on their boundaries only.
\end{proof}

\vskip 10cm

\end{frame}