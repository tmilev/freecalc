\begin{frame}
\frametitle{Line Integrals from Vector Fields}
\begin{itemize}
\item Let $C$ be piecewise smooth, \emph{oriented} curve, parametrized via $\fcv r(t)$.
\item<2-> Let $\diff s$ be the element of arclength.
\item<3-> Let $\fcv{F}$ be a continuous vector defined on $C$.
\item<4-> Let $\fcv{T}$ be unit tangent vector on $C$ compatible with orientation.
\item<5-> Let $\fcv N$ be unit vector perpendicular to $\fcv T$ (only for planar curves).
\uncover<6->{%
\begin{definition}
In any dimension, define the line integral of $\fcv{F}$ along $C$ as
$\displaystyle
\int_C \fcv{F} \cdot \fcv{\diff r} = \int_{C} \fcv{F} \cdot \fcv{T} ~\diff s, \text{ where }  \fcv{\diff r} = \fcv{T}~ \diff s \; .
$

\uncover<7->{In dimension 2, define the line integral of $\fcv F$ across $C$ as 
$\displaystyle \int_{C} \fcv{F} \cdot \fcv{N}~\diff s = \int_C \fcv{F} \cdot \fcv{\diff n}, \text{ where } \fcv{\diff n} = \fcv{N}~\diff s \; .$
}
\end{definition}
}
\item<8-> Line integral = work done by force $\fcv{F}$ on particle moving along $C$.
\item<9-> Line integral across $C$ = flux across a membrane: $\fcv{F}\cdot \fcv{N}$ is the normal component of $\fcv{F}$.
 \end{itemize}

\end{frame}