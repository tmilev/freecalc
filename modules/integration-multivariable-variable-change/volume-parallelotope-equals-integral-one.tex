\begin{frame}
\frametitle{Integral and Algebraic Volume Definitions Agree}
\begin{columns}
\column{0.25\textwidth}
\psset{xunit=1.5cm, yunit=1.5cm}
\begin{pspicture}
\tiny
\fcBoundingBox{-0.2}{-0.2}{1.2}{1.5}%
\renewcommand{\fcScreenStyle}{x}%
\pstVerb{10 dict begin 
/ver1 [1 0 0.05] def 
/ver2 [0 1 0.05] def 
/ver3 [0.3 0.3 1] def  
/normal ver1 ver2 \fcVectorCrossVector def 
/perpFoot ver3 normal normal ver3 \fcVectorScalarVector normal normal \fcVectorScalarVector div \fcVectorTimesScalar \fcVectorMinusVector def
/perpendicular ver3 perpFoot \fcVectorMinusVector def
/innerCorner ver1 ver2 \fcVectorPlusVector 0.1 \fcVectorTimesScalar def
/innerCornerBoxes ver1 ver2 ver3 \fcVectorPlusVector \fcVectorPlusVector 0.2 \fcVectorTimesScalar def
}%
\fcStartIIIdScene%
\fcBoxIIIdInScene[colorUV=cyan]{[0 0 0]}{ver1}{ver2}{ver3}%
\multido{\ra=0+0.2}{5}{%
\fcBoxIIIdInScene[linewidth=0.5, colorUV={1 0.5 0.5}, forceForeground=true] {innerCorner ver3 \ra \space \fcVectorTimesScalar \fcVectorPlusVector}{innerCorner ver3 \ra \space \fcVectorTimesScalar \fcVectorPlusVector ver1 0.8 \fcVectorTimesScalar \fcVectorPlusVector}{innerCorner ver3 \ra \space \fcVectorTimesScalar \fcVectorPlusVector ver2 0.8 \fcVectorTimesScalar \fcVectorPlusVector}{innerCorner ver3 \ra \space \fcVectorTimesScalar \fcVectorPlusVector perpendicular 0.2 \fcVectorTimesScalar \fcVectorPlusVector}%
}%
\fcAxesIIIdInScene{2}{2.5}{2}
\fcFinishIIIdScene[true]

\pstVerb{end}
\end{pspicture}

\column{0.75\textwidth}
\begin{itemize}
\item Let $ \fcv v_1=(v_{11}, \dots, v_{1n})$, $\dots$, $\fcv v_n=(v_{n1},\dots, v_{nn} )$ be $n$-vectors in $n$-dimensional space.
\item Let $\mathcal R_k$ be the parallelotope spanned by $\fcv v_1, \dots, \fcv v_k$.
\item Let $h_k$ be the height of $\mathcal R_k$ with base $\mathcal R_{k-1}$.
\end{itemize}
\end{columns}
\begin{theorem}
$\Vol_n(\mathcal R_n)=h_n \Vol_{n-1}(\mathcal R_{n-1})= \displaystyle {\int\dots \int}_{\mathcal R_n} 1 \cdot \diff x_1\dots \diff x_n $.
\end{theorem}
\begin{itemize}
\item Left hand side: approximate volume with slabs parallel to base.
\item Right hand side: approx. vol. with boxes, sides along coord. axes.
\item Theorem is fully intuitive but its proof is surprisingly laborious.
\end{itemize}

\end{frame}