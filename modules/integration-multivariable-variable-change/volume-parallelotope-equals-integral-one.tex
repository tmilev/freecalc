\begin{frame}
\frametitle{Integral and Algebraic Volume Definitions Agree}
\begin{columns}
\column{0.25\textwidth}
\psset{xunit=1.5cm, yunit=1.5cm}
\begin{pspicture}
\tiny
\fcBoundingBox{-0.2}{-0.2}{1.2}{1.5}%
\renewcommand{\fcScreenStyle}{x}%
\pstVerb{10 dict begin 
/ver1 [1 0 0.05] def 
/ver2 [0 1 0.05] def 
/ver3 [0.3 0.3 1] def  
/normal ver1 ver2 \fcVectorCrossVector def 
/perpFoot ver3 normal normal ver3 \fcVectorScalarVector normal normal \fcVectorScalarVector div \fcVectorTimesScalar \fcVectorMinusVector def
/perpendicular ver3 perpFoot \fcVectorMinusVector def
/innerCorner ver1 ver2 \fcVectorPlusVector 0.1 \fcVectorTimesScalar def
/innerCornerBoxes ver1 ver2 ver3 \fcVectorPlusVector \fcVectorPlusVector 0.2 \fcVectorTimesScalar def
}%
\only<1>{
\fcBoxIIIdFilledNew[dashes={[0.5 3] 0}, colorUV=cyan]{[0 0 0]}{ver1}{ver2}{ver3}%
}
\only<3>{
\fcStartIIIdScene%
\fcBoxIIIdInScene[colorUV=cyan]{[0 0 0]}{ver1}{ver2}{ver3}%
\multido{\ra=0+0.2}{5}{%
\fcBoxIIIdInScene[linewidth=0.5, colorUV={1 0.5 0.5}, forceForeground=true] {innerCorner ver3 \ra \space \fcVectorTimesScalar \fcVectorPlusVector}{innerCorner ver3 \ra \space \fcVectorTimesScalar \fcVectorPlusVector ver1 0.8 \fcVectorTimesScalar \fcVectorPlusVector }{ innerCorner ver3 \ra \space \fcVectorTimesScalar \fcVectorPlusVector ver2 0.8 \fcVectorTimesScalar \fcVectorPlusVector}{innerCorner ver3 \ra \space \fcVectorTimesScalar \fcVectorPlusVector perpendicular 0.2 \fcVectorTimesScalar \fcVectorPlusVector}%
}%
\fcAxesIIIdInScene{2}{2.5}{2}%
\fcFinishIIIdScene[true]%
}
\only<1>{
\pscustom{%
\code{
30 dict begin
/boxSides
[ [ver1 ver2 \fcVectorCrossVector 0] 
  [ver1 ver2 \fcVectorCrossVector -1 \fcVectorScalarVector dup ver3 \fcVectorScalarVector]
  [ver2 ver3 \fcVectorCrossVector 0] 
  [ver2 ver3 \fcVectorCrossVector -1 \fcVectorScalarVector dup ver1 \fcVectorScalarVector]
  [ver3 ver1 \fcVectorCrossVector 0] 
  [ver3 ver1 \fcVectorCrossVector -1 \fcVectorScalarVector dup ver2 \fcVectorScalarVector]
]
def
/isInBox {/currentPoint exch def true boxSides{\fcArrayToStack exch currentPoint \fcVectorScalarVector gt {pop false exit}if }forall } def
/numIt 10 def %
ver1 ver2 ver3 \fcVectorPlusVector \fcVectorPlusVector \fcArrayToStack
/DeltaZ exch numIt div def %
/DeltaY exch numIt div def %
/DeltaX exch numIt div def %
/DeltaXvector [DeltaX 0 0] def %
/DeltaYvector [0 DeltaY 0] def %
/DeltaZvector [0 0 DeltaZ] def %
/counter1 -1 def
numIt {
/counter1 counter1 1 add def
/counter2 -1 def
numIt {
/counter2 counter2 1 add def
/counter3 -1 def
numIt {
/counter3 counter3 1 add def
/startingCorner [DeltaX counter1 mul DeltaY counter2 mul DeltaZ coutner3 mul ] def 
/theCorners 
[startingCorner 
 startingCorner DeltaXvector \fcVectorPlusVector
 startingCorner DeltaYvector \fcVectorPlusVector
 startingCorner DeltaZvector \fcVectorPlusVector
 startingCorner DeltaXvector DeltaYvector \fcVectorPlusVector \fcVectorPlusVector
 startingCorner DeltaYvector DeltaZvector \fcVectorPlusVector \fcVectorPlusVector
 startingCorner DeltaZvector DeltaXvector \fcVectorPlusVector \fcVectorPlusVector
 startingCorner DeltaXvector DeltaYvector DeltaZvector \fcVectorPlusVector \fcVectorPlusVector \fcVectorPlusVector 
 ]
def
true theCorners{isInBox not{pop false exit}if }forall
{startingCorner 
 startingCorner DeltaXvector \fcVectorPlusVector
 startingCorner DeltaYvector \fcVectorPlusVector
 startingCorner DeltaZvector \fcVectorPlusVector
 [\fcGetColorCode{black}] [1 0.5 0.5] true true [\fcDashes] \fcBoxIIIdFilledCode
}if
}repeat
}repeat
}repeat
end
}%
}%
\fcBoxIIIdHollowNew[dashes={[0.5 3] 0}, colorUV=cyan]{[0 0 0]}{ver1}{ver2}{ver3}%
}%
\pstVerb{end}%
\end{pspicture}

\column{0.75\textwidth}
\begin{itemize}
\item Let $ \fcv v_1=(v_{11}, \dots, v_{1n})$, $\dots$, $\fcv v_n=(v_{n1},\dots, v_{nn} )$ be $n$-vectors in $n$-dimensional space.
\item Let $\mathcal R_k$ be the parallelotope spanned by $\fcv v_1, \dots, \fcv v_k$.
\item Let $h_k$ be the height of $\mathcal R_k$ with base $\mathcal R_{k-1}$.
\end{itemize}
\end{columns}
\begin{theorem}
$\Vol_n(\mathcal R_n)=h_n \Vol_{n-1}(\mathcal R_{n-1})= \displaystyle {\int\dots \int}_{ \mathcal R_n} 1 \cdot \diff x_1\dots \diff x_n $.
\end{theorem}
\begin{itemize}
\item<2-> \alert<2>{Right hand side: approx. vol. with boxes, sides along coord. axes.}
\item<3-> \alert<3>{ Left hand side: approximate volume with slabs parallel to base.}
\item<4-> Theorem is fully intuitive but its proof is surprisingly laborious.
\end{itemize}

\end{frame}