\begin{frame}
\frametitle{Theoretical example: Electric force on a lamina}
\begin{itemize}
\item Given:
\begin{itemize}
\item a charge $Q$, located at the origin;
\item charge $q$, uniformly distributed on a planar lamina $\mathcal{R}$.
\end{itemize}
\item What is the resulting (total) force $\fcv F$ on $Q$?
\item<5-> Recall that the attraction force exerted on a \alertNoH{6}{charge $Q$} located at the origin by \alertNoH{7}{a charge $c$ located at a point} with \alertNoH{6}{position vector $\fcv r$} is $\alertNoH{6}{ \varepsilon Q} \alertNoH{7}{c} \alertNoH{6}{\frac{ \fcv r }{|\fcv r|^3}}$.
\end{itemize}
\[
\renewcommand{\arraystretch}{1.8}
\begin{array}{rcl}
\uncover<2->{\alertNoH{8}{\displaystyle \diff q}} &\uncover<2->{\alertNoH{8}{=}} & \uncover<2->{(\text{density of charge})  \diff A} \uncover<3->{ =\displaystyle \alertNoH{8}{ \frac{q}{A(\mathcal{R})} \diff A} }\\
\displaystyle \uncover<4->{ \alertNoH{10}{\diff \fcv{F}}} &\uncover<4->{\alertNoH{10}{=}}& \displaystyle \uncover<4->{\alertNoH{6}{ \varepsilon Q \frac{ \fcv{r }}{ | \fcv{ r}|^3}} \alertNoH{7,8}{\diff q}} \uncover<8->{ = \alertNoH{10}{\varepsilon \frac{Q \alertNoH{8}{q}}{\alertNoH{8}{A(\mathcal{R})} } \frac{ \fcv{r }}{ | \fcv{ r}|^3} \alertNoH{8}{\diff A}}} \\
\displaystyle \alertNoH{12}{\uncover<9->{\fcv{F}}} &\uncover<9->{\alertNoH{12}{=}} & \displaystyle \uncover<9->{\iint_{\mathcal{R}} \alertNoH{10}{\diff \fcv{F}}} \uncover<10->{= \iint_{\mathcal{R} }  \alertNoH{11}{\varepsilon \alertNoH{10}{ \frac{ Q q}{A(\mathcal{R})} }  \frac{ \fcv{r}}{|\fcv{r}|^3}  \diff A}}\\
\uncover<11,12->{&=&\displaystyle  \alertNoH{12}{ \alertNoH{11}{\varepsilon \frac{Q q}{A(\mathcal{R})}}  \iint_{\mathcal{R} } \frac{\fcv r}{ |\fcv{r}|^3} \diff A}}
\end{array}
\]
\end{frame}
