% begin module FTC-part2-ex7
\begin{frame}
\begin{example} %[Example 2, p. 357]
\begin{columns}
\column{0.2\textwidth}
\psset{xunit=0.4cm, yunit=0.4cm}
\begin{pspicture}(-1.000000, -5)(1.500000,5) 
\psframe*[linecolor=white](-1.000000,-5)(7,5) 
\tiny 
\pscustom*[linecolor=cyan]{ %Function formula: \cos{}x 
\psplot[linecolor=\psColorGraph, plotpoints=1000]{0}{1}{x 57.29578 mul cos }\psline(1.000000, 0)(0.000000, 0)}
%Function formula: \cos{}x 
\psplot[linecolor=\psColorGraph, plotpoints=1000]{0}{6}{x 57.29578 mul cos }
\psaxesStandard{-0.6}{-0.6}{6}{1.1}
\end{pspicture} 
\column{0.8\textwidth}
Find the area under the cosine curve from $0$ to $b$, where $0 \leq b \leq \pi /2$.
\begin{itemize}
\item<2->  $\cos x$ is continuous on $[0, \pi /2]$ (in fact, it's continuous everywhere).
\item<3-| alert@3-4>  An antiderivative is \uncover<4->{$\sin x$.}
\end{itemize}
\[
\uncover<5->{%
\int_{\alert<handout:0| 5>{0}}^{\alert<handout:0| 5>{b}} \cos x \ \diff x = \left[ \sin x \right]_{\alert<handout:0| 5>{0}}^{\alert<handout:0| 5>{b}} %
}%
\uncover<6->{%
 = \sin (b) - \sin (0)%
}%
\uncover<7->{%
 = \sin b%
}%
\]
\end{columns}
\end{example}
\end{frame}
% end module FTC-part2-ex7
