% begin module arcsin-ex1
\begin{frame}
\vskip -0.1cm
\uncover<2->{
\begin{observation}
\begin{itemize}
\item $\alertNoH{3}{\Arcsin y}=$ \alertNoH{3,8}{the appropriate angle} \alertNoH{4}{whose sine} \alertNoH{5}{equals $y$}. 
%\item $\alertNoH{3}{\Arcsin y}$ is \alertNoH{3,5}{the appropriate angle $\theta$} for which $\alertNoH{4}{\sin \theta =y}$.
\item<8-> \alertNoH{8}{Important: the output angle must lie in the interval $ \left[-\frac{\pi}{2},\frac{\pi}{2}\right]$.}
\end{itemize}  
\end{observation}
}
\begin{example}
\hfil \hfil Find $\displaystyle  \alertNoH{3}{ \Arcsin \left( \alertNoH{5}{\frac{1}{2}} \right)}$.
\begin{itemize}
\item<3-> $\displaystyle \alertNoH{4}{\sin} \left(\alertNoH{3}{ \fcAnswerUncover{3}{7}{ \worksheet{\alertNoH{8}{\frac{\pi}{ 6}}}} } \right) = \alertNoH{5}{\frac{1}{2}} $.
\item<8-> $\displaystyle -\frac{\pi}{2} \leq \worksheet{\alertNoH{8}{\frac{\pi }{6}}} \leq \frac{\pi}{2}$.
\item<9-> Therefore $ \Arcsin \left( \frac{1}{2}\right) =  \worksheet{\frac{ \pi }{6} }$.
\end{itemize}
\end{example}
\end{frame}
% end module arcsin-ex1
