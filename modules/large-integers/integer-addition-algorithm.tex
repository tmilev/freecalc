\begin{frame}
We covered addition by example; algorithm follows. Feel free to skip.
\begin{algorithm}[Addition base 10]
\begin{itemize}
\item[1.] Set $\bf maxNumberOfDigits$ to the larger number of digits.
\item[2.] For each digit position $\bf{i}$, starting at position $\bf 0$:
\item[2.1.] - Let $\bf{topDigit}$ and $\bf{bottomDigit}$ be the two digits in $\bf{i}^{th}$ position. If smaller number has no digit at the position,  set its digit to $\bf 0$.
\item[2.2.] - Set $\bf{digitSum}$ to $\bf{topDigit} + \bf{bottomDigit}$.
\item[2.3.] 
- If $\bf{digitSum} \geq 10$, set $\bf{resultDigit}=\bf{digitSum} - 10$ and $\bf{carryOver} = 1$.

- Else  $\bf{digitSum} < 10$, so set $\bf{resultDigit}=\bf{digitSum}$ and $\bf{carryOver} = 0$.
\item[2.4.] - Set the result's $\bf i^{th}$ digit to $\bf {resultDigit}$.
\item[3.] If after last step $\bf carryOver$ is $\bf 1$, set $\bf 1$ as the result's $\bf (maxNumberOfDigits +1)^{th}$ digit.
\end{itemize}
\end{algorithm}
\end{frame}