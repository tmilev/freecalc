\begin{frame}
\begin{itemize}
\item Goal: formalize decimal notation \& notation in other bases.
\item In other words, formalize usual number notation: example: $1403$. 
\item We sometimes over-line numbers, for example: $1403 =\overline{1403}$.
\item Over-line is used to parametrize digits, for example we can write $1403=\overline{abcd}$ with $a=1$, $b=4$, $c=0$, $d=3$.
\end{itemize}
\begin{definition}
The $n+1$-digit number in base $b$ denoted by $\overline{a_1a_2\dots a_n}$ is defined as the number
$
\overline{a_0a_1\dots a_{n}} = a_0 b^n + a_1 b^{n-1} \dots + a_n
$
\begin{itemize}
\item The numbers $a_i$ are called digits.
\item We require that $0\leq a_i < b$.
\item We require that $a_0 \neq 0$, i.e., that the leading digit is non-zero.
\end{itemize}
\end{definition}
\begin{itemize}
\item For example, $1403$ has $4$ digits, with $a_0 = 1, a_1 = 4, a_2 = 0, a_3 = 3$.
\item The base is implied to be $b=10$, so $0\leq a_i<10$, i.e., each digits is between $0$ and $9$. 
\end{itemize}
\end{frame}