% begin module limits-one-sided-ex6
\begin{frame}
\frametitle{One-sided Limits}
\begin{example}[Example 6, p. 99]
\begin{columns}[c]
\column{.5\textwidth}
The Heaviside function $H$ is defined by
\[
H(t) = \left\{ \begin{array}{lr}
0 & \textrm{ if } t < 0\\
1 & \textrm{ if } t \geq 0
\end{array}\right. .
\]
\ \includegraphics[height=2.5cm]{limits/pictures/02-02-ex6.pdf}%
\column{.5\textwidth}
\begin{itemize}
\item<2->  As $t$ approaches $0$ from the left, $H(t)$ approaches 0.
\item<3->  As $t$ approaches $0$ from the right, $H(t)$ approaches 1.
\item<4->  There is no single number that $H(t)$ approaches as $t$ approaches 0.
\item<5->  Therefore $\lim_{t\rightarrow 0} H(t)$ doesn't exist.
\end{itemize}
\end{columns}
\end{example}
\end{frame}
% end module limits-one-sided-ex6
