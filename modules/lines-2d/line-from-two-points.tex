\begin{frame}
\begin{example}
\newcommand{\xone}{\makebox[\widthof{$x_1$}]{1}}
\newcommand{\xtwo}{\makebox[\widthof{$x_2$}]{2}}
\newcommand{\yone}{\makebox[\widthof{$y_1$}]{3}}
\newcommand{\ytwo}{\makebox[\widthof{$y_2$}]{6}}
Find an equation of the line passing through $(\alertNoH{13}{\xone},\alertNoH{13}{\yone})$ and $(\alertNoH{13}{\xtwo},\alertNoH{13}{\ytwo})$.
\[
\fcAnswerUncover{3}{9}{(\alertNoH{13}{\xtwo}-\alertNoH{13}{\xone})} \uncover<3->{(\alertNoH{12,5}{y-\alertNoH{13}{\yone}})=}
\fcAnswerUncover{3}{10}{(\alertNoH{13}{\ytwo}-\alertNoH{13}{\yone})} \uncover<3->{(\alertNoH{12,4}{x-\alertNoH{13}{\xone}})} \]
\begin{itemize}

\item<2-> It suffices to manufacture a linear equation such that when we \alertNoH{3}{plug in} \alertNoH{3}{$ (\alertNoH{13}{\xone}, \alertNoH{13}{\yone} )$} and \alertNoH{6,7}{$ (\alertNoH{13}{\xtwo}, \alertNoH{13}{\ytwo} )$} we get an identity.
\item<3-> A (very simple) equation satisfied by $\alertNoH{3, 4}{x =\alertNoH{13}{\xone}}$, $\alertNoH{3, 5}{y=\alertNoH{13}{\yone}}$ is:

\hfil \hfil$
\alertNoH{5}{\alertNoH{7}{y}-\alertNoH{13}{\yone}}=\alertNoH{4}{\alertNoH{6}{x}-\alertNoH{13}{\xone}}.
$

\uncover<4->{This is so because both sides \alertNoH{4,5}{become zero} when $\alertNoH{4}{x=\alertNoH{13}{\xone}}$, $\alertNoH{5}{y=\alertNoH{13}{\yone}}$.}
\item<6-> If we plug in $\alertNoH{6}{x=\alertNoH{13}{\xtwo}}$ and $\alertNoH{7}{y=\alertNoH{13}{\ytwo}}$ in the above \alertNoH{8}{we don't get an identity}\uncover<9->{, but that can be easily fixed:}

\hfil \hfil $
\uncover<9->{\alertNoH{9}{(\alertNoH{13}{\xtwo}-\alertNoH{13}{\xone})}} \alertNoH{10}{(\alertNoH{7,13}{\ytwo}-\alertNoH{13}{\yone})}
\only<handout:0|6-9>{\alertNoH{8}{\neq}}  
\only<10->{=} 
\uncover<10->{\alertNoH{10}{(\alertNoH{13}{\ytwo}-\alertNoH{13}{\yone})}} \alertNoH{9}{(\alertNoH{6,13}{\xtwo}-\alertNoH{13}{\xone})} 
$
\item<11-> Perhaps the last modification caused $x=\alertNoH{13}{\xone}$, $y=\alertNoH{13}{\yone}$ to no longer be solutions? \uncover<12->{No - both sides are still zero when $\alertNoH{12}{x=\alertNoH{13}{\xone}}$, $\alertNoH{12}{y=\alertNoH{13}{\yone}}$.}
\end{itemize}
\end{example}
\end{frame}

\begin{frame}
\begin{example}
Find an equation of the line passing through $(\alertNoH{1}{x_1},\alertNoH{1}{y_1})$ and $(\alertNoH{1}{x_2}, \alertNoH{1}{y_2})$.
\[
\alertNoH{2}{(\alertNoH{1}{x_2}-\alertNoH{1}{x_1})(y-\alertNoH{1}{y_1})=(\alertNoH{1}{y_2}-\alertNoH{1}{y_1})(x-\alertNoH{1}{x_1})}
\]
\begin{itemize}

\item It suffices to manufacture a linear equation such that when we plug in $(\alertNoH{1}{x_1},\alertNoH{1}{y_1})$ and $(\alertNoH{1}{x_2},\alertNoH{1}{y_2})$ we get an identity.
\item A (very simple) equation satisfied by $x=\alertNoH{1}{x_1}$, $ y=\alertNoH{1}{y_2}$ is:

\hfil \hfil$
y-\alertNoH{1}{y_2}=x-\alertNoH{1}{x_1}.
$

This is so because both sides become zero when $x=\alertNoH{1}{x_1}$, $y=\alertNoH{1}{y_1}$.
\item If we plug in $x=\alertNoH{1}{x_2}$ and $y=\alertNoH{1}{y_2}$ in the above we don't get an identity (necessarily), but that can be easily fixed:

\hfil \hfil $
(\alertNoH{1}{x_2}-\alertNoH{1}{x_1} )(\alertNoH{1}{y_2}- \alertNoH{1}{ y_1 } )= ( \alertNoH{1}{y_2}-\alertNoH{1}{y_1}) (\alertNoH{1}{x_2} - \alertNoH{1}{ x_1}) 
$
\item Perhaps the last modification caused $x=\alertNoH{1}{x_1}$, $y=\alertNoH{1}{y_1}$ to no longer be solutions? No - both sides are still zero when $x=\alertNoH{1}{x_1}$, $y=\alertNoH{1}{y_1}$.
\end{itemize}
\uncover<2>{}
\end{example}
\end{frame}