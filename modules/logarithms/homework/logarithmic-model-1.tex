The Richter magnitude $M_L$ of an earthquake is determined from the logarithm of the amplitude $A$ of waves recorded by seismographs (with adjustment to compensate for the distance between the measuring station and the estimated epicenter of the earthquake). The formula is

\hfil \hfil$
M_L=\log_{10}A -J_0(\delta),
$

\noindent where $J_0(\delta)$ depends on the distance $\delta$ from the epicenter. Compare the amplitudes $A_1$ and $A_2$ of the seismographic waves of two hypothetical earthquakes of magnitudes $5$ and $7.2$ with the same epicenter.

\answer{the stronger earthquake has seism. amplitude about $10^{2.2}\approx 158.5$ times larger}

