Find the maximum and minimum values of $f$ on the given interval and the values of $x$ for which they are attained.
\begin{multicols}{2}
\begin{enumerate}[ref={\fcProblemRef}]
\item $\displaystyle f(x)=9+3x-x^2$, $x\in [0,4]$.

\answer{$f_{max}=f\left(\frac{3}{2}\right)=\frac{45}{4} $, $f_{min}=f\left(4\right)=5$}
\item $\displaystyle f(x)=5+4x-2x^3$, $x\in[-1,1] $.

\answer{$f_{max}=f\left(\frac{\sqrt{6}}{3} \right)= \frac{8}{9} \sqrt{6} +5  $, $f_{min} =f\left(-\frac{ \sqrt{6} }{3} \right)=5 - \frac{8 }{9}\sqrt{6} $}

\item $\displaystyle f(x)=2x^3-x^2-20x+1$, $x\in [-4,3]$.

\answer{$f_{max}=f\left( -\frac{5}{3}\right)=\frac{602}{27}$, $f_{min}=f\left(-4\right)=-63$}

\item $\displaystyle f(x)=3x^4-4x^3-12x^2+1$, $x\in [-2, 3]$.

\answer{$f_{max}=f\left(-2\right)=33$, $f_{min}=f\left(2\right)=-31 $}

\item \label{problemmaxminx^3-x^2-x+1over[-1,1]}
$f(x)=x^3-x^2-x+1,$  $x\in [-1,1]$.

\psset{xunit=2cm, yunit=2cm}
\begin{pspicture}(-1.5,-0.5)(1.5,1.5)
\tiny
\fcAxesStandard{-1.5}{-0.5}{1.5}{1.5}
\psplot[linecolor=red]{-1}{1}{x x x mul mul x x -1 mul mul -1 x mul 1 add add add}
\fcXTickWithLabel{1}{$1$}
\fcYTickWithLabel{1}{$1$}
\end{pspicture}
\item \label{problemmaxminx^3-x+1over[-2,1]}
$f(x)=x^3-x+1$,  $x\in[-2,1]$.

\answer{$f_{max}=f\left(-\frac{\sqrt{3}}{3}\right)= \frac{2}{9}\sqrt{3}+1, \quad f_{min}=f\left(-2\right)= -5 $}
\item $\displaystyle f(x)=(x^2-1)^3$, $x\in [-1, 2]$.

\answer{$f_{max}=f\left(2\right)=27$, $f_{min}=f\left(0\right)=-1$}
\item $\displaystyle f(x)=x+\frac{1}{x}$, $x\in [0.2,4 ]$.

\answer{$f_{max}=f\left(0.2\right)=\frac{26}{5}=5.2$, $f_{min}=f\left(1\right)=2$}
\item $\displaystyle f(x)=\frac{x}{x^2-x+1}$, $x\in [0,3 ]$.

\answer{$f_{max}=f\left(1\right)=1$, $f_{min}=f\left(-1\right)=-\frac{1}{3}$}
\item $\displaystyle f(t)=t\sqrt{4-t^2}$, $x\in [-1,2 ]$.

\answer{$f_{max}=f\left(\sqrt{2}\right)=2$, $f_{min}=f\left(-\sqrt{2}\right)=-2$}
\item $\displaystyle f(t)=\sqrt[3]{t}(8-t) $, $x\in [0,8 ]$.

\answer{$f_{max}=f\left(2\right)=6\sqrt[3]{2}$, $f_{min}=f\left(0\right)=f(8)=0$}
\item $\displaystyle f(t)=2\cos t+\sin (2t)$, $x\in [0,\frac{\pi}{2} ]$.

\answer{$f_{max}=f\left(\frac{\pi}{6}\right)=\frac{3}{2}\sqrt{3}$, $f_{min}=f\left(\frac{\pi}{2}\right)=0$}
\item $\displaystyle f(t)=t+\cot \left(\frac{t}{2}\right) $, $x\in [\frac{\pi}{4},\frac{7\pi}{4} ]$.

\answer{$f_{max}=f\left(\frac{3\pi}{2}\right)=\frac{3\pi}{2}-1$, $f_{min}=f\left(\frac{\pi}{2}\right)=\frac{\pi}{2}+1$}

\item $\displaystyle f(t)=t+\cot \left(\frac{t}{2}\right) $, $x\in [\frac{\pi}{4},\frac{7\pi}{4} ]$.
\item $\displaystyle f(x)=x e^{3 x}$, $x\in \left[-3, \frac{1}{6}\right]$.

\answer{$f_{max}=f\left( \frac{1}{6}\right)=\frac{e^{\frac{1}{2}}}{6}\approx 0.274787 $, $f_{min}=f\left( -\frac{1}{3}\right)= -\frac{1}{3e}\approx -0.122626 $}
\item $\displaystyle f(x)=\left(x-2\right) \left(x+1\right) e^{x} $, $x\in \left[-5,2\right]$.

\answer{$\begin{array}{l}
f_{max}=f\left(-\frac{\sqrt{13}}{2}-\frac{1}{2}
\right)= \left(\sqrt{13}+2\right) e^{\left(-\frac{\sqrt{13}}{2}-\frac{1}{2}\right)}\approx 0.560448\\
f_{min}=f\left(\frac{\sqrt{13}}{2}-\frac{1}{2} \right)= \left(-\sqrt{13}+2\right) e^{\left(\frac{\sqrt{13}}{2}-\frac{1}{2}\right)} \approx -5.907619
\end{array}
$}
\item $\displaystyle f(x)=$, $x\in \left[-3,3\right]$.

\answer{$\begin{array}{rcl}
f_{max}&=&f\left( \frac{\sqrt{3}}{2}-\frac{1}{2}
\right) =\left(\frac{\sqrt{3}}{2}+\frac{1}{2}\right) e^{\frac{\sqrt{3}}{2}-1}\approx 1.194743 \\
f_{min}&=&f\left(-\frac{\sqrt{3}}{2}-\frac{1}{2} \right)=\left(-\frac{\sqrt{3}}{2}+\frac{1}{2}\right) e^{-\frac{\sqrt{3}}{2}-1}\approx -0.056638 \end{array}$}
\item \label{problemmaxminxe^(2x)over[-2,1/2]}
$f(x)=x e^{2x}$, $x\in\left[ -2,\frac{1}{2}\right]$.

\answer{$ f_{max}=f\left(\frac{1}{2}\right)= \frac{e}{2}$, $f_{min}=f\left(-\frac{1}{2}\right)=-\frac{1}{2e}$}
\end{enumerate}
\end{multicols}
