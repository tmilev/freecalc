\begin{frame}
\begin{example}
\[\lim_{(x,y) \to (0,0)} \frac{x^2y}{x^2+y^2}
\]
\begin{itemize}
\item<2-> $f$ is not defined at $P_0(0,0)$;
\item<3-> Even if it were, the actual value might be different from the limit.
\end{itemize}
\uncover<4->{Recall polar coordinates: \alert<9,10>{ $x=r\cos\theta$, $y =r \sin\theta$}. \alert<8>{ We have that \alert<5,6>{$(x,y) \to (0,0)$ if an only if $\uncover<-5>{\textbf{?}} \uncover<6->{r\to 0}$  \alert<9,10>{ in polar coordinates}.}}}
\[
\begin{array}{rcl}
\uncover<7->{ \displaystyle \lim\limits_{\alert<8>{(x,y) \to (0,0)}} \alert<9,10>{ \frac{x^2y}{x^2+y^2} } &\alert<9,10>{=}& \displaystyle  \lim\limits_{\alert<8>{r\to 0}}  \uncover<-9>{\alert<9>{\textbf{?}}} \uncover<10->{\alert<10>{\frac{\alert<11>{ r^2}\cos^2\theta r\sin\theta}{\alert<11>{r^2}}}}} \\
\uncover<11->{&\alert<0>{=}&\displaystyle \lim\limits_{r\to 0}\phantom{\textbf{?}} \alert<12>{r\cos^2{\theta}\sin\theta}}\\
\uncover<15->{&\alert<0>{=}&0}
\end{array}	
\]
\uncover<15->{For the last equality, we use the squeeze theorem:} 
\uncover<12->{$\uncover<14->{0=} \uncover<13->{\lim\limits_{r\to 0} } -r \leq \uncover<13->{ \lim \limits_{r \to 0}} \alert<12>{ r\cos^2{\theta}\sin\theta } \leq \uncover<13->{\lim\limits_{r\to 0}}  r \uncover<14->{=0.}$}
\end{example}
\end{frame}