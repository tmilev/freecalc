\begin{frame}\frametitle{Describing multivariable functions}
\begin{itemize}
\item Several ways to define a function of several variables. 
\item Usually: start with \emph{verbal} description, then give specific meanings
to our input and output variables. 
\item We explain by examples.
\end{itemize}
\end{frame}

\begin{frame}\frametitle{Verbal description examples}
\begin{itemize}
  \item The apparent temperature, $W$, felt on exposed skin
  depends on several factors, including the actual
  temperature, $T$, the wind speed, $v$, and the humidity.
  The \emph{wind chill temperature} is a mathematical model
  for $W$ under the assumption that the humidity is 0 and
  that the only factors influencing $W$ are $T$ and $v$:
  %
  $$W = W(T,v)\; .$$
  %
  The domain of the function $W$ consists of all reasonable
  pairs $(T,v)$.

  \item The Cobb-Douglas production function models the
  production output, $P$, under the assumption that the
  only factors are the amount of labor, $L$, and the
  amount of capital, $K$:
  %
  $$P=P(L,K) \; .$$
\end{itemize}

\vskip 8cm %to serve for spacing
\end{frame}
\begin{frame}\frametitle{Verbal description examples}
\begin{itemize}
%  \item The magnitude $G$ of the attraction force between
%  two bodies depends on several factors, including the
%  masses $m$ and $M$ of the bodies and the distance $d$
%  between them:
  %
%  $$G=G(m,M,d)$$

  \item A set $(\rho, \phi, \theta)$ of spherical
  coordinates determines the rectangular coordinates
  $(x,y,z)$ of a point. In this case, both the input
  and the output are multidimensional:
  %
  $$(x,y,z) = \textbf{F}(\rho, \theta, \phi)\; ,$$
  %

  \item The wind velocity $\textbf{v}$ at a point $P$
  depends on the position $\textbf{r}$ of $P$,
  %
  $$\textbf{v} = \textbf{V} (\textbf{r})\; .$$
  %
  In this case both the input and the output are
  vectorial quantities.

  \item The electric force on a charge $q$ displaced
  by $\textbf{r}$ from a charge $Q$ depends on the
  two charges, the displacement, and the medium in
  which the charges are placed:
  %
  $$\textbf{E} =
  \textbf{E}(q, Q,\textbf{r})\; .$$
  %
  Note that in this case the output data
  is a vector and the input data is a mix of scalar
  and vectorial quantities.
\end{itemize}

\vskip 8cm %to serve for spacing

\end{frame}