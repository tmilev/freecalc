\begin{frame}
\begin{example}
\begin{columns}
\column{0.35\textwidth}
\begin{pspicture}(-2,-2)(2,2)
\fcBoundingBox{-0.3}{-0.5}{1.7}{1.1}
\renewcommand{\fcScreen}{[-2 1 -1.2] 0}%
\tiny%
\fcStartIIIdScene%
\fcPatchInScene{[0 0 0]}{[1 0 0]}{[0 0 1]}%
\fcPatchInScene{[0 0 0]}{[0 0 1]}{[0 1 0]}%
\fcPatchInScene{[0 0 0]}{[0 1 0]}{[1 0 0]}%
\fcPatchInScene{[1 1 0]}{[1 1 1]}{[1 0 0]}%
\fcPatchInScene{[1 1 0]}{[0 1 0]}{[1 1 1]}%
\fcFinishIIIdScene[fastsort=true]%
\uncover<2>{%
\fcLineIIId[linecolor=blue, linewidth=2pt]{[0 0 0]}{[1 0 0]}%
\fcLineIIId[linecolor=blue, linewidth=2pt]{[0 0 0]}{[0 1 0]}%
\fcLineIIId[linecolor=blue, linewidth=2pt]{[0 0 0]}{[0 0 1]}%
}%
\fcPutIIId[t]{[0.5 0 -0.1]}{$\uncover<2->{\alert<2>{x}}$}%
\fcPutIIId[b]{[0 0.3 0]}{$\uncover<2->{\alert<2>{y}}$}%
\fcPutIIId[r]{[0 0 0.5]}{$\uncover<2->{\alert<2>{z}}$}%
\end{pspicture}
\column{0.65\textwidth}
Find the maximal volume of a box with no lid whose surface area is $10m^2$.

\uncover<2->{ Let the three dimensions of the box be $\alert<2>{ x,y,z}$.}
\end{columns}
\only<1-15>{%
\uncover<3->{%
We seek to maximize $\alert<7,8>{V=xyz}$ \alert<4,5>{under the restriction}
\[
\fcAnswer{5}{
\alert<14>{\alert<9,10>{ g(x,y,z)=xy+2(zx+yz)-10}=0} \quad .}
\]
}
\uncover<6->{%
By the Lagrange multiplier method, we need to solve the system
\[
\fcAnswer{8}{( \alert<11>{yz}, \alert<12>{zx}, \alert<13>{xy} ) } \uncover<7->{\alert<7,8>{=}} \alert<7,8>{  \nabla V}= \lambda \alert<9,10>{ \nabla g} \uncover<9->{ \alert<9,10>{ =} \alert<11-13>{\lambda}}  \fcAnswer{10}{ (\alert<11>{ 2 z+y},\alert<12>{ 2 z+x}, \alert<13>{2 y+2 x} )\quad .}
\]
}
\uncover<11->{In other words we are solving the following system.}
}%only<1-15>
\uncover<11->{
\[
\alert<15,16>{
\left| \begin{array}{rcl}
\alert<11,17>{y z} &\alert<11,17>{=}& \alert<11,17>{\lambda(2z+y)}\\
\alert<12,17,26>{xz} &\alert<12,17,26>{=}& \alert<12,17,26>{\lambda(2z+x)}\\
\alert<13>{x y} &\alert<13>{=}& \alert<13>{\lambda(2x+2y)} \\
\alert<14,30>{10}&\alert<14,30>{=}&\alert<14,30>{x y+2(z x+y z)}
\end{array}
\right.
}
\uncover<17->{
\Rightarrow
\left| \begin{array}{rcl}
\alert<19>{\alert<18>{x} \alert<17>{y z}} &\alert<17>{=}& \alert<20>{\alert<17>{\lambda(2z+y )} \alert<18>{x}}\\
\alert<19>{\alert<17>{x} \alert<18>{ y} \alert<17>{z}} &\alert<17>{=}& \alert<20>{ \alert<17>{\lambda(2z+x )}\alert<18>{y}}\\
\alert<24>{x y}  &\alert<24>{=}& \alert<24>{ \lambda(2x+2y)}\\
10&=&x y+2(z x+y z)
\end{array}\right. \quad .
}
\]
}
\only<19-22>{%
\uncover<19->{From the first two equalities we get $\alert<20>{2\lambda x z+\alert<21>{\lambda x y}= 2\lambda y z +\alert<21>{\lambda xy}} $}
\uncover<21->{%
and so
\[
\alert<22>{\lambda} x \alert<22>{z} = \alert<22>{\lambda} y\alert<22>{ z} \quad .
\]
}%
\uncover<22->{We have that $\lambda\neq 0, z\neq 0 $ (else the volume would be zero). Therefore }
}
\uncover<22->{%
\[
\alert<22,23,30,33>{x=y}\quad\quad \quad \quad \uncover<29->{ \alert<29,30,33>{ z=\frac{x}{2}} \quad .}
\]
}%only<19-22>

\only<24-28>{
\uncover<24->{We substitute $y=x$ in the third equality  to get $\alert<24>{\alert<25>{ x^2} = 4\lambda\alert<25>{ x}}$} \uncover<25->{and since $x\neq 0$ we get $\alert<25,26>{\lambda = \frac{x}{4}}$.} \uncover<26->{ We substitute $\alert<26>{ \lambda = \frac{x}{4}}$ in the original second equality to get $\alert<26>{z\alert<27>{ x}= \frac{\alert<27>{x} }{4}(2z+x)}$.} \uncover<27->{Since $x\neq 0$ it follows that $\alert<27>{z=\frac{1}{4}(2z+x)} $ \uncover<28->{and so $\alert<28,29>{z=\frac{x}{2}}$.}}
} %only<24-28>
\uncover<30->{%
Finally we substitute $y=x, z=\frac{x}{2}$ in the last equality \uncover<31->{to get $\alert<31,32>{10 = 3 x^2}$.}
}%
\uncover<32->{%
Thus  $\alert<32,34>{x = \frac{\sqrt{30}}{3}}$ \uncover<33->{and therefore \alert<33,34>{$y = \frac{\sqrt{30}}{3}$,  $z = \frac{\sqrt{30}}{6}$,} \uncover<34->{\alert<34>{our final answer.}}}}
\end{example}
\vskip 10cm
\end{frame}
