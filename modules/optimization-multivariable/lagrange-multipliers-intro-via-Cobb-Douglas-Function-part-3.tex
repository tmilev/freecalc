\begin{frame}
  \frametitle{Geometric Method}

  At the point $(K_0,L_0)$ of maximal production:\pause

\begin{itemize}
  \item Level curves $P(K,L) = P(K_0,L_0)$ and $C(K,L) = C(K_0,L_0)$ are tangent;\pause
  \item Gradients of $P$ and $C$ at $(K_0,L_0)$ are  collinear.
\end{itemize}

\pause There exists a scalar $\lambda_0$ such that
%
\begin{align*}
  (\nabla P)_{(K_0,L_0)} = & \lambda_0 (\nabla C)_{(K_0,L_0)} \pause \Longrightarrow
\left\{ \begin{array}{ll}
  P_K(K_0,L_0) = & \lambda_0 C_K(K_0,L_0) \\
  %
  P_L(K_0,L_0) = & \lambda_0 C_L(K_0,L_0)
\end{array}
\right. \\
\Longrightarrow & \left\{ \begin{array}{ll}
  P_K(K_0,L_0) = & 3 \lambda_0 \\
  %
  P_L(K_0,L_0) = & 2 \lambda_0
\end{array}
\right. \; \Longrightarrow \frac{P_K(K_0,L_0)}{3}=\frac{P_L(K_0,L_0)}{2}\; .
\end{align*}
%
\end{frame}

\begin{frame}
  The triple $(K_0,L_0,\lambda_0)$ is a solution of the system
%
$$\left\{ \begin{array}{ll}
  P_K(K,L) = & 3\lambda\\
  %
  P_L(K,L) = & 2\lambda  \\
  %
  C(K,L) = & 16
\end{array} \right.
%
\Longleftrightarrow
%
\left\{ \begin{array}{ll}
  \frac{3}{4} L^{1/4}K^{-1/4} = & 3\lambda\\
  %
  & \\
  %
  \frac{1}{4} L^{-3/4}K^{3/4} = & 2\lambda  \\
  %
  & \\
  %
  3K+2L = & 16
\end{array} \right.
$$

\pause
The system is not linear; we've not studied any methods for solving such systems. \pause

In this particular case:
%
$$\frac{1}{4} L^{1/4}K^{-1/4} = \lambda = \frac{1}{8} L^{-3/4}K^{3/4} \Longrightarrow \frac{K}{L}=2 \Longrightarrow K = 2L\; .$$
%
From $K=2L$ and $3K+2L=16$ we get \pause $L=2$ and $K=4$.\pause

How do we know that $K=4$ and $L=2$ corresponds to a maximum? \pause

Geometric argument using the direction of the gradient $(\nabla P)(4,2)$.

\end{frame}