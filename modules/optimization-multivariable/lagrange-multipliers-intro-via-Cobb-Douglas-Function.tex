 \begin{frame}
  \frametitle{Optimization with Constraints: Motivating Example}

  With $K$ units of capital and $L$ units of labor, the production is given by the following Cobb-Douglas function:
%
$$P(K,L) = L^{1/4} K^{3/4}\; .$$
%
\pause With unlimited resources $(K,L)$ \pause any level of production can be achieved. \pause

\underline{Constraint}: \pause Capital and labor are not free, and the budget is limited.\pause

Suppose
\begin{itemize}
  \item one  unit of $L$ costs 2 units of money;
  \item and one unit of $K$ costs 3 units of money.
\end{itemize}

\pause Cost function:\pause
%
$$C(K,L) = 3K +2L\; .$$
%
\pause \underline{Question}: What is the maximal production that can be achieved with a budget of 16 units of money?
\end{frame}