\begin{frame}
  \frametitle{Minima, Maxima}

  Function $f \colon D \to \mathbb{R}$ defined on a region $D$ in $\mathbb{R}^2$, we want to know:
%
\begin{itemize}
  \item The largest and the smallest values of $f$ attained on $D$, if any;
  %
  \item The points where these extreme values are attained.
\end{itemize}

\pause
A point $P_0$ in $D$ is a point of:
\begin{itemize}
  \item absolute maximum, if $f(P) \leqslant f(P_0)$ for all $P$ in $D$;
  \item absolute minimum, if $f(P) \geqslant f(P_0)$ for all $P$ in $D$.
\end{itemize}

\pause
These notions are relative to the domain $D$. A point $P_0$ that is not an extreme point might become one if we focus only around that point.

\pause
A point $P_0$ in $D$ is a point of:
\begin{itemize}
  \item local maximum, if there exists an open disk $B=B_r(P_0)$ centered at $P_0$ such that $f(P) \leqslant f(P_0)$ for all $P$ in $B \cap D$;
  \item local minimum, if there exists an open disk $B=B_r(P_0)$ centered at $P_0$ such that $f(P) \geqslant f(P_0)$ for all $P$ in $B \cap D$.
\end{itemize}

\pause
How do we find points of extreme?
\end{frame}