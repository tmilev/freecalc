% begin module optimization-intro
\begin{frame}
\frametitle{One variable optimization Problems}
\begin{problem}[One variable optimization problem statement]
\alert<3>{Given a function $\alert<8>{f(\alert<6>{x})}$} find the maximum and/or the minimum of $\alert<8>{f(\alert<6>{x})}$, and the values of $\alert<6>{x}$ for which the minima/maxima are achieved.
\end{problem}
\uncover<2->{The method for finding extreme values solves ``the one variable optimization problem'', which often arises in practice.} \uncover<3->{Optimization problems are usually not formulated directly in \alert<3>{the above form}.} \uncover<4->{Solving an optimization problem involves the following steps.
\begin{enumerate}
\item<5->  Draw a picture of the problem. Assign variable names to the involved quantities.
\item<6->  Express all involved quantities in terms of \alert<6>{only one of them}. \uncover<7->{If you cannot do that then the problem is not in one variable (i.e., lies outside of the scope of Calculus I).}
\item<8->  Determine \alert<8>{which quantity} are we seeking to maximize.
\item<9->  Use calculus to find the maximum (or minimum) value of the desired quantity.
}
\end{enumerate}
\end{frame}
% end module optimization-intro
