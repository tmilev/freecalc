\begin{frame}
\frametitle{Partial Derivatives}
\begin{itemize} 
\item Let $f\colon D \to \RR$,  $P_0(x_0,y_0)$ inside $D$.
\item<2-> Consider the line $\fcv{r}(t) = \fcv{r}_0 + t\fcv {i} = \langle x_0+t, y_0 \rangle$.
\item<3-> Set $g(t) = f(\fcv{r}(t)) = f(x_0+t, y_0)$.
\item<4-> Then $(D_{\fcv{i}} f)(x_0,y_0) = \lim\limits_{t\to 0} \frac{g(t)-g(0)}{t}$
\begin{definition}
The partial derivatives $\frac{\partial}{\partial x} $, $\frac{\partial}{\partial y} $  of $f$ are defined as the directional derivatives of $f$ in the direction of the unit vector along the $x$, $y$ axes. 
\end{definition}
\item<5-> Notations for \alert<1->{partial derivatives}:
$$(D_{\,\fcv{i}}f)(x_0,y_0) = \frac{\partial f}{\partial x}(x_0,y_0) =
f_x(x_0,y_0)$$
$$(D_{\,\fcv{j}}f)(x_0,y_0) =  \frac{\partial f}{\partial y}(x_0,y_0) =
f_y(x_0,y_0)$$
%
\item<6-> Partial derivatives are computed by
\begin{itemize}
\item considering all other variables as constants and
\item applying the rules for differentiation for single variable functions.
\end{itemize}
\end{itemize}
\end{frame}