%begin module partial-fractions-building-blocks-2a-and-3a
%This module may be too long, perhaps a split is needed.
%\begin{comment}
\begin{frame}
\frametitle{Building blocks IIa and IIIa}
Building block IIa: $\int \frac{x}{1+x^2}\diff x.$

Building block IIIa: $\int \frac{1}{1+x^2 }\diff x.$
\begin{example}
Integrate 
\[
\begin{array}{rcl}
\displaystyle \int \frac{\alert<2>{x}}{1+x^2 }\alert<2>{ \diff x} \uncover<2->{&=&\displaystyle \int \frac{1}{(1+x^2) }\alert<2>{ \frac{\alert<3>{\diff (x^2)}}{2}}} \uncover<3->{=\alert<4>{ \int \frac{1}{1+x^2 }} \frac{\alert<3,4>{\diff (1+x^2)}}{2}} \\
\uncover<4->{&=&\displaystyle\frac12\alert<4>{ \ln (1+x^2)}+C\quad .}
\end{array}
\]
\end{example}
\uncover<5->{
\begin{example}
Integrate 
\[
\displaystyle \int \frac{1}{1+x^2 }\diff x \uncover<6->{=\Arctan x +C}
\]
\end{example}
}
\vspace{2cm} 
\end{frame}

\begin{frame}
\frametitle{Linear substitutions leading to blocks IIa and IIIa}
Building block IIa: $ \int \frac{x}{1+x^2}\diff x = \frac{1}{2}\ln(1+x^2)+C$.

\uncover<2->{ Integrals leading to block IIa can be done faster.} \uncover<3->{ We show the ``theoretical way'', then redo the ``fast way''.} \uncover<4->{ \alert<4>{\textbf{Feel free to skip the slide.}}}
\begin{example}
\[
\renewcommand{\arraystretch}{0}
\begin{array}{r{c}ll|l}
 \int \frac{ x }{2x^2+3} 
\diff x \uncover<5->{
&=& \int\frac{x}{3\left(\frac{2}{3} x^2+1\right)} \diff x = \int\frac{x}{3\left(\left(\sqrt{\frac{2}{3}} x\right)^2+1\right)} \diff x\\
&=& \frac{3}{2}\int\frac{\sqrt{\frac{2}{3}} x}{3\left(\left(\sqrt{\frac{2}{3}} x\right)^2+1\right)} \diff \left( \sqrt{\frac{2}{3}}x \right)  && \text{Set } u=\sqrt{\frac{2}{3}}x\\
&=&\frac{1}{2}\int \frac{u}{u^2+1}\diff u= \frac{1}{4} \ln (1+u^2)+C\\
&=& \frac{1}{4} \ln \left(\frac{1}{3} (2x^2+3)\right) +C\\
&=&\frac{1}{4} \ln (2x^2+3) + \frac{\ln \left(\frac{1}{3}\right)}{4}+C\\
&=&\frac{1}{4} \ln (2x^2+3)+K
}
\end{array}
\]

\end{example}



\vspace{4cm}

\end{frame}


\begin{frame}
\frametitle{Linear substitutions leading to blocks IIa and IIIa}
Building block IIa: $ \int \frac{x}{1+x^2}\diff x = \frac{1}{2}\ln(1+x^2)+C$.

In practice, integrals leading to block IIa can be done directly, without transforming to the above form. We illustrate how.
%Building block IIIa: $ \int \frac{1}{1+x^2 }\diff x=\Arctan x+C.$


\begin{example}
\[
\begin{array}{rcll|l}
\displaystyle \int \frac{\alert<2>{ x} }{2x^2+3} \alert<2>{ \diff x} \uncover<2->{&=&\displaystyle \int\frac{1}{2x^2+3} \alert<2,3>{ \diff \left( \frac{x^2}{2} \right)}} \\
\uncover<3->{&=&\displaystyle \int\frac{1}{\alert<5>{2x^2+3}} \diff \left(\frac{\alert<5>{\alert<3>{ 2x^2} \uncover<4->{\alert<4>{+3}}} }{ \alert<3>{4}}\right)} \uncover<5->{&&\text{Set } \alert<5,7>{ u=2x^2+3} } \\
\uncover<5->{&=&\displaystyle \frac{1}{4}\alert<6>{ \int \frac{1}{\alert<5>{u}}\diff \alert<5>{u}}}\\
\uncover<6->{&=&\displaystyle \frac{1}{4}\alert<6>{ \ln \alert<7>{|u|}}+C}\\
\uncover<7->{&=&\displaystyle \frac{1}{4}\ln (\alert<7>{2x^2+3})+C}
\end{array}
\]

\end{example}
\vspace{4cm}

\end{frame}
%\end{comment}

\begin{frame}
\frametitle{Linear substitutions leading to blocks IIa and IIIa}
%Building block IIa: $ \int \frac{x}{1+x^2}\diff x = \frac{1}{2}\ln(1+x^2)+C$.

Building block IIIa: $ \int \frac{1}{1+x^2 }\diff x=\Arctan x+C.$
\begin{example}
\[
\begin{array}{rcll|l}
\displaystyle \alert<8>{\int \frac{1}{\alert<2>{x^2+2}}\diff x} \uncover<2->{ &=&\displaystyle \int\frac{1}{\alert<2>{ 2} \alert<2>{\left(  \alert<3>{\frac{1}{2}x^2} +1\right)}} \alert<4>{\diff x} }\\
\uncover<3->{&=&\displaystyle \int \frac{1}{2\left( \alert<3>{\left(\alert<5>{ \frac{x}{ \sqrt{2}}} \right)^2} +1  \right)} \alert<4>{\sqrt{2}\diff\left(\alert<5>{ \frac{ x}{\sqrt{2}}}\right)}} \uncover<5->{ &&\text{ Set } \alert<5,7>{u= \frac{x}{\sqrt{2}}} }\\
\uncover<5->{&=&\displaystyle \frac{1}{\sqrt{2}}\int \frac{1}{1+{\alert<5>{u}}^2}\diff \alert<5>{u}}\\
\uncover<6->{&=& \frac{1}{\sqrt{2}}\Arctan (\alert<7>{u})+C} \\
\uncover<7->{&=&\displaystyle \alert<8>{ \frac{1}{\sqrt{2}} \Arctan\left(\alert<7>{\frac{x}{\sqrt{2}}}\right)+C}}
\end{array}
\]

\end{example}
\vspace{2cm}

\end{frame}
\begin{frame}
\frametitle{Linear substitutions leading to blocks IIa and IIIa}
%Building block IIa: $ \int \frac{x}{1+x^2}\diff x = \frac{1}{2}\ln(1+x^2)+C$.

Building block IIIa: $ \int \frac{1}{1+x^2 }\diff x=\Arctan x+C$. \alert<1>{\textbf{Let $a>0$.}}
\begin{example}
\[
\begin{array}{rcll|l}
\displaystyle \alert<1>{\int \frac{1}{x^2+a\vphantom{2}}\diff x } &=&\displaystyle \int\frac{1}{ a\left(\frac{1}{a\vphantom{2}}x^2 +1\right)} \diff x \\
\uncover<1->{&=&\displaystyle \int \frac{1}{a\left(\alert<-1>{ \left(\frac{x}{ \sqrt{a\vphantom{2}}} \right)^2} +1  \right)} \sqrt{a\vphantom{2}}\diff\left( \frac{ x}{\sqrt{a\vphantom{2}}}\right)} \uncover<1->{&& \text{ Set } u= \frac{x}{\sqrt{a}}} \\ %the uncover commands are to preserve latex spacing. This may be a LaTeX bug, but please keep the two uncover<1-> commands where they are!
&=&\displaystyle \frac{1}{\sqrt{a\vphantom{2}}} \int \frac{1}{1 +{u}^2}\diff u \\
&=& \frac{1}{\sqrt{a\vphantom{2}}}\Arctan (u)+C \\
&=&\alert<1>{ \displaystyle \frac{1}{\sqrt{a\vphantom{2}}} \Arctan\left(\frac{x}{\sqrt{a}}\right)+C}
\end{array}
\]

\end{example}
\vspace{2cm}

\end{frame}

%\begin{comment}
\begin{frame}
\frametitle{Linear substitutions leading to blocks IIa and IIIa}
Building block IIa: \alert<4>{$ \int \frac{x}{1+x^2}\diff x = \frac{1}{2}\ln(1+x^2)+C$}.

Building block IIIa: \alert<4>{$\int \frac{1}{1+x^2 }\diff x=\Arctan x+C$}.

\begin{itemize}

\item<1-> Let $ax^2+bx+c$ have no real roots.
\item<2-> We can find $p,q$ so that the linear substitution $u=px+q$ transforms the quadratic to:
\[
ax^2+bx+c= r(u^2+1)
\] 
(where $r$ is some number to be determined).
\item<3-> To find $p,q$, we \alert<3>{complete the square}. 
\item<4-> In this way, integrals of the form \alert<4>{$\displaystyle \int \frac{Ax+B}{ax^2+bx+c} \diff x$} are transformed to \alert<4>{combinations of building blocks IIa and IIIa}.

\item<5-> We show examples; the general case is analogous and we leave it to the student.
\end{itemize}
\vspace{5cm}
\end{frame}


\begin{frame}
\frametitle{Linear substitutions leading to blocks IIa and IIIa}
Building block IIa: $ \int \frac{x}{1+x^2}\diff x = \frac{1}{2}\ln(1+x^2)+C$.

Building block IIIa: $\int \frac{1}{1+x^2 }\diff x=\Arctan x+C.$


\begin{example}
\uncover<2->{\alert<2-4>{No real roots $\Rightarrow$ complete the square.}} \uncover<7->{Let \alert<7,27,30>{$u= x+ \frac{1}{2} $}}\uncover<16->{, let \alert<16,28>{$z=\frac{2u}{\sqrt{3}}$}.}  
\[
\begin{array}{rcl}
\displaystyle
%
\vphantom{\int \frac{u}{u^2+\frac{3}{4}}\diff u}
%
\int\frac{x}{\alert<2>{x^2+\alert<3>{x}+1}}\diff x 
\only<1>{{~~~~~~~~~~~~~~~~~~~~~~~~~~~~~~~~~~~~~~~~~~~~~~~~~~~~~~} {~~~~~~~~~~~~~~~~~~~~~~~~~~~~~~~~~~~~~~~~~~~~~~~~~~~~~~} }
\only<1-10>{\uncover<2->{&=& \displaystyle \int \frac{x}{\alert<2>{ \alert<4>{ x^2+\alert<3>{2\frac{1}{2}x} +\frac{1}{4} } \alert<5>{-\frac{1}{4} +1}} } \alert<6>{\diff x} } {~~~~~~~~~~~~~~~~~~~~~~~~~~~~~~~~~~~~~~~~~~~~~~~~~~~~~~} \\
\uncover<4->{&=&\displaystyle \int \frac{\alert<7>{x+\frac{1}{2}}-\frac{1}{2}}{ \alert<4>{\left(\alert<7>{x+\frac{1}2}\right)^2}+\alert<5>{\frac{3}{4}} }\alert<6>{\diff \left(\alert<7>{x+\frac{1}{2}}\right) }}\\
\uncover<7->{&=&\displaystyle \int \frac{\alert<7,8>{u} \alert<9>{ -\frac{1}{2}} }{{\alert<7>{u}}^2+\frac{3}{4}}\diff \alert<7>{u}} \\
\uncover<8->{&=&\alert<10>{ \displaystyle \int \frac{\alert<8>{u}}{u^2+\frac{3}{4}}\diff u\alert<9>{-\frac{1}{2}}\int \frac{1}{u^2+\frac{3}{4}}\diff u}}
}

\only<11->{
&=&\alert<11>{\displaystyle\alert<25>{ \int \frac{u}{u^2+\frac{3}{4}}\diff u} -\frac{1}{2} \alert<12,18>{\int \frac{1}{u^2+\frac{3}{4}}\diff u}} {~~~~~~~~~~~~~~~~~~~~~~~~~~~~~~~~~~~~~~~~~~~~~~~~~~~~~~} \\
}
\only<11-18>{
\only<12->{\displaystyle \alert<12,18>{ \int \frac{1}{\alert<13>{u^2+\frac{3}{4}} }\diff u}\uncover<13->{&=& \displaystyle \int \frac{1}{\alert<13>{\frac{3}{4}\left( \alert<14>{\frac{4}{3}u^2} +1\right)}}\alert<15>{\diff u}}}\\
\uncover<14->{&=&\displaystyle \int \frac{1}{\frac{3}{4}\left(\alert<14>{ \left(\alert<16>{ \frac{2u}{ \sqrt{3}}}\right)^2} +1\right)} \alert<15>{ \frac{\sqrt{3} }{2} \diff \left(\alert<16>{ \frac{2u}{\sqrt{3}}}\right) }}\\
\uncover<16->{&=&\displaystyle \frac{2\sqrt{3}}{3}\int \frac{1}{\alert<16>{z}^2+1}\diff \alert<16>{z}}\uncover<17->{ = \alert<18>{\frac{2\sqrt{3}}{3} \Arctan z}+C}
}
\only<19->{
&=&\displaystyle \vphantom{\int \frac{u}{u^2+\frac{3}{4}}\diff u} \only<19-24>{\alert<20>{\int \frac{u}{u^2+\frac{3}{4}}\diff u}} \only<25->{\alert<25>{ \frac{1}{2}\ln \left(\alert<27>{u}^2+\frac{3}{4}\right) } }-\frac{1}{2}\alert<19>{ \frac{2\sqrt{3}}{3} \Arctan \alert<28>{z} }+C\\
\only<26->{
\uncover<27->{&=&\displaystyle \frac{1}{2}\ln  \left( \alert<29>{ {\alert<27>{\left(x+\frac{1}{2}\right)}}^2 + \frac{3}{4}}\right) - \frac{\sqrt{3}}{3} \Arctan \left( \alert<28>{ \frac{\alert<30>{2u }}{ \sqrt{3}}} \right)+C} \\
\uncover<29->{&=&\displaystyle \frac{1}{2}\ln \left(\alert<29>{x^2+x+1}\right) - \frac{\sqrt{3}}{3} \Arctan \left(\frac{\alert<30>{2x+1}}{\sqrt{3}}\right)+C
}
}
}
\only<20-25>{
\displaystyle \alert<20,25>{\int \frac{\alert<21>{ u} }{u^2+\frac{3}{4}}\alert<21>{ \diff u} } \uncover<21->{ &=& \displaystyle\int \frac{1}{u^2+\frac{3}{4}}\diff \alert<21>{\left(\frac{u^2}{\alert<22>{2}}\right)}}\\
\uncover<22->{&=&\displaystyle \alert<22>{\frac{1}{2}}\int \frac{1}{\alert<24>{u^2+\frac{3}{4}}}\diff \left(\alert<24>{ u^2\uncover<23->{ \alert<23>{ +\frac{3}{4}} }} \right)}\uncover<24->{ =\alert<25>{ \frac{1}{2}\ln \left(\alert<24>{u^2 +\frac{3}{4}} \right)}+C}
}
\end{array}
\]
\end{example}

\vspace{8cm}

\end{frame}

%\end{comment}
%end module partial-fractions-building-blocks-2a-and-3a