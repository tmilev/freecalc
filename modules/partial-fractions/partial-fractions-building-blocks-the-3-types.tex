%begin module partial-fractions-building-blocks-3-and-4-intro
\begin{frame}
\frametitle{The building blocks}
Let $n$ be a positive integer.
\begin{itemize}
\item (Building block I) The first building block integral is:

$\displaystyle \int \frac{1}{x^n }\diff x\quad .$
\item<2-> (Building block II) The second building block integral is:

$\displaystyle \int \frac{\alertNoH{4}{x}}{(\alertNoH{3}{1+x^2})^n }\alertNoH{4}{\diff x}.$ \uncover<3->{ (Note: $\alertNoH{3}{u=1+x^2}, \alertNoH{4}{x\diff x=\frac{1}{2}\diff u}$ transforms II to I).}
\item<5-> (Building block III) The third building block integral is:

$\displaystyle \int \frac{1}{(1+x^2)^n }\diff x\quad .$
\item<6-> The case $n=1$ is special for each of the building blocks:

$\displaystyle \int \frac{1}{x}\diff x$, $\displaystyle \int \frac{x}{1+x^2 }\diff x$ and $\displaystyle \int \frac{1}{1+x^2 }\diff x$.
\item<7-> The case $n=1$ we call respectively building block Ia, IIa and IIIa.
\uncover<8-> {The case $n>1$ we call respectively building block Ib, IIb and IIIb.} \uncover<9->{ This ``building block'' terminology serves our convenience, and is not a part of standard mathematical terminology. }
\end{itemize}

\end{frame}
%end module partial-fractions-building-blocks-3-and-4-intro
