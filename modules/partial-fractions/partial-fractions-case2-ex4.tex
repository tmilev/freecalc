% begin module partial-fractions-case2-ex4
\begin{frame}
\begin{example}[Example 4, p. 513]
Find $\int \frac{x^4-2x^2+4x+1}{x^3-x^2-x+1}\diff x$.
\begin{itemize}
\item<2->  Divide: $\frac{x^4-2x^2+4x+1}{x^3-x^2-x+1} = x + 1 + \frac{4x}{x^3-x^2-x+1}$.
\item<3->  Factor denominator: $x^3-x^2-x+1 = (x-1)^2(x+1)$.
\end{itemize}
\abovedisplayskip=0pt
\belowdisplayskip=0pt
\begin{eqnarray*}
\uncover<4->{%
\frac{4x}{(x-1)^2(x+1)}%
}%
& \uncover<4->{ = } & %
\uncover<4->{%
\frac{A}{x-1} + \frac{B}{(x-1)^2} + \frac{C}{x+1}%
}\\%
\uncover<5->{%
4x%
}%
& \uncover<5->{ = } & %
\uncover<5->{%
A(\alert<handout:0| 8>{x-1})(\alert<handout:0| 6>{x+1}) + B(\alert<handout:0| 6>{x+1}) + C(\alert<handout:0| 8>{x-1})^2%
}\\%
\end{eqnarray*}
\vspace{-.4in}
\begin{itemize}
\item<6->  Plug in $-1$: \uncover<7->{$4(-1) = C(-1-1)^2$, therefore $C = -1$.}
\item<8->  Plug in $1$: \uncover<9->{$4(1) = B(1+1)$ therefore $B = 2$.}
\item<10->  Plug in $0$: $4(0) = A(0-1)(0+1) + 2(0+1) + (-1)(0-1)^2$.
\item<11->  Therefore $A = 1$.
\end{itemize}
\abovedisplayskip=0pt
\belowdisplayskip=0pt
\[
\begin{array}{r@{ \ }c@{ \ }l}
\uncover<12->{%
\int \frac{x^4-2x^2+4x+1}{x^3-x^2-x+1}\diff x%
}%
& \uncover<12->{ = } & %
\uncover<12->{%
\int \left( x + 1 + \frac{1}{x-1} + \frac{2}{(x-1)^2} - \frac{1}{x+1}\right) \diff x%
}\\%
& \uncover<13->{ = } & %
\uncover<13->{%
\frac{x^2}{2} + x + \ln |x-1| - \frac{2}{x-1} -\ln |x+1| + K%
}%
\end{array}
\]
\end{example}
\end{frame}
% end module partial-fractions-case2-ex4
