% begin module partial-fractions-summaryProcess
\begin{frame}\frametitle{Summary of Process}
\begin{itemize}
\item If the degree in the numerator is greater than or equal to the degree in the denominator, you have an improper fraction.  Use long division to convert it to the sum of a polynomial and a ``proper" rational expression.

\item Factor the denominator.
\item Determine the partial fraction form (involving the unknowns A, B, C, ...) and set it  equal to the original expression (as on the previous two slides).
\item Multiply both sides of the equation by the original denominator to obtain the ``Basic Equation".
\item Solve the Basic Equation for the unknowns (see guidelines)
\end{itemize}
\end{frame}

\begin{frame}\frametitle{Guidelines for Finding Coefficients}
\begin{itemize}
\item[1] For distinct linear factors, substitute the roots of the distinct linear factors to determine the constants.
\item[2] For repeated linear factors, first substitute the roots found in (1) to determine some of the constants.  Then rewrite the Basic Equation and use other ``convenient" choices for $ x $ to solve for the remaining coefficients (or use the method of equating coefficients).
\item[3] For quadratic factors, expand the Basic Equation, collect terms according to powers of $ x $ and equate coefficients of like powers of $ x $.  This will give a system of linear equations to be solved. 
\end{itemize}

\end{frame}
% end module partial-fractions-summaryProcess
