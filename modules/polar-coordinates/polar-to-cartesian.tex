% begin module polar-to-cartesian
\begin{frame}
\begin{itemize}
\item  How do we go from polar coordinates to Cartesian coordinates?
\item<2->  Suppose a point has polar coordinates $(r, \theta )$ and Cartesian coordinates $(x,y)$.
\item<8->  How do we go from Cartesian coordinates to polar coordinates?
\end{itemize}
\begin{columns}[c]
\column{0.5\textwidth}
\psset{xunit=4cm, yunit=4cm}
\begin{pspicture}(-0.2, -0.2)(1.400000,0.9)
\tiny
%force a boudning box:
\psline[linecolor=red!1](-0.1, -0.1)(-0.21,0.2)
\psline[linecolor=red!1](1.1, 0.6)(1.1,0.61)
\psaxes[arrows=<->, ticks=none, labels=none](0,0)(-0.2, -0.2)(1, 0.8)
\rput(-0.03, 0.8){$y$}
\rput(1,-0.03){$x$}
%\fcAxesStandard{-0.2}{-0.2}{1}{0.8}

%Calculator input: plotCurve{}(1/5 \cos{}t, 1/5 \sin{}t, 0, 1/6 \pi)
\parametricplot[arrows=->, linecolor=red, plotpoints=1000] {0}{0.523599}{t 57.29578 mul cos 0.2 mul t 57.29578 mul sin 0.2 mul }
\psline[linecolor=blue](0,0)(0.866025404, 0.5)
\rput(0.22, 0.06){$\theta$}

\fcFullDotBlue{0.866025404}{0.5}
\psline(0.866025404,0.5)(0.866025404,0)
\psline(0.846025404, 0)(0.846025404, 0.02)(0.866025404, 0.02)
\rput[l](0.9, 0.5){$P(r,\theta) =(x,y)$}

\uncover<6>{
\psline{<-}(0.89, 0)(0.89, 0.2)
}
\uncover<6->{
\rput(0.89,0.25){$\alertNoH{6,10}{y}$}
}
\uncover<6>{
\psline{->}(0.89, 0.3)(0.89, 0.5)
}
\uncover<4>{
\psline{<-}(0, -0.02)(0.385, -0.02)
}
\uncover<4->{
\rput(0.435,-0.02){$\alertNoH{4,10}{x}$}
}
\uncover<4>{
\psline{->}(0.485, -0.02)(0.866025404, -0.02)
}
\rput[tr](-0.03, -0.03){$O$}
\rput(0.4, 0.3){$\alertNoH{4,6,9,10}{r}$}
\end{pspicture}
%\ \uncover<2->{%
%\includegraphics[height=5cm]{polar-curves/pictures/11-03-conversion.pdf}%
%}%
\column{.5\textwidth}
$
\renewcommand{\arraystretch}{1.5}
\begin{array}{rcl}
\alertNoH{7}{x} &\alertNoH{7}{=}&\uncover<7->{\alertNoH{7}{r\cos \theta} } \\
\alertNoH{7}{y}&\alertNoH{7}{=}& \uncover<7->{\alertNoH{7}{r\sin \theta}} \\\hline
\uncover<3->{\alertNoH{3,4,7}{\cos\theta}} & \uncover<3->{\alertNoH{3,4,7}{=}} &\displaystyle \uncover<4->{\alertNoH{4,7}{\frac{x}{r} }} \\
\uncover<3->{\alertNoH{5,6,7}{\sin \theta}} &\uncover<3->{ \alertNoH{5,6,7}{= }} & \displaystyle \uncover<6->{\alertNoH{6,7}{\frac{y}{r}}}\\
\uncover<9->{\alert<handout:0| 9-10>{r^2} &\alertNoH{9,10}{=}& \uncover<10->{\alertNoH{10}{x^2 + y^2}}}\\\hline
\alertNoH{8,9,10,11}{ r}&\alertNoH{8,9,10,11}{=}&\uncover<11->{\alertNoH{11}{ \sqrt{x^2+y^2}}} \\
\alertNoH{12,13}{\theta} &\alertNoH{12,13}{=}&
\uncover<13->{\alertNoH{13}{\arcsin \left(\frac{y}{r}\right)  \text{\quad if } x>0}}\\
\uncover<13->{
&\alertNoH{13}{=}&\alertNoH{13}{\arccos \left(\frac{x}{r}\right)  \text{\quad if } y>0}}\\
\uncover<13->{&\alertNoH{13}{=}&\alertNoH{13}{\arctan \left(\frac{y}{x}\right)  \text{\quad if } x>0}}
\end{array}
$


\end{columns}
\end{frame}
% end module polar-to-cartesian
