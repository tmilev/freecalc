% begin module polar-intersection-ex3
\begin{frame}
\begin{example} %[Example 3, p. 688]
Find all points of intersection of the polar curves $r = \frac{1}{2}$ and $r = \cos (2\theta)$.
\begin{columns}[c]
\column{.4\textwidth}

\psset{xunit=1.8cm, yunit=1.8cm}
\begin{pspicture}(-1.5, -1.5)(1.5,1.5)
\tiny
\fcAxesStandard{-1.4}{-1.4}{1.4}{1.4}
%Calculator command: drawPolar{}(1/2, 0, 2 \pi)
\parametricplot[linecolor=\fcColorGraph, plotpoints=1000, algebraic=false]{0}{6.28319}{ 0.5 t 57.29578 mul cos mul 0.5 t 57.29578 mul sin mul }
%Calculator command: drawPolar{}(\cos{}(2 t), 0, 2 \pi)
\parametricplot[linecolor=\fcColorGraph, plotpoints=1000, algebraic=false]{0}{6.28319}{t 2 mul 57.29578 mul cos t 57.29578 mul cos mul t 2 mul 57.29578 mul cos t 57.29578 mul sin mul }

\rput[tr](-0.8, -0.8){$r=\frac{1}{2}$}
\psline{->}(-0.75, -0.75)(-0.353553391, -0.353553391)
\rput[lt] (0.2, -1){$r=\cos (2\theta)$}

\uncover<4->{
\fcFullDotBlack{0.433013}{0.25}
\fcFullDotBlack{-0.433013}{0.25}
\fcFullDotBlack{-0.433013}{-0.25}
\fcFullDotBlack{0.433013}{-0.25}
}
\uncover<6->{
\fcFullDotBlack{0.25}{0.433013}
\fcFullDotBlack{0.25}{-0.433013}
\fcFullDotBlack{-0.25}{-0.433013}
\fcFullDotBlack{-0.25}{0.433013}
}
\uncover<9>{
\pscircle*[linecolor=red](0.433013,0.25){0.09}
\pscircle*[linecolor=red](-0.433013,0.25){0.09}
\pscircle*[linecolor=red](-0.433013,-0.25){0.09}
\pscircle*[linecolor=red](0.433013,-0.25){0.09}
}
\uncover<10>{
\pscircle*[linecolor=red](0.25,0.433013){0.09}
\pscircle*[linecolor=red](0.25,-0.433013){0.09}
\pscircle*[linecolor=red](-0.25,-0.433013){0.09}
\pscircle*[linecolor=red](-0.25,0.433013){0.09}
}
\end{pspicture}

%\ \only<handout:0| -3>{%
%\includegraphics[height=5cm]{polar-curves/pictures/11-04-ex3a.pdf}%
%}%
%\only<handout:0| 4-5>{%
%\includegraphics[height=5cm]{polar-curves/pictures/11-04-ex3b.pdf}%
%}%
%\only<6-8>{%
%\includegraphics[height=5cm]{polar-curves/pictures/11-04-ex3c.pdf}%
%}%
%\only<handout:0| 9>{%
%\includegraphics[height=5cm]{polar-curves/pictures/11-04-ex3d.pdf}%
%}%
%\only<handout:0| 10->{%
%\includegraphics[height=5cm]{polar-curves/pictures/11-04-ex3e.pdf}%
%}%
\column{.6\textwidth}
\abovedisplayskip=0pt
\belowdisplayskip=0pt
\begin{eqnarray*}
\uncover<2->{%
\cos 2\theta%
}%
& \uncover<2->{ = } &%
\uncover<2->{%
\frac{1}{2}%
}\\%
\uncover<3->{%
2\theta%
}%
& \uncover<3->{ = } &%
\uncover<3->{%
\frac{\pi}{3}, \frac{5\pi}{3}, \frac{7\pi}{3}, \frac{11\pi}{3}%
}\\%
\uncover<4->{%
\theta%
}%
& \uncover<4->{ = } &%
\uncover<4->{%
\frac{\pi}{6}, \frac{5\pi}{6}, \frac{7\pi}{6}, \frac{11\pi}{6}%
}%
\end{eqnarray*}
\begin{itemize}
\item<5->  This only gives four points.
\item<6->  There are actually eight.
\item<7->  The circle $r = \frac{1}{2}$ also has polar equation $r = -\frac{1}{2}$.
\item<8->  To find all eight points, solve \alert<handout:0| 9>{$\cos (2\theta )= \frac{1}{2}$} and \alert<handout:0| 10>{$\cos (2\theta) = -\frac{1}{2}$}.
\end{itemize}
\end{columns}
\end{example}
\end{frame}
% end module polar-intersection-ex3
