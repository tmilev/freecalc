\begin{frame}
\footnotesize
\frametitle{Cantor-Zassenhaus algorithm}
Input: a polynomial $f\in \mathbb Z_p[x]$. Output: a list of of probably irreducible factors $R$.
\begin{itemize}
\item<2->[0] Initialize the list of non-reduced factors  $N=(f)$ and the list of fully reduced factors as the empty list $R=()$. 
\item<3->[1] If $N$ is empty, terminate the algorithm;\uncover<4->{ else remove the first element $f$ of $N$.  }
\item<5->[2]  If $\gcd(f,f')\neq 1$, add $\gcd(f,f')$ and $f/\gcd(f,f')$ to $N$. Go back to Step 1.
\item<6->[3] Compute $\gcd (x^{p^n}-x, f)$ for $n=1,..., \deg f$. If a non-trivial factor $h$ is found, add $h,\frac{f}{h}$ to $N$ and go back to Step 1.
\item<7->[4] For $X$ rounds, repeat:
\item<8->[4.1] Pick a (pseudo-) random $h$.
\item<9->[4.2] For every positive divisor $k$ of $\deg f$:
\item<10->[4.2.1] Compute $a=\gcd \left(h^{\frac{p^k-1}{2}}-1 \right)$.
\item<11->[4.2.2] If $a$ is non-trivial, add $a,\gcd(a,f)$ to $N$ and go back to Step 1.
\item<12->[5] If $X$ rounds did not produce a divisor of $f$, declare $f$ to be probably irreducible with error probability $1$ in $2^X$. 
\item<13->[6] Append $f$ to $R$ and go back to Step 1.
\end{itemize}
\end{frame}