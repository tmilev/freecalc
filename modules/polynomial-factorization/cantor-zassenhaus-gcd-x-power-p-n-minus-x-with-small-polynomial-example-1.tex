\begin{frame}
\footnotesize
\vskip -0.22cm
\begin{example}\vskip -0.1cm
Compute the remainder of  $x^{3^{6}}-x \mod 3$ divided by $x^2+x+1\mod 3$. Use the fact that $\alertNoH{26}{3^{6}=729=1+2^3+2^4+2^6+2^7+2^9}$.
\uncover<2->{\alertNoH{17,35,36}{All computations are $\mod 3$.}}

\begin{itemize}
\item<2-> \only<-8|handout:1>{ Regular polynomial division would yield $3^{6}=729$ rows. 
\item<3-> Instead, we will compute in the ring $\mathbb Z_3[x]/\langle x^2+x+1\mod 3 \rangle$.
\item<4->In other words, we will compute $\left( x^{3^{6}}-x\right) \mod \left(x^2+x+1 \right)$.

\uncover<5->{
Computing $\mod \left(x^2+x+1 \right)$ amounts to declaring 
}
}

\uncover<5->{
\hfil\hfil$
\alertNoH{8,9} {
\begin{array}{rcll}
\alertNoH{20}{ x^2+\alertNoH{6}{x+1}}&\alertNoH{20}{ \equiv}&\alertNoH{20}{  0} &\mod 3\\
\uncover<6->{ \alertNoH{12,22,29,31,33}{ x^2} &\alertNoH{12,22,29}{\equiv} & \alertNoH{7}{-}(\alertNoH{6}{x+1})\uncover<6->{\equiv\alertNoH{12,22,29,31,33}{ \alertNoH{7}{2}x+\alertNoH{7}{2}} } &\mod 3}
\end{array}
}
$
}


\only<handout:2|9->{
$
\begin{array}{rcrclcl}
\uncover<10->{ \alertNoH{27}{ x^{2^0}} &=& x^1&&& \equiv&\alertNoH{23,27}{ x}}\\
\uncover<11->{ x^{2^1} &=& \alertNoH{12,15,22}{x^2} &&& \alertNoH{12,15,22}{ \equiv}& \alertNoH{12,15,22,23}{ 2x+2}}\\
\uncover<13->{x^{2^2} &=& \alertNoH{13,14,21}{x^4}& \alertNoH{13,14}{\equiv}& \fcAnswer{14}{ \alertNoH{ 30}{ (\alertNoH{15}{ \alertNoH{16}{2}x+ \alertNoH{16}{2}} )^{ \alertNoH{16}{2}}}} \uncover<16->{ =\alertNoH{16,17}{4}(x+1)^2 \uncover<17->{=\alertNoH{18}{ ( x + 1)^2}}} }\\
\uncover<18->{ &&&\equiv&\alertNoH{18}{ x^2+ \alertNoH{19}{ 2x} +1}\uncover<19->{=\alertNoH{20}{x^2+\alertNoH{19}{x}}+\alertNoH{19}{x}\alertNoH{20}{+1}} } \uncover<20->{&\equiv& \alertNoH{21,23,30}{x} } \\
\uncover<21->{ \alertNoH{28}{ x^{2^3}} &=&\alertNoH{21}{ x^8} & \equiv& \alertNoH{21,22}{x^2} \uncover<22->{ &\alertNoH{22}{=} &\alertNoH{22,23,28}{2x+2}} }\\
\uncover<23->{
\alertNoH{27}{x^{2^4}}&=&x^{16}&&&\equiv&\alertNoH{23,27}{ x}\\
x^{2^5}&=&x^{32}&&&\equiv&\alertNoH{23}{ 2x+2}\\
} 
\uncover<24->{
\alertNoH{27}{x^{2^6}}&&&&&\equiv&\alertNoH{27}{x}\\
\alertNoH{28}{x^{2^7}}&&&&&\equiv&\alertNoH{28}{2x+2}\\
} 
\uncover<25->{
x^{2^8}&&&&&\equiv&x\\
\alertNoH{28}{x^{2^9}}&&&&&\equiv&\alertNoH{28}{2x+2}\\
}
\uncover<26->{x^{\alertNoH{26}{729} }&&&=& {\alertNoH{27,28}{ x}}^{ \alertNoH{26}{ \alertNoH{27}{1}+\alertNoH{28}{2^3}+ \alertNoH{27}{ 2^4}+ \alertNoH{27}{ 2^6 }+\alertNoH{28}{2^7}+\alertNoH{28}{2^9}}} \uncover<27->{ \equiv \alertNoH{27,29}{x^3} (\alertNoH{28}{2x +2})^{ \alertNoH{28}{3}}}}\\
\uncover<29->{
&&&=&\alertNoH{29}{x} \alertNoH{30}{(\alertNoH{29}{2x+ 2})^4}\uncover<30->{= x\cdot \alertNoH{30,31}{ x^2}}\uncover<31->{= \alertNoH{32}{x(\alertNoH{31}{2x+2})}}\\
\uncover<32->{&&&=& \alertNoH{32}{2 \alertNoH{33}{x^2}+2x} \uncover<33->{ =2(\alertNoH{33}{2x+2})+2x}\uncover<34->{ =\alertNoH{35}{6x}+\alertNoH{36}{4}} \uncover<35->{&\equiv&\alertNoH{36}{1}}}
}
\end{array}
$
}
\end{itemize}
\end{example}
\vskip 10cm
\end{frame}

\begin{frame}
\footnotesize
\vskip -0.22cm
\begin{example}\vskip -0.1cm
Compute the remainder of  $x^{3^{6}}-x \mod 3$ divided by $x^2+x+1\mod 3$.
\end{example}
\begin{itemize}
\item<2-> As we saw on the preceding slide, the answer is $1$.
\item<3-> We could have computed faster: 
\begin{itemize}
\footnotesize
\item<3-> $ x^{3}\equiv x\left(2x+2\right)\equiv 2x^2+2\equiv 6x+4\equiv 1 \mod \left(x^2+x+1\mod 3\right) $.
\item<4-> $x^{3^6}= \left( x^3\right)^{3^5}\equiv 1^{3^5}=1\mod \left(x^2+x+1\mod 3\right)$.
\end{itemize}
\item<5-> However, the previous method works for all examples.
\begin{itemize}
\footnotesize
\item<6-> We used consecutive squares to raise $x$ to a huge exponent.
\item<7-> We reduced all intermediates $\mod$ the polynomial before squaring again.
\item<8-> $\Rightarrow$ computations were small = good news when coding on a computer!
\end{itemize}
\item<9-> To find the gcd of a polynomial $g$ and $x^{p^n}-x$, we use the Euclidean algorithm.
\item<10-> The Euclidean algorithm only uses the remainder of $x^{p^n}-x$ divided by $g$.
\item<11-> $\Rightarrow$ finding $\gcd\left( x^{p^n}-x,g\right)$ is efficient computationally.
\item<12-> This is so efficient it's doable by hand on not-so-small-examples.
\end{itemize}


\vskip 10cm
\end{frame}