\begin{frame}
\frametitle{Factorization over $\mathbb Z$ using finite fields: the problems}
\begin{itemize}
\item We just described a plan for extracting a factorization over $\mathbb Z$ from a factorization $\mod p$.
\item The plan had a few problems / open questions.
\begin{itemize}
\item How to choose the prime $p$?
\item How to efficiently factor $\mod p$? 
\item \alert<1->{What happens when factoring $\mod p$ yields more factors than we have over $\mathbb Z$?}
\item How to choose $n$ large enough so that factorization $\mod p^n$ helps us read off factorization over $\mathbb Z$?
\end{itemize}
\end{itemize}
\end{frame}

\begin{frame}
\begin{itemize}
\item Suppose $g$ is an irreducible factor of $f$ over $\mathbb Z$.
\item<2-> Suppose $g$ factors $\mod p$ into irreducibles: 
	
\[
\begin{array}{rcll|l}
g&\equiv& a_1\cdots a_s & \mod p  \\
g&\equiv& d_1\cdots d_s & \mod p^n &d_i \text{ is Hensel-lifted from } a_i
\end{array}
\]
\item<3-> Then the coefficients of $g$ can be read off the coefficients of the expanded product $d_1\cdots d_s$.
\end{itemize}
\end{frame}

\begin{frame}
\begin{question}
What happens when factoring $\mod p$ yields more factors than we have over $\mathbb Z$?
\end{question}
\begin{itemize}
\item<2-> Suppose $f\equiv d_1\cdots d_k \mod p^n$.
\item<3-> Suppose $C$ is the leading coefficient of $f$.
\item<4-> For every subset $\{i_1, \dots, i_s\}\subset \{1,\dots, k\}$:
\begin{itemize}
\item<5-> Expand $g=C\cdot d_{i_1}\cdots d_{i_s} \mod p^n$ fully.
\item<6-> Convert $g \in \mathbb Z_{p}^n[x]$ to integer-valued $\bar g \in \mathbb Z[x]$: \uncover<7->{replace every positive (representative) coefficient $z$ larger than $\frac{p^n}{2}$ to $z-p$.}
\item<8-> Divide $f$ by $\bar g$ over $\mathbb Q$; if remainder is zero, we've found a factor of $f$.
\end{itemize}  
\item<9-> We could try all subsets of $\{1,\dots, k\}$ (total $2^k$ possibilities).
\item<10-> However, if a factor involves more than half of the indices, its complement factor has fewer than half of the indices. 
\item<11-> $\Rightarrow$ we need to go \alertNoH{12}{up to subsets of size $\lfloor \frac{k}{2}\rfloor$}.  
\item<12-> $ \alertNoH{12}{\text{total subsets to try}=\begin{cases}
2^{k-1}& \text{ if } k-\text{odd}\\
2^{k-1} + \frac{1}{2}\binom{k}{\frac{k}{2}}& \text{ if } k-\text{odd}\\
\end{cases}}$ .
\item<13-> Try smaller subsets first $\Rightarrow$ discover irreducible factors first.
\end{itemize}
\end{frame}