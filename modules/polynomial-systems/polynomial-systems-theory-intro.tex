\begin{frame}
\frametitle{Systems of polynomial equations}
\begin{definition}
A collection of one or more simultaneous polynomial equations in one or more variables is called a \emph{polynomial system}.
\end{definition}
\begin{itemize}
\only<handout:1|1-7>{
\item<2-> The definition includes usual one-variable polynomial equations such as:

\hfil\hfil$x^2+2x-3=0$

\item<3-> Typical polynomial systems have more than one variable/equation:

\hfil\hfil$
\left|\begin{array}{rcl}
y^{2}+x y-4 y-2 x+4&=&0\\
y^{2} x-2 y x- y-2 x+4&=&0.
\end{array}\right.
$

Here we have \fcAnswer{5}{2} variables \uncover<5->{($x,y$)}, \fcAnswer{5}{2} equations.
\item<6-> The number of variables and equations need not be equal:

\hfil\hfil $
\left|\begin{array}{rcl}
x+y+z+w&=&2\\
y+z^2&=&1\\
y+z w^2&=&1.\\
\end{array}\right.
$

Here we have \fcAnswer{7}{4} variables \uncover<7->{($x,y,z,w$)}, \fcAnswer{7}{3} equations.
}
\only<handout:2|8-13>{
\item<8-> Polynomial systems may have no solutions:
$
\left|\begin{array}{rcl}
x&=&0\\
x y &=&1.
\end{array}\right.
$

\uncover<9->{The first equation implies $x=0$, but then the left hand side of the second equation must equal $0$.}
\item<10-> Polynomial systems may finitely many solutions:
$
\left|\begin{array}{rcl}
x&=&0\\
y+x &=&1.
\end{array}\right.
$
\uncover<11->{The first equation implies $x=0$, and then the second equation implies $y=1$.}
\item<12-> Polynomial systems may have infinitely many solutions:
$
\left|\begin{array}{rcl}
x&=&0\\
y+z &=&1.
\end{array}\right.
$
\uncover<13->{If we set $x=0$, $y=1-z$, we produce infinitely many solutions for every possible value of $z$.}
}
\only<handout:3|14-17>{
\item<14-> The branch of mathematics that studies exact solutions of polynomial systems is called Algebraic Geometry.
\item<15-> The practical aspects of solving such systems are covered under the subject of Elimination Theory.
\item<16-> Solving polynomial systems is an indispensable mathematical tool used in other branches of science and mathematics.
\item<17-> Polynomial systems also have direct practical applications, for example kinematics - the configurations of a robotic arm can be parametrized with polynomials. 
}
\only<handout:4|18->{
\item<18-> Whether a system has finitely or infinitely many solutions and what are they can be computed with a computer algorithm. 
\item<19-> Algorithm: find so-called Gr\"obner basis (named after the Austrian W. Gr\"obner, 1899-1980) using the Buchberger algorithm(named after the Austrian B. Buchberger, 1942-).
\item<20-> These algorithms are too advanced to cover here.
\item<21-> Not well-suited for pen and paper computations: can get notoriously large and require computers/super-computers.
\item<22-> A system doable by hand would typically be solved milliseconds on a modern computer.
\item<23-> A system doable by hand would typically be solved easily using ad-hoc techniques.
}
\end{itemize}


\vskip 10cm
\end{frame}