% begin module power-series-calculus
\begin{frame}
\frametitle{Differentiation and Integration of Power Series}
\begin{theorem}[Differentiation and Integration of Power Series]
If a power series $\sum c_n (x-a)^n$ has radius of convergence $R >0$, then the function $f$ defined by
\abovedisplayskip=0pt
\belowdisplayskip=0pt
\[
f(x) = c_0 + c_1(x-a) + c_2(x-a)^2 + c_3(x-a)^3 +\cdots  = \sum_{n=0}^\infty c_n(x-a)^n%
\]
is differentiable (and therefore continuous) on the interval $(a-R, a+R)$ and
\begin{enumerate}
\item  $\displaystyle f'(x) = c_1 + 2c_2(x-a) + 3c_3(x-a)^2 + \cdots = \sum_{n=1}^\infty nc_n(x-a)^{n-1}$.
\item  $\displaystyle \int f(x) \ \diff x  = C + c_0(x-a) + c_1\frac{(x-a)^2}{2} + c_2\frac{(x-a)^3}{3} + \cdots $ $= C + \sum_{n=0}^\infty c_n\frac{(x-a)^{n+1}}{n+1}$.
\end{enumerate}
\end{theorem}
\end{frame}


\begin{frame}
\begin{itemize}
\item  This is called term-by-term differentiation and integration.
\item  Another way of saying it is
\end{itemize}
\begin{eqnarray*}
\frac{\diff}{\diff x} \left[ \sum_{n=0}^\infty c_n (x-a)^n\right] & = & \sum_{n=0}^\infty \frac{\diff}{\diff x} \left[ c_n (x-a)^n\right]\\
\int \left[ \sum_{n=0}^\infty c_n (x-a)^n\right] \diff x & = & \sum_{n=0}^\infty \int \left[ c_n (x-a)^n\right] \diff x
\end{eqnarray*}
\begin{itemize}
\item  We can treat power series like polynomials with infinitely many terms.
\end{itemize}
\end{frame}
% end module power-series-calculus
