\solution{
\ref{probelmSum_n=3^infty 3/(n^2-3n+2) } 
%Solution contributed by student Kreg Hanning
\[
\begin{array}{rcll|l}
\displaystyle \sum_{n=3}^{\infty} \frac{3}{n^2-3n+2} &=&\displaystyle  \sum_{n=3}^{\infty}\left( \frac{3}{n-2} - \frac{3}{n-1} \right)  &&\text{use partial fractions, see below}\\
&=&\displaystyle  3\sum_{n=3}^{\infty} \left( \frac{1}{n-2} - \frac{1}{n-1} \right)\\
&=&\displaystyle 3\left( \underset{n=3}{\left(1-\frac{1}{2}\right)} + \underset{n=4}{\left(\frac{1}{2}-\frac{1}{3}\right)} + \underset{n=5}{\left(\frac{1}{3}-\frac{1}{4}\right)} + \dots\right)\\~\\
&=&\displaystyle 3\lim\limits_{n\rightarrow\infty} \left( 1 - \frac{1}{n-1} \right) = 3\quad .
\end{array}
\]

In the above we used the partial fraction decomposition of $\displaystyle \frac{3}{n^2-3n+2}$. This decomposition is computed as follows. 

\[\frac{3}{n^{2}-3n +2}=\frac{3}{ \left(n -1\right)\left(n -2\right)}\]We need to find $A_i$'s so that we have the following equality of rational functions. After clearing denominators, we get the following equality. \[3 = A_{1} (n -2)+A_{2} (n -1)\]After rearranging we get that the following polynomial must vanish. Here, by ``vanish'' we mean that the coefficients of the powers of $x$ must be equal to zero.\[(A_{2} +A_{1} )n +(-A_{2} -2A_{1} -3)\]In other words, we need to solve the following system. \[\begin{array}{llll} & -2A_{1} & -A_{2} & =3\\ & A_{1} & +A_{2} & =0\\\end{array}\] \begin{longtable}{cc} System status&Action \\\hline $\begin{array}{llll} & -2A_{1} & -A_{2} & =3\\ & A_{1} & +A_{2} & =0\\\end{array}$& Selected pivot column 2. Eliminated the non-zero entries in the pivot column. \\\hline $\begin{array}{llll} & A_{1} & +\frac{A_{2} }{2} & =-\frac{3}{2}\\ & & \frac{A_{2} }{2} & =\frac{3}{2}\\\end{array}$& Selected pivot column 3. Eliminated the non-zero entries in the pivot column. \\\hline $\begin{array}{llll} & A_{1} & & =-3\\ & & A_{2} & =3\\\end{array}$& Final result.\\ \end{longtable}

Therefore, the final partial fraction decomposition is the following. 
\[
\frac{3}{n^{2}-3n +2}=\frac{-3}{(n -1)}+\frac{3}{(n -2)}.
\]
}