% begin module integral-test-ex4
\begin{frame}
\begin{example}[Example 4, p. 736]
Test the series $\displaystyle \sum_{n=1}^\infty \frac{\ln n}{n}$ for convergence.
\begin{itemize}
\item<2->  $f(x) = \frac{\ln x}{x}$ is continuous and positive.
\item<3->  It's not obvious if it's decreasing, so take the derivative.
\abovedisplayskip=0pt
\belowdisplayskip=0pt
\uncover<4->{%
\[
f'(x) = \frac{\left(\frac{1}{x}\right)(x) - (\ln x)(1)}{x^2} %
\uncover<5->{ = \frac{1-\ln x}{x^2}}
\]
}%
\item<6->  This is negative for all \alert<handout:0| 6-7>{$x >$ \uncover<7->{$e$.}}
\item<8->  Therefore $f$ is decreasing for all $x > e$.
\end{itemize}
\abovedisplayskip=0pt
\belowdisplayskip=0pt
\begin{eqnarray*}
\uncover<9->{\int_1^\infty \frac{\ln x}{x}\diff x} & \uncover<9->{ = } & \uncover<9->{\lim_{t\to\infty} \int_1^t \alert<handout:0| 10-11>{\frac{\ln x}{x}}\diff x} %
 \uncover<10->{ = } %
\uncover<10->{\lim_{t\to\infty}\left[ \uncover<11->{\alert<handout:0| 11>{\frac{(\ln x)^2}{2}}}\right]_{\alert<handout:0| 13-14>{1}}^{\alert<handout:0| 12>{t}}}\\
 & \uncover<12->{ = } & %
\uncover<12->{\alert<handout:0| 15-16>{\lim_{t\to\infty}}\left( \alert<handout:0| 12,15-16>{\frac{1}{2}(\ln t)^2} - \uncover<14->{\alert<handout:0| 14>{0}} \right)} %
 \uncover<15->{ = } %
\uncover<16->{\alert<handout:0| 16>{\infty}}
\end{eqnarray*}
\uncover<17->{%
Therefore $\sum_{n=1}^\infty \frac{\ln n}{n}$ is \uncover<18->{divergent.}%
}%
\end{example}
\end{frame}
% end module integral-test-ex4
