\begin{frame}
  \frametitle{Examples}
The equation $x^2+y^2+z^2 = 1$ implicitly defines $z=\sqrt{1-x^2-y^2}$ as the unique function $z=f(x,y)$ such that
%
\begin{itemize}
  \item $x^2+ y^2+(f(x,y))^2 = 1$ for all $(x,y)$ in a disk around $(0,0)$;
  \item $f(0,0) = 1$.
\end{itemize}

\pause
The equation $x^2+y^2+z^2 = 1$ implicitly defines $z=-\sqrt{1-x^2-y^2}$ as the unique function $z=f(x,y)$ such that
%
\begin{itemize}
  \item $x^2+ y^2+(f(x,y))^2 = 1$ for all $(x,y)$ in a disk around $(0,0)$;
  \item $f(0,0) = -1$.
\end{itemize}

\pause
In general, the implicit function question is quite tricky!

\pause
But for differentiable functions things are somehow simpler.
\end{frame}
