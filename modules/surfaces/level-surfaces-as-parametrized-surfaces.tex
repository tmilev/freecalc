\begin{frame}
  \frametitle{Level Surfaces as Parametrized Surfaces}

Let $f \colon \RR^3 \to R$ be a differentiable function.
\begin{itemize}
  \item $P$ is a \emph{critical point} if $(\nabla f)(P) = 0$;
  \item $c \in \RR$ is a \emph{critical value} if
%
$$f^{-1}(c) = \{ (x,y,z) \; | \; f(x,y,z) = c\}$$
%
contains a critical point.
\item If $c$ is not a critical value then it is a \emph{regular} value.
\end{itemize}

Implicit Function Theorem $\Longrightarrow$

If $c$ is a regular value taken by $f$, then $f^{-1}(c)$ is a smooth surface.

\underline{Examples}:
 \begin{overlayarea}{\textheight}{5cm}
 \only<2>{
 For $f(x,y,z) = ax+by+cz$ with $a$, $b$, $c$ not all zero,
 \begin{itemize}
 \item All points $P$ are regular points
   \item All values $d$ are regular;
   \item $f^{-1}(d)$ is a smooth surface: the plane $ax+by+cz=d$.
 \end{itemize}
 }

 \only<3>{
 For $f(x,y,z) = x^2+y^2-z^2$,
 \begin{itemize}
   \item $(0,0,0)$ is the only critical point $\Longrightarrow$ $0=f(0,0,0)$ is the only critical value.
   \item $H=f^{-1}(1)$ is a smooth surface: $x^2+y^2-z^2=1$, a hyperboloid with one sheet.
 \end{itemize}
 }
 \end{overlayarea}
\end{frame}