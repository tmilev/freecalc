We recall that the substitution $\theta = 2\arctan t$ transforms a trigonometric integral into an integral of a rational function. We now apply the substitution $2\arctan t$ after the substitution $x=\sec \theta$:

\begin{equation*}
\begin{array}{rcll|l}
x&=&\displaystyle \sec \theta=\frac{1}{\cos \theta} && \text{use } \theta =2 \arctan t\\
&=&\displaystyle \frac{1} {\cos(2\arctan t)} && \text{use\refBad{\ref{eqSinCosViaTan}}{ }{ \eqref{eqSinCosViaTan}: }} \displaystyle \cos (2z) = \frac{ 1- \tan^2 z}{1+ \tan^2 z}\\
&=& \displaystyle \frac{1+\tan^2(\arctan t)}{1-\tan^2( \arctan t)} \\
&=&\displaystyle \frac{1+t^2}{1-t^2}\\
&=&\displaystyle-1+\frac{2}{1-t^2} \quad .
\end{array}
\end{equation*}
We can furthermore compute
\begin{equation}\label{eqsqrtxsquareminus1E2}
\begin{array}{rcll|l}
\sqrt{x^2-1 }&=&\displaystyle \sqrt{ \left(\frac{1+t^2}{1-t^2}\right)^2-1}\\
&=& \displaystyle \sqrt{\frac{(1+t^2)^2-(1-t^2)^2}{(1-t^2)^2} }\\
&=& \displaystyle \sqrt{\frac{4t^2}{(1-t^2)^2}} && \begin{array}{l} \displaystyle t \text{ and }1-t^2\text{ have the same}\\ \text{sign for } t\in (-\infty, -1) \cup \left[0, 1 \right)\end{array} \\
&=&\displaystyle \frac{2t}{1-t^2}\quad .
\end{array}
\end{equation}
The differential $\diff x$ can be computed from $x=\frac{1+t^2}{1-t^2}$. Finally, we can express $t$ via $x$ with a little algebra:
\[
\begin{array}{rcll|l}
\displaystyle x&=&\displaystyle  \frac{1+t^2}{ 1- t^2} \\
\displaystyle (1- t^2)x&=&\displaystyle  1+t^2\\
\displaystyle (1+ x)t^2&=&\displaystyle  x-1\\
\displaystyle t^2&=&\displaystyle  \frac{x-1}{x+1}\\
\\
\displaystyle t&=&\displaystyle \doublebrace{ \sqrt{\frac{x-1}{x+1}}}{x>1}{-\sqrt{\frac{x-1}{x+1}}}{x<-1} &&\begin{array}{l} \text{because when } x<-1, \\ \text{ we have } t\in \left( -\infty , -1 \right]\end{array} \\
\displaystyle t&=&\displaystyle \doublebrace{ \frac{\sqrt{ x^2 -1}}{x+1}}{x>1}{-\frac{\sqrt{x^2-1}}{x+1}}{x<-1}  \quad .
\end{array}
\]
The Euler substitution $x= \sec (2\arctan t)$ can be now summarized as:

\[
\begin{array}{rcl}
x&=&\displaystyle \frac{1+t^2}{1-t^2}\\
\sqrt{x^2-1}&=&\displaystyle \frac{2t}{1-t^2}  \\
\diff x&=&\displaystyle  \frac{4 t}{(1- t^{2})^{2}} \diff t\\
t&=&\displaystyle \pm \frac{ \sqrt{x^2-1}}{x+1} \quad .
\end{array}
\]