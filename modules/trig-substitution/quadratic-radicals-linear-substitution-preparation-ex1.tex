%begin module quadratic-radicals-linear-substitution-preparation-ex1
\begin{frame}
\frametitle{Linear substitutions to simplify radicals $\sqrt{ay^2+by+c}$}
\begin{itemize}
\item Using linear substitutions, radicals of form  $\sqrt{ay^2+by+c}$, $a\neq 0$, $b^2-4ac\neq 0$ can be transformed to (multiple of):
\begin{itemize}
\item $\sqrt{x^2+1}$
\item $\sqrt{-x^2+1}$
\item $\sqrt{x^2-1}$.
\end{itemize}
\item We already studied how to do that using completing the square when dealing with rational functions.
\end{itemize}
\end{frame}
\begin{frame}
Recall: linear substitution is subst. of the form $u=px+q$.
\begin{example}
Use linear substitution to transform $\sqrt{x^2+x+1}$ to multiple of $\sqrt{u^2+1}$.

\noindent $
\begin{array}{rcl}
\sqrt{x^2+x+1}&=&\displaystyle \uncover<2->{ \sqrt{ x^2+2\cdot \frac{1}{2}x + \fcAnswer{3}{\frac{1}{4}}  \uncover<2->{ \alertNoH{2,3}{-} } \fcAnswer{3}{  \frac{1}{4}} +1}} \\
\uncover<4->{&=&\displaystyle \sqrt{ {\left(x+\fcAnswer{5}{\frac{1}{2}}  \right)}^2 +  \fcAnswer{5}{ \alertNoH{5, 6}{ \frac{3}{4}}} }} \\
\uncover<6->{&=&\displaystyle \sqrt{ \alertNoH{6,7}{ \frac{3}{4}}\left( \alertNoH{6,8}{\frac{4}{3}} \left(x+\frac{1}{2}\right)^{\alertNoH{8}{2}} +\alertNoH{6}{ 1} \right)}}\\
\uncover<7->{&=&\displaystyle \alertNoH{7}{\frac{\sqrt{3}}{2}} \sqrt{\left(  \alertNoH{9}{\alertNoH{8}{\frac{2}{\sqrt{3}}} \left( x+ \frac{1}{2} \right)}\right)^{\alertNoH{8}{2}}+1}}\\
\uncover<9->{ &=&\displaystyle \frac{\sqrt{3}}{2} \sqrt{ {\alertNoH{9}{u}}^2+1},}
\end{array}
$

\noindent \uncover<9->{ where $\displaystyle \alertNoH{9}{u= \frac{2}{\sqrt{3}}\left( x+\frac{1}{2}\right)}  =\frac{2\sqrt{3}}{3}x +\frac{\sqrt{3}}{3} $.}
\end{example}
\vspace{5cm}
\end{frame}
%end module quadratic-radicals-linear-substitution-preparation-ex1
