\begin{frame}
\frametitle{Angle measure}
\begin{columns}
\column{0.25\textwidth}
\begin{pspicture}(-1,-1)(6, 4)
\tiny
\psline[arrows=->](0,0)(2,0)
\psline[arrows=->](0,0)(! 1 3 sqrt )
\fcAngleBetweenVectors[arrows=->, linecolor=blue]{[2 0]}{[1 3 sqrt]}{0.3}{}%
\fcAngleBetweenVectors[linecolor=red]{[2 0]}{[1 3 sqrt]}{1}{}%
\parametricplot[linestyle=dashed, linecolor=red!60]{60}{360}{t cos t sin}
\fcFullDot[linecolor=blue]{0}{0}%
\rput[t] (-0.2, -0.2){$O$}%
\fcFullDot{1}{0}%
\rput[t] (1, -0.2){$A$}%
\fcFullDot{0.5 }{3 sqrt 0.5 mul}%
\rput[rt] (0.6, 1.2){$B$}%
\end{pspicture}

\column{0.75\textwidth}
\begin{definition}[Radian measure of geometric angle]
The measure of a geometric angle is a number given as follows.
\begin{itemize}
\item The magnitude the measure of a geometric angle is the length of shorter arc of a unit circle cut off from a unit circle by the angle.  
\item If when traversing the arc from the initial arm to the terminal we move clockwise along the circle, the measure is taken with positive sign; else it is taken with negative sign.
\end{itemize}
\end{definition}

\end{columns}
\end{frame}