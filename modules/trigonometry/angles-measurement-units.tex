% begin module angles
\begin{frame}
\frametitle{Degrees and radians}
\begin{itemize}
\item Degrees is a unit for measuring angles, denoted by ${}^\circ$. 
\item<2-> The relationship between degrees and radians is:

\hfil \hfil $\renewcommand{\arraystretch}{1.4}
\begin{array}{rcl}
\displaystyle\alertNoH{3}{ \alertNoH{5}{\pi} \text{ rad}}&=&\displaystyle \alertNoH{4}{{\alertNoH{6}{180}}^{\circ}}\\
\uncover<5->{\displaystyle 1 \text{rad}&=&\displaystyle \frac{180^{\circ}}{\alertNoH{5}{\pi}}\approx 57.3^\circ}\\
\displaystyle \uncover<6->{ 1^{\circ} &=&\displaystyle \frac{\pi}{\alertNoH{6}{180}}\text{ rad}\approx 0.017\text{ rad}.}
\end{array}
$
\item<3-> In other words, \alertNoH{3,4}{a half-turn} is measured by $\alertNoH{3}{\pi \text{rad}}$ or $\alertNoH{4}{180^{\circ}}$.
\item<7-> Degrees are useful because the most frequently encountered fractions of a half turn are measured by a whole number of degrees.
\item<8-> If a measurement unit is not specified, it is implied to be radians. For example, in $\sin 5$, the number $5$ stands for $5$ radians.
\end{itemize}
\end{frame}



% end module angles
