\begin{frame}
\begin{center}
\psset{xunit=2cm, yunit=2cm}
\begin{pspicture}(-1.4,-1.4)(1.6,1.4)
\tiny
\fcAxesStandard{-1.4}{-1.4}{1.6}{1.4}
\fcLabels{1.6}{1.3}
\parametricplot{0}{360}{t cos t sin}
%\newcommand{\rayFormat}{}
\newcommand{\theLine}[1]{%
\psline[linecolor=blue](! ####1 cos 0.9 mul ####1 sin 0.9 mul)(! ####1 cos 1.1 mul ####1 sin 1.1 mul)%
\psline[linestyle=dashed, linewidth=0.5pt, linecolor=blue](0,0)(! ####1 cos 1 mul ####1 sin 1 mul)%
}%
\theLine{30}%
\theLine{45}%
\theLine{60}%
\theLine{90}%
\theLine{120}%
\theLine{135}%
\theLine{150}%
\theLine{180}%
\theLine{210}%
\theLine{225}%
\theLine{240}%
\theLine{270}%
\theLine{300}%
\theLine{315}%
\theLine{330}%
\theLine{360}%
\rput[l](! 30 cos 30 sin){$~~~~\frac{\pi}{6} = 30^\circ$}%
\rput[l](! 45 cos 45 sin){$~~~~\frac{\pi}{4} = 45^\circ$}%
\rput[bl](! 60 cos 60 sin 0.1 add){$\frac{\pi}{3} = 60^\circ$}%
\rput[bl](0,1.02){$\frac{\pi}{2} = 90^\circ$}%
\rput[br](! 120 cos 120 sin 0.1 add){$\frac{2\pi}{3} = 120^\circ$}%
\rput[br](! 135 cos 135 sin 0.1 add){$\frac{3\pi}{4} = 135^\circ$}%
\rput[br](! 150 cos 150 sin 0.1 add){$\frac{5\pi}{6} = 150^\circ$}%
\rput[br](! -1 0.05){$\pi = 180^\circ$}%
\rput[tr](! 210 cos 210 sin 0.1 sub){$\frac{7\pi}{6} = 210^\circ$}%
\rput[tr](! 225 cos 225 sin 0.1 sub){$\frac{5\pi}{4} = 225^\circ$}%
\rput[tr](! 240 cos 240 sin 0.1 sub){$\frac{4\pi}{3} = 240^\circ$}%
\rput[tr](! 0 -1.1){$\frac{3\pi}{2} = 270^\circ$}%
\rput[tl](! 300 cos 300 sin 0.1 sub){$\frac{5}{3}\pi = 300^\circ$}%
\rput[tl](! 315 cos 315 sin 0.1 sub){$\frac{7\pi}{4} = 315^\circ$}%
\rput[tl](! 330 cos 330 sin 0.1 sub){$\frac{11\pi}{6} = 330^\circ$}%
\rput[tl](! 1 -0.1){$2\pi = 360^\circ$}%
\rput[lb](1.2, 1.2){Quadrant I}
\rput[rb](-1.2,1.2){Quadrant II}
\rput[rt](-1.2,-1.2){Quadrant III}
\rput[lt](1.2,-1.2){Quadrant IV}
\end{pspicture}
\end{center}


The most frequently encountered angles are given in the table below.
\[
\begin{array}{|c@{ \ }|c@{ \ }|c@{ \ }|c@{ \ }|c@{ \ }|c@{ \ }|c@{ \ }|c@{ \ }|c@{ \ }|c@{ \ }|c@{ \ }|c@{ \ }|}
\hline
\text{Deg.} &
 0^\circ &
30^\circ &
45^\circ &
60^\circ &
90^\circ &
120^\circ &
135^\circ &
150^\circ &
180^\circ &
270^\circ &
360^\circ \\
\hline
\textrm{Rad.} &
0 &
\displaystyle \frac{ \pi}{6} &
\displaystyle \frac{ \pi}{4} &
\displaystyle \frac{ \pi}{3} &
\displaystyle \frac{ \pi}{2} &
\displaystyle \frac{2\pi}{3} &
\displaystyle \frac{3\pi}{4} &
\displaystyle \frac{5\pi}{6} &
\displaystyle \pi &
\displaystyle \frac{3\pi}{2} &
\displaystyle 2\pi \\
\hline
\end{array}
\]


\end{frame}

\begin{frame}
\begin{center}
\psset{xunit=2cm, yunit=2cm}
\begin{pspicture}(-1.4,-1.4)(1.4,1.4)
\tiny
\fcAxesStandard{-1.3}{-1.3}{1.6}{1.3}
\fcLabels{1.6}{1.3}
\parametricplot{0}{360}{t cos t sin}
%\newcommand{\rayFormat}{}
\newcommand{\theLine}[1]{%
\psline[linecolor=blue](! ####1 cos 0.9 mul ####1 sin 0.9 mul)(! ####1 cos 1.1 mul ####1 sin 1.1 mul)%
\psline[linestyle=dashed, linewidth=0.5pt, linecolor=blue](0,0)(! ####1 cos 1 mul ####1 sin 1 mul)%
}%
\theLine{57.295779513 1 mul}%
\theLine{57.295779513 2 mul}%
\theLine{57.295779513 3 mul}%
\theLine{57.295779513 4 mul}%
\theLine{57.295779513 5 mul}%
\theLine{57.295779513 6 mul}%
\theLine{57.295779513 7 mul}%
\theLine{57.295779513 8 mul}%
\rput[l](! 57.295779513 1 mul dup cos exch sin 0.1 add){$1 \text{ rad} \approx 57.3^\circ$}%
\rput[r](! 57.295779513 2 mul dup cos exch sin 0.1 add){$2 \text{ rad} \approx 114.6^\circ$}%
\rput[r](! 57.295779513 3 mul dup cos exch sin 0.1 add){$3 \text{ rad} \approx 171.9^\circ$}%
\rput[t](! 57.295779513 4 mul dup cos exch sin -0.1 add){$4 \text{ rad} \approx 229.2^\circ$}%
\rput[t](! 57.295779513 5 mul dup cos exch sin -0.1 add){$5 \text{ rad} \approx 286.5^\circ$}%
\rput[tl](! 57.295779513 6 mul dup cos exch sin){$6 \text{ rad} \approx 343.8^\circ$}%
\rput[l](! 57.295779513 7 mul dup cos exch sin){$7 \text{ rad} \approx 401.1^\circ$}%
\rput[b](! 57.295779513 8 mul dup cos exch sin 0.1 add){$8 \text{ rad} \approx 458.4^\circ$}%
\end{pspicture}
\end{center}
\begin{itemize}
\item Integer quantities of radians are not rational multiples of the angle measure of a half-turn and therefore are not easy to compute with. 
\item For example to determine in which quadrant is an angle of $k$ radians located one needs to know the numerical value of $\frac{k}{\pi}$, which requires knowledge of $\pi$ with great numerical accuracy.
\end{itemize}
\end{frame}