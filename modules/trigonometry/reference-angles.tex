\begin{frame}

\uncover<2->{
The computation of the reference angle $\alpha$ depends on the quadrant.

\begin{tabular}{|c|c|c|}\hline

\begin{pspicture}(-1.4,-1.4)(1.4,1.4)
\tiny
\rput[lt](-1.35,1.35){Quadrant II}
\fcAxesStandardNoFrame{-1.37}{-1.37}{1.37}{1.37}
\parametricplot[arrows=->, linecolor=red]{0}{120}{t cos 0.3 mul t sin 0.3 mul}
\parametricplot[arrows=->, linecolor=red]{120}{180}{t cos 0.35 mul t sin 0.35 mul}
\psline[arrows=->](0,0)(! 120 cos 1.3 mul 120 sin 1.3 mul)
\rput[lb](0.3,0.2){$\theta$}
\rput[rb](-0.3,0.2){$\alpha$}
\end{pspicture}
&
\begin{pspicture}(-1.4,-1.4)(1.4,1.4)
\tiny
\rput[lt](-1.35,1.35){Quadrant III}
\fcAxesStandardNoFrame{-1.37}{-1.37}{1.37}{1.37}
\parametricplot[arrows=->, linecolor=red]{0}{230}{t cos 0.3 mul t sin 0.3 mul}
\parametricplot[arrows=->, linecolor=red]{180}{230}{t cos 0.4 mul t sin 0.4 mul}
\psline[arrows=->](0,0)(! 230 cos 1.3 mul 230 sin 1.3 mul)
\rput[lb](0.3,0.2){$\theta$}
\rput[rt](-0.4,-0.2){$\alpha$}
\end{pspicture}
&
\begin{pspicture}(-1.4,-1.4)(1.4,1.4)
\tiny
\rput[lt](-1.35,1.35){Quadrant IV}
\fcAxesStandardNoFrame{-1.37}{-1.37}{1.37}{1.37}
\parametricplot[arrows=->, linecolor=red]{0}{300}{t cos 0.3 mul t sin 0.3 mul}
\parametricplot[arrows=->, linecolor=red]{300}{360}{t cos 0.4 mul t sin 0.4 mul}
\psline[arrows=->](0,0)(! 300 cos 1.3 mul 300 sin 1.3 mul)
\rput[rb](-0.3,0.2){$\theta$}
\rput[lt](0.4,-0.2){$\alpha$}
\end{pspicture}
 \\
$\begin{array}{rcl}
\alertNoH{2,3}{\alpha}&\alertNoH{2,3}{=}&\fcAnswer{3}{\pi-\theta} \\ \alertNoH{2,3}{\alpha}&\alertNoH{2,3}{=}&\fcAnswer{3}{180^\circ-\theta}\end{array}$&
$\begin{array}{rcl}
\alertNoH{4,5}{\alpha}&\alertNoH{4,5}{=}&\fcAnswer{5}{\theta-\pi}\\ 
\alertNoH{4,5}{\alpha}&\alertNoH{4,5}{=}&\fcAnswer{5}{\theta-180^\circ} \end{array}$&
$\begin{array}{rcl}
\alertNoH{6,7}{\alpha}&\alertNoH{6,7}{=}&\fcAnswer{7}{2\pi-\theta}\\ 
\alertNoH{6,7}{\alpha}&\alertNoH{6,7}{=}&\fcAnswer{7}{360^\circ-\theta}\end{array}$\\\hline

\end{tabular}
}

\only<handout:1|1-7>{
To compute trigonometric functions from obtuse ($>90^\circ$) or negative angles, we can use the following visual aid. 
\begin{definition}[Reference Angle]
Let $\theta$ be an angle in standard position. Its reference angle is the acute positive angle formed by the terminal arm and the $x$ axis. 
\end{definition}
}
\only<handout:2|8->{
\begin{observation}
One can find the value of a trigonometric function of $\theta$ as follows.
\begin{itemize}
\item Find the reference angle $\alpha$ associated to $\theta$.
\item Find the trig function of $\alpha$.
\item Use the quadrant in which $\theta$ lies to affix an appropriate sign to the function value.
\end{itemize}
\end{observation}
}

\vskip 10cm
\end{frame}
