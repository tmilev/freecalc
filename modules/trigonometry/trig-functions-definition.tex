\begin{frame}
\frametitle{Definition of the trigonometric functions}
\vskip -0.2cm
\begin{columns}
\column{0.2\textwidth}
%\psset{xunit=0.8cm, yunit=0.8cm}
\begin{pspicture}(-1.2,-1.2)(1.3, 1.3)%
\tiny%
%\fcBoundingBox{-1.2}{-1.3}{1.4}{1.4}%
\pstVerb{20 dict begin}%
\pstVerb{/startAngle 0 def /endAngle 60 def}%
\only<handout:0|2>{\pstVerb{/endAngle 240 def}}%
\only<handout:0|3>{\pstVerb{/endAngle 420 def}}%
\only<handout:0|4>{\pstVerb{/endAngle -60 def}}%
\only<handout:0|5>{\pstVerb{/endAngle -240 def}}%
\parametricplot[linecolor=\fcColorGraph]{0}{360}{t cos t sin}
\fcAxesStandardNoFrame{-1.2}{-1.3}{1.3}{1.34}%
\parametricplot[linecolor=blue, arrows=->]{startAngle}{endAngle }{t cos t sin}%
\rput[b](! startAngle endAngle add 2 div dup cos 1.1 mul exch sin 1.2 mul){$\theta$}%
\psline[arrows=->](0,0)(! endAngle cos 1.5 mul endAngle sin 1.5 mul)%
\uncover<3>{\parametricplot[linecolor=blue, arrows=->]{startAngle}{endAngle}{t cos 0.3 t 2000 div add mul  t sin 0.3 t 2000 div add mul}}%
\fcFullDot{0}{0}%
\fcFullDot{endAngle cos}{endAngle sin}%
\uncover<6->{\fcFullDot{endAngle cos}{endAngle sin}}%
\uncover<7->{%
\fcXTickWithLabel{endAngle cos}{$x$}%
\fcYTickWithLabel{endAngle sin}{$y$}%
\fcPerpendicular{[endAngle cos endAngle sin]}{[1 0]}{0.2}
}%
\only<handout:0|7,9,10,11>{\psline[linewidth=1.5pt, linecolor=green](! 0 0)(! endAngle cos 0)}%
\only<handout:0|8,9,10,12>{\psline[linewidth=1.5pt, linecolor=blue](! endAngle cos 0)(! endAngle cos endAngle sin)}%
\only<handout:0|11,12>{\psline[linewidth=1.5pt, linecolor=orange](0,0)(! endAngle cos endAngle sin)}%
\pstVerb{end}%
\end{pspicture}
\column{0.8\textwidth}
\begin{itemize}
\item For an angle-measure $\theta$ we selected geometric angle with initial arm on $x$ axis and terminal arm selected by traveling $\theta$ units on the unit circle.
\item<6-> Let $(x,y)$ be the intersection of the terminal arm of the geometric angle with the unit circle.
\end{itemize}

\end{columns}

\uncover<7->{\begin{definition}[$\sin$ and $\cos$]
The sine and cosine functions of the angle $\theta$, denoted by $\sin \theta$ and $\cos \theta$, are defined by

\hfil\hfil$
\alertNoH{7}{\cos \theta=x} \qquad \qquad  \alertNoH{8}{\sin \theta=y}.
$

\end{definition}
}
\uncover<9->{
\begin{definition}[additional trigonometric functions]
The functions \alertNoH{9}{tangent}, \alertNoH{10}{cotangent}, \alertNoH{11}{secant} and \alertNoH{12}{cosecant} of the angle $\theta$, denoted by $\tan \theta$, $\cot \theta$, $\sec \theta$, $\csc \theta$, are defined by 

\noindent \mbox{$\displaystyle \!\!
\alertNoH{9}{\tan \theta=}\frac{\only<handout:0|9>{\color{blue}} \sin \theta}{\only<handout:0|9>{\color{green}} \cos \theta} \qquad \alertNoH{10}{\cot\theta=}\frac{\only<handout:0|10>{\color{green}} \cos \theta}{\only<handout:0|10>{\color{blue}} \sin\theta}\qquad \alertNoH{11}{\sec \theta=}\frac{\only<handout:0|11>{\color{orange}}1}{\only<handout:0|11>{\color{green}} \cos \theta} \quad~~~ \alertNoH{12}{\csc \theta=} \frac{\only<handout:0|12>{\color{orange}}1}{\only<handout:0|12>{\color{blue}} \sin\theta}.
$}
\end{definition}
}
\end{frame}

