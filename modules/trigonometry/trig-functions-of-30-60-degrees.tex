\begin{frame}
\vskip -0.1cm
\begin{example}
\begin{columns}
\column{0.25\textwidth}
\psset{xunit=1.3cm, yunit=1.3cm}
\uncover<2->{
\begin{pspicture}(-0.1,-0.25)(2.2,1.95)
\tiny
\fcBoundingBox{-0.1}{-0.25}{2.2}{2}
\only<handout:0|4,7,21,22,23>{\renewcommand{\fcAngleLineWidth}{2pt}}
\fcPerpendicular{[1 3 sqrt ]}{[1 0]}{0.1}
\parametricplot[linecolor=red]{0}{60}{t cos 0.3 mul t sin 0.3 mul}
\uncover<handout:0|3,7,13,16,28,29,30,31,32,33>{\parametricplot[linecolor=red, linewidth=2pt]{0}{60}{t cos 0.3 mul t sin 0.3 mul}
}
\uncover<13->{\parametricplot[linecolor=red]{120}{180}{t cos 0.3 mul 2 add t sin 0.3 mul}}
\uncover<handout:0|13,16>{\parametricplot[linecolor=red, linewidth=2pt]{120}{180}{t cos 0.3 mul 2 add t sin 0.3 mul}}
\parametricplot[linecolor=red]{-90}{-120}{t cos 0.3 mul 1 add t sin 0.3 mul 3 sqrt add}
\uncover<handout:0|5,7,14,16, 34-39>{
\parametricplot[linecolor=red, linewidth=2pt]{-90}{-120}{t cos 0.3 mul 1 add t sin 0.3 mul 3 sqrt add}
}
\uncover<14->{\parametricplot[linecolor=purple]{-90}{-60}{t cos 0.3 mul 1 add t sin 0.3 mul 3 sqrt add}}
\uncover<handout:0|14,16>{\parametricplot[linecolor=purple, linewidth=2pt]{-90}{-60}{t cos 0.3 mul 1 add t sin 0.3 mul 3 sqrt add}}
\uncover<2->{\psline(0,0)(1, 0)(! 1 3 sqrt )(0,0)}
\uncover<handout:0|2,6>{\psline[linewidth=2pt, linecolor=red](0,0)(1, 0)(! 1 3 sqrt )(0,0)(1,0)}
\uncover<handout:0|12>{\psline[linewidth=2pt, linecolor=blue](0,0)(1, 0)(! 1 3 sqrt )(0,0)(1,0)}
\uncover<11->{\psline[linestyle=dashed](1, 0)(2,0)(! 1 3 sqrt )}
\uncover<handout:0|11,12>{\psline[linecolor=red, linewidth=2pt, linestyle=dashed](1, 0)(2,0)(! 1 3 sqrt )(1,0)}
\rput[t](0.5, -0.1){$\alertNoH{15,18,19,25,31,33,35,39}{1}$}
\uncover<11->{\rput[t](1.5, -0.1){$\uncover<15->{\alertNoH{15,18,19}{1}}$}}
\uncover<handout:0|15,35,39>{
\psline[linecolor=blue, linewidth=2pt](0,0)(1,0)
}
\uncover<handout:0|15>{
\psline[linecolor=red, linewidth=2pt](1,0)(2,0)
}
\uncover<handout:0|16>{
\psline[linecolor=blue, linewidth=2pt](0,0)(2,0)(! 1 3 sqrt )(0,0)(2,0)
}
\uncover<handout:0|17,18,19,23,25>{
\psline[linecolor=purple, linewidth=2pt](0,0)(1,0)
}
\uncover<handout:0|17>{
\psline[linecolor=purple, linewidth=2pt](1,0)(2,0)
}
\uncover<handout:0|17,18,19,22,23,24>{
\psline[linecolor=blue, linewidth=2pt](0,0)(! 1 3 sqrt)
}
\uncover<handout:0|29,31,35,37>{
\psline[linecolor=orange, linewidth=2pt](0,0)(! 1 3 sqrt)
}
\uncover<handout:0|18,19>{
\psline[linecolor=blue, linewidth=2pt](1,0)(2,0)
}
\uncover<handout:0|20,21,22,23,27>{
\psline[linewidth=2pt, linecolor=red](1,0)(! 1 3 sqrt)
}
\uncover<handout:0|29,33>{
\psline[linewidth=2pt, linecolor=blue](1,0)(! 1 3 sqrt)
}
\uncover<handout:0|37,39>{
\psline[linewidth=2pt, linecolor=green](1,0)(! 1 3 sqrt)
}
\uncover<handout:0|31,33>{
\psline[linewidth=2pt, linecolor=green](0,0)(1,0)
}
\rput[br](0.4, 0.8){$\uncover<19->{\alertNoH{19,24,29,31,35,37}{2}}$}
\rput[bl](1.6, 0.8){$\uncover<19->{\alertNoH{19}{2}}$}
\rput[l](0.3, 0.2){$\alertNoH{3,7,13,16,28,29,30,31,32,33}{60^\circ}$}
\rput[r](1.7, 0.2){$\uncover<13->{\alertNoH{13,16}{60^\circ}}$}
\rput[br](0.94, 1.2){$\uncover<10->{\alertNoH{10,14,16,34- 39}{30^\circ=}} \alertNoH{5, 7, 14, 16, 34-39}{\gamma}$}
\rput[bl](1, 1.2){$\uncover<14->{~\alertNoH{14,16}{30^\circ}}$}
\rput[l](! 1 3 sqrt 2 div){$\uncover<26->{\alertNoH{26,27,29,33,37,39}{\sqrt{3}}}$}
\rput[tl](-0.1,-0.1){$A$}
\rput[t](1,-0.1){$H$}
\rput[b](! 1 3 sqrt 0.1 add){$C$}
\uncover<11->{\rput[tl](! 2 -0.1){$B$}}
\end{pspicture}
}
\column{0.75\textwidth}
\alertNoH{41}{
Find the values of $ \alertNoH{28,29}{\sin 60^{\circ}}, \alertNoH{30,31}{\cos 60^{\circ}}, \alertNoH{32,33}{\tan 60^{\circ}}$, $\alertNoH{34,35}{\sin 30^{\circ}}, \alertNoH{36,37}{\cos 30^{\circ}}, \alertNoH{38,39}{\tan 30^{\circ}}$.}

\uncover<2->{
Construct a right angled $\triangle AHC$ as indicated: \uncover<3->{angles $\alertNoH{3}{60^\circ}, \alertNoH{4}{90^\circ}, \alertNoH{5}{\gamma}$.} \uncover<6->{Angles \alertNoH{6}{in $\triangle$} \alertNoH{7}{sum to $180^\circ$}:}

\hfil \hfil$
\begin{array}{rcl}
\uncover<6->{\alertNoH{7}{ \alertNoH{8}{60^\circ+90^\circ +}\gamma} &\alertNoH{7}{{=}}& \alertNoH{7}{180^\circ} }\\
\uncover<8->{\alertNoH{14}{\gamma}&=&\alertNoH{9,10}{ 180^\circ\alertNoH{8}{ -90^\circ - 60^\circ} } \uncover<9->{\alertNoH{9,10,14}{=} } \fcAnswer{10}{\alertNoH{14}{ 30^\circ}} \uncover<10->{.}}
\end{array}
$
}
\end{columns}
\uncover<11->{Construct $\alertNoH{11}{\triangle HBC}$ as indicated so that $ \alertNoH{13}{ \alertNoH{11}{ \triangle HBC}\cong\alertNoH{12}{\triangle HAC}}$. \uncover<16->{$\triangle ABC$ has three equal angles ($=60^\circ$)} \uncover<17->{$\Rightarrow$ its sides are of equal length. Therefore }

\hfil\hfil $
\begin{array}{rcll|l}
\uncover<17->{\alertNoH{17}{|AC|}&\alertNoH{17}{{=}}&\alertNoH{17,18}{|AB|}\uncover<18->{\alertNoH{18}{=}\alertNoH{18,19}{1+1} \uncover<19->{\alertNoH{19}{=2}}} }\\
\uncover<20->{\alertNoH{20,21,27}{|CH|}&\alertNoH{20,21}{{=}}& \fcAnswer{21}{ \sqrt{|\alertNoH{22,24}{AC}|^2-|\alertNoH{23,25}{AH}|^2 }}  \uncover<21->{&&\alertNoH{21}{\text{Pythagorean theorem}}}} \\
\uncover<24->{&=&\alertNoH{26}{\sqrt{{\alertNoH{24}{2}}^2- {\alertNoH{25}{1}}^2}}}  \uncover<26->{\alertNoH{26,27}{=\sqrt{3}}}
\end{array}
$ 

$ 
\begin{array}{rcl@{\qquad}rcl@{\qquad}rcl}
\displaystyle \uncover<28->{ \alertNoH{28,29,41}{\sin 60^\circ} &\alertNoH{28,29,41}{{=}}& \displaystyle  \fcAnswer{29}{ \alertNoH{41}{ \frac{\sqrt{3}}{2}} } &\displaystyle  \alertNoH{30,31,41}{\cos 60^\circ }&\alertNoH{30,31,41}{{=}}&\displaystyle  \fcAnswerUncover{ 28 }{31}{\alertNoH{41}{ \frac{1}{2}} } &\displaystyle  \alertNoH{32,33,41}{\tan 60^\circ  }&\alertNoH{32, 33,41}{{=}}& \displaystyle \fcAnswerUncover{28}{33}{ \frac{\sqrt{ 3}} {1}=\alertNoH{41}{\sqrt{3}}}}\\
\displaystyle\uncover<34->{ \alertNoH{34,35,41}{ \sin 30^\circ} &\alertNoH{34,35,41}{{=}}&\displaystyle  \fcAnswerUncover{34}{35}{ \alertNoH{41}{\frac{1}{2}}} &\displaystyle  \alertNoH{36,37,41}{\cos 30^\circ} &\alertNoH{36,37,41}{ {=} }&\displaystyle  \fcAnswerUncover{34}{37}{\alertNoH{41}{\frac{ \sqrt{3}}{2}}} &\displaystyle  \alertNoH{38,39,41}{\tan 30^\circ }&\alertNoH{38,39, 41}{{= }}& \displaystyle \fcAnswerUncover{34}{39}{\alertNoH{40}{ \frac{ 1}{\sqrt{3}}} \uncover<40->{\alertNoH{40}{=\alertNoH{41}{\frac{\sqrt{3}}{3}}}} } \uncover<40->{.}}
\end{array}
$
}
\end{example}

\end{frame}

\begin{frame}
\begin{observation}
\begin{itemize}
\item If the hypotenuse of a right angle triangle is twice larger than one of the sides, then the angle opposite to that side is $30^\circ$. 
\item Conversely, in a right angle triangle with angle $30^\circ$, the hypotenuse is twice longer than the shorter of the two legs.
\end{itemize}
\end{observation}

\hfil \hfil\psset{xunit=4cm, yunit=4cm}
\begin{pspicture}(0,-0.1)(0.6,1.1)
\tiny
\fcPerpendicular{[0.5 3 sqrt 2 div]}{[0.5 0]}{0.1}
\parametricplot[linecolor=red]{0}{60}{t cos 0.1 mul t sin 0.1 mul}
\parametricplot[linecolor=red]{-90}{-120}{t cos 0.1 mul 0.5 add t sin 0.1 mul 3 sqrt 2 div add}
\psline(0,0)(0.5, 0)(! 0.5 3 sqrt 2 div)(0,0)
\rput[t](0.25, -0.05){$x$}
\rput[br](0.2, 0.4){$2x$}
\rput(0.14, 0.07){$60^\circ$}
\rput[tr](0.5, 0.68){$30^\circ$}
\end{pspicture}
\end{frame}