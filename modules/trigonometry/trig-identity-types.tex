\begin{frame}
\frametitle{Types of identites}
\begin{itemize}
\item In the present course we deal with two basic types of trigonometric identities.
\item First, identities that involve operations on the arguments of the trigonometric functions.
\begin{itemize}
\item Example: $\sin(\alpha+\beta)=\sin \alpha\cos \beta+ \cos\alpha\sin\beta $ (this is one of the angle sum identities); $\sin \theta+\sin(-\theta)=0$.
\item Such identities can be proved using the angle sum formulas and the even/odd function properties of $\sin,\cos$.
\end{itemize}
\item Second, identities that involve trigonometric functions of one variable.
\begin{itemize}
\item Example: $\tan^2\theta +1=\sec^2\theta$. 
\item Such identities can be proved only using the already demonstrated Pythagorean identity $\sin^2\theta +\cos^2\theta=1$. 
\end{itemize}
\item The Pythagorean identity follows from the angle sum formulas and the even/odd function properties of $\sin,\cos$, so all trigonometric identities follow from those properties alone.
\end{itemize}
\end{frame}