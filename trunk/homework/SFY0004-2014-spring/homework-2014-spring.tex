\documentclass[12pt]{article}
\usepackage{../homework-problems}
\usepackage{amsthm}
%\usepackage{etex} %avoiding error: too many packages. This is a LaTeX bug (``feature'')

\usepackage[english]{babel}
%\usepackage[latin1]{inputenc}
%\usepackage[all,cmtip]{xy}
\usepackage{times}
\usepackage{pstricks-add}
\usepackage{pst-math}
\usepackage{pst-node}
\usepackage{cancel}
\usepackage{tikz}


%%%%%%%%%%%%%%%%%%%%%%%%%%%%%%%%%%%%%%%%%%%%%%%%%%%%%%%%%
%
%  The following should be set to true if you want 
%  the solutions shown, and false otherwise. 
%  For the time being, this same tag is used 
%  for answer keys which are not solutions.
%
%%%%%%%%%%%%%%%%%%%%%%%%%%%%%%%%%%%%%%%%%%%%%%%%%%%%%%%%%

 \toggletrue{solutions}
 %\togglefalse{solutions}

 %\toggletrue{answers}
 \togglefalse{answers}


% The name of your course goes here. Default course name is Freecalc.
\renewcommand{\course}{SFY0004: Foundation Mathematics 2}
% The time when homework assignments are due goes here. Default due time is: (to be announced)
\renewcommand{\deadline}{10:00\\ (at the end of the lecture)}

\begin{comment}
\homeworkStart{1}{Friday February 7, 2014}
\item % begin homework chain-rule1
In each of the following cases find a simple function $u$ of $x$ such that the given function is a simple function of $u$.  
Use the Chain Rule to differentiate the given function with respect to $x$.   

\begin{enumerate}
\item   $y = \sqrt{1+x^2}$


\pointsii{3}  $y = (\cos x)^{1/2}$
\ans{%
\begin{align*}
\text{Let } \quad u & = \cos x. \\
\text{Then } \quad y & = u^{1/2}. \\
\text{Chain Rule: } \quad \frac{\diff y}{\diff x} & = \frac{\diff y}{\diff u}\frac{\diff u}{\diff x} \\
 & = \big(\frac{1}{2}u^{-1/2}\big) (-\sin x) \\
 & = -\frac{1}{2} \sin x (\cos x)^{-1/2}.
\end{align*}
}%

\item   $y = \sin^3 x$

\pointsii{3}  $y = (1+\cos x)^2$
\ans{%
\begin{align*}
\text{Let } \quad u & = 1+\cos x. \\
\text{Then } \quad y & = u^{2}. \\
\text{Chain Rule: } \quad \frac{\diff y}{\diff x} & = \frac{\diff y}{\diff u}\frac{\diff u}{\diff x} \\
 & = (2u) (-\sin x) \\
 & = -2 \sin x \cos x \\
 & = - \sin 2x. \quad \text{(This last step is optional.)}
\end{align*}
}%

\end{enumerate}
% end homework chain-rule1

\item Differentiate. The answer key has not been proofread, use with caution.
\begin{multicols}{3}
\begin{enumerate}[ref={\fcProblemRef}]
\item $\displaystyle f(x)=\sin (-5x)$. 

\answer{$ -5 \cos(-5 x)=-5\cos (5x)$}
\item $\displaystyle f(x)=\cot (2x)$. 

\answer{$-2\csc^2 (2x)$}
\item $\displaystyle f(x)=e^{-3x}$. 

\answer{$-3 e^{-3 x} $}
\item $\displaystyle f(x)=e^{\frac{1}x}$. 

\answer{$ -\frac{e^{\frac{1}{x}}}{x^2}$}
\item $\displaystyle f(x)=e^{\sqrt{x}}$. 

\answer{$ \frac{1}{2\sqrt{x}}e^{\sqrt{x}} $}
\item $\displaystyle f(x)=\ln (1+x) $

\answer{$\frac{1}{1+x} $}
\item $\displaystyle f(x)=\ln(1+x^3) $

\answer{$\frac{3x^2}{1+x^3}$}
\item $\displaystyle f(x)=\frac{1}{2}\ln\left(\frac{1+x}{1-x}\right) $

\answer{$\frac{1}{1-x^{2}}$}
\end{enumerate}
\end{multicols}

\item (Textbook, page 154, problems 7-46) Compute the derivative.
\begin{multicols}{2}
\begin{enumerate}
\item $\displaystyle f(x)= (x^4+3x^2-2)^5$.

\answer{$ (30 x +20 x^{3}) (-2+3 x^{2}+x^{4})^{4}$}
\item $\displaystyle f(x)= (4x-x^2)^{100}$.

\answer{$(-200 x+400) (4 x- x^{2})^{99}$}
\item $\displaystyle f(x)= \sqrt{1-2x}$.

\answer{$- (1-2 x)^{-\frac{1}{2}}$}
\item $\displaystyle f(x)= \frac{1}{(1+\sec x)^2}$.

\answer{$\frac{-2 \cos{}(x) \sin{}(x)}{(1+\cos{}(x))^{3}} =\frac{- \sin{}(2x)} {(1+\cos{}(x))^{3}} $}
\item $\displaystyle f(z)=\frac{1}{1+z^2} $.

\answer{$\frac{-2 z}{(1+z^{2})^{2}} $}
\item $\displaystyle f(t)= \sqrt[3]{1+\tan t}$.

\answer{$\frac{\frac{1}{3}}{(\cos{}(t))^{2}} \left( \frac{\cos{}(t) +\sin{}(t)}{\cos{}(t)} \right)^{-\frac{2}{3}} $}
\item $\displaystyle f(x)=\cos (a^3+x^3) $.

\answer{$ -3 x^{2}\sin{}(a^{3}+x^{3}) $}
\item $\displaystyle f(x)= a^3+\cos^3 x$.

\answer{$ -3 \sin{}(x) (\cos{}(x))^{2}$}
\item $\displaystyle f(x)= x\sec (k x) $.

\answer{$\frac{\cos{}(k x)+k x \sin{}(k x) }{(\cos{}(k x))^{2}} $}
\item $\displaystyle f(\theta)= 3\cot (n\theta)$.

\answer{$ \frac{-3 n}{(\sin{}(n \theta))^{2}}$}
\item $\displaystyle f(x)= (2x - 3)^4 (x^2 + x + 1)^5$.

\answer{$ (-7-12 x+28 x^{2})(-3+2 x)^{3} (1+x+x^{2})^{4}$}
\item $\displaystyle f(x)= (x^2+1)^3(x^2+2)^6$.

\answer{$\left(24 x+18 x^{3}\right)\left(1+x^{2}\right)^{2} \left(2+x^{2}\right)^{5} $}
\item $\displaystyle f(t)= (t+1)^{\frac{2}{3}}(2t^2-1)^3$.

\answer{$ \left(\frac{40}{3} t^{2}+12 t-\frac{2}{3}\right)\left(2 t^{2}-1\right)^{2}\left(t+1\right)^{-\frac{1}{3}}$}
\item $\displaystyle f(t)= (3t-1)^4(2t+1)^{-3}$.

\answer{$(3 t-1)^{3}\frac{6 t+18}{(2 t+1)^{4}}$}
\item $\displaystyle f(x)=\left(\frac{x^2+1}{x^2-1} \right)^3 $.

\answer{$\frac{-12 x}{\left(x^{2}-1\right)^{2}} \left(\frac{x^{2}+1}{x^{2}-1}\right)^{2} $}
\item $\displaystyle f(s)= \sqrt{\frac{s^2+1}{s^2+4}}$.

\answer{$\frac{3 s}{\left(s^{2}+4\right)^{2}} \left(\frac{s^{2}+1}{s^{2}+4}\right)^{-\frac{1}{2}} $}
\item $\displaystyle f(x)=\sin (x\cos x) $.

\answer{$\cos{}(x) \cos{}(x \cos{}(x))- x \cos{}(x \cos{}(x)) \sin{}(x) $}
\item $\displaystyle f(x)=\frac{x}{\sqrt{7-3x}} $.

\answer{$ \frac{-\frac{3}{2} x+7}{(-3 x+7)^{\frac{3}{2}}}$}
\item $\displaystyle f(z)=\sqrt{\frac{z-1}{z+1}} $.

\answer{$\frac{1}{(z+1)^{2}} \left(\frac{z-1}{z+1}\right)^{ -\frac{1}{2}} $}
\item $\displaystyle f(y)= \frac{(y-1)^4 }{(y^2+2y)^5}$.

\answer{$(y-1)^{3}\frac{-6y^{2}+8y +10}{(y^{2}+2 y)^{6}} $}
\item $\displaystyle f(r)=\frac{r}{\sqrt{r^2+1}} $.

\answer{$ \frac{1}{\left(r^{2}+1\right)^{\frac{3}{2}}}$}
\item $\displaystyle f(x)=\frac{\cos (\pi x)}{\sin (\pi x)+\cos (\pi x) } $.

\answer{$ \frac{- \pi}{(\sin{}(\pi x)+\cos{}(\pi x))^{2}}$}
\item $\displaystyle f(x)=\sin \left(\sqrt{1+x^2}\right) $.

\answer{$x \cos{}\left(\left(x^{2}+1\right)^{\frac{1}{2}}\right) \left(x^{2}+1\right)^{-\frac{1}{2}}$}
\item $\displaystyle f(v)=\left(\frac{v}{v^3+1}\right)^6 $.

\answer{$\frac{-12 v^{3}+6}{\left(v^{3}+1\right)^{2}} \left(\frac{v}{v^{3}+1}\right)^{5} $}
\end{enumerate}
\end{multicols}


\homeworkEnd
\end{comment}

\homeworkStart{2}{Friday February 28, 2014}
\item % begin homework exponent-simplfy
Express each of the following as a single power.  

\begin{enumerate}[ref={\fcProblemRef}]
\item   $\displaystyle\frac{2^5\cdot 2^7}{2\sqrt{2}}$

\answer{$2^{10.5}=2^{\frac{21}{2}}$}
\pointsii{2}  $\label{problemSimplify3^23^(-1)/(3^3sqrt(3^3))} \displaystyle\frac{3^2\cdot 3^{-1}}{3^3\cdot \sqrt{3^3}}$

\answer{$3^{-\frac{7}{2}}$}
%solution to this problem moved to separate file
\item   $\displaystyle \frac{\pi^3}{\pi^{-1}\sqrt{\pi^5}}$

\answer{ $\pi^{\frac{3}{2}}$}
\end{enumerate}
% end homework exponent-simplify

\item % begin homework logarithms-combine
Express each of the following as a single logarithm.  

\begin{enumerate}
\item   $\ln 4 + \ln 6 - \ln 5$

\pointsii{2} $2\ln 2 - 3\ln 3 + 4\ln 4$

\solution{%
\begin{align*}
2\ln 2 - 3\ln 3 + 4\ln 4 & = \ln 2^2 - \ln 3^3 + \ln 4^4 \\
 & = \ln 4 - \ln 27 + \ln 256 \\
 & = \ln \Big( \frac{4}{27}\Big) + \ln 256 \\
 & = \ln \Big( \frac{4\cdot 256}{27}\Big) \\
 & = \ln \Big( \frac{1024}{27}\Big).
\end{align*}
}%

\item   $\ln 36 - 2\ln 3 - 3\ln 2$

\end{enumerate}
% end homework logarithms-combine

\item % begin homework logarithms-basic2
Use the definition of a logarithm to evaluate each of the following without using a calculator.  

\begin{enumerate}
\item   $\log_2 16$

\item   $\log_3 (1/9)$

\item   $\log_{10} 1000$

\item   $\log_{6} 36^{-2/3}$

\item   $\log_{2} (8\sqrt{2})$

\item $\log_7(49^x/343^y)$

\solution{%
\begin{align*}
\log_7(49^x/343^y) & = \log_749^x - \log_7343^y \\
 & = x\log_749 - y\log_7343 \\
\intertext{But $49 = 7^2$ and $343=7^3$, therefore}
\log_7(49^x/343^y) & = 2x-3y.
\end{align*}
}%


\end{enumerate}
% end homework logarithms-basic2

\item % begin homework logarithms-equations2
Solve each equation for $x$.  
Then use a calculator to give an approximate answer in decimal notation.  
\begin{enumerate}
\item $\ln (3x-10)=2$.
\answer{$\frac{e^2+10}{3}\approx 5.796$}

\item $\ln (x^2-1)=3$.
\answer{$\pm \sqrt{e^3+1}\approx \pm 4.592$}

\item $e^{2x}-3e^x+2=0$.
\answer{$x=\ln 2\approx 0.693, ~~~, x=0$}

\item $2^{x-5}=3$.
\answer{$\log_2 3+5= \frac{\ln 3}{\ln 2}+5 \approx 6.585$}

\pointsii{5}  $\ln x+\ln (x-1)=1$.
\hiddenanswer{$\frac{1}{2}\left(1+\sqrt{1+4e}\right)\approx 2.223$}

\solution{%
\begin{align*}
\ln x + \ln (x-1) & = 1 \\
\ln (x^2-x) & = 1 \\
e^{\ln (x^2-x)} & = e^1 \\
x^2-x & = e \\
x^2-x-e & = 0 \\
\text{Quadratic formula:}\quad x & = \frac{-(-1)\pm \sqrt{(-1)^2-4(1)(-e)}}{2(1)} \\
 & = \frac{1\pm \sqrt{1+4e}}{2}.
\end{align*}
But $\frac{1-\sqrt{1+4e}}{2}$ is negative, so $\ln\frac{1-\sqrt{1+4e}}{2}$ is undefined.  
Hence the only solution is $x = \frac{1+\sqrt{1+4e}}{2}\approx 2.2229$.  
}%

\item $e^{3x+1}=k$.
\answer{$\frac{\ln k-1}{3}$}

\item $e- e^{-2x}=1$.
\answer{$-\frac12\ln (e-1)\approx -0.271$}

\item $10(1+e^{-x})^{-1}=3$.
\answer{$-\ln \frac73 =\ln \frac37 \approx -0.847$}

\item $\ln (\ln x)=1$.
\answer{$e^e\approx 15.154$}

\item $e^{2x}-e^x-6=0$.
\answer{$x=\ln 3$}


\end{enumerate}
% end homework logarithms-equations2

\item % begin homework inverse-functions2
Find the inverse function and its domain. 
\begin{enumerate}
\item  $y=\ln (x+3)$.
\answer{$f^{-1}(x)=e^x-3$}

%\solution{%
%\begin{align*}
%y & = \ln (x+3) \\
%e^y & = e^{\ln (x+3)} \\
%e^y & = x + 3 \\
%e^y - 3 & = x \\
%\text{Therefore} \quad f^{-1}(y) & = e^y - 3.
%\end{align*}
%The domain of $e^y$ is all real numbers, so the domain of $f^{-1}$ is all real numbers.  
%}%

\item $f(x)=e^{x^3}$.
\answer{$f^{-1}(x)=\sqrt[3]{\ln x}, \quad x>0$}

\item $y=(\ln x)^2$, $x\geq 1$.
\answer{$f^{-1}(x)=e^{\sqrt{x}}, \quad x\geq 0 $}

\pointsii{5}  $y=\frac{e^x}{1+2e^x}$.
\hiddenanswer{$f^{-1}(x)= \ln \left(\frac{x}{1-2x}\right) $, \quad $x\in (0, \frac12) $}

\solution{%
\begin{align*}
y & = \frac{e^x}{1+2e^x} \\
y(1+2e^x) & = e^x \\
y & = e^x(1-2y) \\
\frac{y}{1-2y} & = e^x \\
\ln\frac{y}{1-2y} & = \ln e^x \\
\ln\frac{y}{1-2y} & = x \\
\text{Therefore} \quad f^{-1}(y) & = \ln\frac{y}{1-2y}.
\end{align*}
The natural logarithm function is only defined for positive input values.  
Therefore the domain is the set of all $y$ for which 
\begin{align*}
\frac{y}{1-2y} & > 0.
\end{align*}
This inequality holds if the numerator and denominator are both positive or both negative.  
This happens if either
\begin{enumerate}
\item  $y > 0$ and $y < 1/2$, or 
\item  $y < 0$ and $y > 1/2$.
\end{enumerate}
The latter option is impossible, so the domain is $\{ y \in \mathbb{R} \ | \ 0 < y < 1/2\}$.  
}%

\end{enumerate}
% end homework inverse-functions2

\item % begin homework inverse-functions3
Find the inverse function. You are asked to do the algebra only; you are not asked to determine the domain or range of the function or its inverse. 
\begin{enumerate} [ref={\fcProblemRef}]
\item $f(x)= 3x^2+4x-7$, where $x\geq -\frac{2}{3}$.

\answer{$f^{-1}(x)= -\frac{2}3+\frac{\sqrt{25+3x}}{3}, \quad x\geq -\frac{25}{3}$}
\item $f(x)= 2x^2+3x-5$, where $x\geq -\frac{3}{4}$.

\answer{$f^{-1}(x)=-\frac{3}{4}+\frac{\sqrt{49+8x}}{4}, \quad x\geq -\frac{49}{8}$}
\item $\displaystyle f(x)= \frac{2x+5}{x-4}$, where $x\neq 4$.

\answer{$f^{-1}(x)=\frac{4x+5}{x-2}, \quad x\neq 2$}
\pointsii{3} \label{problemFindInversef=(3x+5)/(2x-4)} $\displaystyle f(x)= \frac{3x+5}{2x-4}$, where $x\neq 2$.

\hiddenanswer{$\displaystyle f^{-1}(x) = \frac{ 4 x +5}{2x-3}, \quad x\neq \frac{3}{2}$}
\item \label{problemFindIversef=(5x+6)/(4x+5)}  $\displaystyle f(x)= \frac{5x+6}{4x+5}$.

\answer{$f^{-1}(x)= \frac{-5x+6}{4x-5}$, $x\neq \frac{5}{4}$}
\item  $\displaystyle f(x)= \frac{2x-3}{-3x+4}$.

\answer{$f^{-1}(x)=\frac{4x+3}{3x+2}  $, $x\neq -\frac{2}{3}$}
\end{enumerate}
% end homework inverse-functions3

\item % begin homework exponent-derivative
Differentiate each function.  

\begin{enumerate}
\item   $\displaystyle f(x) = \frac{e^x}{1+2e^x}$.  

\pointsii{3}  $r(t) = Ae^{-kt^2}$, where $A$ and $k$ are unknown constants.  

\solution{%
\begin{align*}
r & = Ae^{-kt^2}. \\
\text{Let}\quad u & = -kt^2. \\
\text{Then}\quad r & = Ae^u. \\
\text{Chain Rule}\quad \frac{\diff r}{\diff t} & = \frac{\diff r}{\diff u}\frac{\diff u}{\diff t} \\
 & = (Ae^u)(-2kt) \\
 & = -2Akte^{-kt^2}.
\end{align*}
}%

\item   $y = \frac{e^x}{2}(\sin x + \cos x)$.  
\end{enumerate}
% end homework exponent-derivative

\homeworkEnd

\begin{comment}
\homeworkStart{3}{Friday March 21, 2014}
\points{5} % begin homework implicit-inverse-trig1
The variables $x$ and $y$ are related by
\[
x\arctan y + y\arctan x = \frac{\pi}{2}.
\]

\begin{enumerate}
\item   Show that $(1,1)$ is on the graph of this relation.  

\solution{%
\begin{align*}
\text{LS} & = 1\cdot \arctan 1 + 1\cdot \arctan 1 & \text{RS} & = \frac{\pi}{2}. \\
 & = 1\cdot \frac{\pi}{4} + 1\cdot \frac{\pi}{4}  & & \\
 & = \frac{\pi}{2}.  & & 
\end{align*}
The fact that the left side equals the right side when we plug in $x = 1$ and $y = 1$ means that the point $(1,1)$ is on the graph of the relation.  
}%

\item   Find $\frac{\diff y}{\diff x}$ in terms of $x$ and $y$.  

\solution{%
Differentiate implicitly.
\begin{align*}
\big((x)\frac{\diff}{\diff x}(\arctan y) + (\arctan y)\frac{\diff}{\diff x}(x)\big)  + \big((y)\frac{\diff}{\diff x}(\arctan x) + (\arctan x)\frac{\diff}{\diff x}(y)\big)  & = 0 \\
x\cdot \frac{1}{1+y^2}\cdot \frac{\diff y}{\diff x} + (\arctan y)1 + y\cdot \frac{1}{1+x^2} + (\arctan x)\frac{\diff y}{\diff x} & = 0 \\
\frac{x}{1+y^2}\frac{\diff y}{\diff x} + \arctan y + \frac{y}{1+x^2} + \frac{\diff y}{\diff x}\arctan x & = 0.
\end{align*}
Rearrange to isolate $\frac{\diff y}{\diff x}$ on one side.  
\begin{align*}
\frac{x}{1+y^2}\frac{\diff y}{\diff x} +  \frac{\diff y}{\diff x}\arctan x & = - \frac{y}{1+x^2} - \arctan y \\
\big(\frac{x}{1+y^2} + \arctan x\big)\frac{\diff y}{\diff x} & = - \big(\frac{y}{1+x^2} + \arctan y\big) \\
\frac{\diff y}{\diff x} & = -\frac{\frac{y}{1+x^2}+\arctan y}{\frac{x}{1+y^2}+\arctan x}. 
\end{align*}

}%

\item   Find the equation of the tangent to the graph at $(1,1)$.  

\solution{%
To find the slope of the tangent, plug in $x=1,y=1$ to the formula for $\frac{\diff y}{\diff x}$.  

\begin{align*}
\frac{\diff y}{\diff x} & = -\frac{\frac{1}{1+1^2}+\arctan 1}{\frac{1}{1+1^2}+\arctan 1} \\
& = -1.
\end{align*}

Now use the point $(1,1)$ to find an equation for the tangent line.  
\begin{align*}
y - 1 & = (-1)(x-1) \\
y & = -x +2.
\end{align*}
}%

\end{enumerate}
% end homework implicit-inverse-trig1

\points{5} % begin homework implicit-inverse-trig2
The variables $x$ and $y$ are related by
\[
x^2y+xy^2+\arcsin x = \frac{\pi}{6}.
\]

\begin{enumerate}
\item   Find all points on the graph of this relation for which $x = 1/2$.  

\solution{%
Set $x = 1/2$ and solve for $y$.  
\begin{align*}
\big(\frac{1}{2}\big)^2y+\frac{1}{2}y^2 + \arcsin \frac{1}{2} & = \frac{\pi}{6} \\
\frac{1}{4}y + \frac{1}{2}y^2 + \frac{\pi}{6} & = \frac{\pi}{6} \\
\frac{1}{4}y + \frac{1}{2}y^2  & = 0 \\
\frac{1}{4}y(1 + 2y)  & = 0,
\end{align*}
so $y = 0$ or $y = -1/2$.  
Therefore $(1/2,0)$ and $(1/2,-1/2)$ are the points on the graph of the relation for which $x = 1/2$.  
}%

\item   Find $\frac{\diff y}{\diff x}$ in terms of $x$ and $y$.  

\solution{%
Differentiate implicitly.
\begin{align*}
\big((x^2)\frac{\diff}{\diff x}(y) + (y)\frac{\diff}{\diff x}(x^2)\big) + \big( (x)\frac{\diff}{\diff x}(y^2) + (y^2)\frac{\diff}{\diff x}(x)\big) + \frac{1}{\sqrt{1-x^2}} & = 0 \\
x^2\frac{\diff y}{\diff x} + y(2x) + x(2y)\frac{\diff y}{\diff x} + y^2 + \frac{1}{\sqrt{1-x^2}} & = 0 \\
x^2\frac{\diff y}{\diff x} + 2xy + 2xy\frac{\diff y}{\diff x} + y^2 + \frac{1}{\sqrt{1-x^2}} & = 0.
\end{align*}
Rearrange to isolate $\frac{\diff y}{\diff x}$ on one side.  
\begin{align*}
x^2\frac{\diff y}{\diff x} + 2xy\frac{\diff y}{\diff x} & = -y^2-2xy-\frac{1}{\sqrt{1-x^2}} \\
(x^2+2xy)\frac{\diff y}{\diff x} & = -\Big(y^2+2xy+\frac{1}{\sqrt{1-x^2}}\Big) \\
\frac{\diff y}{\diff x} & = -\frac{y^2+2xy+\frac{1}{\sqrt{1-x^2}}}{x^2+2xy}.
\end{align*}

}%

\item   Find the equation of the tangent to the graph at each of the points you found in the first part.  

\solution{%
To find the slope of the tangent at $(1/2,0)$, plug in $x=1/2,y=0$ to the formula for $\frac{\diff y}{\diff x}$.  

\begin{align*}
\frac{\diff y}{\diff x} & = -\frac{(0)^2+2(1/2)(0) + \frac{1}{\sqrt{1-(1/2)^2}}}{(1/2)^2+2(1/2)(0)} \\
& = -\frac{0+0+\frac{1}{\sqrt{3/4}}}{1/4 + 0} \\
& = -\frac{\frac{1}{\sqrt{3}/2}}{1/4} \\
& = -\frac{2}{\sqrt{3}}\cdot \frac{4}{1} \\
& = -\frac{8}{\sqrt{3}}.
\end{align*}

Now use the point $(1/2,0)$ to find an equation for the tangent line.  
\begin{align*}
y - 0 & = -\frac{8}{\sqrt{3}}(x-1/2) \\
y & = -\frac{8}{\sqrt{3}}x +\frac{4}{\sqrt{3}}.
\end{align*}

This is the equation for one of the tangent lines.  

To find the slope of the tangent at $(1/2,-1/2)$, plug in $x=1/2,y=-1/2$ to the formula for $\frac{\diff y}{\diff x}$.  

\begin{align*}
\frac{\diff y}{\diff x} & = -\frac{(-1/2)^2+2(1/2)(-1/2) + \frac{1}{\sqrt{1-(1/2)^2}}}{(1/2)^2+2(1/2)(-1/2)} \\
& = -\frac{\frac{1}{4}-\frac{1}{2}+\frac{1}{\sqrt{3/4}}}{1/4 -\frac{1}{2}} \\
& = -\frac{-\frac{1}{4}+\frac{2}{\sqrt{3}}}{-\frac{1}{4}} \\
& = \frac{-\frac{1}{4}+\frac{2}{\sqrt{3}}}{\frac{1}{4}} \\
& = 4\big(-\frac{1}{4}+\frac{2}{\sqrt{3}}\big) \\
& = -1 + \frac{8}{\sqrt{3}}.
\end{align*}

Now use the point $(1/2,-1/2)$ to find an equation for the tangent line.  
\begin{align*}
y - (-1/2) & = \Big(-1 + \frac{8}{\sqrt{3}}\Big)(x-1/2) \\
y  & = \Big(-1 + \frac{8}{\sqrt{3}}\Big)x +1/2 - \frac{4}{\sqrt{3}} + 1/2 \\
y  & = \Big(-1 + \frac{8}{\sqrt{3}}\Big)x +1 - \frac{4}{\sqrt{3}},
\end{align*}
and this is the equation of the other tangent line.  
}%


\end{enumerate}
% end homework implicit-inverse-trig2

\homeworkEnd

\end{comment}

