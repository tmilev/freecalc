\documentclass{article}
\ProvidesPackage{homework-problems-UMB}
\addtolength{\hoffset}{-3.5cm}
\addtolength{\textwidth}{6.8cm}
\addtolength{\voffset}{-3.3cm}
\addtolength{\textheight}{6.3cm}
\usepackage{../homework-problems} %warnign folder paths are relative to the file that uses the includepackage

\renewcommand{\answer}[1]{\iftoggle{answers}{ \hfill{~} \rotatebox{180}{\tiny answer: #1}}{} }
\renewcommand{\pointsii}[1]{}


\toggletrue{solutions}
%\togglefalse{solutions}
\toggletrue{answers}
\newtheorem{problem}{Problem}

\newcommand{\hide}[1]{}

\renewcommand{\fcProblemRef}{\theproblem.\theenumi}
\renewcommand{\fcSubProblemRef}{\theenumi.\theenumii}

\begin{document}
\begin{center}
\Large
Master Problem Sheet \\ Math 140 Calculus I \\ 
\end{center}

%\noindent \textbf{Name:} \hfill{~}
%\begin{tabular}{c|c|c|c|c|c|c|c|c||c}
%Problem&1 &2&3&4&5&6&7&8& $\sum$\\ \hline
%Score&&&&&&&&&\\ \hline
%Max&20&20&20&20&20&10&20&20&150
%\end{tabular}




This master problem sheet contains all freecalc problems on the topics studied in Calculus I. The latest \LaTeX{} source of this file (complete with typo and error fixes) can be downloaded from the freecalc project page below. 

\url{https://sourceforge.net/p/freecalculus/code/HEAD/tree/}

For a list of contributors/authors of the freecalc project (and in particular, the present problem collection) see the following file.
\url{https://sourceforge.net/p/freecalculus/code/HEAD/tree/trunk/contributors.tex}


\fcLicenseContent



\tableofcontents

\section{Functions, Basic Facts}\label{secMPSfunctionBasics}
\subsection{Understanding function notation}
\begin{problem}
(Stewart, 7ed., page 21, problems 27-30)
Evaluate the difference and simplify your answer.
\begin{multicols}{2}
\begin{enumerate}
\item $\frac{f(3+h)-f(3)}{h}$, where $f(x)=4+3x-x^2$.
\answer{$-3-h$}
\item $\frac{f(a+h)-f(a)}{h}$, where $f(x)= x^3$.
\answer{$ 3a^2+3ah+h^2$}
\item $\frac{f(x)-f(a)}{x-a}$, where $f(x)=\frac{1}{x}$.
\answer{$-\frac{1}{ax}$.}
\item $\frac{f(x)-f(1)}{x-1}$, where $f(x)=\frac{x+3}{x+1}$.
\answer{$-\frac{1}{x+1}$.}
\end{enumerate}
\end{multicols}

\end{problem}
\subsection{Domains and ranges}
\begin{problem}
(Stewart, 7th ed., page 21, 31-37) Find the implied domain of the function
\begin{multicols}{2}
\begin{enumerate}
\item $f(x)=\frac{x+4}{x^2-9}$.
\item $f(x)=\frac{2x^3-5}{x^2+x-6}$.
\item $f(t)=\sqrt[3]{2t-1}$.
\item $g(t)=\sqrt{3-t}-\sqrt{2+t}$.
\item $h(x)=\frac{1}{\sqrt[4]{x^2-5x}}$.
\item $f(u)=\frac{u+1}{1+\frac{1}{u+1}}$.
\item $F(p)=\sqrt{2-{\sqrt{p}}}$.
\end{enumerate}
\end{multicols}
\end{problem}
\subsection{Piecewise Defined Functions}
\begin{problem}
\item Write down a formula for a function whose graphs is given below. The graphs are up to scale. Please note that there is more than one way to write down a correct answer. The answer key has not been proofread, use with caution.
\begin{multicols}{2}
\begin{enumerate}[ref={\fcProblemRef}]
\item ~
\psset{xunit=0.4cm, yunit=0.4cm}
\begin{pspicture}(-1.2,-1.2)(6.7,5.9)
\tiny
\fcAxesStandard{-1}{-1}{6.5}{5.5}
\fcXTickWithLabel{1}{$1$}
\fcYTickWithLabel{1}{$1$}
\fcLabels{6.5}{5.5}
\fcGrid[linestyle=dashed]{0}{0}{6}{5}{1}{1}{}
\psline[linecolor=red](0,3)(2, 0)(5, 4)
\fcHollowDot{0}{3}
\end{pspicture}

\answer{ $y= \left\{\begin{array}{ll} 
-\frac{3}{2}x+3 &\text{if } 0< x\leq 2 \\ 
\frac{4}{3}x-\frac{8}{3}& \text{if } 2<x\leq 6 \end{array} \right.$}
\item ~
\psset{xunit=0.4cm, yunit=0.4cm}
\begin{pspicture}(-3.7,-3.7)(3.7,3.9)
\tiny
\fcAxesStandard{-3.5}{-3.5}{3.5}{3.5}
\fcXTickWithLabel{1}{$1$}
\fcYTickWithLabel{1}{$1$}
\fcLabels{3.5}{3.5}
\fcGrid[linestyle=dashed]{-3}{-3}{6}{6}{1}{1}{}
\psline[linecolor=red](-3,-1)(1, 2)(2, -2)
\fcHollowDot{1}{2}
\end{pspicture}

\answer{ $y= \left\{\begin{array}{ll} 
\frac{3}{4}x+\frac{5}{4} &\text{if } -3\leq x < 1\\ 
-4x +6& \text{if }1<x\leq 2\end{array} \right.$}
\item ~
\psset{xunit=0.4cm, yunit=0.4cm}
\begin{pspicture}(-3.7,-3.7)(3.7,3.9)
\tiny
\fcAxesStandard{-3.5}{-3.5}{3.5}{3.5}
\fcXTickWithLabel{1}{$1$}
\fcYTickWithLabel{1}{$1$}
\fcLabels{3.5}{3.5}
\fcGrid[linestyle=dashed]{-3}{-3}{6}{6}{1}{1}{}
\psline[linecolor=red](-2,1)(2, -2)(3, 3)
\fcHollowDot{3}{3}
\end{pspicture}

\answer{ $y= \left\{\begin{array}{ll} 
-\frac{3}{4}x-\frac{1}{2} & \text{if } -2\leq x<2\\ 
5x-12 & \text{if } 2\leq x<3
\end{array} \right.$}
\item ~
\psset{xunit=0.4cm, yunit=0.4cm}
\begin{pspicture}(-3.7,-3.7)(3.7,3.9)
\tiny
\fcAxesStandard{-3.5}{-3.5}{3.5}{3.5}
\fcXTickWithLabel{1}{$1$}
\fcYTickWithLabel{1}{$1$}
\fcLabels{3.5}{3.5}
\fcGrid[linestyle=dashed]{-3}{-3}{6}{6}{1}{1}{}
\psline[linecolor=red](-2,2)(-1, -1)(0, 1)(1,-1)(2,2)
\end{pspicture}

\answer{ $y= \left\{\begin{array}{ll} 
-3x-4&\text{if } -2\leq x<-1  \\ 
2x+1&\text{if } -1\leq x<0  \\ 
-2x+1&\text{if } 0\leq x<1  \\ 
3x-4&\text{if } 1\leq x\leq 2  \\ 
\end{array} \right.$}
\end{enumerate}
\end{multicols}

\end{problem}

\begin{problem}
\item Write down formulas for function whose graphs are as follows. The graphs are up to scale. All arcs are parts of circles.

\begin{enumerate}[ref={\fcProblemRef}]
\item
\tiny
\psset{xunit=0.4cm, yunit=0.4cm}
\begin{pspicture}(-4,-1)(4,5)
\psaxes{->}(0,0)(-4.5,-1)(4.5,4)
\psplot[linecolor=red]{-2}{2}{4 x x mul sub sqrt }
\psline[linecolor=red](2,0)(4,1)
\psline[linecolor=red](-2,0)(-4,1)
\rput[b](4.5,0.1 ){$x$}
\rput[l](0.1,5 ){$y$}
\fcFullDot{4}{1}
\rput[b](4, 1.1){$(4, 1)$}
\fcFullDot{-4}{1}
\rput[b](-4, 1.1){$(-4, 1)$}

\end{pspicture}
\end{enumerate}

\end{problem}

\begin{problem}
Plot the piecewise defined functions by hand. Compare your answer to the plot of a computer algebra system.
\begin{multicols}{2}
\begin{enumerate}
\item $G(x)=\frac{x+|x|}{2x}$.
\item $g(x)=|x|-x$.
\item $f(x)=\doublebrace{x}{x\leq 1}{x^2}{x\geq 1}$.
\end{enumerate}
\end{multicols}

\end{problem}

\subsection{Function composition}
\begin{problem}
Find the functions $f\circ g$, $g\circ f$, $f\circ f$ and $g\circ g$ and their implied domains. The answer key has not been proofread, use with caution.

\begin{enumerate}
\item $f(x)=x^2+1$, $g(x)=x+1$. 
\answer{\begin{tabular}{l} Domain, all 4 cases: $x\in \mathbb R$ (all reals)\\ in some order: $(1+x)^{2}+1, (x)^{2}+2, ((x)^{2}+1)^{2}+1, 2+x$\end{tabular} }
\item $f(x)=\sqrt{x+1}$, $g(x)=x+1$. 
\answer{\begin{tabular}{l} Domain of $f\circ g$ is $x\geq -2$. Domain of $g\circ f$ is $x\geq -1$ \\Domain of $f\circ f$ is $ x\geq -1$. Domain of $g\circ g$ is all reals ($x\in \mathbb R$). \\ in some order:$\sqrt{2+x}, 1+\sqrt{1+x}, \sqrt{1+\sqrt{1+x}}, 2+x$\end{tabular}}
\item $f(x)= 2x$, $g(x)= \tan x$. 

You are not required to find the domain.
\answer{\begin{tabular}{l}
Domain  $f\circ f$: all reals ($x\in \mathbb R$). Domain $g\circ f$: $x\neq (2k+1)\frac{\pi}{2}$ for all $k\in \mathbb Z$\\ Domain $ f\circ g$: $x\neq (4k+1)\frac{\pi}{4}$, $x\neq (4k+3)\frac{\pi}{4}$ for all $k\in \mathbb Z$\\
Domain $g\circ g$: $x\neq (2k+1)\frac{\pi}{2}$ and $x\neq k\pi+ \arctan \left(\frac{\pi}{2}\right)$ for all $k\in \mathbb Z$
\\
in some order:$2 \tan{}x, \tan{}(2 x), 4 x, \tan{}(\tan{}x) $
\end{tabular}
}
\item $f(x)=\frac{x+1}{x-1}$, $g(x)=\frac{x-1}{x+1}$.
\answer{ 
\begin{tabular}{l}
Domain $ f\circ f$: $x\neq 1$. Domain $g\circ g$: $x\neq 0$, $x\neq -1$\\
Domain $f\circ g$: $x \neq -1$. Domain $g\circ f$: $x\neq 0$, $x\neq 1$\\
in some order: $- x, \frac{1}{x}, x, -\frac{1}{x} $
\end{tabular}
}
\end{enumerate}

\end{problem}
\begin{problem}
Compute the composite functions $(f\circ g)(x)$, $(g\circ f)(x)$. Simplify your answer to a single fraction. Find the domain of the composite function. The answer key has not been fully proofread, use with caution. 

\begin{enumerate}[ref={\fcProblemRef}]
\item $\displaystyle f{}({{x}})=\frac{x+2}{x-2},
g{}({{x}})=\frac{x-1}{x+2}$.

\answer{
\begin{tabular}{rl}
$(f\circ g)(x)= \frac{3+3 x}{-5- x}$& $x\neq -2, -5$\\ 
$(g\circ f)(x)=\frac{4}{-2+3 x}$& $x\neq 2, \frac{2}{3}$ 
\end{tabular}
}
\item 
$\displaystyle f{}({{x}})=\frac{x+1}{3x-2}, g{}({{x}})= \frac{x-2}{x-1}
$.

\answer{
\begin{tabular}{rl}
$(f\circ g)(x)= \frac{-3+2 x}{-4+x}
$ & $x\neq 4, 1$ \\
$(g\circ f)(x)=\frac{5-5 x}{3-2 x}$
& $x\neq \frac{2}{3}, \frac{3}{2}$
\end{tabular}
}
\item 
$\displaystyle f{}({{x}})=\frac{2x+1}{3x-1},
g{}({{x}})=\frac{x-2}{2x-1}
$.

\answer{
\begin{tabular}{rl}
$(f\circ g)(x)=\frac{-5+4 x}{-5+x}
$ &$x\neq 5, \frac{1}{2}$ \\
$(g\circ f)(x)=\frac{3-4 x}{3+x}
$ &$x\neq -3, \frac{1}{3}$
\end{tabular}
}
\item 
$\displaystyle f{}({{x}})=\frac{x+1}{x-2},
g{}({{x}})=\frac{x+2}{2x-1}
$.

\answer{
\begin{tabular}{rl}
$(f\circ g)(x)= \frac{1+3 x}{4-3 x}
$&$x\neq \frac{4}{3}, \frac{1}{2}$\\ 
$(g\circ f)(x)=\frac{-3+3 x}{4+x}
$&$x\neq -4, 2$
\end{tabular}
}
\item 
$\displaystyle f{}({{x}})=\frac{5x+1}{4x-1},
g{}({{x}})=\frac{4x-1}{3x+1}
$.

\answer{
\begin{tabular}{rl}
$(f\circ g)(x)= \frac{-4+23 x}{-5+13 x}
$&$x\neq \frac{5}{13}, -\frac{1}{3 }$\\
$(g\circ f)(x)=\frac{5+16 x}{2+19 x}
$&$x\neq -\frac{2}{19}, \frac{ 1}{4}$
\end{tabular}
}
\item $\displaystyle  f(x)= \frac{3x-5}{x-2}$, $\displaystyle g(y)=\frac{y-2 }{y-4} $. 

\answer{ 
\begin{tabular}{rl}
$(f\circ g)(x)=\frac{-2 x+14}{- x+6}$&$x\neq 6, 4$\\
$(g\circ f)(x)=\frac{x-1}{- x+3}$&$x\neq 3,2$
\end{tabular}
}
\item $\displaystyle  f(x)= \frac{x-3}{x+2}$, $\displaystyle g(y)=\frac{y+3 }{y-4} $. 

\answer{ 
\begin{tabular}{rl}
$(f\circ g)(x)=\frac{-2x+15}{3 x-5}$&$x\neq \frac{5}{3}, 4$\\ 
$(g\circ f)(x)=\frac{4 x+3}{-3 x-11}$&$x\neq -\frac{11}{3}, -2$
\end{tabular}
}
\end{enumerate}

\end{problem}
\subsection{Linear Transformations and Graphs of Functions}
\begin{problem}
\begin{problem}Graph the functions by hand, by applying consecutively the transformations learned in class.
\begin{multicols}{2}
\begin{enumerate}
\item $y=\frac{1}{x}$.
\item $y=\frac{1}{x+1}$.
\item $y=\frac{1}{2x+1}$.
\item $y=\frac{3}{2x+1}$.
\item $y=\frac{3+x}{2x+1}$.
\item $y=\left|\frac{3+x}{2x+1}\right|$.
\end{enumerate}
\end{multicols}
\end{problem}
\end{problem}

\begin{problem}
Sketch by hand approximately the given function. The function is obtained by transforming linearly the graph of a known function. The known function has been sketched for you by computer.

\begin{enumerate}[ref={\fcProblemRef}]
\item $\displaystyle f(x)=-\frac{1}{2}x^2+1$.

\begin{pspicture}(-3.5,-3.5)(3.5,3.5)
\fcAxesStandard{-3.5}{-3.5}{3.5}{3.5}
\fcGrid[linestyle=dashed]{-3}{-3}{6}{6}{1}{1}{}
\newcommand{\theFun}{x x mul}
\newcommand{\theFunTwo}{x x mul -0.5 mul 1 add}
\psplot[linecolor=gray]{-1.75}{1.75}{\theFun}
\rput[t](0,0){$y=x^2$}
%\psplot[linecolor=\fcColorGraph]{-3}{3}{\theFunTwo}
%\rput[tl](-1.9,-1.5){$y=-\frac{1}{2}x^2+1$}
\end{pspicture}

\answer{
\psset{xunit=0.2cm, yunit=0.2cm}
\begin{pspicture}(-3.5,-3.5)(3.5,3.5)
\fcAxesStandard{-3.5}{-3.5}{3.5}{3.5}
\fcGrid[linestyle=dashed]{-3}{-3}{6}{6}{1}{1}{}
\newcommand{\theFun}{x x mul}
\newcommand{\theFunTwo}{x x mul -0.5 mul 1 add}
\psplot[linecolor=gray]{-1.75}{1.75}{\theFun}
\psplot[linecolor=\fcColorGraph]{-3}{3}{\theFunTwo}
\end{pspicture}
}
\item $\displaystyle f(x)=\frac{1}{2}x^2+x-1 $.

\begin{pspicture}(-3.5,-3.5)(3.5,3.5)
\fcAxesStandard{-3.5}{-3.5}{3.5}{3.5}
\fcGrid[linestyle=dashed]{-3}{-3}{6}{6}{1}{1}{}
\newcommand{\theFun}{x x mul}
\newcommand{\theFunTwo}{x x mul 0.5 mul x add 1 add}
\psplot[linecolor=gray]{-1.75}{1.75}{\theFun}
\rput[t](0,0){$y=x^2$}
%\psplot[linecolor=\fcColorGraph]{-3}{1.3}{\theFunTwo}
%\rput[tl](-2,1.5){$y=\frac{1}{2}x^2+x-1$}
\end{pspicture}

\answer{
\psset{xunit=0.2cm, yunit=0.2cm}
\begin{pspicture}(-3.5,-3.5)(3.5,3.5)
\fcAxesStandard{-3.5}{-3.5}{3.5}{3.5}
\fcGrid[linestyle=dashed]{-3}{-3}{6}{6}{1}{1}{}
\newcommand{\theFun}{x x mul}
\newcommand{\theFunTwo}{x x mul 0.5 mul x add 1 add}
\psplot[linecolor=gray]{-1.75}{1.75}{\theFun}
\psplot[linecolor=\fcColorGraph]{-3}{1.3}{\theFunTwo}
\end{pspicture}
}
\item $\displaystyle f(x)=\frac{1}{2x-1}+1$.

\begin{pspicture}(-3.5,-3.5)(3.5,3.5)
\fcAxesStandard{-3.5}{-3.5}{3.5}{3.5}
\fcGrid[linestyle=dashed]{-3}{-3}{6}{6}{1}{1}{}
\newcommand{\theFun}{1 x div}
\newcommand{\theFunTwo}{1 x 2 mul -1 add div 1 add}
\psplot[linecolor=gray]{-3}{-0.32}{\theFun}
\psplot[linecolor=gray]{0.32}{3}{\theFun}
\rput[bt](1.2,0.5){$~y=\frac{1}{x}$}
%\psplot[linecolor=\fcColorGraph]{-3}{0.37}{\theFunTwo}
%\psplot[linecolor=\fcColorGraph]{0.7}{3}{\theFunTwo}
%\rput[tl](-2,1.5){$y=\frac{1}{2x-1}+1$}
\end{pspicture}

\answer{
\psset{xunit=0.2cm, yunit=0.2cm}
\begin{pspicture}(-3.5,-3.5)(3.5,3.5)
\fcAxesStandard{-3.5}{-3.5}{3.5}{3.5}
\fcGrid[linestyle=dashed]{-3}{-3}{6}{6}{1}{1}{}
\newcommand{\theFun}{1 x div}
\newcommand{\theFunTwo}{1 x 2 mul -1 add div 1 add}
\psplot[linecolor=gray]{-3}{-0.32}{\theFun}
\psplot[linecolor=gray]{0.32}{3}{\theFun}
%\rput[bt](1.2,0.5){$~y=\frac{1}{x}$}
\psplot[linecolor=\fcColorGraph]{-3}{0.37}{\theFunTwo}
\psplot[linecolor=\fcColorGraph]{0.7}{3}{\theFunTwo}
%\rput[tl](-2,1.5){$y=\frac{1}{2x-1}+1$}
\end{pspicture}
}
\item $\displaystyle f(x)=\frac{\frac{1}{2}x + \frac{1}{4}}{x-\frac{1}{2}}+\frac{1}{2}$.

\begin{pspicture}(-3.5,-3.5)(3.5,3.5)
\fcAxesStandard{-3.5}{-3.5}{3.5}{3.5}
\fcGrid[linestyle=dashed]{-3}{-3}{6}{6}{1}{1}{}
\newcommand{\theFun}{1 x div}
\newcommand{\theFunTwo}{1 x 2 mul -1 add div 1 add}
\psplot[linecolor=gray]{-3}{-0.32}{\theFun}
\psplot[linecolor=gray]{0.32}{3}{\theFun}
\rput[bt](1.2,0.5){$~y=\frac{1}{x}$}
%\psplot[linecolor=\fcColorGraph]{-3}{0.37}{\theFunTwo}
%\psplot[linecolor=\fcColorGraph]{0.7}{3}{\theFunTwo}
%\rput[tl](-2,1.5){$y=\frac{1}{2x-1}+1$}
\end{pspicture}

\answer{
\psset{xunit=0.2cm, yunit=0.2cm}
\begin{pspicture}(-3.5,-3.5)(3.5,3.5)
\fcAxesStandard{-3.5}{-3.5}{3.5}{3.5}
\fcGrid[linestyle=dashed]{-3}{-3}{6}{6}{1}{1}{}
\newcommand{\theFun}{1 x div}
\newcommand{\theFunTwo}{1 x 2 mul -1 add div 1 add}
\psplot[linecolor=gray]{-3}{-0.32}{\theFun}
\psplot[linecolor=gray]{0.32}{3}{\theFun}
%\rput[bt](1.2,0.5){$~y=\frac{1}{x}$}
\psplot[linecolor=\fcColorGraph]{-3}{0.37}{\theFunTwo}
\psplot[linecolor=\fcColorGraph]{0.7}{3}{\theFunTwo}
%\rput[tl](-2,1.5){$y=\frac{1}{2x-1}+1$}
\end{pspicture}
}
\item $\displaystyle f(x)= -\sqrt{2x-1}-1$

\begin{pspicture}(-3.5,-3.5)(3.5,3.5)
\fcAxesStandard{-3.5}{-3.5}{3.5}{3.5}
\fcGrid[linestyle=dashed]{-3}{-3}{6}{6}{1}{1}{}
\newcommand{\theFun}{x sqrt}
\newcommand{\theFunTwo}{2 x mul 1 sub sqrt -1 mul -1 add}
\psplot[linecolor=gray]{0}{3}{\theFun}
\rput[bt](1.2,0.5){$~y=\sqrt{x}$}
%\psplot[linecolor=\fcColorGraph]{0.5}{3}{\theFunTwo}
\end{pspicture}

\answer{
\psset{xunit=0.2cm, yunit=0.2cm}
\begin{pspicture}(-3.5,-3.5)(3.5,3.5)
\fcAxesStandard{-3.5}{-3.5}{3.5}{3.5}
\fcGrid[linestyle=dashed]{-3}{-3}{6}{6}{1}{1}{}
\newcommand{\theFun}{x sqrt}
\newcommand{\theFunTwo}{2 x mul 1 sub sqrt -1 mul -1 add}
\psplot[linecolor=gray]{0}{3}{\theFun}
\rput[bt](1.2,0.5){$~y=\sqrt{x}$}
%\psplot[linecolor=\fcColorGraph]{0.5}{3}{\theFunTwo}
\end{pspicture}
}
\item $\displaystyle f(x)= -\sqrt{-2x-1}+1$

\begin{pspicture}(-3.5,-3.5)(3.5,3.5)
\fcAxesStandard{-3.5}{-3.5}{3.5}{3.5}
\fcGrid[linestyle=dashed]{-3}{-3}{6}{6}{1}{1}{}
\newcommand{\theFun}{x sqrt}
\newcommand{\theFunTwo}{-2 x mul 1 sub sqrt -1 mul 1 add}
\psplot[linecolor=gray]{0}{3}{\theFun}
\rput[bt](1.2,0.5){$~y=\sqrt{x}$}
%\psplot[linecolor=\fcColorGraph]{-3}{-0.5}{\theFunTwo}
\end{pspicture}

\answer{
\psset{xunit=0.2cm, yunit=0.2cm}
\begin{pspicture}(-3.5,-3.5)(3.5,3.5)
\fcAxesStandard{-3.5}{-3.5}{3.5}{3.5}
\fcGrid[linestyle=dashed]{-3}{-3}{6}{6}{1}{1}{}
\newcommand{\theFun}{x sqrt}
\newcommand{\theFunTwo}{-2 x mul 1 sub sqrt -1 mul 1 add}
\psplot[linecolor=gray]{0}{3}{\theFun}
\rput[bt](1.2,0.5){$~y=\sqrt{x}$}
\psplot[linecolor=\fcColorGraph]{-3}{-0.5}{\theFunTwo}
\end{pspicture}
}
\end{enumerate}
\end{problem}
\solution{\ref{problemSketch1/(2x-1)+1fromgraphof1/x}.

We are asked to plot $\displaystyle f(x)=\frac{1}{2x-1}+1$ by establishing a connection with the graph of $g(x)=\frac{1}{x}$. To do that we have to compose $g$ with a sequence of linear transformations to obtain $f(x)$. There are two natural ways to do that; we show both by presenting two different solutions.


\noindent\textbf{Solution I.} 
We show how to get from $g(x)=\frac{1}{x}$ by composing $g$ with a sequence of linear transformations.
\[
\begin{array}{r|rcll|l}
g(x)&=&\frac{1}{x} \\
\text{Define } h(x) \text{ via:}& h(x)&=&\displaystyle g(x+1)= \frac{1}{x-1} \\
\text{Define } k(x) \text{ via:}& k(x)&=&\displaystyle h(2x)= \frac{1}{2x-1} \\
\text{Therefore }& f(x)&=&\displaystyle k(x)+1= \frac{1}{2x-1}+1 \\
\end{array}
\]


We plot consecutively the functions $g(x)$, $h(x)$, $k(x)$ and $f(x)$. We start from the given graph of $g(x)$.

\psset{xunit=0.7cm, yunit=0.7cm}
\begin{pspicture}(-3.5,-3.5)(3.5,3.5)
\fcAxesStandard{-3.5}{-3.5}{3.5}{3.5}
\fcGrid[linestyle=dashed]{-3}{-3}{6}{6}{1}{1}{}
\newcommand{\theFun}{1 x div}
\newcommand{\theFunTwo}{1 x 2 mul -1 add div 1 add}
\psplot[linecolor=\fcColorGraph]{-3}{-0.32}{\theFun}
\psplot[linecolor=\fcColorGraph]{0.32}{3}{\theFun}
\rput[bt](1.2,0.5){$~y=g(x)=\frac{1}{x}$}
%\psplot[linecolor=\fcColorGraph]{-3}{0.37}{\theFunTwo}
%\psplot[linecolor=\fcColorGraph]{0.7}{3}{\theFunTwo}
%\rput[tl](-2,1.5){$y=\frac{1}{2x-1}+1$}
\end{pspicture}
\raisebox{2cm}{ $\stackrel{\begin{array}{l} \text{shift graph of }g(x)\\ 1 \text{ unit right} \end{array} }{\longrightarrow}$}
\begin{pspicture}(-3.5,-3.5)(3.5,3.5)
\fcAxesStandard{-3.5}{-3.5}{3.5}{3.5}
\fcGrid[linestyle=dashed]{-3}{-3}{6}{6}{1}{1}{}
\newcommand{\theFun}{1 x div}
\newcommand{\theFunTwo}{1 x  -1 add div}
\psplot[linecolor=gray]{-3}{-0.32}{\theFun}
\psplot[linecolor=gray]{0.32}{3}{\theFun}
\rput[bl](-3,-2.95){$~y=h(x)=g(x-1)=\frac{1}{x-1}$}
\psplot[linecolor=\fcColorGraph]{-3}{0.68}{\theFunTwo}
\psplot[linecolor=\fcColorGraph]{1.32}{3}{\theFunTwo}
%\rput[tl](-2,1.5){$y=\frac{1}{2x-1}+1$}
\end{pspicture}
\raisebox{2cm}{ $\stackrel{\begin{array}{l} \text{compress graph of }h(x)\\ \text{by factor of }2 \\\text{horizontally} \end{array} }{\longrightarrow}$}

\begin{pspicture}(-3.5,-3.5)(3.5,3.5)
\fcAxesStandard{-3.5}{-3.5}{3.5}{3.5}
\fcGrid[linestyle=dashed]{-3}{-3}{6}{6}{1}{1}{}
\newcommand{\theFun}{1 x  -1 add div}
\newcommand{\theFunTwo}{1 x 2 mul -1 add div}
\psplot[linecolor=gray]{-3}{-0.68}{\theFun}
\psplot[linecolor=gray]{1.32}{3}{\theFun}
\rput[bl](-3,-2.95){$~y=k(x)=h(2x)=\frac{1}{2x-1}$}
\psplot[linecolor=\fcColorGraph]{-3}{0.34}{\theFunTwo}
\psplot[linecolor=\fcColorGraph]{0.66}{3}{\theFunTwo}
%\rput[tl](-2,1.5){$y=\frac{1}{2x-1}+1$}
\end{pspicture}
\raisebox{2cm}{ $\stackrel{\begin{array}{l} \text{shift graph of }k(x)\\ 1 \text{ unit vertically }\end{array} }{\longrightarrow}$}
\begin{pspicture}(-3.5,-3.5)(3.5,3.5)
\fcAxesStandard{-3.5}{-3.5}{3.5}{3.5}
\fcGrid[linestyle=dashed]{-3}{-3}{6}{6}{1}{1}{}
\newcommand{\theFun}{1 x 2 mul -1 add div}
\newcommand{\theFunTwo}{1 x 2 mul -1 add div 1 add}
\psplot[linecolor=gray]{-3}{0.34}{\theFun}
\psplot[linecolor=gray]{0.66}{3}{\theFun}
\rput[bl](-3,-2.95){$~y=f(x)=h(x)+1=\frac{1}{2x-1}+1$}
\psplot[linecolor=\fcColorGraph]{-3}{0.38}{\theFunTwo}
\psplot[linecolor=\fcColorGraph]{0.74}{3}{\theFunTwo}
%\rput[tl](-2,1.5){$y=\frac{1}{2x-1}+1$}
\end{pspicture}
\noindent\textbf{Solution II.} 
In the previous solution we used horizontal stretch to transform the graph of $h(x)$ to the graph of $k(x)=h(2x)$. However algebra suggests a another to transform the graph of $g(x)$ to the graph of $f(x)$, this time using a vertical stretch. Indeed, the equality 
\[
\displaystyle f(x)=\frac{1}{2x-1}+1=\frac{1}{2}\cdot \frac{1}{x-\frac{1}{2}}+1
\]
suggests a different way to obtain the graph of $f(x)$ from the graph of $g(x)$.
\[
\begin{array}{r|rcll|l}
g(x)&=&\frac{1}{x} \\
\text{Define } l(x) \text{ via:}& l(x)&=&\displaystyle g\left(x-\frac{1}{2}\right)= \frac{1}{x-\frac{1}{2}} \\
\text{Define } k(x) \text{ via:}& k(x)&=&\displaystyle \frac{1}{2}h(x)= \frac{1}{2} \cdot\frac{1}{\left(x-\frac{1}{2}\right)}= \frac{1}{\left(2x-1\right)} \\
\text{Therefore }& f(x)&=&\displaystyle k(x)+1= \frac{1}{2x-1}+1 \\
\end{array}
\]


We plot consecutively the functions $g(x)$, $h(x)$, $k(x)$ and $f(x)$. We start from the given graph of $g(x)$.

\psset{xunit=0.7cm, yunit=0.7cm}
\begin{pspicture}(-3.5,-3.5)(3.5,3.5)
\fcAxesStandard{-3.5}{-3.5}{3.5}{3.5}
\fcGrid[linestyle=dashed]{-3}{-3}{6}{6}{1}{1}{}
\newcommand{\theFun}{1 x div}
\newcommand{\theFunTwo}{1 x 2 mul -1 add div 1 add}
\psplot[linecolor=\fcColorGraph]{-3}{-0.32}{\theFun}
\psplot[linecolor=\fcColorGraph]{0.32}{3}{\theFun}
\rput[bt](1.2,0.5){$~y=g(x)=\frac{1}{x}$}
%\psplot[linecolor=\fcColorGraph]{-3}{0.37}{\theFunTwo}
%\psplot[linecolor=\fcColorGraph]{0.7}{3}{\theFunTwo}
%\rput[tl](-2,1.5){$y=\frac{1}{2x-1}+1$}
\end{pspicture}
\raisebox{2cm}{ $\stackrel{\begin{array}{l} \text{shift graph of }g(x)\\\frac{1}{2} \text{ units right} \end{array} }{\longrightarrow}$}
\begin{pspicture}(-3.5,-3.5)(3.5,3.5)
\fcAxesStandard{-3.5}{-3.5}{3.5}{3.5}
\fcGrid[linestyle=dashed]{-3}{-3}{6}{6}{1}{1}{}
\newcommand{\theFun}{1 x div}
\newcommand{\theFunTwo}{1 x  -0.5 add div}
\psplot[linecolor=gray]{-3}{-0.32}{\theFun}
\psplot[linecolor=gray]{0.32}{3}{\theFun}
\rput[bl](-3,-2.95){$y=l(x)=g\left(x-\frac{1}{2}\right)=\frac{1}{x-\frac{1}{2}}$}
\psplot[linecolor=\fcColorGraph]{-3}{0.18}{\theFunTwo}
\psplot[linecolor=\fcColorGraph]{0.82}{3}{\theFunTwo}
%\rput[tl](-2,1.5){$y=\frac{1}{2x-1}+1$}
\end{pspicture}
\raisebox{2cm}{ $\stackrel{\begin{array}{l} \text{compress graph of }h(x)\\ \text{by factor of }2 \\\text{vertically} \end{array} }{\longrightarrow}$}

\begin{pspicture}(-3.5,-3.5)(3.5,3.5)
\fcAxesStandard{-3.5}{-3.5}{3.5}{3.5}
\fcGrid[linestyle=dashed]{-3}{-3}{6}{6}{1}{1}{}
\newcommand{\theFun}{1 x  -0.5 add div}
\newcommand{\theFunTwo}{1 x 2 mul -1 add div}
\psplot[linecolor=gray]{-3}{0.18}{\theFun}
\psplot[linecolor=gray]{0.82}{3}{\theFun}
\rput[bl](-3,-2.95){$~y=k(x)=\frac{1}{2}l(x)=\frac{1}{2x-1}$}
\psplot[linecolor=\fcColorGraph]{-3}{0.34}{\theFunTwo}
\psplot[linecolor=\fcColorGraph]{0.66}{3}{\theFunTwo}
%\rput[tl](-2,1.5){$y=\frac{1}{2x-1}+1$}
\end{pspicture}
\raisebox{2cm}{ $\stackrel{\begin{array}{l} \text{shift graph of }k(x)\\ 1 \text{ unit vertically }\end{array} }{\longrightarrow}$}
\begin{pspicture}(-3.5,-3.5)(3.5,3.5)
\fcAxesStandard{-3.5}{-3.5}{3.5}{3.5}
\fcGrid[linestyle=dashed]{-3}{-3}{6}{6}{1}{1}{}
\newcommand{\theFun}{1 x 2 mul -1 add div}
\newcommand{\theFunTwo}{1 x 2 mul -1 add div 1 add}
\psplot[linecolor=gray]{-3}{0.34}{\theFun}
\psplot[linecolor=gray]{0.66}{3}{\theFun}
\rput[bl](-3,-2.95){$~y=f(x)=h(x)+1=\frac{1}{2x-1}+1$}
\psplot[linecolor=\fcColorGraph]{-3}{0.38}{\theFunTwo}
\psplot[linecolor=\fcColorGraph]{0.74}{3}{\theFunTwo}
%\rput[tl](-2,1.5){$y=\frac{1}{2x-1}+1$}
\end{pspicture}
}


\section{Trigonometry}\label{secMPStrigonometry}
\subsection{Angle conversion}
\begin{problem}
Convert from degrees to radians.
\begin{multicols}{3}
\begin{enumerate}
\item $15^\circ$.
\item $30^\circ$.
\item $36^\circ$.
\item $45^\circ$.
\item $60^\circ$.
\item $75^\circ$.
\item $90^\circ$.
\item $120^\circ$.
\item $135^\circ$.
\item $150^\circ$.
\item $180^\circ$.
\item $225^\circ$.
\item $270^\circ$.
\item $305^\circ$.
\item $360^\circ$.
\item $405^\circ$.
\item $1200^\circ$.
\item $-900^\circ$.
\item $-2014^\circ$.
\end{enumerate}
\end{multicols}

\end{problem}
\begin{problem}
Convert from radians to degrees. The answer key has not been proofread, use with caution.
\begin{multicols}{3}
\begin{enumerate}
\item $4\pi$.

\answer{$720^{\circ}$}
\item $-\frac{7}{6}\pi$.

\answer{$-210^{\circ}$}
\item $\frac{7}{12}\pi$.

\answer{$105^{\circ}$}
\item $\frac{4}{3}\pi$.

\answer{$240^{\circ}$}
\item $-\frac{3}{8}\pi$.

\answer{$-67.5^{\circ}$}
\item $2014\pi$.

\answer{$362520^{\circ}$}
\item $5$.

\answer{$\left(\frac{900}{\pi}\right)^{\circ}\approx 286^\circ$}
\item $-2014$.

\answer{$-362520 ^{\circ}$}
\end{enumerate}
\end{multicols}
\end{problem}
\subsection{Trigonometry identities}
\begin{problem}
(Textbook appendix page 32-, problems 45, 46, 47, 48, 49, 50, 51, 52, 56, 57, 58).
Derive the trigonometry identities.
\begin{multicols}{3}
\begin{enumerate}
\item $\sin \theta\cot \theta =\cos \theta$.
\item $(\sin x +\cos x)^2=1+\sin(2x)$.
\item $\sec y - \cos y= \tan y \sin y$.
\item $\tan^2 \alpha-\sin^2 \alpha=\tan^2\alpha\sin^2\alpha$.
\item $\cot^2\theta+\sec^2\theta=\tan^2\theta+\csc^2\theta$.
\item $2\csc 2t= \sec t \csc t$.
\item $\tan 2\theta =\frac{2\tan \theta}{1-\tan^2\theta} $.
\item $\frac{1}{1-\sin \theta}+ \frac{1}{1+\sin \theta}=2\sec^2\theta$.
\item $\tan x + \tan y = \frac{\sin (x+y)}{\cos x \cos y}$.
\item $\sin 3\theta +\sin \theta = 2 \sin 2\theta \cos \theta $.
\item $\cos 3\theta = 4\cos^3\theta-3\cos \theta $.
\end{enumerate} 
\end{multicols}

\end{problem}

\subsection{Trigonometry equations}
\begin{problem}
\begin{problem}(Textbook page A33, problems 65-72).
Find all values of $x$ in the interval $[0,2\pi]$ that satisfy the 
equation.
\begin{multicols}{3}
\begin{enumerate}
\item $2\cos x - 1=0$.
\item $3\cot^2 x= 1$.
\item $2\sin^2 x= 1$.
\item $|\tan x|=1 $.
\item $\sin 2x = \cos x $.
\item $2\cos x +\sin 2x=0$.
\item $\sin x =\tan x$.
\item $2+\cos 2x = 3 \cos x$.
\end{enumerate}
\end{multicols}
\end{problem}

\end{problem}
\solution{ \ref{problemSolve2cos^2x-(1+sqrt(2))cosx+sqrt(2)/2=0}
Set $\cos x=u$. Then 
\[
2\cos^2 x- (1+\sqrt{2})\cos x+\frac{\sqrt 2}2=0 
\] 
becomes 
\[2u^2-(1+\sqrt{2})u+\frac{\sqrt{2}}{2}=0.
\] 
This is a quadratic equation in $u$ and therefore has solutions
\[
\begin{array}{rcl}
u_1, u_2\displaystyle &=& \displaystyle \frac{ 1+\sqrt{2}\pm\sqrt{ (1+\sqrt{2})^2-4 \sqrt{2} } }4\\
&=&\displaystyle \frac{1+\sqrt{2}\pm\sqrt{1-2\sqrt{2}+2} }4\\
&=&\displaystyle \frac{1+\sqrt{2}\pm \sqrt{(1-\sqrt{2})^2}}4\\
&=&\displaystyle \frac{1+\sqrt{2}\pm (1-\sqrt{2}) }4=\doublebrace{\frac{1}2 }{ \mathrm{or}} {\frac{\sqrt{2}}{2}}{}
\end{array}
\]
Therefore $u=\cos x= \frac12$ or $u=\cos x=\frac{\sqrt{2}}2$, and, as $x$ is in the interval $[0,2\pi]$, we get $x=\frac{\pi}{3}, \frac{5\pi}{3}$ (for $\cos x=\frac12$) or $x=\frac{\pi }4 ,\frac{7\pi}4$ (for $\cos x=\frac{\sqrt{2}}{2}$).
} % end solution

\section{Limits and Continuity}
\subsection{Limits as $x$ tends to a number}\label{secMPSlimitsXtendsToNumer}
\begin{problem}
(Problem contributed by Gabe Cunningham) Find the following limits, or show that they do not exist:
\begin{multicols}{2}
\begin{enumerate}
\item ${\displaystyle \lim_{x \to 2} \frac{x^2-4}{x^2-x-2}}$
\answer{$\frac43$}
\item ${\displaystyle \lim_{x \to -\infty} \frac{5x^3+x-1}{2x^3-7}}$
\answer{$5/2$}
\item ${\displaystyle \lim_{x \to 1^{+}} \frac{x-3}{x-1}}$
\answer{$-\infty$}
\item ${\displaystyle \lim_{h \to 0} \frac{2(x+h)^3 - 2x^3}{h}}$
\answer{$6x^2$}
\item ${\displaystyle \lim_{x \to \infty} \frac{\sqrt{9x^2-2}}{x+4}}$
\answer{$3$}
\item ${\displaystyle \lim_{x \to -1} \frac{2x+3}{x+1}}$
\answer{Does not exist}
\end{enumerate}
\end{multicols}

\end{problem}
\begin{problem}
(Textbook page 70, problems 11-32). 
Evaluate the limit if it exists.
\begin{multicols}{2}
\begin{enumerate}
\item $\displaystyle\lim\limits_{x\to 5}\frac{x^2-6x+5}{x-5} $. 
\answer{4}
\item $\displaystyle\lim\limits_{x\to 4}\frac{x^2-4x}{x^2-3x-4} $.
\answer{$\frac{4}5$}
\item $\displaystyle\lim\limits_{x\to 5}\frac{x^2-5x+6}{x-5} $.
\answer{DNE}
\item $\displaystyle\lim\limits_{x\to -1}\frac{x^2-4x}{x^{2}-3x-4} $.
\answer{DNE}
\item $\displaystyle\lim\limits_{t\to -3}\frac{t^2-9}{2t^2+7t+3} $.
\answer{$\frac{6}{5}$}
\item $\displaystyle\lim\limits_{x\to -1}\frac{2x^2+3x+1}{x^2-2x-3} $.
\answer{$\frac{1}{4}$}
\item $\displaystyle\lim\limits_{h\to 0}\frac{(-5+h)^2-25}{h} $.
\answer{$-10$}
\item $\displaystyle\lim\limits_{h\to 0}\frac{(2+h)^3-8}{h} $.
\answer{$12$}
\item $\displaystyle\lim\limits_{x\to -2}\frac{x+2}{x^3+8} $.
\answer{$\frac{1}{12}$}
\item $\displaystyle\lim\limits_{t\to 1}\frac{t^4-1}{t^3-1} $.
\answer{$\frac{4}{3}$}
\item $\displaystyle\lim\limits_{h\to 0}\frac{\sqrt{9+h}-3}{h} $.
\answer{$\frac{1}{6}$}
\item $\displaystyle\lim\limits_{u\to 2} \frac{\sqrt{4u+1}-3}{u-2}$.
\answer{$\frac{2}{3}$}
\item $\displaystyle\lim\limits_{x\to -4} \frac{\frac{1}{4}+ \frac{1}{x}} {4+x}$.
\answer{$-\frac{1}{16}$}
\item $\displaystyle\lim\limits_{x\to -1} \frac{x^2+2x+1}{x^4-1}$.
\answer{$0$}
\item $\displaystyle\lim\limits_{t\to 0} \frac{\sqrt{1+t}- \sqrt{1-t}}{t}$.
\answer{$1$}
\item $\displaystyle\lim\limits_{t\to 0}\left(\frac{1}t -\frac{1}{t^2+t}\right)$.
\answer{$1$}
\item $\displaystyle\lim\limits_{x\to 16} \frac{4-\sqrt{x}}{16x-x^2}$.
\answer{$\frac{1}{128}$}
\item $\displaystyle\lim\limits_{h \to 0}\frac{(3+h)^{-1}-3^{-1}}{h} $.
\answer{$-\frac{1}{9}$}
\item $\displaystyle\lim\limits_{t\to 0} \left(\frac{1}{t\sqrt{1+t}}-\frac{1}{t} \right)$.
\answer{$-\frac{1}{2}$}
\item $\displaystyle\lim\limits_{x\to -4} \frac{\sqrt{x^2+9}-5}{x+4}$.
\answer{$-\frac{4}{5}$}
\item $\displaystyle\lim\limits_{h\to 0}\frac{(x+h)^3-x^3}{h} $.
\answer{$3x^2$}
\item $\displaystyle\lim\limits_{h\to 0}\frac{\frac{1}{(x+h)^2}-\frac{1}{x^2}}{h} $.
\answer{$-\frac{2}{x^3}$}
\solution{ 
~\\
$
\begin{array}{rcl}
\lim\limits_{h\to 0}\frac{\frac{1}{(x+h)^2}-\frac{1}{x^2}}{h}&=&\lim\limits_{h\to 0}\frac{x^2-(x+h)^2}{hx^2(x+h)^2}=\lim\limits_{h\to 0} \frac{x^2-(x^2+2xh+h^2)}{hx^2(x+h)^2}\\
&=&\lim\limits_{h\to 0}\frac{\cancel{h}(-2x+h)}{\cancel{h}x^2(x+h)^2}= \frac{-2x+0}{x^2(x+0)^2}=-\frac{2}{x^3}
\end{array}
$
}
\end{enumerate}
\end{multicols}

\end{problem}
\solution{\ref{problemlim(xto2)(x^2-5x+6)/(x-2)}

$
\begin{array}{rcll|l}
\displaystyle 
\displaystyle \lim\limits_{x\to 2}\frac{x^2-5x+6}{x-2} &=&\displaystyle \lim\limits_{x\to 2}\frac{(x-3)\cancel{(x-2)}}{\cancel{x-2}} &&\text{factor and cancel}\\
&=&\displaystyle 2-3=-1
\end{array}
$
}
\solution{\ref{problemlimxto-2(2x^2+x-6)/(x^2-4)}

$\begin{array}{rcll|l}
\displaystyle \lim\limits_{x\to -2} \frac{2x^2+x-6}{x^2-4}&=&  \displaystyle \lim\limits_{x\to -2}\frac{ (2x -3)\cancel{( x+ 2 ) }}{ (x-2)\cancel{(x+2)}} &&\text{factor and cancel}\\ 
&=&\displaystyle  \frac{(2(-2)-3)}{-2-2} &&\text{substitute}\\
&=&\displaystyle \frac{7}{4}
\end{array}
$

}
\solution{\ref{limproblem(xto-2)(x^2-4)/(2x^2+5x+2)}

$
\begin{array}{rcll|l}
\displaystyle 
\displaystyle \lim\limits_{x\to 2}\frac{x^2-4}{2x^2+5x+2} &=&\displaystyle \lim\limits_{x\to -2} \frac{(x-2)\cancel{(x+2)}}{(2x+1) \cancel{(x+2)}} &&\text{factor and cancel}\\
&=&\displaystyle \frac{(-2)-2}{2(-2)+1}=\frac{4}{3}
\end{array}
$
}
\solution{
\ref{problemlim(xto-1)(2x^2+3x+1)/(3x^2-2x-5)}

$
\begin{array}{rcll|l}
\displaystyle \lim\limits_{x\to-1}\frac{2x^2+3x+1}{3x^2-2x-5} &=&\displaystyle \lim\limits_{x\to -1}\frac{(2x+1)\cancel{(x+1)}}{(3x-5)\cancel{(x+1)}}&&\text{factor and cancel}\\
&=&\displaystyle \frac{2(-1)+1}{3(-1)-5} =\frac{1}{8}.
\end{array}
$
}
\solution{\ref{problemlimxto-4(x^2+7x+12)/(x^2+6x+8)}.

$\begin{array}{rcll|l}
\displaystyle \lim_{x \to -4}\frac{x^{2}+7 x+12}{x^{2}+6 x+8}&=& \displaystyle \lim_{x \to -4}\frac{(x+3)(\cancel{x+4})}{(x+2)(\cancel{x+4})} &&\text{factor}\\
&=&\displaystyle \frac{-4+3}{-4+2}=-\frac{1}{2}.
\end{array}
$

}
\input{\freecalcBaseFolder/modules/limits/homework/limit-x-tends-to-a-difference-quotient-subproblem-22-solution}

\solution{\ref{problemlimhto0(1/(2+h)^2-1/4)/h}.

\textbf{Variant I.}

$\begin{array}{rcll|l}
\displaystyle \lim_{h\to 0} \frac{\frac{1}{(2+h)^2}-\frac{1}{4}}{h}&=&\displaystyle \lim_{h\to 0}\frac{\frac{4-(2+h)^2}{4(2+h)^2}}{h}\\
&=&\displaystyle \lim_{h\to 0} \frac{4- (4+4h+h^2)}{4h(2+h)^2}\\
&=&\displaystyle \lim_{h\to 0} \frac{-4h-h^2}{4h(2+h)^2}\\
&=&\displaystyle \lim_{h\to 0} \frac{\cancel{h}(-4-h) }{4\cancel{h}(2+h)^2}&&\text{substitute }h=0\\
&=&\displaystyle \frac{-4-0}{4(2+0)^2}\\
&=&\displaystyle -\frac{1}{4}
\end{array}
$

\textbf{Variant II.}

$\begin{array}{rcll|l}
\displaystyle \lim_{h\to 0} \frac{\frac{1}{(2+h)^2}-\frac{1}{4}}{h}&=&\displaystyle \frac{\diff }{\diff x}\left(\frac{1}{x^2}\right)_{|x=2}\\
&=&\displaystyle \left(\frac{-2}{x^3}\right)_{|x=2}\\
&=&\displaystyle -\frac{1}{4}
\end{array}
$

}


\solution{\ref{problemlimhto0(1/(1+h)^2-1)/h}.

\textbf{Variant I.}

$\begin{array}{rcll|l}
\displaystyle \lim_{h\to 0} \frac{\frac{1}{(1+h)^2}-1}{h}&=&\displaystyle \lim_{h\to 0}\frac{\frac{1-(1+h)^2}{ (1+h)^2}}{h}\\
&=&\displaystyle \lim_{h\to 0} \frac{1- (1+2h+h^2)}{h(1+h)^2}\\
&=&\displaystyle \lim_{h\to 0} \frac{-2h-h^2}{h(1+h)^2}\\
&=&\displaystyle \lim_{h\to 0} \frac{\cancel{h}(-2-h) }{\cancel{h}(1+h)^2}&&\text{substitute }h=0\\
&=&\displaystyle \frac{-2-0}{(1+0)^2}\\
&=&\displaystyle -2.
\end{array}
$

\textbf{Variant II.}

$\begin{array}{rcll|l}
\displaystyle \lim_{h\to 0} \frac{\frac{1}{(1+h)^2}-1}{h}&=&\displaystyle \frac{\diff }{\diff x}\left(\frac{1}{x^2}\right)_{|x=1}&&\text{derivative definition}\\
&=&\displaystyle \left(\frac{-2}{x^3}\right)_{|x=1}\\
&=&\displaystyle -2.
\end{array}
$

}
\begin{problem}
Evaluate the limit if it exists.
\begin{enumerate}
\item $\lim\limits_{x\to 1} \frac{3x^2+4x-7}{x^3-x}$ \answer{$5 $.}
\item $\lim\limits_{x\to -1} \frac{2x^2-3x-5}{x^3+1}$ \answer{$ -\frac{7}{3}$.}
\end{enumerate}

\end{problem}
\begin{problem}
Evaluate the limits. Justify your computations.
\begin{multicols}{3}
\begin{enumerate}
\item $\displaystyle\lim\limits_{x\to 2} 2x^2-3x-6$.
\answer{$-4$}
\item $\displaystyle\lim\limits_{x\to -1} \frac{x^4-x}{x^2+2x+3}$.
\answer{$1$}
\item $\displaystyle\lim\limits_{x\to -1} \frac{1}{x^2 -3x +2} $.
\answer{$\frac{1}{6}$}
\item $\displaystyle\lim\limits_{x\to -2}\sqrt{x^4+16}$.
\answer{$\sqrt{32}$}
\item $\displaystyle\lim\limits_{x \to 8}(1+ \sqrt[3]{x} )(2-  x)$.
\answer{$-18$}
\end{enumerate}
\end{multicols}

\end{problem}
\subsection{Limits as $x\to\pm \infty$}\label{secMPSlimitsXtoInfty}
\begin{problem}
Find the limit or show that it does not exist. If the limit does not exist, indicate whether it is $\pm\infty$, or neither. The answer key has not been proofread, use with caution.
\begin{multicols}{3}
\begin{enumerate}
\item $\displaystyle \lim\limits_{x\to\infty }\frac{x-2}{2x+1}$.

\answer{$\frac12$}
\item $\displaystyle \lim\limits_{x\to\infty }\frac{1-x^2}{x^3-x-1}$.

\answer{$ 0$}
\item $\displaystyle \lim\limits_{x\to-\infty }\frac{x-2}{x^2+5}$.

\answer{$ 0$}
\item $\displaystyle \lim\limits_{x\to-\infty }\frac{3x^3+2}{2x^3-4x+5}$.

\answer{$ \frac{3}{2}$}
\item $\displaystyle \lim\limits_{x\to\infty }\frac{\sqrt{x}+x^2}{\sqrt{x}-x^2}$.

\answer{$-1$}
\item $\displaystyle \lim\limits_{x\to\infty }\frac{3-x\sqrt{t}}{2x^{\frac{3}{2}}-2}$.

\answer{$-\frac12$}
\item $\displaystyle \lim\limits_{x\to\infty }\frac{(2x^2+3)^2}{(x-1)^2(x^2+1)}$.

\answer{$ 4$}
\item $\displaystyle \lim\limits_{x\to\infty }\frac{x^2-3}{\sqrt{x^4+3}}$.

\answer{$1$}
\item $\displaystyle \lim\limits_{x\to\infty }\frac{\sqrt{16x^6-3x}}{x^3+2}$.

\answer{$3$}
\item $\displaystyle \lim\limits_{x\to-\infty }\frac{\sqrt{16x^6-3x}}{x^3+2}$.

\answer{$-3$}
\item $\displaystyle \lim\limits_{x\to\infty}\sqrt{4x^2+x}-2x$.

\answer{$\frac{1}{4}$}
\item $\displaystyle \lim\limits_{x\to-\infty} x+\sqrt{x^2+3x} $.

\answer{$-\frac{3}{2} $}
\item $\displaystyle \lim\limits_{x\to\infty}\sqrt{x^2+ax}-\sqrt{x^2+bx}$.

\answer{$\frac{a-b}2$}
\item $\displaystyle \lim\limits_{x\to\infty}\cos x$.

\answer{DNE}
\item $\displaystyle \lim\limits_{x\to\infty}\frac{x^4+x}{x^3-x+2}$.

\answer{$\infty$}
\item $\displaystyle \lim\limits_{x\to\infty}\sqrt{x^2+1}$.

\answer{$\infty$}
\item $\displaystyle \lim\limits_{x\to-\infty}(x^4+x^5)$.

\answer{$-\infty$}
\item $\displaystyle \lim\limits_{x\to-\infty}\frac{\sqrt{1+x^6}}{1+x^2}$.

\answer{$\infty$}
\item $\displaystyle \lim\limits_{x\to\infty}(x-\sqrt{x})$.

\answer{$\infty$}
\item $\displaystyle \lim\limits_{x\to\infty}(x^2-x^3)$.

\answer{$-\infty$}
\item $\displaystyle \lim\limits_{x\to\infty}x\sin x$.

\answer{DNE}
\item $\displaystyle \lim\limits_{x\to\infty}\sqrt{x}\sin x$.

\answer{DNE}
\end{enumerate}
\end{multicols}
\end{problem}

\subsection{Limits involving vertical asymptote}\label{secMPSlimitsVerticalAsymptote}
\begin{problem}
Show the following limits do not exist and compute whether they evaluate to $\infty $, $-\infty$, or neither. 
\begin{multicols}{3}
\begin{enumerate}
\item $\displaystyle\lim_{x\to 3^+} \frac{x^{2}+x-1}{x^2-2x-3} $.
\answer{$\infty$.}
\item $\displaystyle\lim_{x\to 3^-} \frac{x^{2}+x-1}{x^2-2x-3} $.
\answer{$-\infty$.}
\item $\displaystyle\lim_{x\to 1^+} \frac{x^2+1}{\sqrt{x^2+3 }-2} $.
\answer{$\infty$.}
\item $\displaystyle\lim_{x\to 1^-} \frac{x^2+1}{\sqrt{x^2+3 }-2} $.
\answer{$-\infty$.}
\item $\displaystyle\lim_{x\to 2^+} \frac{\sqrt{x^3-8}}{ -x^2+x+2} $.
\answer{$-\infty$.}
\item $\displaystyle\lim_{x\to -1^+} \frac{\sqrt[3]{x^2+2x+1}}{ x^2-2x-3} $.
\answer{$-\infty$.}

\end{enumerate}
\end{multicols}

\end{problem}
\begin{problem}
Evaluate the limit if it exists.
\begin{enumerate}[ref={\fcProblemRef}]
\item \label{problemlimxto3+sqrt(x^2/9-1)/(2x^2-3x-9)} $\displaystyle \lim\limits_{x\to 3^+} \frac{ \sqrt{ \frac{x^2 }{9} - 1 }}{2x^2 -3x-9 }$. \answer{$ \infty$.}

\item $\displaystyle \lim\limits_{x\to -2^-} \frac{\sqrt{ \frac{ x^2}{ 4}-1 }}{2x^2 +3x-2 }$. \answer{$ \infty$.}
\end{enumerate}

\end{problem}
\solution{\ref{problemlimxto3+sqrt(x^2/9-1)/(2x^2-3x-9)}.
We have that 
\[
\begin{array}{rcl}
\displaystyle \lim\limits_{x\to 3^+}\frac{\sqrt{\frac{x^2}9-1} }{2x^2-3x-9}&=&\displaystyle  \lim\limits_{x\to 3^+} \frac{\sqrt{(\frac{x}3-1)(\frac{x}3+1)} }{2(x+\frac{3}2)(x-3)}= \lim\limits_{x\to 3^+}
\frac{\left(\frac{1}3 (x-3)(\frac{x}3+1) \right)^{ \frac{1}2}}{ 2(x+\frac{3}2)(x-3)} \\~\\
&=&\displaystyle  \lim\limits_{x\to 3^+} \frac{\sqrt{\frac{1}3\left(\frac{x}3+1\right)} }{2\left(x+\frac{3}2\right)(x-3)^{\frac12}}= \lim\limits_{x\to 3^+} \frac{\sqrt{\underbrace{ \frac{1}3\left(\frac{x}3+1\right)}_{\to\frac23 }} }{\underbrace{2\left(x+\frac{3}2\right)}_{\to 9}\underbrace{(x-3)^{\frac12}}_{\to 0^+}}=\infty,
\end{array}
\]
where the latest term is $+\infty$ because it is of the form $\frac{(+)}{(+)(+)}$.
}
\subsection{Find the Horizontal and Vertical Asymptotes}\label{secMPShorAndVertAsymptotes}
\begin{problem}
\begin{problem}(Textbook, page 235, problems 33-38).
Find the horizontal and vertical asymptotes of each curve. Check your work by plotting the function using the internet.
\begin{multicols}{3}
\begin{enumerate}
\item $y=\frac{2x+1}{x-2}$.
\item $y=\frac{x^2+1}{2x^2-3x-2}$.
\item $y=\frac{2x^2+x-1}{x^2+x-2}$.
\item $y=\frac{1+x^4}{x^2-x^4}$.
\item $y=\frac{x^3-x}{x^2-6x+5}$.
\item $y=\frac{x-9}{\sqrt{4x^2+3x+2}}$.
\end{enumerate}
\end{multicols}
\end{problem}
\end{problem}
\input{../../modules/curve-sketching/homework/asymptotes-vertical-horizontal-1-subproblem-1-solution}

\solution{\ref{problemAsymptotesy=x/(sqrt(x^2+3) -2x)}

\textbf{Vertical asymptotes.} A function $f(x)$ has a vertical asymptote at $x=a$ if $\lim\limits_{x\to a} f(x)=\pm \infty$. 

The function is algebraic, and therefore has a finite limit at every point it is defined (i.e., no asymptote). Therefore the function can have vertical asymptotes only for those $x$ for which $f(x)$ is not defined. The function is not defined for 

\[
\begin{array}{rcll|l}
\sqrt{x^2+3}-2x&=&0\\
\sqrt{x^2+3}&=&2x&&\begin{array}{l} \text{square both sides}\\\text{may introduce extraneous solutions} \end{array}\\
x^2+3&=&4x^2\\
3x^2-3&=&0\\
3(x-1)(x+1)&=&0\\
x=1 \quad &\text{or}& \cancel{ x=-1}\\
&&x=-1 \text{ is extraneous:}\\
&& \sqrt{(-1)^2+3}-(-1)2=4\neq 0
\end{array}
\]

$x=-1$ is indeed a vertical asymptote:
\[
\lim\limits_{x\to 1^+}  \frac{x}{\sqrt{x^2+3} -2x}=\infty \quad \quad \quad 
\lim\limits_{x\to 1^-}  \frac{x}{\sqrt{x^2+3} -2x}=-\infty .
\]
\textbf{Horizontal asymptotes.} 
\[
\begin{array}{rcll|l}
\displaystyle \lim\limits_{x\to -\infty}  \frac{x}{\sqrt{x^2+3} -2x}&=&\displaystyle \lim\limits_{x\to - \infty} \frac{1}{\frac{\sqrt{x^2+3}}{x} -2} \\
&=& \displaystyle \lim\limits_{x\to - \infty} \frac{1}{-\sqrt{\frac{x^2+3}{x^2}} -2}   && \frac{1}{x} =- \sqrt{\frac{1}{x^2} } \text{ when } x<0\\
&=&\displaystyle \lim\limits_{x\to - \infty} \frac{1}{-\sqrt{1+\frac{3}{x^2}} -2}  \\
&=& \displaystyle \frac{1}{-\sqrt{1+0}-2}\\
&=&\displaystyle -\frac{1}{3}.\\
\displaystyle \lim\limits_{x\to -\infty}  \frac{x}{\sqrt{x^2+3} -2x}&=&\displaystyle \lim\limits_{x\to  \infty} \frac{1}{\frac{\sqrt{x^2+3}}{x} -2} \\
&=& \displaystyle \lim\limits_{x\to  \infty} \frac{1}{\sqrt{\frac{x^2+3}{x^2}} -2}   && \frac{1}{x} = \sqrt{\frac{1}{x^2} } \text{ when } x>0\\
&=&\displaystyle \lim\limits_{x\to  \infty} \frac{1}{\sqrt{1+\frac{3}{x^2}} -2}  \\
&=& \displaystyle \frac{1}{\sqrt{1+0}-2}\\
&=&\displaystyle -1.\\
\end{array}
\]
Therefore $y=-\frac{1}{3}$ and $y=-1$ are the two horizontal asymptotes. 


A computer generated graph confirms our computations.

\psset{xunit=0.2cm, yunit=0.2cm}
\begin{pspicture}(-16, -20)(16,17)
\tiny
\fcAxesStandard{-15}{-19.32133}{15}{16.190354}
\fcXTick{10}
\rput[t](10, -0.6){$10$}
\newcommand{\theFun}{x x x mul 3 add sqrt -2 x mul add div}
%Function formula: \frac{2 x}{\sqrt{x^{2}+x+3}-3}
\psplot[linecolor=\fcColorGraph, plotpoints=1000]{1.036}{15}{\theFun }
%Function formula: \frac{2 x}{\sqrt{x^{2}+x+3}-3}
\psplot[linecolor=\fcColorGraph, plotpoints=1000]{-15}{0.961}{\theFun }
\psline[linestyle=dotted](1,-19.3)(1,16.1)
\psline[linestyle=dashed, linecolor=blue](! -15 -1  3 div)(!15 -1 3 div)
\psline[linestyle=dashed, linecolor=blue](-15, -1)(15, -1)
\rput[b](-8, 0.2){$y=-\frac{1}{3}$}
\rput[t](8, -2){$y=-1$}
\rput[bl](5,5){$y=\frac{x}{\sqrt{x^2+3} -2x}$}
\rput[l](1.6,-8){$x=1$}
\end{pspicture}


}



\begin{problem}
Evaluate the limit if it exists.
\begin{enumerate}
\item $\lim\limits_{x\to-\infty}\sqrt{x^2+x}-\sqrt{x^2-x}$. \answer{$-1 $.}
\item $\lim\limits_{x\to\infty}\sqrt{x^2+2x}-\sqrt{x^2-2x} $. \answer{ $2 $.}
\end{enumerate}

\solution{
\[ \begin{array}{rcl}
\displaystyle\lim_{x\to -\infty} \sqrt{x^2+x}-\sqrt{x^2-x} &=&\displaystyle\lim_{x\to -\infty} \left(\sqrt{x^2+x}-\sqrt{x^2-x}\right) \frac{ \left(\sqrt{x^2+x}+\sqrt{x^2-x}\right) }{\left( \sqrt{x^2+x}+\sqrt{x^2-x}\right)}
\\
&=&\displaystyle \lim_{x\to -\infty} \frac{x^2+x-(x^2-x) }{\sqrt{x^2+x}+\sqrt{x^2-x} } = \lim_{x\to -\infty} \frac{2x \frac{1}{x} }{\left(\sqrt{x^2+x}+\sqrt{x^2-x} \right)\frac{1}{x}} 
\\&=&\displaystyle \lim_{x\to -\infty} \frac{2}{\frac{\sqrt{x^2+x}}x+\frac{\sqrt{x^2-x}}x }= \lim_{x\to -\infty} \frac{2}{ {\color{red}-} \sqrt{\frac{x^2+x}{x^2}} {\color{red}-} \sqrt{\frac{x^2-x}{x^2}} }\\
&=&\displaystyle \lim_{x\to -\infty} \frac{2}{ - \sqrt{1+\frac1x} - \sqrt{1-\frac1x} }=\frac{2}{-\sqrt{1+0}-\sqrt{1-0}}=-1
\end{array}
\]
The sign highlighted in red arises from the fact that, for negative $x$, we have that $ x={\color{red}-}\sqrt{x^2}$.
}
\end{problem}

\subsection{Continuity}
\subsection{Conceptual problems} \label{secMPScontinuityConceptual}
\begin{problem}
For which values of $x$ is $f$ continuous?
\begin{itemize}
\item $f(x)=\doublebrace{0}{\mathrm{if~} x\mathrm{~is~rational}}{1}{\mathrm{if~}x~\mathrm{is~irrational}}$
\item $f(x)=\doublebrace{0}{\mathrm{if~} x\mathrm{~is~rational}}{x}{\mathrm{if~}x~\mathrm{is~irrational}}$
\end{itemize}
\end{problem}
\begin{problem}
Show that $f(x)$ is continuous at all irrational points and discontinuous at all rational ones.
\[
f(x)=\doublebrace{\frac{1}{q^2}}{\mathrm{if~}x\mathrm{~is~rational~and~} x=\frac{p}{q} }{0}{\mathrm{if~}x~\mathrm{is~irrational}}
\]
where in the first item $p,q$ are relatively prime integers (i.e., integers without a common divisor).

\end{problem}
\subsection{Continuity and Piecewise Defined Functions} \label{secMPScontinuityPiecewise}
\begin{problem}
\begin{problem}
(Textbook, page 91, problem 23-24). Find the (implied) domain of $f(x)$. Extend the definition of $f$ at $x=2$ to make $f$ continuous at $2$.
\begin{multicols}{2}
\begin{enumerate}
\item $f(x)=\frac{x^2-x-2}{x-2}$.
\item $f(x)=\frac{x^3-8}{x^2-4}$.
\end{enumerate}
\end{multicols}
\end{problem}
\begin{problem}(Textbook, page 92, problem 41-43)
Find the numbers $x$ for which $f$ is discontinuous. At which of these numbers is $f$ continuous from the right, from the left, or neither? 
\begin{multicols}{2}
\begin{enumerate}
\item $f(x)=\triplebrace{1+x^2}{\mathrm{if~} x\leq 0 } {2-x}{\mathrm{if~} 0<x\leq 2}{(x-2)^2}{\mathrm{if~} x>2}$.
\item $f(x)=\triplebrace{x+1}{\mathrm{if~} x\leq 1 }{\frac{1}{x}}{\mathrm{if~}1<x<3}{\sqrt{x-3}}{\mathrm{if~} x\geq 3}$.
\item $f(x)=\triplebrace{x+2}{\mathrm{if~} x<0}{2x^2}{\mathrm{if~} 0\leq x\leq 1 }{2-x}{\mathrm{if~}x>1 }$
\end{enumerate}
\end{multicols}
\end{problem}
\begin{problem} (Textbook, page 92, problem 46)
Find the values of $a$ and $b$ that make $f$ continuous everywhere.
\[f(x)=\triplebrace{\frac{x^2-4}{x-2}}{\mathrm{if~}x<2}{ax^2-bx+3}{\mathrm{if~}2\leq x<3}{2x-a+b}{x\geq 3} \quad .\]
\end{problem}

\end{problem}
\begin{problem}
Find the numbers $x$ for which $f$ is discontinuous. At which of these numbers is $f$ continuous from the right, from the left, or neither? 
\begin{multicols}{2}
\begin{enumerate}
\item $f(x)=
\left\{\begin{array}{ll}
2+x^2 & \text{if~} x\leq 0 \\
-2x& \text{if~} 0<x\leq 2\\
-x^2& \text{if~} x>2\\
\end{array}\right.
$.
\item $f(x)=
\left\{\begin{array}{ll}
x+1 & \text{if~} x\leq 1\\
\frac{1}{x}& \text{if~}1<x<2 \\
\sqrt{x-2}& \text{if~} x\geq 2
\end{array}\right.$.
\item $f(x)=
\left\{\begin{array}{ll}
x+2& \text{if~} x<0\\
2x^2& \text{if~} 0\leq x\leq 1\\ 
2-x& \text{if~}x>1 \end{array}\right.$.
\end{enumerate}
\end{multicols}

\end{problem}
\begin{problem}
Find the values of $a$ and $b$ that make $f$ continuous everywhere.
\begin{enumerate}[ref={\fcProblemRef}]
\item 
$\displaystyle
f(x)= \left\{\begin{array}{ll}
\displaystyle 1 & {\text{if~}x<0}\\ 
\displaystyle {ax+b}& {\text{if~}0\leq x<1}\\ 
{2x}& \text{if~}{x\geq 1}\end{array}\right. .
$
\item
$\displaystyle
f(x)= \left\{\begin{array}{ll}
\displaystyle {\frac{x^2-1}{x-1}} & {\text{if~}x<1}\\ 
\displaystyle {ax^2-bx+3}& {\text{if~}1\leq x<3}\\ 
{2x-a+b}& \text{if~}{x\geq 3}\end{array}\right. .
$

\end{enumerate}

\end{problem}
\subsection{Intermediate Value Theorem}\label{secMPSintermediateValueTheorem}
\begin{problem}
\begin{problem}
(Textbook, page 92, problem 51-56) Use the Intermediate Value Theorem to show that there is a real number solution of the given equation in the specified interval. 
\begin{multicols}{2}
\begin{itemize}
\item $x^4+x-3=0$ where $x\in (1,2)$.
\item $\sqrt[3]{x}=1-x$ where $x\in (0,2) $.
\item $\cos x=x$, where $x\in (0,1)$.
\item $\sin x=x^2-x$, where $x\in (1,2)$.
\item $\cos x=x^3$, where $x\in \mathbb R$ (i.e., $x$ is an arbitrary real number).
\item $x^5-x^2+2x+3$, where $x\in \mathbb R$.
\end{itemize}
\end{multicols}
\end{problem}
\end{problem}
\begin{problem}
\begin{enumerate}
\item (1) Solve the equation $x^2+13x+41=1$.  (2) Use the intermediate value theorem to prove that the equation $x^2+13x+41=\sin  x$ has at least two solutions, lying between the two numbers found in (1).
\solution{
\noindent (1)
\begin{eqnarray*}
x^2+13x+41&=&1\\
x^2+13x+40&=&0\\
(x+5)(x+8)&=&0\quad .
\end{eqnarray*}
Therefore the two solutions are $x_1=-5$ and $x_2=-8$.

\noindent (2) Consider the function
\[
f(x)=x^2+13x+41-\sin x\quad.
\]
Our strategy for proving $f(x)=0$ has a solution consists in finding a number $a$ such that $f(a)<0$ and a number $b$ such that $f(b)>0$, and then using the Intermediate Value Theorem (IVT) with $N=0$.

Let
\[
g(x)=x^2+13x+41,
\]
and so $f(x)=g(x)-\sin x$. We have no techniques for evaluating $\sin x$ without calculator, but we do have all knowledge necessary to evaluate $g(x)$. Indeed, from high school we know that the lowest point of the parabola $g(x)$ is located at $x=-\frac{13}2=-6.5$. Then $g(-6.5)= -1.25$. Therefore
\[
f(-6.5)=g(-6.5)-\sin(-6.5)= g(-6.5)+\sin (6.5)=-1.25+\sin 6.5 \leq -0.25,
\]
where for the very last inequality we use the fact that $\sin 6.5< 1 $ (remember $\sin t\leq 1$ for all real values of $t$).

On the other hand,
\[f(-5)= g(-5)-\sin (-5) = 1+\sin 5> 0\]
as $\sin 5 >-1$ (remember $\sin t\geq -1$ for all real values of $t$). Therefore $f(-5)>0$ and $f(-6.5)<0$ and by the Intermediate Value Theorem (IVT) $f(x)=0$ has a solution in the interval $x\in (-6.5, -5)$.

Proving $f(x)=0$ has a solution in the interval $x\in (-8, -6.5)$ is similar and we leave it to the student.

Below is a computer generated plot of the function with the use of which we can visually verify our answer.

\psset{xunit=1cm, yunit=1cm}
\begin{pspicture}(-9, -5)(1,5)
\psframe*[linecolor=white](-9,-5)(1,5)
\tiny
\psaxes[ticks=none, labels=none]{<->}(0,0)(-9,-4.5)(1,4.5)
\fcLabels{1}{5}
\fcXTickWithLabel{-6.5}{$-6.5$}
\fcXTickWithLabel{-8}{$-8$}
\fcXTickWithLabel{-5}{$-5$}
%Function formula: (x)^{2}+40+13 (x)
\rput(-4.5,-2){$y=x^{2}+13 x+40$}
\psplot[linecolor=grey!30, plotpoints=1000]{-9}{-4}{x 13 mul 40 x 2 exp add add }
\rput(-6.5,3){$y=x^{2}+41+13 x- \sin x$}
\psplot[linecolor=\fcColorGraph, plotpoints=1000]{-9}{-4}{x 57.29578 mul sin -1 mul x 13 mul 41 x 2 exp add add add }

\end{pspicture}
}

\item (1) Solve the equation $x^2-15x+55=1$.  (2) Use the intermediate value theorem to prove that the equation $x^2-15x+55=\cos  x$ has at least two solutions, lying between the two numbers found in (1).
\end{enumerate}

\end{problem}
\input{../../modules/continuity/homework/IVT-problems2-solutions}
\section{Inverse Functions}\label{secMPSInverseFunctions}
\begin{problem}
% begin homework inverse-functions2
Find the inverse function and its domain. 
\begin{enumerate}
\item  $y=\ln (x+3)$.
\answer{$f^{-1}(x)=e^x-3$}

%\solution{%
%\begin{align*}
%y & = \ln (x+3) \\
%e^y & = e^{\ln (x+3)} \\
%e^y & = x + 3 \\
%e^y - 3 & = x \\
%\text{Therefore} \quad f^{-1}(y) & = e^y - 3.
%\end{align*}
%The domain of $e^y$ is all real numbers, so the domain of $f^{-1}$ is all real numbers.  
%}%

\item $f(x)=e^{x^3}$.
\answer{$f^{-1}(x)=\sqrt[3]{\ln x}, \quad x>0$}

\item $y=(\ln x)^2$, $x\geq 1$.
\answer{$f^{-1}(x)=e^{\sqrt{x}}, \quad x\geq 0 $}

\pointsii{5}  $y=\frac{e^x}{1+2e^x}$.
\hiddenanswer{$f^{-1}(x)= \ln \left(\frac{x}{1-2x}\right) $, \quad $x\in (0, \frac12) $}

\solution{%
\begin{align*}
y & = \frac{e^x}{1+2e^x} \\
y(1+2e^x) & = e^x \\
y & = e^x(1-2y) \\
\frac{y}{1-2y} & = e^x \\
\ln\frac{y}{1-2y} & = \ln e^x \\
\ln\frac{y}{1-2y} & = x \\
\text{Therefore} \quad f^{-1}(y) & = \ln\frac{y}{1-2y}.
\end{align*}
The natural logarithm function is only defined for positive input values.  
Therefore the domain is the set of all $y$ for which 
\begin{align*}
\frac{y}{1-2y} & > 0.
\end{align*}
This inequality holds if the numerator and denominator are both positive or both negative.  
This happens if either
\begin{enumerate}
\item  $y > 0$ and $y < 1/2$, or 
\item  $y < 0$ and $y > 1/2$.
\end{enumerate}
The latter option is impossible, so the domain is $\{ y \in \mathbb{R} \ | \ 0 < y < 1/2\}$.  
}%

\end{enumerate}
% end homework inverse-functions2

\end{problem}
\input{../../modules/inverse-functions/homework/inverse-functions2-solutions}

\begin{problem}
% begin homework inverse-functions3
Find the inverse function. You are asked to do the algebra only; you are not asked to determine the domain or range of the function or its inverse. 
\begin{enumerate} [ref={\fcProblemRef}]
\item $f(x)= 3x^2+4x-7$, where $x\geq -\frac{2}{3}$.

\answer{$f^{-1}(x)= -\frac{2}3+\frac{\sqrt{25+3x}}{3}, \quad x\geq -\frac{25}{3}$}
\item $f(x)= 2x^2+3x-5$, where $x\geq -\frac{3}{4}$.

\answer{$f^{-1}(x)=-\frac{3}{4}+\frac{\sqrt{49+8x}}{4}, \quad x\geq -\frac{49}{8}$}
\item $\displaystyle f(x)= \frac{2x+5}{x-4}$, where $x\neq 4$.

\answer{$f^{-1}(x)=\frac{4x+5}{x-2}, \quad x\neq 2$}
\pointsii{3} \label{problemFindInversef=(3x+5)/(2x-4)} $\displaystyle f(x)= \frac{3x+5}{2x-4}$, where $x\neq 2$.

\hiddenanswer{$\displaystyle f^{-1}(x) = \frac{ 4 x +5}{2x-3}, \quad x\neq \frac{3}{2}$}
\item \label{problemFindIversef=(5x+6)/(4x+5)}  $\displaystyle f(x)= \frac{5x+6}{4x+5}$.

\answer{$f^{-1}(x)= \frac{-5x+6}{4x-5}$, $x\neq \frac{5}{4}$}
\item  $\displaystyle f(x)= \frac{2x-3}{-3x+4}$.

\answer{$f^{-1}(x)=\frac{4x+3}{3x+2}  $, $x\neq -\frac{2}{3}$}
\end{enumerate}
% end homework inverse-functions3

\end{problem}
\solution{\ref{problemFindInversef=(3x+5)/(2x-4)}
\begin{align*}
y & = \frac{3x+5}{2x-4} \\
y(2x-4) & = 3x+5 \\
2xy-4y & = 3x+5 \\
2xy-3x & = 4y+5 \\
x(2y-3) & = 4y+5 \\
x & = \frac{4y+5}{2y-3} \\
\text{Therefore}\quad f^{-1}(y) & = \frac{5+4y}{2y-3}.
\end{align*}
}%

\begin{problem}
Find the inverse function $f^{-1}$. Plot roughly by hand $y=f(x)$. Using the plot of $y=f(x)$, plot roughly by hand $f^{-1}(x)$. Indicate the relationship between the graph of $f(x)$ and $f^{-1}(x)$.
\begin{enumerate}
\item $f(x)= x^2+2x-2$,\quad \quad \quad $ x\geq -1$. 
\answer{
$f^{-1}(x)=\sqrt{x+3}-1$
\psset{xunit=0.2cm, yunit=0.2cm}
\begin{pspicture}(-3, -5)(6,5) 
\psframe*[linecolor=white](-3,-5)(6,5) 
\tiny 
\psaxes[ticks=none, labels=none]{<->}(0,0)(-3,-4.5)(6,4.5)
\psLabels{6}{5}
%Function formula: (x+3)^{1/2}-1 
\psplot[linecolor=\psColorGraph, plotpoints=1000]{-3}{6}{-1 3 x add 0.5 exp add }
%Function formula: x^{2}+2 x-2 
\psplot[linecolor=\psColorGraph, plotpoints=1000]{-1}{2}{-2 x 2 mul add x 2 exp add }
\end{pspicture} 
}
\item $f(x)= x^2+x-2$, \quad \quad \quad $ x\geq -\frac{1}{2}$.

\answer{
$f^{-1}(x)=\frac{ \sqrt{4 x+9}-1}2$
\psset{xunit=0.2cm, yunit=0.2cm}
\begin{pspicture}(-2.25, -5)(4,5) 
\psframe*[linecolor=white](-2.25,-5)(4,5) 
\tiny 
\psaxes[ticks=none, labels=none]{<->}(0,0)(-2.25,-4.5)(4,4.5)
\psLabels{4}{5}
%Function formula: 1/2 (4 x+9)^{1/2}-1/2 
\psplot[linecolor=\psColorGraph, plotpoints=1000]{-2.25}{4}{-0.5 9 x 4 mul add 0.5 exp 0.5 mul add }
%Function formula: x^{2}+x-2 
\psplot[linecolor=\psColorGraph, plotpoints=1000]{-0.5}{2}{-2 x add x 2 exp add }
\end{pspicture} 
}
\end{enumerate}
\end{problem}


\section{Logarithms and Exponent Basics}\label{secMPSLogarithmsExponentsBasics}
\subsection{Exponents Basics}
\begin{problem}
% begin homework exponent-simplfy
Express each of the following as a single power.  

\begin{enumerate}[ref={\fcProblemRef}]
\item   $\displaystyle\frac{2^5\cdot 2^7}{2\sqrt{2}}$

\answer{$2^{10.5}=2^{\frac{21}{2}}$}
\pointsii{2}  $\label{problemSimplify3^23^(-1)/(3^3sqrt(3^3))} \displaystyle\frac{3^2\cdot 3^{-1}}{3^3\cdot \sqrt{3^3}}$

\answer{$3^{-\frac{7}{2}}$}
%solution to this problem moved to separate file
\item   $\displaystyle \frac{\pi^3}{\pi^{-1}\sqrt{\pi^5}}$

\answer{ $\pi^{\frac{3}{2}}$}
\end{enumerate}
% end homework exponent-simplify

\end{problem}
\solution{\ref{problemSimplify3^23^(-1)/(3^3sqrt(3^3))}.

\begin{align*}
\frac{3^2\cdot 3^{-1}}{3^3\cdot\sqrt{3^3}} & = \frac{3^2\cdot 3^{-1}}{3^3\cdot (3^3)^{\frac{1}{2}}} \\
 & = \frac{3^2\cdot 3^{-1}}{3^3\cdot 3^{\frac{3}{2}}} \\
 & = \frac{3^{2-1}}{3^{3+\frac{3}{2}}} \\
 & = \frac{3^{1}}{3^{\frac{9}{2}}} \\
 & = 3^{1-\frac{9}{2}} \\
 & = 3^{-\frac{7}{2}}.
\end{align*}
}%


\subsection{Logarithm Basics}
\begin{problem}
% begin homework logarithms-basic2
Use the definition of a logarithm to evaluate each of the following without using a calculator. The answer key has not been proofread, use with caution.

\begin{enumerate}[ref={\fcProblemRef}]
\item   $\displaystyle \log_2 16$


\answer{$4$}
\item   $\displaystyle\log_3 \left(\frac{1}{9}\right)$

\answer{$-2$}
\item   $\displaystyle\log_{10} 1000$

\answer{$3$}
\item   $\displaystyle\log_{6} 36^{-\frac{2}{3}}$

\answer{$-\frac{4}{3}$}
\item   $\displaystyle\log_{2} (8\sqrt{2})$

\answer{$\frac{7}{2}$}
\item   $\displaystyle\log_{\frac{1}{2}} (4)$

\answer{$-2$}
\item   $\displaystyle\log_{\frac{1}{9}} (\sqrt{3})$

\answer{$ -\frac{1}{4}$}
\end{enumerate}
% end homework logarithms-basic2

\end{problem}
\solution{\ref{problemSimplifylog_7(49^x/343^y)}.

\[
\begin{array}{rcl}
\log_7\left(\frac{49^x}{343^y}\right) & =& \log_749^x - \log_7343^y \\
 & =& x\log_749 - y\log_7343 \\
\text{But }49 = 7^2\text{ and }343=7^3,\text{ therefore }
\log_7\left(\frac{49^x}{343^y}\right) & =& 2x-3y.
\end{array}
\]
}%
\begin{problem}
% begin homework logarithms-combine
Express each of the following as a single logarithm. If possible, compute the logarithm without using a calculator. The answer key has not been proofread, use with caution.

\begin{enumerate}[ref={\fcProblemRef}]
\item   $\ln 4 + \ln 6 - \ln 5$.

\answer{$\ln \left(\frac{24}{5} \right) $}
\item \label{problem2ln(2)-3ln(3)+4ln(4)} $2\ln 2 - 3\ln 3 + 4\ln 4$.

\answer{$ \ln \left( \frac{1024}{27}\right)$}
\item   $\ln 36 - 2\ln 3 - 3\ln 2$.

\answer{$-\ln 2=\ln \left(\frac{1}{2}\right) $}

\item $\log_2(24)-\log_{4}9$.

\answer{$3$}

\item $\log_7(24)+\log_{\frac{1}{7}}3-\log_{49} (64)$.

\answer{$0$}
\item $\log_3(24)+\log_{3}\left(\frac{3}{8}\right)$.

\answer{$ 2$}

\end{enumerate}
% end homework logarithms-combine

\end{problem}
\solution{\ref{problem2ln(2)-3ln(3)+4ln(4)}.
\begin{align*}
2\ln 2 - 3\ln 3 + 4\ln 4 & = \ln 2^2 - \ln 3^3 + \ln 4^4 \\
 & = \ln 4 - \ln 27 + \ln 256 \\
 & = \ln \Big( \frac{4}{27}\Big) + \ln 256 \\
 & = \ln \Big( \frac{4\cdot 256}{27}\Big) \\
 & = \ln \Big( \frac{1024}{27}\Big).
\end{align*}

$\frac{1024}{27}$ is not a rational power of $e$, therefore $ \ln \left( \frac{1024}{27}\right)$ is not a rational number and there are no further simplifications of the answer (except possibly a numerical approximation with a calculator or equivalent). 
}%

\solution{\ref{problemlog_7(24)+log_(1/7)3-log_(49)64}

\[
\renewcommand{\arraystretch}{2}
\begin{array}{@{}r@{}c@{}l@{}l|l}
\displaystyle \log_{7}{(24)}+\log_{\frac{1}{7}}{(3)}-\log_{49}{(64)}&=&\displaystyle  \log_{7}{(24)}+ \frac{\log_{7}{(3)}}{ \log_{7}{\left(\frac{1}{7}\right)}}- \frac{\log_{7}{ (64)}}{\log_{7}{(49)}} && \text{common base}\\
&=&\displaystyle \log_{7}{(24)}+ \frac{\log_{7}{(3)}}{-1} -\frac{\log_{7}{(64)}}{2}&&\text{simplify logarithms}\\
&=&\displaystyle \log_{7}{(24)}-\log_{7}{(3)}-\frac{1}{2}\log_{7}{(64)}\\
&=&\displaystyle \log_{7}{\left(\frac{24}{3}\right)}- \log_{7}{ \left(64^{ \frac{ 1}{2}}\right) }&&\renewcommand{\arraystretch}{1.2} \begin{array}{@{}l}\text{rule: } \log_ax-\log_ay=\log_a\left( \frac{x}{y}\right)  \\ \text{rule: } \log_ax^r=r\log_ax \end{array} \\
&=&\displaystyle \log_{7}{\left(8\right)}-\log_{7}\left(\sqrt{64} \right)\\
&=&\displaystyle \log_78-\log_78 =0 &&\text{alternatively: }\\
&=&\displaystyle \log_7\left(\frac{8}{8}\right)\\
&=&\displaystyle \log_7(1)\\
&=&0.
\end{array}
\]
}
\begin{problem}
\input{../../modules/logarithms/homework/logarithms-basic-properties-1}
\end{problem}
\solution{\ref{problemSimplifylog_7(49^x/343^y)}.

\[
\begin{array}{rcl}
\log_7\left(\frac{49^x}{343^y}\right) & =& \log_749^x - \log_7343^y \\
 & =& x\log_749 - y\log_7343 \\
\text{But }49 = 7^2\text{ and }343=7^3,\text{ therefore }
\log_7\left(\frac{49^x}{343^y}\right) & =& 2x-3y.
\end{array}
\]
}%



\subsection{Some Problems Involving Logarithms}
\begin{problem}
% begin homework inverse-functions2
Find the inverse function and its domain. 
\begin{enumerate}
\item  $y=\ln (x+3)$.
\answer{$f^{-1}(x)=e^x-3$}

%\solution{%
%\begin{align*}
%y & = \ln (x+3) \\
%e^y & = e^{\ln (x+3)} \\
%e^y & = x + 3 \\
%e^y - 3 & = x \\
%\text{Therefore} \quad f^{-1}(y) & = e^y - 3.
%\end{align*}
%The domain of $e^y$ is all real numbers, so the domain of $f^{-1}$ is all real numbers.  
%}%

\item $f(x)=e^{x^3}$.
\answer{$f^{-1}(x)=\sqrt[3]{\ln x}, \quad x>0$}

\item $y=(\ln x)^2$, $x\geq 1$.
\answer{$f^{-1}(x)=e^{\sqrt{x}}, \quad x\geq 0 $}

\pointsii{5}  $y=\frac{e^x}{1+2e^x}$.
\hiddenanswer{$f^{-1}(x)= \ln \left(\frac{x}{1-2x}\right) $, \quad $x\in (0, \frac12) $}

\solution{%
\begin{align*}
y & = \frac{e^x}{1+2e^x} \\
y(1+2e^x) & = e^x \\
y & = e^x(1-2y) \\
\frac{y}{1-2y} & = e^x \\
\ln\frac{y}{1-2y} & = \ln e^x \\
\ln\frac{y}{1-2y} & = x \\
\text{Therefore} \quad f^{-1}(y) & = \ln\frac{y}{1-2y}.
\end{align*}
The natural logarithm function is only defined for positive input values.  
Therefore the domain is the set of all $y$ for which 
\begin{align*}
\frac{y}{1-2y} & > 0.
\end{align*}
This inequality holds if the numerator and denominator are both positive or both negative.  
This happens if either
\begin{enumerate}
\item  $y > 0$ and $y < 1/2$, or 
\item  $y < 0$ and $y > 1/2$.
\end{enumerate}
The latter option is impossible, so the domain is $\{ y \in \mathbb{R} \ | \ 0 < y < 1/2\}$.  
}%

\end{enumerate}
% end homework inverse-functions2

\end{problem}

\begin{problem}
(Textbook, page 409, problem 27-36)
Solve each equation for $x$. Using a calculator give an ($\approx$) answer in decimal notation. Using calculator verify your approximate solutions.
\begin{multicols}{2}
\begin{enumerate}
\item $e^{6-4x}=6$.

\answer{$\frac{6-\ln 6 }{4}\approx 1.052$}
\item $\ln (3x-10)=2$.

\answer{$\frac{e^2+10}{3}\approx 5.796$}
\item $\ln (x^2-1)=3$.

\answer{$\pm \sqrt{e^3+1}\approx \pm 4.592$}
\item $e^{2x}-3e^x+2=0$.

\answer{$x=\ln 2\approx 0.693, ~~~, x=0$}
\item $2^{x-5}=3$.

\answer{$\log_2 3+5= \frac{\ln 3}{\ln 2}+5 \approx 6.585$}
\item $\ln x+\ln (x-1)=1$.

\answer{$\frac{1}{2}\left(1+\sqrt{1+4e}\right)\approx 2.223$}
\item $e^{3x+1}=k$.

\answer{$\frac{\ln k-1}{3}$}
\item $\log_2(mx)=c$.

\answer{$\frac{2^c}{m}$}
\item $e- e^{-2x}=1$.

\answer{$-\frac12\ln (e-1)\approx -0.271$}
\item $10(1+e^{-x})^{-1}=3$.

\answer{$-\ln \frac73 =\ln \frac37 \approx -0.847$}
\item $\ln (\ln x)=1$.

\answer{$e^e\approx 15.154$}
\item $e^{e^x}=10$.

\answer{$\ln (\ln 10)\approx 0.834$}
\item $e^{2x}-e^x-6=0$.

\answer{$x=\ln 3$}
\item $\ln(2x+1)=2-\ln x$.

\answer{$\frac{-1+\sqrt{1+8e^2}}{4}\approx 1.688$}
\end{enumerate}
\end{multicols}


\end{problem}

\solution{\ref{problem2^(x-3)=5}
\[\begin{array}{rcll|l}
\displaystyle 2^{x-3} &=& 5 &&\displaystyle  \text{take } \log_2 \\
x-3&=&\displaystyle  \log_2(5) &&\text{add } 3 \text{ to both sides}\\
x&=&\displaystyle \log_2(5)+3 &&\text{answer is complete} \\
&=&\displaystyle \frac{\ln 5}{\ln 2}+3 && \text{optional step: convert to }\ln\\
&\approx &5.321928095 &&\text{calculator}
\end{array}
\]
}

\solution{ \ref{probleme-e^(-2x)=1}

\[
\begin{array}{rcll|l}
\displaystyle e-e^{-2x}&=&1\\
\displaystyle e^{-2x}&=&e-1&& \text{apply }\ln\\
\displaystyle \ln e^{-2x}&=&\displaystyle \ln(e-1)\\
-2x&=&\displaystyle\ln(e-1)\\
x&=&\displaystyle-\frac{1}{2}\ln(e-1)\\
&\approx& -0.270662427&&\text{calculator}
\end{array}
\]

}

\solution{\ref{problemlnx+ln(x-1)=1} %
\[
\begin{array}{rcl}
\displaystyle \ln x + \ln (x-1) & =& 1 \\
\displaystyle \ln \left(x^2-x\right) & =& 1 \\
\displaystyle e^{\ln (x^2-x)} & =& e^1 \\
\displaystyle x^2-x & =& e \\
\displaystyle x^2-x-e & =& 0 \\
\text{Quadratic formula:}\quad x & = &\displaystyle \frac{-(-1)\pm \sqrt{(-1)^2-4(1)(-e)}}{2(1)} \\
& =&\displaystyle  \frac{1\pm \sqrt{1+4e}}{2}.
\end{array}
\]
However $\frac{1-\sqrt{1+4e}}{2}$ is negative, so $\ln\left( \frac{1-\sqrt{1 + 4e}}{2} \right)$ is undefined.  
Hence the only solution is $x = \frac{1+\sqrt{1+4e}}{2}\approx 2.2229$.  
}%

\solution{\ref{problemSolve4^(3x)-2^(3x+2)-5}

\[
\begin{array}{rcll|l}
\displaystyle 4^{3x}-2^{3x+2}-5&=&0 \\
\displaystyle 4^{3x}-4\cdot 2^{3x}-5&=&0&&\text{Set } \begin{array}{rcl}\displaystyle 2^{3x}&=&u\\ \displaystyle 4^{3x}&=&u^2\end{array} \\
\displaystyle u^2-4u-5&=&0\\
\displaystyle (u-5)(u+1)&=&0\\
\displaystyle u=5&\text{or}& u=-1\\
\displaystyle 2^{3x}=5&&\displaystyle 2^{3x}=-1\\
\displaystyle 3x=\log_2(5)&&\text{no real solution}\\
\displaystyle x=\frac{\log_2 5}{3}\\
\text{Calculator: }x\approx 0.773976
\end{array}
\]
}

\solution{\ref{problemSolve32^x+2(1/2)^(x-1)-7=0}
\[
\begin{array}{rcll|l}
\displaystyle 3\cdot 2^{x}+2 \displaystyle \left(\frac{1}{2}\right)^{x-1}-7&=&0\\
3\cdot 2^{x}+2\displaystyle  \left(\frac{1}{2} \right)^{x}\left( \frac{1}{2}\right)^{-1} -7&=&0\\
\displaystyle 3\cdot 2^{x}+4 \left(\frac{1}{2} \right)^{x} -7&=&0&&\text{Set } 2^x=u \\
\displaystyle 3u+\frac{4}{u}-7&=&0&&\text{Multiply by }u\\
\displaystyle 3u^2-7u+4&=&0\\
\displaystyle (u-1)(3u-4)&=&0\\
\displaystyle u=1&\text{or}&\displaystyle 3u-4=0\\
\displaystyle 2^x=1&&\displaystyle u=\frac{4}{3}\\
x=0&&\displaystyle 2^x=\frac{4}{3}\\
&&\displaystyle x=\log_2\frac{4}{3}=\log_2 4- \log_2 3\\
&&\displaystyle x=2-\log_2 3\\
\text{Calculator:}&&x\approx 0.415037
\end{array}
\]
}





\section{Derivatives}
\subsection{Derivatives and Function Graphs: basics}\label{secMPSderivativesFunGraphsBasics}
\begin{problem}
Match the graph of each the following functions
\begin{multicols}{2}
\begin{enumerate}
\item
\psset{xunit=1cm, yunit=1cm}
\begin{pspicture}(-5, -5)(5,5)
\psframe*[linecolor=white](-5,-5)(5,5)
\psaxes[ticks=none, labels=none]{<->}(0,0)(-2.5,-2.5)(2.5,2.5)
%Function formula: -8 ((x) ((x) (x)))+2 (x)
\psplot[linecolor=red, plotpoints=1000]{-0.8}{0.8}{x 2 mul x x mul x mul -8 mul add }
\end{pspicture}
\item
\psset{xunit=1cm, yunit=1cm}
\begin{pspicture}(-5, -5)(5,5)
\psframe*[linecolor=white](-5,-5)(5,5)
\psaxes[ticks=none, labels=none]{<->}(0,0)(-2.5,-2.5)(2.5,2.5)
%Function formula: -2+x
\psplot[linecolor=red, plotpoints=1000]{1}{2.5}{x -2 add } %Function formula: - (x)
\psplot[linecolor=red, plotpoints=1000]{-1}{1}{x -1 mul } %Function formula: 2+x
\psplot[linecolor=red, plotpoints=1000]{-2.5}{-1}{x 2 add }
\end{pspicture}
\item
\psset{xunit=1cm, yunit=1cm}
\begin{pspicture}(-5, -5)(5,5)
\psframe*[linecolor=white](-5,-5)(5,5)
\psaxes[ticks=none, labels=none]{<->}(0,0)(-2.5,-2.5)(2.5,2.5)
%Function formula: - ((1)/((x)^{2}+1))
\psplot[linecolor=red, plotpoints=1000]{-2.5}{2.5}{1 1 x 2 exp add div -1 mul }
\end{pspicture}
\item
\psset{xunit=1cm, yunit=1cm}
\begin{pspicture}(-5, -5)(5,5)
\psframe*[linecolor=white](-5,-5)(5,5)
\psaxes[ticks=none, labels=none]{<->}(0,0)(-2.5,-2.5)(2.5,2.5)
%Function formula: - (((x)^{2}) ((x) (x)))+(x)^{2}
\psplot[linecolor=red, plotpoints=1000]{-1.46}{1.46}{x 2 exp x x mul x 2 exp mul -1 mul add }
\end{pspicture}
\end{enumerate}
\end{multicols}
to the graph of its derivative among the graphs below
\begin{multicols}{2}
\begin{enumerate}
\item
\psset{xunit=1cm, yunit=1cm}
\begin{pspicture}(-5, -5)(5,5)
\psframe*[linecolor=white](-5,-5)(5,5)
\psaxes[ticks=none, labels=none]{<->}(0,0)(-2.5,-2.5)(2.5,2.5)
%Function formula: (x)/(((x)^{2}+1)^{2})
\psplot[linecolor=blue, plotpoints=1000]{-2.5}{2.5}{x 1 x 2 exp add 2 exp div }
\end{pspicture}

\item \psset{xunit=1cm, yunit=1cm}
\begin{pspicture}(-5, -5)(5,5)
\psframe*[linecolor=white](-5,-5)(5,5)
\psaxes[ticks=none, labels=none]{<->}(0,0)(-2.5,-2.5)(2.5,2.5)
%Function formula: -24 ((x) (x))+2
\psplot[linecolor=blue, plotpoints=1000]{-0.427}{0.427}{2 x x mul -24 mul add }
\end{pspicture}
\item
\psset{xunit=1cm, yunit=1cm}
\begin{pspicture}(-5, -5)(5,5)
\psframe*[linecolor=white](-5,-5)(5,5)
\psaxes[ticks=none, labels=none]{<->}(0,0)(-2.5,-2.5)(2.5,2.5)
%Function formula: -4 ((x)^{3})+2 (x)
\psplot[linecolor=blue, plotpoints=1000]{-1.045}{1.045}{x 2 mul x 3 exp -4 mul add }
\end{pspicture}

\item
\psset{xunit=1cm, yunit=1cm}
\begin{pspicture}(-5, -5)(5,5)
\psframe*[linecolor=white](-5,-5)(5,5)
\psaxes[ticks=none, labels=none]{<->}(0,0)(-2.5,-2.5)(2.5,2.5)
%Function formula: 1
\psplot[linecolor=blue, plotpoints=1000]{1}{2.5}{1}
\fcHollowDotBlue{-1}{1}
%Function formula: -1
\fcHollowDotBlue{-1}{-1}
\psplot[linecolor=blue, plotpoints=1000]{-1}{1}{-1}
\fcHollowDotBlue{1}{-1}
%Function formula: 1
\fcHollowDotBlue{1}{1}
\psplot[linecolor=blue, plotpoints=1000]{-2.5}{-1}{1}
\end{pspicture}

\end{enumerate}
\end{multicols}
Give reasons for your choices. Can you guess formulas that would give a similar (or precisely the same) graph, and confirm visually your guess using a graphing device?

\end{problem}
\subsection{Product and Quotient Rules}\label{secMPSproductQuotientRules}

\begin{problem}
(Textbook, page 136, 1-44).
Compute the derivative.
\begin{multicols}{2}
\begin{enumerate}
\item $f(x)=2^{40}$.

\answer{$0$}
\item $f(x)=\pi^2$.

\answer{$0$}
\item $f(t)=2-\frac{2}{3}t$.

\answer{$-\frac{2}{3}$}
\item $F(x)=\frac{3}{4}x^8$.

\answer{$6 x^{7}$}
\item $f(x)=x^3-4x+6$.

\answer{$-4+3 x^{2} $}
\item $f(t)=\frac{1}{2}t^6-3t^4+t$.

\answer{$ 3 t^{5}-12 t^{3}+1$}
\item $g(x)=x^2(1-2x)$. 

\solution{
There are two approaches: 
1. Uncover the parenthesis, and then differentiate:

$
\left(x^2(1-2x)\right)'= \left(x^2-2x^3\right)'=2x-6x^2
$

2. Use first the product rule and then simplify:
$
\begin{array}{rcl}
\left(x^2(1-2x)\right)'&=& (x^2)'(1-2x)+x^2(1-2x)'\\
&=&2x(1-2x)+x^2(-2)\\
&=& 2x-4x^2-2x^2\\
&=& 2x-6x^2.
\end{array}
$

Of course, both approaches lead to the same answer.
}


\answer{$ 2 x-6 x^{2}$}
\item $h(x)=(x-2)(2x+3)$.

\answer{$ 4x-1$}
\item $g(t)=2t^{-3/4}$.

\answer{$-\frac{3}{2} t^{-\frac{7}{4}} $}
\item $B(y)=c y^{-6}$.

\answer{$-6 c y^{-7} $}
\item $A(s)=-\frac{12}{s^5}$.
\answer{$60 s^{-6}$}

\end{enumerate}
\end{multicols}

\end{problem}
\solution{\ref{problemDifferentiatexsquaredtimes1minus2x}
Approach 1. Uncover the parenthesis, and then differentiate:

$
\left(x^2(1-2x)\right)'= \left(x^2-2x^3\right)'=2x-6x^2
$

Approach 2. Use first the product rule and then simplify:
$
\begin{array}{rcl}
\left(x^2(1-2x)\right)'&=& (x^2)'(1-2x)+x^2(1-2x)'\\
&=&2x(1-2x)+x^2(-2)\\
&=& 2x-4x^2-2x^2\\
&=& 2x-6x^2.
\end{array}
$

Of course, both approaches lead to the same answer.
}

\begin{problem}
Compute the derivative.
\begin{multicols}{2}
\begin{enumerate}[ref={\fcProblemRef}]
\item $\displaystyle y=x^{\frac53}-x^{\frac23}$.

\answer{$ \frac53 x^{\frac23}-2/3 x^{-\frac13}$}
\item $\displaystyle f(x)=\sqrt{x}-x$.

\answer{$-1+\frac{1}{2} x^{-\frac{1}{2}} $}
\item $\displaystyle y=\sqrt{x}(x-1)$.

\answer{$ \frac{3}{2} x^{\frac{1}{2}}- \frac{1}{2}  x^{-\frac{1}{2}}$}
\item $\displaystyle f(x)=(2x+1)^2$.

\answer{$4+8 x $}
\item $\displaystyle f(x)=4\pi x^2$.

\answer{$8 \pi x$}
\item $\displaystyle y=\frac{ x^2+4x+3}{\sqrt{x}}$.

\answer{$ 2 x^{-\frac{1}{2}}+\frac{3}{2} x^{\frac{1}{2}}-\frac{3}{2} x^{-\frac{3}{2}}$}
\item $\displaystyle y=\frac{\sqrt{x}+x}{x^2}$.

\answer{$- x^{-2}-\frac{3}{2} x^{-\frac{5}{2}} $}
\item $\displaystyle f(x)=\left(x+x^{-1}\right)^3$.

\answer{$3x^{2}+3-3x^{-2}-3x^{-4} $}
\item $\displaystyle f(x)=\sqrt 2 x +\sqrt{5x}$.

\answer{$ \sqrt{2}+\frac{\sqrt5}{2}  x^{-\frac{1}{2}}=\sqrt{2}+\frac{\sqrt5}{2\sqrt{x}}$}
\item $\displaystyle y=\sqrt[5]x+4\sqrt{x^5}$.

\answer{$10 x^{\frac{3}{2}}+\frac{1}{5} x^{-\frac{4 }{5}} $}
\item \label{problemd/dx((sqrt(x)+1/sqrt[3](x))^2)} $\displaystyle y=\left(\sqrt{x}+ \frac{1}{ \sqrt[3]{x}}\right)^2$.


\answer{$1+\frac{1}{3} x^{-\frac{5}{6}}-\frac{2}{3} x^{-\frac{5}{3}} $}
\item $\displaystyle f(x)=(1+2x^2)(x-x^2)$.

\answer{$1-2 x+6 x^{2}-8 x^{3}$}
\item $\displaystyle f(x)=\frac{x^4-5x^3+ \sqrt{x}}{x^2}$.

\answer{$-5+2 x-\frac{3}{2} x^{-\frac{5}{2}} $}
\item $\displaystyle f(x)=(2x^3+3)(x^4-2x)$.

\answer{$-6-4 x^{3}+14 x^{6}$}
\item $\displaystyle f(x)=(1+x+x^2)(2-x^4)$.

\answer{$ 2+4 x-4 x^{3}-5 x^{4}-6 x^{5}$}
\item $\displaystyle g(y)= \left(\frac{1}{y^2}- \frac{3}{y^4} \right)(y+5y^3)$.

\answer{$5+9 y^{-4}+14 y^{-2} $}
\item $\displaystyle f(x)=(x^3-2x)(x^{-4}+x^{-2})$.

\answer{$1+6 x^{-4}+x^{-2}$}
\item $\displaystyle f(x)=\frac{1+2x}{3-4x}$.

\answer{$ 10 (3-4 x)^{-2}$}
\end{enumerate}
\end{multicols}
\end{problem}
\solution{\ref{problemd/dx((sqrt(x)+1/sqrt[3](x))^2)}
\[
\begin{array}{rcl}
\displaystyle \left(\left(\sqrt{x}+\frac{1}{\sqrt[3]{x}}\right)^2 \right)' &=&\displaystyle \left(\left(x^{\frac{1}{2}}+x^{- \frac{1}{3}}\right)^2 \right)'\\
&=&\displaystyle \left(\left(x^{\frac{1}{2}}\right)^2 +2 x^{\frac{1}{2}}x^{-\frac{1}{3}} + \left(x^{-\frac{1}{3}}\right)^2  \right)'\\
&=&\left(x +2 x^{\frac{1}{6}} + x^{-\frac{2}{3}}\right)'\\
&=&\displaystyle 1+2\cdot \frac{1}{6} x^{\frac{1}{6}-1} + \left(-\frac{2}{3} \right)x^{-\frac{2}{3}-1}\\
&=&\displaystyle 1+\frac{1}{3} x^{-\frac{5}{6}} -\frac{2}{3}x^{-\frac{5}{3}}
\end{array}
\]
}

\begin{problem}
Compute the derivative.
\begin{multicols}{2}
\begin{enumerate}[ref={\fcProblemRef}]

\item $\displaystyle f(x)=\frac{x-3}{x+3}$.

\answer{$6 (3+x)^{-2} $}
\item $\displaystyle y=\frac{x^3}{1-x^2}$.

\answer{$ \frac{3 x^{2}- x^{4}}{(1- x^{2})^{2}}$}
\item $\displaystyle y=\frac{x+1}{x^3+x-2}$.

\answer{$\frac{-3-3 x^{2}-2 x^{3}}{(-2+x+x^{3})^{2}} $}
\item $\displaystyle y=\frac{x^3-2x\sqrt{x}}{x}$.

\answer{$2 x- x^{-\frac{1}{2}}$}
\item \label{problemd/dt(t/(t-1)^2)}  $\displaystyle y=\frac{t}{(t-1)^2}$.

\answer{$-\frac{t +1}{(t-1)^3} $}
\item $\displaystyle y=\frac{t^2+2}{t^4-3t^2+1}$.

\answer{$\frac{14 t-8 t^{3}-2 t^{5}}{(1-3 t^{2}+t^{4})^{2}} $}
\item $\displaystyle g(t)=\frac{t-\sqrt{t}}{t^{\frac{1}{3}}}$.

\answer{$-\frac{1}{6} t^{-\frac{5}{6}}+\frac{2}{3} t^{-\frac{1}{3}} $}
\item $\displaystyle y=a x^2+b x + c$.

\answer{$ b+2 a x$}
\item $\displaystyle y=A+\frac{B}x +\frac{C}{x^2}$.

\answer{$\frac{- Bx-2 C }{x^{3}}$}
\item $\displaystyle f(t)=\frac{2t}{2+\sqrt{t}}$.

\answer{$\frac{4+t^{\frac{1}{2}}}{\left( 2+t^{\frac{1 }{ 2}} \right)^{2}} $}
\item $\displaystyle y=\frac{c x}{1+c x}$.

\answer{$ c (1+c x)^{-2}$}
\item $\displaystyle y=\sqrt[3]{t}(t^2+t+t^{-1}) $.

\answer{$-\frac{2}{3} t^{-\frac{5}{3}}+\frac{4}{3} t^{\frac{1}{3}}+\frac{7}{3} t^{\frac{4}{3}} $}
\item $\displaystyle y=\frac{u^6-2u^3+5}{u^2}$.

\answer{$-2-10 u^{-3}+4 u^{3} $}
\item $\displaystyle f(x)=\frac{a x+b}{c x+ d}$.

\answer{$\frac{a d- b c}{(d+c x)^{2}}$}
\item $\displaystyle f(x)=\frac{1+x }{1+\frac{2}x}$. 

\answer{$\frac{x^{2}+4 x+2}{(2+x)^{2}}$}
\item $\displaystyle f(x)=\frac{1+x }{1+\frac{3}x}$. 

\answer{$\frac{x^{2}+6 x+3}{(3+x)^{2}}$}
\item $\displaystyle f(x)=\frac{x}{x+\frac{c}{x}}$.

\answer{$ \frac{2 x c}{(c+x^{2})^{2}}$}
\end{enumerate}
\end{multicols}

\end{problem}
\solution{\ref{problemd/dt(t/(t-1)^2)}
This can be differentiated more efficiently using the chain rule, however let us show how the problem can be solved directly using the quotient rule.
\[
\begin{array}{rcl}
\displaystyle  \left(\frac{t}{(t-1)^2}\right)'&=&\displaystyle \frac{(t)' (t-1)^2-t\left((t-1)^2\right)' }{(t-1)^4}\\
&=&\displaystyle \frac{(t-1)^2 - t \left(t^2-2t+1\right)' }{(t-1)^4}\\
&=&\displaystyle \frac{(t-1)^2 - t \left(2t-2\right) }{(t-1)^4}\\
&=&\displaystyle \frac{\cancel{(t-1)} \left((t-1) - 2t \right)}{(t-1)^{\cancel{4} ~3}}\\
&=&\displaystyle \frac{-t -1}{(t-1)^3}\\
&=&\displaystyle-\frac{t+1}{(t-1)^3}
\end{array}
\]


}

\begin{problem}
Compute the derivative of the function.
\begin{enumerate}[ref={\fcProblemRef}]
\item $\displaystyle f(x)=\frac{1+x }{1+\frac{2}x}$. 

\answer{$\frac{x^{2}+4 x+2}{(2+x)^{2}}$}

\item $\displaystyle f(x)=\frac{1+x }{1+\frac{3}x}$. 

\answer{$\frac{x^{2}+6 x+3}{(3+x)^{2}}$}
\end{enumerate}

\end{problem}

\subsection{Basic Trigonometric Derivatives}\label{secMPStrigDerivatives}
\begin{problem}

Compute the derivative.
\begin{multicols}{2}
\begin{enumerate}
\item $\displaystyle f(x)= 2x^3 -3 \cos x$.

\answer{$ 6 x^2 +3 \sin x$}
\item $\displaystyle f(x)=\sqrt{x}\cos x$.

\answer{$ -x^{\frac{1}{2}}\sin x +\frac{1}{2}x^{-\frac{1}{2}} \cos x $}


\item $\displaystyle f(x)=\sin x +\frac{1}{3}\cot x$.

\answer{$\frac{-\frac{1}{3}+\cos x \sin^2x}{\sin^2x} = \cos x- \frac{1}{3} \csc^2 x $}
\item $\displaystyle y=2\sec x - \csc x$.

\answer{$ \frac{\cos^3 x+2 \sin^3x}{(\cos x \sin x)^{2}}$}
\item $\displaystyle y=\frac{1+\sin^2\theta}{\cos^3\theta}$.

\answer{$ y'=\displaystyle \frac{2 \sin{}\theta \cos^{2}{}\theta+3 \sin^{3}{}\theta+3 \sin{}\theta}{\cos^{4}{}\theta} $}
\item $\displaystyle g(t)=4 \sec t + \tan t$.

\answer{$4\sec t \tan t +\sec^2t= \frac{1+4 \sin t}{\cos^2 t} $}

\item $\displaystyle y= c\cos t + t^2\sin t$.

\answer{$ - c \sin t+2 t \sin t+ t^{2}\cos t$}
\item $\displaystyle y=u(a\cos u + b \cot u)$.

\answer{$ \frac{- a u \sin^3 u+a \cos u \sin^2u- b u +b \cos u \sin u}{\sin^2u}$}
\item $\displaystyle y=\frac{x}{2-\tan x}$.

\answer{$ \frac{x - \cos x \sin x+2 \cos^2 x}{(2 \cos x- \sin x)^{2}}$}
\item $\displaystyle y=\sin \theta \cos \theta$.

\answer{$\cos (2\theta)= \cos^2\theta- \sin^2\theta$}
\item $\displaystyle f(\theta)=\frac{\sec \theta}{1+\sec \theta}$.

\answer{$\frac{\sin\theta}{(1+\cos\theta)^{2}} $}
\item $\displaystyle y=\frac{\cos x}{1-\sin x}$.

\answer{$\frac{1}{1- \sin x} $}
\item $\displaystyle y=\frac{t\sin t}{1+t}$.

\answer{$ \frac{\sin t+t \cos t+t^{2}\cos t }{(1+t)^{2}}$}
\item 
$\displaystyle y=\frac{1-\sec x}{\tan x}$.

\answer{$\frac{\cos x- 1}{\sin^2x} $}

\item $\displaystyle h(\theta)=\theta \csc \theta -\cot \theta$.

\answer{$\frac{1+\sin\theta- \theta \cos\theta}{\sin^2\theta}$}
\item $\displaystyle y=x^2\sin x\tan x$.

\answer{$\frac{2 x \cos{}x \sin^2{}x+2 x^{2} \sin{}x  \cos^2x+x^{2} \sin^3{}(x)}{\cos^2{}x} $}
\end{enumerate}
\end{multicols}

\end{problem}
\begin{problem}
Differentiate.

\begin{multicols}{2}
\begin{enumerate}
\item $\tan x$.

\answer{$\sec^2 x$}
\item $\cot x$.

\answer{$-\csc^2 x$}
\item $\sec x$.

\answer{$\sec x \tan x= \frac{\sin x}{\cos^2 x}$}
\item $\csc x$.

\answer{$-\csc x \cot x= -\frac{\cos x }{\sin^2x} $}
\item $\sec x\tan x$.

\answer{$\sec x \tan^2 x+\sec^3 x$}
\item $\sec x+\tan x$.

\answer{$\sec x(\tan x +\sec x) $}
\item $\sec^2 x$.

\answer{$2\tan x\sec^2 x$}
\item $\csc^2 x$.

\answer{$ -2\cot x\csc^2 x$}
\item $\displaystyle \frac{\sin x}{x}$.

\answer{$\frac{x \cos{}x- \sin{}x}{x^{2}}$}
\end{enumerate}

\end{multicols}
\end{problem}
\subsection{Natural Exponent Derivatives}
% begin homework exponent-derivative
Differentiate each function.  

\begin{enumerate}
\item   $\displaystyle f(x) = \frac{e^x}{1+2e^x}$.  

\pointsii{3}  $r(t) = Ae^{-kt^2}$, where $A$ and $k$ are unknown constants.  

\solution{%
\begin{align*}
r & = Ae^{-kt^2}. \\
\text{Let}\quad u & = -kt^2. \\
\text{Then}\quad r & = Ae^u. \\
\text{Chain Rule}\quad \frac{\diff r}{\diff t} & = \frac{\diff r}{\diff u}\frac{\diff u}{\diff t} \\
 & = (Ae^u)(-2kt) \\
 & = -2Akte^{-kt^2}.
\end{align*}
}%

\item   $y = \frac{e^x}{2}(\sin x + \cos x)$.  
\end{enumerate}
% end homework exponent-derivative


\subsection{The Chain Rule}\label{secMPSchainRule}

\begin{problem}
% begin homework chain-rule1
In each of the following cases find a simple function $u$ of $x$ such that the given function is a simple function of $u$.  
Use the Chain Rule to differentiate the given function with respect to $x$.   

\begin{enumerate}
\item   $y = \sqrt{1+x^2}$


\pointsii{3}  $y = (\cos x)^{1/2}$
\ans{%
\begin{align*}
\text{Let } \quad u & = \cos x. \\
\text{Then } \quad y & = u^{1/2}. \\
\text{Chain Rule: } \quad \frac{\diff y}{\diff x} & = \frac{\diff y}{\diff u}\frac{\diff u}{\diff x} \\
 & = \big(\frac{1}{2}u^{-1/2}\big) (-\sin x) \\
 & = -\frac{1}{2} \sin x (\cos x)^{-1/2}.
\end{align*}
}%

\item   $y = \sin^3 x$

\pointsii{3}  $y = (1+\cos x)^2$
\ans{%
\begin{align*}
\text{Let } \quad u & = 1+\cos x. \\
\text{Then } \quad y & = u^{2}. \\
\text{Chain Rule: } \quad \frac{\diff y}{\diff x} & = \frac{\diff y}{\diff u}\frac{\diff u}{\diff x} \\
 & = (2u) (-\sin x) \\
 & = -2 \sin x \cos x \\
 & = - \sin 2x. \quad \text{(This last step is optional.)}
\end{align*}
}%

\end{enumerate}
% end homework chain-rule1

\end{problem}
\solution{\ref{problemDifferentialtexDivsqrt(1+2divx^2)}
\[
\begin{array}{rclr|r}
\displaystyle\left(\frac{x }{\sqrt{1+\frac{2}{x^2}}}\right)'&=&\displaystyle\frac{\sqrt{1+\frac{2}{x^2}}- x\left(\sqrt{1+\frac{2}{x^2}}\right)'}{1+\frac{2}{x^2}} =\frac{\sqrt{1+\frac{2}{x^2}}-  x\frac{\frac12}{\sqrt{1 +\frac{ 2}{ x^2 }}}  \left(\frac{2}{x^2}\right)'}{1+\frac{2}{x^2}}\\
&=& \displaystyle\frac{\sqrt{1+\frac{2}{x^2}}+  \frac{2}{x^2\sqrt{1 +\frac{ 2}{ x^2 }}} }{1+\frac{2}{x^2}} = \frac{x^2\left(1+\frac{2}{x^2}\right)+  2 }{x^2\left(1+\frac{2}{x^2}\right)^{\frac32}}= \frac{x^2+4}{x^2\left(1+\frac{2}{x^2}\right)^{\frac32}}
\end{array}
\]
Please note that this problem can be solved also by applying the transformation 
\[
\displaystyle  \frac{x}{\sqrt{1+\frac{2}{x^2}}}= \frac{x}{\sqrt{\frac{x^2+2}{x^2}}} =\frac{x}{\frac{1}{\pm x}\sqrt{x^2+2}} = \frac{\pm x^2}{\sqrt{x^2+2}}
\]
before differentiating, however one must not forget the $\pm $ sign arising from $\sqrt{x^2}=\pm x$. Our original approach resulted in more algebra, but did not have the disadvantage of dealing with the $\pm$ sign.
}

\solution{ \ref{problemd/dx((cosx)^(1/2))}%
\begin{align*}
\text{Let } \quad u & = \cos x. \\
\text{Then } \quad y & = u^{\frac{1}{2}}. \\
\text{Chain Rule: } \quad \frac{\diff y}{\diff x} & = \frac{\diff y}{\diff u}\frac{\diff u}{\diff x} \\
 & = \left(\frac{1}{2}u^{-\frac{1}{2}}\right) (-\sin x) \\
 & = -\frac{1}{2} \sin x (\cos x)^{-\frac{1}{2}}.
\end{align*}
}%

\solution{\ref{problemd/dx((1+cosx)^2)} %
\begin{align*}
\text{Let } \quad u & = 1+\cos x. \\
\text{Then } \quad y & = u^{2}. \\
\text{Chain Rule: } \quad \frac{\diff y}{\diff x} & = \frac{\diff y}{\diff u}\frac{\diff u}{\diff x} \\
 & = (2u) (-\sin x) \\
 & = -2 \sin x (1+\cos x) \\
 & = -2\sin x -2 \sin x \cos x \\
 & = -2\sin x -\sin (2x). \quad \text{(This last step is optional.)}
\end{align*}
}%

\solution{\ref{problemd/dx(sin(sqrt(x)))} %
\begin{align*}
\text{Let } \quad u & = \sqrt{x}. \\
\text{Then } \quad y & = \sin u. \\
\text{Chain Rule: } \quad \frac{\diff y}{\diff x} & = \frac{\diff y}{\diff u}\frac{\diff u}{\diff x} \\
 & = (\cos u) \left(\frac{1}{2}u^{-\frac{1}{2}}\right) \\
 & = \frac{\cos\left(\sqrt{x}\right) }{2\sqrt{x}}.
\end{align*}
}%

\solution{\ref{problemd/dx(sqrt(sec(4x)))} %
\begin{align*}
\text{Chain Rule: } \quad \frac{\diff y}{\diff x} & = \left( \frac{1}{2}(\sec (4x))^{-\frac{1}{2}} \right) \frac{\diff}{\diff x}(\sec (4x)) \\
\text{Chain Rule: } \quad \frac{\diff y}{\diff x} & = \left( \frac{1}{2\sqrt{\sec (4x)}} \right) (\sec (4x) \tan (4x))\frac{\diff}{\diff x}(4x) \\
 & = \left( \frac{1}{2\sqrt{\sec (4x)}}\right) (\sec (4x) \tan (4x))(4) \\
 & =  \frac{2\sec (4x)\tan (4x)}{\sqrt{\sec (4x)}} \\
\intertext{There are many ways to simplify this answer, including both of the following.}
 & =  2\sqrt{\sec (4x)}\tan (4x). \\
 & =  2(\sec (4x))^{\frac{3}{2}}\sin (4x). 
\end{align*}
}%

\solution{\ref{problemd/dx(x^2tan(5x))} %
\begin{align*}
\text{Product Rule: } \quad \frac{\diff y}{\diff x} & = (x^2)\frac{\diff}{\diff x}(\tan (5x)) + (\tan (5x))\frac{\diff}{\diff x}(x^2) \\
\intertext{Use the Chain Rule to differentiate $\tan (5x)$ in the first term.}
\frac{\diff y}{\diff x} & = (x^2)(-5\sec^2 (5x) + (\tan (5x))(2x) \\
 & = 2x\tan (5x) - 5x^2\sec^2 (5x).
\end{align*}
}%


\solution{\ref{problemd/dx((1+sin(x^2))/(1+cos(x^2)))} %
\begin{align*}
\text{Quotient Rule: } \quad \frac{\diff y}{\diff x} & = \frac{\left(1+ \cos \left( x^2 \right)\right)\frac{\diff}{\diff y}(1+\sin \left(x^2\right) ) - (1+\sin \left(x^2\right))\frac{\diff}{\diff x}(1+\cos \left(x^2\right))}{(1+\cos \left(x^2\right))^2} \\
\intertext{By the Chain Rule, $\frac{\diff}{\diff x}(1+\cos \left(x^2\right)) = -2x\sin \left(x^2\right)$ and $\frac{\diff}{\diff x}(1+\sin \left(x^2\right)) = 2x\cos \left(x^2\right)$.}
\frac{\diff y}{\diff x} & = \frac{(1+\cos \left(x^2\right))(2x\cos \left(x^2\right)) - (1+\sin \left(x^2\right))(-2x\sin \left(x^2\right))}{(1+\cos \left(x^2\right))^2} \\
 & = \frac{2x\cos \left(x^2\right) + 2x\cos^2 \left(x^2\right) + 2x\sin \left(x^2\right) + 2x\sin^2 \left(x^2\right)}{(1+\cos \left(x^2\right))^2} \\
 & = \frac{2x(\cos^2 \left(x^2\right) + \sin^2 \left(x^2\right)) + 2x(\cos \left(x^2\right) + \sin \left(x^2\right))}{(1+\cos \left(x^2\right))^2} \\
\intertext{By the Pythagorean Identity, $\cos^2 \left(x^2\right) + \sin^2 \left(x^2\right) = 1$.}
\frac{\diff y}{\diff x} & = \frac{2x + 2x(\cos \left(x^2\right) + \sin \left(x^2\right))}{(1+\cos \left(x^2\right))^2} \\
 & = \frac{2x(1 + \cos \left(x^2\right) + \sin \left(x^2\right))}{(1+\cos \left(x^2\right))^2}.
\end{align*}
}%

\begin{problem}
(Textbook, page 154, problems 7-46) Compute the derivative.
\begin{multicols}{2}
\begin{enumerate}
\item $\displaystyle f(x)= (x^4+3x^2-2)^5$.

\answer{$ (30 x +20 x^{3}) (-2+3 x^{2}+x^{4})^{4}$}
\item $\displaystyle f(x)= (4x-x^2)^{100}$.

\answer{$(-200 x+400) (4 x- x^{2})^{99}$}
\item $\displaystyle f(x)= \sqrt{1-2x}$.

\answer{$- (1-2 x)^{-\frac{1}{2}}$}
\item $\displaystyle f(x)= \frac{1}{(1+\sec x)^2}$.

\answer{$\frac{-2 \cos{}(x) \sin{}(x)}{(1+\cos{}(x))^{3}} =\frac{- \sin{}(2x)} {(1+\cos{}(x))^{3}} $}
\item $\displaystyle f(z)=\frac{1}{1+z^2} $.

\answer{$\frac{-2 z}{(1+z^{2})^{2}} $}
\item $\displaystyle f(t)= \sqrt[3]{1+\tan t}$.

\answer{$\frac{\frac{1}{3}}{(\cos{}(t))^{2}} \left( \frac{\cos{}(t) +\sin{}(t)}{\cos{}(t)} \right)^{-\frac{2}{3}} $}
\item $\displaystyle f(x)=\cos (a^3+x^3) $.

\answer{$ -3 x^{2}\sin{}(a^{3}+x^{3}) $}
\item $\displaystyle f(x)= a^3+\cos^3 x$.

\answer{$ -3 \sin{}(x) (\cos{}(x))^{2}$}
\item $\displaystyle f(x)= x\sec (k x) $.

\answer{$\frac{\cos{}(k x)+k x \sin{}(k x) }{(\cos{}(k x))^{2}} $}
\item $\displaystyle f(\theta)= 3\cot (n\theta)$.

\answer{$ \frac{-3 n}{(\sin{}(n \theta))^{2}}$}
\item $\displaystyle f(x)= (2x - 3)^4 (x^2 + x + 1)^5$.

\answer{$ (-7-12 x+28 x^{2})(-3+2 x)^{3} (1+x+x^{2})^{4}$}
\item $\displaystyle f(x)= (x^2+1)^3(x^2+2)^6$.

\answer{$\left(24 x+18 x^{3}\right)\left(1+x^{2}\right)^{2} \left(2+x^{2}\right)^{5} $}
\item $\displaystyle f(t)= (t+1)^{\frac{2}{3}}(2t^2-1)^3$.

\answer{$ \left(\frac{40}{3} t^{2}+12 t-\frac{2}{3}\right)\left(2 t^{2}-1\right)^{2}\left(t+1\right)^{-\frac{1}{3}}$}
\item $\displaystyle f(t)= (3t-1)^4(2t+1)^{-3}$.

\answer{$(3 t-1)^{3}\frac{6 t+18}{(2 t+1)^{4}}$}
\item $\displaystyle f(x)=\left(\frac{x^2+1}{x^2-1} \right)^3 $.

\answer{$\frac{-12 x}{\left(x^{2}-1\right)^{2}} \left(\frac{x^{2}+1}{x^{2}-1}\right)^{2} $}
\item $\displaystyle f(s)= \sqrt{\frac{s^2+1}{s^2+4}}$.

\answer{$\frac{3 s}{\left(s^{2}+4\right)^{2}} \left(\frac{s^{2}+1}{s^{2}+4}\right)^{-\frac{1}{2}} $}
\item $\displaystyle f(x)=\sin (x\cos x) $.

\answer{$\cos{}(x) \cos{}(x \cos{}(x))- x \cos{}(x \cos{}(x)) \sin{}(x) $}
\item $\displaystyle f(x)=\frac{x}{\sqrt{7-3x}} $.

\answer{$ \frac{-\frac{3}{2} x+7}{(-3 x+7)^{\frac{3}{2}}}$}
\item $\displaystyle f(z)=\sqrt{\frac{z-1}{z+1}} $.

\answer{$\frac{1}{(z+1)^{2}} \left(\frac{z-1}{z+1}\right)^{ -\frac{1}{2}} $}
\item $\displaystyle f(y)= \frac{(y-1)^4 }{(y^2+2y)^5}$.

\answer{$(y-1)^{3}\frac{-6y^{2}+8y +10}{(y^{2}+2 y)^{6}} $}
\item $\displaystyle f(r)=\frac{r}{\sqrt{r^2+1}} $.

\answer{$ \frac{1}{\left(r^{2}+1\right)^{\frac{3}{2}}}$}
\item $\displaystyle f(x)=\frac{\cos (\pi x)}{\sin (\pi x)+\cos (\pi x) } $.

\answer{$ \frac{- \pi}{(\sin{}(\pi x)+\cos{}(\pi x))^{2}}$}
\item $\displaystyle f(x)=\sin \left(\sqrt{1+x^2}\right) $.

\answer{$x \cos{}\left(\left(x^{2}+1\right)^{\frac{1}{2}}\right) \left(x^{2}+1\right)^{-\frac{1}{2}}$}
\item $\displaystyle f(v)=\left(\frac{v}{v^3+1}\right)^6 $.

\answer{$\frac{-12 v^{3}+6}{\left(v^{3}+1\right)^{2}} \left(\frac{v}{v^{3}+1}\right)^{5} $}
\end{enumerate}
\end{multicols}


\end{problem}
\solution{\ref{problemd/dx(cos(x))^(1/2)}%
\begin{align*}
\text{Let } \quad u & = \cos x. \\
\text{Then } \quad y & = u^{\frac{1}{2}}. \\
\text{Chain Rule: } \quad \frac{\diff y}{\diff x} & = \frac{\diff y}{\diff u}\frac{\diff u}{\diff x} \\
 & = \left(\frac{1}{2}u^{-\frac{1}{2}}\right) (-\sin x) \\
 & = -\frac{1}{2} \sin x (\cos x)^{-\frac{1}{2}}.
\end{align*}
}%


\solution{\ref{problemd/dx(1+cos(x))^2} %
\begin{align*}
\text{Let } \quad u & = 1+\cos x. \\
\text{Then } \quad y & = u^{2}. \\
\text{Chain Rule: } \quad \frac{\diff y}{\diff x} & = \frac{\diff y}{\diff u}\frac{\diff u}{\diff x} \\
 & = (2u) (-\sin x) \\
 & = -2 \sin x (1+\cos x) \\
 & = -2\sin x -2 \sin x \cos x \\
 & = -2\sin x -\sin 2x. \quad \text{(This last step is optional.)}
\end{align*}
}%

\solution{\ref{problemd/dx(sin(sqrt(x)))} %
\begin{align*}
\text{Let } \quad u & = \sqrt{x}. \\
\text{Then } \quad y & = \sin u. \\
\text{Chain Rule: } \quad \frac{\diff y}{\diff x} & = \frac{\diff y}{\diff u}\frac{\diff u}{\diff x} \\
 & = (\cos u) \left(\frac{1}{2}u^{-\frac{1}{2}}\right) \\
 & = \frac{\cos\sqrt{x}}{2\sqrt{x}}.
\end{align*}
}%


\begin{problem}
Differentiate. 
\begin{multicols}{2}
\begin{enumerate}
\item $\displaystyle f(x)=\sin (\tan (2x)) $.

\answer{$2\sec^2(2x) \cos (\tan 2x) $}
\item $\displaystyle f(x)=\sec^2(m x) $.


\answer{$ \frac{2 m \sin{}(m x) }{(\cos{}(m x))^{3}} $}
\item $\displaystyle f(x)= \sec^2 x+\tan^2 x$.

\answer{$\frac{4 \sin{}x}{\cos^{3}{}x} $}
\item $\displaystyle f(x)=x\sin\left( \frac{1}{x}\right) $.

\answer{$- x^{-1}\cos{}(x^{-1}) +\sin{}(x^{-1})$}
\item $\displaystyle f(x)= \left(\frac{1-\cos (2x)}{1+\cos (2x)}\right)^4$.

\answer{$\frac{16 \sin{}(2 x)}{(\cos{}(2 x)+1)^{2}} \left(\frac{- \cos{}(2 x)+1}{\cos{}(2 x)+1}\right)^{3} $}
\item $\displaystyle f(x)=\sqrt{\frac{x}{x^2+4}} $.

\answer{$ \frac{-\frac{1}{2} x^{2}+2}{(x^{2}+4)^{2}} \left(\frac{x}{x^{2}+4}\right)^{-\frac{1}{2}}$}
\item $\displaystyle f(t)= \cot^2(\sin t)$.

\answer{$ \frac{-2 \cos{}t \cos{}(\sin{}t)}{\sin^{3}{}(\sin{}t)}$}
\item $\displaystyle f(x)= \left(a x+\sqrt{x^2+b^2}\right)^{-2}$.

\answer{$\frac{-2 x\left(x^{2}+b^{2} \right)^{-\frac{ 1 }{2}} -2 a}{\left(\left(x^{ 2}+b^{2}\right)^{ \frac{1}{2}} +a x\right)^{3}} $}
\item $\displaystyle f(x)= \left(x^2+(1-3x)^5 \right)^3$.

\answer{
\begin{tabular}{l}
$\left(-45 (-3 x+1)^{4} +6 x \right) \left((-3 x+1)^{5}+x^{2}\right)^{2}$
\\
Using computer algebra:
\\
$(-3645x^{4}+4860x^{3}-2430x^{2}+546x -45)\left((-3 x+1)^{5}+x^{2}\right)^{2}$ \\
Using computer algebra full expansion:
\\
$\begin{array}{l}
-215233605x^{14}+1004423490x^{13}-2176250895x^{12}\\
+2903793624x^{11} -2666357595x^{10}+1782098820x^{9}\\
-893713176x^{8} +341444160x^{7} -99805041x^{6} \\ +22199676x^{5}-3697470x^{4}  +447132x^{3}\\
-37125x^{2}+1896x -45
\end{array}
$
\end{tabular} 
}
\item $\displaystyle f(x)=\sin (\sin (\sin x))$.

\answer{$\cos{}x \cos{}(\sin{}x) \cos{}(\sin{}(\sin{}x)) $}
\item $\displaystyle f(x)= \sqrt{x+\sqrt{x}}$.

\answer{$ \left(\frac{1}{2} +\frac{1}{4} x^{-\frac{1}{2}}\right) \left(x^{\frac{1}{2}}+x\right)^{-\frac{1}{2}}$}

\item $\displaystyle f(x)= \sqrt{x+\sqrt{x+\sqrt{x}}}$.

\answer{$\frac{1}{2} \left(\left(x^{\frac{1}{2}}+x\right)^{\frac{1}{2}}+x\right)^{-\frac{1}{2}} \left(\frac{1}{2} \left(x^{\frac{1}{2}}+x\right)^{-\frac{1}{2}} \left(\frac{1}{2} x^{-\frac{1}{2}}+1\right)+1\right) $}
\item $\displaystyle f(x)=(2r \sin (r x)+n)^p $.

\answer{$ p r(2 r \sin{}(r x)+n)^{p-1} \cos{}(r x) $}
\item $\displaystyle f(x)=\cos^4(\sin^3 x) $.

\answer{$-12 \cos{}x \sin^{2}{}x \sin{}(\sin^{3}{}x) \cos^{3}{}(\sin^{3}{}x) $}
\item $\displaystyle f(x)=\cos \sqrt{\sin (\tan (\pi x))} $.

\answer{$ \frac{-\frac{1}{2} \pi \cos{}(\tan{}(\pi x))  \sin{}\left(\sqrt{\sin (\tan{}(\pi x) )} \right)}{\sqrt{\sin{}(\tan{}(\pi x))} \cos^{2}{}(\pi x) }$}
\item $\displaystyle f(x)=\left(x+(x+\sin^2 x)^3 \right)^4 $.

\answer{$4 ((\sin^{2}{}x+x)^{3}+x)^{3} (3 (\sin^{2}{}x+x)^{2} (2 \sin{}x \cos{}x+1)+1) $}
\end{enumerate}
\end{multicols}
\end{problem}
\begin{problem}
Compute the second derivative.
\begin{multicols}{3}
\begin{enumerate}
\item $\displaystyle f(x)=\sin (-5x)$. 

\answer{$ 25 \sin{}(5 x)$}
\item $\displaystyle f(x)=\cot (2x)$. 

\answer{$8\cot(2x)\csc^2(2x)=  \frac{8 \cos{}\left(2 x\right)}{\sin^{3}{}\left(2 x\right)} $}
\item $\displaystyle f(x)=e^{-3x}$. 

\answer{$9 e^{-3 x} $}
\item $\displaystyle f(x)=e^{\frac{1}x}$. 

\answer{$ 2 e^{x^{-1}} x^{-3}+e^{x^{-1}} x^{-4}$}
\item $\displaystyle f(x)=e^{\sqrt{x}}$. 

\answer{$ e^{x^{\frac{1}{2}}} x^{-\frac{3}{2}}+\frac{1}{4} e^{x^{\frac{1}{2}}} x^{-1}$}
\item $\displaystyle f(x)=\frac{e^{x}-e^{-x}}{e^x+e^{-x}} $

\answer{$\frac{-8 \left(- e^{- x}+e^{x}\right)}{\left(e^{- x}+e^{x}\right)^{3}} $}
\item $\displaystyle f(x)=\frac{1}2\ln \left(\frac{1+x}{1-x}\right) $

\answer{$\frac12\left(-\frac{1}{(x+1)^{2}}+\frac{1}{(- x+1)^{2}}\right)= \frac{ 2x}{\left(1-x^2\right)^2} $}
\end{enumerate}
\end{multicols}

\end{problem}
\subsection{Problem Collection All Techniques}
\begin{problem}
Find the derivative of the following functions.
\begin{multicols}{2}
\begin{enumerate}
\item ${\displaystyle \frac{\sin x}{x^2}}$

\answer{${\displaystyle \frac{x \cos x - 2 \sin x}{x^3}}$}
\item ${\displaystyle e^{\sqrt{x^2 + 1}}}$

\answer{
$\begin{array}{l}\displaystyle e^{\sqrt{x^2 + 1}} \cdot \frac{1}{2}\left(x^2+ 1\right)^{ -\frac{ 1}{2}} \cdot 2x \\~\\
= \frac{x e^{\sqrt{x^2+1}} }{ \sqrt{ x^2 +1} } \end{array}$
} 
\item ${\displaystyle \ln \left(x-\frac{1}{x} \right)}$

\answer{${\displaystyle \frac{1}{x-\frac{1}{x}} \cdot \left(1 + \frac{1}{x^2}\right)}$} 
\item ${\displaystyle \sqrt[3]{x} \ln x}$

\answer{${\displaystyle \frac{1}{3} \frac{1}{\sqrt[3]{x^2}} \ln x + \frac{1}{\sqrt[3]{x^2}} }$} 
\item ${\displaystyle \cos(e^x)}$

\answer{${\displaystyle -\sin\left(e^x\right) \cdot e^x}$} 
\item ${\displaystyle \sin^3(2x)}$

\answer{${\displaystyle 3 \sin^2(2x) \cdot \cos(2x) \cdot 2 = 6 \sin^2(2x) \cos(2x)}$} 
\item ${\displaystyle f(x) = \int_x^1 (2+t^4)^5 \; \diff t}$

\answer{${\displaystyle -\left(2+x^4\right)^5}$} 
\item ${\displaystyle g(x) = \int_{0}^{x^3} \cos^2 t \; \diff t}$

\answer{${\displaystyle 3x^2 \cos^2\left(x^3\right)}$} 
\item Find $y'$ if $2x^2 + x + xy = 1$.

\answer{${\displaystyle y' = \frac{-4x-1-y}{x}}$} 
\item Find $y'$ if $x \sin y + y \sin x = 4$.

\answer{${\displaystyle y' = \frac{-\sin y - y \cos x}{\sin x + x \cos y}}$} 
\end{enumerate}
\end{multicols}

\end{problem}
\solution{\ref{problemDifferentiateFTC1int_x^1(2+t^4)^5dt} %(Contributed by student Anamaria Ronayne)

We recall that the Fundamental Theorem of Calculus part 1 states that $\frac{\diff}{\diff x}\left(\int_{a}^{x}h(t)dt\right)=h(x)$
where $a$ is a constant. We can rewrite the integral so it has $x$ as the upper limit:
\[
f(x)=\int_{x}^{1}(2+1^4)^5dt =-\int_{1}^{x}(2+1^4)^5dt\quad.
\]
Therefore
\[
\frac{\diff}{\diff x}\left( -\int_{1}^{x}(2+t^4)^5 \diff t\right)=- \frac{\diff }{\diff x}\left(\int_{1}^{x}(2+t^4)^5\diff t\right)\stackrel{\text{FTC part 1}}{=}
-(2+x^4)^5\quad .
\]

}

\subsection{Implicit Differentiation}\label{secMPSImplicitDifferentiation}
\begin{problem}
(Textbook, page 161, problems 5-20) Express $\frac{\diff y}{\diff x}$ as a function of $x$ and $y$ by implicit differentiation. The answer key has not been proofread, use with caution.
\begin{multicols}{3}
\begin{enumerate}
\item $x^3+y^3=1$.
\answer{$\frac{\diff y}{\diff x}=-\frac{x^2}{y^2}$}
\item $ 2\sqrt x+\sqrt y=3$.
\answer{$\frac{\diff y}{\diff x}=-2\sqrt{\frac{ y}{x}}$}
\item $ x^2+x y-y^2=4$.
\answer{$\frac{\diff y}{\diff x}=\frac{-2x-y}{x-2y}$}
\item $ 2x^3+x^2y-x y^3=2$.
\answer{$\frac{\diff y}{\diff x}=-\frac{6x^2+2xy+y^3}{-3xy^2+x^2}$}
\item $ x^4(x+y)=y^2(3x-y)$.
\answer{$\frac{\diff y}{\diff x}= \frac{ -5x^4 -4x^3y +3y^2}{x^4- 6xy - 3y^2}$}
\item $ y^5+x^2y^3=1+x^4y $.
\answer{$\frac{\diff y}{\diff x}=\frac{ 4x^3y-2xy^3}{5y^4+3x^2y^2- x^4 }$}
\item $ y\cos x=x^2+y^2 $.
\answer{$\frac{\diff y}{\diff x}= \frac{ y\sin x+2x}{\cos x-2y}$}
\item $ \cos (x y)=1+\sin y$.
\answer{$\frac{\diff y}{\diff x}= y\sin (xy) $}
\item $ 4\cos x\sin y=1$.
\answer{$\frac{\diff y}{\diff x}=$}
\item $ y\sin (x^2)=x\sin (y^2)$.
\answer{$\frac{\diff y}{\diff x}=$}
\item $ \tan \left(\frac{x}{y}\right)=x+y$.
\answer{$\frac{\diff y}{\diff x}=$}
\item $ \sqrt{x+y}=1+x^2y^2$.
\answer{$\frac{\diff y}{\diff x}=$}
\item $ \sqrt{xy}=1+x^2 y$.
\answer{$\frac{\diff y}{\diff x}=$}
\item $ x\sin y+y\sin x=1$.
\answer{$\frac{\diff y}{\diff x}=$}
\item $ y\cos x=1+\sin (x y)$.
\answer{$\frac{\diff y}{\diff x}=$}
\item $ \tan (x-y)=\frac{y}{1+x^2}$.
\answer{$\frac{\diff y}{\diff x}=$}
\end{enumerate}
\end{multicols}
\end{problem}
\begin{problem}
Verify that the coordinates of the given point satisfy the given equation. Use implicit differentiation to find an equation of the tangent line to the curve at the given point. 
\begin{multicols}{2}
\begin{enumerate}[ref={\fcProblemRef}]
\item 
\label{problemImplicitTangentysin(2x)=xcos(2y)point(pi/2,pi/4)} $y\sin (2x)=x\cos (2y) $, $\left(\frac{\pi}{2}, \frac{\pi}{4}\right)$. 

\psset{xunit=0.5cm, yunit=0.5cm}
\begin{pspicture}(-5.8,-5.8)(5.8,5.8)
\fcAxesStandard{-5.5}{-5.5}{5.5}{5.5}
\fcLabels{5.5}{5.5}
\fcXTickWithLabel{1}{$1$}
\fcImplicitIId[linestyle=solid, linecolor=red, linewidth-=0.05, showGridImplicitIId=false]{-5}{-5}{1000}{1000}{0.01}{0.01}{2 x mul 180 mul 3.141592654 div sin y mul 2 y mul 180 mul 3.141592654 div cos x mul sub} 

\fcFullDot[linecolor=blue]{3.141592654 2 div}{3.141592654 4 div}
\end{pspicture}

\answer{$y=\frac{1}{2}x$}

\item $ \sin (x+y)=2x-2y$, $(\pi,\pi)$ . 

\answer{$\frac{1}{3} x+\frac{2}{3} \pi $}
\item 
$x^2+x y+y^2=3 $, $(1,-2)$ (ellipse). 

\psset{xunit=0.5cm, yunit=0.5cm}
\begin{pspicture}(-3.6,-3.8)(3.6,3.6)
\tiny
\fcAxesStandard{-3.5}{-3.5}{3.5}{3.5}
\fcLabels{3.5}{3.5}
\fcXTickWithLabel{1}{$1$}
\fcImplicitIId[linestyle=solid, linecolor=red, linewidth-=0.05, showGridImplicitIId=false]{-2}{-2}{400}{400}{0.01}{0.01}{x x mul x y mul add y y mul add 3 sub}
%\psline[linecolor=blue](-3.5,-2)(3.5,-2)
\fcFullDot[linecolor=blue]{1}{-2}
\end{pspicture}


\answer{$y=-2 $}

\item  $x^2+2x y-y^2+x=2 $, $(1,2)$ (hyperbola). 

\psset{xunit=0.4cm, yunit=0.4cm}
\begin{pspicture}(-5.6,-5.6)(5.6,5.6)
\tiny
\fcAxesStandard{-5.5}{-5.5}{5.5}{5.5}
\fcLabels{5.5}{5.5}
\fcXTickWithLabel{1}{$1$}
\fcImplicitIId[linestyle=solid, linecolor=red, linewidth-=0.05, showGridImplicitIId=false]{-5}{-5}{500}{500}{0.02}{0.02}{x x mul x y mul 2 mul add y y mul sub x add 2 sub}
%\psline[linecolor=blue](-3.5,-2)(3.5,-2)
\fcFullDot[linecolor=blue]{1}{2}
\end{pspicture}

\answer{$y= \frac{7}{2} x-\frac{3}{2}$}


\item  \label{problemImplicitTangenty^3+x^3+4xy=3/4}
$\displaystyle y^{3}+x^{3}+4 x y=\frac{3}{4}$, $\left(-\frac{1}{2},-\frac{1}{2}\right)$

\psset{xunit=0.7cm, yunit=0.7cm}
\begin{pspicture}(-2.4,-2.4)(2.4,2.4)
\fcAxesStandard{-2}{-2}{2}{2}
\fcImplicitIId[linecolor=red, linestyle=dashed, dashes={[1 1] 0}, showGridImplicitIId=false, useMidpointImplicitPlots=false]{-3}{-3}{120}{120}{0.05}{0.05}{y y y mul mul x x x mul mul add 4 x y mul mul add 0.75 sub} 
\fcFullDot{-0.5}{-0.5}
\end{pspicture}


\answer{$y=-x-1$}
\item  \label{problemImplicitTangenty^3+x^3+4xy=-4at(1,-1)}
$\displaystyle y^{3}+x^{3}+4 x y=-4$, $(1,-1)$

\psset{xunit=0.7cm, yunit=0.7cm}
\begin{pspicture}(-3.1,-3.1)(3.1,3.1)
\fcAxesStandard{-2}{-2}{2}{2}
\fcImplicitIId[linecolor=red, linestyle=dashed, dashes={[1 1] 0}, showGridImplicitIId=false, useMidpointImplicitPlots=false]{-3}{-3}{300}{300}{0.02}{0.02}{y y y mul mul x x x mul mul add 4 x y mul mul add 4 add} 
\fcFullDot{1}{-1}
\end{pspicture}

\answer{$ $}

\item $x^2+y^2=(2x^2+2y^2-x)^2 $, $(0,\frac{1}{2})$. 

\answer{$y= x+\frac{1}{2}$}
\item $x^{\frac{2}{3}}+y^{\frac{2}{3}}=4$, $(-3\sqrt{3},1)$. 

\answer{$y=\frac{1}{\sqrt{3}}x+4 $}
\item $2(x^2+y^2)^2 =25(x^2-y^2)$, $(3,1)$. 

\answer{$y= -\frac{9}{13} x+\frac{40}{13}$}
\item $y^2(y^2-4)=x^2(x^2-5) $, $(0,-2)$. 

\answer{$y=-2 $}
\item $x^{\frac{4}{3}}+y^{\frac{4}{3}}=10$ at $(-3\sqrt{3}, 1)$. 

\answer{$y=\sqrt{3}x+10$ }
\item $x^2y^3+x^3-y^2=1$ at $(1,1)$. 

\answer{$y=-5 x+6$}
\end{enumerate}
\end{multicols}

\end{problem}
\solution{\ref{problemImplicitTangentysin(2x)=xcos(2y)point(pi/2,pi/4)}

\psset{xunit=0.5cm, yunit=0.5cm}
\begin{pspicture}(-5.8,-5.8)(5.8,5.8)
\fcAxesStandard{-5.5}{-5.5}{5.5}{5.5}
\fcLabels{5.5}{5.5}
\fcXTickWithLabel{1}{$1$}
\fcImplicitIId[linestyle=solid, linecolor=red, linewidth-=0.05, showGridImplicitIId=false]{-5}{-5}{1000}{1000}{0.01}{0.01}{2 x mul 180 mul 3.141592654 div sin y mul 2 y mul 180 mul 3.141592654 div cos x mul sub} 

\psline[linecolor=blue](-5,-2.5)(5,2.5)
\fcFullDot[linecolor=blue]{3.141592654 2 div}{3.141592654 4 div}
\end{pspicture}


First we verify that the point $\displaystyle (x,y)=\left(\frac{\pi}{2}, \frac{\pi}{4}\right)$ indeed satisfies the given equation:

\[
\begin{array}{rcll|l}
\displaystyle y \sin (2x)_{|x=\frac{\pi}{2}, y=\frac{\pi}{4}}= \frac{\pi}{4}\sin \pi &=& \displaystyle 0 && \text{left hand side}\\
\displaystyle x \cos (2y)_{|x=\frac{\pi}{2}, y=\frac{\pi}{4}}= \frac{\pi}{2}\cos \left(\frac{\pi}{2} \right)&=&\displaystyle 0 && \text{right hand side}\\
\end{array}
\]
so the two sides of the equation are equal (both to $0$) when $x=\frac{\pi}{2}$ and $y=\frac{\pi}{4}$.

Since we are looking an equation of the tangent line, we need to find  $\frac{\diff y}{\diff x}_{|x=\frac{\pi}{2}, y= \frac{\pi}{4}}$ - that is, the derivative of $y$ at the point $x=\frac{\pi}{2}$, $y= \frac{\pi}{4}$. To do so we use implicit differentiation.
\[
\begin{array}{rcll|l}
\displaystyle y \sin (2x)&=&\displaystyle x\cos (2y)&&\frac{\diff }{\diff x}\\
\displaystyle \frac{\diff y}{\diff x} \sin (2x) +y \frac{\diff }{\diff x}\left(\sin (2x)\right)&=&\displaystyle  \cos (2y)+x\frac{\diff }{\diff x}(\cos (2y))\\
\displaystyle \frac{\diff y}{\diff x}\sin (2x)+2y\cos (2x)&= & \displaystyle \cos (2y)-2x \sin (2y) \frac{\diff y}{\diff x}\\
\displaystyle \frac{\diff y}{\diff x}(\sin (2x)+2x\sin (2y))&=&\displaystyle \cos (2y)-2y\cos (2x)&& \text{Set }x=\frac{ \pi}{2}, y=\frac{\pi}{4}\\
\displaystyle \frac{\diff y}{\diff x}_{|x=\frac{\pi}{2}, y=\frac{\pi}{4}} \left(\sin \pi+\pi \sin \left(\frac{\pi}{2}\right)\right)&=& \displaystyle \cos \left(\frac{\pi}{2}\right)-\frac{\pi}{2}\cos \pi\\
\displaystyle \pi \frac{\diff y}{\diff x}_{|x=\frac{\pi}{2}, y=\frac{\pi}{4}} &=& -\frac{\pi}{2}\cos \pi\\
\displaystyle \frac{\diff y}{\diff x}_{|x=\frac{\pi}{2}, y=\frac{\pi}{4}}&=& \displaystyle \frac{1}{2}.
\end{array}
\]
Therefore the equation of the line through $x=\frac{\pi}{2}, y=\frac{\pi}{4}$ is 
\[
\begin{array}{rcl}
\displaystyle y-\frac{\pi}{4}&=&\displaystyle \frac{1}{2}\left( x-\frac{\pi}{2} \right)\\
y&=&\displaystyle \frac{1}{2} x .
\end{array}
\]
}

\subsection{Implicit Differentiation and Inverse Trigonometric Functions}
\begin{problem}
% begin homework implicit-inverse-trig1
The variables $x$ and $y$ are related by
\[
x\arctan y + y\arctan x = \frac{\pi}{2}.
\]

\begin{enumerate}
\item   Show that $(1,1)$ is on the graph of this relation.  

\solution{%
\begin{align*}
\text{LS} & = 1\cdot \arctan 1 + 1\cdot \arctan 1 & \text{RS} & = \frac{\pi}{2}. \\
 & = 1\cdot \frac{\pi}{4} + 1\cdot \frac{\pi}{4}  & & \\
 & = \frac{\pi}{2}.  & & 
\end{align*}
The fact that the left side equals the right side when we plug in $x = 1$ and $y = 1$ means that the point $(1,1)$ is on the graph of the relation.  
}%

\item   Find $\frac{\diff y}{\diff x}$ in terms of $x$ and $y$.  

\solution{%
Differentiate implicitly.
\begin{align*}
\big((x)\frac{\diff}{\diff x}(\arctan y) + (\arctan y)\frac{\diff}{\diff x}(x)\big)  + \big((y)\frac{\diff}{\diff x}(\arctan x) + (\arctan x)\frac{\diff}{\diff x}(y)\big)  & = 0 \\
x\cdot \frac{1}{1+y^2}\cdot \frac{\diff y}{\diff x} + (\arctan y)1 + y\cdot \frac{1}{1+x^2} + (\arctan x)\frac{\diff y}{\diff x} & = 0 \\
\frac{x}{1+y^2}\frac{\diff y}{\diff x} + \arctan y + \frac{y}{1+x^2} + \frac{\diff y}{\diff x}\arctan x & = 0.
\end{align*}
Rearrange to isolate $\frac{\diff y}{\diff x}$ on one side.  
\begin{align*}
\frac{x}{1+y^2}\frac{\diff y}{\diff x} +  \frac{\diff y}{\diff x}\arctan x & = - \frac{y}{1+x^2} - \arctan y \\
\big(\frac{x}{1+y^2} + \arctan x\big)\frac{\diff y}{\diff x} & = - \big(\frac{y}{1+x^2} + \arctan y\big) \\
\frac{\diff y}{\diff x} & = -\frac{\frac{y}{1+x^2}+\arctan y}{\frac{x}{1+y^2}+\arctan x}. 
\end{align*}

}%

\item   Find the equation of the tangent to the graph at $(1,1)$.  

\solution{%
To find the slope of the tangent, plug in $x=1,y=1$ to the formula for $\frac{\diff y}{\diff x}$.  

\begin{align*}
\frac{\diff y}{\diff x} & = -\frac{\frac{1}{1+1^2}+\arctan 1}{\frac{1}{1+1^2}+\arctan 1} \\
& = -1.
\end{align*}

Now use the point $(1,1)$ to find an equation for the tangent line.  
\begin{align*}
y - 1 & = (-1)(x-1) \\
y & = -x +2.
\end{align*}
}%

\end{enumerate}
% end homework implicit-inverse-trig1

\end{problem}
\begin{problem}
% begin homework implicit-inverse-trig2
The variables $x$ and $y$ are related by
\[
x^2y+xy^2+\arcsin x = \frac{\pi}{6}.
\]

\begin{enumerate}
\item   Find all points on the graph of this relation for which $x = 1/2$.  

\solution{%
Set $x = 1/2$ and solve for $y$.  
\begin{align*}
\big(\frac{1}{2}\big)^2y+\frac{1}{2}y^2 + \arcsin \frac{1}{2} & = \frac{\pi}{6} \\
\frac{1}{4}y + \frac{1}{2}y^2 + \frac{\pi}{6} & = \frac{\pi}{6} \\
\frac{1}{4}y + \frac{1}{2}y^2  & = 0 \\
\frac{1}{4}y(1 + 2y)  & = 0,
\end{align*}
so $y = 0$ or $y = -1/2$.  
Therefore $(1/2,0)$ and $(1/2,-1/2)$ are the points on the graph of the relation for which $x = 1/2$.  
}%

\item   Find $\frac{\diff y}{\diff x}$ in terms of $x$ and $y$.  

\solution{%
Differentiate implicitly.
\begin{align*}
\big((x^2)\frac{\diff}{\diff x}(y) + (y)\frac{\diff}{\diff x}(x^2)\big) + \big( (x)\frac{\diff}{\diff x}(y^2) + (y^2)\frac{\diff}{\diff x}(x)\big) + \frac{1}{\sqrt{1-x^2}} & = 0 \\
x^2\frac{\diff y}{\diff x} + y(2x) + x(2y)\frac{\diff y}{\diff x} + y^2 + \frac{1}{\sqrt{1-x^2}} & = 0 \\
x^2\frac{\diff y}{\diff x} + 2xy + 2xy\frac{\diff y}{\diff x} + y^2 + \frac{1}{\sqrt{1-x^2}} & = 0.
\end{align*}
Rearrange to isolate $\frac{\diff y}{\diff x}$ on one side.  
\begin{align*}
x^2\frac{\diff y}{\diff x} + 2xy\frac{\diff y}{\diff x} & = -y^2-2xy-\frac{1}{\sqrt{1-x^2}} \\
(x^2+2xy)\frac{\diff y}{\diff x} & = -\Big(y^2+2xy+\frac{1}{\sqrt{1-x^2}}\Big) \\
\frac{\diff y}{\diff x} & = -\frac{y^2+2xy+\frac{1}{\sqrt{1-x^2}}}{x^2+2xy}.
\end{align*}

}%

\item   Find the equation of the tangent to the graph at each of the points you found in the first part.  

\solution{%
To find the slope of the tangent at $(1/2,0)$, plug in $x=1/2,y=0$ to the formula for $\frac{\diff y}{\diff x}$.  

\begin{align*}
\frac{\diff y}{\diff x} & = -\frac{(0)^2+2(1/2)(0) + \frac{1}{\sqrt{1-(1/2)^2}}}{(1/2)^2+2(1/2)(0)} \\
& = -\frac{0+0+\frac{1}{\sqrt{3/4}}}{1/4 + 0} \\
& = -\frac{\frac{1}{\sqrt{3}/2}}{1/4} \\
& = -\frac{2}{\sqrt{3}}\cdot \frac{4}{1} \\
& = -\frac{8}{\sqrt{3}}.
\end{align*}

Now use the point $(1/2,0)$ to find an equation for the tangent line.  
\begin{align*}
y - 0 & = -\frac{8}{\sqrt{3}}(x-1/2) \\
y & = -\frac{8}{\sqrt{3}}x +\frac{4}{\sqrt{3}}.
\end{align*}

This is the equation for one of the tangent lines.  

To find the slope of the tangent at $(1/2,-1/2)$, plug in $x=1/2,y=-1/2$ to the formula for $\frac{\diff y}{\diff x}$.  

\begin{align*}
\frac{\diff y}{\diff x} & = -\frac{(-1/2)^2+2(1/2)(-1/2) + \frac{1}{\sqrt{1-(1/2)^2}}}{(1/2)^2+2(1/2)(-1/2)} \\
& = -\frac{\frac{1}{4}-\frac{1}{2}+\frac{1}{\sqrt{3/4}}}{1/4 -\frac{1}{2}} \\
& = -\frac{-\frac{1}{4}+\frac{2}{\sqrt{3}}}{-\frac{1}{4}} \\
& = \frac{-\frac{1}{4}+\frac{2}{\sqrt{3}}}{\frac{1}{4}} \\
& = 4\big(-\frac{1}{4}+\frac{2}{\sqrt{3}}\big) \\
& = -1 + \frac{8}{\sqrt{3}}.
\end{align*}

Now use the point $(1/2,-1/2)$ to find an equation for the tangent line.  
\begin{align*}
y - (-1/2) & = \Big(-1 + \frac{8}{\sqrt{3}}\Big)(x-1/2) \\
y  & = \Big(-1 + \frac{8}{\sqrt{3}}\Big)x +1/2 - \frac{4}{\sqrt{3}} + 1/2 \\
y  & = \Big(-1 + \frac{8}{\sqrt{3}}\Big)x +1 - \frac{4}{\sqrt{3}},
\end{align*}
and this is the equation of the other tangent line.  
}%


\end{enumerate}
% end homework implicit-inverse-trig2

\end{problem}


\subsection{Derivative of non-Constant Exponent with non-Constant Base}\label{secMPSDerivativeNonConstExponent}
\begin{problem}
Differentiate
\begin{enumerate}
\item \label{problemDifferentiatex^sinx} $x^{\sin x}$.
\answer{$x^{\sin x}\left(\frac{\sin x}{x} +\cos x\ln x\right) $}
\item $x^{\tan x}$.
\answer{$x^{\tan x}\left(\frac{\tan x}{x}+ (\ln x)\sec^2 x \right) $}
\end{enumerate}

\solution{\ref{problemDifferentiatex^sinx}.
$\displaystyle \left(x^{\sin x}\right)'=\left(e^{(\ln x)\sin x}\right)'= e^{(\ln x)\sin x} (\ln x\sin x)'=x^{\sin x}\left( \frac{\sin x}{x}+\ln x\cos x\right) $.
}

\end{problem}
\solution{\ref{problemDifferentiatex^sinx}.
$\displaystyle \left(x^{\sin x}\right)'=\left(e^{(\ln x)\sin x}\right)'= e^{(\ln x)\sin x} (\ln x\sin x)'=x^{\sin x}\left( \frac{\sin x}{x}+\ln x\cos x\right) $.
}


\begin{problem}
Differentiate.
\begin{multicols}{2}
\begin{enumerate}
\item $10^{x^3}$.\answer{$\ln 10 x^{2} (10)^{x^{3}}$}
\item $2^{\tan x}$. \answer{ $(\ln 2) 2^{\tan x}  \sec^2 x $  }
\item $x^x $. \answer{$x^x(\log{}(x) +1)$}
\item $x^{x^x}$. \answer{$(\ln(x))^{2}  x^{x^{x}+x}+x^{x^{x}+x-1}+(\ln x) x^{x^{x}+x}$}
\item $(\sin x)^{\cos x}$. \answer{$\frac{- \ln(\sin{}x)  (\sin{}x)^{\cos{}x+2} +(\sin{}x)^{\cos{}x} \cos^{2}{}x}{\sin{}x}$}
\item $(\ln x)^{\ln x}$. \answer{$\ln{}(\ln{}(x)) x^{-1} (\ln{}(x))^{\ln{}(x)}+x^{-1} (\ln{}(x))^{\ln{}(x)}$}
\end{enumerate}
\end{multicols}
\end{problem}

\subsection{Related Rates}\label{secMPSrelatedRates}
\begin{problem}
\begin{enumerate}[ref={\fcProblemRef}]
% Related Rates
\item \label{problemRelatedRatesFinddr/dtIfds/dtIsKnownS-sphere} A spherical soap bubble is slowly shrinking. If its surface area is decreasing at a rate of 50 square millimeters per
second, how quickly is the radius decreasing when the surface area is 1000 square millimeters?

\answer{$\frac{\diff r}{\diff t} = - \frac{ 50}{ 8 \pi \sqrt{\frac{250}{\pi}}} =- \frac{ \sqrt{10 } }{8 \sqrt{\pi}}$.}

\item \label{problemRelatedRatesCarAlongEllipticalTrack}A car drives along an elliptical track. The track can be modeled by the equation $x^2 + 5y^2 = 14$, where $x$ and $y$ are 
measured in kilometres of distance from the center of the track. As the car passes the point $(3, 1)$, the $x$-coordinate is 
increasing at a rate of $1.5$ km/min. How quickly is the $y$-coordinate changing at that point?

\answer{$\frac{\diff y}{\diff t} = -\frac{9}{10}$ km/min.}
\item \label{problemRelatedRatesGravelCone} Gravel is being dumped from a conveyor belt at a constant rate of $ 500 $ litres per minute. The gravel pile forms a cone with circular base, the diameter of which remains equal to the hight of the cone at any given moment. Use related rates to approximate how fast is the height of the pile increasing when it is 2 meters tall?
\answer{$\frac{\diff h}{\diff t} =  \frac{ 1}{ 2 \pi}$ m/min.} 
\end{enumerate}
\end{problem}
\solution{\ref{problemRelatedRatesFinddr/dtIfds/dtIsKnownS-sphere}
Let $t$ denote time. Let $R$ be the radius of the sphere. Let $S$ be the surface area of the sphere. We are given that 
\[
\frac{\diff S}{\diff t}=-50\quad ,
\] 
where the sign is negative because the bubble is decreasing. We are asked to find 
\[
\frac{\diff R}{\diff t} = ?
\]
$R$ and $S$ are related via
\[
\begin{array}{rcll|l}
\displaystyle S&=&\displaystyle 4\pi R^2 &&\text{differentiate with respect to time}\\
\displaystyle \frac{\diff S}{\diff t}&=&\displaystyle  4\pi *2R*\frac{\diff R}{\diff t}\\
\displaystyle \frac{\diff R}{\diff t}&=&\displaystyle \frac{1}{8\pi R }\frac{\diff S}{\diff t}\quad .
\end{array}
\]
When $S= 1000$, we can find the corresponding value of $R$:
\[
\begin{array}{rcl}
1000&=&\displaystyle  4\pi R^2\\
R^2&=&\displaystyle \frac{250}{\pi }\\
R&=&\displaystyle  \sqrt{\frac{250}{\pi }}\quad .
\end{array}
\]
Finally, we can compute:
\[
\frac{\diff R}{\diff t}_{|S=1000}= \frac{1}{8\pi \underbrace{\sqrt{\frac{250}{\pi }}}_{=R \text{ when }S=1000} }*(\underbrace{-50}_{=\frac{\diff S}{\diff t}})= -\frac{\sqrt{10\pi}}{ 8\pi}\quad.
\]
}

\solution{\ref{problemRelatedRatesCarAlongEllipticalTrack}. 
Let $t$ denote time, and let $t_0$ be the point in time in which the measurements take place. We are given that $\frac{\diff x}{\diff t}_{|t=t_0}=1.5 km/min$,  $x_{|t=t_0}=3$, $y_{|t=t_0}=1$. The problem asks us to find $\frac{\diff y}{\diff t}_{|t=t_0}$. 

Compute:
\[
\begin{array}{rcll|l}
\displaystyle x^2+5y^2&=&14&&\displaystyle \text{apply }\frac{\diff }{\diff t}\\
\displaystyle 2x \frac{\diff x}{\diff t}+10 y \frac{\diff y}{\diff t}&=&0 &&\displaystyle \text{fix time }t=t_0 \\
\displaystyle 2 \cdot 3 \cdot 1.5 +10 \cdot 1\cdot \frac{\diff y}{\diff t}_{|t=t_0}&=&0\\
\displaystyle 10\frac{\diff y}{\diff t}_{|t=t_0}&=&-9\\
\displaystyle \frac{\diff y}{\diff t}_{|t=t_0}&=&\displaystyle -\frac{9}{10}\quad .\\
\end{array}
\]
The measurement unit of $\diff y$ is $km$, the measurement unit of  $t$ is minutes, therefore the answer is $-\frac{9}{10}$km/min.
}


\solution{\ref{problemRelatedRatesGravelCone}. 
\begin{minipage}[t][][b]{0.6\textwidth}
At  time $ t $ (minutes) let $ h $ be the height of the pile (m), let $ r $ be the radius of the base (m), and let $ V $ be the volume of the pile (m$ ^3 $). \\
We are given that $ \frac{dV}{dt}=500 $L/min $ =\frac12 $ m$ ^3 $/min.
We are also given the diameter is equal to the height, so $ 2r=h$.\\ 
We are asked to find $ \frac{dh}{dt} $ when $ h=2$.\\
The formula for the volume of a cone is 
\[  
V=\frac13\pi r^2 h. 
\] 
Since $ r=\frac{h}{2} $ we obtain 
\[
V=\frac13 \pi \frac{h^2}{4}h = \frac{1}{12}\pi h^3
\]
Differentiating with respect to $ t $ then gives
\[
\frac{dV}{dt} = \frac14 \pi h^2 \frac{dh}{dt} \;\; \Rightarrow \;\;\frac{dh}{dt}=\frac{4}{h^2\pi}\frac{dV}{dt}
\]
Substituting $ \frac{dV}{dt}= \frac12 $m$ ^3$/min, and $ h=2 $ we obtain
\[
 \frac{dh}{dt}=\frac{4}{4\pi}\cdot\frac12=\frac{1}{2\pi}\textrm{ m/min}
\]
\end{minipage} 
\begin{minipage}[t][][c]{0.4\textwidth}
\begin{tikzpicture}[scale=1]
    \draw[dashed] (0,0) arc (170:10:2cm and 0.4cm)coordinate[pos=0] (a);
    \draw (0,0) arc (-170:-10:2cm and 0.4cm)coordinate (b);
    \draw[densely dashed] ([yshift=4cm]$(a)!0.5!(b)$) -- node[right,font=\footnotesize] {$h$}coordinate[pos=0.95] (aa)($(a)!0.5!(b)$)
                            -- node[above,font=\footnotesize] {$r$}coordinate[pos=0.1] (bb) (b);
    \draw (aa) -| (bb);
    \draw (a) -- ([yshift=4cm]$(a)!0.5!(b)$) -- (b);
  \end{tikzpicture}
  \end{minipage} 
} %  
\begin{problem}
\begin{problem} (page 180)
If $V$ is the volume of a cube with edge length $x$ and the cube expands as time passes, find $\frac{dV}{dt}$ in terms of $\frac{dx}{dt}$.
\end{problem}
\begin{problem} (page 180)
Each side of a square is increasing at a rate of 6cm/s. At what rate is the area of the square increasing when the area of the square is 16 $cm^2$?
\end{problem}
\begin{problem}(page 180)
The radius of a ball is increasing at a rate of 4 mm/s. How fast is the volume increasing when the diameter is 80mm?
\end{problem}
\begin{problem}(page 181)
A street light is mounted at the top of a 4.5m tall pole. A man 180 cm tall walks away from the pole at a speed of 5km/h along a straight path. How fast is the tip of his shadow moving when he is 12m from the pole?
\end{problem}
\begin{problem}(page 181)
A boat is pulled into a dock by a rope attached to the bow of the boat and passing through a pulley on the dock that is 1m higher than the bow of the boat. If the rope is pulled in at a rate of 1m/s, how fast is the boat approaching the dock when it is 8m from the dock?
\end{problem}
\begin{problem}(page 182)
A Ferris wheel with a radius of 10m is rotating at a rate of one revolution every 2 minutes. How fast is a riding rising when his seat is 16 m above ground level?
\end{problem}
\begin{problem}(page 183)
The minute hand on a watch is 8mm long and the hour hand is 4mm long. How fast is the distance between the tips of the hands changing at one' clock?
\end{problem}
\end{problem}

\solution{
\ref{problemRelatedRatesCubeExpands}. The volume $V$ is given by $V=x^3$, therefore
\[
\frac{\diff V}{\diff t}= \frac{\diff }{\diff t}\left(x^3\right)= 3x^2\frac{\diff x}{\diff t}.
\]
}
\solution{\ref{problemRelatedRatesSquareExpands}
The area $A$ of the square is given by $A=x^2$, therefore
\[
\frac{\diff A}{\diff t}= \frac{\diff }{\diff t}\left(x^2\right)= 2x\frac{\diff x}{\diff t}.
\]
When $A=9 \text{cm}^2$, $x= 3\text{cm} $ ($=\sqrt{9\text{cm}^2}$), and so $\frac{\diff A}{\diff t}_{|x=3, \frac{\diff x}{\diff t}=1} = 2\cdot 3\text{cm} \cdot 1\text{cm/s} =6\text{cm}^2/\text{s}$.
}
\solution{\ref{problemRelatedRatesBallSurfaceAreaChanges}
Let $S$ denote the surface area and $r$ the radius of the ball. Then $S=4\pi r^2$. Let the point of time be $t_0$. We are given that $\frac{\diff S}{\diff t}_{|t=t_0}= 5 \text{cm}^2/\text{s}$. On the other hand, 

\[
\begin{array}{rcl}
\frac{\diff S}{\diff t}&=& \frac{\diff }{\diff t}\left(4\pi r^2\right)= 8\pi r \frac{\diff r}{\diff t}\\
\frac{1}{8\pi}\frac{\diff S}{\diff t}
\end{array}
\]
}
\solution{\ref{problemRelatedRatesWatchHands}
Let the angle between the two arrows be $\theta$. The cosine law states that  for a triangle with angle $\theta$ and sides $a, b, c$ we have that $c^2=a^2+b^2-2ab\cos \theta$ (where $c$ is the length of the side opposite to the angle $\theta$).

Then by the cosine law, the distance between the tips of the two hands is
1\[ 
y=\sqrt{5^2+10^2+2\cdot 5\cdot 10 \cos  \theta}=\sqrt{125+100 \cos \theta}.
\]

The short hand makes 1 full revolution every 12 hours, and the long hand makes 1 full revolution every 1 hour. Therefore the angle $\theta$ measured from the small hand to the long hand changes at the (constant) rate of $\frac{11}{12}$ revolutions per hour, or what is the same, at the rate $\frac{d \theta}{d t} = \frac{11}{12} (2\pi)=\frac{11}{6}\pi$.

The problem asks us to compute $\frac{d y}{d t}$ at two o'clock, i.e., at $t=2$. This is a straightforward computation using the chain rule:
\[
\begin{array}{rcll|l}
\displaystyle \frac{\diff y}{\diff t}&=&\displaystyle \frac{1}{2} \frac{-100\sin \theta}{ \sqrt{125+100\cos \theta}}\frac{\diff \theta}{\diff t}&&\text{at 2 o'clock } \theta=\frac{\pi}{3} \text{ and } \frac{\diff \theta}{\diff t}=\frac{11}{6} \pi\\~\\
\displaystyle {\frac{\diff y}{\diff t}}_{|t=2}&=&\displaystyle \frac{1}{2} \frac{-100 \sin \frac{\pi}{3} }{ \sqrt{125+100 \cos \frac{\pi}{3}}} \frac{11\pi}{6}=-\frac{55}{42}\sqrt{21} \pi \quad .
\end{array}
\]
The measurement unit of speed is mm/hour, so the distance is changing at the rate of $-\frac{55}{42}\sqrt{21}$ mm/hour.
}


\section{Graphical Behavior of Functions}
\subsection{Mean Value Theorem}\label{secMPS-MVT}
\begin{problem}
Use the Intermediate Value theorem and the Mean Value Theorem/Rolle's Theorem to prove that the function has \textbf{exactly one} real root.
\begin{enumerate}
\item \label{problemIVTandMVTx^3+4x+7} $f(x)=x^3+4x+7$.
\item $f(x)= x^3 +x^2+x+1$.
\item \label{problemIVTandMVTcos3xdiv3+sinx-3x} $f(x)=\cos^3 \left({\frac{x}{3}}\right) +\sin x-  3x$.
\end{enumerate}

\end{problem}
\textbf{Solution \ref{problemIVTandMVTx^3+4x+7}.}  $f(-2) = -9$ and $f(1) = 12$. Since $f(x)$ is continuous and has both negative and positive outputs, it must have a zero. In other words, for some $c$ between $-2$ and $1$, $f(c) = 0$. If there were solutions $x = a$ and $x = b$,  then we would have $f(a) = f(b)$, and Rolle's Theorem would guarantee that for some $x$-value, $f'(x) = 0$. However, $f'(x) = 3x^2 + 4$, which always positive and therefore is never 0. Therefore there cannot be 2 or more solutions. 

The above can be stated informally as follows. Note that $f'(x) = 3x^2 + 4$, which is always positive. Therefore, the graph of $f$ is increasing from left to right. So once the graph crosses the $x$-axis, it can never turn around and cross again, so there can only be a single zero (that is, a single solution to $f(x) = 0$).


\textbf{Solution \ref{problemIVTandMVTcos3xdiv3+sinx-3x}.} $f(5)= \cos^3 \left( \frac{5}{ 3} \right) +\sin 5-15 \leq 2-15=-13<0 $ (because $\cos a, \sin b\in [-1,1]$ for arbitrary $a, b$). Similarly $f(-5)=\cos^3\left(-\frac{5}{3}\right) +\sin (-5)+15 \geq 15-2>0$. Therefore by the Intermediate Value Theorem $f(x)=0$ has at least one solution in the interval $[-5,5]$.

Suppose on the contrary to what we are trying to prove, $f(x)=0$ has two or more solutions; call the first 2 solutions $a,b$. That means that $f(a)=f(b)=0$, so by the Mean value theorem, there exists a $c\in (a,b)$ such that $f'(c)=(f(a)-f(b))/(a-b)=(0-0)/(a-b)=0$. On the other hand we may compute:
\[ 
f'(x)=-3+\cos x-\cos^{2}\left(\frac{x}3\right)\sin\left(\frac{x}{3}\right) \leq -1<0,
\] 
where the first inequality follows from the fact that $\sin x,\cos x\in [-1,1]$. So we got that $f'(c)=0$ for some $c$ but at the same time $f'(x)<0$ for all $x$, which is a contradiction. Therefore $f(x)=0$ has exactly one solution. 


\subsection{Maxima, Minima}\label{secMPSoneVariableMinMax}
\subsubsection{Closed Interval method}\label{secMPSclosedInterval}
\begin{problem}
(Textbook page 205)
Find the absolute maximum and absolute minimum values of $f$ on the given interval.
\begin{multicols}{3}
\begin{enumerate}
\item $\displaystyle f(x)=12+4x-x^2$, $x\in [0,5]$.
\item $\displaystyle f(x)=5+54x-2x^3$, $x\in[0,4] $.
\item $\displaystyle f(x)=2x^3-3x^2-12x+1$, $x\in [-2,3]$.
\item $\displaystyle f(x)=x^3-6x^2+5$, $x\in [-3, 5]$.
\item $\displaystyle f(x)=3x^4-4x^3-12x^2+1$, $x\in [-2, 3]$.
\item $\displaystyle f(x)=(x^2-1)^3$, $x\in [-1, 2]$.
\item $\displaystyle f(x)=x+\frac{1}{x}$, $x\in [0.2,4 ]$.
\item $\displaystyle f(x)=\frac{x}{x^2-x+1}$, $x\in [0,3 ]$.
\item $\displaystyle f(t)=t\sqrt{4-t^2}$, $x\in [-1,2 ]$.
\item $\displaystyle f(t)=\sqrt[3]{t}(8-t) $, $x\in [0,8 ]$.
\item $\displaystyle f(t)=2\cos t+\sin (2t)$, $x\in [0,\frac{\pi}{2} ]$.
\item $\displaystyle f(t)=t+\cot (t/2) $, $x\in [\frac{\pi}{4},\frac{7\pi}{4} ]$.
\end{enumerate}
\end{multicols}

\end{problem}


\begin{problem}
% begin homework logarithm-physics
A particle moves in such a way that, after $t$ seconds, it is $s(t) = \ln (2-t+t^2)$ m to the right of the origin.  
\begin{enumerate}
\item  What is the closest it comes to the origin?  

\solution{%
\begin{align*}
s'(t) & = \frac{-1+2t}{2-t+t^2}. \\
\text{Set } \quad s'(t) & = 0. \\
\frac{-1+2t}{2-t+t^2} & = 0 \\
-1+2t & = 0 \\
t & = 1/2.
\end{align*}

Therefore the position function has a critical number at $t = 1/2$.  
The parabola $2-t+t^2$ has a global minimum at $t = 1/2$, and the natural logarithm function is an increasing function, so $\ln(2-t+t^2)$ also has a global minimum at $t=1/2$.  
The minimum value is $s(1/2) = 2-1/2+(1/2)^2 = 7/4$m.  
}%

\item  What is its acceleration when it is closest to the origin?  

\item  For what values of $t$ is the position function $s(t)$ defined?  


\end{enumerate}

\end{enumerate}
% end homework logarithm-physics

\end{problem}

\subsubsection{Optimization}\label{secMPSoptimization}
\begin{problem}

\begin{enumerate}
\item Find the dimensions of a rectangle with area 1000 $m^2$ whose perimeter is as small as possible.
\item A box with an open top is to be constructed from a square piece of cardboard, 1m wide, by cutting out a square from each of the four corners and bending up the sides. Find the largest volume that such a box can have.
\item A right circular cylinder is inscribed in a sphere of radius $r$. Find the largest possible volume of such a cylinder.
\item A wedge of radius $2$ (depicted below) is folded into a cone cup. The volume varies depending on the angle of the wedge. Find the maximal possible volume of the cone cup and the angle of the wedge for which this maximal volume is achieved.
\psset{xunit=0.5cm, yunit=0.5cm}
\begin{pspicture}(-1.5, -1.5)(1.5,1.5) 
\tiny 
\pscustom*[linecolor=cyan!30]{ \psparametricplot[algebraic] {2.35619}{7.06858} {0+1*cos(t)| 0+1*sin(t)} \psline(0.707107, 0.707107)(0, 0)(-0.707107, 0.707107)}

\psparametricplot[algebraic,linecolor=blue]{2.35619}{7.06858}{cos(t)| sin(t)} 
\psline[linecolor=red](0.707107, 0.707107)(0, 0)(-0.707107, 0.707107)

\rput[t](0.4, 0.2){$r$}
\rput[lb](0.8,0.8){$B$}
\rput[rb](-0.8,0.8){$A$}
\rput[b](0,0.3){$O$}
\end{pspicture} 

\end{enumerate}

\end{problem}
\begin{problem}
\begin{enumerate}[ref={\fcProblemRef}]
% Optimization
\item \label{problemponthyperbolax^2-4y^2closestTo1,1} What is the $x$-coordinate of the point on the hyperbola $x^2 - 4y^2 = 16$ that is closest to the point $(1, 0)$?

\psset{xunit=0.3cm, yunit=0.3cm}
\begin{pspicture}(-10.500000, -5)(10.500000,5)
\psframe*[linecolor=white](-10.500000,-5)(10.500000,5)
\tiny
\fcAxesStandard{-10.000000}{-4.5}{10.000000}{4.5} %Function formula: - (1/4 x^{2}-4)^{1/2}
\psplot[linecolor=\fcColorGraph, plotpoints=1000]{-10.000000}{-4.000000}{-4 x 2 exp 0.25 mul add 0.5 exp -1 mul }
%Function formula: (1/4 x^{2}-4)^{1/2}
\psplot[linecolor=\fcColorGraph, plotpoints=1000]{-10.000000}{-4.000000}{-4 x 2 exp 0.25 mul add 0.5 exp }
%Function formula: - (1/4 x^{2}-4)^{1/2}
\psplot[linecolor=\fcColorGraph, plotpoints=1000]{4.000000}{10.000000}{-4 x 2 exp 0.25 mul add 0.5 exp -1 mul }
%Function formula: (1/4 x^{2}-4)^{1/2}
\psplot[linecolor=\fcColorGraph, plotpoints=1000]{4.000000}{10.000000}{-4 x 2 exp 0.25 mul add 0.5 exp }
\fcFullDot{1}{0}
\fcFullDot{4}{0}
\pscircle[linestyle=dotted](1,0){0.9}
\end{pspicture}

\answer{$x = 4$}

\item What is the $x$-coordinate of the point on the ellipse $x^2+4y^2=16$ closest to the point $(1,0)$?

\psset{xunit=0.3cm, yunit=0.3cm}
\begin{pspicture}(-4.500000, -5)(4.500000,5)
\psframe*[linecolor=white](-4.500000,-5)(4.500000,5)
\tiny
\psline[linecolor=red!1](-10,0)(-9.99,0)
\fcAxesStandard{-4.000000}{-4.5}{4.000000}{4.5} %Function formula: - (-1/4 x^{2}+4)^{1/2}
\psplot[linecolor=\fcColorGraph, plotpoints=1000]{-4.000000}{4.000000}{4 x 2 exp -0.25 mul add 0.5 exp -1 mul }
%Function formula: (-1/4 x^{2}+4)^{1/2}
\psplot[linecolor=\fcColorGraph, plotpoints=1000]{-4.000000}{4.000000}{4 x 2 exp -0.25 mul add 0.5 exp }
\pscircle[linestyle=dotted](1,0){0.574456}
\fcFullDot{1}{0}
\fcFullDot{1.333333}{1.885618}
\fcFullDot{1.333333}{-1.885618}
\end{pspicture}
\answer{$x=\frac43$}
\item \label{problemMaxVolumeBoxFixedAreaDoubleBottomNoLid} You want to build a rectangular box with a square base out of sheet metal. You are going to use 2 pieces of sheet metal for the bottom of the box to reinforce it, and only a single piece of sheet metal for all of the sides and the top. If you want to use no more than $36$ sq. ft. of material, what is the largest possible volume you can enclose?

\answer{12 cubic feet.}
\end{enumerate}

\end{problem}
\solution{\ref{problemponthyperbolax^2-4y^2closestTo1,1}

The distance function between an arbitrary point $(x,y)$ and the point $(1,0)$ is $d=\sqrt{(x-1)^2+(y-0)^2}$. On the other hand, when the point $(x,y)$ lies on the hyperbola we have $y^2= \frac{x^2 -16 }{4 }$. In this way, the problem becomes that of minimizing the distance function

\[
dist(x)=\sqrt{(x-1)^2+y^2}=\sqrt{(x-1)^2+\frac{x^2-16}{4}} \quad .
\]
This is a standard optimization problem: we need to find the critical endpoints, i.e., the points where $dist'=0$. As the square root function is an increasing function, the function $\displaystyle \sqrt{(x-1)^2+\frac{x^2-16}{4}}$ achieves its minimum when the function 
\[
l=dist^2=(x-1)^2+\frac{x^2-16}{4}\quad 
\]
does. $l$ is a quadratic function of $x$ and we can directly determine its minimimum via elementary methods. Alternatively, we find the critical points of $l$:
\[
\begin{array}{rcl}
\displaystyle l'&=&\displaystyle 0\\
\displaystyle 2(x-1)+\frac{x}{2} &=&\displaystyle 0\\
\displaystyle \frac{5}{2}x-2&=&0\\
\displaystyle x&=&\displaystyle \frac{4}{5}\quad .
\end{array}
\]
On the other hand, $x^2=16+4y^2$ and therefore $|x|\geq \sqrt{16} = 4$. Therefore $x\in (-\infty, -4]\cup [4,\infty)$. As $x= \frac{4}{5 }$ is outside of the allowed range, it follows that our either function attains its minimum at one of the endpoints $\pm 4$ or the function has no minimum at all. It is clear however that as $x$ tends to  $\infty$, so does $dist$. Therefore $dist$ attains its maximum for $x=4$ or $-4$ and $y=\pm\sqrt{(\pm4)^2-16}=0$. Direct check shows that $dist_{|x=4} =\sqrt{(4-1)^2 +\frac{4^2- 16}{4 }}=3$ and $dist_{|x=-4}=\sqrt{(-4-1)^2+\frac{4^2-16}{4}}=5$  so our function $dist$ has a minimal value of $3$ achieved when $x=4$, which is our final answer. Notice that this answer can be immediately given without computation by looking at the figure drawn for \ref{problemponthyperbolax^2-4y^2closestTo1,1}. Indeed, it is clear that there are no points from the hyperbola lying inside the dotted circle centered at $(1,0)$. Therefore the point where this circle touches the hyperbola must have the shortest distance to the center of the circle.
}

\subsection{Function Graph Sketching}\label{secMPSfunctionGraphSketching}
\begin{problem}

\begin{problem}
Find the 
\begin{itemize}
\item $x$ and $y$ intercepts of $f$.
\item horizontal and vertical asymptotes.
\item intervals of increase and decrease
\item local and global minima, maxima,
\item intervals of concavity 
\item points of inflection
\end{itemize}
Label all relevant points on the graph. 
\begin{enumerate}
\item $\displaystyle f(x)=\frac{x+1/2}{x^{2}+x+1}$
\psset{xunit=1cm, yunit=1cm}
\begin{pspicture}(-5, -5)(5,5) 
\psframe*[linecolor=white](-5,-5)(5,5) 
\tiny 
\psaxes[ticks=none, labels=none]{<->}(0,0)(-5,-0.5)(5,1.5)
\psLabels{5}{1.5}
%Function formula: \frac{x+1/2}{x^{2}+x+1} 
\psplot[linecolor=\psColorGraph, plotpoints=1000]{-5}{5}{0.5 x add 1 x add x 2 exp add div }
\end{pspicture} 
\answer{
\begin{tabular}{l}
Intervals of decrease: $ (-\infty,\frac{-1-\sqrt{3}}{2})\cup (\frac{-1+\sqrt{3}}{2}, \infty) $, intervals of decrease $(\frac{-1-\sqrt{3}}{2}, \frac{-1+\sqrt{3}}{2})$ \\
Inflection points at: $x=-2$, $x= -\frac12$, $x=1 $ \\
\end{tabular}
}

\item $\displaystyle f(x)=\frac{2 x^{2}-5 x+9/2}{x^{2}-3 x+3}$
\psset{xunit=1cm, yunit=1cm}
\begin{pspicture}(-5, -5)(5,5) 
\psframe*[linecolor=white](-5,-5)(5,5) 
\tiny 
\psaxes[ticks=none, labels=none]{<->}(0,0)(-5,-0.5)(5,3.5)
\psLabels{5}{3.5}
%Function formula: \frac{2 x^{2}-5 x+9/2}{x^{2}-3 x+3} 
\psplot[linecolor=\psColorGraph, plotpoints=1000]{-5}{5}{4.5 x -5 mul add x 2 exp 2 mul add 3 x -3 mul add x 2 exp add div }
\end{pspicture} 
\item $\displaystyle f(x)=\frac{2 \sqrt{- x^{2}+1}+1}{\sqrt{- x^{2}+1}+1}$,  $f(x)=\frac{1}{\sqrt{- x^{2}+1}+1}$ 
\psset{xunit=1cm, yunit=1cm}
\begin{pspicture}(-1, -5)(1,5) 
\psframe*[linecolor=white](-1,-5)(1,5) 
\tiny 
\psaxes[ticks=none, labels=none]{<->}(0,0)(-1,-0.5)(1,3.5)
\psLabels{1}{3.5}
%Function formula: \frac{2 (- x^{2}+1)^{1/2}+1}{(- x^{2}+1)^{1/2}+1} 
\rput(1,3){} 
\psplot[linecolor=brown, plotpoints=1000]{-1}{1}{1 1 x 2 exp -1 mul add 0.5 exp 2 mul add 1 1 x 2 exp -1 mul add 0.5 exp add div }
%Function formula: \frac{1}{(- x^{2}+1)^{1/2}+1} 
\rput(1,3){} 
\psplot[linecolor=\psColorGraph, plotpoints=1000]{-1}{1}{1 1 1 x 2 exp -1 mul add 0.5 exp add div }

\end{pspicture} 

Both functions are drawn on the same plot. Indicate which part of the graph is the graph of which function.
\item $\displaystyle f(x)=\frac{e^x+e^{-x}}{e^x-e^{-x}}$
\psset{xunit=0.5cm, yunit=0.5cm}
\begin{pspicture}(-4, -5)(4,5) 
\psframe*[linecolor=white](-4,-5)(4,5) 
\tiny 
\psaxes[ticks=none, labels=none]{<->}(0,0)(-4,-4.5)(4,4.5)
\psLabels{4}{5}
%Function formula: \frac{e^{- x}+e^{x}}{- e^{- x}+e^{x}} 
\psplot[linecolor=\psColorGraph, plotpoints=1000]{0.2}{4}{2.718281828 x exp 2.718281828 x -1 mul exp add 2.718281828 x exp 2.718281828 x -1 mul exp -1 mul add div }
%Function formula: \frac{e^{- x}+e^{x}}{- e^{- x}+e^{x}} 
\psplot[linecolor=\psColorGraph, plotpoints=1000]{-4}{-0.2}{2.718281828 x exp 2.718281828 x -1 mul exp add 2.718281828 x exp 2.718281828 x -1 mul exp -1 mul add div }
\end{pspicture} 
\item $\displaystyle f(x)=\frac{- e^{- x}+e^{x}}{e^{- x}+e^{x}}$
\psset{xunit=1cm, yunit=1cm}
\begin{pspicture}(-4, -5)(4,5) 
\psframe*[linecolor=white](-4,-5)(4,5) 
\tiny 
\psaxes[ticks=none, labels=none]{<->}(0,0)(-4,-1.1)(4,1.1)
\psLabels{4}{1.1}
%Function formula: \frac{- e^{- x}+e^{x}}{e^{- x}+e^{x}} 
\psplot[linecolor=\psColorGraph, plotpoints=1000]{-4}{4}{2.718281828 x exp 2.718281828 x -1 mul exp -1 mul add 2.718281828 x exp 2.718281828 x -1 mul exp add div }
\end{pspicture}
\item $\displaystyle f(x)=\ln{}\left(\frac{{{x}}+1}{- {{x}}+1}\right)$
\psset{xunit=1cm, yunit=1cm}
\begin{pspicture}(-0.9, -5)(1,5) 
\psframe*[linecolor=white](-0.9,-5)(1,5) 
\tiny 
\psaxes[ticks=none, labels=none]{<->}(0,0)(-1.3,-4)(1.3,4)
\psLabels{1.3}{4}
%Function formula: \log{}(\frac{x+1}{- x+1}) 
\psplot[linecolor=\psColorGraph, plotpoints=1000]{-0.94}{0.94}{1 x add 1 x -1 mul add div ln }

\end{pspicture} 
\item $f(x)=\frac{x^{2}+3 x+1}{x^{2}+2 x}$
\psset{xunit=0.7cm, yunit=0.7cm}
\begin{pspicture}(-5, -5)(5,5) 
\psframe*[linecolor=white](-5,-5)(5,5) 
\tiny 
\psaxes[ticks=none, labels=none]{<->}(0,0)(-5,-4.5)(5,4.5)
\psLabels{5}{5}
%Function formula: \frac{x^{2}+3 x+1}{x^{2}+2 x} 
\psplot[linecolor=\psColorGraph, plotpoints=1000]{0.1}{5}{1 x 3 mul add x 2 exp add x 2 mul x 2 exp add div }
%Function formula: \frac{x^{2}+3 x+1}{x^{2}+2 x} 
\psplot[linecolor=\psColorGraph, plotpoints=1000]{-1.9}{-0.1}{1 x 3 mul add x 2 exp add x 2 mul x 2 exp add div }
%Function formula: \frac{x^{2}+3 x+1}{x^{2}+2 x} 
\psplot[linecolor=\psColorGraph, plotpoints=1000]{-5}{-2.1}{1 x 3 mul add x 2 exp add x 2 mul x 2 exp add div }

\end{pspicture} 
\end{enumerate}
\end{problem}
\end{problem}
\solution{\ref{problemSketchCurve(2x^2-5x+9/2)/(x^2-3 x+3)}

\textbf{Domain.} We have that $f$ is not defined only when we have division by zero, i.e.,  if $x^2-3x+3$ equals zero. However, the roots of $x^{2}-3x+3$ are not real numbers: they are $\frac{3\pm \sqrt{3^2-4\cdot 3 }}{2}= \frac{3\pm \sqrt{-3}}{2}$, and therefore $x^2-3x+3$ can never equal zero. Alternatively, completing the square shows that the denominator is always positive:
\[
x^2-3x+3=x^2-2\cdot \frac{3}{2} x+\frac{9}{4}-\frac{9}{4}+3=\left(x-\frac{3}{2}\right)^2+\frac{3}{4} >0 
\]
Therefore the domain of $f$ is all real numbers.

\textbf{$x$, $y$-intercepts.}  The $y$-intercept of $f$ equals by definition $\displaystyle f(0)= \frac{ 2\cdot 0^2-5\cdot 0+ \frac{9}{2}}{0^2-3\cdot 0 + 3}=\frac{\frac{9}{2}}{3}= \frac{3}{2}$. The $x$ intercept of $f$ is those values of $x$ for which $f(x)=0$. The graph of $f$ shows no such $x$, and that is confirmed by solving the equation $f(x)=0$:

\[
\begin{array}{rcll|l}
f(x)&=&0\\
\displaystyle \frac{2x^2-5x+\frac{9}{2}}{x^2-3 x+3}&=&0&&\text{Mult. by }x^2-3 x+3\\
\displaystyle 2x^2-5x+\frac{9}{2}&=&0\\
\displaystyle x_1, x_2&=&\displaystyle \frac{5 \pm \sqrt{25- 4\cdot 2\cdot \frac{9}{2}}}{4}=\frac{5\pm \sqrt{-9}}{4}\quad ,
\end{array}
\]
so there are no real solutions (the number $\sqrt{-9}$ is not real).

\textbf{Asymptotes.} Since $f$ is defined for all real numbers, its graph has no vertical asymptotes. To find the horizontal asymptote(s), we need to compute the limits $\lim\limits_{x\to \infty } f(x)$ and $\lim\limits_{x\to -\infty} f (x)$. The two limits are equal, as the direct computation below shows:
\[
\begin{array}{rcll|l}
\displaystyle \lim_{x\to \pm\infty} \frac{2x^2-5x+\frac{9}{2}}{x^2-3 x+3}&=& \displaystyle  \lim_{x\to \pm\infty}\frac{\left(2x^2-5x+ \frac{9}{2}\right)\frac{1}{x^2}}{\left(x^2-3 x+3\right)\frac{1}{x^2}} &&\begin{array}{l}\text{Divide by leading}\\ \text{monomial in denominator}\end{array}\\
&=&\displaystyle\lim_{x\to \pm \infty}\frac{2-\frac{5}{x} +\frac{9}{2x^2}}{1-\frac{3}{x}+\frac{3}{x^2}}\\
&=&\displaystyle \frac{2-0+0}{1-0+0}\\
&=& 2
\end{array}
\]
Therefore the graph of $f(x)$ has a single horizontal asymptote at $y=2$.

\textbf{Intervals of increase and decrease.}
The intervals of increase and decrease of $f$ are governed by the sign of $f'$. We compute:

\[
\begin{array}{rcl}
f'(x)&=&\displaystyle \left(\frac{2x^2-5x+\frac{9}{2} }{x^2- 3 x+3} \right)' \\
&=&\displaystyle \frac{\left(2x^2-5x+\frac{9}{2}\right)'\left(x^2- 3 x+3\right)-\left(2x^2-5x+\frac{9}{2}\right)\left(x^2- 3 x+3\right)' }{ \left(x^2- 3 x+3\right)^2}\\
&=&\displaystyle \frac{- x^{2}+3 x-\frac{3}{2} }{ \left(x^2- 3 x+3\right)^2}
\end{array}
\]
As the denominator is a square, the sign of $f'$ is governed by the sign of $- x^{2}+3 x-\frac{3}{2}$. To find where $- x^{2}+3 x-\frac{3}{2}$ changes sign, we compute the zeroes of this expression:

\[
\begin{array}{rcll|l}
\displaystyle - x^{2}+3 x-\frac{3}{2}&=&0&& \text{Mult. by }-2\\
\displaystyle  2x^{2}-6 x+3&=&0\\
x_1, x_2&=&\displaystyle \frac{ 6\pm \sqrt{36-24 }}{4}=\frac{6\pm \sqrt{12}}{4}\\
x_1, x_2&=&\displaystyle \frac{3\pm \sqrt{3}}{2} 
\end{array}
\]
Therefore the quadratic $- x^{2}+3 x-\frac{3}{2}$ factors as 
\begin{equation}
\label{eq1problemSketch(2x^2-5x+9/2)/(x^2-3 x+3)}
-(x-x_1)(x-x_2)=-\left(x-\left(\frac{3- \sqrt{3}}{2} \right)\right)\left(x-\left(\frac{3+ \sqrt{3}}{2}\right)\right)
\end{equation} 

The points $x_1, x_2$ split the real line into three intervals: $\left(-\infty, \frac{3- \sqrt{3}}{2}\right)$, $\left(\frac{3- \sqrt{3}}{2}, \frac{3+ \sqrt{3}}{2} \right)$ and $\left(\frac{3+ \sqrt{3}}{2}, \infty \right)$, and each of the factors of \eqref{eq1problemSketch(2x^2-5x+9/2)/(x^2-3 x+3)} has constant sign inside each of the intervals. If we choose $x$ to be a very negative number, it follows that $-(x-x_1)(x-x_2)$ is a negative, and therefore $ f'(x)$ is negative for $x\in(-\infty, \frac{3- \sqrt{3}}{2})$. For $x\in (\frac{3- \sqrt{3}}{2}, \frac{3+ \sqrt{3}}{2})$, exactly one factor of $f'$ changes sign and therefore $f'(x)$ is positive in that interval; finally only one factor of $f'(x)$ changes sign in the last interval so $f'(x)$ is negative on $(\frac{3+ \sqrt{3}}{2}, \infty )$.

Our computations can be summarized in the following table. 

\begin{tabular}{|lll|}\hline
Interval & $f'(x)$ & $f(x)$   \\\hline
$\left(-\infty, \frac{3- \sqrt{3}}{2}\right)$ & $-$& $\searrow $ \\\hline
$\left(\frac{3- \sqrt{3}}{2}, \frac{3+ \sqrt{3}}{2} \right)$ &$+$&$\nearrow$\\\hline
$\left( \frac{3+ \sqrt{3}}{2}, \infty\right)$&$-$&$\searrow$ \\\hline
\end{tabular}

\textbf{Local and global minima and maxima. } The table above shows that $f(x)$ changes from decreasing to increasing at $x=x_1=\frac{3- \sqrt{3}}{2}$ and therefore $f$ has a local minimum at that point. The table also shows that $f(x)$ changes from increasing to decreasing at $ x=x_2=\frac{3+ \sqrt{3}}{2}$ and therefore $f$ has a local maximum at that point. The so found local maximum and local minimum turn out to be global: there are two things to consider here. First, no other finite point is critical and thus cannot be maximum or minimum - however this leaves out the possibility of a maximum/minimum ``at infinity''. This possibility can be quickly ruled out by looking at the graph of $f$. To do so via algebra, compute first $f(x_1)$ and $f(x_2)$:

\[
\begin{array}{rcl}
\displaystyle f(x_1)= f\left(\frac{3- \sqrt{3}}{2} \right)&=& \displaystyle \frac{2\left(\frac{3- \sqrt{3}}{2} \right)^2-5\left(\frac{3- \sqrt{3}}{2} \right)+\frac{9}{2} }{\left(\frac{3- \sqrt{3}}{2} \right)^2- 3 \left(\frac{3- \sqrt{3}}{2} \right)+3}=2-\frac{\sqrt{3}}{3} \\

\displaystyle f(x_2)= f\left(\frac{3+ \sqrt{3}}{2} \right)&=& \displaystyle \frac{2\left(\frac{3+ \sqrt{3}}{2} \right)^2-5\left(\frac{3+ \sqrt{3}}{2} \right)+\frac{9}{2} }{\left(\frac{3+ \sqrt{3}}{2} \right)^2- 3 \left(\frac{3+ \sqrt{3}}{2} \right)+3}=2+\frac{\sqrt{3}}{3}\quad . 
\end{array}
\]
On the other hand, while computing the horizontal asymptotes, we established that $\lim\limits_{x\to\pm \infty}f(x)=2$. This implies that all $x$ sufficiently far away from $x=0$, we have that $f(x)$ is close to $2 $. Therefore $f(x)$ is larger than $f(x_1)$ and smaller than $f(x_2)$ for all sufficiently far away from $x=0$. This rules out the possibility for a maximum or a minimum ``at infinity'', as claimed above.

\textbf{Intervals of concavity. } 
The intervals of concavity of $f$ are governed by the sign of $f''$. The second derivative of $f$ is:
\[
\begin{array}{rcll|@{}l}
f''(x)&=&\displaystyle (f'(x))'= \left( \frac{- x^{2}+3 x-\frac{3}{2} }{ \left(x^2- 3 x+3\right)^2 } \right)'\\
&=&\displaystyle \left(- x^{2}+3 x-\frac{3}{2} \right)' \left(\frac{1 }{\left(x^2- 3 x+3\right)^2}\right)+ \left(- x^{2}+3 x-\frac{3}{2} \right)\left(\frac{1}{\left(x^2- 3 x+3\right)^2}\right)' &&\begin{array}{@{}l}\text{second differentiation:}\\\text{chain rule }\end{array}\\
&=&\displaystyle (-2x+3)\left(\frac{1}{\left(x^2- 3 x+3\right)^2} \right)+\left(- x^{2}+3 x-\frac{3}{2} \right)(-2)\frac{\left(x^2- 3 x+3\right)'}{\left(x^2- 3 x+3\right)^{3}}\\
&=&\displaystyle (-2x+3)\left(\frac{1}{\left(x^2- 3 x+3\right)^2}\right) +\left(2x^{2}-6 x+3 \right) \frac{(2x-3)}{\left(x^2- 3 x+3\right)^{3}}&&\text{factor out }\frac{(2x-3)}{\left(x^2- 3 x+3\right)^2}\\
&=&\displaystyle \frac{(2x-3)}{\left(x^2- 3 x+3\right)^2}\left(-1+\frac{(2x^{2}-6 x+3)}{\left(x^2- 3 x+3\right)}\right)\\
&=&\displaystyle \frac{(2x-3)}{\left(x^2- 3 x+3\right)^2}\left(\frac{-\left(x^2- 3 x+3\right)+(2x^{2}-6 x+3)}{\left(x^2- 3 x+3\right)} \right)\\
&=&\displaystyle \frac{(2x-3)(x^{2}-3 x )}{\left(x^2- 3 x+3\right)^3}\\
&=&\displaystyle \frac{(2x-3)x(x-3)}{\left(x^2- 3 x+3\right)^3}
\end{array}
\]
When computing the domain of $f$, we established that the denominator of the above expression is always positive. Therefore $f''(x)$ changes sign when the terms in the numerator change sign, namely, at $x=0$, $x=\frac{3}{2}$ and $x=3$. 

Our computations can be summarized in the following table. In the table, we use the $\cup$ symbol to denote that the function is concave up in the indicated interval, and $\cap$ to denote that the function is concave down.

\begin{tabular}{|lll|}\hline
Interval & $f''(x)$ & $f(x)$   \\\hline
$(-\infty, 0)$ & $-$& $\cap$ \\\hline
$(0, \frac{3}{2})$ &$+$&$\cup$\\\hline
$(\frac{3}{2}, 3)$&$-$&$\cap$ \\\hline
$(3, \infty)$&$+$&$\cup$ \\\hline
\end{tabular}

\textbf{Points of inflection.} The preceding table shows that $f''(x)$ changes sign at $0, \frac{3}{2}, 3$ and therefore the points of inflection are located at $x=0, x=\frac{3}{2}$ and $x=3$, i.e., the points of inflection are $\left(0, f(0)\right)= \left(0, \frac{3}{2} \right) $, $\left(\frac {3}{2}, f\left(\frac{3}{2}\right)\right) =\left(\frac{3}{2}, 2\right)$, $\left(3, f(3)\right)=\left(3, \frac{5}{2}\right)$.

We can command our graphing device to use the so computed information to label the graph of the function. Finally, we can confirm visually that our function does indeed behave in accordance with our computations.

\psset{xunit=0.6cm, yunit=0.6cm}
\begin{pspicture}(-5, -5)(5,5)
\psframe*[linecolor=white](-5,-5)(5,5)
\tiny
\psaxes[ticks=none, labels=none]{<->}(0,0) (-5,-0.5) (5, 3.5)
\fcLabels{5}{3.5}
%Function formula: \frac{2 x^{2}-5 x+9/2}{x^{2}-3 x+3}
\psplot[linecolor=\fcColorGraph, plotpoints=1000]{-5}{5 } {4.5 x -5 mul add x 2 exp 2 mul add 3 x -3 mul add x 2 exp add div }
\fcFullDot[linecolor=green]{0}{3 2 div}
\rput[r](-2, 0.2){infl.: $\left(0, \frac{3}{2}\right)$}
\psline[linestyle=dotted, arrows=->](-2, 0.2)(-0.05, 1.45)
\fcFullDot[linecolor=green]{3 2 div }{2}
\rput[r](1.2, 0.2){infl.: $\left(\frac{3}{2},2 \right)$}
\psline[linestyle=dotted, arrows=->](1.2, 0.2)(1.45, 1.95)
\fcFullDot[linecolor=green]{3 }{5 2 div}
\rput[l](2, 0.2){infl.: $\left(3, \frac{5}{2}\right)$}
\psline[linestyle=dotted, arrows=->](2, 0.2)(2.95, 2.45)
\fcFullDot[linecolor=blue]{3 3 sqrt add 2 div}{2 3 sqrt 3 div add}
\rput(2, 3){$\left(\frac{3+\sqrt{3}}{2}, 2+\frac{\sqrt{3}}{3} \right)$}
\fcFullDot[linecolor=blue]{3 3 sqrt sub 2 div}{2 3 sqrt 3 div sub}
\rput[r](-2, 3){$\left(\frac{3-\sqrt{3}}{2}, 2-\frac{\sqrt{3}}{3} \right)$}
\psline[linestyle=dotted, arrows=->](-2, 3)(! 3 3 sqrt sub 2 div 0.05 add 2 3 sqrt 3 div sub 0.05 add)
\end{pspicture}

}

\solution{\ref{problemSketch(x+1)/(x^2+2x+4)} 

\textbf{This problem is very similar to Problem \ref{problemSketchCurve(2x^2-5x+9/2)/(x^2-3 x+3)}. We recommend to the student to solve the problem first ``with closed textbook'' and only then to compare with the present solution.}

\textbf{Domain.} As $f$ is a quotient of two polynomials (rational function), its implied domain is all $x$ except those for which we get division by zero for $f$. Consequently the domain of $f$ is all $x$ for which $x ^2+2x+4=0$. However, the polynomial $x^2+2x+4$ has no real roots - its roots are $\displaystyle \frac{-2\pm \sqrt{4-16} }{2}=-1\pm \sqrt{-3}$, and therefore the domain of $f$ is all real numbers. Alternatively, we can complete the square: $x^2+2x+4=(x+1)^2+3$ and so $x^2+2x+4$ is positive for all values of $x$. 

\textbf{$x$, $y$-intercepts.} The $y$-intercept of $f$ equals by definition $\displaystyle f(0)= \frac{ 0+ 1}{0^2+2\cdot 0 + 4}=\frac{1}{4}$. The $x$ intercept of $f$ is those values of $x$ for which $f(x)=0$. We compute

\[
\begin{array}{rcl}
\displaystyle f(x)&=&0\\
\displaystyle \frac{x+1}{x^2+2x+4}&=&0\\
x+1&=&0\\
x&=&-1\quad ,
\end{array}
\]
and the $x$-intercept of $f$ is $x=-1$. 

\textbf{Asymptotes.} The line $x=a$ is a vertical asymptote when $\lim\limits_{x\to a^{\pm}}f(x)=\pm \infty$; as $f$ is defined for all real numbers, this implies that there are no vertical asymptotes. 
 
The line $y=L$ is a horizontal asymptote if $\lim\limits_{x\to\pm \infty}f(x)$ exists and equals $L$. We compute:
\[
\lim\limits_{x\to \infty} f(x)=\lim\limits_{x\to \infty} \frac{(x+1)\frac{1}{x^2}}{ (x^2+ 2x +4)\frac{1}{x^2}} = \lim\limits_{x\to \infty}\frac{\frac{1}{x}+\frac{1}{x^2}}{ 1+ \frac{2}{x} +\frac{4}{x^2}}=\frac{0+0}{1+0+0}=0
\]
Therefore $y=0$ is a horizontal asymptote for $f$. An analogous computation shows that $\lim\limits_{x\to\pm \infty}f(x)=0$ and so $y=0$ is the only horizontal asymptote of $f$.

\textbf{Intervals of increase and decrease.} 
The intervals of increase and decrease of $f$ are governed by the sign of $f'$. We compute:
\[
\begin{array}{rcll|l}
f'(x)&=&\displaystyle \left(\frac{x+1}{x^2+2x+4}\right)' &&\text{qutotient rule}\\
&=&\displaystyle \frac{(x+1)'\left(x^2+ 2x+4\right)- (x+1)\left( x^2 +2 x+4\right)'}{\left(x^2+2x+4 \right)^2}\\
&=&\displaystyle\frac{ x^2+2x+4-(x+1)(2x+2)}{\left(x^2+2x+4 \right)^2}\\
&=&\displaystyle \frac{x^2+2x+4-\left( 2x^2+ 4x+ 2 \right)}{ \left( x^2 +2x+4 \right)^2}\\
&=&\displaystyle \frac{-x^2-2x+2}{\left(x^2+2x+4 \right)^2}
\end{array}
\] 
As $x^2+2x+4$ is positive, the sign of $f'$ is governed by the sign of $-x^2+2x+2$. To find out where $-x^2+2x+2$ changes sign, we compute the zeroes of this expression:
\[\begin{array}{rcll|l}
-x^2-2x+2&=&0\\
x^2+2x-2&=&0 &&\text{use the quadratic formula}\\
x_1, x_2&=& -1\pm \sqrt{3}\quad .
\end{array}
\]
Therefore the quadratic $-x^2+2x+2$ factors as 
\begin{equation}
\label{eq1problemSketch(x+1)/(x^2+2x+4)}
-(x-x_1)(x-x_2)=-\left(x-\left(-1-\sqrt{3}\right)\right)\left(x-\left(-1+\sqrt{3}\right)\right)
\end{equation} 
The points $x_1, x_2$ split the real line into three intervals: $(-\infty, -1-\sqrt{3})$, $(-1-\sqrt{3}, -1+\sqrt{3})$ and $(-1+ \sqrt{3}, \infty )$, and each of the factors of \eqref{eq1problemSketch(x+1)/(x^2+2x+4)} has constant sign inside each of the intervals. If we choose $x$ to be a very negative number, it follows that $-(x-x_1)(x-x_2)$ is a negative, and therefore $ f'(x)$ is negative for $x\in(-\infty, -1-\sqrt{3})$. For $x\in (-1-\sqrt{3}, -1+\sqrt{3})$, exactly one factor of $f'$ changes sign and therefore $f'(x)$ is positive in that interval; finally only one factor of $f'(x)$ changes sign in the last interval so $f'(x)$ is negative on $(-1+ \sqrt{3}, \infty )$.

Our computations can be summarized in the following table. 

\begin{tabular}{|lll|}\hline
Interval & $f'(x)$ & $f(x)$   \\\hline
$(-\infty, -1-\sqrt{3})$ & $-$& $\searrow $ \\\hline
$(-1-\sqrt{3}, -1+\sqrt{3})$ &$+$&$\nearrow$\\\hline
$( -1+\sqrt{3}, \infty)$&$-$&$\searrow$ \\\hline
\end{tabular}

\textbf{Local and global minima and maxima. } The table above shows that $f(x)$ changes from decreasing to increasing at $x=x_1=-1-\sqrt{3}$ and therefore $f$ has a local minimum at that point. The table also shows that $f(x)$ changes from increasing to decreasing at $ x=x_2=-1+\sqrt{3}$ and therefore $f$ has a local maximum at that point. The so found local maximum and local minimum turn out to be global: indeed, no other finite point is critical and thus cannot be maximum or minimum; on the other hand $\lim\limits_{x\to\pm \infty}f(x)=1$ and this implies that all $x$ sufficiently far away from $x=0$ have that $f(x)$ is close to $0$, and therefore $f(x)$ is larger than $f(x_1)$ and smaller than $f(x_2)$ for all $x$.

\textbf{Intervals of concavity. } 
The intervals of concavity of $f$ are governed by the sign of $f''$. The second derivative of $f$ is:
\[
\begin{array}{rcll|l}
f''(x)&=&\displaystyle (f'(x))'= \left(\frac{-x^2-2x+2}{\left(x^2+2x+4 \right)^2}\right)'\\
&=&\displaystyle (-x^2-2x+2)'\left(\frac{1}{(x^2+2x+4)^2}\right)+(-x^2-2x+2)\left(\frac{1}{(x^2+2x+4)^2}\right)' &&\begin{array}{l}\text{use chain rule }\\\text{for second differentiation}\end{array}\\
&=&\displaystyle (-2x-2)\left(\frac{1}{(x^2+2x+4)^2}\right)+(-x^2-2x+2)(-2)\frac{(x^2+2x+4)'}{(x^2+2x+4)^{3}}\\
&=&\displaystyle -(2x+2)\left(\frac{1}{(x^2+2x+4)^2}\right) +(2x^2+4x-4)\frac{(2x+2)}{(x^2+2x+4)^{3}}&&\text{factor out }\frac{(2x+2)}{(x^2+2x+4)^2}\\
&=&\displaystyle \frac{(2x+2)}{(x^2+2x+4)^2}\left(-1+\frac{(2x^2+4x-4)}{(x^2+2x+4)}\right)\\
&=&\displaystyle \frac{(2x+2)}{(x^2+2x+4)^2}\left(\frac{-(x^2+2x+4)+(2x^2+4x-4)}{(x^2+2x+4)} \right)\\
&=&\displaystyle \frac{(2x+2)(x^{2}+2 x-8)}{(x^2+2x+4)^3}&& \text{factor } (x^2+2x-8)\\
&=&\displaystyle \frac{(2x+2)(x+4)(x-2)}{(x^2+2x+4)^3}
\end{array}
\]
As we previously established, the denominator of the above expression is always positive. Therefore the expression above changes sign when the terms in the numerator change sign, namely, at $x=-1$, $x=-4$ and $x=2$. 

Our computations can be summarized in the following table. 

\begin{tabular}{|lll|}\hline
Interval & $f''(x)$ & $f(x)$   \\\hline
$(-\infty, -4)$ & $-$& $\cap$ \\\hline
$(-4, -1)$ &$+$&$\cup$\\\hline
$(-1, 2)$&$-$&$\cap$ \\\hline
$(2, \infty)$&$+$&$\cup$ \\\hline
\end{tabular}

\textbf{Points of inflection.} The preceding table shows that $f''(x)$ changes sign at $-4, -1, 2$ and therefore the points of inflection are located at $x=-4, x=-1$ and $x=2$, i.e., the points of inflection are $\left(-4, -\frac{1}{4}\right)$, $\left(-1, 0\right)$, $\left(2, \frac{1}{4}\right)$.

}

\begin{problem}
\begin{enumerate}
\item Sketch the graph of $y = x^4 - 8x^2 + 8$ by determining the intervals of increase and decrease, finding the local mins and maxes, determining where the graph is concave up and concave down, and plotting a few key points.

\answer{
\begin{tabular}{l}
Check your graph with a calculator or online graphing program. \\
Local max at 0, local mins at 2 and -2. Concave down between $-\sqrt{4/3}$ and $\sqrt{4/3}$, and concave up otherwise.
\end{tabular}
}

\item Sketch the graph of $y = \frac{x-1}{x^2-9}$ by graphing any vertical and horizontal asymptotes, finding the $x$- and $y$-intercepts, and then sketching a graph that fits this information.

\answer{
\begin{tabular}{l}
Check your graph with a calculator or online graphing program. \\
Vertical asymptotes at $x = 3$ and $x = -3$. \\ Horizontal
asymptote at $y = 0$. \\
$y$-intercept of $\frac{1}{9}$; $x$-intercept of $1$.
\end{tabular}
}
\end{enumerate}

\end{problem}

\section{Linearizations and Differentials} \label{secMPSLinearizationAndDifferentials}
\begin{problem}
\begin{enumerate}[ref={\fcProblemRef}]
\item Find the linearization of the function $f(x) = \sqrt{x}$ at $a = 100$, and then use your new function to approximate
$\sqrt{99.8}$.

\answer{$L(x) = 10 + 0.05(x-100)$. Therefore $\sqrt{99.8} \approx L(99.8) = 9.99$.}

\item Use a linear approximation to estimate $(1.001)^9$. 

\answer{$(1.001)^9 \approx 1.009$.}

\item $f(x)=\sqrt{8+x}$ at $a=1$ and use it to approximate $\sqrt{9.02}$.

\answer{ $f(x)\approx 3+ \frac16 (x-1)=\frac{1}{6} x+\frac{17}{6}$. Therefore $\sqrt{9.02}\approx \frac{901}{300} \approx 3.003333$}
\item $f(x)=\sqrt[3]{8+x}$ at $a=0$ and use it to approximate $\sqrt[3] {7.97}$.

\answer{ $\sqrt[3]{8+x}\approx \frac{1}{12}x+2$. Therefore $\sqrt[3]{7.97}\simeq \frac{799}{400} =1.9975$}

\item \label{problem-linearization-estimate0.9999power2014} Use a linear approximation to estimate $(0.9999)^{2014}$. 

\answer{$(0.9999)^{2014} \approx 0.7986$.}

\item Find the linearization of $f(x)=\ln x$ at $a=1$ and use it to approximate $\ln 1.01$.

\answer{ $f(x)\approx f(1)+f'(1)(x-1)=x-1 $, $\ln 1.01\approx 0.01$. }

\end{enumerate}

\end{problem}
\input{../../modules/differentials/homework/linearization-problems-1-solutions}

\section{Integration Basics}
\subsection{Riemann Sums}\label{secMPSRiemannSums}
\begin{problem}
Estimate the integral using a Riemann sum using the indicated sample points and interval length.
\begin{enumerate}[ref={\fcProblemRef}]
% Riemann sums
\item\label{problemRiemannSum-sqrt(8x+1)} $\displaystyle \int_0^4 \left(\sqrt{8x+1}\right)\diff x$. Use four intervals of equal width, choose the sample point to be the left endpoint of each interval. 

\answer{ $\Delta x = 1$ and $f(x) = \sqrt{8x+1}$. Thus ${\displaystyle \int_0^4 f(x) \diff x \approx 9 + \sqrt{17}}$.}

\item $\displaystyle \int_0^6 \frac{1}{x^2+1} \diff x$. Use three intervals of equal width, choose the sample point to be the left endpoint. 

\answer{ $\Delta x = 2$ and $f(x) = \frac{1}{x^2+1}$. Thus ${\displaystyle \int_0^6 f(x) \diff x \approx \frac{214}{85}}$.}
\item\label{problemRiemannSum-1div1plusxsquared} $\displaystyle \int\limits_{-3.5}^{-0.5} \frac{\diff x}{x^2+1} $. Use three intervals of equal width, choose the sample point to be the midpoint of each interval. 

\answer{ $\Delta x = 1$ and $f(x) = \frac{1}{x^2+1}$. Thus $\displaystyle \int \limits_{-3.5}^{-0.5} f(x) \diff x  \approx \Delta x\left(f{} \left(-3 \right)+ f{}\left( -2\right)+f{}\left(-1\right)\right)=\frac{4}{5}=0.8$.}

\item $\displaystyle\int_{0}^2 \frac{\diff x}{1+x+x^3}$. Use $\Delta x=\frac{1}2 $ and right endpoint sampling points.

\answer{$ \frac{1}{2}\left(\frac{8}{13}+\frac{1}{3}+\frac{8}{47}+\frac{1}{11}\right)=\frac{12197}{20163}\approx 0.604920$}
\item $\displaystyle\int_{-2}^{0} \frac{\diff x}{1+x+x^2}$. Use $\Delta x=\frac23 $ and left endpoint sampling points.

\answer{$\frac23\left(\frac{1}{3}+\frac{9}{13}+\frac{9}{7}\right)=\frac{1262}{819}\approx 1.540904$}

\item $\displaystyle \int\limits_0^2 \frac{\diff x}{1+x^3}$. Use four intervals of equal width, choose the sample point to be the left endpoint of each interval. 

\answer{ $\Delta x = 0.5$ and $f(x) = \frac{1}{1+x^3}$. Thus $\displaystyle \int\limits_0^2 f(x) \diff x  \approx \Delta x\left(f{}\left(0\right)+f{}\left(1\right)+f{}\left(\frac{1}{2}\right)+f{}\left(\frac{3}{2}\right)\right)=\frac{1649}{1260}\approx 1.30873$.}

\item $\displaystyle \int\limits_{-2}^{0} \frac{\diff x}{x^4+1} $. Use four intervals of equal width, choose the sample point to be the right endpoint. 

\answer{ $\Delta x = 0.5$ and $f(x) = \frac{1}{1+x^3}$. Thus $\displaystyle \int\limits_0^2 f(x) \diff x  \approx \Delta x\left(f{}\left(-\frac{3}{2}\right)+f{}\left(-1\right)+f{}\left(-\frac{1}{2}\right)+f{}\left(0\right)\right)=\frac{8595}{6596}\approx 1.303062$.}

\end{enumerate}



\end{problem}
\input{\freecalcBaseFolder/modules/integration/homework/riemann-sum-problems-1-problem-1-solution}
\input{\freecalcBaseFolder/modules/integration/homework/riemann-sum-problems-1-problem-3-solution}
\solution{\ref{problemRiemannSum1/(3x^2+1)from-1to0with3intervalsLeftEndpt}

$\Delta x = \frac{1}{3}$ and $f(x) =\frac{1}{3 {{x}}^{2}+1}$. Thus $\displaystyle \int\limits_{-1}^0 f(x) \diff x$  is approximated by $\Delta x \left(f{}\left(-1\right)+f{}\left(-\frac{2}{3}\right)+f{}\left(-\frac{1}{3}\right)\right)=\frac{10}{21}$.

}


\subsection{Antiderivatives}\label{secMPSantiderivatives}
\begin{problem}
\input{../../modules/antiderivatives/homework/antiderivatives-basic-integrals-1}
\end{problem}
\begin{problem}
\input{../../modules/antiderivatives/homework/antiderivatives-basic-integrals-initial-condition-1}
\end{problem}
\begin{problem}
Verify by differentiation that the formula is correct.
\begin{multicols}{2}
\begin{enumerate}
\item $\displaystyle \int\sqrt{1+x^2}\diff x=  \frac{1}{2}\left( x \sqrt{1+x^2} +\ln \left(x+\sqrt{1+x^2}\right)+C \right) $.
\item $\displaystyle \int \sin^2x \diff x=-\frac{1}{4} \sin{}(2 x)+\frac{1}{2} x+C$.
\item $\displaystyle \int \sin^3 x \diff x =\frac{1}{3} \cos^{3}{}x- \cos{}x+C$.
\item $\displaystyle \int \frac{x}{\sqrt{1+x}}\diff x= \frac{2}{3}(x-2)\sqrt{1+x}+C$
\end{enumerate}
\end{multicols}
\end{problem}

\subsection{Basic Definite Integrals} \label{secMPSBasicDefiniteIntegrals}
\begin{problem}
Evaluate the definite integral.
\begin{multicols}{3}
\begin{enumerate}[ref={\fcProblemRef}]
\item $\displaystyle \int\limits_{-2}^{3} \left(x^2-1 \right)  \diff x$.

\answer{$\left[ \frac{1}{3} x^{3}- x \right]_{-2}^3=\frac{20}{3}$}

\item $\displaystyle \int\limits_{1}^{2} \left(4x^3+3x^2+2x+1\right)  \diff x$.

\answer{$\left[ x^{4}+x^{3}+x^{2}+x\right]_{1}^{2}=26$}
\item $\displaystyle \int\limits_{0}^{2}(x-1)(x^2+1)  \diff x$.

\answer{$\left[\frac{1}{4} x^{4}-\frac{1}{3} x^{3}+\frac{1}{2} x^{2}- x \right]_{0}^{2} = \frac{4}{3}$}
\item $\displaystyle \int\limits_{-1}^{1} \left( \frac{x(x+1) }{ 2} \right)^2  \diff x$.

\answer{$\left[\frac{1}{20} x^{5}+\frac{1}{8} x^{4}+\frac{1}{12} x^{3} \right]_{-1}^{1}=\frac{4}{15}$}
\item $\displaystyle \int\limits_{0}^{1}(1+x^2)^3 dx$.

\answer{$\left[\frac{1}{7} x^{7}+\frac{3}{5} x^{5} + x^{3} + x \right]_{0}^{1}=\frac{96}{35}$}
\item $\displaystyle \int \limits_{1}^{2} \left(\frac{1}{x} - \frac{4}{x^2} \right)  \diff x$.

\answer{$\left[ 4 x^{-1}+\ln x\right]_{1}^{2}=\ln 2-2$}
\item $\displaystyle \int\limits_{1}^{4}\sqrt{x}(1+x) \diff x$.

\answer{$\left[\frac{2}{5} x^{\frac{5}{2}}+\frac{2}{3} x^{\frac{3}{2}} \right]_{1}^{4}=\frac{256}{15}$}

\item $\displaystyle \int\limits_{1}^{4} \sqrt{ \frac{6 }{x }} \diff x$.

\answer{$\left[2 \sqrt{6} \sqrt{x} \right]_{1}^{4}=2 \sqrt{6} $}
\item $\displaystyle \int \limits_{1}^{4} \frac{ \frac{ 1}{ \sqrt{x}}+1+x}{ \sqrt{x}}  \diff x$.

\answer{$\left[ \frac{2}{3} x^{\frac{3}{2}}+2 \sqrt{x}+\ln{}\left|x\right|\right]_{1}^{4}=\ln{}\left(4\right)+\frac{20}{3}$}

\item $\displaystyle \int \limits_{1}^{8} \frac{1+x}{ \sqrt[3]{x}} \diff x$.

\answer{$\left[\frac{3}{5} x^{\frac{5}{3}}+\frac{3}{2} x^{\frac{2}{3}}  \right]_{1}^{8}=\frac{231}{10}$}
\item $\displaystyle \int\limits_{1}^{64} \frac{\frac{1}{ \sqrt[3]{x}} +\sqrt[3]{x}}{ \sqrt{x}}\diff x$.

\answer{$\left[\frac{6}{5} x^{\frac{5}{6}}+6 x^{\frac{1}{6}}  \right]_{1}^{64}=\frac{216}{5}$}
\item $\displaystyle \int\limits_0^{1} \left(\sqrt[5]{x^6} + \sqrt[6]{x^5}\right) \diff x $.

\answer{$\left[\frac{5}{11} x^{\frac{11}{5}}+\frac{6}{11} x^{\frac{11}{6}} \right]_{0}^{1}=1$}

\item $\displaystyle \int\limits_{1}^{2} \left(x + \frac{1}{x} \right)^2 \diff x$.

\answer{$\left[frac{1}{3} x^{3}- x^{-1}+2 x  \right]_{1}^{2}= \frac{29}{6}$}
\item $\displaystyle \int\limits_{1}^{2} \left(x + \frac{1}{x} \right)^3 \diff x$.

\answer{$\left[\frac{1}{4} x^{4}+\frac{3}{2} x^{2}-\frac{1}{2} x^{-2}+3 \ln{}x  \right]_{1}^{2}=\frac{69}{8}+3 \ln{}2 $}
\item $\displaystyle \int\limits_{1}^{2} \left(\sqrt{x} + \frac{1}{\sqrt{x}} \right)^2 \diff x$.

\answer{$\left[ \frac{1}{2} x^{2}+\ln{}x+2 x \right]_{1}^{2}=\frac{7}{2} +\ln{}\left(2\right) $}
\item $\displaystyle \int\limits_{1}^{2} \left(\sqrt{x} +\frac{1}{\sqrt{x}} \right)^3 \diff x$.

\answer{$\left[\frac{2}{5} x^{\frac{5}{2}}+2 x^{\frac{3}{2}}+6 \sqrt{x}-2 x^{-\frac{1}{2}} \right]_{1}^{2}= \frac{53}{5}\sqrt{2}-\frac{32}{5} $}
\item $\displaystyle \int\limits_{0}^{2}|x-1| \diff x$.

\answer{$1$}
\item \label{problemIntegralAbsoluteValuexminushalf} $\displaystyle \int\limits_{0}^{1} \left|x-\frac{1}{2}\right| \diff x$.

\answer{$\frac{1}{4}$}
\item $\displaystyle \int\limits_{-1}^{1}(x-3|x|) \diff x$.

\answer{$-3$}
\item $\displaystyle \int\limits_{\frac{\pi}{4}}^{\frac{\pi}{2}} \csc^2\theta \diff \theta$.

\answer{$\left[ \right]_{}^{}=$}
\item $\displaystyle \int\limits_{0}^{\frac{\pi}{4}}\frac{1-\cos^2\theta}{\cos^2\theta} \diff \theta$.

\answer{$\left[ \right]_{}^{}=$}
\item $\displaystyle \int\limits_{0}^{\frac{\pi}{4}}\frac{\sin^2\theta}{\cos^2\theta} \diff \theta$.

\answer{$\left[ \right]_{}^{}=$}
\item $\displaystyle \int\limits_{0}^{\frac{\pi}{4}}\tan^2\theta \diff \theta$.

\answer{$\left[ \right]_{}^{}=$}
\item $\displaystyle \int\limits_{0}^{\frac{\pi}{3}} \frac{\sin \theta +\sin \theta \tan^2\theta}{\sec^2\theta} \diff \theta$.

\answer{$\left[ \right]_{}^{}=$}
\item $\displaystyle \int\limits_{0}^{\pi} (\sin \theta -\cos \theta) \diff \theta$.

\answer{$\left[ \right]_{}^{}=$}
\item $\displaystyle \int\limits_{0}^{\pi}|\sin x| \diff x$.
\end{enumerate}
\end{multicols}

\end{problem}
\solution{\ref{problemIntegralAbsoluteValuexminushalf}
\[
\begin{array}{rcll|l}
\displaystyle \int\limits_{0}^{1} \left|x-\frac{1}{2}\right| \diff x
&=&\displaystyle  \int\limits_{0}^{\frac{1}{2}}  \left|x-\frac{1}{2}\right| \diff x+\int\limits_{\frac{1}{2}}^1  \left|x-\frac{1}{2}\right| \diff x&&\begin{array}{l}\left|x-\frac{1}{2}\right| =\frac{1}{2}-x \text{ when }x\leq \frac{1}{2} \\
\left|x-\frac{1}{2}\right| =x-\frac{1}{2} \text{ when }x\geq \frac{1}{2}
\end{array}\\
&=&\displaystyle  \int\limits_{0}^{\frac{1}{2}} \left(\frac{1}{2}-x\right) \diff x+\int \limits_{\frac{1}{2}}^1 \left(x-\frac{1}{2}\right) \diff x\\
&=&\displaystyle \left[-\frac{x^2}{2}+\frac{x}{2}\right]_{0}^{\frac{1}{2}} +\left[\frac{x^2}{2}-\frac{x}{2}\right]_{\frac{1}{2}}^{1}\\
&=&\displaystyle \left(-\frac{1}{8}+\frac{1}{4}\right) +\left(\frac{1}{2} -\frac{1}{2}-\left(\frac{1}{8}-\frac{1}{4} \right)\right)\\
&=&\displaystyle \frac{1}{4}
\end{array}
\]
}
\begin{problem}
Evaluate the definite integral.
\begin{enumerate}
\item $\displaystyle\int\limits_{1}^{8} \frac{t-t^{\frac{1}{3}}+ 2}{ t^{\frac{4}{3}}} \diff t\quad .$
\answer{$- \ln8+\frac{15}{2}$}
\item $\displaystyle\int\limits_{1}^{4} \left(x+\sqrt{x}\right)^2 dx\quad .$
\answer{$\frac{119}{2}$}
\end{enumerate}

\end{problem}
\subsection{Fundamental Theorem of Calculus Part I}\label{secMPSFTCpart1}
\begin{problem}
Differentiate $f(x)$ using the Fundamental Theorem of Calculus part 1.
\begin{enumerate}
\item $f(x)=\int\limits_{0}^{x^2} t^2\diff t $.
\answer{$f'(x)=2x^5$ }

\item $f(x)=\int\limits_{\ln x}^{e^x} t\diff t $.

\answer{$f'(x)=e^{2x}-\frac{\ln x}{x}$ }


\end{enumerate}
\end{problem}
\solution{\ref{problemd/dx(int_0^(x^2)t^2dt)}


}


\subsection{Integration with The Substitution Rule}
\label{secMPSintegrationSubstitutionRule}
\subsubsection{Substitution in Indefinite Integrals}
\label{secMPSintegrationSubstitutionRuleIndefinite}
\begin{problem}
Evaluate the indefinite integral.
\begin{multicols}{3}
\begin{enumerate}
\item $\displaystyle\int (1+3x)^9 \diff x $.

\answer{$\displaystyle \frac{ (1+3x)^{10}}{30} +C$.}
\item $\displaystyle\int \left(\sqrt{2x+1}\right)\diff x $.

\answer{$\displaystyle \frac{(2x+1)^{\frac{3}{2}}}{3}+C$}
\item $\displaystyle\int (3x+2)^{2.4}\diff x $.

\answer{$\displaystyle \frac{(3x+2)^{3.4}}{10.2}+C$}
\item $\displaystyle\int (x-1)\sqrt{2x-x^2} \diff x $.

\answer{$\displaystyle -\frac{ \left(2x-x^2\right)^{\frac{3}{2 }}}{ 3} +C$}
\item $\displaystyle\int x\sqrt{1-x^2} \diff x $.

\answer{$\displaystyle $}
\item $\displaystyle\int \frac{1+x^2}{\sqrt{3x+x^3}}\diff x $.
\item $\displaystyle\int (x^2+1)(x^3+4x)^5 \diff x $.
\item $\displaystyle\int \frac{x^2}{\sqrt[3]{1+x^3}} \diff x $.
\item $\displaystyle\int x^2\left(\sqrt{1+x}\right)\diff x $.
\item $\displaystyle\int x(2x+5)^{2014} \diff x $.
\item $\displaystyle\int x^3\left(\sqrt{x^2+1}\right) \diff x $.

\item $\displaystyle\int \sqrt{x}\sin (2+x^{\frac{3}{2}}) \diff x $.
\item $\displaystyle\int \frac{\cos\left(\frac{\pi}{x}\right)}{x^2} \diff x $.

\item $\displaystyle\int \csc^2(2t) \diff t$.
\item $\displaystyle \int \sec (5t) \tan (5t) \diff t $.
\item $\displaystyle\int \frac{\cos t}{\sin t} \diff t $.
\item $\displaystyle\int \tan t \diff t $.
\item $\displaystyle\int \cot (2t) \diff t $.
\item $\displaystyle\int \frac{\sin \sqrt{t}}{\sqrt{t}} \diff t $.
\item $\displaystyle\int \sec^2t \tan^3t \diff t$.
\item $\displaystyle\int \cos^4t\sin t \diff t$.
\item $\displaystyle\int \frac{\diff t}{\cos^2 t\sqrt{1+\tan t}} $.
\item $\displaystyle\int \sqrt{\cot t} \csc^2 t\diff t $.
\item $\displaystyle\int \sin t \sec^2(\cos t)\diff t $.
\item $\displaystyle\int \sec^3 t \tan t\diff t $.
\item $\displaystyle\int t \sin(t^2) \diff t $.

\end{enumerate}
\end{multicols}


\end{problem}
\solution{\ref{problemIntegrate(1+3x)^9}
\[
\begin{array}{rcll|l}
\displaystyle\int (1+3x)^9\diff x&=&\displaystyle \int (1+3x)^9 \frac{\diff (3x)}{3} &&\text{differentials are linear: } \diff (3x)= (3x)' \diff x= 3\diff x\\
&=&\displaystyle \int (1+3x)^9 \frac{\diff (1+3x)}{3} &&\text{differentials don't change when we add constants}\\
&=&\displaystyle \frac{1}{3}\int u^9 \diff u &&\text{Set } u=1+3x\\
&=&\frac{1}{30}u^{10}+C=\frac{(1+3x)^{10}}{30}+C
\end{array}
\]
}
\begin{problem}
Evaluate the integral. The answer key has not been proofread, use with caution.
\begin{multicols}{3}
\begin{enumerate}[ref={\fcProblemRef}]
\item $\displaystyle\int \frac{\diff x}{3x+5} $.

\answer{$\frac{1}{3}\ln |3x+5|+C$}

\item $\displaystyle\int \frac{\diff x}{2-3x}$.

\answer{$-\frac{1}{3}\ln |2-3x|+C$}
\item $\displaystyle\int e^x\cos (e^x) \diff x$.

\answer{$ \sin \left(e^x\right)+C$}
\item $\displaystyle\int \frac{(\ln x)^3}{x} \diff x$.

\answer{$ \frac{(\ln x)^4}{4} +C$}

\item \label{probleminte^x(sqrt(e^x+1))dx} ${\displaystyle \int e^x \left(\sqrt{e^x + 1}\right) \diff x}$

\answer{${\displaystyle \frac23 (e^x + 1)^{\frac{3}{2}} + C}$}
\item $\displaystyle\int e^x\sqrt{1-e^x} \diff x$.

\answer{$ -\frac{2}{3}\left(1-e^x\right)^{\frac{3}{2}} +C$}
\item $\displaystyle\int e^{\sin t }\cos t \diff t$.

\answer{$ e^{\sin t}+C$}
\item $\displaystyle\int e^{\cot x}\csc^2x \diff x$.

\answer{$ -e^{\cot x}+C$}
\item $\displaystyle\int \frac{x}{1+x^2} \diff x$. 

\answer{$\frac{1}{2} \ln\left(x^2+1\right)+C $}

\item $\displaystyle\int \frac{x}{2+3x^2} \diff x$. 

\answer{$\frac{1}{6} \ln{}\left(x^{2}+\frac{2}{3}\right)+C$}

\item $\displaystyle\int \frac{x}{\sqrt{1-x^2}} \diff x$. 

\answer{$-\sqrt{1-x^2}+C $}
\item $\displaystyle\int \frac{\cos \left(\ln x\right)}{x} \diff x$.

\answer{$ \sin (\ln x)+C$}

\item \label{problemintsin(2x)/(2+cos^2x)dx} $\displaystyle\int \frac{\sin (2x)}{2+\cos^2x}\diff x$.

\answer{$-\ln (2+\cos^2x)+C $}
\item ${\displaystyle \int \frac{\cos x}{\sin x} \diff x}$

\answer{${\displaystyle \ln |\sin x| + C}$}
\item $\displaystyle\int \cot x \diff x$.

\answer{$\ln |\sin x|+C $}

\item $\displaystyle \int \cot \left(\frac{x}{2}\right) \diff x$

\answer{$2\ln\left|\sin \left(\frac{x}{2}\right) \right|+C$}

\item $\displaystyle\int \tan (2x) \diff x$.

\answer{$-\frac{1}{2}\ln|\cos (2x) | +C$}
\item ${\displaystyle \int \frac{x^4 + 3x}{x^2} \diff x}$

\answer{${\displaystyle \frac{x^3}{3} + 3 \ln |x| + C}$}

\item ${\displaystyle \int x^2 e^{x^3} \diff x}$

\answer{${\displaystyle \frac{1}{3}e^{x^3} + C}$}
\item \label{problemIntArctan(x)/(1+x^2)dx} $\displaystyle\int \frac{\Arctan x}{1+x^2} \diff x$. 

\answer{$ \frac{(\Arctan x)^2}{2}+C$}
\end{enumerate}
\end{multicols}
\end{problem}
\solution{\ref{probleminte^x(sqrt(e^x+1))dx}.
\[
\begin{array}{rcll|l}
\displaystyle\int e^x\sqrt{e^x+1} ~ \diff x&=& \displaystyle \int \sqrt{e^x+1} ~ \diff \left(e^x\right)\\
&=&\displaystyle \int \sqrt{e^x+1}~ \diff \left(e^x+1\right)
\displaystyle &&\text{Set }u=e^x+1\\
&=&\displaystyle \int \sqrt{u}~ \diff u\\
&=&\displaystyle \frac{2}{3}u^{\frac{3}{2}}+C\\
&=&\displaystyle \frac{2}{3}\left(e^x+1\right)^{\frac{3}{2}}+C
\end{array}
\]
}

\solution{\ref{problemintsin(2x)/(2+cos^2x)dx}
\[
\begin{array}{rcll|l}
\displaystyle\int \frac{\sin (2x)}{2+\cos^2x}\diff x&=&\displaystyle\int \frac{2\cos x \sin x  \diff x }{2+\cos^2x}&&\text{use } \sin (2x)=2\sin x \cos x\\
&=& \displaystyle\int \frac{2\cos x   \diff(-\cos x) }{2+\cos^2x}&&\text{use }\diff (\cos x)=-\sin x \diff x\\
&=&\displaystyle-\int \frac{2 u   \diff(u) }{2+u^2} &&\text{set }u=\cos x\\
&=&\displaystyle-\int \frac{   \diff\left(2+u^2\right) }{2+u^2}&&\text{use }\diff (u^2+2)=2u\diff u\\
&=&\displaystyle -\int \frac{   \diff z }{z}&&\text{set }z= 2+u^2\\
&=&\displaystyle -\ln |z| +C &&\text{Substitute back }z=u^2+2\\
&=&\displaystyle -\ln (u^2+2) +C &&\begin{array}{l} u^2+2\text{ is positive}\\\Rightarrow \text{omit the abs. value}\\ \text{Substitute back }u=\cos x \end{array}\\
&=&\displaystyle -\ln (\cos^2x+2) +C .
\end{array}
\]

}

\solution{\ref{problemIntegrate(sin(lnx))/xdx}
\[
\begin{array}{rcll|l}
\displaystyle \int\frac{\sin(\ln x)}{x}\diff x &=&\displaystyle \int \sin(\ln x)\diff \left(\ln x\right)&&u=\ln x\\
&=&\displaystyle \int \sin u\diff u\\
&=&-\cos u+C\\
&=&-\cos(\ln x)+C
\end{array}
\]
}


\subsubsection{Substitution in Definite Integrals}
\label{secMPSintegrationSubstitutionRuleDefinite}
\begin{problem}
Evaluate the definite integral. The answer key has not been proofread, use with caution.
\begin{enumerate}[ref={\fcProblemRef}]
\item $\displaystyle\int\limits_{e}^{e^3}\frac{\diff x}{x \sqrt[3]{\ln x}} $.

\answer{$ \frac{3}{2}\left( \sqrt[3]{9} -1\right)$}

\item $\displaystyle\int\limits_{0}^{1}xe^{-x^2} \diff x$.

\answer{$ \frac{1-e^{-1}}{2}$}
\item $\displaystyle\int\limits_{0}^{1}\frac{e^x+1}{e^x+x} \diff x$.

\answer{$\ln(e+1) $}
\item \label{problemIntx/(2x^2+1)} $\displaystyle\int\limits_{1}^{2} \frac{x}{2x^2+1 }  \diff x$.

\answer{$\frac14 \ln 3$}

\item \label{problemIntegratefrom-3to2_x/(1-x^2)dx}

$\displaystyle\int_{-3}^{-2} \frac{x}{1-x^2}\diff x$.

\answer{$\left[-\frac{1}{2}\ln \left|1-x^2\right| \right]_{-3}^{-2}=\frac{1}{2} \ln{}\left(\frac{8}{3}\right) $}
\item \label{problemintfrom-3to-2of3x/(2-x^2)dx}

$\displaystyle\int_{-3}^{-2} \frac{3x}{2-x^2}\diff x$.

\answer{$\left[-\frac{3}{2}\ln\left| 2-x^2\right|\right]_{-3}^{-2}= \frac{3}{2}\ln\frac{7}{2}$}

\item $\displaystyle\int\limits_{0}^{\frac{1}4}\frac{x }{\sqrt{1-3x^2}}\diff x$.

\answer{$\frac{1}3\left(1-\sqrt{\frac{13}{16}} \right)$}

\end{enumerate}

\end{problem}
\solution{\ref{problemIntx/(2x^2+1)}
\[
\begin{array}{rcll|l}
\displaystyle\int\limits_{1}^{2} \frac{x}{2x^2+1 }  \diff x&=&\displaystyle \int \limits_{x=1}^{x=2} \frac{\frac{1}{4}\diff (2x^2)}{2x^2+1 }  =
\frac{1}{4}\int \limits_{x=1}^{x=2} \frac{\diff (2x^2+1)}{2x^2+1 } &&\text{Set }u=2x^2+1\\
&=&\displaystyle \frac{1}{4}\int\limits_{\substack{x=1 \\u=3} }^{\substack{ x=2\\u=9}} \frac{\diff u}{u} = \frac{1}{4}\left[ \ln u\right]_{3}^9=\frac{1}{4}\left(\ln 9-\ln 3\right)=\frac{\ln 3}{4}.
\end{array}
\]
}
\solution{\ref{problemIntegratefrom-3to2_x/(1-x^2)dx}

\[\begin{array}{rcll|l}
\displaystyle \int_{-3}^{-2} \frac{x}{1-x^2}\diff x&=&\displaystyle \int\limits_{\tiny \begin{array}{rcl}x&=&-3\\ u&=&-8\end{array}}^{\tiny  \begin{array}{rcl}x&=&-2\\ u&=&-3\end{array}} \frac{1}{u}\left(-\frac{1}{2}\diff u\right)&& \begin{array}{rcl}
u&=&1-x^2\\
\diff u&=&-2x\diff x\\
x\diff x&=&-\frac{1}{2}\diff u
\end{array}\\
&=&\displaystyle -\frac{1}{2}\left[\ln |u|\right]_{-8}^{-3}\\~\\
&=&\displaystyle -\frac{1}{2} \left(\ln|3|-\ln|8| \right)\\~\\
&=&\displaystyle \frac{\ln\left|\frac{8}{3}\right|}{2}
\end{array}
\]

}





\section{First Applications of Integration}
\subsection{Area Between Curves}\label{secMPSareaBetweenCurves}
\begin{problem}
\begin{enumerate}[ref={\fcProblemRef}]
% Area problems
\item 
\label{problemAreaBetweeny=2x^2,y=4+x^2} Find the area of the region bounded by the curves $y = 2x^2$ and $y = 4 + x^2$.

\answer{$\frac{32}{3}$}
\item \label{problemAreaBetweeny=2-x,x=4-y^2} Find the area of the region bounded by the curves $x = 4 - y^2$ and $y = 2 - x$.

\answer{$\frac92$}
\item \label{problemAreaBetweeny=x^2} Find the area of the region bounded by the curves $y=x^2$ and $y=2x^2+x-2$.

\answer{$\frac{9}{2}$}


\item \label{problemareabetweeny=x^2andy=2x^2+x-2}
\begin{itemize}
\item Sketch the region bounded by the curves $y=x^2$ and $y=2x^2+x-2$.

\psset{xunit=0.5cm, yunit=0.5cm}
\begin{pspicture}(-3.4,-3.6)(3,5.7)
\fcAxesStandardNoFrame{-3.5}{-3.5}{2.5}{5.5}
\fcGrid[linestyle=dashed, linewidth=0.5, linecolor=gray]{-3}{-3}{5}{8}{1}{1}{}
\rput[t](0.9,-0.2){$1$}
\fcLabels{3.5}{5.5}
%\psplot{-3}{2}{x x mul}
%\psplot{-3}{2}{x x mul 2 mul x -2 add add}
\end{pspicture}

\item Find the area of the region.

\answer{$\frac{9}{2}$}
\end{itemize}
\item \label{problemAreaBetween-x^2+2x-1and-2x^2+3x+1}
~
\begin{itemize}
\item Sketch the region bounded by the curves $y=- x^{2}+2 x-1$ and $y=-2 x^{2}+3 x+1$. Make sure to indicate the points where the curves intersect.

\psset{xunit=0.5cm, yunit=0.5cm}
\begin{pspicture}(-3.5,-8.8)(3.7,5.7)
\fcAxesStandard{-3.5}{-8.4}{3.5}{5.5}
\fcGrid[linestyle=dashed, linewidth=0.5, linecolor=gray]{-2}{-8}{5}{13}{1}{1}{}
\rput[t](0.9,-0.2){$1$}
\fcLabels{3.5}{5.5}
%\psplot[linecolor=\fcColorGraph]{-1.3}{2.7}{x x -1 mul mul 2 x mul -1 add add}
%\psplot[linecolor=\fcColorGraph]{-1.3}{2.7}{x x -2 mul mul 3 x mul 1 add add}
\end{pspicture}
\item Find the area of the region.
\end{itemize}
\end{enumerate}

\end{problem}
\input{../../modules/area-between-curves/homework/area-between-curves-problems-1-problem-2-solution}

\solution{\ref{problemareabetweeny=x^2andy=2x^2+x-2}

\textbf{Region plot.}

\psset{xunit=0.5cm, yunit=0.5cm}
\begin{pspicture}(-3.5,-3.5)(3,5.7)
\pscustom*[linecolor=\fcColorAreaUnderGraph]{%
\psplot{-2}{1}{x x mul}%
\psplot{1}{-2}{x x mul 2 mul x -2 add add}%
}%
\fcAxesStandardNoFrame{-3.5}{-3.5}{2.5}{5.5}
\psplot{-3}{2}{x x mul}
\psplot{-3}{2}{x x mul 2 mul x -2 add add}
\fcGrid[linestyle=dashed, linewidth=0.5, linecolor=gray]{-3}{-3}{5}{8}{1}{1}{}
\rput[t](0.9,-0.2){$1$}
\fcLabels{3.5}{5.5}
\end{pspicture}

The intersection between the two parabolas are found via
\[
\begin{array}{rcl}
x^2&=&2x^2+x-2\\
x^2+x-2&=&0\\
(x-1)(x+2)&=&0\\
x=1&& x=-2\\
y=1&&y=4.
\end{array}
\]

\textbf{Area of the region.} 
\[
\begin{array}{rcll|l}
A&=&\displaystyle\int_{1}^{-2}\left|x^2-(2x^2+x-2) \right|\diff x&&x^2>(2x^2+x-2) \text{ for }x\in [-2,1] \text{ (from plot)}\\
&=&\displaystyle\int_{1}^{-2}\left(x^2-(2x^2+x-2) \right)\diff x\\
&=&\displaystyle \left[-\frac{1}{3} x^{3}-\frac{1}{2} x^{2}+2 x \right]_{-2}^1\\
&=&\displaystyle \frac{9}{2}.
\end{array}
\]
}




\subsection{Volumes of Solids of Revolution}\label{secMPSvolumesSolidsRevolution}
\subsubsection{Problems Geared towards the Washer Method}\label{secMPSvolumesSolidsRevolutionWashers}
\begin{problem}
\begin{enumerate}[ref={\fcProblemRef}]
% Volume problems
\item 
\label{problemVolumeRegionBoundedByy=2x^2-x+1,y=x^2+1rotatedAroundx=0} Consider the region bounded by the curves $y = 2x^2-x+1$ and $y =x^2+1$. What is the volume of the solid obtained by rotating this region about the line $x = 0$?

\answer{$\frac{2}{5}\pi$.} 
\item Consider the region bounded by the curves $y = 1-x^2$ and $y =0$. What is the volume of the solid obtained by
rotating this region about the line $y = 0$?

\answer{$\frac{16 \pi}{15}$}
 
\item Consider the region bounded by the curves $y = x^2$ and $x = y^2$. What is the volume of the solid obtained by
rotating this region about the line $x = 2$?

\answer{ $\frac{31 \pi}{30}$}
\item \label{problemVolumeAreay=-x^2+2andy=0rotatedAroundy=0andy=-3}
Set up \textsc{but do not evaluate} an integral to calculate the volume of the solid obtained by rotating the region bounded by $y=-x^2+2$ and $y=0$ about the given line. 

\begin{itemize}
\item The $x$ axis.
\item The line $y=-3$.
\end{itemize}

\item \label{problemVolumeRevolution-x^2+1aroundy=0andy=-4}
Set up \textsc{but do not evaluate} an integral to calculate the volume of the solid obtained by rotating the region bounded by $y=-x^2+1$ and $y=0$ about the given line. 
\begin{itemize}
\item The $x$ axis.
\item The line $y=-4$.
\end{itemize}


\end{enumerate}

\end{problem}
\solution{\ref{problemVolumeRegionBoundedByy=2x^2-x+1,y=x^2+1rotatedAroundx=0}
First, plot $y=2x^2-x+1$ and $y=x^2+1$. 

\psset{xunit=2cm, yunit=2cm}
\begin{pspicture}(-0.5,-0.5)(1.3,3)
\tiny
\fcAxesStandard{-1}{-0.4}{1.3}{3}
\pscustom*[linecolor=\fcColorAreaUnderGraph]{
\psplot[linecolor=\fcColorGraph]{0}{1}{x x 2 mul mul x sub 1 add}
\psplot[linecolor=\fcColorGraph]{1}{0}{x x mul 1 add}
}
\psplot[linecolor=\fcColorGraph]{-0.5}{1.2}{x x 2 mul mul x sub 1 add}
\psplot[linecolor=\fcColorGraph]{-0.5}{1.2}{x x mul 1 add}
\psline[](! 0.7 11 8 div)(! 0.8 11 8 div)
\psline[](! 0.7 25 16 div)(! 0.8 25 16 div)
\psline[](0.75, 0)(! 0.75 25 16 div)
\rput[br](! 0.75 25 16 div){$r_{outer}~$}
\rput[tl](! 0.8 11 8 div 0.05 sub){$~r_{inner}$}
\psline[linewidth=2pt, linecolor=blue](0, 0)(1,0)
\psline[linestyle=dashed](1,2)(1,0)
\fcFullDot[linecolor=blue]{0}{0}
\fcFullDot[linecolor=blue]{1}{0}
\end{pspicture}
\psset{xunit=2cm, yunit=2cm}
\begin{pspicture}(-1,-2.3)(3,3)
\renewcommand{\fcScreenStyle}{x}
\renewcommand{\fcScreen}{[-0.2 -0.2 -1] 0}
\fcStartIIIdScene
\fcAxesIIIdInScene{2.2}{2.2}{2.2}
\fcSurfaceInScene[iterationsV=7, iterationsU=4, linewidth=0.3, arrows=(none)]{0}{0}{1}{360}{[u 2 u u mul mul u sub 1 add v cos mul 2 u u mul mul u sub 1 add v sin mul]}{}
\fcSurfaceInScene[iterationsV=7, iterationsU=4, arrows=(none), linewidth=0.3]{0}{0}{1}{360}{[u u u mul 1 add v cos mul u u mul 1 add v sin mul]}{}
\fcSurfaceInScene[iterationsV=1, iterationsU=3, colorUV=cyan, colorVU=cyan, forceForeground=true]{ 0 }{0}{1}{1}{[u u u 2 mul mul u sub 1 add v mul u u mul 1 add 1 v sub mul add 0]}{}
\fcLineIIIdInScene[linewidth=2, linecolor=blue]{[0 0 0]}{[1 0 0]}
\fcFinishIIIdScene[true]
\fcDotIIId[linecolor=blue]{[0 0 0]}
\fcDotIIId[linecolor=blue]{[1 0 0]}
\end{pspicture}
\psset{xunit=2cm, yunit=2cm}
\begin{pspicture}(-1,-2)(3,3)
\renewcommand{\fcScreenStyle}{x}
\renewcommand{\fcScreen}{[-0.2 -0.2 -1] 0}
\fcStartIIIdScene
\fcAxesIIIdInScene{2.2}{2.2}{2.2}
\fcSurfaceInScene[arrows=(none), iterationsV=1, iterationsU=4, colorUV=cyan, colorVU=cyan, linewidth=0.3, forceForeground=true]{ 0 }{0}{1}{1}{[u u u 2 mul mul u sub 1 add v mul u u mul 1 add 1 v sub mul add 0]}{}
\fcSurfaceInScene[arrows=(none), iterationsV=20, iterationsU=1, colorUV=blue, colorVU=blue, linewidth=0.3]{0.75 0.75 mul 2 mul 0.75 sub 1 add}{0}{0.75 0.75 mul 1 add}{360}{[0.75 v cos u mul v sin u mul]}{}
\fcFinishIIIdScene[true]
\fcLineIIId[]{[0.75  0 0]}{[0.75 11 8 div 0]}
\fcPutIIId[br]{[0.75 25 16 div 0]}{$r_{outer}~$}
\fcPutIIId[tl]{[0.75 11 8 div 0.1 sub 0]}{$~r_{inner}$}
\fcLineIIId{[0 0 0]}{[1 0 0]}
\fcDotIIId[linecolor=blue]{[0 0 0]}
\fcDotIIId[linecolor=blue]{[1 0 0]}

\end{pspicture}

\noindent The two curves intersect when 
\[ 
\begin{array}{rcl}
2x^2-x+1&=&x^2+1\\
x^2-x&=&0\\
x(x-1)&=&0\\
x=0 &\text{or}& x= 1.
\end{array}
\]
Therefore the two points of intersection have $x$-coordinates between $x=0$ and $x=1$. Therefore we need we need to integrate the volumes of washers with inner radii $r_{inner}=2x^2-x+1 $, outer radii $r_{outer}=x^2+1$ and infinitesimal heights $\diff x$. The volume of an individual infinitesimal washer is then $ \pi(r^2_{outer}- r^2_{inner})\diff x$
\[
\begin{array}{rcl}
V&=&\displaystyle \int_{0}^1\pi\left(\left(x^2+1\right)^2- \left(2x^2-x+1\right)^2\right)\diff x\\
&=&\displaystyle \pi\int_{0}^1\left(-3 x^{4}+4 x^{3}-3 x^{2}+2 x \right)\diff x\\
&=&\displaystyle \pi\left[-\frac{3}{5} x^{5}+x^{4}- x^{3}+x^{2} \right]_0^1 \\
&=& \displaystyle \frac{2}{5} \pi.
\end{array}
\]
}
\solution{\ref{problemVolumeAreay=-x^2+2andy=0rotatedAroundy=0andy=-3}
First, we plot the 2d region. The two curves intersect when $-x^2+2=0$, i.e., when $x=\pm \sqrt{2}$


\hfil \hfil \begin{pspicture}(-6.2,-3.2)(6.2,3.2)\tiny
\pscustom*[linecolor=\fcColorAreaUnderGraph]{
\psplot[linecolor=\fcColorGraph]{2 sqrt -1 mul}{2 sqrt}{x x mul -1 mul 2 add}
}
\newcommand{\theFuN}{x x mul -1 mul 2 add\space}%
\psplot[linecolor=\fcColorGraph]{-2}{2}{x x mul -1 mul 2 add}
\psline[linecolor=\fcColorGraph](-2,0)(2,0)
\fcAxesStandardNoFrame{-2}{-3.1}{2}{3}
\rput[t](! 2 sqrt -0.1){$\sqrt{2}$}
\rput[t](! 2 sqrt -1 mul -0.1){$-\sqrt{2}$}
\psline[arrows=<->](1, 0)(! 1 1 dict begin /x 1 def \theFuN end)
\rput[l](1.1,0.3){cross-section rad., $y=0$}
\psline[arrows=<->](-1, -3)(! -1 1 dict begin /x -1 def \theFuN end)
\rput[r](-1.1,-1){cross-section rad., $y=-3$}
\psline[linecolor=green](-2,-3)(2,-3)
\end{pspicture}

\textbf{Rotation about $y=0$. }

Unless explicitly stated in the problem, a 3d plot of the solid is not required in the solution. Nevertheless generating such a plot helps to understand the problem. 

To generate a 3d plot of the solid, we draw the circular cross-sections of the solid of revolution. By hand, this can be done roughly by drawing ovals (circles look like ovals when observed at an angle) centered at the axis about which we revolve the 2d-region. We include a computer-generated plot below; the plot's precision is above what is expected on an exam.

\hfil \hfil \begin{pspicture}(-3,-3)(4.2,3.2)%
\newcommand{\theFun}{u u mul -1 mul 2 add\space}%
\renewcommand{\fcScreenStyle}{x}
\renewcommand{\fcScreen}{[-1 -0.2 -0.75] -1}
\fcStartIIIdScene%
\fcAxesIIIdFullInScene{-3}{-3}{-3}{3}{3}{3}%
\fcSurfaceInScene[arrows=(none), iterationsV=15, iterationsU=8, colorVU={1 0.5 0.5}]{2 sqrt -1 mul 0.001 add}{0}{2 sqrt -0.001 add}{360}{[u v cos \theFun mul v sin \theFun mul]}{}%
\fcSurfaceInScene[arrows=(none), iterationsV=4, iterationsU=3, colorUV={0.3 0.7 1}, forceForeground=true]{2 sqrt -1 mul }{0}{2 sqrt}{1}{[u \theFun v mul  0]}{}%
\fcFinishIIIdScene[true]%
\fcPutIIId{[3 0 0]}{$x$}
\fcPutIIId{[0 3 0]}{$y$}
\fcPutIIId{[0 0 3]}{$z$}
\end{pspicture}

The volume of a solid (and in particular, of a solid of revolution) is computed by integrating the area $A(x)=\pi(\text{radius cross-section})= \pi (-x^2+2)^2 $ of the cross-section of the solid. Therefore the volume $V$ equals
\[
\begin{array}{rcll|l}
V&=&\displaystyle \int_{a}^bA(x)\diff x\\
&=&\displaystyle\int_{-\sqrt{2}}^{\sqrt{2}}\pi (-x^2+2)^2  \diff x\\
&=&\displaystyle\pi\left[\frac{1}{5} x^{5}-\frac{4}{3} x^{3}+4 x\right]_{-\sqrt{2}}^{\sqrt 2}&&\text{step not required by problem}\\
&=&\displaystyle \pi \frac{64}{15}\sqrt{2}&&\text{step not required by problem.}
\end{array}
\]


\textbf{Rotation about $y=-3$. } The cross-section of this solid of revolution is a washer with inner radius $ 3$ and outer radius $-x^2+2-(-3)=5-x^2$. Therefore the area of the cross-section is $\pi (5-x^2)^2-\pi 3^2$ and the volume is computed via

\[
\begin{array}{rcll|l}
V&=&\displaystyle \int_{a}^bA(x)\diff x\\
&=&\displaystyle\int_{-\sqrt{2}}^{\sqrt{2}} \pi \left( (5-x^2)^2- 3^2\right)  \diff x\\
&=&\displaystyle\pi\left[\frac{1}{5} x^{5}-\frac{10}{3} x^{3}+16 x  \right]_{-\sqrt{2}}^{\sqrt 2}&&\text{step not required by problem}\\
&=&\displaystyle \pi  \frac{304}{15}\sqrt{2}&&\text{step not required by problem.}
\end{array}
\]
\hfil \hfil \begin{pspicture}(-4,-9)(5,4.2)%
\newcommand{\theFuN}{u u mul -1 mul 2 add\space}%
\renewcommand{\fcScreenStyle}{x}
\fcStartIIIdScene%
\fcAxesIIIdFullInScene{-3}{-9}{-4}{3}{3}{4}%
\renewcommand{\fcScreen}{[-1 -0.2 -0.75] -1}
\fcSurfaceInScene[arrows=(none), iterationsV=15, iterationsU=8, colorVU={0.7 0.2 0.2}, colorUV={0.7 0.2 0.2}]{2 sqrt -1 mul }{0}{2 sqrt -0.001 add}{360}{[u v cos 3 mul -3 add v sin 3 mul]}{}%
\fcSurfaceInScene[arrows=(none), iterationsV=15, iterationsU=8, linecolor=black,colorUV={1 0.5 0.5}, colorVU={1 0.5 0.5}]{2 sqrt -1 mul 0.01 add}{0}{2 sqrt -0.01 add}{360}{[u v cos \theFuN 3 add mul -3 add v sin \theFuN 3 add mul]}{}%
\fcSurfaceInScene[arrows=(none), iterationsV=1, iterationsU=8, colorUV={0.3 0.7 1}, forceForeground=true]{2 sqrt -1 mul }{0}{2 sqrt}{1}{[u \theFuN v mul  0]}{}%
\fcLineIIIdInScene[linecolor=green, linewidth=2]{[-6 -3 0]}{[6 -3 0]}
\fcCurveIIIdInScene[linecolor=red, arrows=(none), linewidth=2]{2 sqrt -1 mul}{2 sqrt}{[1 dict begin /u t def u \theFuN 0 end]}
\fcFinishIIIdScene[true]%
\fcPutIIId{[3 0 0]}{$x$}
\fcPutIIId{[0 3 0]}{$y$}
\fcPutIIId{[0 0 3]}{$z$}
\end{pspicture}
}




\subsubsection{Problems Geared towards the Cylindrical Shells Method}\label{secMPSvolumesSolidsRevolutionShells}
\begin{problem}
\begin{enumerate}
% Volumes by cylindrical shells
\item Consider the region bounded by the curves $y=\sqrt{x}$, $x=0$, $y=2$. Use the method of cyllindrical shells to find the volume of
the solid obtained by rotating this region about the $x$-axis. 
\answer{$8\pi$}
\item Consider the region bounded by the vurves $y=x^2$ and $y=2-x^2$. Use the method of cylindrical shells to find the volume of the solid obtained by rotating this region about the line $x=1$.
\answer{$16\frac{\pi}{3}$}
\end{enumerate}

\end{problem}
\end{document}
