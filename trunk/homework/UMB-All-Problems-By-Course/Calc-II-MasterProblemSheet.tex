\documentclass{article}
\ProvidesPackage{homework-problems-UMB}
\addtolength{\hoffset}{-3.5cm}
\addtolength{\textwidth}{6.8cm}
\addtolength{\voffset}{-3cm}
\addtolength{\textheight}{6cm}
\usepackage{../homework-problems} %warning folder paths are relative to the file that uses the includepackage

\renewcommand{\answer}[1]{\iftoggle{answers}{ \hfill{~} \rotatebox{180}{\tiny answer: #1}}{} }
\renewcommand{\pointsii}[1]{}
\renewcommand{\hiddenanswer}{\answer}
\renewcommand{\points}[1]{\item}
\renewcommand{\pointsii}[1]{\item}
\renewcommand{\Arctan}{\arctan}
\renewcommand{\Arcsin}{\arcsin}
\renewcommand{\Arccot}{\operatorname{arccot}}

\toggletrue{solutions}
%\togglefalse{solutions}
\toggletrue{answers}
\newtheorem{problem}{Problem}

\newcommand{\hide}[1]{}

\renewcommand{\fcProblemRef}{\theproblem.\theenumi}
\renewcommand{\fcSubProblemRef}{\theenumi.\theenumii}
\begin{document}
\begin{center}
\Large
Master Problem Sheet \\
Version \today
 \\ Math 141 Calculus II \\ \normalsize Instructor: Todor Milev

\end{center}

%\noindent \textbf{Name:} \hfill{~}
%\begin{tabular}{c|c|c|c|c|c|c|c|c||c}
%Problem&1 &2&3&4&5&6&7&8& $\sum$\\ \hline
%Score&&&&&&&&&\\ \hline
%Max&20&20&20&20&20&10&20&20&150
%\end{tabular}

This master problem sheet contains all freecalc problems on the topics studied in Calculus II. The latest \LaTeX{} source of this file (complete with typo and error fixes) can be downloaded from the freecalc project page below. 

\url{https://sourceforge.net/p/freecalculus/code/HEAD/tree/}

For a list of contributors/authors of the freecalc project (and in particular, the present problem collection) see the following file.
\url{https://sourceforge.net/p/freecalculus/code/HEAD/tree/trunk/contributors.tex}

\tableofcontents

\section{Problem selections that have appeared as UMB exams}

Below are problems collections that have appeared on final UMass Boston Calc II exams. The problems were given in the same ordered as the order below. 

\textbf{Fall 2014 Exam. Time work: 3 hours. }
\begin{enumerate}
\item Problem \ref{problemint(3x^2+2x-1)/((x-1)(x^2+1))dx}.
\item Problem \ref{problemint(sqrt(9-x^2)/(x^2)dx)}.
\item Problem \ref{problemint(e^(-sqrtx)dx)}.
\item Problem \ref{problemConvergencesum_2^infty1/(xlnx)dx} (the problem was formulated slightly differently - as an improper integral).
\item Problem \ref{problemlimxtoinftysin(2/x)}.
\item Problem \ref{problemlimxtoinfty((x-2)/x)^x}.
\item Problem \ref{problemsumn=0^infty(2^n+5^n)/10^n}.
\item Problem \ref{problemsumn=2^inftyln(1-1/n^2)}.
\item Problem \ref{problemz^3=i}.
\item Problem \ref{problemConvergencesumn=1^infty(-1)^nlnn}.
\item Problem \ref{problemConvergencesumn=2^infty(-1)^n/lnn}.
\item Problem \ref{problemTaylorlnxarounda=2}.
\item Problem \ref{problemIntervalConvergence_sum(x-2)^n/(3sqrt(n+1))}.
\item Problem \ref{problemlengthx=sqrt(t)-2t,y=8/3t^(3/4)}.
\item Problem \ref{problem-Area-swept-byr=sin2theta_outsider=1/2}.
\item Problem \ref{problemy'=y^2(1+x),y(0)=3}. 
\end{enumerate}
\textbf{Spring 2015 Exam. Time work: 3 hours.} The material was slightly reduced due to reduced semester (bad weather).



\section{Derivative non-constant exponent }
\begin{problem}
Differentiate
\begin{enumerate}
\item $x^{\sin x}$.
\solution{
$\displaystyle \left(x^{\sin x}\right)'=\left(e^{(\ln x)\sin x}\right)'= e^{(\ln x)\sin x} (\ln x\sin x)'=x^{\sin x}\left( \frac{\sin x}{x}+\ln x\cos x\right) $
}
\answer{$x^{\sin x}\left(\frac{\sin x}{x} +\cos x\ln x\right) $}
\item $x^{\tan x}$.
\answer{$x^{\tan x}\left(\frac{\tan x}{x}- (\ln x)\sec^2 x \right) $}
\end{enumerate}
\end{problem}
\solution{\ref{problemDifferentiatex^sinx}.
$\displaystyle \left(x^{\sin x}\right)'=\left(e^{(\ln x)\sin x}\right)'= e^{(\ln x)\sin x} (\ln x\sin x)'=x^{\sin x}\left( \frac{\sin x}{x}+\ln x\cos x\right) $.
}


\begin{problem}
Differentiate.
\begin{multicols}{2}
\begin{enumerate}
\item $10^{x^3}$. \answer{$3(\ln 10) x^{2} (10)^{x^{3}}$}
\item $2^{\tan x}$. \answer{ $(\ln 2) 2^{\tan x}  \sec^2 x $  }
\item $x^x $. \answer{$x^x(\log{}(x) +1)$}
\item $x^{x^x}$. \answer{$(\ln(x))^{2}  x^{x^{x}+x}+x^{x^{x}+x-1}+(\ln x) x^{x^{x}+x}$}
\item $(\sin x)^{\cos x}$. \answer{$\frac{- \ln(\sin{}x)  (\sin{}x)^{\cos{}x+2} +(\sin{}x)^{\cos{}x} \cos^{2}{}x}{\sin{}x}$}
\item $(\ln x)^{\ln x}$. \answer{$\ln{}(\ln{}(x)) x^{-1} (\ln{}(x))^{\ln{}(x)}+x^{-1} (\ln{}(x))^{\ln{}(x)}$}
\end{enumerate}
\end{multicols}
\end{problem}
\section{The number $e$ as a limit}
\begin{problem}
Compute the limit. 
\begin{enumerate}
\item $\displaystyle\lim\limits_{x\to\infty} \left(\frac{x-2 }{x} \right)^{2 x}$ 

\answer{$e^{-4}$.}
\item $\displaystyle \lim\limits_{x\to\infty} \left(\frac{x}{ x + 3} \right)^{2x}$ 

\answer{$e^{-6}$}
\end{enumerate}
\end{problem}


\solution{\ref{problemlimxtoinfty((x-2)/x)^x}.

\noindent\textbf{Variant I.}
\[
\begin{array}{rcll|l}
\displaystyle \lim_{x\to \infty }\left( \frac{x-2}{x}\right)^x&=& \displaystyle \lim\limits_{ x \to \infty }\left( 1- \frac{2}{x}\right)^x &&\text{use } \lim\limits_{x\to\infty} \left( 1+ \frac{k}{x}\right)^x=e^k\\
&=&e^{-2}\quad .
\end{array}
\]

\noindent\textbf{Variant II.}
\[
\begin{array}{rcll|l}
\displaystyle \lim_{x\to \infty }\left( \frac{x-2}{x}\right)^x&=& \displaystyle \lim_{ x \to \infty } e^{\ln \left( \left( \frac{x-2}{x}\right)^x\right)} \\
\displaystyle \lim_{x\to \infty} \ln \left( \left( \frac{x-2}{x}\right)^x\right)&=&\displaystyle  \lim_{x\to \infty} x\left( \ln (x-2) -\ln (x) \right)\\
&=&\displaystyle \lim_{x\to\infty } \frac{ \ln (x-2) -\ln (x)}{\frac{1}{x}} &&\text{L'Hospital rule}\\
&=&\displaystyle \lim_{x\to\infty } \frac{ \frac{1}{ x-2} -\frac{1}{x}}{- \frac{ 1}{ x^2 }}\\
&=&\displaystyle \lim_{x\to \infty }\frac{-2x^2}{x^2-2x}=-2 &&\text{Therefore}\\
\displaystyle \lim_{x\to \infty }\left( \frac{x-2}{x}\right)^x&=& \displaystyle \lim_{ x \to \infty } e^{\ln \left( \left( \frac{x-2}{x}\right)^x\right)} \\
&=&e^{\lim\limits_{x\to \infty} \ln \left( \left( \frac{x-2}{x}\right)^x\right) }\\
&=&e^{-2}\quad .
\end{array}
\]
}
\begin{problem}
Find the limit.
\begin{multicols}{2}
\begin{enumerate}
\item $\displaystyle \lim\limits_{x\to \infty} \left(1-\frac{2}{x} \right)^x$. \answer{$e^{-2}$}
\item $\displaystyle \lim\limits_{x\to 0} \left(1-x\right)^{\frac{1}{x}}$.
\answer{ $e^{-1}$}
\item $\displaystyle \lim\limits_{x\to \infty} \left(\frac{x}{x-5}\right)^{x}$.
\answer{$e^5$}
\item $\displaystyle \lim\limits_{x\to \infty} \left(\frac{x}{x-2}\right)^{3x+2}$.
\answer{$e^6$}
\end{enumerate}
\end{multicols}
\end{problem}
\begin{problem}
\begin{multicols}{2}
\begin{enumerate}
\item A sum is held under a yearly compound interest of 1\%. Make an approximation by hand (no calculators allowed) by what factor will have the money increased after 200 years. Can you do the computation in your head?
\item Decide, without using a calculator, which is more profitable: earning a yearly compound interest of 2\% for 150 years or earning yearly simple interest of 11\% for 150 years?
\end{enumerate}
\end{multicols}
\end{problem}
\solution{\ref{problemEasLimitAndCompoundInterest1procentInterest200years}
Each year, the sum increases by a factor of $\left(1+\frac{1}{100}\right)$. Therefore in $200$ years the sum will have increased by 
\[
\begin{array}{rcll|l}
\left( 1+\frac{1}{100}\right)^{200} &=&\left(\left(1+\frac{1}{100}\right)^{100}\right)^{2}&&\text{equals } \left(\left(1+\frac{1}{n}\right)^n\right)^2 \text{ for } n=100\\
&\approx& e^2.
\end{array}
\]
As a rough estimate for $e$ we can take $e\approx 2.7$, and so $e^2\approx 2.7^2=7.29$. Our sum will have increased approximately $7.3$ times. A calculator computation shows that 
\[
\left( 1+\frac{1}{100}\right)^{200}\approx 7.316018 ,
\] 
so our ``in the head'' estimate is fairly accurate. Notice that the calculator computation is on its own an approximation - it was carried using double floating point precision arithmetics, which does introduce some minimal errors. Such round off errors, of course, are also present in modern banking transactions, so we do not need to adjust for those.
}

\solution{\ref{problemEasLimitAndCompoundInterestWhatIsMorecompoundInterest2percent150yearsOrsimpleInterest11percent}
Simple interest of $11\%$ per $150$ years a profit of 
\[
0.11*150= 15+1.5=16.5,
\]
or altogether $17.5$-fold increase of our initial sum. A $2\%$ compound interest for $150$ years yields a 
\[
\begin{array}{rcl}
\left(1+\frac{2}{100}\right)^{150}&=&\left(\left(1+\frac{1}{50}\right)^{50}\right)^3\\
&\approx& e^3
\end{array}
\]
-fold increase of our sum. To establish which of the two options yields more money, we need to compare $e^{3}$ to $17.5$ (without using a calculator). In the solution of \ref{problemEasLimitAndCompoundInterest1procentInterest200years} we established that $e^2\approx 7.3$, so $e^3\approx e\cdot 7.3\approx 2.7\cdot 7.3=2\cdot 7 +2\cdot 0.3 +0.7\cdot7+0.7\cdot 0.3=14+0.6+4.9+0.21=19.71\approx 19.7$. We can say that the compound interest results in approximately $19.7$-fold increase of the initial sum, so the compound interest is more profitable. A calculator computation shows that 
\[
\left(1+\frac{2}{100}\right)^{150}\approx 19.499603\quad .
\]
Our error of approximately $0.2$ was not optimal, yet fairly accurate for an ``in the head'' computation.



}

\begin{problem}
$1,000,000$ servers are handling Internet users. Suppose we distribute the computation load as follows. The computation load distributing program directs every new user to a server chosen at random (one server is allowed to process more than one user at a time). Suppose one server has defective hardware and crashes. We are testing the system by simulating $X$ Internet users.
\begin{itemize}
\item What is the chance we catch the defective server using $1$ simulated user?
\item Without using a calculator, estimate the chance we fail to catch the defective server using $1,000,000$ simulated users.
\item Using a calculator, estimate the chance we fail to catch the defective server using $100,000$ simulated users. Write an expression using $e$ which approximates this chance. Evaluate the latter with a calculator. Are the two numbers close? 
\end{itemize}

\textbf{Remark.} While such a simple system architecture would not be practical, it is not to be immediately dismissed as terrible. For example, if we need to handle 2 million users per second, our load distributing mechanism might not be fast enough to keep track of each server's load. On the other hand, an inexpensive modern pc will easily generate 2 million random numbers per second.

\end{problem}
\section{Inverse trigonometry}
\begin{problem}
Let $x\in (0,1)$. Express the following using $x$ and $\sqrt{1-x^2}$.  
\begin{multicols}{2}
\begin{enumerate} [ref={\fcProblemRef}]
\item $\sin(\Arcsin (x))$. 

\answer{$x$}

\pointsii{3} \label{problemsin(2arcsin x)}  $\sin(2\Arcsin (x))$. 

\answer{$2x\sqrt{1-x^2} $}

\item \label{problemsin(3arcsin x)} $\sin(3\Arcsin (x))$. 

\answer{$ -4x^3+3x $}
\item $\sin(\Arccos (x))$. 

\answer{$\sqrt{1-x^2} $}
\item $\sin(2\Arccos (x))$. 

\answer{$2x \sqrt{1-x^2 }$}
\item \label{problemsin(3arccos(x))}  $\sin(3\Arccos (x))$. 

\answer{
\begin{tabular}{l}
$\left(4x^2-1\right)\sqrt{1-x^2}$ \\= $-4\left(\sqrt{1-x^2}\right)^3+3\sqrt{1-x^2} $
\end{tabular}
}

\item $\cos(2\Arcsin (x))$. 

\answer{$ 1-2x^2$}

\item $\cos(3\Arccos (x))$. 

\answer{$4x^3-3x $}

\end{enumerate}
\end{multicols}
\end{problem}
\solution{\ref{problemsin(2arcsin x)}.
Let $y = \Arcsin x$.  Then $\sin x = y$, and we can draw a right triangle with opposite side length $x$ and hypotenuse length $1$ to find the other trigonometric ratios of $y$.  

\begin{center}
\psset{xunit=1.0cm,yunit=1.0cm,algebraic=true,dotstyle=o,dotsize=3pt 0,linewidth=0.8pt,arrowsize=3pt 2,arrowinset=0.25}
\begin{pspicture*}(-3.33,-6.11)(14.05,6.58)
\psline(0,0)(4,0)
\psline(0,0)(4,3)
\psline(4,3)(4,0)
\psline(4,0.2)(3.8,0.2)
\psline(3.8,0.2)(3.8,0)
\rput[tl](0.83,0.5){$y$}
\rput[tl](1.56,1.82){$1$}
\rput[tl](4.1,1.4){$x$}
\rput[tl](1.7,-0.05){$\sqrt{1-x^2}$}
\parametricplot{0.0}{0.6435011087932844}{1*0.66*cos(t)+0*0.66*sin(t)+0|0*0.66*cos(t)+1*0.66*sin(t)+0}
\end{pspicture*}
\end{center}

Then $\cos y = \sqrt{1-x^2}/1 = \sqrt{1-x^2}$.  
Now we use the double angle formula to find $\sin(2\Arcsin x)$.  

\begin{align*}
\sin (2 \Arcsin x) & = \sin (2y) \\
& = 2\sin y\cos y \\
& = 2x\sqrt{1-x^2}.
\end{align*}
}

\solution{\ref{problemsin(3arcsin x)}. Use the result of problem \ref{problemsin(2arcsin x)}. This also requires the addition formula for sine: 
\[
\sin(A+B) = \sin A \cos B + \sin B\cos A,
\]
and the double angle formula for cosine:
\[
\cos (2y) = \cos^2 y - \sin^2 y.
\]  
\begin{align*}
\sin(3\Arcsin x) & = \sin(3y) \\
& = \sin (2y + y) \\
\text{Addition formula: } \quad & = \sin(2y)\cos y + \sin y \cos (2y) \\
\text{Double angle formulas: } \quad & = (2\sin y \cos y)\cos y + \sin y (\cos^2 y - \sin^2 y) \\
& = 2\sin y \cos^2 y + \sin y \cos^2 y - \sin^3 y \\
& = 3\sin y \cos^2 y - \sin^3 y \\
& = 3x(1-x^2) - x^3 \\
& = 3x - 4x^3.
\end{align*}
}

\begin{problem}
Rewrite as algebraic expression of $x$ which uses $\sqrt{~}$ radicals (i.e., get rid of the trigonometric and inverse trigonometric expressions). The answer key has not been fully proofread, use with caution.

\begin{multicols}{2}
\begin{enumerate}
\item $\cos^2(\Arctan x)$. \answer{$\frac{1}{1+x^2} $}
\item $-\sin^2(\Arccot x)$. \answer{ $-\frac{1}{1+x^2}$}
\item $\frac{1}{\cos(\Arcsin x)}$. \answer{$\frac{1}{\sqrt{1-x^2}}$}
\item $-\frac{1}{\sin(\Arccos x)}$.\answer{$-\frac{1}{\sqrt{1-x^2}}$}
\end{enumerate}
\end{multicols}

\end{problem}
\solution{\ref{problem-sin^2(arccot x) }. We follow the strategy outlined in the end of the solution of Problem \ref{problemsin(3arcsin x)}. We set $y=\Arccot x$. Then we need to express $-\sin^2 y$ via $\cot y$. That is a matter of algebra:

\[\begin{array}{rcll|l}
-\sin^2(\Arccot x)&=&\displaystyle -\sin^2 y&&\text{Set } y=\Arccot x\\
&=&\displaystyle-\frac{\sin^2 y}{\sin^2y+\cos^2 y} && \text{use } \sin^2y+\cos^2y=1\\
&=&\displaystyle-\frac{1}{\frac{\sin^2 y+\cos^2y}{\sin^2 y}}\\
&=&\displaystyle-\frac{1}{1+\cot^2 y}&& \text{Substitute back } \cot y=x\\
&=&\displaystyle-\frac{1}{1+x^2}\quad.
\end{array}
\]

}

\begin{problem}
Rewrite as a rational function of $t$.

\begin{enumerate}
\item $\cot (2\arctan t)$.
\answer{$ \frac{1}{2}\left(\frac{1}{t}-t\right)$.}
\item $\cos (2\arctan t)$.
\answer{$ \frac{1-t^2}{1+t^2}$.}
\item $\sec (2\arctan t)$.
\answer{$\frac{1+t^2}{1-t^2}$}
\end{enumerate}
\end{problem}
\solution{\ref{problemcos(2arctan t)}
Set $z=\Arctan t$, and so $\tan z= t$. Then
\[
\begin{array}{rcll|l}
\cot (2\Arctan t)&=&\displaystyle \cot (2z) \\
&=&\displaystyle \frac{\cos (2z)}{\sin (2z)}&&\text{use double angle formulas} \\
&=&\displaystyle \frac{\cos^2 z- \sin^2 z}{2\sin z \cos z}\\
&=&\displaystyle \frac{1- \tan^2 z}{2 \tan z}\\
&=&\displaystyle \frac{1-t^2}{2t}\\
&=&\displaystyle \frac{1}{2}\left(\frac{1}{t}-t\right)\quad .
\end{array}
\]
}
\solution{\ref{problemcot(2arctan t)}
Set $z=\Arctan t$, and so $\tan z= t$. Then
\[
\begin{array}{rcll|l}
\cot (2\Arctan t)&=&\displaystyle \cot (2z) \\
&=&\displaystyle \frac{\cos (2z)}{\sin (2z)}&&\text{use double angle formulas} \\
&=&\displaystyle \frac{\cos^2 z- \sin^2 z}{2\sin z \cos z}\\
&=&\displaystyle \frac{1- \tan^2 z}{2 \tan z}\\
&=&\displaystyle \frac{1-t^2}{2t}\\
&=&\displaystyle \frac{1}{2}\left(\frac{1}{t}-t\right)\quad .
\end{array}
\]
}

\begin{problem}
Compute the derivative (derive the formula).

\begin{multicols}{2}
\begin{enumerate}
\item $(\Arctan x) '$.
\item $(\Arccot x)'$.
\item $(\Arcsin x)'$.
\item $(\Arccos x)'$.
\item Let $\text{arcsec}$ denote the inverse of the secant function. Compute $(\text{arcsec} x)'$.
\end{enumerate}
\end{multicols}
\end{problem}

\begin{problem}
\begin{enumerate}[ref={\fcProblemRef}]
\item \label{problemTangentAngleSumLaw}  Let $a+b\neq k\pi $, $a\neq k\pi+\frac{\pi}{2}$ and $b\neq k\pi +\frac{\pi}{2}$ for any $k\in \mathbb Z$ (integers). Prove that
\[
\frac{\tan a + \tan b}{1-\tan a \tan b }= \tan (a+b)\quad .
\]
\item \label{problemArctangentAngleSumLaw} Let $x$ and $y$ be real. Prove that, for $xy\neq 1$, we have
\[\Arctan x+\Arctan y= \Arctan\left( \frac{x+y}{1-xy}\right)
\]
if the left hand side lies between $\left(-\frac{\pi}{2}, \frac{\pi}{2}\right)$.
\end{enumerate}

\end{problem}
\solution{\ref{problemTangentAngleSumLaw} We start by recalling the formulas \[\begin{array}{rcl}\cos(a+b)&=&\cos a \cos b-\sin a \sin b\\ \sin (a+b)&=&\sin a \cos b+\sin b \cos a\quad .
\end {array}
\] 
These formulas have been previously studied; alternatively they follow from Euler's formula and the computation 
\[
\begin{array}{rcl}
\cos (a+b) +i\sin (a+b)&=& e^{i(a+b)}= e^{ia}e^{ib}=(\cos a+ i\sin a)(\cos b +i \sin b)\\
&=&\cos a \cos b-\sin a \sin b +i(\sin a \cos b+\sin b \cos a)\quad .
\end{array}
\]
Now \ref{problemTangentAngleSumLaw} is done via a straightforward computation:
\begin{equation}\label{eqTangentAngleSumLaw}
\begin{array}{rcl}
\tan(a+b)&=&\displaystyle \frac{\sin (a+b)}{\cos (a+b)}=\frac{\sin a \cos b+\sin b \cos a}{\cos a\cos b-\sin a\sin b}=\frac{(\sin a\cos b+\sin b \cos a)\frac{1}{\cos a\cos b}}{(\cos a \cos b-\sin a \sin b)\frac{1}{\cos a\cos b}}\\
&=&\displaystyle \frac{\tan a +\tan b}{1-\tan a \tan b}\quad .
\end{array}
\end{equation}
\noindent \ref{problemArctangentAngleSumLaw} is a consequence of \ref{problemTangentAngleSumLaw}. Let $a=\Arctan x$, $b=\Arctan y$. Then \eqref{eqTangentAngleSumLaw} becomes 
\[
\tan (\Arctan x+\Arctan y)= \frac{\tan (\Arctan x)+\tan(\Arctan y)}{1-\tan (\Arctan x)\tan (\Arctan y)}=\frac{x+y}{1-xy}\quad,
\]
where we use the fact that $\tan (\Arctan w)=w$ for all $w$. We recall that $\Arctan (\tan z)=z$ whenever $z\in \left(-\frac{\pi}{2}, \frac{\pi}{2}\right)$. Now take $\Arctan$ on both sides of the above equality to obtain 
\[
\Arctan x +\Arctan y=\Arctan\left(\frac{x+y}{1-xy}\right)\quad .
\]

}


\section{Integration by parts}
\begin{problem}
Evaluate the indefinite integral. Illustrate the steps of your solutions.
\begin{multicols}{2}
\begin{enumerate}[ref={\fcProblemRef}]
\item \label{problemIntegratex*sin(x)dx} $\displaystyle \int x \sin x \diff x$.

\answer{$ -x \cos x +\sin x +C$}
\item $\displaystyle \int x e^{-x}\diff x$.

\answer{$-(1+x)e^{-x} +C$}
\item \label{problemIntegratex^2e^xdx} $\displaystyle \int x^2 e^x \diff x$.

\answer{$ x^2e^x-2xe^x+2e^x+C$}

\item $\displaystyle \int x\sin (-2x)\diff x$.

\answer{$ \frac{x}{2}\cos (-2x) +\frac{1}{4}\sin (-2x)+C$}

\item $\displaystyle \int x^2\cos (3x)\diff x$.

\answer{$ \frac{x^2}{3}\sin (3x)+\frac{2x}{9}\cos (3x)-\frac{2}{27}\sin (3x)  +C$}

\item \label{problemintx^2e^(-2x)dx} $\displaystyle \int x^2 e^{-2x}\diff x$.

\answer{$-\frac{x^2e^{-2x}}{2}-\frac{xe^{-2x}}{2}- \frac{ e^{-2x}}{4}+C$}

\item $\displaystyle \int x \sin (2x)\diff x$.

\answer{$-\frac{x}{2}\cos (2x)+\frac{1}{4}\sin (2x) +C$}
\item $\displaystyle \int x \cos (3x)\diff x$.

\answer{$\frac{x}{3}\sin (3x)+\frac{1}{9}\cos (3x) +C$}
\item $\displaystyle \int x^2 e^{2x}\diff x$.

\answer{$\frac{x^2}{2}e^{2x}-\frac{x}{2}e^{2x} +\frac{e^{2x}}{4}+C$}
\item $\displaystyle \int x^3 e^x \diff x$.

\answer{$x^3 e^x-3x^2e^x+6x e^x-6e^x +C$}
\end{enumerate}
\end{multicols}
\end{problem}
\solution{\ref{problemIntegratex*sin(x)dx}. 
\[
\begin{array}{rcl}
\displaystyle\int x \underbrace{\sin x \diff x}_{=\diff(-\cos x)} = -\int x \diff(\cos x) = -x\cos x + \int \cos x \diff x = -x \cos x +\sin x +C\quad .
\end{array}
\]
}

\solution{\ref{problemIntegratex^2e^xdx}.
\[
\begin{array}{rcl}
\displaystyle\int x^2 \underbrace{e^x \diff x}_{\diff  (e^x)} &=&\displaystyle \int x^2 \diff e^x= x^2e^x - \int e^x2x\diff x=  x^2e^x - \int 2x\diff e^x \\
&=&\displaystyle x^2e^x- 2xe^x+ \int 2e^x \diff x= x^2e^x-2xe^x+2e^x+C\quad .
\end{array}
\]
}

\solution{\ref{problemintx^2e^(-2x)dx}.
\[
\begin{array}{rcll|l}
\displaystyle \int x^2 e^{-2x}\diff x&=&\displaystyle \int x^2 \diff \left( \frac{e^{-2x}}{-2}\right)&&\text{Integrate by parts}\\
&=&\displaystyle -\frac{x^2e^{-2x}}{2}-\int \left(\frac{ e^{-2 x}}{ -2} \right) \diff \left(x^2\right)\\
&=&\displaystyle -\frac{x^2e^{-2x}}{2}+ \int xe^{-2x}\diff x\\
&=&\displaystyle  -\frac{x^2e^{-2x}}{2}+ \int x \diff \left( \frac{e^{ -2 x}}{ -2}\right)&&\text{Integrate by parts}\\
&=&\displaystyle  -\frac{x^2e^{-2x}}{2}-\frac{xe^{-2x}}{2}+ \frac{1}{2} \int e^{ -2x} \diff x\\
&=&\displaystyle -\frac{x^2e^{-2x}}{2}-\frac{xe^{-2x}}{2}- \frac{ e^{-2x }}{ 4}+C\quad .
\end{array}
\]
}

\begin{problem}
Evaluate the indefinite integral. Illustrate the steps of your solutions.
\begin{multicols}{2}
\begin{enumerate}[ref={\fcProblemRef}]
\item $\displaystyle\int x^2\cos (2x) \diff x$.

\answer{$ \frac{1}{2}x^2\sin(2x)+\frac{1}{2}x\cos(2x) - \frac{1}{4} \sin(2x) +C$}
\item 
$\displaystyle\int x^2e^{a x} \diff x$, where $a$ is a constant.

\answer{$ \frac{1}{a} x^2 e^{a x} -\frac{2}{a^2}x e^{a x}+\frac{2}{a^3} e^{a x}+C$}
\item 
$\displaystyle\int x^2e^{-ax}\diff x$, where $a$ is a constant.

\answer{$ -\frac{1}{a} x^2 e^{-a x} -\frac{2}{a^2}x e^{- a x}-\frac{2}{a^3} e^{-a x}+C$}
\item \label{problemintx^2(e^(ax)+e^(-ax))^2/4dx}
$\displaystyle\int x^2\frac{(e^{ax}+e^{-ax})^2}4\diff x$, where $a$ is a constant. 

\answer{$\begin{array}{l} \frac{1}{8}\left(a^{-1} x^{2} e^{2 a x}- a^{-1} x^{2} e^{-2 a x} \right. \\
\left. - a^{-2} x e^{2 a x}- a^{-2} x e^{-2 a x}+\frac{1}{2} a^{-3} e^{2 a x} \right.\\
\left. -\frac{1}{2} a^{-3} e^{-2 a x}+\frac{2}{3} x^{3} \right) +C
\end{array}
$ }
\item \label{problemint(e^(-sqrtx)dx)}
$ \displaystyle 
\int e^{-\sqrt{x}}\diff x\quad .
$

\answer{$-2\sqrt{x}e^{-\sqrt{x}} -2e^{-\sqrt{x}}+C$}
\item 
$\displaystyle\int \cos^2x ~ \diff x$ 

\answer{$\frac{1}{4}\sin(2x) +\frac{x}{2} +C$}
\item 
$\displaystyle \int x\ln |x|  ~ \diff x $

\answer{$ \frac{1}{2}x^2\ln |x| -\frac{x^2}{4} +C$}
\item $\displaystyle\int \frac{x}{1+x^2} \diff x$  (Hint: use substitution rule, don't use integration by parts)

\answer{$\frac{\ln\left(1+x^2\right)}{2}+C$}
\item 
$\displaystyle \int (\Arctan x) \diff x$

\answer{$x\Arctan x -  \frac{\ln\left(1+ x^2\right) }{2}+C$}
\item 
$\displaystyle \int (\Arcsin x) \diff x
$

\answer{$x\Arcsin x+ \sqrt{1-x^2}+C$}
\item $\displaystyle\int \frac{\ln x}{\sqrt{x}}\diff x $

\answer{$ 2\sqrt{x}(\ln x-2)+C$ }
\item \label{problemIntegrateArcsinSquared} $\displaystyle\int (\Arcsin x)^2 \diff x $ \quad \quad (Hint: Try substituting $x=\sin y$)

\answer{$x(\arcsin x)^2+ 2\sqrt{1-x^2}\arcsin x - 2x+C$}

\item $\displaystyle\int \frac{1}{\cos^2 x}\diff x$\quad \quad (Hint: What is the derivative of $\tan x$?)

\answer{$\tan x+C$}
\item $\displaystyle\int (\tan^2 x) \diff x $ \quad \quad (Hint: $\tan^2 x = \frac{1}{\cos^2x }-1$ )


\item \label{problemintxtan^2xdx} $\displaystyle\int x \tan^2 x \diff x $ \quad \quad (Hint: $\tan^2 x \diff x= \diff (F(x))$, where $F(x)$ is the answer from the preceding problem).

\answer{$-\frac{x^2}{2}+x\tan x + \ln |\cos x|+C$}
\item 
$\displaystyle
\int\Arctan \left(\frac{1}x\right)\diff x
$
\item \label{problemintsin(ln x)dx}

$\displaystyle\int \sin (\ln (x)) \diff x $

\answer{$ \frac{x}{2}\left(\sin (\ln x)-\cos (\ln x) \right) +C$}
\item 
$\displaystyle\int \cos (\ln (x)) \diff x $

\answer{$ \frac{x}{2}\left(\cos (\ln x)+\sin (\ln x) \right) +C$}

\item $\displaystyle\int (\ln x)^2 \diff x$.
\item $\displaystyle\int (\ln x)^3 \diff x$.
\item $\displaystyle\int x^2\cos^2x \diff x$ (This problem can be solved directly with integration by parts. An alternative and quicker solution is to use the fact that $\cos x= \frac{ e^{ix} + e^{-ix}}{2}$ and problem \ref{problemintx^2(e^(ax)+e^(-ax))^2/4dx}).
\end{enumerate}
\end{multicols}


\end{problem}
\solution{\ref{problemint(e^(-sqrtx)dx)}
\[
\begin{array}{rcll|l}
\displaystyle \int e^{-\sqrt{x}}\diff x&=& \displaystyle \int 2y e^{-y}\diff y&&\text{Subst.: } \begin{array}{rcl} \sqrt{x}&=&y\\
\frac{1}{2\sqrt{x}}\diff x&=&\diff y\\
\diff x&=&2y\diff y
\end{array}
\\
&=&\displaystyle \int 2y \diff \left(-e^{-y}\right) &&\text{int. by parts}\\
&=&-2ye^{-y}+2\int e^{-y}\diff y\\
&=&-2ye^{-y}-2e^{-y}+C\\
&=&-2\sqrt{x}e^{-\sqrt{x}} -2e^{-\sqrt{x}}+C\quad .
\end{array}
\]
}

\solution{\ref{problemintlnx/sqrt(x)dx}
\[
\begin{array}{rcll|l}
\displaystyle \int \frac{\ln x}{\sqrt{x}}\diff x&=&\int \ln x 2\diff \left(\sqrt{x}\right)&&\text{integrate by parts}\\
&=&\displaystyle (\ln x)2\sqrt{x}- \int 2\sqrt{x}\diff (\ln x)\\ 
&=&\displaystyle 2\sqrt{x}\ln x - 2\int \frac{\sqrt{x}}{x}\diff x\\
&=&\displaystyle 2\sqrt{x}\ln x - 2\int x^{-\frac{1}{2}}\diff x\\
&=&\displaystyle 2\sqrt{x}\ln x - 4\sqrt{x}+C\\
&=&\displaystyle 2\sqrt{x}(\ln x -2)+C\quad .
\end{array}
\]
}

\solution{
\ref{problemIntegrateArcsinSquared}. 
\[
\begin{array}{rcll|l}
\displaystyle \int (\Arcsin x)^2 \diff x &=&\displaystyle  \int \left(\Arcsin(\sin y)\right)^2 \diff  (\sin y) &&\text{Set } x= \sin y\\
&=&\displaystyle \int y^2 \cos y \diff y =  \int y^2 \diff (\sin y)  &&\text{Integrate by parts}\\
&=&\displaystyle y^2\sin y - \int 2y\sin y  \diff y\\
&=&\displaystyle y^2\sin y + \int 2y  \diff (\cos y ) & &\text{Integrate by parts}\\
&=&\displaystyle y^2\sin y + 2y \cos y - 2\int \cos y \diff y\\
&=&\displaystyle y^2\sin y+2y \cos y - 2\sin y +C &&\text{Substitute } y=\Arcsin x\\
&=& \displaystyle x(\arcsin x)^2 \\
&&\displaystyle+2\sqrt{1-x^2}\arcsin x - 2x+C\quad .
\end{array}
\]
}

\solution{\ref{problemintxtan^2xdx}
\[
\begin{array}{rcll|l}
\displaystyle\int x \tan^2 x \diff x &=&\displaystyle\int x\left(\sec^2 x-1 \right)\diff x &&\text{use }\sec^2x-1=\tan^2 x\\
&=&\displaystyle\int x\left(\sec^2 x-1 \right)\diff x \\
&=&\displaystyle - \int x\diff x+\int x\sec^2x\diff x&&\text{use }\diff (\tan x)= \sec^2x\diff x\\
&=&\displaystyle - \frac{x^2}{2} +\int x\diff (\tan x)&&\text{integrate by parts}\\
&=&\displaystyle-\frac{x^2}{2} +x\tan x - \int \tan x \diff x \\
&=&\displaystyle-\frac{x^2}{2} +x\tan x - \int \frac{\sin x}{\cos x}\diff x &&\text{use }\sin x\diff x=-\diff (\cos x)\\
&=&\displaystyle -\frac{x^2}{2}+x\tan x+ \int \frac{\diff (\cos x)}{\cos x} &&\text{Set }y=\cos x \\
&=&\displaystyle-\frac{x^2}{2}+x\tan x + \int \frac{1}{y}\diff y\\
&=&\displaystyle-\frac{x^2}{2}+x\tan x + \ln |y|+C&&\text{Substitute back } y=\cos x\\
&=&\displaystyle -\frac{x^2}{2}+x\tan x + \ln |\cos x|+C\quad .

\end{array}
\]

}


\solution{\ref{problemintsin(ln x)dx}

\noindent$
\begin{array}{@{\!\!}r@{~}c@{~}ll|l}
\displaystyle \int \sin (\ln x)\diff x&=&\displaystyle  x\sin (\ln x)-\int x\diff (\sin (\ln x)) &&\text{int. by parts}\\
&=&\displaystyle  x\sin (\ln x ) - \int x \left(\cos (\ln x)\right) \left(\ln x\right)'\diff x\\
&=&\displaystyle x\sin (\ln x)-\int \cos (\ln x)\diff x &&\text{int. by parts}\\
&=&\displaystyle  x\sin (\ln x)- \left( x\cos (\ln x)-\int  x\diff (\cos (\ln x)) \right)\\
&=&\displaystyle x\sin (\ln x)-x\cos (\ln x)+\int x (-\sin (\ln x))(\ln x)'\diff x\\
&=&\displaystyle x\sin (\ln x)-x\cos (\ln x)-\int \sin (\ln x)\diff x &&\begin{array}{l}\text{add } \int \sin (\ln x)\diff x\\ \text{to both sides}\end{array}\\
\displaystyle 2\int \sin (\ln x)\diff x&=&\displaystyle  x\sin (\ln x)-x\cos (\ln x)\\
\displaystyle \int \sin (\ln x)\diff x&=&\displaystyle \frac{x}{2} \left(\sin (\ln x)-\cos (\ln x) \right)\quad .
\end{array}
$
}




\begin{problem}
\label{problemintx^ne^xdx}
Compute $\displaystyle \int x^n e^x \diff x$.
\end{problem}
\solution{\ref{problemintx^ne^xdx}

\[
\begin{array}{rcl}
\displaystyle\int x^n e^x \diff x &=& \displaystyle \int x^n \diff e^x\\
&=&\displaystyle  x^n e^x- \int  e^x \diff x^n\\
&=&\displaystyle  x^n e^x- n\int  x^{n-1}e^x \diff x\\
&=&\displaystyle x^ne^x- n\left(\int x^{n-1}\diff e^x \right)\\
&=&\displaystyle  x^ne^x - n\left(x^{n-1}e^x- \int (n-1)x^{n-2}e^x \diff x \right) \\
&=&\displaystyle x^{n}e^x-nx^{n-1}xe^x+n(n-1)\int x^{n-2}e^x \diff x \\
&=&\dots \text{(continue above process)}\dots\\
&=&\displaystyle x^ne^{x}- nx^{n-1}e^x + n(n-1 )x^{n-2 } e^x + \dots \\
&&\displaystyle +(-1)^kn(n-1)(n-2)\dots(n-k+1)x^{n-k}e^x\\
&&+\dots +(-1)^n n! e^x+C\\
&=&\displaystyle C+\sum_{k=0}^{n} (-1)^n\frac{n!}{(n-k)!} x^{n-k} e^x \quad .
\end{array}
\]
}


\section{Integration of rational functions}
\subsection{Building block integrals}
\begin{problem}
Integrate. Illustrate the steps of your solution.
\begin{multicols}{2}
\begin{enumerate}[ref={\fcProblemRef}]
\item $\displaystyle \int \frac{1}{x+1}\diff {}x$

\answer{$\ln |x+1|+C$}
\item $\displaystyle \int \frac{x-1}{x+1}\diff {}x$

\answer{$x-2\ln |x+1|+C$}
\item $\displaystyle \int \frac{ 1}{(x+1)^2}\diff {}x$

\answer{$-\frac{1}{x+1}+C$}
\item $\displaystyle \int \frac{x}{(x+1)^2}\diff {}x$

\answer{$\ln |x+1|+\frac{1 }{x+ 1 }+C$}
\item $\displaystyle \int \frac{ 1}{ (2x+3)^2}\diff {}x$

\answer{$-\frac{1}{2(2x+3)}+C$ }

\item $\displaystyle
\int \frac{x}{ 2x^2+3}\diff x
$

\answer{$\frac{1}{4}\ln \left(2 x^2+ 3\right)+C$ }
\item 
$\displaystyle
\int \frac{1}{ 2x^2+3}\diff x
$

\answer{$\frac{\sqrt{6}}{6} \Arctan \left(\sqrt{ \frac{ 2}{ 3} } x \right) $}

\item \label{problemIntegrate x/(2x^2+x+1) dx}
$\displaystyle\int \frac{x }{2x^2+x+1}\diff{}x\quad .
$

\answer{$\frac{1}{4}\ln \left(x^2+ \frac{1}{2}x+ \frac{ 1 }{2} \right) - \frac{\sqrt{7}}{14} \Arctan\left( \frac{ 4x+1}{\sqrt{7}} \right) +C$}

\item $\displaystyle
\int \frac{x}{ 2x^2+x+3}\diff x
$

\answer{ $\frac{1}{4}\ln \left( 2 x^{2}+x+3 \right)- \frac{1}{ 2 \sqrt{23}} \arctan \left( \frac{ 4x +1}{ \sqrt{23} } \right) +C$}

\item $\displaystyle
\int \frac{x}{ x^2-x+3}\diff x
$

\answer{$ \frac{1}{2} \ln{} \left| x^{2}- x+3\right| + \frac{ \sqrt{11}}{11} \Arctan{} \left( \frac{ x - \frac{ 1}{2} }{ \frac{\sqrt{11}}{2}} \right) + C $}

\item \label{problemint1/(x^2+1)^2dx}  $\displaystyle
\int \frac{1}{ \left(x^2+1\right)^2}\diff x
$

\answer{$ \frac{1}{2} x \left(x^{2}+1\right)^{ -1} + \frac{ 1}{2} \Arctan{}\left( x\right) +C$}
\item \label{problemint1/(x^2+x+1)^2dx} $\displaystyle
\int \frac{1}{ \left(x^2+x +1 \right)^2 } \diff x
$

\answer{$ \frac{2}{3} x \left(x^{2}+x+1\right)^{ -1} + \frac{ 1}{3} \left(x^{2}+x+ 1\right)^{ -1} +\frac{4 }{ 9} \sqrt{3} \Arctan{} \left( \frac{2}{3} \sqrt{3} x +\frac{\sqrt{3}}{3}\right) $}

\item $\displaystyle \int \frac{1}{ \left(x^2+ 1 \right)^3 }\diff x
$

\answer{$\frac{3}{8} x \left(x^{2}+ 1\right)^{ -1} + \frac{1}{4} x \left(x^{2}+1\right)^{-2}+\frac{3}{8} \Arctan{}\left( x\right)+C$}
\end{enumerate}
\end{multicols}

\end{problem}
\solution{\ref{problemIntegrate x/(2x^2+x+1) dx}.

\noindent$
\begin{array}{r@{}c@{}l@{}l@{}|l}
\displaystyle\int \frac{x }{2x^2+x+1}\diff x&=&\displaystyle \int \frac{x}{2\left( x^2 + 2x \frac{1}{4} +\frac{1}{2} \right) }\diff x \\
&=&\displaystyle  \int \frac{x}{2\left( x^2 + 2 x \frac{1}{4} +\frac{1}{16}- \frac{1}{16} +\frac{1}{2} \right) }\diff x &&\begin{array}{l} \text{complete square}\\\text{in denominator}\end{array}\\ 
&=&\displaystyle \frac{1}{2}\int \frac{x}{ \left(x+ \frac{1}{4} \right)^2 + \frac{7}{16} } \diff x \\
&=&\displaystyle \frac{1}{2}\int \frac{  x+ \frac{1}{4} - \frac{1}{4}}{ \left(x+ \frac{1}{4} \right)^2 + \frac{7}{16} } \diff \left(x+\frac{1}{4}\right) &&\text{Set } u=x+\frac{1}{4}\\
&=&\displaystyle \frac{1}{2}\int \frac{u-\frac{1}{4}}{u^2+\frac{7}{16}} \diff u \\
&=&\displaystyle \frac{1}{2}\left(\int \frac{u}{u^2+\frac{7}{16}}\diff u - \frac{1}{4}\int \frac{1}{u^2+\frac{7}{16}}\diff u \right) \\
&=&\displaystyle \frac{1}{2} \left(\frac{1}{2}\ln \left(u^2+\frac{7}{16}\right) - \frac{1}{4\sqrt{\frac{7}{16}}}\Arctan\left(\frac{u}{\sqrt{\frac{7}{16}}}\right)\right) +K\\
&=& \displaystyle \frac{1}{4}\ln \left(x^2+ \frac{1}{2}x+\frac{1}{2} \right) - \frac{\sqrt{7}}{14}\Arctan\left(\frac{4x+1}{\sqrt{7}} \right) +K \quad .
\end{array}
$
}

\solution{\ref{problemint1/(x^2+x+1)^2dx}
\[
\begin{array}{rcll|l}
\displaystyle\int \frac{1}{\left(x^2+x+1\right)^2}\diff x &=&\displaystyle \int \frac{1}{\left(\left(x^2 +2x \frac{1}{2} +\frac{1}{4}\right) - \frac{1}{4}+1\right)^2}\diff x &&\text{complete the square}\\
&=&\displaystyle \int \frac{1}{\left(\left(x+\frac{1}{2}\right)^2+\frac{3}{4}\right)^2}\diff \left(x+\frac{1}{2}\right) &&\text{Set } w=x+\frac{1}{2}\\
&=&\displaystyle \int \frac{1}{ \left(w^2+\frac{3}{4}\right)^2} \diff w\\
&=&\displaystyle \int \frac{1}{ \left(\frac{3}{4} \left( \left( \frac{2w}{\sqrt{3}} \right)^2+1 \right)\right)^2 } \frac{\sqrt{3}}{2} \diff \left(\frac{2 w}{\sqrt{3}}\right) &&\text{Set }  z=\frac{2w}{\sqrt{3}} \\
&=&\displaystyle \frac{\frac{\sqrt{3}}{2}}{\left(\frac{3}{4}\right)^2}\int \frac{1}{\left(z^2+1\right)^2}\diff z\\
&=&\displaystyle\frac{8\sqrt{3}}{9} \int \frac{1}{\left(z^2+1\right)^2}\diff z\quad .
\end{array}
\]
The integral $\int \frac{1}{\left(z^2 + 1 \right)^2 } \diff z$ was already studied; it was also given as an exercise in Problem \ref{problemint1/(x^2+1)^2dx}. We leave the rest of the problem to the reader.
}


\begin{problem}
\label{problemIntegrateBuildingBlockIIaandIIIa} Let $a,b,c,A, B$ be real numbers. Suppose in addition $a\neq 0$ and  $b^2-4ac<0$. Integrate
\[
\int \frac{Ax +B}{ax^2+bx+c}\diff x\quad .
\]

The purpose of this exercise is to produce a formula in form ready for implementation in a computer algebra system.
\solution{\ref{problemIntegrateBuildingBlockIIaandIIIa}. 

\[
\begin{array}{rcll|l}
\displaystyle\int \frac{Ax +B}{ax^2+bx+c}\diff x&=&\displaystyle \int \frac{Ax+B}{a\left( x^2 + 2x \frac{b}{2a} +\frac{c}{a} \right) }\diff x =  \int \frac{Ax+B}{a\left( x^2 + 2x \frac{b}{2a}+\frac{b^2}{4a^2}- \frac{b^2}{4a^2} +\frac{c}{a} \right) }\diff x &&\begin{array}{l} \text{complete square}\\\text{in denominator}\end{array}\\ 
&=&\displaystyle \frac{1}{a}\int \frac{Ax+B}{ \left(x+ \frac{b}{2a} \right)^2 + \frac{4ac-b^2}{4a^2} } \diff x &&\text{Set }D=  \frac{4ac-b^2}{4a^2} \\
&=&\displaystyle \frac{1}{a}\int \frac{A \left( x+ \frac{b}{2a} - \frac{b}{2a}\right) +B}{ \left(x+ \frac{b}{2a} \right)^2 + D } \diff \left(x+\frac{b}{2a}\right) &&\text{Set } u=x+\frac{b}{2a}\\
&=&\displaystyle \frac{1}{a}\int \frac{Au+ B- \frac{Ab}{2a}}{u^2+D} \diff u &&\text{Set }C=B-\frac{Ab}{2a} \\
&=&\displaystyle \frac{1}{a}\left(A\int \frac{u}{u^2+D}\diff u + C\int \frac{1}{u^2+D}\diff u \right) \\
&=&\displaystyle \frac{1}{a} \left(\frac{A}{2}\ln (u^2+D) + \frac{C}{\sqrt{D}}\Arctan\left(\frac{u}{\sqrt{D}}\right)\right) +K\\
&=& \displaystyle \frac{1}{a} \left(\frac{A}{2}\ln \left(x^2+ \frac{b}{a}x+\frac{c}{a} \right) + \frac{C}{\sqrt{D}}\Arctan\left(\frac{x+\frac{b}{2a}}{\sqrt{D}} \right)\right) +K \quad .
\end{array}
\] 
The solution is complete. Question to the student: where do we use $b^2-4ac<0$?
}
\end{problem}
\solution{\ref{problemIntegrateBuildingBlockIIaandIIIa}. 

\[
\begin{array}{rcll|l}
\displaystyle\int \frac{Ax +B}{ax^2+bx+c}\diff x&=&\displaystyle \int \frac{Ax+B}{a\left( x^2 + 2x \frac{b}{2a} +\frac{c}{a} \right) }\diff x =  \int \frac{Ax+B}{a\left( x^2 + 2x \frac{b}{2a}+\frac{b^2}{4a^2}- \frac{b^2}{4a^2} +\frac{c}{a} \right) }\diff x &&\begin{array}{l} \text{complete square}\\\text{in denominator}\end{array}\\ 
&=&\displaystyle \frac{1}{a}\int \frac{Ax+B}{ \left(x+ \frac{b}{2a} \right)^2 + \frac{4ac-b^2}{4a^2} } \diff x &&\text{Set }D=  \frac{4ac-b^2}{4a^2} \\
&=&\displaystyle \frac{1}{a}\int \frac{A \left( x+ \frac{b}{2a} - \frac{b}{2a}\right) +B}{ \left(x+ \frac{b}{2a} \right)^2 + D } \diff \left(x+\frac{b}{2a}\right) &&\text{Set } u=x+\frac{b}{2a}\\
&=&\displaystyle \frac{1}{a}\int \frac{Au+ B- \frac{Ab}{2a}}{u^2+D} \diff u &&\text{Set }C=B-\frac{Ab}{2a} \\
&=&\displaystyle \frac{1}{a}\left(A\int \frac{u}{u^2+D}\diff u + C\int \frac{1}{u^2+D}\diff u \right) \\
&=&\displaystyle \frac{1}{a} \left(\frac{A}{2}\ln (u^2+D) + \frac{C}{\sqrt{D}}\Arctan\left(\frac{u}{\sqrt{D}}\right)\right) +K\\
&=& \displaystyle \frac{1}{a} \left(\frac{A}{2}\ln \left(x^2+ \frac{b}{a}x+\frac{c}{a} \right) + \frac{C}{\sqrt{D}}\Arctan\left(\frac{x+\frac{b}{2a}}{\sqrt{D}} \right)\right) +K \quad .
\end{array}
\] 
The solution is complete. Question to the student: where do we use $b^2-4ac<0$?
}

\begin{problem}

Let $a, b, c, A, B$ be real numbers and let $n>1$ be an integer. Suppose in addition $a\neq 0$ and $b^2-4ac<0$. Let 
\[
J(n)=\int \frac{1}{ \left( x^2+\frac{b}{a}x + \frac{ c}{a}\right)^n}\diff x\quad .
\]
\begin{enumerate}
\item \label{problemIntegrateBuildingBlockIIandIIIbPart1} Express the integral 
\[
\int \frac{Ax +B}{ \left( ax^2 +bx +c\right)^n}\diff x
\]
via $J(n)$.
\item \label{problemIntegrateBuildingBlockIIandIIIbPart2} Express $J(n)$ recursively via $J(n-1)$
\end{enumerate}
The purpose of this exercise is to produce a formula in form ready for implementation in a computer algebra system.

\solution{\ref{problemIntegrateBuildingBlockIIandIIIbPart1}.\[
\begin{array}{rcll|l}
\displaystyle\int \frac{Ax +B}{(ax^2+bx+c)^n}\diff x&=&\displaystyle \int \frac{Ax+B}{a^n\left( x^2 + 2x \frac{b}{2a} +\frac{c}{a} \right)^n }\diff x =  \int \frac{Ax+B}{a^n\left( x^2 + 2x \frac{b}{2a}+\frac{b^2}{4a^2}- \frac{b^2}{4a^2} +\frac{c}{a} \right)^n }\diff x &&\begin{array}{l} \text{complete square}\\\text{in denominator}\end{array}\\ 
&=& \displaystyle \frac{1}{a^n} \int \frac{Ax+B}{ \left( \left( x + \frac{b}{2a} \right)^2 + \frac{ 4ac-b^2}{4a^2} \right)^n } \diff x &&\text{Set }D=  \frac{ 4ac-b^2}{4a^2} \\
&=&\displaystyle \frac{1}{ a^n}\int \frac{A \left( x+ \frac{b}{2a} - \frac{b}{ 2a} \right) +B}{ \left(\left(x+ \frac{b}{2a} \right)^2 + D \right)^n } \diff \left(x + \frac{ b}{2a}\right) && \text{Set } u=x+\frac{b}{2a}\\
&=& \displaystyle \frac{1}{a^n} \int \frac{Au+ B- \frac{Ab}{2a} }{ \left(u^2+D\right)^n} \diff u &&\text{Set }C = B - \frac{A b}{2 a} \\
&=&\displaystyle \frac{1 }{a^n} \left( A\int \frac{u}{ \left( u^2 +D\right)^n}\diff u + C\int \frac{1}{\left(u^2+D\right)^n}\diff u \right) \\
&=& \displaystyle \frac{1}{a^n} \left(\frac{A}{2(1-n)} \left( u^2 +D\right)^{1-n} + C J(n ) \right) \\
&=&\displaystyle \frac{1}{a^n} \left(\frac{A}{2(1-n)} \left( x^2 + \frac{ b}{a}x + \frac{c }{a } \right)^{1-n} + C J(n ) \right) \\
\end{array}
\]

}
\solution{\ref{problemIntegrateBuildingBlockIIandIIIbPart2}.
We use all notation and computations from the previous part of the problem. According to theory, in order to solve that integral, we are supposed to integrate by parts the simpler integral 

\[
\begin{array}{rcll|l}
\displaystyle J\left(n-1\right) &=&\displaystyle \int \frac{1}{\left( x^2 +  \frac{b}{a}x +\frac{c}{a} \right)^{n-1}}\diff x = \int \frac{1}{\left(u^2+D\right)^{n-1}}\diff u &&\text{Integrate by parts}\\
&=&\displaystyle  \frac{u}{\left(u^2+D\right)^{n-1}}- \int u ~\diff \left(\frac{1}{\left(u^2+D\right)^{n-1}}\right)\\
&=&\displaystyle \frac{u}{\left(u^2+D\right)^{n-1}}+  2(n-1) \int \frac{u^2}{\left(u^2+D\right)^{n}}\diff u \\
&=&\displaystyle \frac{u}{\left(u^2+D\right)^{n-1}}+  2(n-1) \int \frac{u^2+D-D}{\left(u^2+D\right)^{n}}\diff u \\
&=&\displaystyle \frac{u}{\left(u^2+D\right)^{n-1}}+  2(n-1)J\left(n-1\right)-2D(n-1)  \int \frac{1}{\left(u^2+D\right)^{n}}\diff u \\
&=&\displaystyle \frac{u}{\left(u^2+D\right)^{n-1}}+  2(n-1)J\left(n-1\right)-2D(n-1)  J\left(n \right)
\end{array}
\]
In the above equality, we rearrange terms to get that
\[
\begin{array}{rcl}
\displaystyle 2D(n-1)  J\left(n \right)& = & \displaystyle \frac{u}{ \left(u^2 + D \right)^{n-1}} + (2n- 3) J\left( n-1\right) \\
J(n)&=&\displaystyle \frac{1}{D} \left( \frac{ u}{2( n-1) \left( u^2 +D\right)^{n-1}} +\frac{2n-3}{2n-2}J(n-1)\right) \\
&=& \displaystyle \frac{1}{D} \left( \frac{x +\frac{b }{2a}}{ (2n-2)\left(x^2+\frac{b}{a}x +\frac{c}{ a} \right)^{n-1}} +\frac{2n-3}{2n-2}J(n-1)\right)\quad .
\end{array}
\]
}
\end{problem}
\solution{\ref{problemIntegrateBuildingBlockIIandIIIbPart1}.

\noindent$
\begin{array}{@{}r@{}c@{}l@{}l@{}|l}
\displaystyle\int \frac{Ax +B}{(ax^2+bx+c)^n}\diff x&=&\displaystyle \int \frac{Ax+B}{a^n\left( x^2 + 2x \frac{b}{2a} +\frac{c}{a} \right)^n }\diff x \\
&=&\displaystyle  \int \frac{Ax+B}{a^n\left( x^2 + 2x \frac{b}{2a}+\frac{b^2}{4a^2}- \frac{b^2}{4a^2} +\frac{c}{a} \right)^n }\diff x &&\begin{array}{l} \text{complete square}\\\text{in denominator}\end{array}\\ 
&=& \displaystyle \frac{1}{a^n} \int \frac{Ax+B}{ \left( \left( x + \frac{b}{2a} \right)^2 + \frac{ 4ac-b^2}{4a^2} \right)^n } \diff x &&\text{Set }D=  \frac{ 4ac-b^2}{4a^2} \\
&=&\displaystyle \frac{1}{ a^n}\int \frac{A \left( x+ \frac{b}{2a} - \frac{b}{ 2a} \right) +B}{ \left(\left(x+ \frac{b}{2a} \right)^2 + D \right)^n } \diff \left(x + \frac{ b}{2a}\right) && \text{Set } u=x+\frac{b}{2a}\\
&=& \displaystyle \frac{1}{a^n} \int \frac{Au+ B- \frac{Ab}{2a} }{ \left(u^2+D\right)^n} \diff u &&\text{Set }C = B - \frac{A b}{2 a} \\
&=&\displaystyle \frac{1 }{a^n} \left( A\int \frac{u}{ \left( u^2 +D\right)^n}\diff u + C\int \frac{1}{\left(u^2+D\right)^n}\diff u \right) \\
&=& \displaystyle \frac{1}{a^n} \left(\frac{A}{2(1-n)} \left( u^2 +D\right)^{1-n} + C J(n ) \right) \\
&=&\displaystyle \frac{1}{a^n} \left(\frac{A}{2(1-n)} \left( x^2 + \frac{ b}{a}x + \frac{c }{a } \right)^{1-n} + C J(n ) \right) \\
\end{array}
$
}

\solution{\ref{problemIntegrateBuildingBlockIIandIIIbPart2}.
We use all notation and computations from the previous part of the problem. According to theory, in order to solve that integral, we are supposed to integrate by parts the simpler integral 

\noindent $
\begin{array}{@{}r@{}c@{}l@{}l@{}|l}
\displaystyle J\left(n-1\right) &=&\displaystyle \int \frac{1}{\left( x^2 +  \frac{b}{a}x +\frac{c}{a} \right)^{n-1}}\diff x = \int \frac{1}{\left(u^2+D\right)^{n-1}}\diff u &&\text{int. by parts}\\
&=&\displaystyle  \frac{u}{\left(u^2+D\right)^{n-1}}- \int u ~\diff \left(\frac{1}{\left(u^2+D\right)^{n-1}}\right)\\
&=&\displaystyle \frac{u}{\left(u^2+D\right)^{n-1}}+  2(n-1) \int \frac{u^2}{\left(u^2+D\right)^{n}}\diff u \\
&=&\displaystyle \frac{u}{\left(u^2+D\right)^{n-1}}+  2(n-1) \int \frac{u^2+D-D}{\left(u^2+D\right)^{n}}\diff u \\
&=&\displaystyle \frac{u}{\left(u^2+D\right)^{n-1}}+  2(n-1)J\left(n-1\right)-2D(n-1)  \int \frac{1}{\left(u^2+D\right)^{n}}\diff u \\
&=&\displaystyle \frac{u}{\left(u^2+D\right)^{n-1}}+  2(n-1)J\left(n-1\right)-2D(n-1)  J\left(n \right)
\end{array}
$
In the above equality, we rearrange terms to get that
\[
\begin{array}{rcl}
\displaystyle 2D(n-1)  J\left(n \right)& = & \displaystyle \frac{u}{ \left(u^2 + D \right)^{n-1}} + (2n- 3) J\left( n-1\right) \\
J(n)&=&\displaystyle \frac{1}{D} \left( \frac{ u}{2( n-1) \left( u^2 +D\right)^{n-1}} +\frac{2n-3}{2n-2}J(n-1)\right) \\
&=& \displaystyle \frac{1}{D} \left( \frac{x +\frac{b }{2a}}{ (2n-2)\left(x^2+\frac{b}{a}x +\frac{c}{ a} \right)^{n-1}} +\frac{2n-3}{2n-2}J(n-1)\right)\quad .
\end{array}
\]
}

\subsection{Complete algorithm: partial fractions}
\subsubsection{Quadratic term in the denominator}
\begin{problem}
Integrate. Some of the examples require partial fraction decomposition and some do not. Illustrate the steps of your solution. 
\begin{multicols}{2}
\begin{enumerate}[ref={\fcProblemRef}]
\item $\displaystyle\int \frac{1}{4x^2+4x+1}\diff x$

\answer{$-\frac{1}{2} (2 x+1)^{-1}+C$}
\item $\displaystyle \int \frac{1}{1-x^2}\diff x$

\answer{$-\frac{1}{2} \ln{}\left|x-1\right|+\frac{1}{2} \ln{} \left| x+1\right|+C $}

\item $\displaystyle \int \frac{1}{5-x^2}\diff x$

\answer{$-\frac{\sqrt{5}}{10} \ln{}\left|x- \sqrt{5} \right|+ \frac{\sqrt{5}}{10} \ln{}\left|x+ \sqrt{5} \right|+C $}

\item $\displaystyle \int \frac{x}{4x^2+x+5}\diff {}x$

\answer{$\frac{1}{8} \ln{}\left(x^{2}+\frac{1}{4} x+\frac{5}{4}\right)-\frac{1}{316}\sqrt{79} \arctan{}\left(\frac{x+\frac{1}{8}}{\frac{1}{8}\sqrt{79}}\right) +C $}
\item $\displaystyle \int \frac{x}{4x^2+x-5}\diff {}x$
\answer{$ \frac{5}{36} \ln{}\left|x+\frac{5}{4}\right|+\frac{1}{9} \ln{}\left|x-1\right|+C$}
\item $\displaystyle \int \frac{x}{4x^2 +x+ \frac{ 1}{16}}\diff {}x$
\answer{$\frac{1}{4} (8 x+1)^{-1}+\frac{1}{4} \ln{}\left|8 x+1\right|+C $}

\item $\displaystyle \int \frac{x+1}{2x^2+x}\diff {}x$

\answer{$-\frac{1}{2} \ln{}\left|2 x+1\right|+\ln{}\left|x\right| +C$}
\item $\displaystyle \int \frac{x}{2x^2+x+1}\diff {}x$

\answer{$\frac{1}{4} \ln\left(x^{2}+\frac{1}{2} x+\frac{1}{2}\right)-\frac{1}{14}\sqrt{7} \Arctan{}\left(\frac{x+\frac{1}{4}}{\frac{1}{4}\sqrt{7}}\right) +C $}
\item \label{problemIntegrate x/(2x^2+x-1)dx}
$\displaystyle\int \frac{x }{2x^2+x-1}\diff{}x
$

\answer{$\displaystyle \frac{1}{3} \ln |x+1|+\frac{1}{6} \ln \left|x-\frac{1}{2}\right| +C$}

\item $\displaystyle\int \frac{1}{x^2+x+1}\diff x$

\answer{$\frac{2}{3}\sqrt{3} \arctan{} \left( \frac{ x+ \frac{1}{2}}{\frac{1}{2}\sqrt{3}}\right) +C $}
\item $\displaystyle\int \frac{1}{2x^2+5x+1}\diff x$

\answer{$ \frac{\sqrt{17}}{17} \ln{}\left|x- \frac{ \sqrt{17} }{4} + \frac{5}{4}\right|- \frac{ \sqrt{17} }{17} \ln{}\left|x+ \frac{\sqrt{17}}{4}  + \frac{ 5}{4}\right| $}
\end{enumerate}
\end{multicols}

\end{problem}
\solution{\ref{problemIntegrate x/(2x^2+x-1)dx}
The quadratic in the denominator has real roots and therefore can be factored using real numbers. We therefore use partial fractions.
\[
\begin{array}{rcll|l}
\displaystyle \int \frac{x }{2x^2+x-1}\diff{}x&=&\displaystyle \int \frac{\frac{1}{2}x}{\left( x+1\right)\left(x-\frac{1}{2}\right)} \diff x &&\text{partial fractions, see below}\\
&=&\displaystyle  \int \frac{\frac{1}{3} }{\left( x+1\right)}\diff {}x +\int \frac{\frac{1}{6}}{ \left(x - \frac{1}{2}\right)} \diff {}x\\
&=&\displaystyle \frac{1}{3} \ln |x+1|+\frac{1}{6} \ln \left|x-\frac{1}{2}\right| +C\quad .
\end{array}
\]
Except for showing how the partial fraction decomposition was obtained, our solution is complete. 

We proceed to compute the partial fraction decomposition used above. In what follows, we will use the most straightforward technique - the method of coefficient comparison. This technique is the most laborious for a human but is perhaps the easiest to implement on a computer. The computations below were indeed carried out by a computer program written for the purpose. We note that the method of coefficient comparison is fast enough for a human, but  techniques such as the one used in the solution of Problem \ref{problemint(3x^2+2x-1)/((x-1)(x^2+1))dx} are much easier when not equipped with a computer.

We aim to decompose into partial fractions the following function (the denominator has been factored). 
\[
\frac{x }{2x^{2}+x -1}=\frac{x }{ \left(x +1\right)\left(2x -1\right)} = \frac{A_1}{x+1}+\frac{A_2}{2x-1}\quad .
\]
After clearing denominators, we get the following equality. 
\[x = A_{1} (2x -1)+A_{2} (x +1)
\]
After rearranging we get that the following polynomial must vanish. Here, by ``vanish'' we mean that the coefficients of the powers of $x$ must be equal to zero.
\[(A_{2} +2A_{1} -1)x +(A_{2} -A_{1} )\quad .
\]
In other words, we need to solve the following system.
\[
\begin{array}{llll} & 2A_{1} & +A_{2} & =1\\ & -A_{1} & +A_{2} & =0\\\end{array}
\] 

\begin{longtable}{cc} System status&Action \\\hline $\begin{array}{llll} & 2A_{1} & +A_{2} & =1\\ & -A_{1} & +A_{2} & =0\\\end{array}$ & Selected pivot column 2. Eliminated the non-zero entries in the pivot column. \\\hline $\begin{array}{llll} & A_{1} & +\frac{A_{2} }{2} & =\frac{1}{2}\\ & & \frac{3}{2}A_{2} & =\frac{1}{2}\\\end{array}$& Selected pivot column 3. Eliminated the non-zero entries in the pivot column. \\\hline $\begin{array}{llll} & A_{1} & & =\frac{1}{3}\\ & & A_{2} & =\frac{1}{3}\\\end{array}$& Final result.\\ \end{longtable}
Therefore, the final partial fraction decompocsition is: \[\frac{\frac{x }{2}}{x^{2}+\frac{x }{2} -\frac{1}{2} } =\frac{ \frac{ 1}{3}}{(x +1)}+ \frac{\frac{1}{3}}{(2x -1)}\]
}

\subsubsection{Complete algorithm}
\begin{problem}
Evaluate the indefinite integral. Illustrate all steps of your solution. 
\begin{multicols}{2}
\begin{enumerate}[ref={\fcProblemRef}]
\item $\displaystyle  \int \frac{x^3+4}{x^2+4}\diff x$

\answer{$\displaystyle \frac{x^2}{2} +2\arctan \left(\frac{x}{2}\right)-2\ln \left(x^2+4\right)+C$}
\item $\displaystyle\int \frac{4x^2 }{2x^2-1}\diff x$

\answer{$-\frac{1}{2}\sqrt{2} \ln{}\left(x+\frac{1}{2} \sqrt{2}\right) + \frac{1}{2}\sqrt{2} \ln{}\left(x- \frac{1}{2} \sqrt{2} \right)+2 x+C$}
\item $\displaystyle\int \frac{x^3}{x^2+2x-3}\diff x$

\answer{$\frac{1}{4} \ln{}|x-1|+\frac{27}{4} \ln{}|x+3|+\frac{1}{2} x^{2}-2 x$}
\item $\displaystyle\int \frac{x^3}{x^2+3x-4}\diff x$

\answer{$\frac{1}{2}x^2-3x+\frac{64}{5}\ln|x+4| + \frac{1}{5}\ln|x-1| + C$}
\item $\displaystyle\int \frac{x^3 }{2x^2+3x-5}\diff x$ 

\answer{$\frac{125}{56} \ln{}\left(x+\frac{5}{2}\right)+\frac{1}{7} \ln{}\left(x-1\right)+\frac{1}{4} x^{2}-\frac{3}{4} x+C$}
\item $\displaystyle \int \frac{x^2+1}{(x-3)(x-2)^2}\diff x$

\answer{$\displaystyle  10\ln |x-3|-9\ln |x-2|+\frac{5}{x-2}+C $}

\item $\displaystyle \int \frac{x^4 }{(x+1)^2(x+2) } \diff x$

\answer{$\displaystyle \frac{x^2}2 -4x -5\ln|x +1|+16\ln |x+2|-\frac{1}{x +1}+C$}

\item $\displaystyle\int \frac{15x^2-4x-81}{ (x-3)(x +4)(x-1)}\diff x$

\answer{$ 5 \ln{}\left|- x-4\right|+3 \ln{}\left|x-3 \right|+ 7 \ln{}\left|x-1\right|$}

\item $\displaystyle \int \frac {x^{4}+10x^{3}+ 18x^{2} +2x - 13}{x^{4} +4x^{3} +3x^{2}- 4x-4} \diff x$ 
\\~\\

Check first that $(x-1)(x+2)^2(x+1)= x^{4}+4x^{3}+3x^{2}-4x-4$. 

\answer{$3 (x+2)^{-1}+2 \ln{}\left|x+2\right| +\ln{}\left|x-1\right| +3 \ln{}\left|x+1\right|+x+C $}
\item $\displaystyle \int\frac{x^4 }{(x^2+2)(x+2) } \diff x$

\answer{$\displaystyle \frac{x^2}{2} -2x + \frac{8}{3}\ln|x +2|- \frac{1}{3}\ln \left(x^2+2\right)+\frac{2\sqrt{2}}{3} \Arctan \left(\frac{\sqrt{2}}{2}x\right)+C$ }

\item \label{problemintx^5/(x^3-1)dx} $\displaystyle \int \frac{x^5 }{x^3-1}\diff x$

\answer{$
\begin{array}{l}
\frac{1}{3} \ln{} \left| x^{2} + x + 1\right|+\frac{1}{3} \ln{} \left|x-1\right|+\frac{1}{3} x^{3}+C\\
=\frac{1}{3} \ln{} \left| x^3-1\right|+\frac{1}{3} x^{3}+C
\end{array}
$}
\item \label{problemIntegral x^4/((x^2+2)(x+1)^2)} $\displaystyle\int \frac{x^4}{(x^2+2)(x+1)^2} \diff x $

\answer{$\displaystyle x -\frac{1}{3} (x+1)^{-1}-\frac{10}{9} \ln{}\left|x+1\right| -\frac{4}{9} \ln\left|x^{2} +2\right| -\frac{ 2}{ 9} \sqrt{ 2} \Arctan{}\left(\frac{\sqrt{2}}{2} x\right)+C$}
\item \label{problemint(3x^2+2x-1)/((x-1)(x^2+1))dx} $\displaystyle\int \frac{3x^2 + 2x - 1}{(x-1)(x^2+1)} \diff x$

\answer{$2 \ln|x-1| + \frac{1}{2} \ln\left(x^2+1\right) + 3 \Arctan x+C$}

\item $\displaystyle \int \frac{x^2-1}{x(x^2+1)^2}\diff x$

\answer{$-\left(x^{2}+1\right)^{-1}+\frac{1}{2} \ln{}\left(x^{2}+1\right)- \ln{} |x|$}
\end{enumerate}
\end{multicols}
\end{problem}
\solution{ \ref{problemIntegral x^4/((x^2+2)(x+1)^2)}
To integrate a rational function, we need to decompose it into partial fractions. 

Since the numerator of the function is of degree greater than or equal to the denominator, we start the partial fraction decomposition by polynomial division.
\renewcommand{\arraystretch}{1.2}\begin{longtable}{|cccccc|} \hline&\multicolumn{5}{|c|}{\textbf{Remainder}}\\\multicolumn{1}{|c|}{} & &$\color{red}{-2x^{3}}\color{black}$ & $\color{red}{-3x^{2}}\color{black}$ & $\color{red}{-4x }\color{black}$ & $\color{red}{-2}\color{black}$ \\\hline\textbf{Divisor(s)} &\multicolumn{5}{|c|}{\textbf{Quotient(s)}}\\$x^{4}+2x^{3}+3x^{2}+4x +2$& \multicolumn{5}{|l|}{$1$}\\\hline& \multicolumn{5}{|c|}{\textbf{Dividend}}\\\multicolumn{1}{|c|}{$\underline{~}$} &$x^{4}$ & &&&\\&$x^{4}$ & $+2x^{3}$ & $+3x^{2}$ & $+4x $ & $+2$ \\\cline{2-6}&&$\color{red}{-2x^{3}}\color{black}$ & $\color{red}{-3x^{2}}\color{black}$ & $\color{red}{-4x }\color{black}$ & $\color{red}{-2}\color{black}$ \\\hline\end{longtable}
Our next step is to factor the denominator: 
\[
x^{4}+2x^{3}+3x^{2}+4x +2=\left(x +1\right)^2 \left(x^{2}+2\right).
\]
Next, we combine the two steps:
\[
\renewcommand{\arraystretch}{1.7}
\begin{array}{rcl}
\displaystyle \frac{x^{4}}{x^{4}+2x^{3}+3x^{2}+4x +2}&=&\displaystyle 1+ \frac{-2x^3-3x ^2-4x-2}{x^{4}+2x^{3} +3x^{2} +4x +2}\\
\displaystyle \frac{-2x^3-3x^2-4x-2}{x^{4}+2x^{3}+3x^{2}+4x +2} & =&\displaystyle  \frac{-2x^3-3x^2-4x-2 }{\left(x +1\right)^2\left(x^{2}+2\right)}\\
&=&\displaystyle  \frac{A_1}{(x+1)}+\frac{A_2}{(x+1)^2}+\frac{A_3+A_4x}{(x^2+2)}.
\end{array}
\]
We seek to find $A_i$'s that turn the above expression into an identity. Just as in the solution of Problem \ref{problemIntegrate x/(2x^2+x-1)dx}, we will use the method of coefficient comparison. We note that the solutions of Problems \ref{problemint(3x^2+2x-1)/((x-1)(x^2+1))dx} and  \ref{problemIntegrate x/(2x^2+x-1)dx} provide a shortcut method.


After clearing denominators, we get the following equality. 

\noindent$\begin{array}{@{}rcl@{}}
-2x^{3}-3x^{2}-4x -2 &=& A_{1} (x +1)(x^{2}+2)+A_{2} (x^{2}+2)\\
&&+(A_{3} + A_{4} x)(x +1)^{2}\\
0&=&(A_{4} +A_{1} +2)x^{3}\\
&&+(2A_{4} +A_{3} +A_{2} +A_{1} +3)x^{2}\\
&&+(A_{4} +2A_{3} +2A_{1} +4)x \\
&&+(A_{3} +2A_{2} +2A_{1} +2)\quad .
\end{array}
$

\noindent In order to turn the above into an identity we need to select $A_i$'s such that the coefficients of all powers of $x$ become zero. In other words, we need to solve the following system. 
\[
\begin{array}{llllll} & A_{1} & & & +A_{4} & =-2\\ & A_{1} & +A_{2} & +A_{3} & +2A_{4} & =-3\\ & 2A_{1} & & +2A_{3} & +A_{4} & =-4\\ & 2A_{1} & +2A_{2} & +A_{3} & & =-2\quad .\\
\end{array}
\] 
This is a system of linear equations. There exists a standard method for solving such systems called Gaussian Elimination (this method is also known as the row-echelon form reduction method). This method is very well suited for computer implementation. We illustrate it on this particular example; for a description of the method in full generality we direct the reader to a standard course in Linear algebra. 

\begin{longtable}{cc} System status & Action \\\hline $\begin{array}{llllll} & A_{1} & & & +A_{4} & =-2\\ & A_{1} & +A_{2} & +A_{3} & +2A_{4} & =-3\\ & 2A_{1} & & +2A_{3} & +A_{4} & =-4\\ & 2A_{1} & +2A_{2} & +A_{3} & & =-2\\\end{array}$& Sel. pivot column 2. Eliminate non-pivot entries. \\\hline $\begin{array}{llllll} & A_{1} & & & +A_{4} & =-2\\ & & A_{2} & +A_{3} & +A_{4} & =-1\\ & & & 2A_{3} & -A_{4} & =0\\ & & 2A_{2} & +A_{3} & -2A_{4} & =2\\\end{array}$& Sel. pivot column 3. Eliminate non-pivot entries. \\\hline $\begin{array}{llllll} & A_{1} & & & +A_{4} & =-2\\ & & A_{2} & +A_{3} & +A_{4} & =-1\\ & & & 2A_{3} & -A_{4} & =0\\ & & & -A_{3} & -4A_{4} & =4\\\end{array}$& Sel. pivot column 4. Eliminate non-pivot entries. \\\hline $\begin{array}{llllll} & A_{1} & & & +A_{4} & =-2\\ & & A_{2} & & +\frac{3}{2}A_{4} & =-1\\ & & & A_{3} & -\frac{A_{4} }{2} & =0\\ & & & & -\frac{9}{2}A_{4} & =4\\\end{array}$& Sel. pivot column 5. Eliminate non-pivot entries. \\\hline $\begin{array}{llllll} & A_{1} & & & & =-\frac{10}{9}\\ & & A_{2} & & & =\frac{1}{3}\\ & & & A_{3} & & =-\frac{4}{9}\\ & & & & A_{4} & =-\frac{8}{9}\\\end{array}$& Final result.\\ \end{longtable}

Therefore, the final partial fraction decomposition is the following. 
\[\begin{array}{rcl}
\displaystyle \frac{x^{4}}{x^{4}+2x^{3}+3x^{2}+4x +2}&=&\displaystyle 1+ \frac{-2x^{3}- 3x^{2}-4x -2}{x^{4}+2x^{3}+3x^{2}+4x +2}\\
&=&\displaystyle 1+ \frac{-\frac{10}{9}}{(x +1)}+\frac{\frac{1}{3}}{(x +1)^{2}}+\frac{-\frac{8}{9}x -\frac{4}{9}}{(x^{2}+2)}
\end{array}
\]

Therefore we can integrate as follows.
\[
\begin{array}{rcl}
\displaystyle \int \frac{x^4}{(x^2+2)(x+1)^2} \diff x &=&\displaystyle \int \left( 1+ \frac{-\frac{10}{9}}{(x +1)}+\frac{\frac{1}{3}}{(x +1)^{2}}+\frac{-\frac{8}{9}x -\frac{4}{9}}{(x^{2}+2)}\right)\diff x\\
&=&\displaystyle \int \diff x-\frac{10}{9}\int \frac{1}{(x +1)}\diff x + \frac{ 1}{3}\int \frac{ 1}{(x +1)^{2}}\diff x \\
&&\displaystyle -\frac{8}{9} \int \frac{x}{x^{2}+2}\diff x-\frac{4}{9} \int \frac{ 1}{x^{2}+2}\diff x \\
&=&\displaystyle x -\frac{1}{3} (x+1)^{-1}-\frac{10}{9} \log{}\left(x+1\right)\\
&&\displaystyle -\frac{4}{9} \log{}\left(x^{2} +2\right) -\frac{ 2}{ 9} \sqrt{ 2} \arctan{}\left(\frac{\sqrt{2}}{2} x\right)+C
\end{array}
\]
}

\solution{\ref{problemintx^5/(x^3-1)dx}
This problem can be solved directly with a substitution shortcut, or by the standard method. 

\textbf{Variant I (standard method).}

\noindent$\begin{array}{@{}r@{}c@{}l@{}l@{}|l}
\displaystyle \int \frac{x^5}{x^3-1}\diff x&=&\displaystyle \int\left( x^2+\frac{x^2}{x^3-1}\right)\diff x  &&\text{Polyn. long div. }\\
&=&\displaystyle\frac{x^3}{3}+\int \frac{x^2}{(x-1)(x^2+x+1)}\diff x&&\text{part. frac.}\\
&=&\displaystyle\frac{x^3}{3}+\int \left(\frac{\frac{1}{3}}{x -1}+\frac{\frac{2}{3}x +\frac{1}{3}}{x^{2}+x +1}\right)\diff x &&\text{complete square}\\
&=&\displaystyle \frac{x^3}{3}+\frac{1}{3}\ln |x-1|+\frac{2}{3}\int \frac{x+\frac{1}{2}}{\left(x+\frac{1}{2}\right)^2+ \frac{3}{4}}\diff x &&\text{Set } \begin{array}{rcl} u&=&\left(x+\frac{1}{2}\right)^2+ \frac{3}{4}\\\frac{1}{2}\diff u&= &\left(x+\frac{1}{2}\right) \diff x\end{array}\\
&=&\displaystyle \frac{x^3}{3}+\frac{1}{3}\ln |x-1|+\frac{1}{3} \int \frac{\diff u}{u}\\
&=&\displaystyle \frac{x^3}{3}+\frac{1}{3}\ln |x-1|+\frac{1}{3}\ln |u|+C\\
&=&\displaystyle \frac{x^3}{3}+\frac{1}{3}\ln |x-1|+\frac{1}{3}\ln |x^2+x+1|+C\\
\end{array}
$

\textbf{Variant II (shortcut method).}
\[
\begin{array}{rcll|l}
\displaystyle \int \frac{x^5}{x^3-1}\diff x&=&\displaystyle \int \frac{x^5-x^2+x^2}{x^3-1}\diff x\\
&=&\displaystyle \int \frac{x^2(x^3-1)+x^2}{x^3-1}\diff x\\
&=&\displaystyle\int x^2\diff x+ \int \frac{x^2}{x^3-1}\diff x\\
&=&\displaystyle \frac{x^3}{3}+\int \frac{\diff \left(\frac{x^3}{3}\right)}{x^3-1}\\
&=& \displaystyle \frac{x^3}{3}+\frac{1}{3} \int \frac{\diff \left(x^3-1\right)}{x^3-1}&&\text{Set }u=x^3-1\\
&=&\displaystyle \frac{x^3}{3}+\frac{1}{3}\int \frac{\diff u}{u}\\
&=&\displaystyle\frac{x^3}{3}+\frac{1}{3}\ln |u|+C\\
&=&\displaystyle \frac{x^3}{3}+\frac{1}{3} \ln \left|x^3-1\right|+C\quad .
\end{array}
\]
The answers obtained in the two solution variants are of course equal since 
\[
\ln |x-1|+\ln |x^2+x+1|= \ln \left|\left(x-1\right)\left(x^2+x+1\right)\right|=\ln \left|x^3-1\right|\quad .
\]
}

\solution{\ref{problemint(3x^2+2x-1)/((x-1)(x^2+1))dx}. 
This is a concise solution written in a form suitable for exam taking. To make this solution as short as possible we have omitted many details. On an exam, the student would be expected to carry out those omitted computations on the side. We set up the partial fraction decomposition as follows.
\[
\displaystyle \frac{3x^2 + 2x - 1}{(x-1)(x^2+1)} = \frac{A}{x-1} + \frac{Bx+C}{x^2+1}\quad .
\]
Therefore $3x^2 + 2x - 1 = A(x^2+1) + (Bx+C)(x-1)$. 
\begin{itemize}
\item We set $x = 1$ to get $4 = 2A$, so $A = 2$.
\item We set $x = 0$ to get $-1=A-C$, so $C=3$.
\item Finally, set $x = 2$ to get $15=5A+2B+C$, so $B=1$.
\end{itemize}
We can now compute the integral as follows.
\[
\displaystyle \int\left( \frac{2}{x-1} + \frac{x+3}{x^2+1} \right) \diff x = 2 \ln(|x-1|) + \frac{1}{2} \ln(x^2+1) + 3 \Arctan x+K\quad .
\]
}

\subsubsection{A large example illustrating the complete algorithm}
\begin{problem}
\label{problemint(x^6-x^5+9/2x^4-4x^3+13/2x^2-7/2x+11/4)/(x^5-x^4+3x^3-3x^2+9/4x-9/4)dx}
Integrate
\[
\int \frac{x^{6}-x^{5}+\frac{9}{2} x^{4}-4 x^{3}+\frac{13}{2} x^{2}-\frac{7}{2} x+\frac{11}{4}}{x^{5}-x^{4}+3 x^{3}-3 x^{2}+\frac{9}{4} x-\frac{9}{4}} \diff x\quad .
\]

\end{problem}
\solution{
\ref{problemint(x^6-x^5+9/2x^4-4x^3+13/2x^2-7/2x+11/4)/(x^5-x^4+3x^3-3x^2+9/4x-9/4)dx}.

\noindent\textbf{Step 1.} The first step of our algorithm is to reduce the fraction so that numerator has smaller degree than the denominator. This is done using polynomial long division as follows.

Variable name(s): $x $1 division steps total.\renewcommand{\arraystretch}{1.2}\begin{longtable}{|cccccccc|} \hline&\multicolumn{7}{|c|}{\textbf{Remainder}}\\\multicolumn{1}{|c|}{} & &&$\color{red}{\frac{3}{2}x^{4}}\color{black}$ & $\color{red}{-x^{3}}\color{black}$ & $\color{red}{+\frac{17}{4}x^{2}}\color{black}$ & $\color{red}{-\frac{5}{4}x }\color{black}$ & $\color{red}{+\frac{11}{4}}\color{black}$ \\\hline\textbf{Divisor(s)} &\multicolumn{7}{|c|}{\textbf{Quotient(s)}}\\$x^{5}-x^{4}+3x^{3}-3x^{2}+\frac{9}{4}x -\frac{9}{4}$& \multicolumn{7}{|l|}{${\color{blue}{x}} $}\\\hline& \multicolumn{7}{|c|}{\textbf{Dividend}}\\\multicolumn{1}{|c|}{$\underline{~}$} &$x^{6}$ & $-x^{5}$ & $+\frac{9}{2}x^{4}$ & $-4x^{3}$ & $+\frac{13}{2}x^{2}$ & $-\frac{7}{2}x $ & $+\frac{11}{4}$ \\&$x^{6}$ & $-x^{5}$ & $+3x^{4}$ & $-3x^{3}$ & $+\frac{9}{4}x^{2}$ & $-\frac{9}{4}x $ & \\\cline{2-8}&&&$\color{red}{\frac{3}{2}x^{4}}\color{black}$ & $\color{red}{-x^{3}}\color{black}$ & $\color{red}{+\frac{17}{4}x^{2}}\color{black}$ & $\color{red}{-\frac{5}{4}x }\color{black}$ & $\color{red}{+\frac{11}{4}}\color{black}$ \\\hline\end{longtable}

In other words, 

\noindent$
\begin{array}{@{}r@{}c@{}l}
x^{6}-x^{5}+\frac{9}{2} x^{4}-4 x^{3}+\frac{13}{2} x^{2}-\frac{7}{2} x + \frac{11}{4}  &=& (x^{5}-x^{4}+3 x^{3}-3 x^{2} + \frac{9}{4} x-\frac{9}{4}) {\color{blue}{x}} \\&&+{\color{red}{\frac{3}{2} x^{4}-x^{3} +\frac{17}{4} x^{2}-\frac{5}{4} x +\frac{11}{4}}} \quad ,
\end{array}
$

\noindent and therefore

\noindent$
\begin{array}{@{}r@{}c@{}l@{}}
\displaystyle \frac{x^{6}-x^{5}+\frac{9}{2} x^{4}-4 x^{3}+\frac{13}{2} x^{2}-\frac{7}{2} x+\frac{11}{4}}{x^{5}-x^{4}+3 x^{3}-3 x^{2}+\frac{9}{4} x-\frac{9}{4}} &=&\displaystyle {\color{blue}{x}} +\frac{\color{red}{\frac{3}{2} x^{4}-x^{3} +\frac{17}{4} x^{2}-\frac{5}{4} x +\frac{11}{4}}}{x^{5}-x^{4}+3 x^{3}-3 x^{2}+\frac{9}{4} x-\frac{9}{4} }
\\ 
&=&\displaystyle  x+\frac{6 x^{4}-4 x^{3}+17 x^{2}-5 x+11}{4x^{5}-4 x^{4}+12 x^{3}-12 x^{2}+9 x-9}.
\end{array}
$

\noindent Set
\[
N(x)= 6 x^{4}-4 x^{3}+17 x^{2}-5 x+11
\]
and
\[
D(x)= 4x^{5}-4 x^{4}+12 x^{3}-12 x^{2}+9 x-9\quad .
\]

\noindent\textbf{Step 2.} (Split into partial fractions). Factor the denominator $D(x)=4x^{5}-4 x^{4}+12 x^{3}-12 x^{2}+9 x-9$. 

We recall from elementary algebra that there is a trick to find all rational roots of $D(x)$ on condition $D(x)$ has integer coefficients. It is well known that when $\frac{p}{q}$ is a rational number, then $\pm \frac{p}{q}$ may be a root of the integer coefficient polynomial $D(x)$ only if $p$ is a divisor of the constant term of $D(x)$, and $q$ is a divisor of the leading coefficient of $D(x)$. Since in our case the leading coefficient is 4 and the constant term is -9, the only possible rational roots of $D(x)$ are $\pm 1, \pm 3, \pm 9, \pm \frac{1}{2}, \pm \frac{3}{2}, \pm \frac{9}{2}, \pm \frac{1}{4}, \pm \frac{3}{4}, \pm \frac{9}{4}$. A rational number $r$ is a root of $D(x)$ if and only if substituting $x=r$ yields 0. Direct check shows that, for example,  $D(-1)=-50$. However, $D(1)=0$ and therefore using polynomial division we get that $D(x)=(x-1)(4x^{4}+12x^{2}+9)$. We recognize that the second multiplicand is an exact square and therefore $D(x)=(x-1)(2x^2+3)^2$.


So far we got
\[
\frac{N(x)}{D(x)}= \frac{6 x^{4}-4 x^{3}+17 x^{2}-5 x+11}{(x-1)(2x^2+3)^2}\quad .
\]
In order to split $\frac{N(x)}{D(x)}$ into partial fractions, we need to find numbers $A, B, C, D, E$ such that
\[
\frac{6 x^{4}-4 x^{3}+17 x^{2}-5 x+11}{(x-1)(2x^2+3)^2}= \frac{A}{(x-1)}+\frac{Bx+C}{(2x^2+3)}+\frac{Dx+E}{(2x^2+3)^2}\quad .
\]
After clearing denominators, we see that this amounts to finding $A, B, C, D, E$ such that
\[
6 x^{4}-4 x^{3}+17 x^{2}-5 x+11= A(2x^2+3)^2+ (Bx+C)(2x^2+3)(x-1) + (Dx+E)(x-1)\quad .
\]
Plugging in $x=1$ we see that $25=25A $ and so $A=1$. We may plug back $A=1$ and regroup to get
\[
2x^{4}-4x^{3}+5x^{2}-5x+2= (Bx+C)(2x^2+3)(x-1) + (Dx+E)(x-1)\quad .
\]
Dividing both sides by $(x-1)$ we get
\[
2x^{3}-2x^{2}+3x-2= (Bx+C)(2x^2+3)+Dx+E\quad .
\]
Regrouping we get
\[
x^{3}(2- 2B) + x^2(-2-2C)+x(3-3B-D)+(-2-3C-E)=0\quad.
\]
As $x$ is an indeterminate, the above expression may vanish only if all coefficients in the preceding expression vanish. Therefore we get the system
\[
\left| \begin{array}{rcl}
2-2B&=&0\\
-2-C&=&0\\
3-3B-D&=&0\\
-2-3C-E&=&0\quad .
\end{array}   \right.
\]
We may solve the above linear system using the standard algorithm for solving linear systems (the algorithm is called row reduction and is also known as Gaussian elimination). The latter algorithm is studied in any standard the Linear algebra course. Alternatively, we see from the first equations $B=1$, $C=-1$, and substituting in the remaining equations we see $D=0$, $E=1$. Finally, we check that
\[
\frac{x^{6}-x^{5}+\frac{9}{2} x^{4}-4 x^{3}+\frac{13}{2} x^{2}-\frac{7}{2} x+\frac{11}{4}}{x^{5}-x^{4}+3 x^{3}-3 x^{2}+\frac{9}{4} x-\frac{9}{4}}
=x+\frac{1}{(x-1)}+\frac{x-1}{(2x^2+3)}+\frac{1}{(2x^2+3)^2}\quad .
\]
\textbf{Step 3.} (Find the integral of each partial fraction).
\[
\begin{array}{rcl}
\displaystyle\int x \diff x &= &\displaystyle \frac{x ^2}2+C\\
\displaystyle\int \frac{1}{x-1} \diff x &=&\displaystyle  \ln|x-1|+C\\
\displaystyle\int \frac{x-1}{2x^2+3} \diff x &=& \displaystyle \int\frac{x}{2x^2+3}\diff x -\frac{1}{3}\int \frac{1}{\frac23x^2+1}\diff x\\
&=& \displaystyle  \int \frac{\diff\left(\frac{x^2}2\right) }{2x^2+3}\diff x-\frac{1}{3} \int \frac{1}{\left(\sqrt{\frac23}x\right)^2+1} \diff x\\
&=&\displaystyle  \frac{1}{4} \int \frac{\diff(2x^2+3)}{2x^2+3}\diff x- \frac{1}{3}\int \frac{\frac{\diff \left(\sqrt{\frac23}x\right)} {\sqrt{\frac23}}} {\left( \sqrt{\frac23}x\right)^2+1} \\
&=&\displaystyle   \frac{1}{4}\ln (2x^2+3)-\frac{\sqrt{6}}{6}\Arctan \left(\sqrt{\frac23}x\right)+C
\quad .
\end{array}
\]
The last integral is
\[
\begin{array}{rcll|l}
\displaystyle \int \frac{1}{(2x^2+3)^2}\diff x&=&\displaystyle  \frac{1}{9} \int \frac{\frac{\diff\left(\sqrt{\frac{2}{3}}x\right)}{\sqrt{\frac23}}}{\left(\left(\sqrt{\frac23}x\right)^2+1\right)^2}\\
&=&\displaystyle  \frac{\sqrt{6}}{18}\int \frac{\diff\left(\sqrt{\frac23}x\right)}{\left(\left(\sqrt{\frac23}x\right)^2+1\right)^2} &&\text{Set }y=\sqrt{\frac23}x\\
&=&\displaystyle  \frac{\sqrt{6}}{18}\int \frac{\diff y}{(y^2+1)^2}\quad.
\end{array}
\]
The general form of the integral $\displaystyle\int \frac{\diff y}{(y^2+1)^2}$ \refBad{\ref{eqBuildingBlock3N}}{is solved in the theoretical discussion}{is solved in \eqref{eqBuildingBlock3N}} by integration by parts. As a review of the theory, we redo the computations directly.
\[
\begin{array}{rcl}
C+\arctan y &=&\displaystyle \int \frac{\diff y}{y^2+1}\\
&=&\displaystyle \frac{y}{y^2+1} +\int \frac{2y^2dy }{( y^2+ 1 )^2}= \frac{y}{y^2+1}+\int \frac{2(y^2+1-1)\diff y}{(y^2+1)^2}\\
&=&\displaystyle \frac{y}{y^2+1} + 2\int \frac{\diff y}{ ( y^2 +1)}- 2\int \frac{\diff y}{(y^2+1)^2}\quad.
\end{array}
\]
Transferring summands we get
\[
\int \frac{\diff y}{(y^2+1)^2}= \frac{1}{2} \left( \frac{ y }{y^2+1} +\arctan y\right) +C\quad .
\]
We recall that $y=\sqrt{\frac{2}{3}}x$ and therefore
\[
\int \frac{\diff x}{(2x^2+3)^2}=\frac{\sqrt{6}}{36}\left(\frac{\sqrt{\frac{2}{3}}x}{\left(\sqrt{\frac{2}{3}}x\right)^2+1}+\arctan \left( \sqrt{\frac{2}{3}}x\right)\right) +C
\]

To get the final answer we need to collect all terms, to get a final answer:
\[
\frac{1}{6}\left(\frac{x}{2x^2+3}\right) - \frac{5\sqrt{6}}{36} \arctan \left(\sqrt{\frac{2}{3}}x \right) +\frac{1}{4} \ln (2x^2+3) +\ln|x-1|+\frac{x ^2 } 2+ C\quad .
\]
}
\section{Trigonometric integrals}
\begin{problem}
Integrate.
\begin{enumerate}
\item $\displaystyle \int \sin (3 x) \cos (2x)\diff x$.
\item $\displaystyle \int \sin x \cos (5x)\diff x$.
\item $\displaystyle \int \cos (3x) \sin (2x)\diff x$.
\end{enumerate}

\end{problem}

\begin{problem}
Integrate.
\begin{multicols}{2}
\begin{enumerate}[ref={\fcProblemRef}]
\item $\displaystyle \int \sin^2 x \cos x\diff x$.

\answer{$ \frac{1}{3}\sin^3 x+C$}

\item $\displaystyle \int \sin^2 x\diff x$.

\answer{$ \frac{x}{2} -\frac{1}{4}\sin (2x) +C$}
\item $\displaystyle \int \cos^3 x\diff x$.

\answer{$ \sin x -\frac{1}{3}\sin^3 x+C$}


\item 
$\displaystyle \int \sin^3 x \cos^4 x\diff x$.

\answer{$ \frac{1}{7} \cos^{7}{}x-\frac{1}{5} \cos^{5}{}x +C$}

\end{enumerate}
\end{multicols}
\end{problem}

\begin{problem}
Integrate 
\begin{enumerate}
\item $\displaystyle \int \sec^3x  \diff x$.
\item $\displaystyle \int \tan^3x \diff x$.
\item $\displaystyle \int \sec^2x\tan^2x \diff x$.

\end{enumerate}
\end{problem}

\begin{problem}
Integrate.
\begin{enumerate}
\item $\displaystyle \int \sin (5 x) \sin (2x)\diff x$.

\answer{$\frac{1}{2}\left(\frac{\sin (3x)}{3}- \frac{\sin (7x)}{7} \right)+C $}
\item $\displaystyle \int \sin x \cos (2x)\diff x$.

\answer{$\frac{1}{2} \left(\cos x-\frac{\cos (3x)}{3} \right)+C$}
\item $\displaystyle \int \sec \theta  \diff \theta$.

\answer{ $\ln |\sec \theta +\tan \theta|+C$}
\item $\displaystyle \int \sec^3\theta \diff \theta$.

\answer{$\frac{1}{2}\left( \ln|\tan \theta+\sec \theta|+\sec x\tan x \right)+C$}

\item $\displaystyle \int \tan \theta \diff \theta$.
\answer{$\ln |\sec \theta| +C$}
\end{enumerate}

\end{problem}
\subsection{Trigonometric integrals solved via general method $x=2\arctan t$}
\begin{problem}
Integrate.
\begin{multicols}{2}
\begin{enumerate}[ref={\fcProblemRef}]
\item \label{problemInt1/(3+cos x)dx} $\displaystyle \int \frac{1}{3+\cos x}\diff x$.

\answer{$\displaystyle\frac{1}{\sqrt{2}} \Arctan \left( \frac{ 1}{ \sqrt{2}} \tan \left( \frac{x }{2} \right)\right)+C$}
\item $\displaystyle \int \frac{1}{4+\cos x}\diff x$.

\answer{$\displaystyle  \frac{2}{15}\sqrt{15} \Arctan{}\left(\frac{\sqrt{15}}{5}\tan\left(\frac{x}{2}\right)\right)+C$}
\item $\displaystyle \int \frac{1}{3+\sin x}\diff x$.

\answer{$\displaystyle \frac{1}{\sqrt{2}} \Arctan \left(\frac{3 \tan \left(\frac{x}{2} \right)+1 }{2\sqrt{2}}\right) +C$}
\item \label{problemInt1/(2+tan x)dx}$\displaystyle \int \frac{1}{2+\tan x}\diff x$.  (Hint: this integral can be done simply with the substitution $x=\Arctan t$.)

\answer{$\displaystyle \frac{1}{5} \ln \left(\sin x+2\cos x\right)+\frac{2}{5}x+C$}

\item \label{problemint1/(2sinx-cosx+5)dx} $\displaystyle \int \frac{ \diff x }{ 2\sin x - \cos x +5}$.


\answer{$\displaystyle \frac{ \sqrt{5}}{5}\Arctan \left( \frac{3}{ \sqrt{5}} \left({ \tan \left(\frac{\theta}{2} \right)}+\frac{1}{3} \right) \right)+C$}
\end{enumerate}
\end{multicols}
\end{problem}
\solution{\ref{problemInt1/(3+cos x)dx}
This integral is of none of the forms that can be integrated quickly. Therefore we have to use the standard rationalizing substitution $x=2\Arctan t$, $t=\tan \left(\frac{x}{2}\right)$. We recall that from the double angle formulas it follows that 
\[
\cos (2\Arctan t)=\frac{\cos^2(\Arctan t)- \sin^2(2\Arctan t)}{\cos^{2}(\Arctan t)+\sin^2(\Arctan t)}=\frac{1-t^2}{1+t^2}\quad .
\]
Therefore we can solve the integral as follows.
\[
\begin{array}{rcll|l}
\displaystyle \int \frac{1}{3+\cos x}\diff x&=&\displaystyle \int \frac{1}{3+\cos (2\Arctan t)}\diff \left(2\Arctan t\right) &&\text{Substitute } x=2\Arctan t\\
&=& \displaystyle \int \frac{1}{\left(3+\frac{1- t^2}{ 1+t^2}\right)} \frac{ 2}{\left(1+ t^2\right) } \diff t\\
&=&\displaystyle \int\frac{2}{4+2t^2}\diff t\\
&=&\displaystyle \int \frac{1}{2 +t^2}\diff t\\
&=&\displaystyle \frac{\sqrt{2}}{2} \Arctan\left(\frac{\sqrt{2 }}{2} t \right) +C\\
&=&\displaystyle \frac{\sqrt{2}}{2} \Arctan\left(\frac{\sqrt{2 }}{2} \tan\left(\frac{x}{2}\right) \right) +C\quad .
\end{array}
\]
}


\solution{\ref{problemInt1/(2+tan x)dx}
This integral is of none of the forms that can be integrated quickly. Therefore we can solve it using the standard rationalizing substitution $x=2\Arctan t$, $t=\tan \left(\frac{x}{2}\right)$. This results in somewhat long computations and we invite the reader to try it. 

However, as proposed in the hint, the substitution $x=\Arctan t$ works much faster:
\[
\begin{array}{rcll|l}
\displaystyle \int \frac{1}{2+\tan x}\diff x&=&\displaystyle \int \frac{1}{2+\tan (\Arctan t)}\diff \left(\Arctan t\right) &&\text{Substitute } x=\Arctan t\\
&=& \displaystyle \int \frac{1}{\left(2+t\right)} \frac{ 1}{\left(1+ t^2\right) } \diff t && \text{Decompose into partial fractions}\\
&=&\displaystyle \int \left( \frac{\frac{1}{5}}{(t +2)}+\frac{-\frac{t }{5}+\frac{2}{5}}{(t^{2}+1)}\right)\diff t
\\
&=&\displaystyle \frac{1}{5}\ln |t+2|-\frac{1}{10}\ln (t^2+1)+\frac{2}{5}\arctan t+C &&\text{Substitute back } t=\tan x \\
&=&\displaystyle \frac{1}{5}\ln \left|\tan x+ 2\right|-\frac{1}{10}\ln (\tan^2 x+1)+\frac{2}{5}x+C\\
&=&\displaystyle \frac{1}{5}\ln |\tan x+ 2|+\frac{1}{5}\ln \left|\cos x\right|+\frac{2}{5}x+C\\
&=&\displaystyle \frac{1}{5} \ln \left|(\tan x+2)\cos x\right|+\frac{2}{5}x+C\\
&=&\displaystyle \frac{1}{5} \ln \left|\sin x+2\cos x\right|+\frac{2}{5}x+C.\\
\end{array}
\]
}

\section{Integrals of the form $R(x, \sqrt{ax^2+bx+c})$}
\subsection{Transforming radicals of quadratics to the forms $\sqrt{u^2+1}$, $\sqrt{1-u^2}$, $\sqrt{u^2-1}$}
\begin{problem}
Find a linear substitution (via completing the square) to transform the radical to a multiple of an expression of the form $\sqrt{u^2+1}$, $\sqrt{u^2-1}$ or $\sqrt{1-u^2}$.

\begin{enumerate}[ref={\fcProblemRef}]
\item \label{problemcompletesquaresqrt(x^2+x+1)} $\sqrt{x^2+x+1}$.
\item \label{problemcompletesquaresqrt(-2x^2+x+1)} $\sqrt{-2x^2+x+1}$.
\end{enumerate}

\end{problem}
\solution{\ref{problemcompletesquaresqrt(x^2+x+1)}
\[
\begin{array}{rcl}
\displaystyle \sqrt{x^2+x+1}&=&\displaystyle \sqrt{ x^2+2\frac{1}{2}x +\frac{1}{4}-\frac{1}{4} +1} \\
&=&\displaystyle  \sqrt{ \left(x+\frac{1}{2} \right)^2- }\\
&=&\displaystyle \sqrt{\frac{3}{4}\left( \frac{4}{3} \left(x+\frac{1}{2}\right)^2+1 \right)}\\
&=&\displaystyle \frac{\sqrt{3}}{2} \sqrt{\left( \frac{2}{ \sqrt{3}} \left( x+\frac{1}{2}\right)\right)^2+1}\\
&=&\displaystyle \frac{\sqrt{3}}{2} \sqrt{u^2+1},
\end{array}
\]
where $u=\frac{2}{\sqrt{3}}\left( x+\frac{1}{2}\right) = \frac{2 \sqrt{3}}{3}x+\frac{\sqrt{3}}{3}$.
}

\solution{\ref{problemcompletesquaresqrt(-2x^2+x+1)}
\[
\begin{array}{rcl}
\displaystyle \sqrt{-2x^2+x+1}&=&\displaystyle \sqrt{ -2\left( x^2-\frac{1}{2}x -\frac{1}{2}\right) } \\
&=&\displaystyle \sqrt{ -2\left(x^2-2\frac{1}{4}x +\frac{1}{16}-\frac{1}{16}-\frac{1}{2}\right) }\\
&=&\displaystyle \sqrt{-2\left( \left(x- \frac{1}{16} \right)^2 -\frac{9}{16} \right)}\\
&=&\displaystyle  \sqrt{\frac{9}{8}\left(- \frac{16}{9} \left( x-\frac{1}{16}\right)^2+1 \right)}\\
&=&\displaystyle  \frac{3}{\sqrt{8}} \sqrt{-\left( \frac{4}{3} \left(x-\frac{1}{16}\right)\right)^2+1 }\\
&=&\displaystyle \frac{ 3}{\sqrt{8}} \sqrt{-u^2+1}
\end{array}
\]
where $u=\frac{4}{3}\left(x-\frac{1}{16}\right) = \frac{4}{3}x - \frac{1}{12}$.
}

\subsection{Trig or Euler substitution, solutions use trig substitution}
\subsubsection{Case 1: $\sqrt{x^2+1}$}
\begin{problem} 
Compute the integral. 

\begin{enumerate}[ref={\fcProblemRef}]
\item \label{problemintsqrt(1+x^2)/x^2dx} $\displaystyle\int \frac{\sqrt{1+x^2}}{x^2}\diff x$.

\answer{$\ln \left(\sqrt{1+x^2}+x \right)-\frac{\sqrt{1+x^2}}{x}+C$}
\end{enumerate}

\end{problem}
\solution{\ref{problemintsqrt(1+x^2)/x^2dx}

\textbf{Variant I.} In this variant, we use the trigonometric substitution $x=\tan \theta$ and then solve the integral using a few algebraic tricks. 
\[
\begin{array}{rcll|l}
\displaystyle \int \frac{\sqrt{1+x^2}}{x^2}\diff x&=& \displaystyle \int \frac{\sqrt{1+\tan^2 \theta}}{\tan^2\theta}\diff (\tan \theta) &&\begin{array}{l}\text{Set}\\ x=\tan \theta\\ \theta\in \left(-\frac{\pi}{2},\frac{\pi}{2}\right)\end{array} \\
&=&\displaystyle \int \frac{|\sec\theta|}{\tan^2 \theta}\sec^2\theta\diff \theta && \begin{array}{l}|\sec \theta|=\sec \theta\\ \text{for }\theta\in \left(-\frac{\pi}{2}, \frac{\pi}{2}\right)\end{array}\\
&=&\displaystyle \int \frac{\cos^2 \theta}{\cos^3\theta \sin^2 \theta}\diff \theta\\
&=&\displaystyle \int \frac{\cos\theta}{\cos^2\theta \sin^2\theta}\diff \theta\\
&=&\displaystyle \int \frac{\diff (\sin \theta)}{(1-\sin^2\theta)\sin^2\theta } &&
\begin{array}{l}
\text{Set } \\
u=\sin\theta\\
\text{for }\theta \in \left( 0,\frac{\pi}{2}\right)\\
u=\sqrt{1-\cos^2\theta} \\
u=\sqrt{1-\frac{1}{\sec^2\theta}}\\
u=\sqrt{1-\frac{1}{1+\tan^2\theta}}\\
u=\sqrt{\frac{\tan^2\theta}{1+\tan^2\theta}}\\
u=\frac{\tan\theta}{\sqrt{1+\tan^2\theta}}\\
u=\frac{x}{\sqrt{1+x^2}}
\end{array}
\\
&=&\displaystyle \int \frac{\diff u}{(1-u^2)u^2}\\
&=&\displaystyle \int \frac{\diff u}{(1-u)u^2 (u+1)}&&\text{use part. frac.}\\
&=&\displaystyle \int \left( \frac{\frac{1}{2}}{u +1}+\frac{-\frac{1}{2}}{u -1}+\frac{1}{u ^{2}}\right)\diff u \\
&=&\displaystyle -\frac{1}{2} \ln{}\left|u-1\right|+ \frac{ 1}{2} \ln{}\left(u+1\right)- u^{-1}   +C\\
&=&\displaystyle  -\frac{1}{2} \ln{}\left(1-u\right)+ \frac{ 1}{2} \ln{}\left(u+1\right)- u^{-1}   +C&& u=\frac{x}{\sqrt{1+x^2}}<1\\
&=&\displaystyle \frac{1}{2} \ln\left(\frac{1+u}{1-u} \right) - u^{-1}+C  \\
&=&\displaystyle \frac{1}{2} \ln\left(\frac{(1+u)}{(1-u)} \cdot \frac{(1+u)}{(1+u)} \right) - u^{-1}+C  \\
&=&\displaystyle \frac{1}{2} \ln \left( \frac{(1+u)^2}{1-u^2 }\right) - u^{-1}+C&&\text{use }u=\frac{x}{\sqrt{1+x^2}}\\
&=&\displaystyle \frac{1}{2}\ln \left( \frac{ (1+u)^2 }{ \frac{ 1}{1+x^2}} \right)-\frac{\sqrt{1+x^2}}{x}+C\\
&=&\displaystyle \frac{1}{2} \ln \left( \left((1+ u) \sqrt{1 + x^2} \right)^2\right)-\frac{\sqrt{1+x^2}}{x}+C\\
&=&\displaystyle \ln \left(\sqrt{1+x^2}+x \right)-\frac{\sqrt{1+x^2}}{x}+C\quad .
\end{array}
\]
\textbf{Variant II. } In this variant, we use directly the Euler substitution 

$\begin{array}{rcl}
x&=&\cot (2\Arctan t)\\
&=& \frac{1 }{2} \left( \frac{ 1}{t}-t\right)\\ 
\diff x &=&-\frac{1}{2}\left(\frac{1}{t^2}+1\right) \diff t \\
\sqrt{1+x^2}&=&\frac{1}{2}\left(\frac{1}{t}+t\right)\\
t&=&\sqrt{x^2+1}-x\\
\frac{1}{t}&=&\sqrt{x^2+1}+x\quad .
\end{array}
$
\[
\begin{array}{rcll|l}
\displaystyle \int \frac{\sqrt{1+x^2}}{x^2}\diff x&=&\displaystyle \int \frac{ \frac{1}{2} \left( \frac{1}{t}+t\right) }{\frac{1}{4}\left(\frac{1}{t}-t\right)^2} \left(-\frac{1}{2} \right)\left( \frac{1}{t^2}+1\right) \diff t\\
&=&\displaystyle  \int \frac{-t^{4}-2t^{2}-1}{(t-1)^2t(t+1)^2 }\diff t&&\text{Part. frac}\\
&=&\displaystyle \int\left(-\frac{1}{t }+\frac{1}{(t +1)^{2}}-\frac{1}{(t -1)^{2}}\right)\diff t\\
&=&\displaystyle - \ln{}t - \frac{1}{t+1}+\frac{1}{t-1}+C\\
&=&\displaystyle \ln \left(\frac{1}{t} \right) +\frac{2}{t^2-1} +C\\
&=&\displaystyle \ln \left(\sqrt{1+x^2}+x \right)+ \frac{1}{t\frac{1 }{2} \left( t-\frac{1}{t}\right)}+C\\
&=&\displaystyle \ln \left(\sqrt{1+x^2}+x \right)+ \frac{1}{t}\cdot \frac{1}{\frac{1}{2}\left(t-\frac{1}{t}\right)}+C\\
&=&\displaystyle \ln \left(\sqrt{1+x^2}+x \right)- \left(\sqrt{x^2+1}+x\right)\cdot \frac{1}{x}+C\\
&=&\displaystyle \ln \left(\sqrt{1+x^2}+x \right)- \frac{\sqrt{x^2+1}}{x}-1+C\quad .
\end{array}
\]
}

\subsubsection{Case 2: $\sqrt{1-x^2}$}
\begin{problem}
Compute the integral using a trigonometric substitution.
\begin{enumerate}[ref={\fcProblemRef}]
\item \label{problemint(sqrt(9-x^2)/(x^2)dx)}
$ \displaystyle
\int \frac{\sqrt{9-x^2}}{x^2} \diff x\quad .
$

\answer{$-\frac{\sqrt{9-x^2}}{x} - \Arcsin \left( \frac{x}{3}\right) + C$}
\end{enumerate}

\end{problem}
\solution{
\ref{problemint(sqrt(9-x^2)/(x^2)dx)}
\[
\begin{array}{rcll|l}
\displaystyle \int \frac{\sqrt{9-x^2}}{x^2}\diff x&=&\displaystyle  \int \frac{3 \sqrt{\cos^2\theta }}{ 9 \sin^2 \theta }(3\cos\theta)\diff \theta && \begin{array}{l}
\text{Set }x=3\sin \theta \\
\text{for }\theta\in  \left[\frac{\pi}{2},0\right)\cup  \left(0,\frac{\pi}{2}\right] \\
\diff x=3\cos \theta\diff \theta
\end{array}
\\
&=&\displaystyle  9 \int \frac{|\cos \theta|}{\sin^2\theta}\cos\theta \diff \theta &&\begin{array}{l}
\text{For } \theta\in \left[\frac{\pi}{2},0\right)\cup  \left(0,\frac{\pi}{2}\right] \\
\text{we have} |\cos \theta|=\cos \theta\\
\end{array}\\
&=&\displaystyle \int \cot^2 \theta \diff \theta\\
&=&\displaystyle \int (\csc^2\theta -1)\diff \theta\\
&=&\displaystyle -\cot \theta-\theta+C \\
&=&\displaystyle -\frac{\sqrt{9-x^2}}{x} - \Arcsin \left( \frac{x}{3}\right) + C,
\end{array}
\]
where we expressed $\cot \theta$ via $\sin\theta$ by considering the following triangle.
\psset{xunit=1.5cm, yunit=1.5cm}
\begin{pspicture}(-0.15,-0.4)(3.3,1.2)
\tiny
\psframe*[linecolor=white](-0.1,-0.4)(3.3,1.2)
\psline(0,0)(3, 0)(3,1)(0,0)
\psline(2.9,0)(2.9, 0.1)(3,0.1)
\parametricplot{0}{1 3 div arcsin}{t cos 0.4 mul t sin  0.4 mul}
\rput(0.5, 0.05){$\theta$}
\rput[l](3.1, 0.5){$x$}
\rput[br](1.5, 0.55){$3$}
\rput[t](1.5, -0.1){$\sqrt{9-x^2} $}
%bounding box for pdflatex compilation:
\psline[linecolor=red!1](-0.11, -0.4 )(-0.105, -0.4)
\psline[linecolor=red!1](3.3, 1.21)(3.3, 1.205)
\end{pspicture}
}

\subsection{Trig or Euler substitution, solutions use Euler substitution}
\subsubsection{Case 1: $\sqrt{x^2+1}$}

\begin{problem}
Compute the integral.

\begin{enumerate}[ref={\fcProblemRef}]
\item \label{problemIntegratesqrt(x^2+1)dx}
$\displaystyle
\int \sqrt{x^2+1}\diff x
$

\answer{$\frac{1}{2}x\sqrt{x^2+1}+\frac{1}{2}\ln\left( \sqrt{x^2+1}+x\right)+C $}

\item $\displaystyle
\int \sqrt{x^2+2}\diff x
$

\answer{$ \ln{}\left(\sqrt{\frac{1}{2} x^{2}+1}+\frac{\sqrt{2}}{2} x\right)+\frac{\sqrt{2}}{2} x \sqrt{\frac{1}{2} x^{2}+1}+C$}

\item 
$\displaystyle
\int \sqrt{x^2+x+1}\diff x
$

\answer{$\frac{3}{4} \left(\frac{1}{2} \ln{}\left(\sqrt{\frac{4}{3} (x+\frac{1}{2})^{2}+1}+\frac{2}{3}\sqrt{3} \left(x+\frac{1}{2}\right)\right)+\frac{\sqrt{3}}{3} \left(x+\frac{1}{2}\right) \sqrt{\frac{4}{3} (x+\frac{1}{2})^{2}+1}\right)  +C$}
\item \label{problemIntegrate sqrt(2x^2+2x+1)dx}
$\displaystyle
\int \sqrt{\left(2x^2+2x+1\right)}\diff x
$

\answer{$\frac{\sqrt{2}}{4} \left( \frac{1}{2} (2x+1)\sqrt{(2x+1)^2+1 }+ \frac{1}{2}\ln \left( \sqrt{(2x+1)^2+1 }+2x+1\right) \right)+C$}

\item \label{problemIntegrate sqrt(3x^2+2x+1)dx}
$\displaystyle
\int \sqrt{\left(3x^2+2x+1\right)}\diff x
$

\answer{$
\begin{array}{l}
\frac{2}{9}\sqrt{3} \left(\frac{1}{2} \ln{}\left(\sqrt{\frac{9}{2} (x+\frac{1}{3})^{2}+1}+\frac{3}{2}\sqrt{2} \left(x+ \frac{1}{3} \right)\right)\right. \\\left. +\frac{3}{4}\sqrt{2} \left(x+\frac{1}{3}\right) \sqrt{\frac{9}{2} (x+\frac{1}{3})^{2}+1}\right) +C \end{array}$}

\item \label{problemintsqrt(x^2+1)/(x+1)dx}
$\displaystyle\int \frac{\sqrt{x^2+1}}{x+1}\diff x $

\answer{$ \begin{array}{l} -\sqrt{2} \ln{}\left(\sqrt{x^{2}+1}- x+\sqrt{2}-1\right) \\
+\sqrt{2} \ln{}\left(\sqrt{x^{2}+1}- x-\sqrt{2}-1\right)\\
+ \ln{} \left( \sqrt{x^{2}+1}- x\right)\\
+ \sqrt{x^{2}+1}
\end{array}
$}
\end{enumerate}

\end{problem}
\solution{\ref{problemIntegratesqrt(x^2+1)dx}. 

This problem can be solved both via the Euler substitution and by transforming to a trigonometric integral and solving the trigonometric integral on its own. We present both variants. 


\noindent\textbf{Variant I.}
We recall the Euler substitution for $ \sqrt{x^2+1 } $\refBad{\ref{eqEulerSub-case1-cot(2arctant)}}{}{  given in  \eqref{eqEulerSub-case1-cot(2arctant)}}:

$\begin{array}{rcl}
\displaystyle x&=&\displaystyle \frac{1}{2}\left(\frac{1}{t}-t\right)\\
\displaystyle \sqrt{x^2+1}&=&\displaystyle \frac{1}2\left(\frac 1 t +t\right)\\
\displaystyle \diff x&=&\displaystyle -\frac{1}{2} \left(\frac1{t^2} +1\right)\diff t\\
\displaystyle t&=&\displaystyle \sqrt{x^2+1}-x\quad .
\end{array}$

Therefore

\noindent$
\begin{array}{@{}r@{}c@{}l@{}l@{}|l}
\displaystyle\int \sqrt{(x^2+1)}\diff x&=&\displaystyle-\int  \frac14 \left(\frac1t +t\right)\left(\frac 1 {t^2} +1\right)\diff t\\
&=&\displaystyle -\frac 1 4\int \left( \frac{1}{t^3}+ 2\frac{1}t +t\right)\diff t\\
&=&\displaystyle -\frac{1}4 \left(- \frac{t^{-2}}{2}+ 2\ln|t|+ \frac{t^2}2 \right)+C\\
&=&\displaystyle\frac{1}{8}\left(t^{-2}-t^2\right) - \frac{1}{2}\ln |t|+C &&\begin{array}{l}a^2-b^2=\\(a-b)(a+b)\end{array}\\
&=&\displaystyle \frac{1}{2} \left( \underbrace {\frac{1}{2}\left(t^{-1}  -t \right) }_{=x} \right) \left(\underbrace{\frac{1}{2}\left(t^{-1}+t\right)}_{=\sqrt{x^2+1} } \right)-\frac{1}{2}\ln |t|+C\\
&=&\displaystyle \frac{1}{2} x\sqrt{x^2+1}-\frac{1}{2} \ln \left|\sqrt{x^2+1}-x\right|+C &&\text{See below}\\
&=&\displaystyle \frac{1}{2} x\sqrt{x^2+1}+\frac{1}{2} \ln \left(\sqrt{x^2+1}+x\right) +C\quad .
\end{array}
$

\noindent Our problem is solved.  

A few comments are in order. In the above expression we would have obtained a perfectly good answer if we plugged in $t=\sqrt{x^2+1}-x$ into the fourth line, however our answer would look much more complicated. Indeed, had we not used the formula $a^2-b^2=(a-b)(a+b)$ in the fourth line, the term $t^{-2}-t^{2}$ would be equal to $\frac{1}{ (\sqrt{x^2+ 1}- x)^2 }- (\sqrt{x^2+1}-x)^2$. In turn, the term $\frac{1}{ (\sqrt{ x^2+1}-x)^2}- (\sqrt{x^2+1}-x)^2$ can be simplified to $4x\sqrt{x^2+1}$ as follows. We carry out the simplifications to illustrate some of the algebraic issues arising when dealing with integrals of radicals.

\noindent$
\begin{array}{@{}r@{}c@{}l@{}}
\displaystyle t^{-2}-t^{2}&=&\displaystyle \frac{1}{(\sqrt{x^2+1}-x)^2}- (\sqrt{x^2+1}-x)^2\\
&=&\displaystyle
\frac{(\sqrt{x^2+1}+x)^2}{(\sqrt{x^2+1}-x)^2  (\sqrt{x^2+1}+x)^2 } \\
&&\displaystyle - (\sqrt{x^2+1}-x)^2 \\
&=&\displaystyle \frac{(\sqrt{x^2+1}+x)^2}{\underbrace{((\sqrt{x^2+1})^2-x^2)^2}_{=1} }- (\sqrt{x^2+1}-x)^2 \\
&=&\displaystyle 4x\sqrt{x^2+1}\quad .
\end{array}
$

Of course, the above computations are unnecessary if we use the formula $a^2-b^2=(a-b) (a+b)$ as done in the original solution. 

\noindent We note that in the last transformation we transformed $\ln \left| \sqrt{x^2+1}-x\right|$ to $\ln \left( \sqrt{ x^2+1}-x\right)$ because the quantity $\sqrt{x^2+1}-x$ is always positive. The proof of that fact we leave for the reader's exercise.

Finally, we note that as a last simplification to our solution, we used the transformation $\ln |t|= \ln \left( \sqrt{x^2+1} -x\right) = -\ln|\frac{1}{t}|= -\ln \left( \sqrt{x^2+1} +x \right)$. This is seen as follows.

\noindent $\begin{array}{rcll|l}
\displaystyle \ln |t|&=&\displaystyle  -\ln \left| \frac{1}{ t} \right| \\
&=&\displaystyle -\ln \left(\frac{1}{ \sqrt{x^2+1} - x} \right) && \text{rationalize}\\
&=&\displaystyle - \ln \left(\frac{ \left( \sqrt{x^2+1}+ x\right) }{\left( \sqrt{x^2+1} - x\right) \left(\sqrt{x^2+1 }+x\right) } \right)\\
&=& \displaystyle -\ln \left(\frac{\sqrt{x^2+1}+x}{x^2 +1 -x^2 }\right)\\
&=& \displaystyle  -\ln \left(\sqrt{x^2+1}+x\right)\quad .\\
\end{array}
$

\noindent\textbf{Variant II.} In this variant we transform to a trigonometric integral and solve it using ad-hoc methods. We recall that if we decided to solve the trigonometric integral using the standard substitution $\theta=2\arctan t$, we would arrive at the Euler substitution given in Variant I.

\noindent $
\begin{array}{@{}r@{~}c@{~}l@{}l@{}|l}
\displaystyle \int \sqrt{x^2+1}\diff x&=&\displaystyle \int \sqrt{\tan^2\theta+1}~\diff (\tan \theta) && \begin{array}{l} \text{Set }\\ x=\tan \theta \\ \theta\in \left(-\frac{\pi}{2}, \frac{\pi }{2}\right)\end{array}\\
&=&\displaystyle \int \sqrt{sec^2\theta}\sec^2\theta\diff \theta && \sec\theta>0\\
&=&\displaystyle \int \sec^3\theta\diff \theta&&\text{Problem } \refBad{\ref{problemintsec^3xdx}}{\text{solved already}}{\ref{problemintsec^3xdx}} \\
&=&\displaystyle \frac{1}{2}\left(\tan \theta \sec \theta+\ln|\sec\theta +\tan \theta | \right)+C&&
\begin{array}{@{}l}
\psset{xunit=0.3cm, yunit=0.3cm}
\begin{pspicture}(-0.8,-0.8)(4.6,3.2)
\tiny
\fcBoundingBox{-1}{-0.8}{4.6}{3.2}
\psline(0,0)(4, 0)(4,3)(0,0)
\psline(3.8,0)(3.8, 0.2)(4,0.2)
\fcAngle{0}{3 4 div ATAN}{0.8}{}
\rput[bl](1, 0.2){$\theta$}
\rput[l](4.2, 1.5){$x$}
\rput[t](2, -0.2){$1$}
\rput[br](2, 1.5){$\sqrt{x^2+1}$}
%bounding box for pdflatex compilation:
\psline[linecolor=red!1](-0.11, -0.3 )(-0.105, -0.3)
\psline[linecolor=red!1](2.3, 1.21)(2.3, 1.205)
\end{pspicture}\\
\sec\theta=\sqrt{x^2+1}\\
\tan \theta=x
\end{array}
\\
&=&\displaystyle \frac{1}{2}\left(x \sqrt{x^2+1} +\ln\left(\sqrt{x^2+1}+x \right) \right)+C
\end{array}
$

}

\solution{\ref{problemIntegrate sqrt(2x^2+2x+1)dx}
\[
\begin{array}{@{}r@{}c@{}l@{}l@{}|l}
\displaystyle \int \sqrt{\left(2x^2+2x+1\right)}\diff x &=&\displaystyle  \int \sqrt{2}\sqrt{\left(\left(x+\frac{1}{2}\right)^2+\frac{1}{4}\right)} \diff x &&\text{complete square}\\
&=&\displaystyle  \frac{\sqrt{2}}{2} \int \sqrt{\left( 4 \left( x + \frac{ 1}{2}\right)^2+1\right)}   \diff {}x\\
&=&\displaystyle  \frac{\sqrt{2}}{2}  \int \sqrt{\left( \left( 2x +1\right)^2+ 1\right)} \frac{1}{2}\diff {}\left(2x+1\right) &&\text{Set }u=2x+1 \\
&=&\displaystyle \frac{\sqrt{2}}{4} \int \sqrt{\left(u^2+1\right)}\diff {}u && \begin{array}{@{}l} \text{Euler subst.: } \\ u = \frac{1}{2}\left(\frac{1}{t}-t\right),\\ t> 0 \\~\\ \diff u = -\frac{1}{2}\left(\frac{1}{t^{2}}+1\right)\diff t \\~\\ \sqrt{u^2+1}= \frac{1}{2}\left(\frac{1}{t}+t\right)\\~\\ t= \sqrt{u^2+1}-u \end{array}\\
&=&\displaystyle - \frac{\sqrt{2} }{16} \int\left(\frac{1}{t}+t\right)\left(\frac{1}{t^2}+1 \right) \diff t\\
&=&\displaystyle - \frac{\sqrt{2} }{16} \int \left( t^{-3}+2t^{-1}+t\right)\diff t\\
&=&\displaystyle - \frac{\sqrt{2} }{16}  \left( -\frac{t^{-2}}{2} + 2\ln |t| + \frac{t^2}{2}\right)+C&& \begin{array}{@{}l}
\text{simplify as} \\
\text{in Problem \ref{problemIntegratesqrt(x^2+1)dx}}
\end{array}
\\
&=&\displaystyle \frac{\sqrt{2}}{8} \left( u\sqrt{u^2+1}+ \ln\left(\sqrt{u^2+1}+u \right) \right) +C \\
&=&\displaystyle \frac{\sqrt{2}}{8} \left( (2x+1)\sqrt{(2x+1)^2+1 }\right.\\
&&\displaystyle ~~~~~~ \left.+\ln \left( \sqrt{(2x+1)^2+1 }+2x+1\right) \right)+C.
\end{array}
\]
}



\solution{\ref{problemintsqrt(x^2+1)/(x+1)dx}

$\begin{array}{@{}r@{}c@{}l@{}l@{}|l}
\displaystyle \int \frac{\sqrt{ x^2+1}}{x+1}\diff x&=& \displaystyle \int \frac{ \frac{1}{2}\left(\frac{1}{t}+t \right) }{ \frac{1}{2}\left(\frac{1}{t}-t \right)+1 }\diff \left(\frac{1}{2} \left(\frac{1}{t}-t \right) \right) &&\begin{array}{l}\text{Euler sub: }\\ \begin{array}{@{\!\!\!\!\!}r@{}c@{}l} x&=&\frac{1}{2}\left(\frac{1}{t}-t\right) \\
\sqrt{x^2+1}&=&\frac{1}{2}\left(\frac{1}{t}+t\right)
\end{array}\end{array}\\
&=&\displaystyle \int \left(\frac{1+t^2}{1-t^2+2t }\right)\frac{1}{2} \left(-t^{-2}-1 \right)\diff t\\
&=&\displaystyle \int \frac{1}{2} \frac{(1+t^2)\left(-t^{-2}-1 \right)}{1-t^2+2t } \diff t\\
&=&\displaystyle \frac{1}{2} \int \frac{ t^{4}+2 t^{2}+1}{ t^{4}-2 t^{3}-t^{2}} \diff t &&\text{pol. long div.}\\
&=&\displaystyle \frac{1}{2}\int \left(1+\frac{ 2t^{3}+ 3t^{2} +1}{ t^2\left(t^{2}-2 t-1\right)}\right) \diff t &&\text{part. fractions}\\
&=&\displaystyle \frac{1}{2}\int \left(1+ \frac{2\sqrt{2}}{t -\sqrt{2}-1}+\frac{-2\sqrt{2}}{t +\sqrt{2}-1}+\frac{2}{t }+\frac{-1}{t^{2}}\right)\diff t\\
&=&\displaystyle -\sqrt{2} \ln{}\left|t+\sqrt{2}-1\right|+\sqrt{2} \ln{}\left|t-\sqrt{2}-1\right|\\
&&\displaystyle+\frac{1}{2} t^{-1}+\ln{}\left|t\right|+\frac{1}{2} t +C&& t=\sqrt{x^2+1}-x\\
&=&\displaystyle -\sqrt{2} \ln{}\left(\sqrt{x^{2}+1}- x+\sqrt{2}-1\right)\\
&&\displaystyle +\sqrt{2} \ln{}\left(\sqrt{x^{2}+1}- x-\sqrt{2}-1\right)\\
&&\displaystyle + \ln{} \left( \sqrt{x^{2}+1}- x\right)\\
&&\displaystyle +\frac{1}{2} \left(\sqrt{x^{2}+1}- x\right)^{-1}+\frac{1}{2} \sqrt{x^{2}+1}-\frac{1}{2} x+C &&\begin{array}{l}\text{Last 3 terms}\\ \text{simplify}\end{array}\\
&=&\displaystyle -\sqrt{2} \ln{}\left(\sqrt{x^{2}+1}- x+\sqrt{2}-1\right) \\
&&\displaystyle +\sqrt{2} \ln{}\left(\sqrt{x^{2}+1}- x-\sqrt{2}-1\right)\\
&&\displaystyle + \ln{} \left( \sqrt{x^{2}+1}- x\right)\\
&&+ \sqrt{x^{2}+1}+C\quad .

\end{array}
$

}
\begin{problem}
\label{problemIntegrate sqrt(ax^2+bx+c)dx}
Let $b^2-4ac<0$ and $a>0$ be (real) numbers. Show that 

\noindent $
\int \sqrt{\left(a x^2+b x+c\right)}\diff x = \frac{\sqrt{a} D}{2} \left( \ln{}\left( \frac{2\sqrt{D} a \sqrt{\left(\frac{2 x a+b}{2 \sqrt{D} a}\right)^{2}+1} +2 x a+b}{2 \sqrt{D} a}\right)+\frac{ \left(2 x a+b\right)}{2 \sqrt{D} a} \sqrt{\left(\frac{2 x a+b}{2 \sqrt{D} a}\right)^{2}+1}\right)+C,
$

where 
$\displaystyle D=\frac{4a c-b^2}{4a^2}$.





\end{problem}
\subsubsection{Case 2: $\sqrt{1-x^2}$}
\begin{problem}
Integrate
\begin{enumerate}[ref={\fcProblemRef}]
\item  \label{problemintsqrt(1-x^2)dx}
$\displaystyle
\int \sqrt{1-x^2}\diff x$

\answer{$ $}
\item 
$\displaystyle
\int \sqrt{2-x^2}\diff x
$
\item 
$\displaystyle
\int \sqrt{-x^2+x+1}\diff x
$
\item 
$\displaystyle
\int \sqrt{2-x-x^2}\diff x
$

\item \label{problemintsqrt(1-x^2)/(1+x)dx}
$\displaystyle
\int \frac{\sqrt{1-x^2} }{1+x}\diff x
$
\item 
$\displaystyle
\int \frac{\sqrt{1-x^2} }{2+x}\diff x
$
\end{enumerate}
\end{problem}
\solution{\ref{problemintsqrt(1-x^2)dx} 

\textbf{Variant I.} This integral can quickly be solved using a trig substitution. The Euler substitution results in a slightly longer solution, shown in the next solution variant.

\noindent $
\begin{array}{rcl@{}l@{}|l}
\displaystyle \int \sqrt{1-x^2}\diff x &=&\displaystyle  \int \sqrt{1-\cos^2\theta}\diff (\cos \theta) &&\text{Set }x=\cos \theta, \theta\in[0,\pi]\\
&=&\displaystyle \int \sqrt{\sin^2\theta } (-\sin \theta)\diff \theta &&\theta\in [0,\pi]\Rightarrow \sin \theta\geq 0\\
&=&\displaystyle -\int \sin^2\theta \diff \theta && \sin^2\theta= \frac{1-\cos (2\theta) }{2}\\
&=&\displaystyle-\int \frac{1-\cos (2\theta)}{2}\diff \theta\\
&=&\displaystyle - \frac{\theta}{2}+\frac{ \sin (2\theta)}{4}+C\\
&=&\displaystyle - \frac{\theta}{2}+\frac{ 2\sin \theta\cos \theta }{4}+C
&&
\begin{array}{@{}r@{}c@{}l}
x &=&\cos \theta\\
\theta &=&\Arccos x\\
\sin \theta&=&\sin \left(\Arccos x\right)\\
&=&\sqrt{1-x^2}
\end{array}
\\
&=&\displaystyle -\frac{\Arccos x}{2}+ \frac{x\sqrt{1-x^2} }{2} + C\\
&=&\displaystyle \frac{\Arcsin x}{2}+\frac{x\sqrt{1- x^2} }{2} + K\quad ,
\end{array}
$
where for the last equality we recall that the derivative of $\Arcsin x$ is minus the derivative of $\Arccos x$.



\textbf{Variant II.} We show how to do this integral via the Euler substitution $x=\cos (2\Arctan t)$.

$
\begin{array}{rcl@{}l@{}|l}
\displaystyle \int \sqrt{1-x^2}\diff x &=&\displaystyle  \int \sqrt{1-\cos^2\theta}\diff (\cos \theta) &&\begin{array}{@{}r@{}c@{}l} &&\text{Set }\\
x&=&\cos(2\Arctan t)\\
\frac{1}{2}\Arccos x&=&\Arctan t\\ 
x&=&\frac{1-t^2}{1+t^2}\\
&=& \frac{2}{1+t^2}-1\\
\sqrt{1-x^2}&=&\frac{2t}{1+t^2}\\
\end{array}\\
&=&\displaystyle \int \frac{2t}{1+t^2}\diff \left(\frac{1-t^2}{1+t^2} \right) \\
&=&\displaystyle \int \frac{2t}{1+t^2} \left(\frac{-4t}{\left(1+t^2\right)^2}\right)\diff t &&\begin{array}{l}
\text{Integral rational}\\
\text{function}\\
\text{we skip details}
\end{array} \\
&=&\displaystyle \frac{-t}{t^{2}+1}+\frac{2 t}{ \left(t^{2}+1\right)^{2}}\\
&&\displaystyle - \Arctan{}t+C \\
&=&\displaystyle -\frac{1}{2}\sqrt{1-x^2} +\frac{\sqrt{1-x^2}}{t^2+1} \\
&&- \Arctan t+C\\
&=&\displaystyle \frac{1}{2} \sqrt{1-x^2} \left(\frac{ 2}{ t^2 +1} -1\right)\\
&&-\Arctan t+C\\
&=&\displaystyle \frac{x\sqrt{1-x^2}}{2}-\frac{1}{2}\arccos x+C\\
&=&\displaystyle \frac{x\sqrt{1-x^2}}{2}+\frac{1}{2}\arcsin x+K,
\end{array}
$
where for the very last equality we used the fact that the derivatives of $\arcsin x$ and $\arccos x$ are negatives of one another.

\textbf{Variant III. } We show how to do this integral geometrically, provided already know the area of a sector of circle. Of course, here we assume we have already derived the formula for an area of a circle. We warn the reader that most methods for deriving the formula of a sector area rely on integrals, so it is possible we are making a circular reasoning argument. Since we already did the integral purely algebraically in the preceding solution variants, we can safely ignore the danger of the aforementioned circular reasoning argument. In other words, the present solution Variant is a geometric interpretation of the problem which relies on the formula for sector area of a circle (which we assumed proved elsewhere, possibly using similar integration techniques to the ones presented in Variant I and II).

By the Fundamental Theorem of Calculus, the indefinite integral measures up to a constant the area locked under the graph of $\sqrt{1-x^2}$. This graph is a part of a circle. Therefore, up to a constant, $\int\sqrt{1-t^2}\diff t$ equals $\int_{0}^{x}\sqrt{1-t^2}\diff t$. In turn $\int_{0}^{x}\sqrt{1-t^2}\diff t$ is given by the area highlighted in the picture below.

\psset{xunit=2cm, yunit=2cm}
\begin{pspicture}(-0.3,-0.4)(1.3,1.3)
\tiny
\pscustom*[linecolor=\fcColorAreaUnderGraph]{
\psplot[linecolor=\fcColorGraph]{0}{0.75}{1 x x mul sub sqrt}
\psline(! 0.75 1 0.75 0.75 mul sub sqrt)(0.75, 0)
\psline(0.75, 0)(0,0)
}
\fcAxesStandardNoFrame{-0.3}{-0.3}{1.2}{1.2}

\psplot[linecolor=\fcColorGraph, linewidth=1.5pt]{0}{0.75}{1 x x mul sub sqrt}
\psplot{0.75}{1}{1 x x mul sub sqrt}
\rput[t](0.75,-0.05){$P$}
\fcFullDot{0.75}{0}
\rput(0.375, -0.1){$x$}
\psline(0,0)(! 0.75 1 0.75 dup mul sub sqrt)(0.75, 0)
\rput[l](0.2, 0.5){$A$}
\rput[b](0.5, 0.2){$B$}
\rput[lb](0.45, 0.9){$\Arcsin x $}
\fcFullDot{0}{1}
\rput[r](-0.1, 1){$P$}
\rput[l](0.8, 0.3){$\sqrt{1-x^2}$} 
\rput[l](0.8, 0.65){$Q$}
\fcFullDot{0.75}{1 0.75 0.75 mul sub sqrt}
\rput[tr](-0.05, -0.05){$O$}
\end{pspicture}
\[
\begin{array}{rcll|l}
\text{Area}(A)&=&\displaystyle  \frac{\text{length }\left( \stackrel{\frown}{ PQ}
\right) }{2\pi } \pi= \frac{\text{length }\left(
\stackrel{\frown}{ PQ}
\right)}{2} = \frac{\Arcsin x}{2}\\
\text{Area}(B)&=&\displaystyle \text{Area}(\triangle OPQ)= \frac{x \sqrt{1-x^2} }{2}\\
\displaystyle \int_{0}^x\sqrt{1-t^2}\diff t&=&\displaystyle  \text{Area}(A)+\text{Area}(B)\\
&=&\displaystyle \frac{\Arcsin x}{2}+\frac{x\sqrt{1-x^2}}{2}\\
&\Rightarrow\\
\displaystyle\int\sqrt{1-x^2}\diff x&=&  \displaystyle \frac{\Arcsin x}{2} + \frac{x\sqrt{1-x^2}}{2}+C\quad .
\end{array}
\] 
}

\solution{\ref{problemintsqrt(1-x^2)/(1+x)dx} In this problem solution we use the standard Euler substitution $x=\cos (2\Arctan t)$. We recall \refBad{\ref{eqEulerSubx=cos(2arctant)}}{that}{from \eqref{eqEulerSubx=cos(2arctant)} that}

\noindent $\begin{array}{rcl}
\displaystyle x& =&\displaystyle \cos(2\Arctan t)= \frac{1-t^2}{1+t^2}\\
\Arccos (x)&=& 2 \Arctan t\\
\displaystyle \diff x &=&\displaystyle  -\frac{4t}{(1+t^2)^2}\diff t\\
\displaystyle \sqrt{1-x^2}&=&\displaystyle \sin (2\Arctan t)= \frac{2t}{1+t^2}\\
t&=&\displaystyle \frac{\sqrt{1-x^2}}{x+1}\quad .
\end{array}
$

\noindent $
\begin{array}{@{}r@{}c@{}l@{}l@{}|l}
\displaystyle \int  \frac{\sqrt{1-x^2}}{1+x}\diff x&=& \displaystyle \int t \left(-\frac{4t}{\left(1+t^2\right)^2} \right)\diff t&&\begin{array}{l} \text{Set }x=\frac{1-t^2}{1+t^2}\\
\text{Use f-las above}
\end{array}
\\
&=&\displaystyle -4\int\frac{t^2}{\left(1+t^2\right)^2}\diff t\\
&=&\displaystyle -4\int\frac{1+t^2-1}{\left(1+t^2\right)^2 }\diff t\\
&=&\displaystyle -4\int\left(\frac{1}{1+t^2} -\frac{1}{\left(1+t^2\right)^2}\right)\diff t\\
&=&\displaystyle -4\left(\Arctan t - \frac{1}{2}\left(\Arctan t+ \frac{t}{1+t^2} \right) \right)+C\\
&=&\displaystyle -2\left(\Arctan {}t - \frac{t}{1+t^2} \right) +C\\
&=&\displaystyle -2\left( \Arctan {}\left(\frac{\sqrt{1-x^2}}{1+x} \right) - \frac{1}{2}\sqrt{1-x^2} \right) +C\\
&=&\displaystyle -2 \Arctan {}t  +\sqrt{1-x^2} +C &&\text{Use f-las above}
\\
&=&\displaystyle -\Arccos x+ \sqrt{1-x^2}  +C\\
&=&\displaystyle \Arcsin x +\sqrt{1-x^2}+K\quad .
\end{array}
$

We have included the last equality to remind the student that derivatives of $\Arcsin(x)$ and $\Arccos x$ are negatives of one another.


}

\subsubsection{Case 3: $\sqrt{x^2-1}$}
\begin{problem}
Integrate
\begin{enumerate}[ref={\fcProblemRef}]
\item 
\[
\int \sqrt{x^2-1}\diff x
\]
\item 
\[
\int \sqrt{x^2-2}\diff x
\]
\item 
\[
\int \sqrt{2x^2+x-1}\diff x
\]
\item 
\[
\int \sqrt{x^2+x-1}\diff x
\]
\end{enumerate}
\end{problem}
\input{../../modules/trig-substitution/homework/integration-euler-substitution-case3-solutions}

\subsection{Theory through problems (Optional material)}
\subsection{Case 1: $\sqrt{x^2+1}$}
\subsubsection{$x=\cot \theta$}
\begin{problem} 
\begin{enumerate}[ref={\fcProblemRef}]
\item \label{problemTheoreticalTrigSubx=cott} Express $x, \diff x $ and $\sqrt{x^2+1 }$ via $\theta$ and $\diff \theta$ for the trigonometric substitution $x=\cot \theta $, $\theta\in \left(0,\pi\right)$.
\item \label{problemTheoreticalTrigSubx=cot(2arctant)} Express $x, \diff x $ and $\sqrt{x^2+1}$ via $t$ and $\diff t$ for the Euler substitution $x=\cot(2\arctan t)$, $t>0$. Express $t$ via $x$.
\end{enumerate}


\end{problem} 
\solution{\ref{problemTheoreticalTrigSubx=cott}
The trigonometric substitution $x=\cot \theta$ is given by
\begin{equation*}
\begin{array}{rcll|l}
\displaystyle \sqrt{ x^2+1}&=&\displaystyle \sqrt{\cot^2 \theta+1}\\
&=&\displaystyle \sqrt{\frac{\cos^2\theta}{ \sin^2 \theta} + 1}\\
&=&\displaystyle \sqrt{ \frac{ \cos^2 \theta+\sin^2\theta}{ \sin^2 \theta}} \\
&=& \displaystyle  \sqrt{\frac{1}{\sin^2\theta}} && \begin{array}{l}\displaystyle \text{when }\theta\in \left(0 , \pi\right) \text{ we have }\\ ~ \sin \theta \geq 0\text{ and so } \sqrt{\sin^2 \theta}=\sin\theta  \end{array}\\
&=&\displaystyle  \frac{1}{\sin \theta}= \csc \theta\quad .
\end{array}
\end{equation*}
The differential $\diff x$ can be expressed via $\diff \theta$ from $x=\cot \theta$. The substitution $x=\cot \theta$ can be now summarized as:
\[
\begin{array}{rcl}
x&=&\displaystyle \cot \theta\\
\sqrt{x^2+1}&=&\displaystyle \frac{1}{\sin \theta}=\csc \theta\\
\diff x&=&\displaystyle  -\frac{\diff \theta}{\sin^2\theta} = - \csc^2 \theta \diff \theta\\
\theta& =& \Arccot x\quad .
\end{array}
\]

}
\solution{\ref{problemTheoreticalTrigSubx=cot(2arctant)}
We recall that the substitution $\theta=2\arctan t$ transforms a trigonometric integral into an integral of a rational function. We now apply the substitution $\theta=2\arctan t$ after the substitution $x=\cot\theta$:
\[
\begin{array}{rcll|l}
x&=&\displaystyle \cot \theta &&\text{use } \theta=2\arctan t\\
&=& \displaystyle \cot \left(2\arctan t\right) &&\displaystyle  \text{use\refBad{\ref{eqSinCosViaTan}}{ }{ \eqref{eqSinCosViaTan}: }} \cot 2z=\frac{\cos (2z)}{\sin (2z)}=\frac{1-\tan^2z}{2\tan z } \\
&=&\displaystyle \frac{1-\tan^2 (\arctan t)}{2 \tan (\arctan t)} \\
&=&\displaystyle \frac{1-t^2}{2t}\\
&=&\displaystyle \frac{1}{2}\left(\frac{1}t -t\right)\quad .
\end{array}
\]
We can furthermore compute
\begin{equation} \label{eqsqrtx2plus1Euler2}
\begin{array}{rcll|l}
\displaystyle \sqrt{x^2+1}&=& \displaystyle  \sqrt{ \frac{1}{4} \left(\frac{1}t -t \right)^2 +1}\\
&=&\displaystyle \frac{1}{2} \sqrt{\left( \frac{1}{t} +t \right)^2} & &\displaystyle \sqrt{\left(\frac{1}{t}+t\right)^2} = \frac{1}{t} +t \text{ because }t>0\\
&=&\displaystyle \frac{1}{2}\left(\frac{1}{t}+t\right)\quad .
\end{array}
\end{equation}
The differential $\diff x$ can via $\diff x $ as follows.
\[
\diff x=\diff \left(\frac{1}{2} \left( \frac{1}{t} - t\right)\right) = -\frac{1}{2} \left(\frac{1}{t^2}-1\right)\quad .
\]
Finally, we can subtract $\displaystyle x=\frac{1}{2} \left( \frac{1}{t} - t\right)$ from  $\displaystyle \sqrt{x^2+1}= \frac{1}{2} \left( \frac{1 }{ t} +t\right)$ to get that \[t=\sqrt{x^2+1}-x \quad .\]
The Euler substitution $x=\cot \theta= \cot (\arctan 2t)$ can be now summarized as:
\begin{equation}\label{eqEulerSub-case1-cot(2arctant)}
\begin{array}{rcl}
x&=&\displaystyle \frac12\left(\frac{1}{t}- t\right)\\
\displaystyle\sqrt{x^2+1}&=& \displaystyle \frac12 \left(\frac1t +t\right)\\
\displaystyle \diff x&=&\displaystyle -\frac12\left(\frac{1}{t^2}+1\right) \diff t\\
t &=&\sqrt{x^2+1}-x\quad .
\end{array}
\end{equation}
}


\begin{problem} 
\label{problemEulerSub-case1-cot(2arctant)-alternative-exposition}
Let the variables $x$ and $t$ be related via $\sqrt{x^2+1}=x+t$.
\begin{enumerate}[ref={\fcProblemRef}]
\item \label{problemEulerSub-case1-cot(2arctant)-alternative-exposition-x-via-t} Express $x$ via $t$.
\item \label{problemEulerSub-case1-cot(2arctant)-alternative-exposition-radical-via-t} Express $\sqrt{x^2+1}$ via $t$ alone.
\item Express $\diff x$ via $t$ and $\diff t$.
\end{enumerate}
\end{problem} 

\solution{\ref{problemEulerSub-case1-cot(2arctant)-alternative-exposition-x-via-t}.
\[
\begin{array}{rcll|l}
\sqrt{x^2+1}&=& x+t &&\text{square both sides}  \\
x^2+1&=&x^2+2xt+t^2\\
-2xt&=&t^2-1\\
x&=&\displaystyle\frac{1}{2}\left(\frac{1}{t}-t \right) \quad .
\end{array}
\]
}

\solution{\ref{problemEulerSub-case1-cot(2arctant)-alternative-exposition-radical-via-t}. We use \ref{problemEulerSub-case1-cot(2arctant)-alternative-exposition-x-via-t}:
\[
\sqrt{x^2+1}= x+t = \frac{1}{2}\left(\frac{1}{t}-t \right)+t=\frac{1}{2}\left(\frac{1}{t}+t \right)\quad.
\]

}



\subsubsection{$x=\tan \theta$}
\begin{problem} 
\begin{enumerate}[ref={\fcProblemRef}]
\item \label{problemTheoreticalTrigSubx=tant} Express $x, \diff x $ and $\sqrt{x^2+1 }$ via $\theta$ and $\diff \theta$ for the trigonometric substitution $x=\tan \theta $, $\theta\in \left(-\frac{\pi}{2},\frac{\pi}{2}\right)$.
\item \label{problemTheoreticalTrigSubx=tan(2arctant)} Express $x, \diff x $ and $\sqrt{x^2+1}$ via $t$ and $\diff t$ for the Euler substitution $x=\tan(2\Arctan t)$, $t\in(-1,1)$. Express $t$ via $x$.
\end{enumerate}


\end{problem} 
\solution{\ref{problemTheoreticalTrigSubx=tant}
The trigonometric substitution $x=\tan \theta$ is given by
\[
\begin{array}{rcll|l}
\displaystyle \sqrt{ x^2+1}&=&\displaystyle \sqrt{\tan^2 \theta+1}\\
&=&\displaystyle   \sqrt{ \frac{ \sin^2 \theta}{ \cos^2 \theta} +1}\\
&=&\displaystyle \sqrt{ \frac{ \sin^2\theta+\cos^2 \theta}{ \cos^2 \theta}} \\
&=& \displaystyle \sqrt{\frac{1}{\cos^2\theta}} && \begin{array}{l} \displaystyle \text{when }\theta\in \left(-\frac{\pi}{2}, \frac{\pi}{2}\right) \text{ we have }\\ ~ \cos \theta > 0\text{ and so } \sqrt{\cos^2 \theta}=\cos\theta \end{array}\\
&=&\displaystyle  \frac{1}{\cos \theta}= \sec \theta\quad .
\end{array}
\]
The differential $\diff x$ can be expressed via $\diff \theta$ from $x=\tan \theta$. The substitution $x=\tan \theta$ can be now summarized as:
\[
\begin{array}{rcl}
x&=&\displaystyle \tan \theta\\
\sqrt{x^2+1}& =& \displaystyle \frac{1}{\cos \theta}=\sec \theta\\
\diff x &=&\displaystyle \frac{\diff \theta}{\cos^2\theta}= \sec^2 \theta \diff \theta\\
\theta& =& \arctan x\quad .
\end{array}
\]

}
\solution{\ref{problemTheoreticalTrigSubx=tan(2arctant)}
We recall that the substitution $\theta=2\arctan t$ transforms a trigonometric integral into an integral of a rational function. We now apply the substitution $\theta=2\arctan t$ after the substitution $x=\tan\theta$:
\[
\begin{array}{rcll|l}
x&=&\displaystyle \tan \theta &&\text{use } \theta=2\arctan t \\
&=&\displaystyle \tan \left(2\arctan t\right)&&\displaystyle  \text{use\refBad{\ref{eqSinCosViaTan}}{}{ \eqref{eqSinCosViaTan}}: } \tan 2z=\frac{\sin (2z)}{\cos (2z)}=\frac{2\tan z}{1-\tan^2 z}\\
&=&\displaystyle \frac{2\tan (\arctan t)}{1-\tan^2 (\arctan t)}\\
&=&\displaystyle \frac{2t}{1-t^2}\quad .
\end{array}
\]
We can furthermore compute
\begin{equation}\label{eqsqrtx2plus1Euler1}
\begin{array}{rcll|l}
\displaystyle \sqrt{x^2+1}&=&\displaystyle  \sqrt{ \left( \frac{ 2t}{1-t^2}\right)^2+1 }\\
&=&\displaystyle \sqrt{\frac{4t^2+(1-t^2)^2}{(1-t^2)^2} }\\
&=&\displaystyle \sqrt{\frac{(1+t^2)^2}{(1-t^2)^2}} && \sqrt{(1-t^2)^2 } = 1-t^2  \text{ because } |t|<1\\
&=&\displaystyle \frac{1+t^2}{1-t^2}\\
&=&\displaystyle \frac{2-(1-t^2)}{1-t^2}\\
&=&\displaystyle -1+ \frac{2 }{1-t^2 } \quad .
\end{array}
\end{equation}
From $\displaystyle \sqrt{x^2+1}=-1+\frac{2}{1-t^2}$ and $\displaystyle x=\frac{2t}{1-t^2}$ we can express $t$ via $x$:
\[
\begin{array}{rcll|l}
\displaystyle \sqrt{x^2+1}&=&\displaystyle -1+ \frac{2}{ 1 - t^2 }\\
&=&\displaystyle -1+\frac{1}{t}\left(\frac{2t}{1 -t^2} \right) &&\displaystyle \text{use } x= \frac{2t}{1-t^2}\\
&=&\displaystyle -1+\frac{x}{t}\\
\displaystyle 1+\sqrt{x^2+1}&=&\displaystyle \frac{x}{t}\\
\displaystyle t&=&\displaystyle \frac{x}{1+\sqrt{x^2+1}}\\ &=& \displaystyle \frac{x}{1+\sqrt{x^2+1}} \left(\frac{ 1-\sqrt{x^2+1} }{1 - \sqrt{ x^2+1}}\right)\\
&=& \displaystyle  \frac{x(1- \sqrt{x^2 +1 } ) }{ 1- x^2-1}\\
&=&\displaystyle \frac{\sqrt{x^2+1}-1}{x}\quad .
\end{array}
\]
The differential $\diff x$ can expressed via $\diff t$ from $\displaystyle x=1+ \frac{2 }{t^2 -1}$. The Euler substitution  $x=\tan \theta=\tan (2\arctan t)$ can now be summarized as follows.
\begin{equation}\label{eqEulerSub-case1-tan(2arctant)}
\begin{array}{rcl}
x&=&\displaystyle \frac{2t}{1-t^2}\\
\displaystyle\sqrt{x^2+1}&=& \displaystyle -1+ \frac{2 }{1-t^2} \\
\displaystyle \diff x&=&\displaystyle \frac{2(1+t^2)} {(1-t^2)^2} \diff t\\
t &=&\displaystyle \frac{\sqrt{x^2+1}-1}{x}\quad .
\end{array}
\end{equation}
}

\begin{problem} 
Let the variables $x$ and $t$ be related via $\sqrt{x^2+1}=\frac{x}{t}-1$.
\begin{enumerate}
\item Express $x$ via $t$.
\item Express $\sqrt{x^2+1}$ via $t$ alone.
\item Express $\diff x$ via $t$ and $\diff t$.
\end{enumerate}
\end{problem} 

\subsection{Case 2: $\sqrt{1-x^2}$}
\subsubsection{$x=\cos \theta$}
\begin{problem} 
\begin{enumerate}[ref={\fcProblemRef}]
\item \label{problemTheoreticalTrigSubx=cost} Express $x, \diff x $ and $\sqrt{1-x^2 }$ via $\theta$ and $\diff \theta$ for the trigonometric substitution $x=\cos \theta $, $\theta\in \left[0, \pi\right]$.
\item \label{problemTheoreticalTrigSubx=cos(2arctant)} Express $x, \diff x $ and $\sqrt{1-x^2}$ via $t$ and $\diff t$ for the Euler substitution $x=\cos(2\Arctan t)$, $t\geq 0$. Express $t$ via $x$.
\end{enumerate}


\end{problem} 
\solution{\ref{problemTheoreticalTrigSubx=cost}
The trigonometric substitution $x=\cos \theta$ is given by
\[
\begin{array}{rcll|l}
\displaystyle \sqrt{-x^2+1}&=&\displaystyle \sqrt{1-\cos^2\theta}\\
&=&\displaystyle \sqrt{\sin^2\theta} &&\begin{array}{l} \text{when }\theta\in\left[0,\pi \right] \text{we have}\\
\sin \theta\geq 0 \text{ and so } \sqrt{\sin^2\theta}=\sin \theta
\end{array} \\
&=&\displaystyle \sin \theta\quad .
\end{array}
\]
The differential $\diff x$ can be expressed via $\diff \theta$ from $x=\cos \theta$. The substitution $x=\cos \theta $ can be now summarized as:
\begin{equation*}
\begin{array}{rcl}
x&=&\cos \theta\\
\sqrt{-x^2+1}&=&\sin \theta\\
\diff x&=& -\sin \theta \diff \theta\\
\theta&=&\arccos x \quad .
\end{array}
\end{equation*}

}
\solution{\ref{problemTheoreticalTrigSubx=cos(2arctant)}
We recall that the substitution $\theta = 2\arctan t$ transforms a trigonometric integral into an integral of a rational function. We now apply the substitution $2\arctan t$ after the substitution $x=\cos \theta$:

\begin{equation*}
\begin{array}{rcll|l}
x&=&\displaystyle \cos \theta &&\text{use } \theta=2\arctan t\\
&=&\displaystyle \cos (2\arctan t) &&\text{use }\refBad{\ref{eqSinCosViaTan}}{~}{\eqref{eqSinCosViaTan}:}  \displaystyle \cos (2z) = \frac{ 1-\tan^2 z }{1+ \tan^2 z}\\
&=&\displaystyle \frac{1-\tan^2(\arctan t)}{1+\tan^2( \arctan t)} \\
&=&\displaystyle \frac{1-t^2}{1+t^2} \quad .
\end{array}
\end{equation*}
We can furthermore compute
\begin{equation}\label{eqsqrt1minusxsquaredE2}
\begin{array}{rcll|l}
\sqrt{-x^2+1 }&=&\displaystyle \sqrt{1- \left(\frac{1-t^2}{1+t^2}\right)^2}\\
&=&\displaystyle \sqrt{\frac{(1+t^2)^2-(1-t^2)^2}{(1+t^2)^2} }\\
&=&\displaystyle \sqrt{\frac{4t^2}{(1+t^2)^2}} &&\displaystyle \sqrt{4t^2}=2t\text{ because } t\geq 0\\
&=&\displaystyle \frac{2t}{1+t^2}\quad .\\
\end{array}
\end{equation}
The differential $\diff x$ can be computed from $x=\frac{1-t^2}{1+t^2}$. Finally, we can express $t$ via $x$ with a little algebra:

\[
\begin{array}{rcll|l}
\displaystyle x&=&\displaystyle \frac{1-t^2}{1+t^2}\\
\displaystyle (1+t^2)x&=&\displaystyle 1-t^2\\
\displaystyle t^2(x+1)&=&\displaystyle 1-x\\
\displaystyle t^2&=&\displaystyle \frac{1-x}{1+x}\\
\displaystyle t&=&\displaystyle \sqrt{\frac{1-x}{1+x}}&& \text{here we use } t>0\\
\displaystyle t&=&\displaystyle \frac{\sqrt{1-x}}{\sqrt{ 1+x}} \frac{ \sqrt{1+x}}{\sqrt{1+x}} \\
\displaystyle t&=&\displaystyle \frac{\sqrt{-x^2+1}}{x+1}\quad .
\end{array}
\]
The Euler substitution $x= \cos (2\arctan t)$ can be now summarized as:
\[
\begin{array}{rcl}
x&=&\displaystyle \frac{1-t^2}{1+t^2}\\
\sqrt{-x^2+1}&=&\displaystyle \frac{2t}{1+t^2}  \\
\diff x&=&\displaystyle  -\frac{4 t}{(t^{2}+1)^{2}} \diff t\\
t&=&\displaystyle \frac{\sqrt{-x^2+1}}{x+1} \quad .
\end{array}
\]
}


\begin{problem} 
Let the variables $x$ and $t$ be related via $\sqrt{-x^2+1}= ( 1-x)t$.
\begin{enumerate}[ref={\fcProblemRef}]
\item \label{problemEulerSub-case1-cos(2arctant)-alternative-exposition-x-via-t} Express $x$ via $t$.
\item \label{problemEulerSub-case1-cos(2arctant)-alternative-exposition-radical-via-t} Express $\sqrt{-x^2+1}$ via $t$ alone.
\item Express $\diff x$ via $t$ and $\diff t$.
\end{enumerate}

\end{problem} 

\solution{\ref{problemEulerSub-case1-cos(2arctant)-alternative-exposition-x-via-t}.
\[
\begin{array}{rcll|l}
\sqrt{-x^2+1}&=&(1-x)t&&\text{square both sides}\\
(1-x)(1+x)&=&(1-x)^2t^2&&\text{divide by } (1-x)\\
1+x&=&(1-x)t^2\\
x(1+t^2)&=&t^2-1\\
x&=&\displaystyle \frac{t^2-1}{t^2+1}= 1-\frac{2}{t^2+1}\quad .
\end{array}
\]

}

\solution{\ref{problemEulerSub-case1-cos(2arctant)-alternative-exposition-radical-via-t}.

Using \ref{problemEulerSub-case1-cos(2arctant)-alternative-exposition-x-via-t} we get
\[
\sqrt{-x^2+1}=(1-x)t=\left(1-\left(1-\frac{2t}{t^2+1}\right)\right)t= \frac{ 2t}{ t^2+1}\quad .
\]

}





\subsubsection{$x=\sin \theta$}
\begin{problem} 
\begin{enumerate}[ref={\fcProblemRef}]
\item \label{problemTheoreticalTrigSubx=sint} Express $x, \diff x $ and $\sqrt{1-x^2 }$ via $\theta$ and $\diff \theta$ for the trigonometric substitution $x=\sin \theta $, $\theta\in \left[-\frac{\pi}{2}, \frac{\pi}{2}\right]$.
\item \label{problemTheoreticalTrigSubx=sin(2arctant)} Express $x, \diff x $ and $\sqrt{1-x^2}$ via $t$ and $\diff t$ for the Euler substitution $x=\sin(2\Arctan t)$, $t\in(-1,1)$. Express $t$ via $x$.
\end{enumerate}


\end{problem} 
\solution{\ref{problemTheoreticalTrigSubx=sint}
The trigonometric substitution $x=\sin \theta$ is given by
\[
\begin{array}{rcll|l}
\displaystyle \sqrt{-x^2+1}&=&\displaystyle \sqrt{1-\sin^2\theta}\\
&=&\displaystyle \sqrt{\cos^2\theta} &&\begin{array}{l} \text{when }\theta\in\left[-\frac{\pi}{2}, \frac{\pi}{2} \right] \text{we have}\\
\cos \theta\geq 0 \text{ and so } \sqrt{\cos^2\theta}=\cos \theta
\end{array} \\
&=&\displaystyle \cos \theta\quad .
\end{array}
\]
The differential $\diff x$ can be expressed via $\diff \theta$ from $x=\sin \theta$. The substitution $x=\sin \theta $ can be now summarized as:
\begin{equation*}
\begin{array}{rcl}
x&=&\sin \theta\\
\sqrt{-x^2+1}&=&\sin \theta\\
\diff x&=& \cos \theta \diff \theta\\
\theta&=&\arcsin x \quad .
\end{array}
\end{equation*}

}
\solution{\ref{problemTheoreticalTrigSubx=sin(2arctant)}
We recall that the substitution $\theta = 2\arctan t$ transforms a trigonometric integral into an integral of a rational function. We now apply the substitution $2\arctan t$ after the substitution $x=\sin \theta$:

\begin{equation*}
\begin{array}{rcll|l}
x&=&\displaystyle \sin \theta &&\text{use } \theta=2\arctan t\\
&=&\displaystyle \sin (2\arctan t) &&\text{use} \refBad{\ref{eqSinCosViaTan}}{~}{\eqref{eqSinCosViaTan}: } \displaystyle \sin (2z) = \frac{ 2 \tan z}{1+ \tan^2 z}\\
&=&\displaystyle \frac{2\tan(\arctan t)}{1+\tan^2( \arctan t)} \\
&=&\displaystyle \frac{2t}{1+t^2} \quad .
\end{array}
\end{equation*}
We can furthermore compute
\begin{equation}\label{eqsqrt1minusxsquaredE1}
\begin{array}{rcll|l}
\sqrt{-x^2+1 }&=&\displaystyle \sqrt{1- \left(\frac{2t}{1+t^2}\right)^2}\\
&=&\displaystyle \sqrt{\frac{(1+t^2)^2-4t^2}{(1+t^2)^2} }\\
&=&\displaystyle \sqrt{\frac{(1-t^2)^2}{(1+t^2)^2}} &&\displaystyle \sqrt{(1-t^2)^2}=1-t^2\text{ because } |t|\leq 1\\
&=&\displaystyle \frac{1-t^2}{1+t^2}\\
&=&\displaystyle \frac{2-(1+t^2) }{1+t^2}\\
&=&\displaystyle -1+\frac{2}{1+t^2}\quad .
\end{array}
\end{equation}
The differential $\diff x$ can be computed from $x=\frac{2t}{1+t^2}$. Finally, we can express $t$ via $x$ with a little algebra:

\[
\begin{array}{rcll|l}
\displaystyle \sqrt{-x^2+1}&=&\displaystyle -1 + \frac{2}{ 1+ t^2} \\
&=& \displaystyle  -1 +\frac{1 }{t} \left( \frac{2t}{1+t^2}\right) &&\text{use } x= \frac{2t}{1+t^2}\\
&=&\displaystyle -1 +\frac{x}{t} &&+1 \text{ to both sides}\\
\displaystyle \frac{x}{t}&=&1+\displaystyle \sqrt{-x^2+1}\\
t&=&\displaystyle \frac{x}{1+\sqrt{-x^2+1}}\\ &=&\displaystyle\frac{x}{(1+\sqrt{-x^2+1})} \frac{(1-\sqrt{- x^2+ 1})}{(1 - \sqrt{ -x^2+1})} \\
&=&\displaystyle \frac{ 1-\sqrt{-x^2+1}}{x} \quad .
\end{array}
\]

The Euler substitution $x= \sin (2\arctan t)$ can be now summarized as:

\[
\begin{array}{rcl}
x&=&\displaystyle \frac{2t}{1+t^2}\\
\sqrt{-x^2+1}&=&\displaystyle -1+\frac{2}{1+t^2}  \\
\diff x&=&\displaystyle  2\left(\frac{1-t^2 }{(1+ t^2)^2} \right) \diff t\\
t&=&\displaystyle \frac{ 1-\sqrt{-x^2+1}}{x} \quad .
\end{array}
\]
}


\begin{problem} 
Let the variables $x$ and $t$ be related via $\sqrt{-x^2+1}=1-xt$.
\begin{enumerate}
\item \label{problemEulerSub-case1-sin(2arctant)-alternative-exposition-x-via-t}  Express $x$ via $t$.
\item \label{problemEulerSub-case1-sin(2arctant)-alternative-exposition-radical-via-t} Express $\sqrt{-x^2+1}$ via $t$ alone.
\item Express $\diff x$ via $t$ and $\diff t$.
\end{enumerate}
\end{problem} 

\subsection{Case 3: $\sqrt{x^2-1}$}
\subsubsection{$x=\sec \theta$}
\begin{problem} 
\begin{enumerate}[ref={\fcProblemRef}]
\item \label{problemTheoreticalTrigSubx=sect} Express $x, \diff x $ and $\sqrt{x^2-1 }$ via $\theta$ and $\diff \theta$ for the trigonometric substitution $x=\csc \theta $, $\theta\in \left[0, \frac{\pi}{2}\right]\cup \left[\pi, \frac{3\pi}{2} \right) $.
\item \label{problemTheoreticalTrigSubx=sec(2arctant)} Express $x, \diff x $ and $\sqrt{1-x^2}$ via $t$ and $\diff t$ for the Euler substitution $x=\sec(2\Arctan t)$, $t\in (-\infty, -1)\cup[1,0)$. Express $t$ via $x$.
\end{enumerate}

\end{problem} 
\solution{\ref{problemTheoreticalTrigSubx=sect}
The trigonometric substitution $x=\sec \theta$ is given by
\[
\begin{array}{rcll|l}
\displaystyle \sqrt{x^2-1}&=&\displaystyle \sqrt{\sec^2\theta-1}=\sqrt{\frac{1}{\cos^2\theta}-1}\\
&=&\displaystyle \sqrt{\frac{\sin^2\theta}{\cos^2\theta}} =\sqrt{ \tan^2 \theta} &&\begin{array}{l} \text{when }\theta\in \theta \in \left[0, \frac{ \pi}{2 }\right)\cup \left[\pi, \frac{3\pi}{2}\right) \text{we have}\\
\tan \theta\geq 0 \text{ and so } \sqrt{\tan^2\theta}=\tan \theta
\end{array} \\
&=&\displaystyle \tan \theta\quad .
\end{array}
\]
The differential $\diff x$ can be expressed via $\diff \theta $ from $x=\sec \theta$. The substitution $x=\sec \theta $ can be now summarized as:
\begin{equation*}
\begin{array}{rcl}
\displaystyle x&=&\displaystyle \sec\theta= \frac{1}{\cos \theta}\\
\displaystyle \sqrt{x^2-1}& =&\displaystyle  \tan \theta\\
\displaystyle \diff x&=&\displaystyle \frac{\sin\theta}{ \cos^2\theta} \diff \theta= \sec\theta\tan\theta  \diff \theta\\
\displaystyle \theta&=&\Arcsec x \quad .
\end{array}
\end{equation*}

}
\solution{\ref{problemTheoreticalTrigSubx=sec(2arctant)}
We recall that the substitution $\theta = 2\arctan t$ transforms a trigonometric integral into an integral of a rational function. We now apply the substitution $2\arctan t$ after the substitution $x=\sec \theta$:

\begin{equation*}
\begin{array}{rcll|l}
x&=&\displaystyle \sec \theta=\frac{1}{\cos \theta} && \text{use } \theta =2 \arctan t\\
&=&\displaystyle \frac{1} {\cos(2\arctan t)} && \text{use\refBad{\ref{eqSinCosViaTan}}{ }{ \eqref{eqSinCosViaTan}: }} \displaystyle \cos (2z) = \frac{ 1- \tan^2 z}{1+ \tan^2 z}\\
&=& \displaystyle \frac{1+\tan^2(\arctan t)}{1-\tan^2( \arctan t)} \\
&=&\displaystyle \frac{1+t^2}{1-t^2}\\
&=&\displaystyle-1+\frac{2}{1-t^2} \quad .
\end{array}
\end{equation*}
We can furthermore compute
\begin{equation}\label{eqsqrtxsquareminus1E2}
\begin{array}{rcll|l}
\sqrt{x^2-1 }&=&\displaystyle \sqrt{ \left(\frac{1+t^2}{1-t^2}\right)^2-1}\\
&=& \displaystyle \sqrt{\frac{(1+t^2)^2-(1-t^2)^2}{(1-t^2)^2} }\\
&=& \displaystyle \sqrt{\frac{4t^2}{(1-t^2)^2}} && \begin{array}{l} \displaystyle t \text{ and }1-t^2\text{ have the same}\\ \text{sign for } t\in (-\infty, -1) \cup \left[0, 1 \right)\end{array} \\
&=&\displaystyle \frac{2t}{1-t^2}\quad .
\end{array}
\end{equation}
The differential $\diff x$ can be computed from $x=\frac{1+t^2}{1-t^2}$. Finally, we can express $t$ via $x$ with a little algebra:
\[
\begin{array}{rcll|l}
\displaystyle x&=&\displaystyle  \frac{1+t^2}{ 1- t^2} \\
\displaystyle (1- t^2)x&=&\displaystyle  1+t^2\\
\displaystyle (1+ x)t^2&=&\displaystyle  x-1\\
\displaystyle t^2&=&\displaystyle  \frac{x-1}{x+1}\\
\\
\displaystyle t&=&\displaystyle \doublebrace{ \sqrt{\frac{x-1}{x+1}}}{x>1}{-\sqrt{\frac{x-1}{x+1}}}{x<-1} &&\begin{array}{l} \text{because when } x<-1, \\ \text{ we have } t\in \left( -\infty , -1 \right]\end{array} \\
\displaystyle t&=&\displaystyle \doublebrace{ \frac{\sqrt{ x^2 -1}}{x+1}}{x>1}{-\frac{\sqrt{x^2-1}}{x+1}}{x<-1}  \quad .
\end{array}
\]
The Euler substitution $x= \sec (2\arctan t)$ can be now summarized as:

\[
\begin{array}{rcl}
x&=&\displaystyle \frac{1+t^2}{1-t^2}\\
\sqrt{x^2-1}&=&\displaystyle \frac{2t}{1-t^2}  \\
\diff x&=&\displaystyle  \frac{4 t}{(1- t^{2})^{2}} \diff t\\
t&=&\displaystyle \pm \frac{ \sqrt{x^2-1}}{x+1} \quad .
\end{array}
\]
}


\begin{problem} 
Let the variables $x$ and $t$ be related via $\sqrt{x^2-1}=(x+1)t$.
\begin{enumerate}[ref={\fcProblemRef}]
\item \label{problemEulerSub-case1-sec(2arctant)-alternative-exposition-x-via-t}  Express $x$ via $t$.
\item \label{problemEulerSub-case1-sec(2arctant)-alternative-exposition-radical-via-t} Express $\sqrt{x^2-1}$ via $t$ alone.
\item Express $\diff x$ via $t$ and $\diff t$.
\end{enumerate}

\end{problem} 

\solution{\ref{problemEulerSub-case1-sec(2arctant)-alternative-exposition-x-via-t}.

\[
\begin{array}{rcll|l}
\displaystyle \sqrt{x^2-1}&=&\displaystyle (x+1)t&&\text{square both sides}\\
\displaystyle (x-1)(x+1)&=&\displaystyle (x+1)^2t^2&&\text{divide by } (x+1)\\
\displaystyle x-1&=&\displaystyle (x+1)t^2\\
\displaystyle x(1-t^2)&=&\displaystyle 1+t^2\\
\displaystyle x&=& \displaystyle \frac{1+t^2}{1-t^2}= -1+\frac{2}{1-t^2}
\end{array}
\]
}
\solution{\ref{problemEulerSub-case1-sec(2arctant)-alternative-exposition-radical-via-t}. 

We use Problem \ref{problemEulerSub-case1-sec(2arctant)-alternative-exposition-x-via-t} to get
\[
\sqrt{x^2-1}=(x+1)t=\left(-1+\frac{2}{1-t^2}+1\right)t=\frac{2t}{1-t^2}
\]
}


\subsubsection{$x=\csc \theta$}
\begin{problem} 
\begin{enumerate}[ref={\fcProblemRef}]
\item \label{problemTheoreticalTrigSubx=csct} Express $x, \diff x $ and $\sqrt{1-x^2 }$ via $\theta$ and $\diff \theta$ for the trigonometric substitution $x=\csc \theta $, $\theta\in \left[0, \frac{\pi}{2}\right]\cup \left[\pi, \frac{3\pi}{2}\right)$.
\item \label{problemTheoreticalTrigSubx=csc(2arctant)} Express $x, \diff x $ and $\sqrt{1-x^2}$ via $t$ and $\diff t$ for the Euler substitution $x=\csc(2\Arctan t)$, $t\in(-\infty, -1)\cup [0,1)$. Express $t$ via $x$.
\end{enumerate}

\end{problem} 
\solution{\ref{problemTheoreticalTrigSubx=csct}
The trigonometric substitution $x=\csc \theta$ is given by
\[
\begin{array}{rcll|l}
\displaystyle \sqrt{x^2-1}&=&\displaystyle \sqrt{\frac{1}{\sin^2\theta}-1}\\
&=&\displaystyle \sqrt{\frac{\cos^2\theta}{\sin^2\theta}}=\sqrt{\cot^2 \theta} &&\begin{array}{l} \text{when }\theta\in \theta \in \left[0, \frac{\pi}{2}\right)\cup \left[\pi, \frac{3\pi}{2}\right) \text{we have}\\
\cot \theta\geq 0 \text{ and so } \sqrt{\cot^2\theta}=\tan \theta
\end{array}  \\
&=&\displaystyle \cot \theta\quad .
\end{array}
\]
The differential $\diff x$ can be expressed via $\diff \theta$ from $x=\csc \theta$. The substitution $x=\csc \theta $ can be now summarized as:
\begin{equation*}
\begin{array}{rcl}
x&=&\displaystyle \csc \theta\\
\sqrt{x^2-1}&=&\displaystyle \cot \theta\\
\diff x&=&\displaystyle  -\frac{\cos \theta }{\sin^2 \theta} \diff \theta = -\csc\theta \cot\theta \diff \theta\\
\theta&=&\displaystyle \Arccsc x \quad .
\end{array}
\end{equation*}

}
\solution{\ref{problemTheoreticalTrigSubx=csc(2arctant)}
We recall that the substitution $\theta = 2\arctan t$ transforms a trigonometric integral into an integral of a rational function. We now apply the substitution $2\arctan t$ after the substitution $x=\csc \theta$:
\begin{equation*}
\begin{array}{rcll|l}
x&=&\displaystyle \csc \theta =\frac{1}{\sin \theta} & & \text{use } \theta=2\arctan t\\
&=&\displaystyle \frac{1}{\sin (2\arctan t)} && \text{use\refBad{\ref{eqSinCosViaTan}}{ }{\eqref{eqSinCosViaTan}: }} \displaystyle \sin (2z) = \frac{ 2\tan z }{1+ \tan^2 z}\\
&=&\displaystyle \frac{1+\tan^2(\arctan t)}{2\tan( \arctan t)} \\
&=&\displaystyle \frac{1+t^2}{2t} \\
&=&\displaystyle \frac{1}{2}\left(\frac{1}t+t\right)\quad .
\end{array}
\end{equation*}
We can furthermore compute
\begin{equation}\label{eqsqrtxsquareminus1E1}
\begin{array}{rcll|l}
\sqrt{x^2-1 }&=&\displaystyle \sqrt{ \left( \frac{1+t^2}{ 2t } \right)^2 -1}\\
&=& \displaystyle \sqrt{\frac{(1+t^2)^2-4t^2 }{ 4t^2} }\\
&=&\displaystyle \sqrt{\frac{(1-t^2)^2}{4t^2}} &&\displaystyle \frac{1-t^2}{2t}>0\text{ when } t\in (-\infty, -1) \cup \left[0, 1 \right)  \\
&=&\displaystyle \frac{1-t^2}{2t}\\
&=&\displaystyle\frac{1}{2}\left(\frac{1}{t}-t\right)\quad .\\
\end{array}
\end{equation}
The differential $\diff x$ can be computed from $x=\frac{1}{2}\left(\frac{1}{t}-t\right)$. Finally, we can express $t$ via $x$ with a little algebra:

\[
\begin{array}{rcll|l}
\displaystyle \sqrt{x^2-1 }&=&\displaystyle \frac{1-t^2}{2t}\\
\displaystyle \sqrt{x^2-1 }&=&\displaystyle \frac{2-(1+t^2)}{2t} &&\displaystyle \text{use } x=\frac{1+t^2}{2t}\\
\displaystyle \sqrt{x^2-1}&=&\displaystyle \frac{1}{t}-x\\
\displaystyle \frac{1}{t}&=&\displaystyle \sqrt{x^2-1}+x\\
t&=& \displaystyle \frac{1}{\sqrt{x^2-1}+x}= \frac{1}{(\sqrt{x^2-1}+x)} \frac{ (-\sqrt{x^2-1}+x)} {(-\sqrt{x^2-1}+x)}\\
t&=&\displaystyle x-\sqrt{x^2-1}
\end{array}
\]
The Euler substitution $x= \cos (2\arctan t)$ can be now summarized as:
\[
\begin{array}{rcl}
x&=&\displaystyle \frac{1}{2}\left(\frac{1}{t}+t\right)\\
\sqrt{-x^2+1}&=&\displaystyle\frac12 \left(\frac{1}{t} - t \right) \\
\diff x&=&\displaystyle  -\frac{1}{2}\left(\frac{1}{t^2}+1\right) \diff t\\
t&=&\displaystyle x-\sqrt{x^2-1}\quad .
\end{array}
\]
}


\begin{problem} 
Let the variables $x$ and $t$ be related via $\sqrt{x^2-1}=\frac{1}{t}-x$.
\begin{enumerate}[ref={\fcProblemRef}]
\item Express $x$ via $t$.
\item Express $\sqrt{x^2-1}$ via $t$ alone.
\item Express $\diff x$ via $t$ and $\diff t$.
\end{enumerate}
\end{problem} 
\section{L'Hospital's rule}

\begin{problem}
Compute the limits. The answer key has not been fully proofread, use with caution.
\begin{multicols}{2}
\begin{enumerate}
\item $\displaystyle \lim\limits_{x\to 0} \frac{\sin x  }{x}$. 

\answer{$1$}
\item $\displaystyle \lim\limits_{x\to 0} \frac{x}{\ln (1+x)}$. 

\answer{$ 1$}
\item $\displaystyle \lim\limits_{x\to 0} \frac{x^2}{x-\ln (1+x)}$. 

\answer{$2$}
\item $\displaystyle \lim\limits_{x\to 0} \frac{x^2}{\sin x\ln (1+x)}$. 

\answer{$ 1$}
\item $\displaystyle \lim\limits_{x\to 0} \frac{\sin^2 x  }{\left(\ln (1+x)\right)^2}$.

\answer{$ 1$}
\item $\displaystyle \lim\limits_{x\to 0} \frac{\cos x- 1}{\sin x\ln (1+x)}$.

\answer{$- \frac{1}{2} $}
\item $\displaystyle \lim\limits_{x\to 0} \frac{\arctan x -x}{x^3} $.

\answer{$ -\frac{1}{3} $}
\item $\displaystyle \lim\limits_{x\to 0} \frac{\arcsin x -x}{x^3} $.

\answer{$ \frac{1}{6}$}
\item $\displaystyle \lim\limits_{x\to 1} \frac{x}{x-1}-\frac{1}{\ln x}$.

\answer{$ \frac{1}{2}$}
\item $\displaystyle \lim\limits_{x\to 0} \frac{\cos (nx) -\cos (mx)}{x^2 }$.

\answer{$\frac{m^2-n^2}2 $}
\item \label{eqProblemLimlixto0(arcsinx-x-x^3/6)/(sin^5 x)}  $\displaystyle \lim \limits_{x\to 0} \frac{\arcsin x-x-\frac{1}{6}x^3}{\sin^5 x} $. 

\answer{$\frac{3}{40}$}
\item \label{problemLHospital (sin (pi x) ln x )/ (cos pi x +1)}  $\displaystyle \lim\limits_{x\to 1} \frac{\sin \left(\pi x \right)\ln x }{\cos(\pi x)+1 } $.

\answer{$-\frac{2}{\pi}$}

\item $\displaystyle \lim\limits_{x\to 0} \frac{\sin x-x }{\arcsin x-x } $.

\answer{$ -1$}
\item \label{problemlim x to 0 (sin x - x)/(arctan x - x)} $\displaystyle \lim\limits_{x\to 0}\frac{\sin x- x}{\Arctan x -x}$.

\answer{$\frac{1}{2}$}

\item 
\label{problemlimxtoinftysin(2/x)}
$ {\displaystyle \lim_{x \to \infty} x \sin\left(\frac{2}{x}\right)}.$

\answer{$2$}

\end{enumerate}
\end{multicols}
\end{problem}

\begin{problem}
Find the limit.
\begin{multicols}{2}
\begin{enumerate}
\item $\displaystyle \lim\limits_{x\to 0} \frac{\sin x-x }{\arcsin x-x } $.
\answer{$ -1$}
\item \label{problemLHospital (sin (pi x) ln x )/ (cos pi x +1)}  $\displaystyle \lim\limits_{x\to 1} \frac{\sin \left(\pi x\right)\ln x }{\cos(\pi x)+1 } $.
\answer{$-\frac{2}{\pi}$}
\item \label{problemlim x to 0 (sin x - x)/(arctan x - x)} $\lim\limits_{x\to 0}\frac{\sin x- x}{\Arctan x -x}$.
\answer{$\frac{1}{2}$}

\end{enumerate}
\end{multicols}
\end{problem}
\solution{\ref{problemLHospital (sin (pi x) ln x )/ (cos pi x +1)}
The limit is of the form ``$\frac{0}{0}$'' so we are allowed to use L'Hospital's rule.
\[
\begin{array}{rcll|l}
\displaystyle
\lim\limits_{x\to 1} \frac{\sin \left(\pi x\right)\ln x }{\cos(\pi x)+1 } &=&\displaystyle \lim\limits_{x\to 1}  \frac{\left(\sin \left(\pi x\right)\ln x\right)' }{\left(\cos(\pi x)+1\right)' }\\
&=&\displaystyle \lim\limits_{x\to 1}  \frac{\left(\pi \cos \left(\pi x\right)\ln x+ \sin\left(\pi x\right)\frac{1}{x}\right) }{\left(-\pi\sin(\pi x)\right) } &&\text{type ``$\frac{0}{0}$'',  L'Hospital's rule}\\
&=&\displaystyle \lim\limits_{x\to 1}  \frac{\left(\pi \cos \left(\pi x\right)\ln x+ \sin\left(\pi x\right)\frac{1}{x}\right)' }{\left(-\pi\sin(\pi x)\right)' } \\
&=&\displaystyle \lim\limits_{x\to 1}  \frac{- \pi^{2} \sin{}\left(\pi x\right) \ln{}\left(x\right)+2 \pi \cos{}\left(\pi x\right) x^{-1}- \sin{}\left(\pi x\right) x^{-2} }{\left(-\pi^2\cos(\pi x)\right) } \\
&=&\displaystyle  \frac{- \pi^{2} \sin{}\left(\pi \right) \ln(1)+2 \pi \cos{}\left(\pi \right) - \sin{}\left(\pi \right)  }{\left(-\pi^2\cos(\pi )\right) } \\
&=&\displaystyle -\frac{2}{\pi}\quad .
\end{array}
\]
}
\solution{
\ref{problemlim x to 0 (sin x - x)/(arctan x - x)}
\noindent \textbf{Solution I.} 
\[
\begin{array}{rcll|l}
\displaystyle \lim\limits_{x\to 0}\frac{\sin x- x}{\Arctan x -x}&=&\displaystyle \lim\limits_{x\to 0} \frac{\cos x -1 }{ \frac{1}{1+x^2}-1 } &&\text{L'Hospital rule}\\
&=&\displaystyle \lim\limits_{x\to 0}\frac{-\sin x }{\frac{ -2x}{(1+x^2)^2} }&& \text{L'Hospital rule again}\\
&=&\displaystyle \lim\limits_{x\to 0} \frac{(1+x^2)^2}{2}\frac{\sin x}{x}  \\
&=&\displaystyle \lim\limits_{x\to 0} \frac{(1+x^2)^2}{2}\lim\limits_{x\to 0}\frac{\sin x}{x} \\
&=&\displaystyle \frac{1}{2}\quad .
\end{array}
\] 
}

\noindent \textbf{Solution II.}
\[
\begin{array}{rcll|l}
\displaystyle \lim\limits_{x\to 0}\frac{\sin x- x}{\Arctan x -x}&=&\displaystyle \lim\limits_{x\to 0} \frac{\left( x-\frac{x^3}{3!}+\frac{x^5}{5!}-\dots\right) -x}{\left( x-\frac{x^3}{3}+\frac{x^5}{5}-\dots\right)-x} &&\text{use the Maclaurin series of }\sin, \Arctan \\
&=&\displaystyle \lim\limits_{x\to 0}\frac{- \frac{x^3 }{6} + x^5\left(\frac{1}{5!}-\dots\right) }{- \frac{ x^3}{3} + x^5 \left(\frac{1}{5}-\dots \right)  }&& \begin{array}{rcl}
\text{The expressions in parenthesis }\\
\text{are continous functions in x }
\end{array}\\
&=&\displaystyle \lim\limits_{x\to 0} \frac{-\frac{1}{6}+ x^2 \left(\frac{1}{5!}-\dots\right) }{- \frac{1}{3} +x^2 \left( \frac{1}{5}-\dots \right)  }\\
&=&\displaystyle \frac{-\frac{1}{6}+0}{\frac{1}{3}+0}\\
&=&\displaystyle \frac{1}{2}\quad .
\end{array}
\] 

\solution{\ref{problemlimxtoinftysin(2/x)}.
\[
\begin{array}{rcll|l}
\displaystyle \lim_{x\to\infty }x\sin\left( \frac{2}{x} \right)& =&\displaystyle \lim_{x \to \infty} \frac{\sin \left(\frac{2}{x}\right) }{ \frac{ 1}{x}}
 &&\begin{array}{l}
\text{indeterminate form }\\
\text{Use L'Hospital's rule}
\end{array}\\
&=&\displaystyle \lim_{x \to \infty} \frac{\cos\left(\frac{2}{x}\right) \left (-\frac{2}{x^2}\right)}{-\frac{1}{x^2}} \\ 
&=&\displaystyle \lim_{x \to \infty} 2 \cos\left(\frac{x}{2}\right) \\
&=&\displaystyle 2\quad .
\end{array}
\]
}



\begin{problem}
Evaluate the limit, or show that it does not exist.

\begin{multicols}{2}
\begin{enumerate}
\item $\displaystyle \lim\limits_{x\to 0}\frac{x^2}{1-\cos x} $
\answer{$2$}
\item $\displaystyle \lim\limits_{x\to\infty}x \tan \left(\frac{1}{x}\right) $
\answer{$1$}
\item $\displaystyle \lim \limits_{x\to 0^+}x^{\sqrt{x}}$
\answer{$1$}
\end{enumerate}
\end{multicols}
\end{problem}
\section{Improper Integrals}
\begin{problem}
Determine whether the integral is convergent or divergent. Motivate your answer.
\begin{multicols}{2}
\begin{enumerate}[ref={\fcProblemRef}]
\item $\displaystyle \int\limits_{2}^{\infty}\frac{1}{(x-1)^{\frac32}} \diff x$.

\answer{convergent}
\item $\displaystyle \int\limits_{-1}^{1}\frac{1}{\sqrt[5]{1+x}} \diff x$.

\answer{convergent}
\item $\displaystyle \int\limits_{1}^{\infty }\frac{1}{\sqrt[5]{1+x}} \diff x$.

\answer{divergent}
\item $\displaystyle \int\limits_{-1}^{\infty }\frac{1}{\sqrt[5]{1+x}} \diff x$.

\answer{divergent}
\item $\displaystyle \int\limits_{-\infty }^{0}\frac{1}{2-3x} \diff x$.

\answer{divergent}
\item $\displaystyle \int\limits_{-\infty }^{0}\frac{1}{(2-3x)^{2}} \diff x$.

\answer{convergent}
\item $\displaystyle \int\limits_{-\infty }^{0}\frac{1}{(2-3x)^{1.00000001}} \diff x$.

\answer{convergent}
\item $\displaystyle \int\limits_{-2}^{ \frac{1}{ 2}} \frac{1}{2x-1} \diff x$.

\answer{divergent}

\item $\displaystyle \int\limits_{-1}^{\infty} e^{-3x} \diff x$.

\answer{convergent, equals $\frac{e^{3}}{3}$}

\item $\displaystyle \int\limits_{-\infty }^{5}  2^x \diff x$.

\answer{convergent}
\item $\displaystyle \int\limits_{-\infty }^{\infty}x^3 \diff x$.

\answer{divergent}
\item $\displaystyle \int\limits_{-\infty}^{\infty} x e^{-x^2} \diff x$.

\answer{convergent, equals $0$}

\item \label{problemConvergencesqrt(x)e^-sqrt(x)zerotoinfty} $\displaystyle \int\limits_{0}^{\infty} \sqrt{x} e^{-\sqrt{x}} \diff x$.

\answer{convergent, equals $4 $}

\item $\displaystyle \int\limits_{0}^{\infty}\sin^2 x \diff x$.

\answer{divergent}
\item $\displaystyle \int\limits_{0}^{5}\frac{1}{x^2+x-2} \diff x$.

\answer{divergent}
\item $\displaystyle \int\limits_{0}^{\infty}\frac{1}{x^2+x+1} \diff x$.

\answer{convergent}
\item $\displaystyle \int\limits_{2}^{\infty}\frac{1}{x^2-x-1} \diff x$.

\answer{convergent}
\item $\displaystyle \int\limits_{0}^{\infty}\frac{1}{x^2-x-1} \diff x$.

\answer{divergent}
\item \label{problemConvergencex^2/(x^4+2)from-inftyto+infty}

$\displaystyle \int\limits_{-\infty}^{\infty} \frac{x^2}{x^4+2} \diff x$.
\answer{convergent}

\item 
$\displaystyle \int\limits_{100}^{\infty} \frac{1}{x\ln x} \diff x$.

\answer{divergent}

\item $\displaystyle \int\limits_{100}^{\infty} \frac{1}{x(\ln x)^2} \diff x$.

\answer{convergent}
\item $\displaystyle \int\limits_{0}^{1}\ln x \diff x$.

\answer{convergent}
\item $\displaystyle \int\limits_{0}^{1}\frac{\ln x}{\sqrt{x}} \diff x$.

\answer{convergent}
\item $\displaystyle \int\limits_{0}^{2}x^3\ln x \diff x$.

\answer{convergent, equals $-1+4 \ln 2 $}

\item $\displaystyle \int\limits_{0}^{1} \frac{e^{\frac{1}{x}}}{x^2} \diff x$.

\answer{divergent}
\item $\displaystyle \int\limits_{-1}^{0} \frac{e^{\frac{1}{x}}}{x^2} \diff x$.

\answer{convergent}

\end{enumerate}
\end{multicols}

\end{problem}

\solution{\ref{problemConvergencesqrt(x)e^-sqrt(x)zerotoinfty}
It is possible to show that this integral is convergent by using the comparison theorem. However, we shall use direct integration instead. First, we solve the indefinite integral:

\[
\begin{array}{rcll|l}
\displaystyle \int \sqrt{x} e^{-\sqrt{x}} \diff x&=& \displaystyle \int \sqrt{x} e^{-\sqrt{x}} \frac{2\sqrt{x} \diff x}{2\sqrt{x}} &&\text{use }  \diff \sqrt{x} = \frac{\diff x}{2\sqrt{x}} \\
&=& \displaystyle \int \sqrt{x} e^{-\sqrt{x}} \left(2 \sqrt{x} \diff \sqrt{x}\right) &&\text{Set } \sqrt{x}=u\\
&=& \displaystyle 2\int u^2 e^{-u}\diff u\\
&=& \displaystyle 2\left( -\int u^2 \diff \left( e^{-u} \right) \right) &&\text{integrate by parts} \\
&=& \displaystyle 2\left(- u^2e^{-u}+\int e^{-u}\diff \left(u^2\right) \right) \\
&=&\displaystyle 2\left(- u^2e^{-u}+\int 2 u e^{-u}\diff u \right)\\
&=&\displaystyle 2\left(- u^2e^{-u}-\int 2 u \diff e^{ -u} \right) &&\text{integrate by parts again}\\
&=&\displaystyle 2\left(- u^2e^{-u}- 2 u  e^{-u}+ \int 2e^{-u}\diff u \right) \\
&=&\displaystyle 2\left( - u^2e^{-u} -2ue^{-u} -2e^{ -u} \right)+C\\
&=&\displaystyle 2\left( - xe^{-\sqrt{x}} -2\sqrt{x } e^{ -\sqrt{x}} -2e^{-\sqrt{x}}\right)+C
\end{array}
\]
Therefore 
\[
\begin{array}{rcll|l}
\displaystyle \int\limits_{0}^\infty \sqrt{x} e^{-\sqrt{x}} \diff x&=& \displaystyle \lim\limits_{t\to \infty} 2\left[ - xe^{- \sqrt{ x} } -2\sqrt{x}e^{-\sqrt{x}} -2e^{- \sqrt{x}} \right]_{ 0}^{\infty}\\
&=&\displaystyle 4+ \lim\limits_{t\to \infty} 4\left( -te^{ -\sqrt{t}} -\sqrt{t}e^{-\sqrt{t}}- e^{-\sqrt{t}} \right) && \text{Set }u=\sqrt{t}\\
&=&\displaystyle 4- 4\lim\limits_{u\to \infty} \left( u^2 e^{-u} + ue^{-u}+e^{-u}\right)\\
&=& \displaystyle 4- 4\lim\limits_{u\to \infty} \frac{ u^2 + u+1}{e^u} &&\text{use L'Hospital's rule for limit, see below}\\
&=& 4\quad ,
\end{array}
\]
and the integral converges to $4$. In the above computation we used the following limit computation
\[
\begin{array}{rcll|l}
\displaystyle \lim\limits_{u\to \infty}\frac{u^2+u+1}{e^u}&=&\displaystyle  \lim\limits_{u\to \infty} \frac{2u+1}{e^u}&&\text{Apply L'Hospital's rule}\\
&=&\displaystyle \lim\limits_{u\to \infty} \frac{2}{e^u}\\
&=&\displaystyle 0\quad .
\end{array}
\]

}

\solution{\ref{problemConvergencex^2/(x^4+2)from-inftyto+infty}
The integrand is a rational function and therefore we can solve this problem by finding the indefinite integral and then computing the limit. We would need to start by factoring $x^4+2$ into irreducible quadratic factors - that is already quite laborious:
\[
x^4+2= \left(x^2+\sqrt[4]{8}x+\sqrt{2} \right)\left(x^2-\sqrt[4]{8}x+\sqrt{2}\right)\quad.
\]
The problem asks us only to establish the convergence of the integral; it does not ask us to compute its actual numerical value. Therefore we can give a much simpler solution. The function is even and therefore it suffices to establish whether $\displaystyle \int\limits_0^\infty \frac{x^2}{x^4+2}\diff x$ is convergent.

We have that 
\[
\int\limits_{0}^\infty \frac{x^2}{x^4+2}\diff x= \int\limits_{0}^1 \frac{x^2}{ x^4 +2 }\diff x+\int\limits_{1}^\infty \frac{x^2}{x^4+2}\diff x\quad.
\]
The function $ \frac{x^2}{x^4+2}$ is continuous so $\int\limits_{0}^1 \frac{x^2}{x^4+2}\diff x$ integrates to a number, which does not affect the convergence of the above expression. Therefore the convergence of our integral is governed by the convergence of $\int\limits_{1}^\infty \frac{x^2}{x^4+2}\diff x$. To establish that that integral is convergent, we use the comparison theorem as follows.
\[
\begin{array}{rcll|l}
\displaystyle
\int\limits_{1}^\infty \frac{x^2}{x^4+2}\diff x&\leq & \displaystyle \int \limits_{ 1}^\infty \frac{x^2}{x^4}\diff x && \begin{array}{l} \text{we have that } x^4+2>x^4\\ \text{and therefore }\displaystyle \frac{x^2}{x^4+2}\leq \frac{x^2}{x^4}\end{array}\\
&=&\displaystyle \int_1^{\infty} x^{-2}\diff x\\
&=&\displaystyle \lim\limits_{t\to \infty} \left[ -\frac{1}{x}\right]_{1}^{t}\\
&=&\displaystyle  \lim\limits_{t\to\infty} 1- \frac{1}{t}\\
&=&\displaystyle  1\quad .
\end{array}
\]
In this way we showed $ \displaystyle\int_{1}^\infty \frac{x^2}{x^4+2} \diff x \leq 1$. Therefore, as $\displaystyle \frac{x^2 }{ x^4 + 2}\geq 0 $ is positive, we can apply the comparison theorem to get that $\displaystyle\int_{1}^\infty \frac{ x^2}{x^4+2} \diff x $ is convergent.
}


\begin{problem}
Determine whether the integral is convergent or divergent. Motivate your answer. The answer key has not been proofread, use with caution.

\begin{multicols}{2}
\begin{enumerate}
\item $\displaystyle \int\limits_{100}^{\infty} \frac{1}{x\ln x} \diff x$.
\answer{divergent}
\item $\displaystyle \int\limits_{100}^{\infty} \frac{1}{x(\ln x)^2} \diff x$.
\answer{convergent}
\item $\displaystyle \int\limits_{0}^{1}\ln x \diff x$.
\answer{convergent}
\item $\displaystyle \int\limits_{0}^{1}\frac{\ln x}{\sqrt{x}} \diff x$.
\answer{convergent}
\item $\displaystyle \int\limits_{0}^{2}x^3\ln x \diff x$.
\answer{convergent}
\item $\displaystyle \int\limits_{0}^{1} \frac{e^{\frac{1}{x}}}{x^2} \diff x$.
\answer{divergent}
\item $\displaystyle \int\limits_{-1}^{0} \frac{e^{\frac{1}{x}}}{x^2} \diff x$.
\answer{convergent}
\item $\displaystyle \int \limits_{0}^{\infty}\sin x^2\diff x$ (This problem is more difficult and may require knowledge of sequences to solve).
\answer{convergent}
\end{enumerate}
\end{multicols}

\end{problem}
\begin{problem}
Determine if the integral is convergent or divergent. If it is convergent, compute the value of the integral.
\begin{multicols}{2}
\begin{enumerate}
\item $\displaystyle \int\limits_{1}^{2} \frac{x}{\sqrt{x^2-1}}\diff x$ 
\answer{$\sqrt{3}$}

\item $\displaystyle \int\limits_{0}^{1} x^2\ln x\diff x$ \label{problemImproperIntegral x^2ln x dx}
\answer{$- \frac{1}{9} $}

\item \label{problemImproperIntegral e^(-sqrt x)/sqrt(x)dx}
$\displaystyle \int\limits_{0}^{\infty} \frac{e^{-\sqrt{x}}}{ \sqrt{x}} \diff x $
\answer{ $2 $} 

\item \label{problemImproperIntegral1/(x ln x)dx} $\displaystyle \int\limits_{100}^{\infty} \frac{1}{x\ln x}\diff x $
\answer{$\infty$- the integral is divergent} 

\item $\displaystyle \int\limits_{0}^{1} \frac{1}{x\ln x}\diff x $
\answer{$-\infty$- the integral is divergent} 

\item \label{problemImproperIntegral(1+e^(-x))/(xlnx)dx}
$\displaystyle \int\limits_{100}^{\infty} \frac{1+e^{-x}}{x\ln x }\diff x $ 
\answer{$\infty$-the integral is divergent}
\end{enumerate}
\end{multicols}
\end{problem}
\solution{\ref{problemImproperIntegral x^2ln x dx}
\[
\begin{array}{rcll|l}
\displaystyle \int \limits_{0}^1 x^2\ln x \diff x&=& \displaystyle \int \limits_{0}^1 \ln x \diff \left( \frac{ x^3}{ 3} \right) &&\text{Integrate by parts}\\ 
&=&\displaystyle \left[\frac{x^3}{3}\ln x\right]_{0}^1-\int \frac{x^3 }{3} \diff (\ln x)\\
&=&\displaystyle  \left[\frac{x^3}{3}\ln x\right]_{0}^1- \int \limits_{0 }^1 \frac{x^2}{3} \diff x\\
&=&\displaystyle \left[\frac{x^3}{3}\ln x- \frac{ x^3}{9} \right]_{0}^1\\
&=&\displaystyle  \frac{1}{3} \ln 1 - \frac{1}{9}- \left(\lim\limits_{ x \to 0} \frac{x^3\ln x}{3}-0 \right)\\
&=&\displaystyle -\frac{1}{9} -\lim \limits_{x\to 0} \frac{x^3\ln x}{3}\\
&=&\displaystyle -\frac{1}{9} -\lim \limits_{x\to 0} \frac{\ln x}{ \frac{3 }{ x^3}} &&\text {Use L'Hospital's rule} \\
&=&\displaystyle -\frac{1}{9} - \lim \limits_{x\to 0} \frac{ \frac{1 }{x}}{- \frac{9 }{ x^4}} = -\frac{1}{9} -\lim \limits_{x\to 0}\frac{x^3}{-9}\\
&=&\displaystyle -\frac{1}{9}\quad .
\end{array}
\]

}



\solution{\ref{problemImproperIntegral e^(-sqrt x)/sqrt(x)dx}
\[
\begin{array}{rcl}
\displaystyle \int\limits_{0}^{\infty} \frac{e^{\sqrt{x}}}{\sqrt{x}} \diff x &=& \displaystyle  2\int\limits_{x=0}^{\infty} e^{-\sqrt{x}}\diff \sqrt{x}\\
&=&\displaystyle \left[-2e^{-\sqrt{x}}\right]_{x=0}^{\infty}\\
&=&\displaystyle \lim\limits_{t\to \infty} -2e^{-\sqrt{x}}- \left(-2e^{-\sqrt{0}}\right)\\
&=& 2\quad .
\end{array}
\]

}

\solution{\ref{problemImproperIntegral1/(x ln x)dx}
\[
\begin{array}{rcl}
\displaystyle \int\limits_{100}^{\infty} \frac{1}{x \ln x}\diff x&=&\displaystyle  \int\limits_{x=100}^{\infty} \frac{1}{\ln x}\diff (\ln x)\\
&=&\displaystyle \int\limits_{x=100}^{\infty} \diff \left(\ln (\ln x)\right)\\~\\
&=&\displaystyle \left[ \ln (\ln x)\right]^{\infty}_{100}\\~\\
&=&\displaystyle \lim\limits_{t\to \infty} \ln (\ln t)- \ln (\ln 100)\\~\\
&=&\displaystyle \infty\quad .
\end{array}
\]
The integral diverges to $\infty$.
}
\solution{\ref{problemImproperIntegral(1+e^(-x))/(xlnx)dx}
\[
\int\limits_{100}^\infty \frac{1+e^{-x}}{x\ln x}\diff x >\int \limits_{100}^{\infty}\frac{1}{x\ln x}\diff x \stackrel{\text{Problem }\ref{problemImproperIntegral1/(x ln x)dx}}{=} \infty\quad .
\]
Therefore by the comparison test, our integral diverges to $\infty$.
}

\begin{problem}
Determine if the integral is convergent or divergent. If it is convergent, compute the value of the integral.

\begin{enumerate}
\item $\displaystyle \int\limits_{1}^\infty \frac{x^2}{x^3+1}\diff x$
\answer{$\infty$- the integral diverges.}
\item $\displaystyle \int \limits_{1}^{\infty} \frac{1 }{ x^2+1}\diff x$
\answer{$\frac{\pi }{4}$}
\item  $\displaystyle\int\limits_{6}^8\frac{4}{(x-6)^3}\diff x$
\answer{$\infty$- the integral diverges.}
\end{enumerate}
\end{problem}

\section{Sequences}
\subsection{Understanding sequence notation}
\begin{problem}
Give a simple sequence formula that matches the pattern below. 

\begin{multicols}{2}
\begin{enumerate}
\item $\displaystyle \left(1, \frac{1}{3}, \frac{1}{5}, \frac{1}{7},\frac{1}{9},\dots \right)$.

\answer{$a_n=\frac{1}{2n-1}$}
\item $\displaystyle \left(-1, \frac{1}{5}, -\frac{1}{25}, \frac{1}{125},-\frac{1}{625}, \frac{1}{3125}\dots \right)$

\answer{$a_n=-\left(-\frac{1}{5}\right)^{n-1}$}
\item $\displaystyle \left(-5, 2, -\frac{4}{5}, \frac{8}{25}, -\frac{16}{125}, \frac{32}{625},\dots \right)$

\answer{$a_n=-5\left(-\frac{2}{5}\right)^{n-1}$}

\item $\displaystyle \left(4, 7, 10, 13, 16, 19,\dots\right)$

\answer{$a_n=3 {{n}}+1 $}

\item $\left(-2, \frac{3}{4}, -\frac{4}{9}, \frac{5}{16}, -\frac{6}{25}, \frac{7}{36}, \dots \right)$

\answer{$a_n=(-1)^{n}\left(\frac{n +1}{n^{2}}\right)$}

\item $\left(0,-1, 0, 1,0,-1, 0, 1,0,-1, 0, 1,\dots \right)$

\answer{$a_n=\cos\left(n\frac{\pi}{2}\right)$}
\end{enumerate}
\end{multicols}
\end{problem}
\begin{problem}
List the first 5 elements of the sequence. 
\begin{multicols}{2}
\begin{enumerate}
\item $\displaystyle a_{n+1}=\frac{1}{2}\left(a_n+ \frac{3}{a_n}\right)$, $a_1=1$.
\item $\displaystyle a_n=a_{n-1}+a_{n-2}$, $a_1=1$, $a_2=1$.
\item $\displaystyle a_n= \frac{\left(\frac{1}{2}-n\right)}{n} a_{n-1} $, $a_0=1$.
\item $\displaystyle a_n= a_{n-1}+2n+1$, $a_0=1$.
\item $\displaystyle a_n:=\frac{1}{n} a_{n-1}$, $a_1=1$.
\end{enumerate}
\end{multicols}
\end{problem}
\begin{problem}
List the first 4 elements of the sequence. 
\begin{multicols}{2}
\begin{enumerate}
\item $\displaystyle a_n= \frac{(-1)^n}{n}$.

\answer{$(a_1, a_2, a_3, a_4, a_5)=\left(-1, \frac12, -\frac13, \frac14\right)$ }
\item $\displaystyle a_n=\frac{1}{n!}$.

\answer{$(a_1, a_2, a_3, a_4, a_5)=\left(1, \frac{1}{2}, \frac{1}{6}, \frac{1}{24}\right)$ }
\item $\displaystyle a_n=\cos (\pi n)$.

\answer{$(a_1, a_2, a_3, a_4, a_5)=(-1, 1, -1, 1)$ }
\item $\displaystyle a_n=\frac{(-1)^n}{2n+1}$.

\answer{$(a_1, a_2, a_3, a_4, a_5)=\left(-\frac{1}{3}, \frac{1}{5}, -\frac{1}{7}, \frac{1}{9}\right)$ }
\item $\displaystyle a_n=\frac{\sqrt{5}}{5}\left( \left(\frac{1+\sqrt{5 }}{2} \right)^n- \left(\frac{1-\sqrt{5}}{2}\right)^n\right) $

\answer{$(a_1, a_2, a_3, a_4, a_5)=(1, 1, 2, 3)$ }
\end{enumerate}
\end{multicols}
\end{problem}

\subsection{Convergence}
\begin{problem}
Determine if the sequence is convergent or divergent. If convergent, find the limit of the sequence.
\begin{multicols}{2}
\begin{enumerate}
\item $\displaystyle a_n=n$.
\answer{divergent}
\item $\displaystyle a_n=2^n$.
\answer{divergent}
\item $\displaystyle a_n=1.0001^n$.
\answer{divergent}
\item $\displaystyle a_n=0.999999^n$.
\answer{convergent, $\lim_{n\to \infty} a_n=0$}
\item $\displaystyle a_n=n-\sqrt{n+1}\sqrt{n+2}$
\answer{convergent, $\lim_{n\to \infty} a_n=-\frac{3}{2}$}
\item $\displaystyle a_n=\frac{\ln n}{n}$.
\answer{convergent, $\lim_{n\to \infty} a_n=0$}
\item $\displaystyle a_n=\frac{\ln n}{\sqrt[10]{n}}$.
\answer{convergent, $\lim_{n\to \infty} a_n=0$}
\item $\displaystyle a_n=\frac{1}{n}$.
\answer{convergent, $\lim_{n\to \infty} a_n=0$}
\item $\displaystyle a_n=\frac{1}{n!}$.
\answer{convergent, $\lim_{n\to \infty} a_n=0$}
\item $\displaystyle a_n=\frac{n^n}{n!}$.
\answer{divergent}
\item $\displaystyle a_n=\cos n$.
\answer{divergent}
\item $\displaystyle a_n=\cos\left(\frac{1}{n}\right)$
\answer{convergent, $\lim_{n\to \infty} a_n=1$}

\item $\displaystyle a_n= \left(\frac{n+1}{n}\right)^{n}$.
\answer{convergent, $\lim_{n\to \infty} a_n=e$}
\item $\displaystyle a_n= \left(\frac{2n+1}{n}\right)^{n}$.
\answer{divergent}

\item $\displaystyle a_n= \left(\frac{n+1}{n}\right)^{2n}$.
\answer{convergent, $\lim_{n\to \infty} a_n=e^2$}

\item $\displaystyle a_n= \left(\frac{n+1}{2n}\right)^{n}$.
\answer{convergent, $\lim_{n\to \infty} a_n=0$}
\end{enumerate}
\end{multicols}
\end{problem}
\begin{problem}
Find the limit of the sequence or prove that the sequence is divergent.
\begin{enumerate}
\item $\displaystyle a_n=\left(\frac{n}{n-1}\right)^{2n}$.
\answer{convergent, $\lim\limits_{n\to \infty} a_n = e^2$}
\item $\displaystyle a_n=\frac{n!}{n^n}$.
\answer{convergent, $\lim\limits_{n\to \infty} a_n=0$}
\end{enumerate}

\end{problem}
\section{Series}
\subsection{Some explicit series summations}
\subsubsection{Geometric series}
\begin{problem}
Express the infinite decimal number as a rational number.
\begin{multicols}{2}
\begin{enumerate}
\item $1.\overline{6}=1.6666\dots$

\answer{ $\frac{5}{3} $}
\item $1.\overline{3}=1.3333\dots$

\answer{ $\frac{4}{3} $}
\item $2.\overline{16}=2.16161616\dots$

\answer{$\frac{214}{99}$ }
\item $2014.\overline{2014}=2014.2014201420142014\dots$

\answer{$\frac{20140000}{9999}$}
\end{enumerate}
\end{multicols}
\end{problem}
\solution{\ref{problemExpressAsRational2014.20142014...}

\[
\begin{array}{rcl}
\displaystyle 2014.201420142014\dots &=&\displaystyle  2014 + \frac{2014}{10^4} + \frac{2014}{10^8} + \dots \\
&=&\displaystyle  2014 +\frac{2014}{10000} \left(1+\frac{1}{10000}+\dots +\frac{1}{10^{4n}}+\dots \right)\\
&=&\displaystyle  2014 +\frac{2014}{10000} \left(\frac{1}{1-\frac{1}{10^4}}\right)\\
&=&\displaystyle  2014 +\frac{2014}{10000}\cdot\frac{10000}{9999}\\
&=&\displaystyle  2014 +\frac{2014}{9999} \\
&=&\displaystyle  \frac{2014\cdot 9999+2014}{9999}\\
&=&\displaystyle  \frac{2014\cdot 10000}{9999}\\
&=&\displaystyle  \frac{20140000}{9999}
\end{array}
\]
Our answer cannot be reduced any further as the greatest common divisor of $20140000$ and $9999$ is $1$.

}

\begin{problem}
Express the sum of the series as a rational number.
\begin{multicols}{2}
\begin{enumerate}[ref={\fcProblemRef}]
\item 
\label{problemSum(2^n+3^n)/(5^n)}
$
\displaystyle \sum\limits_{n=1}^{\infty} \frac{2^n+3^n}{5^n}
$

\answer{$\frac{13}{6}$}

\item \label{problemsumn=0^infty(2^n+5^n)/10^n}
$\displaystyle\sum_{n=0}^{\infty} \frac{2^n+5^n}{10^n}$

\answer{$\frac{13}{4}$}
 
\item \label{problemSum(3^n+5^n)/(7^n)}
$\displaystyle
\sum\limits_{n=1}^{\infty} \frac{5^n-3^n}{7^n}
$

\answer{$\frac{7}{4}$}
\item \label{sum_n=1^infty(3^(n+1)+7^(n-1))/21^n}
$\displaystyle
\sum_{n=1}^\infty \frac{3^{n+1}+7^{n-1}}{21^n}
$
\answer{$ \frac{4}{7}$}

\item \label{sum_n=0^infty(2^(n+1)+(-3)^(n-1))/5^n}
$\displaystyle
\sum_{n=0}^\infty \frac{2^{n+1}+(-3)^{n-1}}{5^n}
$

\answer{$ \frac{25}{8} $}
 
\end{enumerate}
\end{multicols}
\end{problem}
\solution{\ref{problemSum(2^n+3^n)/(5^n)}.

\[
\begin{array}{rcll|l}
\displaystyle \sum\limits_{n=1}^{\infty} \frac{2^n+3^n}{5^n}&=&\displaystyle \sum\limits_{n=1}^{\infty} \left(\frac{2}{5}\right)^n
+\sum\limits_{n=1}^{\infty} \left(\frac{3}{5}\right)^n\\
&=&\displaystyle  \frac{2}{5}\sum\limits_{n=0}^{\infty} \left(\frac{2}{5} \right)^n+\frac{3}{5} \sum\limits_{n=0}^{ \infty} \left(\frac{3}{5}\right)^n&&
\begin{array}{l}
\text{Use geometric series sum f-la: }\\
\sum\limits_{n=0}^{\infty}r^n=\frac{1}{1-r},\\
\text{provided } |r|<1
\end{array}\\
&=&\displaystyle  \frac{2}{5}\cdot  \frac{1}{\left(1-\frac{2 }{5} \right)} +\frac{3}{5}\cdot  \frac{1}{ \left(1- \frac{3 }{5} \right)}\\
&=&\displaystyle \frac{13}{6}\quad .
\end{array}
\]
}

\solution{\ref{problemsumn=0^infty(2^n+5^n)/10^n}.
\[
\begin{array}{rcll|l}
\displaystyle \sum\limits_{n=0}^{\infty}\frac{2^n+5^n}{10^n}&=&\displaystyle  \sum \limits_{ n=0}^{\infty}\left(\frac{1}{5^n}+\frac{1}{2^n}\right)&&\text{use } \sum\limits_{ n= 0}^{\infty} r^n=\frac{1}{1-r}, \text{ for } |r|<1\\
&=&\displaystyle \frac{1}{1-\frac{1}{2}} +\frac{1}{1-\frac{1}{5}}\\
&=&\displaystyle \frac{13}{4}\quad .
\end{array}
\]
} 
\solution{\ref{sum_n=1^infty(3^(n+1)+7^(n-1))/21^n}.
\[
\begin{array}{rcll|l}
\displaystyle \sum\limits_{n=1}^{\infty} \frac{3^{n+1}+ 7^{ n-1}}{21^n} &=& \displaystyle \sum \limits_{n=1}^{ \infty}\left(3 \cdot \frac{3^{n}}{21^n}+ \frac{ 1 }{7}\cdot \frac{ 7^n }{21^n}\right)\\
&=&\displaystyle 3\sum_{n =1}^{\infty} \left(\frac{1}{7} \right)^n + \frac{1 }{7} \sum_{n=1}^{\infty}  \left(\frac{1}{3} \right)^n\\
&=&\displaystyle \frac{3}{7} \sum_{n=0}^{\infty} \left( \frac{1}{7 } \right)^n +\frac{1 }{21} \sum_{ n=0}^{ \infty}\left(\frac{1}{3 } \right) ^n &&\text{use }\sum_{n= 0 }^\infty r^n=\frac{1}{1-r}, |r|<1\\
&=&\displaystyle \frac{3}{7}\cdot  \frac{1}{ \left(1 -\frac{ 1}{7} \right)}+ \frac{ 1}{21}\cdot  \frac{1 }{ \left(1-\frac{1 }{ 3} \right)}\\
&=&\displaystyle \frac{4}{7}\quad .
\end{array}
\]
} 
\solution{\ref{sum_n=0^infty(2^(n+1)+(-3)^(n-1))/5^n}.
\[
\begin{array}{rcll|l}
\displaystyle \sum\limits_{n=0}^{\infty} \frac{2^{n+1}+ (-3)^{ n-1}}{5^n} &=& \displaystyle \sum \limits_{n=0}^{ \infty} \left(2\cdot \frac{2^{n}}{5^n}-\frac{ 1 }{3}\cdot  \frac{ (-3)^n }{5^n}\right)\\
&=&\displaystyle 2 \sum_{n=0}^{ \infty}\left(\frac{2}{5 } \right)^n -\frac{1}{3} \sum_{ n=0}^{\infty}\left(-\frac{3}{5 } \right) ^n &&\text{use }\sum_{n= 0 }^\infty r^n=\frac{1}{1-r}, |r|<1\\
&=&\displaystyle 2\cdot  \frac{1}{ \left(1 -\frac{ 2}{5} \right)}- \frac{ 1}{3}\cdot  \frac{1 }{ \left(1-\left(-\frac{3}{ 5}\right) \right)} \\
&=&\displaystyle \frac{25}{8} \quad .
\end{array}
\]

} 
\subsubsection{Telescoping series}
\begin{problem}
Use partial fractions to split the summand of each sum into two. Sum the telescoping series (a sum is ``telescoping'' if it can be broken into summands so that consecutive terms cancel). The answer key has not been proofread, use with caution.
\begin{enumerate}[ref={\fcProblemRef}]
\item $\displaystyle \sum\limits_{n=0}^{\infty} \frac{-6}{9 n^{2}+3 n-2}$\quad.
\answer{$ 2$}
\item \label{probelmSum_n=3^infty 3/(n^2-3n+2) }  $\displaystyle \sum \limits_{n=3}^{\infty} \frac{3}{n^{2}-3 n+2} $\quad.
\answer{$3$}
\end{enumerate}
\end{problem}
\solution{
\ref{probelmSum_n=3^infty 3/(n^2-3n+2) } 
%Solution contributed by student Kreg Hanning
\[
\begin{array}{rcll|l}
\displaystyle \sum_{n=3}^{\infty} \frac{3}{n^2-3n+2} &=&\displaystyle  \sum_{n=3}^{\infty}\left( \frac{3}{n-2} - \frac{3}{n-1} \right)  &&\text{using partial fractions}\\
&=&\displaystyle  3\sum_{n=3}^{\infty} \left( \frac{1}{n-2} - \frac{1}{n-1} \right)\\
&=&\displaystyle 3\left( \underset{n=3}{\left(1-\frac{1}{2}\right)} + \underset{n=4}{\left(\frac{1}{2}-\frac{1}{3}\right)} + \underset{n=5}{\left(\frac{1}{3}-\frac{1}{4}\right)} + \dots\right)\\~\\
&=&\displaystyle 3\lim\limits_{n\rightarrow\infty} \left( 1 - \frac{1}{n-1} \right) = 3\quad .
\end{array}
\]
}
\begin{problem}
Use partial fractions to sum the telescoping series (a sum is ``telescoping'' if it can be broken into summands so that consecutive terms cancel).
\begin{multicols}{2}
\begin{enumerate}
\item $\displaystyle \sum\limits_{x=1}^\infty \frac{1}{x^{2}+x}$

\answer{$1$}
\item $\displaystyle \sum\limits_{x=2}^\infty\frac{2 x+1}{x^{4}+2 x^{3}- x^{2}-2 x}$

\answer{$\frac{1}{3}$}

\item $\displaystyle \sum\limits_{x=1}^\infty \frac{2 x}{x^{4}-3 x^{2}+1}$

\answer{$-1$}

\item $\displaystyle \sum\limits_{x=1}^\infty \frac{x^{2}+x+2}{ x^{4}- 5  x^{2}+4}$
\answer{$-\frac{1}{2}$}

\end{enumerate}
\end{multicols}
\end{problem}
\solution{\ref{problemsum_(n=3)^(infty)(n^2+n+2)/(n^4-5n^2+4)}

The partial fractions decomposition algorithm shows that 
\[
\frac{n^{2}+n +2}{n^{4}-5n^{2}+4}=\frac{1}{3} \left( \frac{2}{n -2}-\frac{2}{n -1}+\frac{1}{n +1}-\frac{1}{n +2}\right)\quad .
\]
We omit the details of the partial fraction decomposition as it is quite laborious, but otherwise straightforward. Therefore 
\[
\begin{array}{rc@{}ll|l}
\displaystyle \sum\limits_{n=3}^{\infty}\frac{n^{2}+n +2}{n^{4}-5n^{2}+4}&=&\displaystyle \phantom{+} \frac{1}{3} \sum\limits_{n=3}^{\infty} \left( \frac{2}{n -2}-\frac{2}{n -1}+\frac{1}{n +1}-\frac{1}{n +2}\right)\\
&=&\displaystyle\phantom{+}  \frac{2}{3} \sum\limits_{ n=3}^{ \infty} \left( \frac{1}{n -2}-\frac{1}{n -1}\right)\\
&&\displaystyle +\frac{1}{3}\sum\limits_{n=3}^{\infty} \left(\frac{1}{n +1}-\frac{1}{n +2}\right)\\
&=& \displaystyle\phantom{+} \frac{2}{3} \left( \left(1- \cancel{ \frac{1}{2}} \right) + \left(\cancel{\frac{1}{2}}-\cancel{ \frac{2}{3}}\right)+\cancel{\dots} +\left(\cancel{\frac{1}{n-2}}-\frac{1}{n-1}\right)+\dots  \right)\\
&&\displaystyle+\frac{1}{3}\left( \left(\frac{1}{4} - \cancel{ \frac{1}{5}} \right) +\left(\cancel{ \frac{1}{5}}- \cancel{ \frac{1}{6}} \right)+\cancel{\dots} +\left( \cancel{ \frac{1 }{n+1}} -\frac{1}{n+2} \right)+\dots \right)\\
&=&\displaystyle \phantom{+}\lim\limits_{n\to \infty}  \frac{2}{3}\left(1-\frac{1}{n-1} \right)+\lim\limits_{n\to \infty}  \frac{1}{3}\left( \frac{1}{4} - \frac{1}{n+2} \right)\\
&=&\displaystyle \phantom{+}\frac{2}{3}+ \frac{1}{3} \cdot \frac{1}{4}\\
&=&\displaystyle \phantom{+}\frac{3}{4}


\quad .

\end{array}
\]


}


\subsection{Series convergence tests}
\subsubsection{Basic tests}
\begin{problem}
Find whether the series is convergent or divergent using an
appropriate test. Some of the problems require the alternating series test. The test states the following.

\medskip

{\textbf{Alternating series test.} Suppose $b_n \searrow 0$. Then $\sum (-1)^n b_n$ is convergent.} 

\medskip

Here, $b_n\searrow 0$ means the following.
\begin{itemize}
\item The sequence of numbers $b_n$ is decreasing.
\item The sequence decreases to $0$, that is, 
\[\lim\limits_{n\to \infty} b_n=0\quad .
\]
\end{itemize}
%
\begin{multicols}{2}
\begin{enumerate}[ref={\fcProblemRef}]
\item \label{problemConvergencesumn=1^infty(-1)^nlnn} $\displaystyle\sum_{n=1}^{\infty} (-1)^n\ln n  . $

\answer{diverges, basic divergence test}
\item \label{problemConvergencesumn=2^infty(-1)^n/lnn} $\displaystyle \sum_{n=2}^{\infty} \frac{(-1)^n }{\ln n}  .$

\answer{converges, alternating series test}
\item \label{problemConvergencesum_n=2^infty(-1)^nn/ln(n)} $\displaystyle \sum\limits_{n=2}^{\infty}\frac{n}{\ln n}$

\answer{diverges, basic divergence test}
\item \label{problemConvergencesum_n=2^infty(-1)^nln(n)/n} $\displaystyle \sum\limits_{n=2}^{\infty}\frac{\ln n}{n}$

\answer{converges, alternating series test}
\end{enumerate}
\end{multicols}
\end{problem}
\solution{\ref{problemConvergencesumn=1^infty(-1)^nlnn}. $\lim\limits_{n\to\infty }(-1)^n \ln n$ does not exist and therefore the sum is not convergent.
}

\noindent \solution{\ref{problemConvergencesumn=2^infty(-1)^n/lnn}. For $n>2$, we have that $\ln n$ is a positive increasing function and therefore $\frac{1}{\ln n}$ is a decreasing positive function. Furthermore $\displaystyle \lim_{n\to \infty}\frac{1}{\ln n} =0 $. Therefore the series is convergent by the alternating series test.
}


\subsubsection{Integral and comparison tests}

\begin{problem}
Use the integral test, the comparison test or the limit comparison test to determine whether the series is convergent or divergent. Justify your answer.
\begin{multicols}{2}
\begin{enumerate}[ref={\fcProblemRef}]
\item $\displaystyle \sum\limits_{n=1}^{\infty} \frac{1}{2n+1}$.

\answer{divergent}

\item $\displaystyle \sum\limits_{n=1}^{\infty} \frac{1}{2n^2+n^3}$.

\answer{convergent, compare to $\sum\limits_{n=1}^{\infty} \frac{1}{2n^2}$}

\item $\displaystyle \sum\limits_{n=1}^{\infty}\frac{n^2+3}{3n^5+n}$

\answer{convergent, can use limit comparison test}
\item $\displaystyle \sum\limits_{n=0}^{\infty} \frac{1}{3^n+5}$.

\answer{convergent, compare to $\sum_{n=0}^{\infty} \frac{1}{3^n}$}

\item \label{problemConvergencesum_2^infty1/(xlnx)dx}
$\displaystyle \sum_{n=2}^\infty \frac{1}{n \ln n}$

\answer{divergent, integral test}
\item  \label{problemConvergencesum_2^infty1/((2n+1)ln(n)}
$\displaystyle \sum\limits_{n=2}^{\infty} \frac{1}{(2n+1)\ln (n)}$.

\answer{divergent}
\item $\displaystyle \sum\limits_{n=2}^{\infty}\frac{1}{n(\ln n)^2}$

\answer{convergent, can use integral test}
\item 
$\displaystyle \sum\limits_{n=2}^{\infty} \frac{1}{(2n+1)(\ln (n))^2}$.

\answer{convergent}
\item 
Determine all values of $p$, $q$ $r$ for which the series 
\[
\displaystyle \sum_{n=30}^{\infty} \frac{1}{n^p(\ln n)^q(\ln (\ln n))^r}
\]
is convergent.


\end{enumerate}
\end{multicols}


\end{problem}
\solution{\ref{problemConvergencesum_2^infty1/(xlnx)dx}.
\[
\begin{array}{rcl}
\displaystyle \int_{2}^{\infty}\frac{1}{x\ln x}\diff x&=&\displaystyle \lim_{t\to\infty} \int_{2}^{t}\frac{1}{x\ln x}\diff x\\
&=&\displaystyle \lim_{t\to\infty} \int_{2}^{t}\frac{1}{\ln x}\diff(\ln x)\\
&=&\displaystyle \lim_{t\to\infty} \int_{2}^{t}\diff(\ln(\ln x))\\
&=&\displaystyle \lim_{t\to \infty}\left[ \ln(\ln x)\right]_{x=2}^{x=t}\\
&=&\displaystyle \lim_{t\to \infty}\left(\ln(\ln t)-\ln (\ln 2)\right)\\
&=&\displaystyle \infty,
\end{array}
\]
therefore the integral is divergent (and diverges to $+\infty$).

The function $\frac{1}{x\ln x}$ is decreasing, as for $x>2$, it is the quotient of $1$ by increasing positive functions. $\frac{ 1}{ x\ln x}$ tends to $0$ as $x\to \infty$, and therefore the integral criterion implies that $\sum\limits_{2}^{ \infty} \frac{1 }{n \ln n}$ is divergent.
}

\subsubsection{Root, ratio tests}
\begin{problem}
Establish whether the series is convergent or divergent. Use the ratio or root tests. Show all your work. The answer key has not been proofread, use with caution.
\begin{enumerate}
\item $\displaystyle \sum\limits_{n=1}^{\infty}\left(\frac{n+1 }{4n}\right)^n$
\answer{convergent, easier to see with root test}
\item $\displaystyle \sum\limits_{n=1}^{\infty}\left(\frac{4n+1 }{n}\right)^n$
\answer{divergent, easier to see with root test}
\item $\displaystyle \sum\limits_{n=1}^{\infty} \frac{n^n }{4^n n!}$
\answer{convergent, use ratio test}
\item $\displaystyle \sum\limits_{n=1}^{\infty} \frac{(4n)^n }{ n!}$
\answer{divergent, use ratio test}
\end{enumerate}
\end{problem}
\solution{\ref{problemConvergenceSum_n=1^infty(4n)^n/nfactorial}[Contributed by student Bingjie Wu]
In order to establish the convergence of 
\[ 
\sum_{n=1}^{\infty }\frac{(4n)^{n}}{n!}\quad ,
\]
we shall use the ratio test. We recall that the ratio test states that if $\displaystyle \lim\limits_{n\to \infty} \frac{a_n}{a_{n+1 }} $ exists and is equal to $L$, then if $L>1$ the series is divergent and if $L<1$ the series is convergent (if $L=1$ the test is inconclusive). 

We compute:
\[
\begin{array}{rcl}
\displaystyle\lim\limits_{n\to \infty} \frac{a_{n+1}}{a_{n}}&=&\displaystyle \lim \limits_{ n \to\infty} \left|\frac{(4n+4)^{n+1}n !}{ (n+ 1)! ( 4n )^{ n}} \right|\\
&=&\displaystyle
\lim\limits_{n\to\infty} \left| \frac{ (4n+4)(4n+4 )^{n}}{(n+1)(4n)^{n}}\right|\\
&=&\displaystyle \left(\lim\limits_{n\to\infty} \frac{4n+4}{n+1} \right) \left(\lim\limits_{n\to \infty} \left( \frac{n+1}{n}\right)^{n}\right)=4e> 1\quad,
\end{array}
\]
and therefore the series is divergent.
}

\begin{problem}
Except for $x=\pm e$, use the ratio test to determine all real values of $x$ for which 
\[
\sum_{n=0}^{\infty}x^n\frac{n!}{n^n}
\]

You are expected to use in your solution the already studied fact that 
\[
\lim_{x\to 0}\left(1+\frac{x}{n}\right)^n=e^x\quad .
\]
\end{problem}
\subsection{Problems collection, all techniques}
\begin{problem}
Determine if the series converges or diverges. Present a detailed motivation for your answer. 

\begin{enumerate}
\item $\displaystyle\sum\limits_{n=1}^\infty \frac{(2n+1)^n}{n^{2n}}  $
\answer{converges, root test}
\item $\displaystyle\sum\limits_{n=1}^\infty \frac{1}{n\sqrt{n^2+1}}  $
\answer{converges, comparison test}
\item $\displaystyle\sum\limits_{n=1}^\infty \frac{(-1)^n \sqrt{n}}{n+5}  $
\answer{converges, alternating series test}
\item $\displaystyle\sum\limits_{n=1}^\infty \frac{3n^2+4}{10n^2+1} $
\answer{diverges, summands do not tend to $0$}
\item $\displaystyle\sum\limits_{n=1}^\infty \frac{(n!)^2}{(n+1)!} $
\answer{diverges, ratio test, alternatively, summands tend to $\infty$}
\item $\displaystyle\sum\limits_{n=1}^\infty \frac{1}{e^{n^2}} $
\answer{converges, comparison test}

\end{enumerate}
\end{problem}

\section{Power series, Taylor and Maclaurin series}
\subsection{Interval of convergence}
\begin{problem}
Determine the interval of convergence for the following power series. 
\begin{enumerate}[ref={\fcProblemRef}]
\item \label{problemIntervalConvergence_sum(x-2)^n/(3sqrt(n+1))} $\displaystyle \sum_{ n=1}^{ \infty} \frac{(x-2)^n}{3\sqrt{n+1}}.$

\answer{$x\in [1, 3)$.}

\item $\displaystyle \sum \limits_{n= 1}^{\infty} \frac{ 10^nx^n}{n^3}$.

\answer{$x\in \left[-\frac{1}{10}, \frac{1}{10} \right] $.}
\item $\displaystyle \sum \limits_{n= 1}^{\infty} \frac{ 10^n(x-1)^n}{n^3}$.

\answer{$x\in [0.9, 1.1]$.}

\item $\displaystyle \sum\limits_{n=0}^{\infty}(-1)^n \frac{(x+1)^n }{ 2n+1} $.

\answer{$x\in (-2, 0]$.}
\item $\displaystyle \sum\limits_{n=0}^{\infty}(-1)^n \frac{(x- 3)^n }{ 2n+1} $.

\answer{$x\in (2, 4]$.}
\item $\displaystyle \sum\limits_{n=0}^{\infty} \frac{x^n}{n!}$.

\answer{converges for all $x$.}
\item $\displaystyle \sum\limits_{n=0}^{\infty} (n+1)x^n $.

\answer{converges for $|x|<1$.}

\item $\displaystyle \sum\limits_{n=1}^{\infty} \frac{x^n}{n}$.

\answer{converges for $x\in[-1,1)$.}
\item $\displaystyle\sum \limits_{n=1}^{\infty} (-1)^n\frac{x^{2n+1}}{2n+1}$.

\answer{converges for $x\in [-1, 1]$.}
\item $\displaystyle \sum\limits_{n=1}^{\infty} \binom{\frac{1}{2}}{n}x^{n}$, where we recall that the binomial coefficient $\displaystyle \binom{q}{n}$ stands for $\displaystyle\frac{q (q-1)\dots (q-n+1)}{n!}$.

\answer{converges for $x\in (-1,1]$. } 
\end{enumerate}
\end{problem}
\solution{\ref{problemIntervalConvergence_sum(x-2)^n/(3sqrt(n+1))}.
We apply the Ratio Test to get that $\lim\limits_{n\to \infty }\left| \frac{ a_{n+1} }{a_n }\right| = |x-2|$. Therefore the power series converges at least in the interval $x\in (1, 3)$. When $x = 3$, the series becomes $\sum\limits_{n=1}^{\infty} \frac{1}{3\sqrt{n+1}}$, which diverges - this can be seen, for example, by comparing to the $p$-series $\frac{1}{\sqrt{n}}$. When $x = 1$, the series becomes $\sum\limits_{n=1}^{\infty} \frac{ (-1 )^n }{ 3 \sqrt{n+1}}$, which converges by the Alternating Series Test. Our final answer $x\in [1, 3)$.
}

\subsection{Taylor, Maclaurin series}
\begin{problem}
\begin{enumerate}
\item Find the Maclaurin series for $xe^{x^3}$.

\answer{$\displaystyle \sum\limits_{n=0}^{\infty} \frac{ x^{3n+1}}{ n!}$ }

\item Use your series to find the Maclaurin series of $\displaystyle\int x e^{x^3}\diff x$ 

\answer{ 
\begin{tabular}{l}
$\displaystyle C+\sum\limits_{n=0}^{\infty} \frac{ x^{3n+2}}{ (3n+2) n!}$\\
note the integral \\
can't be integrated with elementary\\
functions.
\end{tabular}
}

\end{enumerate}

\end{problem}
\begin{problem}
For each of the items below, do the following.
\begin{itemize}
\item Find the Maclaurin series of the function (i.e., the power series representation of the function around $a=0$).
\item Find the radius of convergence of the series you found in the preceding point. You are not asked to find the entire interval of convergence, but just the radius.
\end{itemize}
\begin{multicols}{2}
\begin{enumerate}
\item $e^x$.

\answer{ $\displaystyle  \sum\limits_{n=0}^\infty \frac{x^n}{n!} $  }
\item $x e^{-2x}$.

\answer{ $
\begin{array}{l}
\sum\limits_{n=0}^{\infty} (-1)^n 2^{n} x^{n+1}= \sum\limits_{n=1}^{\infty} (-1)^{n-1} 2^{n-1} x^{n} \\
\text{converges for }x\in \left(  \right) 
\end{array}
$
}

\item $e^{2x}$.

\answer{ $\displaystyle  e^{2x}=\sum\limits_{n=0}^\infty \frac{2^nx^n}{n!} $  }

\item $e^{x^2}$.

\answer{ $\displaystyle e^{x^2}= \sum\limits_{n=0}^\infty \frac{x^{2n}}{n!} $  }
\item $e^{-3x^2}$.

\answer{ $\displaystyle  e^{-3x^2}=\sum\limits_{n=0}^\infty \frac{(-1)^n 3^nx^{2n}}{n!} $  }

\item $x^2e^{2x}$.

\answer{ $\displaystyle  x^{2} e^{2x}=\sum\limits_{n=0}^\infty \frac{ 2^n x^{n+2}}{n!} $  }

\item $\sin x$.

\answer{ $\displaystyle  \sin x=\sum\limits_{n=0}^\infty (-1)^n \frac{x^{2n+1}}{(2n+1)!} $  }
\item $\cos x$.

\answer{ $\displaystyle  \cos x=\sum\limits_{n=0}^\infty (-1)^n \frac{x^{2n}}{(2n)!} $  }
\item $\sin (2x)$.

\answer{ $\displaystyle  \sin (2x)=\sum\limits_{n=0}^\infty (-1)^n \frac{2^{2n+1} x^{2n+1}}{(2n+1)!} $  }
\item $\cos (2x)$.

\answer{ $\displaystyle  \cos (2x) =\sum\limits_{n=0}^\infty (-1)^n 2^{2n} \frac{x^{2n}}{(2n)!} $  }
\item $\cos^2 (x)$.

\answer{ $\displaystyle  \cos^2 x= \frac{1}{2} + \sum\limits_{ n=0}^\infty (-1)^n 2^{2n-1} \frac{x^{2n}}{(2n)!} $  }

\item $x\sin x$.

\answer{ $\displaystyle  x\sin x=\sum\limits_{n=0}^\infty (-1)^n \frac{x^{2n+2}}{(2n+1)!} $  }

\end{enumerate}
\end{multicols}
\end{problem}
\begin{problem}
For each of the items below, do the following.
\begin{itemize}
\item Find the Maclaurin series of the function (i.e., the power series representation of the function around $a=0$).
\item Find the radius of convergence of the series you found in the preceding point. 
\end{itemize}
\begin{multicols}{2}
\begin{enumerate}[ref={\fcProblemRef}]
\item $\displaystyle \frac{1}{3-x}$.

\answer{$
\begin{array}{l}
\displaystyle \sum\limits_{n=0}^{\infty} \frac{x^n}{3^{n+1}}\\
R=3 \\
\text{converges for }x\in \left(-3,3 \right) 
\end{array}
$}
\item $\displaystyle \frac{1}{3-2x}$.

\answer{$
\begin{array}{l}
\displaystyle \sum\limits_{n=0}^{\infty} \frac{ 2^n}{3^{n+1}}x^n \\
R=\frac{3}{2}\\
\displaystyle \text{converges for }x\in \left(-\frac{3}{2},\frac{3}{2} \right) 
\end{array}
$}

\item $\displaystyle \frac{1}{2x+3}$.

\answer{ $\begin{array}{l}
\displaystyle \frac{1}{3}\left(1- \frac{2x}{3} +\left(\frac{2 x}{ 3} \right)^2 -\left(\frac{2x}{3}\right)^3+\dots\right) \\
\displaystyle = \sum\limits_{0}^{ \infty} \frac{(-1)^n}{3}\left(\frac{2}{3}\right)^n x^n  \\
R=\frac{3}{2}\\
\displaystyle \text{converges for }x\in \left(-\frac{3}{2}, \frac{3}{2} \right) 
 \end{array}
 $}

\item $\displaystyle \frac{1}{1+x^2}$.

\answer{$\begin{array}{l} 
\displaystyle \sum\limits_{n=0}^{\infty } (-1)^nx^{2n} \\
R=1\\
\text{converges for }x\in \left(-1,1 \right) 
\end{array}
$ }
\item $\displaystyle \frac{1}{1-2x^2}$.

\answer{ $ \begin{array}{l} 
\displaystyle \sum\limits_{n=0}^{\infty } 2^n x^{2n} \\
\displaystyle R=\frac{1}{2}\\
\displaystyle \text{converges for }x\in \left(-\frac{1}{2}, \frac{1}{2}\right)
\end{array}
$ }
\item \label{problemMaclaurin(1/(x^2-1))} $\displaystyle \frac{1}{x^2-1}$. 

\answer{$\begin{array}{l}
\displaystyle - \sum\limits_{n=0}^\infty x^{2n}\\
\displaystyle R=1\\
\text{converges for } x\in (-1,1)
\end{array}
$}
\item $\displaystyle\frac{\frac12}{x-1}-\frac{\frac12}{x+1}$.

\answer{same as \ref{problemMaclaurin(1/(x^2-1))}}
\item \label{problemMaclaurin(1/(1-x)^2)} $\displaystyle \frac{1}{(1-x)^2}$.

\answer{ $
\begin{array}{l}
1+2x+3x^2+4x^3+\dots\\
=\sum\limits_{n=0}^{\infty} (n+1) x^n \\
R=1\\
\text{converges for }x\in \left(-1,1 \right) 
\end{array}
$}

\item $\displaystyle \frac{1}{(1-x)^3}$.

\answer{ $
\begin{array}{l}
\frac{1}{2}\left( 2+6x+12x+\dots+n(n-1)x^{n-2}+\dots \right) \\
=\sum\limits_{n=0}^{\infty} \frac{(n+1)(n+2)}{2} x^n \\
R=1\\
\text{converges for }x\in \left(-1, 1  \right) 
\end{array}
$
}
\item $\displaystyle\ln (1+x)$.

\answer{ $
\begin{array}{l}
\displaystyle \sum\limits_{n=1}^\infty (-1)^{n+1} \frac{x^n}{n} \\
R=1\\
\text{converges for } x\in \left(-1,1\right]
\end{array}
$}
\item \label{problemMaclaurinSeriesln(1-x)} $\ln (1-x)$.

\answer{ $
\begin{array}{l}
\displaystyle -\sum\limits_{n=1}^\infty \frac{x^n}{n} \\
R=1\\
\text{converges for } x\in \left[-1,1\right)
\end{array}
$}

\item $\ln (1-3x)$.

\answer{ $
\begin{array}{l}
\displaystyle -\sum\limits_{n=0}^\infty \frac{3^n x^n}{n} \\
\displaystyle R=\frac{1}{3}\\
\displaystyle \text{converges for }x\in\left( - \frac{ 1}{ 3}, \frac{1}{3} \right] 
\end{array}
$}

\item $\ln (1-3x^2)$.

\answer{ $\begin{array}{l}
\displaystyle -\sum\limits_{n=1}^\infty \frac{3^nx^{2n }}{n} \\
\displaystyle R= \frac{1}{\sqrt{3}}\\
\displaystyle \text{converges for }x\in \left(- \frac{ 1}{ \sqrt{3}}, \frac{1}{\sqrt{3}}\right)
\end{array}
$}

\item \label{problemMaclaurin(ln(3-2x^2))} $\ln (3-2x^2)$.

\answer{ $\begin{array}{l}
\displaystyle \ln 3-\sum\limits_{n=1}^\infty \left( \frac{ 2}{3}\right)^n \frac{x^{2n }}{n} \\
\displaystyle R=\sqrt{\frac{2}{3}}\\
\displaystyle \text{converges for }x\in \left(-\sqrt{\frac{2}{3}}, \sqrt{\frac{2}{3}} \right)
\end{array}
$}

\item $x\ln (3-2x^2)$.

\answer{ $
\begin{array}{l}
\displaystyle x\ln 3-\sum\limits_{n=1}^\infty \left( \frac{ 2 }{3}\right)^n \frac{x^{2n+1 }}{n} \\
\displaystyle R=\sqrt{\frac{2}{3}}\\
\displaystyle \text{converges for }x\in \left(-\sqrt{\frac{2}{3}}, \sqrt{\frac{2}{3}} \right)
\end{array}
$}

\item $\displaystyle\arctan x$.

\answer{$
\begin{array}{l}
\displaystyle \sum\limits_{n=0}^\infty (-1)^n \frac{x^{2 n+1 }}{2n+1}\\
R=1\\
\text{converges for } x\in\left(-1 ,1 \right]
\end{array}
$ }
\item $\displaystyle\arctan (2x)$.

\answer{$
\begin{array}{l}
\displaystyle \sum\limits_{n=0}^\infty \frac{(-1)^n 2^{ 2n +1} x^{2n+1}}{2n+1}\\
\displaystyle R=\frac{1}{2}\\
\displaystyle \text{converges for }x\in \left( -\frac{ 1}{ 2}, \frac{1}{2} \right] 
\end{array}
$ }
\item $\displaystyle\arctan \left(2x^2\right)$.

\answer{$
\begin{array}{l}
\displaystyle \sum\limits_{n=0}^\infty (-1)^n \frac{ 2^{ 2n +1} x^{4n+2}}{2n+1}\\
\displaystyle R=\frac{1}{\sqrt{2}}\\
\displaystyle \text{converges for } x \in \left( -\frac{ 1}{ \sqrt{2}}, \frac{1}{\sqrt{2}} \right] 
\end{array}
$ }

\end{enumerate}
\end{multicols}
\end{problem}
\solution{\ref{problemMaclaurin(1/(1-x)^2)}
\[
\begin{array}{rcll|l}
\displaystyle \frac{1}{1-x}&=&\displaystyle \frac{\diff }{\diff x}\left(1+x+x^2+x^3+\dots  \right)&&
\begin{array}{l}
\text{geometric series,}\\
\text{converges if and only if}\\
|x|<1
\end{array}
\\
\displaystyle \frac{\diff }{\diff x}\left(\frac{1}{1-x} \right)&=& \displaystyle \frac{\diff }{\diff x}\left(1+x+x^2+x^3+\dots  \right)&&\text{apply }\frac{\diff }{\diff x}\\
\displaystyle  -\frac{(1-x)'}{(1-x)^2}= \frac{1}{(1-x)^2}&=& \displaystyle 1+2x+3x^2+\dots \\
\displaystyle \frac{1}{(1-x)^2} &=&\displaystyle \sum\limits_{n=0}^\infty (n+1)x^n&& \text{rewrite in } \sum 
\text{ notation.}
\end{array}
\]
The radius of convergence of the geometric series is $1$. Differentiating does not change the radius of convergence. We have that the radius of convergence of $1+x+x^2+\dots$ is $1$ and therefore we have that $\frac{1}{ (1-x)^2}= \sum \limits_{ n=0}^ \infty (n+1)x^n$ converges for $|x|<1$ and the radius of convergence is $R=1$.

The problem does not ask us to determine the interval of convergence, however let us do it for exercise. The endpoints of the interval of convergence are $-1$ and $1$. The series is divergent for both of them: indeed at $x=-1$ the series becomes $\sum\limits_{n=0}^\infty (-1)^n(n+1)x^n $ and at $x=1$ the series becomes $\sum\limits_{n=0}^\infty (n+1)x^n $. Both of these series are divergent as their terms do not tend to zero as $n$ tends to infinity. Thus the interval of convergence is $(-1,1)$.

\refBad{\ref{problemMaclaurin 1/(1-x)^k}}{}{We generalize this problem in Problem \ref{problemMaclaurin 1/(1-x)^k}.}

}

\solution{\ref{problemMaclaurinSeriesln(1-x)}
\[
\begin{array}{rcll|l}
\displaystyle \frac{\diff }{\diff x} \left( \ln(1 -x) \right) &=&\displaystyle  \frac{-1}{1-x}&&\begin{array}{@{}l} \text{expand as geometric series}\\\text{for }|x|< 1\end{array}\\
&=&\displaystyle  -\left(1+x+x^2+x^3+\dots\right) &&\text{Integrate indefinitely, } |x|< 1 \\
\displaystyle 
\int 
\frac{\diff }{\diff x}(\ln(1-x))\diff x  &=& \displaystyle -\int\left(1+x+x^2+x^3+\dots \right)\diff x  && \begin{array}{@{}l} \text{For power series, }\\
\text{integral of infinite}\\
\text{sum equals}\\
\text{infinite sum of integrals} \\
\text{inside the convergence radius}
\end{array}
\\
\ln(1-x)&=&\displaystyle -\left(x+\frac{x^2}{2}+\frac{x^3}{3} +\dots \right)+C&&\text{To find }C \text{ set }x=0\\~\\
0=\ln 1&=&-0+C=C\\~\\
\ln (1-x)&=&-\left(x+\frac{x^2}{2}+\frac{x^3}{3}+\dots \right)\\
&=&\displaystyle - \sum_{n=1}^{\infty}\frac{x^n}{n} \quad .
\end{array}
\]
The radius of convergence of the geometric series $1+x+x^2+\dots$ is $1$. Since the series for $\ln (1-x)$ is obtained from the geometric series via integration, its radius of convergence is again $1$. 

We note that the interval of convergence for the series  $-\sum\limits_{n=1}^\infty \frac{x^n}{n}$ is $\left[-1, 1 \right)$ - the series is convergent at $x=-1$ by the alternating series test and divergent at $x=1$ (at $x=1$ the series is minus the harmonic series). This shows that integration of power series can change convergence at the endpoints of the interval of convergence.  
}

\solution{\ref{problemMaclaurin(ln(3-2x^2))}. 
We solve this problem by reducing it to Problem \ref{problemMaclaurinSeriesln(1-x)}, which asserts the power series expansion $\displaystyle \ln (1-y) = -\sum\limits_{n=1}^\infty \frac{y^n}{n}$ for $|y|<1$.
\[
\begin{array}{rcll|l}
\ln \left(3-2x^2\right)&=& \displaystyle \ln \left(3 \left(1 -\frac{2}{3} x^2\right)\right)\\
&=&\displaystyle \ln 3+ \ln  \left(1 -\frac{2}{3} x^2\right)&&\text{Set } y=\frac{2}{3}x^2\\
&=&\displaystyle \ln 3+\ln (1-y)\\
&=&\displaystyle \ln 3-\sum\limits_{n=1}^\infty \frac{y^n}{n}&& \begin{array}{l}
\ln (1-y)= -\sum\limits_{n=1}^\infty \frac{y^n}{n} \text{ for } |y|<1\\
\text{above does not hold for } |y|>1\\
\text{above may (not) hold for } y=\pm 1
\end{array}
\\
&=&\displaystyle \ln 3-\sum\limits_{n=1}^\infty \left(\frac{2}{3}\right)^n\frac{x^{2n}}{n}\quad . &&\text{Substituted back } y=\frac{2}{3}x^2\quad .
\end{array}
\]
As indicated above, the equality $\ln (1-y)= -\sum\limits_{n=1}^\infty \frac{y^n}{n}$ holds for $|y|<1$ and fails for $|y|>1$ (for $|y|>1$ the series $\sum\limits_{ n=1}^\infty \frac{y^n}{n}$ diverges). Therefore interval of convergence is given by 

\[\begin{array}{rcll|l}
|y|&<&1 &&\text{use }y=\frac{2}{3}x^2\\
\left|\frac{2}{3 }x^2\right|&<&1\\
|x^2|&<&\frac{3}{2}\\
|x|&<&\sqrt{\frac{3}{2}},
\end{array}
\]
i.e., the radius of convergence is $R=\sqrt{\frac{3}{2}}$.
}


\begin{problem}
\label{problemMaclaurin 1/(1-x)^k}
Compute the Maclaurin series of 
\[
\left(\frac{1}{(1-x)^k}\right)\quad ,
\]
where $n\geq 1$ is an integer. 
\end{problem}
\solution{
\ref{problemMaclaurin 1/(1-x)^k}
We have that
\[
\begin{array}{r@{~}c@{~}lcl}
\displaystyle \frac{\diff }{\diff x}\left(\frac{1}{1-x}\right)&=&\displaystyle  \frac{ (1-x)'}{(1-x)^2} &=&\displaystyle  \frac{1}{(1-x)^2}\\
\displaystyle \frac{\diff^2 }{\diff x^2}\left(\frac{1}{1-x}\right)&=&\displaystyle \frac{\diff }{\diff x}\left(\frac{1}{ (1-x)^2}\right)=-2 \frac{(1-x)'}{(1-x)^3} &=&\displaystyle  \frac{2}{(1-x)^3}\\
\displaystyle \frac{\diff^3 }{\diff x^3}\left(\frac{1}{1-x}\right)&=&\displaystyle  \frac{\diff }{\diff x}\left( \frac{2}{ (1-x)^3}\right)=2(-3) \frac{(1-x)'}{(1-x)^4}&=&\displaystyle  \frac{2\cdot 3}{(1-x)^4}\\
\vdots \\
\displaystyle \frac{\diff^{k-2} }{\diff x^{k-2}}\left(\frac{1}{1-x}\right)&&&=& \displaystyle   \frac{(k-2)!}{(1-x)^{k-1}}\\
\displaystyle \frac{\diff^{k-2} }{\diff x^{k-2}}\left(\frac{1}{1-x}\right)&=&\displaystyle \displaystyle \frac{\diff }{\diff x} \left(\frac{(k-2)!}{(1-x)^{k-1}}\right) &=&\displaystyle   \frac{(k-1)!}{(1-x)^{k}}\\
\vdots 
\end{array}
\]
We can now compute Maclaurin series as follows:
\[
\begin{array}{r@{~}c@{~}ll|l}
\displaystyle
\maclaurin\left(\frac{1}{(1-x)^{k}}\right)&=& \displaystyle \maclaurin\left(\frac{1}{(k-1)!}  \frac{\diff^{k-1}}{\diff x^{k-1}}\left( \frac{1}{(1-x)}\right)\right)\\
&=&\displaystyle \frac{1}{(k-1)!}  \frac{\diff^{k-1}}{\diff x^{k-1}} \left(\maclaurin\left(\frac{1}{1-x}\right)\right)\\
&=&\displaystyle \frac{1}{(k-1)!}  \frac{\diff^{k-1}}{\diff x^{k-1}}\left(\sum\limits_{n=0}^{\infty} x^n\right)\\
&=&\displaystyle \frac{1}{(k-1)!}  \left(\sum\limits_{n=0}^{\infty} n(n-1)\dots (n-k+2) x^{n-k+1}\right)&&\text{Recall }\binom{n}{k}= \frac{n(n-1)\dots (n-k+1)}{k!} \\
&=&\displaystyle  \sum\limits_{n=0}^{\infty} \binom{n}{k-1} x^{n-k+1}&& 
\text{Set } n-k+1=m\\
&=&\displaystyle \sum\limits_{m=-k+1}^{\infty} \binom{m+k-1}{k-1} x^{m} &&
\text{first }k-2 \text{ summands are zero}
\\
&=&\displaystyle  \sum\limits_{m=0}^{\infty} \binom{m+k-1}{k-1} x^{m} 
\end{array} 
\]
}

\begin{problem}
\label{problemMaclaurin(1+x)^q}
Compute the Maclaurin series of 
\[
(1+x)^q\quad ,
\]
where $q\in \mathbb R$ is an arbitrary real number. 


\end{problem}
\solution{\ref{problemMaclaurin(1+x)^q}
Since $q$ does not have to be an integer, we cannot directly relate its power series to the power series of $\frac{1}{1+x}$ or its derivatives. We therefore compute the Maclaurin series directly using \refBad{\ref{eqMacLaurinDef}}{their definition}{their definition (Definition \ref{eqMacLaurinDef})}.
\[
\begin{array}{rcl}
\frac{\diff}{\diff x}\left( (1+x)^q\right)&=& q (1+x)^{q-1}\\
\frac{\diff^{2}}{\diff x^2}\left( (1+x)^q\right)&=& q(q-1) (1+x)^{q-2}\\
\vdots \\
\frac{\diff^{n}}{\diff x^n}\left( (1+x)^q\right)&=& q(q-1)(q-2)\dots (q-n+1) (1 +x )^{ q-n}\quad .
\end{array}
\]
Therefore $\frac{\diff^{n}}{\diff x^n}\left( (1+x)^q\right)_{|x=0}=q( q-1) (q-2) \dots (q-n+1) (1+0)^{q-n}= q(q-1)(q-2)\dots (q-n+1)  $. Therefore
\begin{equation}\label{eqNewtonBinomialGeneralized}
\begin{array}{rcl}
\displaystyle \maclaurin \left( (1+x)^q\right) &=& \displaystyle\sum_{ n=0}^{ \infty} \frac{ 1}{n!}\frac{\diff^n }{\diff x^n} \left( (1+x)^q \right)_{ |x=0 } x^n  \\ &=& \displaystyle \sum_{n=0}^{\infty}  \frac{q( q-1) (q-2)\dots (q-n+1) }{n!}x^n= \sum_{ n=0 }^{\infty} \binom{q}{n}x^n\quad .
\end{array}
\end{equation}
For the last equality we recall the definition of binomial coefficient $\binom{ q }{n} = \frac{q(q-1)\dots (q-n+1)}{n!}$ and that it allows for $q$ to be an arbitrary complex number \refBad{\ref{eqBinomialCoeffDefinition}}{}{(see \eqref{eqBinomialCoeffDefinition})}. The above formula is a generalization of the Newton binomial formula\refBad{\ref{eqNewtonBinomialFormula}}{}{, \eqref{eqNewtonBinomialFormula}}. \index{binomial!generalized formula}
}

\begin{problem}
Compute the Maclaurin series of the function.
\begin{multicols}{2}
\begin{enumerate}
\item $\sqrt{1+x}$.
\item $\frac{1}{\sqrt{1+x}}$.
\item $ \frac{1}{\sqrt{1-x^2}}$.
\item $ \arcsin x$.
\end{enumerate}
\end{multicols}
\end{problem}
\solution{\ref{problemMaclaurinsqrt(1+x)}
This problem follows directly from the formula $(1+x)^q=\sum\limits_{n=0}^\infty \binom{q}{n} x^n$. 
\[
\maclaurin \left(\sqrt{1+x}\right)=\maclaurin \left((1+x)^{\frac{1}{2}}\right)= \sum\limits_{n=0}^\infty \binom{\frac{1}{2}}{n} x^n\quad .
\]
}

\solution{\ref{problemMaclaurin(1+x)^(-1/2)}
This problem can be solved by computing the derivative of the preceding problem. However, it is easier to simply apply the generalized Newton Binomial formula.
\[
\maclaurin \left((1+x)^{-\frac{1}{2}}\right)= \sum\limits_{n=0}^\infty \binom{- \frac{1}{2 }}{ n} x^n\quad .
\]
}

\solution{\ref{problemMaclaurin(1-x^2)^(-1/2)}
This problem is solved by replacing $ x$ with $-x^2$ in Problem \ref{problemMaclaurin(1+x)^(-1/2)}. To avoid the possible confusion, we carry out the substitution by introducing an intermediate variable $y$.
\[
\begin{array}{rcll|l}
\displaystyle \maclaurin \left(\left(1-x^2\right)^{-\frac{1}{2}} \right)&=& \displaystyle   \maclaurin \left(\left(1+y\right)^{-\frac{1}{2}} \right) && \text{Set }y= -x^2 \\
&=&\displaystyle  \sum\limits_{n=0}^\infty \binom{- \frac{1}{2 }}{ n} y^n &&\text{Substitute back } y=-x^2\\
&=& \displaystyle  \sum\limits_{ n=0 }^\infty (-1)^n \binom{- \frac{1}{2 }}{ n} x^{2n}\quad .
\end{array}
\]
}

\solution{\ref{problemMaclaurin(arcsin x)} 
We have that $\frac{\diff }{\diff x}\left(\Arcsin x\right)=\frac{1}{ \sqrt{1- x^2}}$, and the Maclaurin series of $\frac{1}{ \sqrt{1- x^2}}$ were computed in Problem \ref{problemMaclaurin(1-x^2)^(-1/2)}. The power series of $\Arcsin x$ are therefore obtained via integration. 
\[
\begin{array}{rcll|l}
\displaystyle \frac{\diff }{\diff x}\maclaurin (\Arcsin x)&=&\displaystyle  \maclaurin \left(\frac{\diff}{\diff x} \left(\Arcsin x\right) \right)\\
&=&\displaystyle \maclaurin \left(\frac{1}{\sqrt{1-x^2}}\right) && \text{use Problem }\ref{problemMaclaurin(1-x^2)^(-1/2)} \\
&=&\displaystyle \sum\limits_{ n=0 }^\infty (-1)^n \binom{- \frac{1}{2 }}{ n} x^{2n}\\
\maclaurin \left(\Arcsin x\right)&=&\displaystyle \int \left(\sum\limits_{ n=0 }^\infty (-1)^n \binom{- \frac{1}{2 }}{ n} x^{2n}\right)\diff x\\
&=&\displaystyle C+\sum\limits_{ n=0 }^\infty (-1)^n \binom{- \frac{1}{2 }}{ n} \int x^{2n}\diff x\\
&=&\displaystyle C+\sum\limits_{ n=0 }^\infty (-1)^n \binom{- \frac{1}{2 }}{ n} \frac{ x^{2n+1}}{2n+1}&&C=0 \text{ since }\Arcsin 0=0\\
&=&\displaystyle \sum\limits_{ n=0 }^\infty (-1)^n \binom{- \frac{1}{2 }}{ n} \frac{ x^{2n+1}}{2n+1}\quad .
\end{array}
\]
}


\begin{problem}
Find the Taylor series of the function at the indicated point.
\begin{enumerate}
\item  $\frac{1}{x^2}$ at $a=-1$.

\answer{$\displaystyle 1+ 2(x+1)+3(x+1)^2+\dots = \sum\limits_{n=0}^\infty(n+1)(x+1)^n$ } 

\item \label{problemTaylorSeries a=1 ln(sqrt(x^2-2x+2))}
$\ln \left( \sqrt{x^2-2x+2} \right)$ at $a=1$.

\answer{$\displaystyle \sum_{n=1}^{\infty} (-1)^{n+1} \frac{(x-1)^{2n}}{2n}  $}
\end{enumerate}


\end{problem}
\solution{
\ref{problemTaylorSeries a=1 ln(sqrt(x^2-2x+2))}
\[
\begin{array}{rcll|l}
\displaystyle \ln \left( \sqrt{x^2-2x+2} \right) &=&\displaystyle  \frac{ 1}{2} \ln \left( (x-1)^2+1\right)&&\text{use } \ln(1+y) =  \sum \limits_{n=1 }^ \infty (-1)^{n+1} \frac{y^n}{n}, |y|< 1 \\
&=&\displaystyle \frac{1}{2}\sum\limits_{n=1}^{\infty} (-1)^{n+1}\frac{\left((x-1)^2\right)^{n}}{n}\\
&=&\displaystyle \sum \limits_{n=1}^{\infty} (-1)^{n+1} \frac{( x-1)^{2n}}{2n}\quad .
\end{array}
\]

We will determine the interval of convergence of the series as well. If we use the fact that $\ln(1+y) =  \sum \limits_{n=1 }^ \infty (-1)^{n+1} \frac{y^n}{n}$ holds for $ -1<y\leq 1$, it follows immediately that the above equality holds for $ 0<(x-1)^2\leq 1$, which holds for $x\in[0,2]$. Let us however compute the interval of convergence without using the aforementioned fact.

Let $a_n$ be the $n^{th}$ term of our series, i.e., let 
\[
a_n= (-1)^{n+1} \frac{( x-1)^{2n}}{2n}\quad .
\]
We use the ratio test:
\[
\begin{array}{rcl}
\displaystyle \lim \limits_{ n\to \infty}\left| \frac{a_{n + 1} }{a_n }\right|&=&\displaystyle \lim \limits_{ n\to \infty }\left| \frac{(-1)^{n+2}(x-1)^{2n+2} }{ (2n+2)} \frac{ 2 n}{ (-1)^{n+1} ( x-1)^{2n} } \right|\\
&=&\displaystyle \lim_{n\to \infty} (x-1)^2 \frac{n}{n+1}\\
&=&\displaystyle (x-1)^2\quad .
\end{array}
\] 
By the ratio test, the series is divergent for $ (x-1)^2>1$, i.e., for $|x-1|>1$, and convergent for $(x-1)^2<1$, i.e., for $|x-1|<1$. The ratio test is inconclusive at only two points: $x-1=1$, i.e., $x=2$ and $x-1=-1$, i.e., $x=0$. At both points the series becomes $\displaystyle \sum\limits_{n=1}^{\infty} (-1)^{n+1} \frac{ 2^{ 2n}}{2n}$ and the series is convergent at both points by the alternating series test.
}

\begin{problem}
Find the Taylor series around the indicated point. The answer key has not been proofread, use with caution.
\begin{enumerate}
\item  $\frac{1}{x}$ at $a=1$.
\answer{$\displaystyle 1- (x-1)+(x-1)^2-(x-1)^3+\dots = \sum\limits_{n=0}^\infty(-1)^n(x-1)^n$ } 
\item  $\frac{1}{x^2}$ at $a=1$.
\answer{$\displaystyle 1- 2(x-1)+3(x-1)^2-4(x-1)^3+\dots = \sum\limits_{n=0}^\infty(n+1)(-1)^n(x+1)^n$ } 
\end{enumerate}

\end{problem}
\subsection{Example of differentiable function not equal to its Maclaurin series}

\begin{problem}
Let $f(x)$ be defined as 
\[
f(x):=\doublebrace{e^{-\frac{1}{x^2}}}{\mathrm{if~} x>0}{0}{\mathrm{otherwise.}}
\]
\begin{enumerate}[ref={\fcProblemRef}]
\item Prove that if $R(x)$ is an arbitrary rational function, 
\[
\lim\limits_{\substack{x\to 0\\ x>0}} R(x)e^{-\frac{1}{x^2}}=0
\]
\item Prove that $f(x)$ is differentiable at $0$ and $f'(0)=0$.
\item Prove that the Maclaurin series of $f(x)$ are 0 (but $f(x)$ is clearly a non-zero function).
\end{enumerate}
\end{problem}

\section{Complex numbers}
\begin{problem}
\label{probComplexNumbersBasicOperations}
Carry out the operations. For some of the problems you may want to review the Newton Binomial formula\refBad{\ref{eqNewtonBinomialFormula}}{}{ \eqref{eqNewtonBinomialFormula}}.
\begin{multicols}{3}
\begin{enumerate}[ref={\fcProblemRef}]
\item $\displaystyle(5+3i)^2$.

\answer{$30 i+16 $}
\item $\displaystyle\frac{5+3i}{2-3i}$.

\answer{$ \frac{21}{13} i+\frac{1}{13} $}
\item $(5+3i)^{-2}$.

\answer{$-\frac{15}{578} i+\frac{4}{289} $}
\item $(1+i)^3$.

\answer{$2 i-2  $}
\item $(1+i)^4$.

\answer{$-4 $}
\item \label{eq(1+i)^5} $(1+i)^5$.

\answer{$ -4 i-4 $}
\item \label{eq(1+i)^-5} $(1+i)^{-5}$.

\answer{$\frac{1}{8} i-\frac{1}{8} $}
\end{enumerate}
\end{multicols}

\end{problem}
\solution{\ref{eq(1+i)^5}.
By the Newton Binomial formula\refBad{\ref{eqNewtonBinomialFormula}}{}{ \eqref{eqNewtonBinomialFormula}}, we have that 
\[
(1+i)^5= 1 + 5i + 10 i^2+ 10i^3+5i^4+i^5= 1-10+5 +i(5-10+1)=-4-4i.
\]
}

\solution{
\ref{eq(1+i)^-5}. Using the preceding example, we have that \[
(1+i)^{-5}=\frac{1}{(1+i)^5}=\frac{1}{ -4-4i}=\frac{-4+4i}{(-4-4i)(-4+4i)}=\frac{-4+4i}{32}=-\frac{1}{8}+\frac{1}{8}i\quad .
\]
}

\begin{problem}

Plot the number $z$ on the complex plane (you may use one drawing only for all the numbers). Find all real numbers $\varphi$ and $\rho$ for which $z=e^{\rho+i\varphi}$. Your answer may contain expressions of the form $\arcsin x$, $\arccos x$, $\arctan x$, $\ln x$, only if $x$ is a real number.
\begin{multicols}{2}
\begin{enumerate}[ref={\fcProblemRef}]
\item \label{prob1plussqrt3} $z=1+i\sqrt{3}$.

\answer{$z= e^{\ln 2 + i\left(\frac{\pi}3 +2k\pi\right)}$, $k\in \mathbb Z$}
\item \label{prob2plus3i} $z=-2-3i$.

\answer{$z= e^{\frac{1}{2}\ln (13) + i\left(\Arctan\left(\frac{3}{2}\right)+2k\pi\right)}$, $k\in \mathbb Z$}
\item $z=1-i\sqrt{3}$.

\answer{$z= e^{\ln 2 + i\left(-\frac{\pi}3 +2k\pi\right)}$, $k\in \mathbb Z$}
\item $z=1+i$.

\answer{$z= e^{\frac{1}{2}\ln 2 + i\left(\frac{\pi}4 +2k\pi\right)}$, $k\in \mathbb Z$}
\item $z=-1-i$.

\answer{$z= e^{\frac{1}{2}\ln 2 + i\left(\frac{5\pi}{4} +2k\pi\right)}$, $k\in \mathbb Z$}
\item $z=\frac{\sqrt{3}+i}{4}$.

\answer{$z= e^{-\ln 2 + i\left(\frac{\pi}{6} +2k\pi\right)}$, $k\in \mathbb Z$}
\item $z=-i$.

\answer{$z= e^{ i\left(-\frac{\pi}{2} +2k\pi\right)}$, $k\in \mathbb Z$}
\item $z=3+4i$.

\answer{$z= e^{\ln 5 + i\left( \Arctan\left(\frac{4}{3} \right)+ 2k\pi\right)}$, $k\in \mathbb Z$}
\end{enumerate}
\end{multicols}
\end{problem}
\solution{\ref{prob1plussqrt3}.

\noindent Solution I. We have that
\[
|z|=\sqrt{z\bar z}= \sqrt{\left(1+i\sqrt{3}\right)\left(1-i\sqrt{3}\right)}=\sqrt{1^2+\sqrt{3}^2}=\sqrt{4}=2\quad .
\]
Recall \refBad{\ref{eqEulerExp}}{that}{from \eqref{eqEulerExp} that} $e^{\rho +i \varphi}= e^{\rho}(\cos\varphi+i\sin \varphi) $ and therefore
\[
\begin{array}{rcl}
\cos \varphi &=&\displaystyle \frac{|z|\cos \varphi}{|z|}= \frac{\Re z}{|z|} =  \frac{1}{2}\\
\sin \varphi &=&\displaystyle \frac{|z|\sin \varphi}{|z|}= \frac{\Im z}{|z|} =  \frac{\sqrt{3}}2\\
\tan \varphi &=&\displaystyle \frac{\sin \varphi}{\cos \varphi}=  \frac{\sqrt{3}}3\quad .
\end{array}
\]
Therefore $\varphi$ is of the form $\varphi =  \arctan \left(\frac{\sqrt{3}}3\right)= \frac{\pi}3+k\pi$. However $\varphi$ cannot be of the form $ \frac{\pi}3+(2k+1)\pi$ because $\cos \left(\frac{\pi}3+(2k+1)\pi\right)=-\frac 12$. On the other hand, $\sin \left(\frac{\pi}{3}+2k\pi\right) = \frac{\sqrt{3}}2$ and $\cos \left(\frac{\pi}{3}+2k\pi\right) = \frac {1}{2})$. Therefore
\[
\varphi =\frac{\pi}{3}+2k\pi, \quad \quad \mathrm{for~all~} k\in \mathbb Z
\]
(Recall that $\mathbb Z$ denotes the integers).

As studied in class $e^{\rho}=|z|=2$, and therefore $\rho = \ln (e^\rho)= \ln |z|=\ln 2 $. Therefore we get the answer
\[
1+i\sqrt{3} = e^{\ln 2 +i \left(\frac{\pi}3+2k\pi  \right) }
\]
for all $k\in \mathbb Z$. To finish the task we need to plot the number $z$.

\optionalDisplay{
\psset{xunit=1cm,yunit=1cm}
\begin{pspicture*}(-3,-4)(4,4)
\psline[linecolor=gray](-1.5,0)(2.5,0) % x-axis
\psline[linecolor=gray](0,-1.5)(0,2.5) % y-axis
\rput[l](2.5,0){$\Re z$}
\rput[b](0,2.5){$\Im z$}
\rput[bl](0.03,1.03){$i$}
\rput[bl](1,-0.1){$1$}
\rput[c](0,1){$\bullet$}
\rput[c](1,0){$\bullet$}
\psplot{-1}{0.5}{1 x x mul sub sqrt}

\psplot[linecolor=red]{0.5}{1}{1 x x mul sub sqrt}

\psplot{-1}{1}{1 x x mul sub sqrt -1 mul}

\psline(0,0)(0.5,0.866)
\rput[c](0.5,0.866){$\bullet$}
\rput[l](0.47,1){$\frac{z}{|z|}$}
\psline[linestyle=dotted](0.5,0.866)(1,1.74)
\rput[l](1.1,1.81){$z=1+i\sqrt{3} $}
\rput[c](1,1.74){$\bullet$}

\rput[t](-2,-3){$-2-3i$}
\psline[linestyle=dotted](0,0)(-2,-3)
\rput[c](-2,-3){$\bullet$}

\rput[l](0.84,0.65){$\varphi= \frac\pi 3 $}
\end{pspicture*}
} %optionalDisplay

\noindent Solution II. We draw the number $z$ as above. We compute that $\sin \varphi = \frac{\Im z}{|z|}= \frac{ \sqrt{3} }{2}$, $\cos \varphi= \frac{\Re z}{|z| } = \frac{1 }{2}$. Therefore we have that
\[
1+i\sqrt{3}= e^{\ln|1+i\sqrt{3}| + i\left(\frac{\pi}3 +2 k \pi \right)}=  e^{\ln 2 + i\left(\frac{\pi}3 +2k \pi \right) }\quad .
\]
}

\solution{\ref{prob2plus3i}. 

We draw the number as indicated on the figure. We compute that $\sin \varphi =-\frac{3}{\sqrt{13}}$, $\cos \varphi = - \frac{ 2}{\sqrt{13}}$, $\tan \varphi = \frac{3}{2}$. By the convention of our course, $\arctan \varphi\in \left(- \frac{ \pi}{2}, \frac{\pi}{2}\right)$. Therefore $\varphi= \left( \Arctan \left( \frac{3}{2}\right) +\pi\right)+2k\pi $ for all $k\in \mathbb Z$. Finally, we get
\[
\begin{array}{rcl}
-2-3i&=& e^{\ln|-2-3i| + i \left(\left(\Arctan \left( \frac{ 3}{2} \right) +\pi\right) +2k\pi\right)}= e^{\ln \sqrt{13}  + i\left(\left(\Arctan\left(\frac{3}{2}\right) + \pi \right) +2k\pi\right)} \\&=&  e^{\frac{1}2\ln 13  + i \left( \left( \Arctan\left(\frac{3}{2}\right) +\pi\right) +2k \pi \right) } \quad .
\end{array}
\]
}


\begin{problem}
Find the fourth roots of $-16$

\answer{$\pm \sqrt{2}\pm \sqrt{2}i$ (in all four combinations)}
\end{problem}
\noindent \solution{\ref{problemz^3=i}. Let $z=|z|(\cos \theta+i\sin \theta)$ be the polar form of $|z|$ for which $\theta\in(-\pi, \pi]$. We have  $|z|^3=\left| i \right| = 1$. Therefore $|z|=1$.


We can write $\displaystyle i$ in polar form as $\displaystyle i = \cos\left(\frac{ \pi}{2} \right) + i\sin\left(\frac{\pi}{2}\right)$. Therefore 
\[
\begin{array}{rcll|l}
z^3&=& i&&\text{use de Moivre's formula} \\
|z|^3 \left(\cos (3\theta)+i\sin (3\theta)\right)&=& \cos\left(\frac{ \pi}{2} \right) + i\sin\left(\frac{\pi}{2}\right) &&\text{use }|z|=1\\
\cos (3\theta)+i\sin (3\theta) &=&\cos\left(\frac{ \pi}{2} \right) + i\sin\left(\frac{\pi}{2}\right) &&\begin{array}{l}\text{when sines and cosines} \\
\text{coincide the angles differ}\\
\text{by even multiple of }\pi\end{array}
\\
3\theta&=&\frac{\pi}{2}+2k\pi, &&k-\text{integer}\\
\theta&=&\frac{\pi}{6}+k\frac{2\pi}{3}&&\theta\in(-\pi,\pi] \Rightarrow k=-1, 0, \text{ or }1  \\
\theta&=&-\frac{\pi}{2}, \frac{\pi}{6}, \text{ or } \frac{5\pi}{6}\quad .
\end{array}
\]


To find out the values of $z$ in non-polar form, we simply plot the numbers $z=(\cos \theta +i\sin \theta)$. The three complex solutions lie on a circle of radius $1$; the numbers form an equilateral triangle, as shown on the picture. To find the actual values for these complex numbers, we use known values of the trigonometric functions. Our final answer is as follows.

\begin{pspicture} (-1.2,-1.2)(1.2,1.2)
\tiny
\psaxes[labels=none, ticks=none]{<->}(0,0)(-1.4,-1.4)(1.4,1.4)
\rput[t](1.4,-0.1){$Re$}
\rput[r](-0.1,1.4){$Im$}
\parametricplot[linecolor=gray]{0}{360}{t cos t sin }
\psline[linecolor=blue](0,0)(!150 cos 150 sin )
\psdot*(! 150 cos 150 sin )
\rput[rt](! 150 cos  150 sin ){$-\frac{ \sqrt{3 }}{ 2} +\frac{i}{2}~$}
\psline[linecolor=blue](0,0)(!30 cos 30 sin)
\psdot*(!30 cos  30 sin )
\rput[lt](! 30 cos 30 sin ){$~\frac{ \sqrt{3 }}{ 2} +\frac{i}{2}$}
\psline[linecolor=blue](0,0)(!-90 cos -90 sin )
\psdot*(!-90 cos -90 sin )
\rput[lt](0, -1.1){$-i$}
\end{pspicture}
\raisebox{1.2cm}{
$
\begin{array}{c|c}
\text{polar form }&\text{ value}\\\hline 
\cos \left(\frac{5\pi}{6}\right) +i\sin \left( \frac{ 5\pi }{6}\right)&  -\frac{\sqrt{3}}{2}+\frac{i}{2} \\
\cos \left(\frac{\pi}{6}\right) +i\sin \left( \frac{ \pi}{6} \right) &  \frac{ \sqrt{3 }}{ 2} +\frac{i}{2} \\
\cos \left(-\frac{\pi}{2}\right) +i\sin \left(-\frac{ \pi } {2} \right)& -i
\end{array}
$
}
}

\noindent \solution{
\ref{problemz^3=-i/8}

Let $z=|z|(\cos \theta+i\sin \theta)$ be the polar form of $|z|$ for which $\theta\in(-\pi, \pi]$. We have  $|z|^3=\left| \frac{i}{8 } \right| = \frac{1}{8}$. Since $|z|$ is a positive real number it follows that $\displaystyle |z|=\sqrt[3]{\frac{1}{8}}=\frac{1}{2}$.

We can write $\displaystyle-\frac{i}{8}$ in polar form as $\displaystyle - \frac{ i}{8} = \frac{1 }{8}\left( \cos\left(- \frac{ \pi}{2} \right) + i\sin\left(-\frac{\pi}{2}\right)  \right)$. Therefore 
\[
\begin{array}{rcll|l}
z^3&=& \frac{-i}{8}&&\text{use de Moivre's formula} \\
|z|^3\left(\cos (3\theta)+i\sin (3\theta)\right)&=&\frac{1 }{8}\left( \cos\left(- \frac{ \pi}{2} \right) + i\sin\left(-\frac{\pi}{2}\right)  \right) &&\text{use }|z|=\frac{1}{2}\\
\cos (3\theta)+i\sin (3\theta)&=&\cos\left(- \frac{ \pi}{2} \right) + i\sin\left(-\frac{\pi}{2}\right) &&\begin{array}{l} \text{when sines and cosines} \\
\text{coincide the angles differ}\\
\text{by even multiple of }\pi\end{array}
\\
3\theta&=&-\frac{\pi}{2}+2k\pi, &&k-\text{integer}\\
\theta&=&-\frac{\pi}{6}+k\frac{2\pi}{3}&&\theta\in(-\pi,\pi] \Rightarrow k=-1, 0, \text{ or }1  \\
\theta&=&-\frac{5\pi}{6}, -\frac{\pi}{6}, \text{ or } \frac{\pi}{2}\quad .
\end{array}
\]
To find out the values of $z$ in non-polar form, we simply plot the numbers $z=\frac{1}{2}(\cos \theta +i\sin \theta)$. The three complex solutions lie on a circle of radius $\frac{1}{2}$; the numbers form an equilateral triangle, as shown on the picture. To find the actual values for these complex numbers, we use known values of the trigonometric functions. Our final answer is as follows.

\begin{pspicture} (-1.2,-1.2)(1.2,1.2)
\tiny
\fcAxesStandard{-1.2}{-1.2}{1.2}{1.2}
\fcLabels[$\Re$][$\Im$]{1.2}{1.2}
\parametricplot[linecolor=gray]{0}{360}{t cos 0.5 mul t sin 0.5 mul}

\psline[linecolor=blue](0,0)(!-150 cos 0.5 mul -150 sin 0.5 mul)
\fcFullDot{-150 cos 0.5 mul}{-150 sin 0.5 mul}
\rput[rt](! -150 cos 0.5 mul -150 sin 0.5 mul){$\frac{ \sqrt{3 }}{ 4} -\frac{i}{4}~$}

\psline[linecolor=blue](0,0)(!-30 cos 0.5 mul -30 sin 0.5 mul)
\fcFullDot{-30 cos 0.5 mul}{-30 sin 0.5 mul}
\rput[lt](! -30 cos 0.5 mul -30 sin 0.5 mul){$~-\frac{ \sqrt{3 }}{ 4} -\frac{i}{4}$}

\psline[linecolor=blue](0,0)(!90 cos 0.5 mul 90 sin 0.5 mul)
\fcFullDot{90 cos 0.5 mul}{90 sin 0.5 mul}
\rput[lb](0, 0.5){$~\frac{i}{2}$}

\end{pspicture}
\raisebox{1.2cm}{
$
\begin{array}{c|c}
\text{polar form }&\text{ value}\\\hline 
\frac{1}{2}\left(\cos \left(-\frac{5\pi}{6}\right) +i\sin \left(- \frac{ 5\pi }{6}\right) \right)&  -\frac{\sqrt{3}}{4}-\frac{i}{4} \\
\frac{1}{2}\left(\cos \left(-\frac{\pi}{6}\right) +i\sin \left( -\frac{ \pi}{6} \right) \right)&  \frac{ \sqrt{3 }}{ 4} -\frac{i}{4} \\
\frac{1}{2}\left(\cos \left(\frac{\pi}{2}\right) +i\sin \left(\frac{ \pi } {2} \right) \right)& \frac{i}{2} 
\end{array}
$
}
}
\begin{problem}
Express the number in polar form and compute the indicated power. The answer key has not been proofread, use with caution.

\begin{enumerate}
\item $z=\sqrt{3}+i$, find $z^3$.

\answer{$z=\sqrt{3}+i= 2\left(\cos\left( \frac{\pi}{6} \right) +i\sin\left( \frac{\pi}{6}\right) \right)$, $z^3= 8\left( \cos\left( \frac{\pi}{2} \right)+ i\sin\left( \frac{ \pi }{2} \right) \right)=8i.$}

\item $z=\sqrt{3}i-1$, find $z^{10}$.

\answer{$z=2\left(\cos\left(\frac{2\pi}{3} \right)+i\sin \left( \frac{2\pi}{3} \right) \right)$, $z^{10}= 2^{10} \left(- \frac{1}{2} +\frac{\sqrt{3 }}{2} i\right) = -512 +512 \sqrt{3}i .$ }

\item $z= -1-i$, find $z^{21}$.

\answer{$z=\sqrt{2} \left(\cos\left(\frac{5}{4}\pi\right) + \sin \left( \frac{5}{4} \pi\right) \right)$, $z^{21}= 1024+ 1024 i $.}
\end{enumerate}  


\end{problem}
\begin{problem}
The de Moivre follows directly from Euler's formula and states that $(\cos (n\alpha) +i\sin (n\alpha) )= (\cos \alpha +i\sin\alpha)^n$. Expand the indicated expression and use it to express $\cos (n\alpha)$ and $\sin (n\alpha)$ via $\cos \alpha$ and $\sin \alpha$.

You may want to use the Newton binomial formulas (derived, say, via Pascal's triangle). The formulas you may want to use are:
\[
\begin{array}{rcl}
(a+b)^2&=&a^2+2ab+b^2\\
(a+b)^3&=&a^3+3a^2b+3ab^2+b^3\\
(a+b)^4&=&a^4+4a^3b+6a^2b^2+4ab^3+b^4\quad .\\
\end{array}
\]

\begin{enumerate}
\item Expand $(\cos \alpha +i\sin \alpha)^2$. Express $\cos (2\alpha)$ and $\sin(2\alpha) $ via $\cos \alpha$ and $\sin \alpha$.

\answer{ $\begin{array}{rcl}
\cos (2\alpha)&=&\cos^2\alpha-\sin^2\alpha \\
\sin (2\alpha)&=&2\sin\alpha\cos\alpha.
\end{array}
$
}
\item Expand $(\cos \alpha +i\sin \alpha)^3$. Express $\cos (3\alpha)$ and  $\sin(3\alpha) $ via $\cos \alpha$ and $\sin \alpha$.

\answer{$
\begin{array}{rcl}
\cos (3\alpha)&=&\cos^3\alpha-3\cos\alpha \sin^2 \alpha \\ \sin (3\alpha)&=&-\sin^3\alpha+ 3\sin\alpha \cos^2\alpha.
\end{array}
$}
\item Expand $(\cos \alpha +i\sin \alpha)^4$. Express $\cos (4\alpha)$ and  $\sin(4\alpha) $ via $\cos \alpha$ and $\sin \alpha$.

\answer{$
\begin{array}{rcl}
\cos (4\alpha)&=&\cos^4\alpha-6\cos^2\alpha \sin^2 \alpha-\sin^4\alpha \\
\sin (4\alpha)&=& 4\sin\alpha \cos^3 \alpha - 4\sin^3\alpha \cos\alpha.
\end{array}
$}

\end{enumerate}
\end{problem}
\section{Curves}
\begin{problem}
Match the graphs of the parametric equations $x=f(t)$, $y=g(t)$ with the graph of the parametric curve $ \gamma \left| \begin{array}{rcl}x&=&f(t)\\y&=&g(t) \end{array}\right.$
\psset{xunit=0.5cm, yunit=0.5cm, algebraic=true}
\begin{multicols}{2}
\begin{enumerate}
\item 
\begin{pspicture}(-0.2, -1.2)(2.2,1.2)
\tiny
\psaxesStandard{-0.5}{-1.2}{2.2}{1.2}
\psXTickWithLabel{1}{$1$}
\parametricplot[linecolor=\psColorGraph, plotpoints=500]{-3.14159}{3.14159}{sin(t)+1|sin(2*t)}
\end{pspicture}
\answer{matches to \ref{itemMatchx=1+sin(t),y=sin(2t)}}

\item 
\begin{pspicture}(-1.5, -1.5)(1.5,1.5)
\tiny
\psaxesStandard{-1.4}{-1.4}{1.4}{1.4}
\psXTickWithLabel{1}{$1$}
\parametricplot[linecolor=\psColorGraph, plotpoints=500]{-3.14159}{3.14159}{sin(2*t)|sin(3*t)}
\end{pspicture}
\answer{matches to \ref{itemMatchx=sin2t,y=sin3t}}

\item 
\begin{pspicture}(-2.7, -2.7)(2.7,2.7)
\tiny
\psaxesStandard{-2.5}{-0.5}{2.5}{2.5}
\psXTickWithLabel{1}{$1$}
\parametricplot[linecolor=\psColorGraph,  plotpoints=500]{-2}{2}{t*(t-1.75)*(t+1.75)|sqrt(4-t*t) }
\end{pspicture}
\answer{matches to \ref{itemMatchx=cubic,y=sqrt}}
\end{enumerate}

\columnbreak
\begin{enumerate}
\item \label{itemMatchx=sin2t,y=sin3t}
\begin{pspicture}(-3.3, -1.2)(3.3,1.2)
\tiny
\psaxes[ticks=none, labels=none, arrows=<->](0,0)(-3.2, -1.1)(3.2, 1.1)
\psLabels[$t$][$x$]{3.2}{1.1}
\psXTickWithLabel{1}{$1$}
\psplot[linecolor=blue, plotpoints=500]{-3.14159}{3.14159}{sin(2*x)}
\end{pspicture}
\begin{pspicture}(-3.3, -1.2)(3.3,1.2)
\tiny 
\psaxes[ticks=none, labels=none, arrows=<->](0,0)(-3.2, -1.1)(3.2, 1.1)
\psLabels[$t$][$y$]{3.2}{1.1}
\psXTickWithLabel{1}{$1$}
\psplot[linecolor=blue, plotpoints=500]{-3.14159}{3.14159}{sin(3*x)}
\end{pspicture}
\item \label{itemMatchx=cubic,y=sqrt}
\begin{pspicture}(-2.7, -2.7)(2.7,2.7)
\tiny
\psaxes[ticks=none, labels=none, arrows=<->](0,0)(-2.5, -2.5)(2.5, 2.5)
\psLabels[$t$][$x$]{2.5}{2.5}
\psXTickWithLabel{1}{$1$}
\psplot[linecolor=blue, plotpoints=500]{-2}{2}{x*(x-1.75)*(x+1.75)}
\end{pspicture}
\begin{pspicture}(-2.3, -0.5)(2.3,2,3)
\psaxes[ticks=none, labels=none, arrows=<->](0,0)(-2.2, -0.5)(2.2, 2.2)
\psLabels[$t$][$y$]{2.2}{2.2}
\psXTickWithLabel{1}{$1$}
\psplot[linecolor=blue, plotpoints=500]{-2}{2}{sqrt(4-x*x)}
\end{pspicture}

\item \label{itemMatchx=1+sin(t),y=sin(2t)}
\begin{pspicture}(-3.3, -0.5)(3.3,2.2)
\psaxes[ticks=none, labels=none, arrows=<->](0,0)(-3.2, -0.5)(3.2, 2.5)
\psLabels[$t$][$x$]{3.2}{2.5}
\psXTickWithLabel{1}{$1$}
\psplot[linecolor=blue, plotpoints=500]{-3.14159}{3.14159}{sin(x)+1}
\end{pspicture}
\begin{pspicture}(-3.3, -1.2)(3.3,1.2)
\psaxes[ticks=none, labels=none, arrows=<->](0,0)(-3.2, -1.1)(3.2, 1.1)
\psLabels[$t$][$y$]{3.2}{1.1}
\psXTickWithLabel{1}{$1$}
\psplot[linecolor=blue, plotpoints=500]{-3.14159}{3.14159}{sin(2*x)}
\end{pspicture}


\end{enumerate}
\end{multicols}
\end{problem}
\subsection{Curves in polar coordinates}
\begin{problem}
Match the graph of the curve to its graph in polar coordinates and to its polar parametric equations.
\psset{xunit=0.5cm, yunit=0.5cm}
\begin{multicols}{3}
\begin{enumerate}
\item \begin{pspicture}(-2.323256, -3.101443)(3.400000,3.201443) 
\tiny 
\psaxesStandard{-2.073256}{-2.851443}{3.150000}{2.851443}
%Calculator command: drawPolarExtended{}(\cos{}(3 t)+2, 0, 2 \pi) 
\parametricplot[linecolor=\psColorGraph, plotpoints=1000, algebraic=false]{0}{6.28319}{2.0000000 t 3.0000000 mul 57.29578 mul cos add t 57.29578 mul cos mul 2.0000000 t 3.0000000 mul 57.29578 mul cos add t 57.29578 mul sin mul }
\end{pspicture} 
\answer{matches \ref{itemMatchPolarGraph,r=2+cos(3*t)}, \ref{itemMatchPolarFormula,r=2+cos(3*t)}}
\item 
\begin{pspicture}(-3.269100, -2.893230)(3.268991,3.499862) 
\tiny 
\psaxesStandard{-3.019100}{-2.643230}{3.018991}{3.149862}
%Calculator command: drawPolarExtended{}(\sin{}(5 t)+2, 0, 2 \pi) 
\parametricplot[linecolor=\psColorGraph, plotpoints=1000, algebraic=false]{0}{6.28319}{2.0000000 t 5.0000000 mul 57.29578 mul sin add t 57.29578 mul cos mul 2.0000000 t 5.0000000 mul 57.29578 mul sin add t 57.29578 mul sin mul }
\end{pspicture} 
\answer{matches \ref{itemMatchPolarGraph,r=2+sin(5t)}, \ref{itemMatchPolarFormula,r=2+sin(5t)}}
\item 
\begin{pspicture}(-3.515530, -0.900000)(3.541593,2.319691) 
\tiny 
\psaxesStandard{-3.265530}{-0.650000}{3.291593}{1.969691}
\parametricplot[linecolor=\psColorGraph, plotpoints=1000, algebraic=false]{-3.14159}{3.14159}{t t 57.29578 mul cos mul t t 57.29578 mul sin mul }
\end{pspicture} 
\answer{matches \ref{itemMatchPolarGraph,r=t}, \ref{itemMatchPolarFormula,r=t} }
\item 
\begin{pspicture}(-0.962477, -1.280083)(1.400000,1.380083) 

\tiny 
\psaxesStandard{-0.712477}{-1.030083}{1.150000}{1.030083}
%Calculator command: drawPolarExtended{}(\cos{}(3 t), 0, 2 \pi) 
\parametricplot[linecolor=\psColorGraph, plotpoints=1000, algebraic=false]{0}{6.28319}{t 3.0000000 mul 57.29578 mul cos t 57.29578 mul cos mul t 3.0000000 mul 57.29578 mul cos t 57.29578 mul sin mul }

\end{pspicture}
\answer{matches \ref{itemMatchPolarGraph,r=cos(3t)}, \ref{itemMatchPolarFormula,r=cos(3t)}}

\item 
\begin{pspicture}(-1.122338, -1.122308)(2.550479,2.650486) 
\tiny 
\psaxesStandard{-0.872338}{-0.872308}{2.300479}{2.300486}
%Calculator command: drawPolarExtended{}(\sin{}t+\cos{}t+1, 0, 2 \pi) 
\parametricplot[linecolor=\psColorGraph, plotpoints=1000, algebraic=false]{0}{6.28319}{1.0000000 t 57.29578 mul cos add t 57.29578 mul sin add t 57.29578 mul cos mul 1.0000000 t 57.29578 mul cos add t 57.29578 mul sin add t 57.29578 mul sin mul }
\end{pspicture} 
\answer{matches \ref{itemMatchPolarGraph,r=1+sin(t)+cos(t)}, \ref{itemMatchPolarFormula,r=1+sin(t)+cos(t)}}

\item 
\begin{pspicture}(-1.729462, -1.765832)(1.787465,1.792057) 
\tiny 
\psaxesStandard{-1.479462}{-1.515832}{1.537465}{1.442057}
%Calculator command: drawPolarExtended{}(1/4 \sqrt{t}, 0, 10 \pi) 
\parametricplot[linecolor=\psColorGraph, plotpoints=1000, algebraic=false]{0}{31.4159}{t sqrt 0.2500000 mul t 57.29578 mul cos mul t sqrt 0.2500000 mul t 57.29578 mul sin mul }
\end{pspicture}
\answer{matches \ref{itemMatchPolarGraph,r=sqrt(t)}, \ref{itemMatchPolarFormula,r=1/4sqrt(t)}}

\end{enumerate}

\columnbreak
\begin{enumerate}

\item \label{itemMatchPolarGraph,r=sqrt(t)}
\begin{pspicture}(-0.900000, -0.900000)(5.9,1.898448) 
\tiny 
\psaxesStandard{-0.650000}{-0.650000}{5.565927}{1.548448}
\parametricplot[linecolor=\psColorTangent, plotpoints=1000, algebraic=false]{0}{5.9}{t t sqrt 0.2500000 mul }
\end{pspicture} 

\item \label{itemMatchPolarGraph,r=2+cos(3*t)}
\begin{pspicture}(-0.900000, -0.900000)(6.683185,3.500000) 
\tiny 
\psaxesStandard{-0.650000}{-0.650000}{6.433185}{3.150000}
\parametricplot[linecolor=\psColorTangent, plotpoints=1000, algebraic=false]{0}{6.28319}{t 2.0000000 t 3.0000000 mul 57.29578 mul cos add }
\end{pspicture} 

\item \label{itemMatchPolarGraph,r=2+sin(5t)}
\begin{pspicture}(-0.900000, -0.900000)(6.683185,3.499995) 
\tiny 
\psaxesStandard{-0.650000}{-0.650000}{6.433185}{3.149995}
\parametricplot[linecolor=\psColorTangent, plotpoints=1000, algebraic=false]{0}{6.28319}{t 2.0000000 t 5.0000000 mul 57.29578 mul sin add }
\end{pspicture} 

\item \label{itemMatchPolarGraph,r=1+sin(t)+cos(t)}
\begin{pspicture}(-0.900000, -0.900000)(6.683185,2.914198) 
\tiny 
\psaxesStandard{-0.650000}{-0.650000}{6.433185}{2.564198}
\parametricplot[linecolor=\psColorTangent, plotpoints=1000, algebraic=false]{0}{6.28319}{t 1.0000000 t 57.29578 mul cos add t 57.29578 mul sin add }
\end{pspicture} 
\item \label{itemMatchPolarGraph,r=t}
\begin{pspicture}(-3.541593, -3.541593)(3.541593,3.616510) 
\tiny 
\psaxesStandard{-3.291593}{-3.291593}{3.291593}{3.266510}\parametricplot[linecolor=\psColorTangent, plotpoints=1000, algebraic=false]{-3.14159}{3.14159}{t t}
\end{pspicture} 

\item \label{itemMatchPolarGraph,r=cos(3t)}
\begin{pspicture}(-0.900000, -1.399823)(6.683185,1.500000) 
\tiny 
\psaxesStandard{-0.650000}{-1.149823}{6.433185}{1.150000}
\parametricplot[linecolor=\psColorTangent, plotpoints=1000, algebraic=false] {0}{6.28319}{t t 3.0000000 mul 57.29578 mul cos }

\end{pspicture} 


\end{enumerate}
\columnbreak
\renewcommand\theenumii{\roman{enumii}}
\begin{enumerate}
\item \label{itemMatchPolarFormula,r=1+sin(t)+cos(t)} $r=1+\sin(\theta)+cos(\theta)$
\item \label{itemMatchPolarFormula,r=t} $r= \theta, \theta\in [-\pi, \pi]$.
\item \label{itemMatchPolarFormula,r=cos(3t)} $r= \cos(3\theta), \theta\in [0, 2\pi]$.
\item \label{itemMatchPolarFormula,r=1/4sqrt(t)}
$r=\frac{1}4\sqrt{\theta}, \theta\in [0, 10\pi]$.
\item \label{itemMatchPolarFormula,r=2+sin(5t)} $r=2+\sin (5\theta) $.
\item \label{itemMatchPolarFormula,r=2+cos(3*t)} $r=2+cos(3\theta)$.
\end{enumerate}
\end{multicols}
\end{problem}
\begin{problem}
~\begin{enumerate}[ref={\fcProblemRef}]
\item Sketch the curve given in polar coordinates by $r=2\sin \theta $. What kind of a figure is this curve? Find an equation satisfied by the curve in the $(x,y)$-coordinates.
\item Sketch the curve given in polar coordinates by $r=4\cos \theta $. What kind of a figure is this curve? Find an equation satisfied by the curve in the $(x,y)$-coordinates.
\item \label{problemPolarSketchr=2sec(theta)}  Sketch the curve given in polar coordinates by $r=2\sec \theta $. What kind of a figure is this curve? Find an equation satisfied by the curve in the $(x,y)$-coordinates.
\answer{the curve is the line $x=2$}
\item Sketch the curve given in polar coordinates by $r=2\csc \theta $. What kind of a figure is this curve? Find an equation satisfied by the curve in the $(x,y)$-coordinates.
\item \label{problemPolarSketchr=2sec(theta+pi/4)} Sketch the curve given in polar coordinates by $r=2\sec \left(\theta + \frac{\pi}{4} \right) $. What kind of a figure is this curve? Find an equation satisfied by the curve in the $(x,y)$-coordinates.
\answer{the curve is the line $y=x-2\sqrt{2}$}

\item Sketch the curve given in polar coordinates by $r=2\csc\left(\theta +\frac{\pi}{6}\right)$. What kind of a figure is this curve? Find an equation satisfied by the curve in the $(x,y)$-coordinates.

\end{enumerate}


\end{problem}
\solution{\ref{problemPolarSketchr=2sec(theta)}.
Recall from trigonometry that if we draw a unit circle as shown below, $\sec \theta$ is given by the signed distance as indicated on the figure. Therefore it is clear that the curve given in polar coordinates by $y=\sec \theta$ is the vertical line passing through $x=1$. Analogous considerations can be made for a circle of radius $2$, from where it follows that $y=2\sec \theta$ is the vertical line passing through $x=2$.

Alternatively, we can find an equation in the $(x,y)$-coordinates of the cuve by the direct computation: \[x= r\cos \theta= 2\sec\theta \cos \theta = 2\quad .
\]
\psset{xunit=1cm, yunit=1cm}
\begin{pspicture}(-1.39998, -1.399995)(1.4,2.7)
\tiny
\fcAxesStandard{-1.14998}{-1.149995}{1.15}{2.6}
%Calculator command: drawPolar{}(1, 0, 2 \pi)
\parametricplot[linecolor=\fcColorGraph, plotpoints=1000, algebraic=false]{0}{6.28319}{ 1 t 57.29578 mul cos mul 1 t 57.29578 mul sin mul }
\fcAngle{0}{1.107149}{0.2}{$\theta$}
\psline(0,0)(1,2)
\psline(1,-1)(1,2.6)
\fcLengthIndicator{-0.1}{0.05}{0.9}{2.05}{}
\rput[r](0.5,1.1){$\sec \theta$}
\end{pspicture}
}
\solution{
\ref{problemPolarSketchr=2sec(theta+pi/4)}.

\noindent \textbf{Approach I.}
Adding an angle $\alpha$ to the angle polar coordinate of a point corresponds to rotating that point counterclockwise at an angle $\alpha$ about the origin. Therefore a point $P$ with polar coordinates $P\left( 2\sec \left(\theta + \frac{\pi }{ 4} \right) ,\theta\right)$ is obtained by rotating at an angle $-\frac{\pi}{4}$ the point $Q$ with polar coordinates $Q\left( 2\sec \left(\theta + \frac{\pi }{ 4} \right) ,\theta+ \frac{\pi}{4} \right)$. The point $P$ lies on the curve with equation $r=2\sec \left(\theta+ \frac{\pi}{4}\right)$ and the point $Q$ lies on the curve with equation $r=2\sec \theta$ - the latter curve is the curve from problem \ref{problemPolarSketchr=2sec(theta)}. Thus the curve in the current problem is obtained by rotating the curve from \ref{problemPolarSketchr=2sec(theta)} at an angle of $-\frac{\pi}{4}$. As the curve in Problem \ref{problemPolarSketchr=2sec(theta)} is the vertical line $x=2$, the curve in the present problem is also a line. Rotation at an angle of $-\frac{\pi}{4}$ of a vertical line yields a line with slope $1$. When $\theta=0$, $x=\frac{2}{\frac{\sqrt{2}}{2}}= 2\sqrt{2}$, $y=0$ and the curve passes through $(2\sqrt{2}, 0)$. We know the slope of a line and a point through which it passes; therefore the $(x,y)$-coordinates of our curve satisfy
\[
y=x-2\sqrt{2}\quad .
\]

\noindent \textbf{Approach II. } We compute
\[
\begin{array}{rcll|l}
x&=&\displaystyle r\cos \theta = \frac{2\cos \theta}{\cos (\theta +\frac{\pi}{4})} &&\text{multiply by }\cos \left(\frac{\pi}{4}\right)=\frac{\sqrt{2}}{2} \\
y&=&\displaystyle r\sin \theta = \frac{2\sin \theta}{\cos (\theta +\frac{\pi}{4})} &&\text{multiply by }-\sin \left(\frac{\pi}{4}\right)= -\frac{\sqrt{2 }}{2} \\\hline
&&&&\text{add the above }
\\
x\cos \left(\frac{\pi}{4} \right) -y \sin\left( \frac{\pi}{4}\right)&=&\displaystyle  2 \frac{\cos \theta\cos  \left(\frac{\pi}{4} \right) - \sin \theta \sin  \left(\frac{\pi}{4} \right) }{\cos  \left(\theta +\frac{\pi}{4} \right)} &&\text{use } \cos (\alpha+\beta)=\cos \alpha\cos \beta-\sin\alpha\sin\beta\\
\frac{\sqrt{2}}{2}\left(x-y\right)&=&\displaystyle 2\frac{\cos\left(\theta +\frac{\pi}{4} \right)}{\cos\left(\theta +\frac{\pi}{4} \right)}=2\\
y&=&\displaystyle x-2\sqrt{2},
\end{array}
\]
and therefore our curve is the line given by the equation above.
}

\subsection{Curve tangents}
\begin{problem}
Find the values of the parameter $t$ for which the curve has horizontal and vertical tangents.
\begin{multicols}{2}
\begin{enumerate}
\item $y=t^2-t+1$, $x=t^2+t-1$

\psset{xunit=0.25cm, yunit=0.25cm}
\begin{pspicture}(-0.9, -1.65)(13.4,11.416228)
\tiny
\fcAxesStandard{-0.65}{-1.4}{13.15}{11.066228}

%Calculator input: plotCurve{}(t^{2}- t+1, t^{2}+t-1, -3, 3)
\parametricplot[linecolor=\fcColorGraph, plotpoints=1000]{-3}{3}{ 1 t -1 mul add t 2 exp add -1 t add t 2 exp add }
\end{pspicture}
\item $x=t^3-t^2-t+1$, $y=t^2-t-1 $.

\psset{xunit=1cm, yunit=1cm}
\begin{pspicture}(-0.9, -1.649998)(3.358221,1.5)
\tiny
\fcAxesStandard{-0.65}{-1.399998}{3.108221}{1.15}

%Calculator input: plotCurve{}(t^{3}- t^{2}- t+1, t^{2}- t-1, -1, 2)
\parametricplot[linecolor=\fcColorGraph, plotpoints=1000]{-1}{2}{ 1 t -1 mul add t 2 exp -1 mul add t 3 exp add -1 t -1 mul add t 2 exp add }
\end{pspicture}
\item $x=\cos (t)$, $y=\sin (3t)$

\psset{xunit=1cm, yunit=1cm}
\begin{pspicture}(-1.4, -1.399999)(1.4,1.499999)
\tiny
\fcAxesStandard{-1.15}{-1.149999}{1.15}{1.149999}

%Calculator input: plotCurve{}(\cos{}t, \sin{}(3 t), - \pi, \pi)
\parametricplot[linecolor=\fcColorGraph, plotpoints=1000]{-3.14159}{3.14159}{t 57.29578 mul cos t 3 mul 57.29578 mul sin }
\end{pspicture}
\item $x=\cos (t)+\sin (t)$ , $y=\sin (t)$.

\psset{xunit=1cm, yunit=1cm}
\begin{pspicture}(-1.814213, -1.399999)(1.81421,1.499999)
\tiny
\fcAxesStandard{-1.564213}{-1.149999}{1.56421}{1.149999}

%Calculator input: plotCurve{}(\sin{}t+\cos{}t, \sin{}t, - \pi, \pi)
\parametricplot[linecolor=\fcColorGraph, plotpoints=1000]{-3.14159}{3.14159}{t 57.29578 mul cos t 57.29578 mul sin add t 57.29578 mul sin }
\end{pspicture}
\end{enumerate}
\end{multicols}

\end{problem}
\begin{problem}
Show that the parametric curve has multiple tangents at the point and find their slopes.
\begin{multicols}{2}
\begin{enumerate}
\item $x=\cos t$, $y=2\sin (2t)$, two tangents at $(x,y)=(0,0)$.

\psset{xunit=1cm, yunit=1cm}
\begin{pspicture}(-1.4, -2.399998)(1.4,2.499998)
\tiny
\fcAxesStandard{-1.15}{-2.149998}{1.15}{2.149998}
%Calculator input: plotCurve{}(\cos{}t, 2 \sin{}(2 t), - \pi, \pi)
\parametricplot[linecolor=\fcColorGraph, plotpoints=1000]{-3.14159}{3.14159}{t 57.29578 mul cos t 2 mul 57.29578 mul sin 2 mul }
\end{pspicture}
\item $x=\cos t \sin (3t)$, $y=\sin(t)\sin (3t)$, six tangents at $(x,y)=(0,0)$.
\psset{xunit=1cm, yunit=1cm}
\begin{pspicture}(-1.280086, -1.399988)(1.4,1.0625)
\tiny
\fcAxesStandard{-1.030086}{-1.149988}{1.15}{0.7125}
%Calculator input: plotCurve{}(\cos{}t \sin{}(3 t), \sin{}t \sin{}(3 t), -2 \pi, 2 \pi)
\parametricplot[linecolor=\fcColorGraph, plotpoints=1000]{-6.28319}{6.28319}{t 3 mul 57.29578 mul sin t 57.29578 mul cos mul t 3 mul 57.29578 mul sin t 57.29578 mul sin mul }
\end{pspicture}
\item $x=\cos t, y=\sin (3t)$, find the two points at which the curve has double tangent and find the slopes of both pairs of tangents.
\psset{xunit=1cm, yunit=1cm}
\begin{pspicture}(-1.399995, -1.399999)(1.4,1.499999)
\tiny
\fcAxesStandard{-1.149995}{-1.149999}{1.15}{1.149999}

%Calculator input: plotCurve{}(\cos{}t, \sin{}(3 t), -2 \pi, 2 \pi)
\parametricplot[linecolor=\fcColorGraph, plotpoints=1000]{-6.28319}{6.28319}{t 57.29578 mul cos t 3 mul 57.29578 mul sin }
\end{pspicture}
\item $x=t^3-t^2-t+1$, $y=t^2-t-1 $, find a point where the curve has double tangent and find the slopes of the tangents.

\psset{xunit=1cm, yunit=1cm}
\begin{pspicture}(-0.9, -1.649998)(3.358221,1.5)
\tiny
\fcAxesStandard{-0.65}{-1.399998}{3.108221}{1.15}
%Calculator input: plotCurve{}(t^{3}- t^{2}- t+1, t^{2}- t-1, -1, 2)
\parametricplot[linecolor=\fcColorGraph, plotpoints=1000]{-1}{2}{ 1 t -1 mul add t 2 exp -1 mul add t 3 exp add -1 t -1 mul add t 2 exp add }
\end{pspicture}

\end{enumerate}
\end{multicols}

\end{problem}
\subsection{Curve lengths}
\begin{problem}
Find the length of the curve. 
\begin{enumerate}[ref={\fcProblemRef}]

\item \label{problemlengthy=x^2from1to2}
$y=x^2$, $x\in [1, 2]$.

\answer{$L=\sqrt{17}+\frac{1}{4} \log{}\left(\sqrt{17}+4\right)-\frac{1}{4} \log{}\left(\sqrt{5}+2\right)-\frac{\sqrt{5}}{2} \approx 3.167841$}

\item \label{problemlengthy=sqrt(x)from1to2}

$y=\sqrt{x}$, $x\in [1, 2]$.

\answer{$L= \frac{1}{8}\left( 12\sqrt{2} +\ln (17+12\sqrt{2})-4\sqrt{5}-\ln (9+4\sqrt{5})\right) \approx 1.083$}
\item \label{problemlengthx=sqrt(t)-2t,y=8/3t^(3/4)} $\displaystyle x = \sqrt{t} - 2t$ and $\displaystyle y = \frac{8}{3}t^{\frac{3}{4}}$ from $t = 1$ to $t = 4$.

\answer{$L=7$}



\item $\gamma:\left| 
\begin{array}{rcl}
x(t)&=&\frac{1}{t}+\frac{t^3}{3}\\
y(t)&=&2t\\
\end{array}\right., t\in [1,2]\quad . $

\answer{$L=\frac{17}{6}$}
\item  $\gamma:\left| 
\begin{array}{rcl}
x(t)&=&\frac{1}{t}+t\\
y(t)&=&2\ln t\\
\end{array}\right., t\in [1,2]\quad . $

\answer{$L=\frac{3}{2}$}
\item One arch of the cycloid 
\[
\gamma: \left|\begin{array}{rcl}
x(t)&=& t-\sin t  \\
y(t)&=&1-\cos t \\
\end{array} \right., t\in[0,2\pi]
\]

\begin{pspicture}(-0.5,-0.5)(8.1,2.1)
\pstVerb{5 dict begin /pi 3.141592654 def}
\fcAxesStandard{-0.5}{-0.5}{7}{2}
\fcLabels{7}{2}
\parametricplot[linecolor=red]{0}{360}{t 180 div pi mul t sin sub 1 t cos sub}
\pstVerb{end}
\end{pspicture}

\answer{$L=8$}
\item The cardioid

\[
\gamma: \left|\begin{array}{rcl}
x(t)&=&(1+\sin t)\cos t  \\
y(t)&=&(1+\sin t)\sin t \\
\end{array} \right., t\in[0,2\pi]
\]

\begin{pspicture}(-2,-1.1)(2,2.4)
\pstVerb{5 dict begin /pi 3.141592654 def}
\fcAxesStandard{-1.8}{-1}{1.8}{2.3}
\fcLabels{1.8}{2.3}
\parametricplot[linecolor=red]{0}{360}{t sin 1 add t cos mul t sin 1 add t sin mul}
\pstVerb{end}
\end{pspicture}

\answer{$L=8$}

\end{enumerate}

\end{problem}
\solution{\ref{problemlengthy=x^2from1to2}
The length of the parametric curve is given by
\[
\begin{array}{rcll|l}
L&=&\displaystyle \int_{1}^{2}\sqrt{1+\left(\frac{\diff y}{\diff x}\right)^2 }\diff x\\
&=&\displaystyle \int_{x=1}^{x=2}\sqrt{1+4x^2 }\diff x &&\begin{array}{rcl}\text{Substitute }2x&=&u\\\diff x&=&\frac{1}{2}\diff u\\ \end{array}\\
&=&\displaystyle \int_{u=2}^{u=4} \sqrt{u^2+1}\left(\frac{1}{2}\diff u\right)\\
&=&\displaystyle \frac{1}{2}\int_{u=2}^{u=4} \sqrt{u^2+1}\diff u&& 
\begin{array}{l}\displaystyle
\int \sqrt{u^2+1}\diff u \\
= \displaystyle \frac{1}{2}\left(u\sqrt{u^2+1}+\ln\left(u+\sqrt{u^2+1}\right) \right)+C\\
\text{previously studied}
\end{array}
\\
&=&\frac{1}{4} \left[ u\sqrt{u^2+1}+\ln\left(u+\sqrt{u^2+1}\right)\right]_{2}^4\\
&=&\sqrt{17}+\frac{1}{4} \log{}\left(\sqrt{17}+4\right)-\frac{1}{4} \log{}\left(\sqrt{5}+2\right)-\frac{\sqrt{5}}{2} \\
&\approx& 3.167841
\end{array}
\]

}

\solution{\ref{problemlengthy=sqrt(x)from1to2}

\textbf{Solution I.} The curve can be rewritten in the form $x=y^2$, $y\in[1, \sqrt{2}]$. 
\[
\begin{array}{rcll|l}
L&=&\displaystyle \int_{1}^{\sqrt{2}}\sqrt{\left(\frac{\diff x}{\diff y}\right)^2+1 }~\diff y  \\
&=&\displaystyle \int_{y=1}^{y=\sqrt{2}}\sqrt{4y^2+1 }~\diff y && \begin{array}{rcl}\text{Substitute }2y&=&u \\ \diff y&=&\frac{1}{2}\diff u\end{array} \\
&=&\displaystyle \int_{u=2}^{u=2\sqrt{2}}\sqrt{u^2+1 }\left(\frac{1}{2}\diff u\right)  \\
&=&\displaystyle \frac{1}{2}\int \sqrt{u^2+1}\diff u&&
\begin{array}{l}\displaystyle
\int \sqrt{u^2+1}\diff u \\
= \displaystyle \frac{1}{2}\left(u\sqrt{u^2+1}+\ln\left(u+\sqrt{u^2+1}\right) \right)+C\\
\text{previously studied}
\end{array}\\
\\
&=&\displaystyle \frac{1}{4} \left[ u\sqrt{u^2+1}+\ln\left(u+\sqrt{u^2+1}\right)\right]_{2}^{2\sqrt{2}}\\
&=&\displaystyle \frac{3}{2}\sqrt{2}+\frac{1}{4} \ln{}\left(2\sqrt{2}+3\right)-\frac{1}{4} \ln{}\left(\sqrt{5}+2\right)-\frac{\sqrt{5}}{2} \\
&\approx&1.083
\end{array}
\]


\textbf{Solution II. } The length of the parametric curve is given by
\[
\begin{array}{rcll|l}
L&=&\displaystyle \int_{1}^{2}\sqrt{1+\left(\frac{\diff y}{\diff x}\right)^2 }\diff x\\
&=&\displaystyle \int_{1}^{2}\sqrt{1+\left(\frac{1}{2\sqrt{x}}\right)^2 }\diff x\\
&=&\displaystyle \int_{x=1}^{x=2} \sqrt{1 +\frac{1}{4x}}\diff x &&\begin{array}{rcl}\text{Substitute }4x&=&u\\\diff x&=&\frac{1}{4}\diff u\\ \end{array}\\
&=&\displaystyle \int_{u=4}^{u=8} \sqrt{1 +\frac{1}{u}}\left(\frac{1}{4}\diff u\right)\\
&=&\displaystyle \frac{1}{4}\int_{4}^{8}\sqrt{\frac{u+1}{u}}\diff u\\
&=&\displaystyle \frac{1}{4}\int_{4}^{8}\sqrt{\frac{u(u+1)}{u^2}}\diff u\\
&=&\displaystyle \frac{1}{4}\int_{4}^{8}\frac{\sqrt{u^2+u }}{u}\diff u\\
&=&\displaystyle \frac{1}{4}\int_{4}^{8}\frac{\sqrt{u^2+u+\frac{1}{4}-\frac{1}{4} }}{u}\diff u\\
&=&\displaystyle \frac{1}{4}\int_{4}^{8}\frac{\sqrt{\left(u+\frac{1}{2}\right)^2-\frac{1}{4} }}{u}\diff u\\
&=&\displaystyle \frac{1}{4}\int_{4}^{8}\frac{ \sqrt{\frac{1}{4}\left( \left(2u+1\right)^2-1\right) }}{u}\diff u\\
&=&\displaystyle \frac{1}{8}\int_{u=4}^{u=8}\frac{ \sqrt{ \left(2u+1\right)^2-1 }}{u}\diff u &&\begin{array}{rcl}\text{Substitute }2u+1&=&z\\ u&=&\frac{z-1}{2}\\\diff u&=&\frac{1}{2}\diff z \end{array}\\ 
&=&\displaystyle \frac{1}{8}\int_{z=9}^{z=17}\frac{ \sqrt{ z^2-1 }}{\frac{z-1}{2}}\frac{1}{2}\diff z \\
&=&\displaystyle \frac{1}{8}\int_{z=9}^{z=17}\frac{ \sqrt{ z^2-1 }}{z-1}\diff z&&  \begin{array}{rcl}
\text{Trig. subst.: }z&=&\sec\theta \\ 
\sqrt{z^2-1}&=&\tan \theta \\
\diff z &= &\tan \theta\sec\theta \diff \theta
\end{array}\\
&=&\displaystyle \frac{1}{8}\int_{\theta=\Arcsec(9) }^{\theta=\Arcsec(17)}\frac{\tan \theta}{\sec \theta - 1} \sec \theta \tan \theta \diff \theta  && 
\begin{array}{rcl}
\text{Set }\alpha&=&\Arcsec(9)\\ 
\text{Set }\beta&=&\Arcsec(17)\\ 
\end{array}
\\
&=&\displaystyle \frac{1}{8}\int_{\alpha }^{\beta}\frac{\tan^2 \theta}{\sec \theta - 1} \sec \theta \diff \theta && \text{Use} \tan^2\theta=\sec^2\theta-1\\
&=&\displaystyle \frac{1}{8}\int_{\alpha }^{\beta}\frac{\sec^2 \theta-1}{\sec \theta - 1} \sec \theta \diff \theta \\
&=&\displaystyle \frac{1}{8}\int_{\alpha }^{\beta}\frac{(\cancel{ \sec\theta-1})(\sec \theta+1)}{\cancel{\sec \theta - 1}} \sec \theta \diff \theta \\
&=&\displaystyle \frac{1}{8}\int_{\alpha }^{\beta}\left(\sec^2\theta+ \sec \theta \right)\diff \theta && \begin{array}{l}\displaystyle\int\sec \theta\diff \theta= \ln\left|\sec\theta+\tan\theta\right|+C\\ \text{previously studied}\end{array}\\
&=&\displaystyle \frac{1}{8}\left[\tan \theta +\ln \left|\sec\theta+\tan\theta\right| \right]_{\alpha}^{\beta} && \begin{array}{rlc}\tan \theta &=& \sqrt{\sec^2\theta-1}, \theta\in\left[0,\frac{\pi}{2}\right) \\ \tan \alpha&=&\sqrt{9^2-1}=4\sqrt{5}\\ 
\tan \beta&=& \sqrt{17^2-1}=12\sqrt{2} \end{array}  \\
&=&\displaystyle \frac{1}{8}\left( 12\sqrt{2} +\ln (17+12\sqrt{2})-4\sqrt{5}-\ln (9+4\sqrt{5})\right) \\
&=&\displaystyle \frac{1}{8} \ln{}\left(12\sqrt{2}+17\right)-\frac{1}{8} \ln{}\left(4\sqrt{5}+9\right)-\frac{\sqrt{5}}{2}+\frac{3}{2}\sqrt{2} \\
&\approx& 1.083\quad.
\end{array}
\]
The two answers are both approximately 1.083, so that serves to cross verify our two solutions against one another. 

Comparing the two answers we notice that the logarithmic parts in the two answers look different (yet they must be equal). It follows that 
\[
\frac{1}{8} \ln{}\left(12\sqrt{2}+17\right)-\frac{1}{8} \ln{}\left(4\sqrt{5}+9\right) = \frac{1}{4} \ln{}\left(2\sqrt{2}+3\right)-\frac{1}{4} \ln{}\left(\sqrt{5}+2\right).
\]
A short computation (which computation?), left to the reader, confirms that indeed those two expressions are equal.

}
\solution{ \ref{problemlengthx=sqrt(t)-2t,y=8/3t^(3/4)}.
The length of the parametric curve is given by
\[
L={\displaystyle \int_{1}^4 \sqrt{\left(\frac{\diff x}{\diff t}\right)^2 + \left( \frac{\diff y}{\diff t} \right)^2}  \diff t}\quad .
\]
We have that 
\[
\begin{array}{rclll}
\displaystyle \frac{\diff x}{\diff t} &=&\displaystyle  \frac{1}{2\sqrt{t}} - 2\\
\displaystyle \frac{\diff y}{\diff t} &=&\displaystyle  2t^{-\frac{1}4}\\
\displaystyle \left(\frac{\diff x}{\diff t}\right)^2 &=&\displaystyle  \frac{1}{4t} - \frac{2}{\sqrt{t}} + 4\\
\displaystyle \left(\frac{\diff y}{\diff t}\right)^2 &=&\displaystyle  4t^{-\frac{1}{2}} = \frac{4}{\sqrt{t}}\\
\displaystyle \left(\frac{\diff x}{\diff t}\right)^2+\left(\frac{\diff y}{\diff t}\right)^2 & =&\displaystyle  \frac{1}{4t} + 2\frac{1}{\sqrt{t}} + 4 = \left(\frac{1}{2\sqrt{t}} + 2\right)^2\quad .
\end{array}
\]

$\frac{1}{2\sqrt{t}} +2$ is positive and $\sqrt{\left(\frac{ 1}{2 \sqrt{t}} +2\right)^2} =\frac{1}{2\sqrt{t}} +2$. So the integral becomes 
\[\displaystyle 
L= \int_1^4 \left(\frac{1}{2\sqrt{t}} +2\right)  \diff t=\left[\sqrt{t} + 2t\right]_{t=1}^{t=4}=(2+8)-(1+2)=7\quad .
\]
}



\begin{problem}
Set up an integral that expresses the length of the curve and find the length of the curve.
\begin{enumerate}
\item
$
\left|
\begin{array}{rcl}
x(t)&=&e^t+e^{-t}\\
y(t)&=&5-2t\\
\end{array}\right., t\in [0,3]
$

\psset{xunit=0.3cm, yunit=0.3cm, algebraic=false}
\begin{pspicture}(-0.9, -1.388012)(20.415591,5.5)
\tiny
\fcAxesStandard{-0.65}{-1.138012}{20.165591}{5.15}
%Calculator input: plotCurve{}(e^{- t}+e^{t}, -2 t+5, 0, 3)
\parametricplot[linecolor=\fcColorGraph, plotpoints=1000 ]{0}{3}{2.718281828 t exp 2.718281828 t -1 mul exp add 5 t -2 mul add}
\end{pspicture}

\answer{ $ e^3-e^{-3}$}

\item
$
\left|
\begin{array}{rcl}
x(t)&=&\sin t +\cos t\\
y(t)&=&\sin t-\cos t\\
\end{array}\right., t\in [0,\pi]
$

\psset{xunit=1cm, yunit=1cm}
\begin{pspicture}(-1.81421, -1.814213)(1.814213,1.914213)
\tiny
\fcAxesStandard{-1.56421}{-1.564213}{1.564213}{1.564213}
%Calculator input: plotCurve{}(\sin{}t+\cos{}t, \sin{}t- \cos{}t, 0, 2 \pi)
\parametricplot[linecolor=\fcColorGraph, linestyle=dashed, plotpoints=1000]{0}{6.28319}{t 57.29578 mul cos t 57.29578 mul sin add t 57.29578 mul cos -1 mul t 57.29578 mul sin add }
%Calculator input: plotCurve{}(\sin{}t+\cos{}t, \sin{}t- \cos{}t, 0, \pi)
\parametricplot[linecolor=\fcColorGraph, plotpoints=1000]{ 0}{3.14159}{t 57.29578 mul cos t 57.29578 mul sin add t 57.29578 mul cos -1 mul t 57.29578 mul sin add}
\end{pspicture}

\answer{ $\sqrt{2} \pi$}

\end{enumerate}

\end{problem}
\subsection{Area under curve}
\begin{problem}
Give a geometric definition of the cycloid curve using a circle of radius 1. Using that definition, derive equations for the cycloid curve. Find area locked between one ``arch'' of the cycloid curve and the $x$ axis.



\end{problem}
\subsection{Area locked by  curve}
\begin{problem}
\begin{enumerate}
\item \label{problem-Area-swept-by-r=1+sin2theta} The curve given in polar coordinates by $r=1+\sin 2\theta$ is plotted below by computer. Find the area lying outside of this curve and inside of the circle $x^2+y^2=1$.
\psset{xunit=1cm, yunit=1cm}
\begin{pspicture}(-2.016386, -2.016424)(2.016407,2.116335) 
\tiny 
\psaxesStandard{-1.766386}{-1.766424}{1.766407}{1.766335}
%Calculator command: drawPolar{}(\sin{}(2 t)+1, 0, 2 \pi) 
\parametricplot[linecolor=\psColorGraph, plotpoints=1000, algebraic=false]{0}{6.28319}{ 1 t 2 mul 57.29578 mul sin add t 57.29578 mul cos mul 1 t 2 mul 57.29578 mul sin add t 57.29578 mul sin mul }
\end{pspicture} 
\answer{$a=2-\frac{\pi}{4}$}
\item \label{problem-Area-swept-by-r=cos2theta} The curve given in polar coordinates by $r=\cos (2\theta)$ is plotted below by computer. Find the area lying inside the curve and outside of the circle $x^2+y^2=\frac14$.
\psset{xunit=1cm, yunit=1cm}
\begin{pspicture}(-1.399902, -1.399975)(1.4,1.499975) 
\tiny 
\psaxesStandard{-1.149902}{-1.149975}{1.15}{1.149975}
%Calculator command: drawPolar{}(\cos{}(2 t), 0, 2 \pi) 
\parametricplot[linecolor=\psColorGraph, plotpoints=1000, algebraic=false]{0}{6.28319}{t 2 mul 57.29578 mul cos t 57.29578 mul cos mul t 2 mul 57.29578 mul cos t 57.29578 mul sin mul }
\end{pspicture}

%\item The curve given in polar coordinates by $r=1+\cos (3t)$ is plotted below. Find the area \textbf{outside} of the curve and \textbf{inside} the circle $x^2+y^2=\frac14$.

%\begin{pspicture}(-1.611991, -2.182513)(2.4,2.282513) \tiny \psaxesStandard{-1.361991}{-1.932513}{2.15}{1.932513}
%Calculator command: drawPolar{}(\cos{}(3 t)+1, 0, 2 \pi) 
%\parametricplot[linecolor=\psColorGraph, plotpoints=1000, algebraic=false]{0}{6.28319}{ 1 t 3 mul 57.29578 mul cos add t 57.29578 mul cos mul 1 t 3 mul 57.29578 mul cos add t 57.29578 mul sin mul }
%\end{pspicture} 
 
\end{enumerate}


\end{problem}
\solution{\ref{problem-Area-swept-by-r=1+sin2theta}. A computer generated plot of the two curves is included below. The circle $x^2+y^2=1$ has one-to-one polar representation given by $r=1, \theta\in [0,2\pi)$. Except the origin, which is traversed four times by the curve $r=1+\sin (2\theta)$, the second curve is in a one-to-one correspondence with points in the $r,\theta$-plane given by the equation $r=1+\sin (2\theta), \theta\in [0,2\pi)$. Since the two curves do not meet in the origin, we may conclude that the two curves may intersect only when their values for $r$ and $\theta$ coincide. Therefore we have an intersection when 
\[\begin{array}{rcll|l}
1+\sin (2\theta)&=&1\\
\sin (2\theta)&=&0\\
\theta &=& 0,\frac{\pi}{2}, \pi, \frac{3\pi}{2}&&\text{because } \theta\in [0,2\pi) \\
\end{array}
\]
Therefore the two curves meet in the points $(0,1)(-1,0)$ and $(0,-1),(1,0)$. 

Denote the investigated region by $A$. From the computer-generated plot, it is clear that when a point has polar coordinates $\theta\in [\frac{\pi}{2}, \pi] \cup[\frac{3\pi}{2}, 2\pi]$, $r\in [1+\sin(2\theta),1]$ it lies in $A$. Furthermore, the points $r,\theta$ lying in the above intervals are in one-to-one correspondence with the points in $A$. 

Suppose we have a curve $r=f(\theta), \theta\in [a,b]$ for which no two points lie on the same ray from the origin. Recall from theory that the area swept by that curve is given by
\[
\int\limits_{a}^b\frac{1}{2} f^2(\theta)\diff \theta\quad .
\]

Therefore the area $a$ of $A$ is computed via the integrals
\[
\begin{array}{rcll|l}
a&=&\displaystyle \int\limits_{\frac{\pi}{2}}^{\pi} \frac{1}{2} \left( {\underbrace{ 1}_{\text{inner curve}}}^2- \left(\underbrace{1+\sin(2\theta)}_{\text{outer curve}}\right)^2 \right)\diff \theta + \int \limits_{ \frac{3\pi}{2}}^{ 2\pi} \frac{1}{2} \left(1^2- (1+\sin(2\theta) )^2 \right) \diff \theta &&\text{use the symmetry of } A\\
&=&\displaystyle  \int\limits_{\frac{\pi}{2}}^{\pi} \left(1^2-(1+\sin(2\theta))^2\right)\diff \theta= \int\limits_{\frac{\pi}{2}}^{\pi} \left( - 2\sin(2\theta) - \sin^2(2\theta)\right) \diff \theta  &&\text{use } \sin^2 z=\frac{1-\cos (2z)}{2} \\
&=&\displaystyle  \int\limits_{\frac{\pi}{2}}^{\pi}  \left( -2\sin(2\theta) -\frac{1}{2} +\frac{1}{2}\cos (4\theta)\right)\diff \theta = \left[\cos (2\theta) -\frac{1}{2}\theta -\frac{1}{8}\sin (4\theta) \right]_{\frac{\pi}{2}}^{\pi} \\
&=&2-\frac{\pi}{4}\quad .
\end{array}
\]

\psset{xunit=1cm, yunit=1cm}
\begin{pspicture}(-2.016386, -2.016424)(2.016407,2.116335) 
\tiny 
\psaxesStandard{-1.766386}{-1.766424}{1.766407}{1.766335}
\pscustom*[linecolor=\psColorAreaUnderGraph]{
%Calculator command: drawPolar{}(1, 1/2 \pi, \pi) 
\parametricplot[linecolor=\psColorGraph, plotpoints=1000, algebraic=false]{1.5708}{3.14159}{ 1 t 57.29578 mul cos mul 1 t 57.29578 mul sin mul }
%Calculator command: drawPolar{}(\sin{}(2 t)+1, 1/2 \pi, \pi) 
\parametricplot[linecolor=\psColorGraph, plotpoints=1000, algebraic=false]{1.5708}{3.14159}{ 1 t 2 mul 57.29578 mul sin add t 57.29578 mul cos mul 1 t 2 mul 57.29578 mul sin add t 57.29578 mul sin mul }
} %pscustom
\pscustom*[linecolor=\psColorAreaUnderGraph]{
%Calculator command: drawPolar{}(\sin{}(2 t)+1, -1/2 \pi, 0) 
\parametricplot[linecolor=\psColorGraph, plotpoints=1000, algebraic=false]{-1.5708}{0}{ 1 t 2 mul 57.29578 mul sin add t 57.29578 mul cos mul 1 t 2 mul 57.29578 mul sin add t 57.29578 mul sin mul }
%Calculator command: drawPolar{}(1, -1/2 \pi, 0) 
\parametricplot[linecolor=\psColorGraph, plotpoints=1000, algebraic=false]{-1.5708}{0}{ 1 t 57.29578 mul cos mul 1 t 57.29578 mul sin mul }
} %pscustom

%Calculator command: drawPolar{}(1, 0, 2 \pi) 
\parametricplot[linecolor=\psColorGraph, plotpoints=1000, algebraic=false]{0}{6.28319}{ 1 t 57.29578 mul cos mul 1 t 57.29578 mul sin mul }
%Calculator command: drawPolar{}(\sin{}(2 t)+1, 0, 2 \pi) 
\parametricplot[linecolor=\psColorGraph, plotpoints=1000, algebraic=false]{0}{6.28319}{ 1 t 2 mul 57.29578 mul sin add t 57.29578 mul cos mul 1 t 2 mul 57.29578 mul sin add t 57.29578 mul sin mul }
\end{pspicture} 
}
\solution{\ref{problem-Area-swept-by-r=cos2theta} A computer generated plot of the figure is included below. The circle $x^2+y^2=\frac{1}{4} $ is centered at $0$ and of radius $\frac{1}{2}$ and therefore can be parametrized in polar coordinates via $r=\frac{1}{2}, \theta\in [0, 2\pi]$.

Points with polar coordinates $(r_1, \theta_1) $ and $(r_2,\theta_2)$ coincide if one of the three holds:
\begin{itemize}
\item[$\bullet$] $r_1=r_2\neq 0$ and $\theta_1=\theta_2+2k\pi, k\in \mathbb Z $,
\item[$\bullet$] $r_1=-r_2\neq 0$ and $\theta_1=\theta_2+(2k+1)\pi, k\in \mathbb Z$,
\item[$\bullet$] $r_1=r_2=0 $ and $\theta$ is arbitrary.
\end{itemize}
To find the intersection points of the two curves we have to explore each of the cases above. The third case is not possible as the circle does not pass through the origin. Suppose we are in the first case. Then the value of $r$ (as a function of $\theta$)  is equal for the two curves. Thus the two curves intersect if 
\[
\begin{array}{rcll|l}
r=\cos (2\theta)&=&\frac12\\
2\theta&=& \pm\frac{\pi}{3}+2k\pi&&\text{where }k\in \mathbb Z\\
\theta &=& \pm\frac{\pi}{6}+k\pi &&\text{where }k\in \mathbb Z\\
\theta &=& \frac{\pi}{6}, \frac{\pi}{6}+\pi, -\frac{\pi }{6}+\pi, -\frac{\pi }{6}+2\pi &&\text{all other values discarded as }\theta\in [0,2\pi]\\
\theta&=&\frac{\pi}{6}, \frac{7\pi}{6}, \frac{5\pi}{6}, \frac{11\pi}{6}
\end{array}
\]
This gives us only four intersection points, and the computer-generated plot shows eight. Therefore the second case must occur as well: the two curves intersect also when 
\[
\begin{array}{rcll|l}
r=\cos (2\theta)&=&-\frac{1}{2}\\
2\theta &=& \pm \frac{2\pi}{3} +2k\pi &&\text{where } k\in \mathbb Z\\
\theta &=& \pm \frac{\pi}{3} +k\pi &&\text{where } k\in \mathbb Z\\
\theta&=& \frac{\pi }{3}, \frac{\pi}{3}+\pi, \frac{-\pi}{3} +\pi, \frac{-\pi}{3}+2\pi &&\text{all other values are discarded as }\theta \in [0,2\pi]\\
\theta&=&\frac{\pi}{3}, \frac{4\pi}3, \frac{2\pi}{3}, \frac{5\pi}{3}  \quad .
\end{array}
\]
From the computer-generated plot below, we can see that the area we are looking for is 4 times the area locked between the two curves for $\theta\in \left[\frac{-\pi}{6}, \frac{\pi}{6}\right] $. Therefore the area we are looking for is given by
\[
4\int\limits_{-\frac{\pi}{6}}^{\frac{\pi}{6}} \frac{1}{2}\left(\cos^2(2\theta)-\left(\frac{1}{2}\right)^2 \right)\diff \theta\quad .
\]
We leave the above integral to the reader.
\psset{xunit=2cm, yunit=2cm}
\begin{pspicture}(-1.399902, -1.399975)(1.4,1.499975) 
\tiny 
\pscustom*[linecolor=\psColorAreaUnderGraph]{
%Calculator command: drawPolar{}(1/2, 1/6 \pi, -1/6 \pi) 
\parametricplot[linecolor=\psColorGraph, plotpoints=1000, algebraic=false]{0.523599}{-0.523599}{ 0.5 t 57.29578 mul cos mul 0.5 t 57.29578 mul sin mul }
%Calculator command: drawPolar{}(\cos{}(2 t), -1/6 \pi, 1/6 \pi) 
\parametricplot[linecolor=\psColorGraph, plotpoints=1000, algebraic=false]{-0.523599}{0.523599}{t 2 mul 57.29578 mul cos t 57.29578 mul cos mul t 2 mul 57.29578 mul cos t 57.29578 mul sin mul }
}
\pscustom*[linecolor=\psColorAreaUnderGraph]{
%Calculator command: drawPolar{}(1/2, 5/3 \pi, 4/3 \pi) 
\parametricplot[linecolor=\psColorGraph, plotpoints=1000, algebraic=false]{5.23599}{4.18879}{ 0.5 t 57.29578 mul cos mul 0.5 t 57.29578 mul sin mul }
%Calculator command: drawPolar{}(\cos{}(2 t), 1/3 \pi, 2/3 \pi) 
\parametricplot[linecolor=\psColorGraph, plotpoints=1000, algebraic=false]{1.0472}{2.0944}{t 2 mul 57.29578 mul cos t 57.29578 mul cos mul t 2 mul 57.29578 mul cos t 57.29578 mul sin mul }
}
\pscustom*[linecolor=\psColorAreaUnderGraph]{
%Calculator command: drawPolar{}(1/2, 7/6 \pi, 5/6 \pi) 
\parametricplot[linecolor=\psColorGraph, plotpoints=1000, algebraic=false]{3.66519}{2.61799}{ 0.5 t 57.29578 mul cos mul 0.5 t 57.29578 mul sin mul }
%Calculator command: drawPolar{}(\cos{}(2 t), 5/6 \pi, 7/6 \pi) 
\parametricplot[linecolor=\psColorGraph, plotpoints=1000, algebraic=false]{2.61799}{3.66519}{t 2 mul 57.29578 mul cos t 57.29578 mul cos mul t 2 mul 57.29578 mul cos t 57.29578 mul sin mul }
}
\pscustom*[linecolor=\psColorAreaUnderGraph]{
%Calculator command: drawPolar{}(1/2, 2/3 \pi, 1/3 \pi) 
\parametricplot[linecolor=\psColorGraph, plotpoints=1000, algebraic=false]{2.0944}{1.0472}{ 0.5 t 57.29578 mul cos mul 0.5 t 57.29578 mul sin mul }
%Calculator command: drawPolar{}(\cos{}(2 t), 4/3 \pi, 5/3 \pi) 
\parametricplot[linecolor=\psColorGraph, plotpoints=1000, algebraic=false]{4.18879}{5.23599}{t 2 mul 57.29578 mul cos t 57.29578 mul cos mul t 2 mul 57.29578 mul cos t 57.29578 mul sin mul }
}
\parametricplot[linecolor=\psColorGraph, plotpoints=1000, algebraic=false]{0}{6.28319}{ 0.5 t 57.29578 mul cos mul 0.5 t 57.29578 mul sin mul }
%Calculator command: drawPolar{}(\cos{}(2 t), 0, 2 \pi) 
\parametricplot[linecolor=\psColorGraph, plotpoints=1000, algebraic=false]{0}{6.28319}{t 2 mul 57.29578 mul cos t 57.29578 mul cos mul t 2 mul 57.29578 mul cos t 57.29578 mul sin mul }
\psaxes[ticks=none, labels=none, arrows = <->](0,0)(-1.149902,-1.149975)(1.15,1.149975)
\psLabels{1.15}{1.149975}
\end{pspicture} 
%Calculator command: drawPolar{}(1/2, 0, 2 \pi) 

}


\begin{problem}
\begin{enumerate}
\item
Sketch the graph of the curve given in polar coordinates by $r=3\sin (2\theta)$ and find the area of one petal.

\answer{$\frac{9\pi}{8}$, curve sketch: \psset{xunit=0.1cm, yunit=0.1cm}
\begin{pspicture}(-2.709395, -2.709285)(2.709395,2.809368) 
\tiny 
\psaxesStandard{-2.459395}{-2.459285}{2.459395}{2.459368}
%Calculator command: drawPolar{}(3 \sin{}(2 t), 0, 2 \pi) 
\parametricplot[linecolor=\psColorGraph, plotpoints=1000, algebraic=false]{0}{6.28319}{t 2 mul 57.29578 mul sin 3 mul t 57.29578 mul cos mul t 2 mul 57.29578 mul sin 3 mul t 57.29578 mul sin mul }
\end{pspicture} }

\item Sketch the graph of the curve given in polar coordinates by $r=4+3\sin \theta$ and find the area enclosed by the curve. 

\answer{$\frac{41\pi}{2}$ , curve sketch: \psset{xunit=0.1cm, yunit=0.1cm}
\begin{pspicture}(-5.177597, -1.733324)(5.177712,7.499951) 
\tiny 
\psaxesStandard{-4.927597}{-2.5}{4.927712}{7.149951}
%Calculator command: drawPolar{}(3 \sin{}t+4, 0, 2 \pi) 
\parametricplot[linecolor=\psColorGraph, plotpoints=1000, algebraic=false]{0}{6.28319}{ 4 t 57.29578 mul sin 3 mul add t 57.29578 mul cos mul 4 t 57.29578 mul sin 3 mul add t 57.29578 mul sin mul }
\end{pspicture} }
\end{enumerate}
\end{problem}
\section{A Bit of Differential Equations}
\subsection{Separable Differential equations}
\subsubsection{The Mixing Problem}
\begin{problem}
\begin{enumerate}
\item \label{problemMixingProblem1}
A tank contains 30 kg of salt dissolved in water to form $10000$ liters of solution. Brine that contains $0.05$ kg of salt per liter enters the tank at a rate of 10 liters per minute. The solution is kept thoroughly mixed and drains from the tank at the same rate (10 liters per minute). Determine how much salt remains in the tank after half an hour.
\item \label{problemMixingProblem2} A tank contains $1000$ kg of salt dissolved in water to form $10000$ liters of solution. Brine that contains $0.01$ kg of salt per liter of water enters the tank at a rate of $30$ liters per minute. The solution is kept thoroughly mixed and drains from the tank at the same rate ($30$ liters per minute). 
\begin{enumerate} 
\item Determine how much salt remains in the tank after an hour. 
\item How long should the procedure continue  so that the solution in the tank gets to a salt concentration of $0.101$ kg/L? 
\end{enumerate}
\end{enumerate}


\solution{\ref{problemMixingProblem1}


}
\end{problem}
\solution{\ref{problemDFQseparable-mixing-problem-1}. Let 
\[
y(t)=\text{salt in the tank after } t \text{ minutes (in kg)}\quad .
\] 
We are given $y(0)= 30$kg, the initial amount of salt. We are looking to find $y(45)$, the amount of salt after $45$ minutes. We have that 
\[
\frac{\diff y}{\diff t}= \text{(rate in)} - \text{(rate~out)} \quad .
\]
The rate of salt entering the tank is constant: 
\[
\text{(rate in)}=0.05 kg/L \cdot 10 L/min= 0.5 kg/min\quad .
\] 
As the solution is thoroughly mixed, at any time the concentration of salt in the tank is 
\[
\displaystyle \frac{y}{ 10000} kg/L.
\] 
Therefore the rate of salt going out of the tank is 
\[
\text{(rate out)}=\frac{y}{10000} kg/L * 10 L/min = \frac{y}{1000} kg/min\quad .
\] 
Therefore the differential equation for the amount of salt in the tank is
\[
\frac{\diff y}{\diff t}= \underbrace{ 0.5}_{\text{(rate  in)}}- \underbrace{ \frac{y}{1000} }_{\text{(rate out)}}\quad .
\]
To find $y(45)$, we integrate from $t=0$ to $t=45$:
\[
{\renewcommand{\arraystretch}{1.5}
\begin{array}{rcll|l}
\displaystyle \int\limits_{t=0}^{45} \frac{1000}{500- y} \underbrace{ \frac{\diff y}{\diff t} \diff t}_{\diff (y(t))} &=& \displaystyle \int\limits_{t=0}^{45} \diff t \\
\displaystyle \int\limits_{t=0}^{t=45} \frac{1000}{500-y(t)}\diff (y(t))&=& 45&&\text{set }z=y(t) \\
\displaystyle -1000 \int \limits_{z=y(0)=30}^{z=y(45)} \frac{1}{500-z}d(500-z)&=& 45 \\
\displaystyle \left. -1000 \ln |500-y| \right]_{y(0)=30}^{y(45)}&=& 45 \\
\displaystyle -1000 \left( \ln |500-y(45)|\right. \\
\left. -\ln |500- 30|  \right) &=& 45 \\
\displaystyle \ln \left| \frac{470}{500-y(45)} \right|  &=& \displaystyle \frac{45}{1000}
\\
\displaystyle \ln \left( \frac{470}{500-y(45)} \right)  &=& \displaystyle \frac{45}{1000} &&\text{see below}
\\
\displaystyle \frac{470}{500-y(30)}&=&\displaystyle e^{\frac{45}{1000}}\\
\displaystyle 500-y(30)&=&\displaystyle  470e^{-\frac{9}{200}}\\
\displaystyle y(30)&=&\displaystyle 500-470e^{-\frac{9}{200}}\\
&\approx& 500-470\cdot 0.955997 \\
&\approx& 50.681184 \quad ,
\end{array}
}
\]
where we have used that $\displaystyle \frac{470}{500-y(t)}>0 $. The fact that $\displaystyle \frac{470}{500-y(t)}>0 $ can be seen as follows. As $500-y(0)=470>0$ and $y(t)$ is continuous, in order to have $500-y(t)<0$ there must exist some $x_1$ for which $y(x_1)=500$. However this is impossible since $\displaystyle x=\ln \left|\frac{470}{500-y(x)}\right|  $. 

As the unit of measurement is $kg$, the final answer to the problem is $\approx 50.68 kg$ salt.
}

\begin{problem}
Mixing problem. A tank contains $1000$ kg of salt dissolved in 10000 liters of water. Brine that contains $0.05$ kg of salt per liter of water enters the tank at a rate of $30$ liters per minute. The solution is kept thoroughly mixed and drains from the tank at the same rate ($30$ liters per minute). 


\begin{enumerate}
\item Determine how much salt remains in the tank after an hour. The answer key has not been proofread, use with caution.

\answer{$\displaystyle 500+ 500 e^{-0.18}\approx 917.64kg$}
\item Determine how much time will be needed in order to have the concentration of salt in the tank reach $0.0501$kg/liter. The answer key has not been proofread, use with caution.

\answer{$\frac{1000}{3}\ln 500\approx 2071.54min\approx 34.53 hours$}
\end{enumerate}

\end{problem}

\subsubsection{General Separable Problems}
\begin{problem}
\begin{enumerate}

\item \label{problemDFQseparable-yprime=ysquared-1}
\begin{equation}\label{eqDFQseparable-yprime=ysquared-1}
\frac{\diff y}{\diff x}= y^2-1\quad .
\end{equation}
\begin{enumerate}
\item \label{problemDFQseparable-yprime=ysquared-1-part1} Find all solutions of the differential equation above.
\item \label{problemDFQseparable-yprime=ysquared-1-part2} Find a solution for which $y(0)=-\frac{3}{5}$.
\end{enumerate}

\item 
\begin{enumerate}
\item Find the general solution to the differential equation 
\[
\frac{\diff y}{\diff x}= y^2-4\quad .
\]
The drawing below is a computer-generated plot of the direction field  $\displaystyle \frac{dy}{dx}=y^2-4$, you may use it to get a feeling for what your answer should look like.

\begin{pspicture}(-6,-6)(6,6)
\newcommand{\Dconst}{4}
\psplot[linecolor=green]{-4}{4}{1 \Dconst\space 2.718281828 4 x mul exp mul sub 1 \Dconst\space 2.718281828 4 x mul exp mul add div 2 mul} 

\psaxes{<->}(0,0)(-6,-2)(6,6)
\rput (5,5){The direction field  $\frac{dy}{dx}=y^2-4$}
  \psset{arrows=->}
  \multido{\ra=-4+0.5}{17}{%
    \multido{\rb=-4+0.5}{17}{%
      \pstVerb{/xC \ra\space def
               /yC \rb\space def
               /F  yC yC mul 4 sub \space def
}
%\psline[linecolor=blue](! xC  yC )(! xC yC)
\psdot[linecolor=red!60](! xC yC)
\psline[linecolor=blue](! xC F ATAN 57.295 mul cos 0.2 mul sub yC F ATAN 57.295 mul sin 0.2 mul sub)(! xC F ATAN 57.295 mul cos 0.2 mul add yC F ATAN 57.295 mul sin 0.2 mul add )
}}
\end{pspicture}

\item  Find a solution of the above equation for which $ y(0)= -\frac{6}{5}$. 
\end{enumerate}
\end{enumerate}

\solution{\ref{problemDFQseparable-yprime=ysquared-1}
(\ref{problemDFQseparable-yprime=ysquared-1-part1}). We proceed as explained in the theory.

\noindent Case 1. Suppose there exists a number $x_0$ such that $(y(x_0) )^2 - 1\neq 0$. Since $y$ is a differentiable function of $x$, it is also continuous. Therefore for some $t$ sufficiently close to $x_0$, all numbers $x$ in the interval between $t$ and $x_0$ satisfy $ y(x)^2-1\neq 0$.
\[
\begin{array}{rcll|l}
\displaystyle \frac{\frac{ \diff y}{ \diff x}}{y^2-1}&=&1 \\
\displaystyle\int\limits_{x=x_0}^{x=t} \frac{1}{ y^2- 1} \underbrace{ \frac{\diff y}{\diff x}\diff x}_{=\diff (y(x)) }&=&\displaystyle\int\limits_{x=x_0}^{x=t}\diff x &&\text{can integrate as }  y(x)^2-1\neq 0\\
\displaystyle\int\limits_{t=x_0 }^{x=t} \frac{\diff (y(x))}{ (y(x))^2-1}& =& \displaystyle\left.x \right|_{ x=x_0}^{x=t} &&\text{set } z=y(x)\\
\displaystyle\int\limits_{z=y(x_0) }^{z=y(t)} \frac{\diff z}{ z^2-1}& =& \displaystyle t-x_0 \\
\displaystyle\int\limits_{z=y(x_0)}^{z=y(t)} \left(\frac{\frac12 }{z-1}- \frac{\frac12}{z+1}\right)\diff z&=& t-x_0\\
\displaystyle\left .\frac{1}2 \ln \left|\frac{z-1}{z+1}\right|\right]_{z=y(x_0)}^{z=y(t)}&=& t-x_0\\
\displaystyle \ln \left|\frac{y(t)-1}{y(t)+1}\right|&=& 2t - C&&\text{relabel dummy variable } t \text { to } x \\
\displaystyle
\ln \left|\frac{y(x)-1}{y(x)+1}\right|&=& 2x - C
\end{array}
\]
where we have set 
\[\displaystyle C=2x_0-  \ln \left|\frac{y(x_0)-1}{y(x_0)+1}\right| \quad .
\] 
Set  
\[
D:=e^{-C}\quad .
\] 
By the assumption of our case, $ (y(x_0))^2-1\neq 0$, so there are two remaining cases: $ (y(x_0))^2-1>0$ and $ (y(x_0))^2-1<0$.

\noindent Case 1.1. Suppose $\displaystyle \frac{y(x_0)-1}{ y(x_0)+1}>0$. As the function $y(x)$ is differentiable, it is also continuous. Therefore $\displaystyle \frac{y(x)-1}{y(x)+1}>0$ for all $x$ near $x_0$. Then we can remove the absolute values around from the equality above to get that for all $x$ close to $x_0$ we have that
\[
\begin{array}{rcl}
\displaystyle \ln \left(\frac{y(x)-1}{y(x)+1}\right)&=& 2x - C\\
\displaystyle \frac{y(x)-1}{y(x)+1}&=& D e^{2x}\\
\displaystyle y(x)-1&=&\displaystyle  De^{2x}(y(x)+1)\\
\displaystyle y(x)\left(1- De^{2x}\right)&=&\displaystyle  De^{2x}+1\\
\displaystyle y(x)&=&\displaystyle  \frac{ 1+De^{2x}}{1- De^{2x}}\quad .\\
\end{array}
\]
The solution $y(x)$ given above satisfies $\displaystyle \frac{y(x)-1}{y(x)+1}= De^{2x}$ for all $x$. As $D>0$, this implies that $\displaystyle \frac{y(x)-1}{ y(x)+1}>0$. Therefore the considerations above are valid for all $x$, rather than only for those $x$ near $x_0$. Therefore our first case yields the solution
\[
y(x)=\frac{ 1+De^{2x}}{1- De^{2x}}\quad .
\]

\noindent Case 1.2. Suppose  $\displaystyle \frac{y(x_0) -1}{y(x_0) +1} <0$. Then for all $x$ near $x_0$ we get $\displaystyle \ln \left| \frac{y(x) -1}{y(x) +1}\right|= \ln \left( \frac{ 1- y(x) }{ y( x) +1}\right)$ and, similarly to Case 1, we get 
\[
\begin{array}{rcl}
\displaystyle \frac{1-y(x)}{y(x)+1}&=& D e^{2x}\\
1-y(x)&=& De^{2x}(y(x)+1)\\
y(x)\left(1+ De^{2x}\right)&=& 1-De^{2x}\\
y(x)&=&\displaystyle \frac{1- De^{2x}}{1+ De^{2x}}\quad .
\end{array}
\]
Since $D$ is a positive constant, we conclude in a fashion analogous to Case 1 that $y(x)<0$ for all $ x$.

Case 2.  Suppose $\displaystyle  (y(x_0))^2-1=0 $.  Then $y(x_0)=\pm 1$. Clearly the constant functions $y(x)= \pm 1$ are two solutions: if we can plug back $y=\pm 1$ in the original equation we get that $\frac{\diff y}{\diff x}= 0$ and $y$ is a constant function of $x$. From the preceding two cases we know that if $\frac{y(x) -1}{y(x) +1}$ is defined and not equal to zero for some value of $x$, then $\frac{y(x)-1}{y(x)+1}$ is defined and not equal to zero for all values of $x$. Therefore the present case yields only two solutions, the constant functions $y(x)=\pm 1$. 

Our final answer is 
\[
y(x)= \frac{1+De^{2x}}{1-De^{2x}} \quad \text{ or }\quad y(x)=0,
\]
where $D$ is an arbitrary rea(l number. Notice that in the above answer, we have combined Cases 1.1, 1.2 and the case $y(x)=1 $: by allowing $D$ to be negative we included Case 1.2 and by allowing $D $ to be zero we included the case $y(x)=1$. Finally, we note that if we let $D\to \infty$, the solution $y(x) = \frac{1+De^{2x }}{ 1- De^{2x}}  $ tends to the solution $y(x)=-1$ (for all values of $x$).

We may plot solutions for a few values of $D$ as follows. We overlay the solutions on top of the direction field of the differential equation. The picture tells us a lot about the properties of the solutions of the differential equations. 
%\optionalDisplay{
\begin{pspicture}(-6,-6)(6,6)
\directionField{}

\newcommand{\Dconst}{1}
\psplot[linecolor=green]{-4}{4}{1 \Dconst\space 2.718281828 2 x mul exp mul sub 1 \Dconst\space 2.718281828 2 x mul exp mul add div} 
\renewcommand{\Dconst}{0.25}
\psplot[linecolor=green]{-4}{4}{1 \Dconst\space 2.718281828 2 x mul exp mul sub 1 \Dconst\space 2.718281828 2 x mul exp mul add div} 
\renewcommand{\Dconst}{4}
\psplot[linecolor=green]{-4}{4}{1 \Dconst\space 2.718281828 2 x mul exp mul sub 1 \Dconst\space 2.718281828 2 x mul exp mul add div} 
\rput[l](5,2 ){$\frac{1- \frac{1}4 e^{2x}}{1+\frac 14 e^{2x}}$ }
\rput[l](5,0.5 ){$\frac{1- e^{2x}}{1+e^{2x}}$ }
\rput[l](5,-2 ){$\frac{1- 4e^{2x}}{1+4e^{2x}}$ }
\psline[arrows=->, linestyle=dotted](5,2)(0,0.6)
\psline[arrows=->, linestyle=dotted](5,0.5)(0,0)
\psline[arrows=->, linestyle=dotted](5,-2)(0,-0.6)

\renewcommand{\Dconst}{1}
\psplot[linecolor=green]{-4}{-0.17}{1 \Dconst\space 2.718281828 2 x mul exp mul add 1 \Dconst\space 2.718281828 2 x mul exp mul sub  div} 
\psplot[linecolor=green]{0.17}{4}{1 \Dconst\space 2.718281828 2 x mul exp mul add 1 \Dconst\space 2.718281828 2 x mul exp mul sub  div} 
\renewcommand{\Dconst}{4}

\psplot[linecolor=green]{-4}{-0.863147181}{1 \Dconst\space 2.718281828 2 x mul exp mul add 1 \Dconst\space 2.718281828 2 x mul exp mul sub  div} 
\psplot[linecolor=green]{-0.523147181}{4}{1 \Dconst\space 2.718281828 2 x mul exp mul add 1 \Dconst\space 2.718281828 2 x mul exp mul sub  div} 

\renewcommand{\Dconst}{0.25}
\psplot[linecolor=green]{-4}{0.523147181}{1 \Dconst\space 2.718281828 2 x mul exp mul add 1 \Dconst\space 2.718281828 2 x mul exp mul sub  div} 
\psplot[linecolor=green]{0.863147181}{4}{1 \Dconst\space 2.718281828 2 x mul exp mul add 1 \Dconst\space 2.718281828 2 x mul exp mul sub  div} 
\rput[r](-5,0.5 ){$\frac{1+\frac 14 e^{2x}}{1- \frac{1}4 e^{2x}}$ }
\rput[r](-5,2 ){$\frac{1+e^{2x}}{1- e^{2x}}$ }
\rput[r](-5,-2 ){$\frac{1+4e^{2x}}{1- 4e^{2x}}$ }
\psline[arrows=->, linestyle=dotted](-5,0.5)(0,1.6667)
\psline[arrows=->, linestyle=dotted](-5,0.5)(1,-3.360539267)
\psline[arrows=->, linestyle=dotted](-5,2)(-0.2,5.066489563)
\psline[arrows=->, linestyle=dotted](-5,2)(0.2,-5.066489563)
\psline[arrows=->, linestyle=dotted](-5,-2)(0,-1.6667)
\psline[arrows=->, linestyle=dotted](-5,-2)(-1,3.360539267)
\psaxes{<->}(0,0)(-4.5,-4.5)(4.5,4.5)
\rput (5,5){The direction field  $\frac{\diff y}{\diff x}=y^2-1$}
  \psset{arrows=->}
  \multido{\ra=-4+0.5}{17}{%
    \multido{\rb=-4+0.5}{17}{%
      \pstVerb{/xC \ra\space def
               /yC \rb\space def
               /F  yC yC mul 1 sub \space def
}
%\psline[linecolor=blue](! xC  yC )(! xC yC)
\psdot[linecolor=red!60](! xC yC)
\psline[linecolor=blue](! xC F ATAN 57.295 mul cos 0.2 mul sub yC F ATAN 57.295 mul sin 0.2 mul sub)(! xC F ATAN 57.295 mul cos 0.2 mul add yC F ATAN 57.295 mul sin 0.2 mul add )
}}
\end{pspicture}
%} %optionalDisplay

\noindent \ref{problemDFQseparable-yprime=ysquared-1} (\ref{problemDFQseparable-yprime=ysquared-1-part1}).
From the computer generated picture above, we may visually estimate that $y(x)=\frac{1-4 e^{2x} }{1+4 e^{2x} }$ intersects the $x$-axis at $(0, -\frac 35)$. Furthermore, we may and check directly that for 
\[
y(x)=\frac{1-4 e^{2x} }{1+4 e^{2x} }
\] 
we have $y(0)= \frac{1-4}{1+5}= \frac{-3}{5}= -\frac 35$ and that is a solution to our problem (this however does not prove the solution is unique). 

Alternatively, let us give an algebraic solution. As we are given that $y(0)=-\frac35$ and so 
\[
\begin{array}{rcl}
\displaystyle -\frac35&=&\displaystyle y(0)= \frac{1-De^{2\cdot 0}}{1+ De^{2\cdot 0}}= \frac{1-D}{1+D}\\
\displaystyle -\frac{3}{5} (1+D)&=&1-D\\
\displaystyle \frac{2}{5} D&=&\displaystyle \frac{8}{5}\\
D&=&4\quad ,
\end{array}
\] 
and $D=4$ is our final answer.
}

\end{problem}
\solution{\noindent \ref{problemDFQseparable-yprime=ysquared-1-part1}.
There are two variants for solving this problem. The first variant uses indefinite integration and is slightly informal, but easier to apply and remember. The second variant is more rigorous but more difficult to write up. Both solutions are acceptable for full credit in a Calculus exam. Variant I is recommended when taking exams and Variant II is recommended when writing scientific texts.

\textbf{Variant I}

\renewcommand{\arraystretch}{2}
\[
\begin{array}{rcll|l}
\displaystyle \frac{ \diff y}{ \diff x}&=&y^2-1 && \text{Suppose } y^2-1\neq 0 \\
\displaystyle \frac{\frac{ \diff y}{ \diff x}}{y^2-1}&=&1 \\
\displaystyle\int\frac{1}{ y^2- 1} \underbrace{ \frac{\diff y}{\diff x}\diff x}_{=\diff y } &=& \displaystyle\int \limits \diff x \\
\displaystyle\int \frac{\diff y}{ y^2-1}& =& \displaystyle x +C \\
\displaystyle\int \left(\frac{\frac{1}{2} }{y-1}- \frac{\frac{1 }{2}}{ y+1}\right)\diff y&=& x +C \\
\displaystyle \frac{1}2 \ln \left|\frac{y-1}{y+1}\right| &=& x+C\\
\displaystyle \ln \left|\frac{y-1}{y+1}\right|&=& 2 x + 2C\\
\displaystyle\left|\frac{y-1}{y+1}\right|&=& e^{ 2x +2C} \\
\displaystyle\frac{y-1}{y+1}&=& \pm e^{ 2x +2C} \\
\displaystyle y-1&=&\displaystyle \pm e^{2x+2C} (y+1)\\
\displaystyle y(1\mp e^{2x+2C})&=&\displaystyle 1\pm e^{2x+2C} \\
\displaystyle y&=&\displaystyle \frac{1\pm e^{2x+2C}}{1\mp e^{2x+2C}}  \\
\displaystyle y&=&\displaystyle \frac{1\pm e^{2C} e^{2x}}{1\mp e^{2C}e^{2x}}&&\text{Set }D=\pm e^{2C}\\
\displaystyle y&=&\displaystyle  \frac{1+D e^{2x}}{1- De^{2x}}\quad .
\end{array}
\]
The above solution works on condition that $y^2-1\neq 0$. So the only case not covered is that of $y^2-1=0$, which yields the two solutions $y=\pm 1$.

Our final answer is
\[
y(x)= \frac{1+De^{2x}}{1-De^{2x}} \quad \text{ or }\quad y(x)=-1,
\]
where $D$ is an arbitrary real number. Notice that in the above answer, by allowing $D=0$, we have covered the case $y(x)=1 $. Finally, we note that if we let $D\to \infty$, the solution $y(x) = \frac{1+De^{2x }}{ 1- De^{2x}}  $ tends to the solution $y(x)=-1$ (here we fix a value of $x$ before we let $D\to \infty$).


\textbf{Variant II}

\noindent Case 1. Suppose there exists a number $x_0$ such that $(y(x_0) )^2 - 1\neq 0$. Since $y$ is a differentiable function of $x$, it is also continuous. Therefore for some $t$ sufficiently close to $x_0$, all numbers $x$ in the interval between $t$ and $x_0$ satisfy $ y(x)^2-1\neq 0$.
\[
\begin{array}{rcll|l}
\displaystyle \frac{\frac{ \diff y}{ \diff x}}{y^2-1}&=&1 \\
\displaystyle\int\limits_{x=x_0}^{x=t} \frac{1}{ y^2- 1} \underbrace{ \frac{\diff y}{\diff x}\diff x}_{=\diff (y(x)) }&=&\displaystyle\int\limits_{x=x_0}^{x=t}\diff x &&\text{can integrate as }  y(x)^2-1\neq 0\\
\displaystyle\int\limits_{t=x_0 }^{x=t} \frac{\diff (y(x))}{ (y(x))^2-1}& =& \displaystyle\left.x \right|_{ x=x_0}^{x=t} &&\text{set } z=y(x)\\
\displaystyle\int\limits_{z=y(x_0) }^{z=y(t)} \frac{\diff z}{ z^2-1}& =& \displaystyle t-x_0 \\
\displaystyle\int\limits_{z=y(x_0)}^{z=y(t)} \left(\frac{\frac12 }{z-1}- \frac{\frac12}{z+1}\right)\diff z&=& t-x_0
\\
\displaystyle\left .\frac{1}2 \ln \left|\frac{z-1}{z+1}\right|\right]_{z=y(x_0)}^{z=y(t)}&=& t-x_0 && \text{Set } C=2x_0-  \ln \left|\frac{y(x_0)-1}{ y(x_0)+ 1} \right|\\
\displaystyle \ln \left|\frac{y(t)-1}{y(t)+1}\right|&=& 2t - C&&\text{relabel dummy variable } t \text { to } x \\
\displaystyle
\ln \left|\frac{y(x)-1}{y(x)+1}\right|&=& 2x - C
\end{array}
\]
Set
\[
D=e^{-C}\quad .
\]
By the assumption of our case, $ (y(x_0))^2-1\neq 0$, so there are two remaining cases: $ (y(x_0))^2-1>0$ and $ (y(x_0))^2-1<0$.

\noindent Case 1.1. Suppose $\displaystyle \frac{y(x_0)-1}{ y(x_0)+1}>0$. As the function $y(x)$ is differentiable, it is also continuous. Therefore $\displaystyle \frac{y(x)-1}{y(x)+1}>0$ for all $x$ near $x_0$. Then we can remove the absolute values in the equality above to get that for all $x$ close to $x_0$ we have that
\[
\begin{array}{rcll|l}
\displaystyle \ln \left(\frac{y(x)-1}{y(x)+1}\right)&=& 2x - C&&\text{exponentiate, recall }D=e^{-C}\\
\displaystyle \frac{y(x)-1}{y(x)+1}&=& D e^{2x}\\
\displaystyle y(x)-1&=&\displaystyle  De^{2x}(y(x)+1)\\
\displaystyle y(x)\left(1- De^{2x}\right)&=&\displaystyle  De^{2x}+1\\
\displaystyle y(x)&=&\displaystyle  \frac{ 1+De^{2x}}{1- De^{2x}}\quad .\\
\end{array}
\]
The solution $y(x)$ given above satisfies $\displaystyle \frac{y(x)-1}{y(x)+1}= De^{2x}$ for all $x$. As $D>0$, this implies that $\displaystyle \frac{y(x)-1}{ y(x)+1}>0$. Therefore the considerations above are valid for all $x$, rather than only for those $x$ near $x_0$. Therefore our first case yields the solution
\[
y(x)=\frac{ 1+De^{2x}}{1- De^{2x}}\quad .
\]

\noindent Case 1.2. Suppose  $\displaystyle \frac{y(x_0) -1}{y(x_0) +1} <0$. Then for all $x$ near $x_0$ we get $\displaystyle \ln \left| \frac{y(x) -1}{y(x) +1}\right|= \ln \left( \frac{ 1- y(x) }{ y( x) +1}\right)$ and, similarly to Case 1, we get
\[
\begin{array}{rcl}
\displaystyle \frac{1-y(x)}{y(x)+1}&=& D e^{2x}\\
1-y(x)&=& De^{2x}(y(x)+1)\\
y(x)\left(1+ De^{2x}\right)&=& 1-De^{2x}\\
y(x)&=&\displaystyle \frac{1- De^{2x}}{1+ De^{2x}}\quad .
\end{array}
\]
Since $D$ is a positive constant, we conclude in a fashion analogous to Case 1 that $y(x)<0$ for all $ x$.

Case 2.  Suppose $\displaystyle  (y(x_0))^2-1=0 $.  Then $y(x_0)=\pm 1$. Clearly the constant functions $y(x)= \pm 1$ are two solutions: if we can plug back $y=\pm 1$ in the original equation we get that $\frac{\diff y}{\diff x}= 0$ and $y$ is a constant function of $x$. From the preceding two cases we know that if $\frac{y(x) -1}{y(x) +1}$ is defined and not equal to zero for some value of $x$, then $\frac{y(x)-1}{y(x)+1}$ is defined and not equal to zero for all values of $x$. Therefore the present case yields only two solutions, the constant functions $y(x)=\pm 1$.

Our final answer is
\[
y(x)= \frac{1+De^{2x}}{1-De^{2x}} \quad \text{ or }\quad y(x)=-1,
\]
where $D$ is an arbitrary real number. Notice that in the above answer, we have combined Cases 1.1, 1.2 and the case $y(x)=1 $: by allowing $D$ to be negative we included Case 1.2 and by allowing $D $ to be zero we included the case $y(x)=1$. Finally, we note that if we let $D\to \infty$, the solution $y(x) = \frac{1+De^{2x }}{ 1- De^{2x}}  $ tends to the solution $y(x)=-1$ (for all values of $x$).


\textbf{Solution plots.}

We may plot solutions for a few values of $D$ as follows. We overlay the solutions on top of the direction field of the differential equation. The picture tells us a lot about the properties of the solutions of the differential equations.

\optionalDisplay{
\begin{pspicture}(-6,-6)(6,6)
\newcommand{\Dconst}{1}
\psplot[linecolor=green]{-4}{4}{1 \Dconst\space 2.718281828 2 x mul exp mul sub 1 \Dconst\space 2.718281828 2 x mul exp mul add div}
\renewcommand{\Dconst}{0.25}
\psplot[linecolor=green]{-4}{4}{1 \Dconst\space 2.718281828 2 x mul exp mul sub 1 \Dconst\space 2.718281828 2 x mul exp mul add div}
\renewcommand{\Dconst}{4}
\psplot[linecolor=green]{-4}{4}{1 \Dconst\space 2.718281828 2 x mul exp mul sub 1 \Dconst\space 2.718281828 2 x mul exp mul add div}
\rput[l](5,2 ){$\frac{1- \frac{1}4 e^{2x}}{1+\frac 14 e^{2x}}$ }
\rput[l](5,0.5 ){$\frac{1- e^{2x}}{1+e^{2x}}$ }
\rput[l](5,-2 ){$\frac{1- 4e^{2x}}{1+4e^{2x}}$ }
\psline[arrows=->, linestyle=dotted](5,2)(0,0.6)
\psline[arrows=->, linestyle=dotted](5,0.5)(0,0)
\psline[arrows=->, linestyle=dotted](5,-2)(0,-0.6)

\renewcommand{\Dconst}{1}
\psplot[linecolor=green]{-4}{-0.17}{1 \Dconst\space 2.718281828 2 x mul exp mul add 1 \Dconst\space 2.718281828 2 x mul exp mul sub  div}
\psplot[linecolor=green]{0.17}{4}{1 \Dconst\space 2.718281828 2 x mul exp mul add 1 \Dconst\space 2.718281828 2 x mul exp mul sub  div}
\renewcommand{\Dconst}{4}

\psplot[linecolor=green]{-4}{-0.863147181}{1 \Dconst\space 2.718281828 2 x mul exp mul add 1 \Dconst\space 2.718281828 2 x mul exp mul sub  div}
\psplot[linecolor=green]{-0.523147181}{4}{1 \Dconst\space 2.718281828 2 x mul exp mul add 1 \Dconst\space 2.718281828 2 x mul exp mul sub  div}

\renewcommand{\Dconst}{0.25}
\psplot[linecolor=green]{-4}{0.523147181}{1 \Dconst\space 2.718281828 2 x mul exp mul add 1 \Dconst\space 2.718281828 2 x mul exp mul sub  div}
\psplot[linecolor=green]{0.863147181}{4}{1 \Dconst\space 2.718281828 2 x mul exp mul add 1 \Dconst\space 2.718281828 2 x mul exp mul sub  div}
\rput[r](-5,0.5 ){$\frac{1+\frac 14 e^{2x}}{1- \frac{1}4 e^{2x}}$ }
\rput[r](-5,2 ){$\frac{1+e^{2x}}{1- e^{2x}}$ }
\rput[r](-5,-2 ){$\frac{1+4e^{2x}}{1- 4e^{2x}}$ }
\psline[arrows=->, linestyle=dotted](-5,0.5)(0,1.6667)
\psline[arrows=->, linestyle=dotted](-5,0.5)(1,-3.360539267)
\psline[arrows=->, linestyle=dotted](-5,2)(-0.2,5.066489563)
\psline[arrows=->, linestyle=dotted](-5,2)(0.2,-5.066489563)
\psline[arrows=->, linestyle=dotted](-5,-2)(0,-1.6667)
\psline[arrows=->, linestyle=dotted](-5,-2)(-1,3.360539267)
\psaxes[arrows=<->](0,0)(-4.5,-4.5)(4.5,4.5)

\rput (5,5){The direction field  $\frac{\diff y}{\diff x}=y^2-1$}

\fcDirectionFieldDefault{y y mul 1 sub}{-4}{-4}{0.5}{17}
\end{pspicture}
} %optionalDisplay

\noindent \ref{problemDFQseparable-yprime=ysquared-1-part2}.
From the computer generated picture above, we may visually estimate that $y(x)=\frac{1-4 e^{2x} }{1+4 e^{2x} }$ intersects the $x$-axis at $\left(0, -\frac {3}{ 5}\right)$. Furthermore, we may check directly that for
\[
y(x)=\frac{1-4 e^{2x} }{1+4 e^{2x} }
\]
we have $y(0)= \frac{1-4}{1+5}=  -\frac{3}{5}$ and that is a solution to our problem (this however does not prove the solution is unique).

Alternatively, let us give an algebraic solution. As we are given that $y(0)=-\frac35$ and so
\[
\begin{array}{rcl}
\displaystyle -\frac{3}{5}&=&\displaystyle y(0)= \frac{1-De^{2\cdot 0}}{1+ De^{2\cdot 0}}= \frac{1-D}{1+D}\\
\displaystyle -\frac{3}{5} (1+D)&=&1-D\\
\displaystyle \frac{2}{5} D&=&\displaystyle \frac{8}{5}\\
D&=&4\quad ,
\end{array}
\]
which is our final answer.
}

\solution{\ref{problemy'=y^2(1+x),y(0)=3}. 

This is a concise solution written up in a form suitable for exam taking.
\[\begin{array}{rcl}
\displaystyle \frac{\diff y}{\diff x}&=&\displaystyle y^2(1+x)\\
\displaystyle \frac{\diff y}{y^2} &=&\displaystyle  (1+x) \diff x\\
\displaystyle \int \frac{\diff y}{y^2} &=&\displaystyle \int (1+x) \diff x\\
\displaystyle -\frac{1}{y} &=&\displaystyle  x + \frac{x^2}{2} + C\\
\displaystyle -\frac{1}{3}& =&\displaystyle  0 + 0 + C\\
\displaystyle y &=&\displaystyle  -\frac{1}{\frac{x^2}{2}+x  - \frac{1}{3}} = -\frac{3}{3x^2+6x-2}\quad .
\end{array}
\]
}

\solution{\ref{problemDFQseparabley'=xtany_initial_condition1} and 
\ref{problemDFQseparabley'=xtany_initial_condition2}
\[\begin{array}{rcll|l}
\displaystyle y'&=&\displaystyle x\tan y\\
\displaystyle\frac{y'}{\tan y}&=&\displaystyle x\\
\displaystyle\frac{(\cos y) y'}{\sin y}&=&\displaystyle x &&\text{Integrate from }0\\
\displaystyle \int\limits_{t=0}^{t=x} \frac{\cos(y(t))}{\sin (y(t))} (y' \diff t)&=&\displaystyle  \int\limits_{t=0}^xt \diff t \\
\displaystyle \int\limits_{t=0}^{t=x} \frac{\cos(y(t))}{\sin (y(t))}\diff (y(t))&=&\displaystyle  \frac{x^2}{2} &&\text{Set }z=y(t)\\
\displaystyle \int \limits_{z=y(0)}^{z=y(x)} \frac{\cos z}{\sin z} \diff z&=&\displaystyle  \frac{x^2}{2}\\
\displaystyle \int \limits_{z=y(0)}^{z=y(x)} \frac{\diff (\sin z)}{\sin z} &=&\displaystyle  \frac{x^2}{2}\\
\displaystyle \left[\ln | \sin z|\right]_{y(0)}^{y} & = & \displaystyle  \frac{x^2}{2}\\
\displaystyle \ln |\sin y|- \ln |\sin (y(0))|&=& \displaystyle  \frac{x^2}{2} \\
\displaystyle \ln |\sin y|&=& \displaystyle  \frac{x^2}{2}+\ln |\sin (y(0))| \\
|\sin y|&=&\displaystyle  e^{\frac{x^2}{2}+\ln |\sin (y(0))|}\\ |\sin y|&=&\displaystyle \doublebrace{e^{\frac{x^2}{2}+\ln \left|\sin \left(\Arcsin \left(\frac{1}{e}\right) \right)\right|}}{\text{for problem \ref{problemDFQseparabley'=xtany_initial_condition1}}}{e^{\frac{x^2}{2}+\ln \left|\sin \left(\pi+ \Arcsin \left(\frac{1}{e}\right) \right)\right|} }{\text{for problem \ref{problemDFQseparabley'=xtany_initial_condition2}}} \\
|\sin y|&=&\displaystyle  e^{\frac{x^2}{2}+ \ln \left(\frac{1}{ e}\right)}\\
\displaystyle |\sin y|&=&\displaystyle e^{\frac{x^2}{2}-1} &&\begin{array}{l} y(0)>0 \text{ for both problems}\\
\text{therefore }  \sin y(0) > 0\end{array}\\
\displaystyle \sin y&=&e^{\frac{x^2}{2}-1} \quad.
\end{array}
\]
From the elementary properties of the trigonometric functions, we know that  $\sin y=\sin \alpha$ implies that either
\begin{itemize}
\item $y=\alpha +2k\pi$, where $k$ is an arbitrary integer or
\item $y=(2k+1)\pi-\alpha$, where $k$ is an arbitrary integer.
\end{itemize}
In other words, if we are given $\sin y$, we know $y$ up to a choice of sign and a choice of an integer $k$. For our problem, this means that 

\[
y=\left\{\begin{array}{ll} 2 k \pi+\Arcsin\left( e^{\frac{x^2}{2}-1} \right) &{k -\text{integer}} \\{\text{or}}\\{(2k+1)\pi- \Arcsin\left( e^{\frac{x^2}{2}-1} \right)}&{k-\text{integer}}\end{array}\right.
\]

For problem \ref{problemDFQseparabley'=xtany_initial_condition1}, 
the only choice for $k$ and sign which fits the initial condition $y(0)= \Arcsin\left(\frac{1}{e}\right)$ is
\[
y=\Arcsin\left(e^{\frac{x^2}{2}-1} \right)\quad ,
\]
which is our final answer. 

For problem \ref{problemDFQseparabley'=xtany_initial_condition2}, 
the only choice for $k$ and sign which fits the initial condition $y(0)=\pi+ \Arcsin\left(-\frac{1}{e}\right)=\pi- \Arcsin \left( \frac{1}{e}\right) $ is
\[
y=\pi- \Arcsin\left(e^{\frac{x^2}{2}-1} \right)\quad, 
\]
which is our final answer.
}


\begin{problem}
\begin{enumerate}
\item Find the general solution to the differential equation
\[
\frac{\diff y}{\diff x}= y^2-4\quad .
\]
Below is a computer-generated plot of the direction field  $\displaystyle \frac{\diff y}{\diff x}=y^2-4$, you may use it to get a feeling for what your answer should look like.

\optionalDisplay{
\begin{pspicture}(-6,-6)(6,6)
\newcommand{\Dconst}{4}
\psplot[linecolor=green]{-4}{4}{1 \Dconst\space 2.718281828 4 x mul exp mul sub 1 \Dconst\space 2.718281828 4 x mul exp mul add div 2 mul}
\psaxes{<->}(0,0)(-5,-5)(5,5)
\rput[l](0.4,5){The direction field  $\frac{\diff y}{\diff  x}=y^2-4$}
\fcDirectionFieldDefault{y y mul 4 sub}{-4}{-4}{0.5}{17}
\end{pspicture}
}
\item {Find a solution of the above equation for which $ y(0)= -\frac{6}{5}$.}
\end{enumerate}

\end{problem}
\begin{problem}
\begin{enumerate}
\item Solve the initial-value differential equation problem 
\[
y'=xe^{-y} 
\quad,\quad  \quad \quad y(4)=0.
\]
Below is a computer-generated plot of the corresponding direction field, you may use it to get a feeling for what your answer should look like.

\optionalDisplay{
\begin{pspicture}(-6,-6)(9.5,9.5)
\psaxes{<->}(0,0)(-5,-5)(9,9)
\psFullDot{4}{0}
\rput[lt](0.4,9){The direction field  $\frac{\diff y}{\diff  x}=y'=xe^{-y} $}
\directionFieldDefault{2.718281828 y -1 mul exp x mul}{-4}{-4}{0.5}{25}
\psplot[linecolor=\psColorGraph]{3.745}{8.2}{-7 x 2 exp 0.5 mul add ln}
\end{pspicture}
}

\answer{$y(x)=\ln \left(\frac{x^2}{2}-7\right)$}
\item Solve the initial-value differential equation problem 
\[
y'=\frac{\ln x}{x y}
\quad,\quad  \quad \quad y(1)=2.
\]
Below is a computer-generated plot of the corresponding direction field, you may use it to get a feeling for what your answer should look like.

\optionalDisplay{
\begin{pspicture}(-1,-6)(11,6)
\psaxes{<->}(0,0)(-0.5,-5)(10,6)
\psFullDot{1}{2}
\rput[l](0.4,5){The direction field  $\frac{\diff y}{\diff  x}=y'=\frac{\ln x}{x y} $}
\directionFieldDefault{x ln y x mul div}{0.01}{-4.01}{0.5}{17}
\psplot[linecolor=\psColorGraph, plotpoints=600 ]{0.2}{8}{4 x ln 2 exp add 0.5 exp}
\end{pspicture}
}
\answer{$y(x)= \sqrt{(\ln x)^2+4}$}
\end{enumerate}
\end{problem}
\solution{\ref{problemy'=y^2(1+x),y(0)=3}. 
\[\begin{array}{rcl}
\displaystyle \frac{\diff y}{\diff x}&=&\displaystyle y^2(1+x)\\
\displaystyle \frac{\diff y}{y^2} &=&\displaystyle  (1+x) \diff x\\
\displaystyle \int \frac{\diff y}{y^2} &=&\displaystyle \int (1+x) \diff x\\
\displaystyle -\frac{1}{y} &=&\displaystyle  x + \frac{x^2}{2} + C\\
\displaystyle -\frac{1}{3}& =&\displaystyle  0 + 0 + C\\
\displaystyle y &=&\displaystyle  -\frac{1}{\frac{x^2}{2}+x  - \frac{1}{3}} = -\frac{3}{3x^2+6x-2}\quad .
\end{array}
\]
}

\begin{problem}
\begin{enumerate}
\item \label{problemDFQseparabley'=xtany_initial_condition1} Solve the initial-value differential equation problem 
\[
y'=x\tan y 
\quad,\quad  \quad \quad y(0)=\Arcsin\left(\frac{1}{e}\right)\approx 0.376728.
\]

\answer{$y(x)=\Arcsin\left(e^{\frac{x^2}{2}-1}\right)$}
\item \label{problemDFQseparabley'=xtany_initial_condition2} Solve the same differential equation with initial condition $y(0)= \pi+\Arcsin \left(-\frac{1}{e}\right)\approx 2.764865$.

\answer{$y(x)= \pi+ \Arcsin\left(- e^{ \frac{x^2}{2}- 1}\right)$}


Below is a computer-generated plot of corresponding direction field, you may use it to get a feeling for what your answer should look like.

\optionalDisplay{
\begin{pspicture}(-6,-6)(9.5,9.5)
\psaxes{<->}(0,0)(-5,-5)(9,9)
\rput[lt](0.4,9){The direction field  $\frac{y}{\diff  x}=y'=x\tan y $}
\directionFieldDefault{y 57.29578 mul sin x mul y 57.29578 mul cos div}{-4}{-4}{0.5}{25}
\psplot[linecolor=\psColorGraph, plotpoints=500]{-1.41421356}{1.41421356}{2.718281828 -1 x 2 exp 0.5 mul add exp ASIN}
\psplot[linecolor=green, plotpoints=500]{-1.41421356}{1.41421356}{3.141592654 2.718281828 -1 x 2 exp 0.5 mul add exp -1 mul ASIN add}
\end{pspicture}
}
\end{enumerate}
\end{problem}
\solution{\ref{problemDFQseparabley'=xtany_initial_condition1} and 
\ref{problemDFQseparabley'=xtany_initial_condition2}
\[\begin{array}{rcll|l}
\displaystyle y'&=&\displaystyle x\tan y\\
\displaystyle\frac{y'}{\tan y}&=&\displaystyle x\\
\displaystyle\frac{(\cos y) y'}{\sin y}&=&\displaystyle x &&\text{Integrate from }0\\
\displaystyle \int\limits_{t=0}^{t=x} \frac{\cos(y(t))}{\sin (y(t))} (y' \diff t)&=&\displaystyle  \int\limits_{t=0}^xt \diff t \\
\displaystyle \int\limits_{t=0}^{t=x} \frac{\cos(y(t))}{\sin (y(t))}\diff (y(t))&=&\displaystyle  \frac{x^2}{2} &&\text{Set }z=y(t)\\
\displaystyle \int \limits_{z=y(0)}^{z=y(x)} \frac{\cos z}{\sin z} \diff z&=&\displaystyle  \frac{x^2}{2}\\
\displaystyle \int \limits_{z=y(0)}^{z=y(x)} \frac{\diff (\sin z)}{\sin z} &=&\displaystyle  \frac{x^2}{2}\\
\displaystyle \left[\ln | \sin z|\right]_{y(0)}^{y} & = & \displaystyle  \frac{x^2}{2}\\
\displaystyle \ln |\sin y|- \ln |\sin (y(0))|&=& \displaystyle  \frac{x^2}{2} \\
\displaystyle \ln |\sin y|&=& \displaystyle  \frac{x^2}{2}+\ln |\sin (y(0))| \\
|\sin y|&=&\displaystyle  e^{\frac{x^2}{2}+\ln |\sin (y(0))|}\\ |\sin y|&=&\displaystyle \doublebrace{e^{\frac{x^2}{2}+\ln \left|\sin \left(\Arcsin \left(\frac{1}{e}\right) \right)\right|}}{\text{for problem \ref{problemDFQseparabley'=xtany_initial_condition1}}}{e^{\frac{x^2}{2}+\ln \left|\sin \left(\pi+ \Arcsin \left(\frac{1}{e}\right) \right)\right|} }{\text{for problem \ref{problemDFQseparabley'=xtany_initial_condition2}}} \\
|\sin y|&=&\displaystyle  e^{\frac{x^2}{2}+ \ln \left(\frac{1}{ e}\right)}\\
\displaystyle |\sin y|&=&\displaystyle e^{\frac{x^2}{2}-1} &&\begin{array}{l} y(0)>0 \text{ for both problems}\\
\text{therefore }  \sin y(0) > 0\end{array}\\
\displaystyle \sin y&=&e^{\frac{x^2}{2}-1} \quad.
\end{array}
\]
From the elementary properties of the trigonometric functions, we know that  $\sin y=\sin \alpha$ implies that either
\begin{itemize}
\item $y=\alpha +2k\pi$, where $k$ is an arbitrary integer or
\item $y=(2k+1)\pi-\alpha$, where $k$ is an arbitrary integer.
\end{itemize}
In other words, if we are given $\sin y$, we know $y$ up to a choice of sign and a choice of an integer $k$. For our problem, this means that 

\[
y=\triplebrace{ 2 k \pi+\Arcsin\left( e^{\frac{x^2}{2}-1} \right)}{k -\text{integer}}{\text{or}}{}{(2k+1)\pi- \Arcsin\left( e^{\frac{x^2}{2}-1} \right)}{k-\text{integer}}
\]

For problem \ref{problemDFQseparabley'=xtany_initial_condition1}, 
the only choice for $k$ and sign which fits the initial condition $y(0)= \Arcsin\left(\frac{1}{e}\right)$ is
\[
y=\Arcsin\left(e^{\frac{x^2}{2}-1} \right)\quad ,
\]
which is our final answer. 

For problem \ref{problemDFQseparabley'=xtany_initial_condition2}, 
the only choice for $k$ and sign which fits the initial condition $y(0)=\pi+ \Arcsin\left(-\frac{1}{e}\right)=\pi- \Arcsin \left( \frac{1}{e}\right) $ is
\[
y=\pi- \Arcsin\left(e^{\frac{x^2}{2}-1} \right)\quad, 
\]
which is our final answer.
}
\end{document}
