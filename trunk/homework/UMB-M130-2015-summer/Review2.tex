\documentclass{article}
\ProvidesPackage{homework-problems-UMB}
\addtolength{\hoffset}{-3.5cm}
\addtolength{\textwidth}{6.8cm}
\addtolength{\voffset}{-3cm}
\addtolength{\textheight}{6cm}
\usepackage{../homework-problems} %warning folder paths are relative to the file that uses the includepackage

\renewcommand{\answer}[1]{\iftoggle{answers}{ \hfill{~} \rotatebox{180}{\tiny answer: #1}}{} }
\renewcommand{\pointsii}[1]{}
\renewcommand{\hiddenanswer}{\answer}
\renewcommand{\points}[1]{\item}
\renewcommand{\pointsii}[1]{\item}
\renewcommand{\Arctan}{\arctan}
\renewcommand{\Arcsin}{\arcsin}
\renewcommand{\Arccot}{\operatorname{arccot}}

\toggletrue{solutions}
\toggletrue{answers}
\renewcommand{\fcProblemRef}{\theproblem.\theenumi}
\renewcommand{\fcSubProblemRef}{\theenumi.\theenumii}


\newcommand{\hide}[1]{}
\newtheorem{problem}{Problem}
\pagestyle{empty}
\begin{document}
\begin{center}
\Large
Review sheet Test 2 \\ Math 130 Precalculus \\ \normalsize Summer 2015 \\ Instructor: Todor Milev
\end{center}
%\noindent \textbf{Name:\underline{~~~~~~~~~~~~~~~~~~~~~~~} } \hfill{~}



\noindent The exam is closed textbook. \textbf{No electronic devices are allowed during the exam. } The exam will contain 6 problems, of the 6 problem types given in this review sheet. The material on the test includes the material from Lecture 8 (Inverse Functions) up to and including Lecture 14 (Trigonometry Basics). %You are allowed one single formula sheet, handwritten by you. No template problem solutions are allowed. The sheet will be collected with the test. Photocopied formula sheets are not allowed. 

\begin{problem}
% begin homework inverse-functions3
Find the inverse function. You are asked to do the algebra only; you are not asked to determine the domain or range of the function or its inverse. 
\begin{enumerate} [ref={\fcProblemRef}]
\item $f(x)= 3x^2+4x-7$, where $x\geq -\frac{2}{3}$.
\answer{$f^{-1}(x)= -\frac{2}3+\frac{\sqrt{25+3x}}{3}, \quad x\geq -\frac{25}{3}$}

\item $f(x)= 2x^2+3x-5$, where $x\geq -\frac{3}{4}$.
\answer{$f^{-1}(x)=-\frac{3}{4}+\frac{\sqrt{49+8x}}{4}, \quad x\geq -\frac{49}{8}$}

\item $\displaystyle f(x)= \frac{2x+5}{x-4}$, where $x\neq 4$.
\answer{$f^{-1}(x)=\frac{4x+5}{x-2}, \quad x\neq 2$}

\pointsii{3} \label{problemFindInversef=(3x+5)/(2x-4)} $\displaystyle f(x)= \frac{3x+5}{2x-4}$, where $x\neq 2$.

\hiddenanswer{$\displaystyle f^{-1}(x) = \frac{ 4 x +5}{2x-3}, \quad x\neq \frac{3}{2}$}
\item \label{problemFindIversef=(5x+6)/(4x+5)}  $\displaystyle f(x)= \frac{5x+6}{4x+5}$.

\answer{$f^{-1}(x)= \frac{-5x+6}{4x-5}$, $x\neq \frac{5}{4}$}
\item  $\displaystyle f(x)= \frac{2x-3}{-3x+4}$.

\answer{$f^{-1}(x)=\frac{4x+3}{3x+2}  $, $x\neq -\frac{2}{3}$}
\item $f(x)=2^{2x}+2^{x}-2$.
\answer{$f^{-1}(x) =\log_2\frac{-1+\sqrt{9+4x}}{2}, \quad x\geq -2$}

\end{enumerate}
% end homework inverse-functions3

\end{problem}
\solution{
\ref{problemFindIversef=(5x+6)/(4x+5)}. Set $f(x)=y$. Then
\[
\begin{array}{rcl}
y&=&\displaystyle \frac{5x+6}{4x+5}\\
y(4x+5)&=&5x+6\\
x(4y-5)&=&-5y+6\\
x&=&\displaystyle \frac{-5y+6}{4y-5} .
\end{array}
\]
Therefore the function $\displaystyle x=g(y)=\frac{-5y+6}{4y-5}$ is the inverse of $f(x)$. We write $g=f^{-1}$. The function $g=f^{-1}$ is defined for $\displaystyle y\neq \frac{5}{4}$. For our final answer we relabel the argument of $g$ to $x$:

\[
g(x)=f^{-1}(x)= \frac{-5x+6}{4x-5}\quad .
\]

Let us check our work. In order for $f$ and $g$ to be inverses, we need that $g(f(x))$ be equal to $x$.
\[
g(f(x))=  \frac{-5f(x) +6}{4f(x)-5}=  \frac{-5\frac{(5x+6)}{4x+5} +6}{4\frac{(5x+6)}{4x+5}-5}= \frac{-5(5x+6) +6(4x+5)}{4(5x+6)-5(4x+5)}=\frac{-x}{-1}=x\quad ,
\]
as expected.
}

\solution{\ref{problemFindInversef=(3x+5)/(2x-4)}
This is a concise solution written in form suitable for test taking.
\[
\begin{array}{rcl}
y & =& \displaystyle \frac{3x+5}{2x-4} \\
y(2x-4) & =& 3x+5 \\
2xy-4y & =& 3x+5 \\
2xy-3x & =& 4y+5 \\
x(2y-3) & =& 4y+5 \\
x & =& \displaystyle  \frac{4y+5}{2y-3} \\
\text{Therefore}\quad \displaystyle  f^{-1}(y) & =& \displaystyle  \frac{5+4y}{2y -3 } \\
\displaystyle f^{-1}(x) & =& \displaystyle \frac{5+4x}{2x-3}.
\end{array}
\]
}%

\begin{problem}
% begin homework logarithms-combine
Express each of the following as a single logarithm. If possible, compute the logarithm without using a calculator. The answer key has not been proofread, use with caution.

\begin{enumerate}[ref={\fcProblemRef}]
\item   $\ln 4 + \ln 6 - \ln 5$.

\answer{$\ln \left(\frac{24}{5} \right) $}
\item \label{problem2ln(2)-3ln(3)+4ln(4)} $2\ln 2 - 3\ln 3 + 4\ln 4$.

\answer{$ \ln \left( \frac{1024}{27}\right)$}
\item   $\ln 36 - 2\ln 3 - 3\ln 2$.

\answer{$-\ln 2=\ln \left(\frac{1}{2}\right) $}

\item $\log_2(24)-\log_{4}9$.

\answer{$3$}

\item $\log_7(24)+\log_{\frac{1}{7}}3-\log_{49} (64)$.

\answer{$0$}
\item $\log_3(24)+\log_{3}\left(\frac{3}{8}\right)$.

\answer{$ 2$}

\end{enumerate}
% end homework logarithms-combine

\end{problem}
\solution{\ref{problem2ln(2)-3ln(3)+4ln(4)}.
\begin{align*}
2\ln 2 - 3\ln 3 + 4\ln 4 & = \ln 2^2 - \ln 3^3 + \ln 4^4 \\
 & = \ln 4 - \ln 27 + \ln 256 \\
 & = \ln \Big( \frac{4}{27}\Big) + \ln 256 \\
 & = \ln \Big( \frac{4\cdot 256}{27}\Big) \\
 & = \ln \Big( \frac{1024}{27}\Big).
\end{align*}

$\frac{1024}{27}$ is not a rational power of $e$, therefore $ \ln \left( \frac{1024}{27}\right)$ is not a rational number and there are no further simplifications of the answer (except possibly a numerical approximation with a calculator or equivalent). 
}%

\solution{\ref{problemlog_7(24)+log_(1/7)3-log_(49)64}

\[
\renewcommand{\arraystretch}{2}
\begin{array}{@{}r@{}c@{}l@{}l|l}
\displaystyle \log_{7}{(24)}+\log_{\frac{1}{7}}{(3)}-\log_{49}{(64)}&=&\displaystyle  \log_{7}{(24)}+ \frac{\log_{7}{(3)}}{ \log_{7}{\left(\frac{1}{7}\right)}}- \frac{\log_{7}{ (64)}}{\log_{7}{(49)}} && \text{common base}\\
&=&\displaystyle \log_{7}{(24)}+ \frac{\log_{7}{(3)}}{-1} -\frac{\log_{7}{(64)}}{2}&&\text{simplify logarithms}\\
&=&\displaystyle \log_{7}{(24)}-\log_{7}{(3)}-\frac{1}{2}\log_{7}{(64)}\\
&=&\displaystyle \log_{7}{\left(\frac{24}{3}\right)}- \log_{7}{ \left(64^{ \frac{ 1}{2}}\right) }&&\renewcommand{\arraystretch}{1.2} \begin{array}{@{}l}\text{rule: } \log_ax-\log_ay=\log_a\left( \frac{x}{y}\right)  \\ \text{rule: } \log_ax^r=r\log_ax \end{array} \\
&=&\displaystyle \log_{7}{\left(8\right)}-\log_{7}\left(\sqrt{64} \right)\\
&=&\displaystyle \log_78-\log_78 =0 &&\text{alternatively: }\\
&=&\displaystyle \log_7\left(\frac{8}{8}\right)\\
&=&\displaystyle \log_7(1)\\
&=&0.
\end{array}
\]
}


\begin{problem}~
\begin{enumerate}
\item $1$ day after the start of hypothetical experiment a population of fruit flies was measured to have $110$ individuals. $3$ days after the start there were $190$ flies. Write down an exponential growth law that fits this data. According to the model, how may fruit were there at the start of the experiment? After $5$ days? The answer key has not been proofread, use with caution.

\answer{$P(t)=be^{at}$  $a=\frac{1}{2}\ln\left(\frac{19}{11}\right)\approx 0.273$, $b=110e^{-a}\approx 84$, $P(0)=b\approx 84$, $P(5)\approx 328$}

\item In a hypothetical experiment, the number of E. Coli bacteria cells is modeled with a logistic curve $\displaystyle E(t)= \frac{2.8\times 10^{11}}{1+ (3.5\times 10^9) e^{-1.2t }} $, where $t$ measures time in hours since the start of the experiment. 
\begin{itemize}
\item According to the model, approximately how many cells were there at the start of the experiment?
\item According to the model, how many hours are needed for the number of cells to be approximately $10^{10}$?
\end{itemize}
The answer key has not been proofread, use with caution.

\answer{about $80$ cells at the start; about $15.6$ hours needed for $10^{10}$ cells.}
\item The Richter magnitude $M_L$ of an earthquake is determined from the logarithm of the amplitude $A$ of waves recorded by seismographs (with adjustment to compensate for the distance between the measuring station and the estimated epicenter of the earthquake). The formula is

\hfil \hfil$
M_L=\log_{10}A -J_0(\delta),
$

\noindent where $J_0(\delta)$ depends on the distance $\delta$ from the epicenter. Compare the amplitudes $A_1$ and $A_2$ of the seismographic waves of two hypothetical earthquakes of magnitudes $5$ and $7.2$ with the same epicenter.

\answer{the stronger earthquake has seism. amplitude about $10^{2.2}\approx 158.5$ times larger}


\end{enumerate}
\end{problem}

\begin{problem}
Find the indicated circle arc-length. The answer key has not been proofread, use with caution. 
\begin{enumerate}
\item Circle of radius $3$, arc of measure $36^\circ$.

\answer{$\frac{3\pi}{5}\approx 1.884956 $}
\item Circle of radius $\frac{1}{2}$, arc of measure $100^\circ$.

\answer{$\frac{5\pi}{18}\approx 0.872665 $}
\item Circle of radius $1$, arc of measure $3$ (radians).

\answer{$3$}
\item Circle of radius $3$, arc of measure $300^\circ$.

\answer{$5\pi\approx 15.707963 $}
\end{enumerate}
\end{problem}

\begin{problem}
Find the $6$ trigonometric functions of the indicated angle in the indicated right triangle.
\begin{enumerate}[ref={\fcProblemRef}]
\item ~
\begin{pspicture}(-1,-1)(2.5,3.5)
\pstVerb{20 dict begin
/theX 2 def
/theY 3 def
/adjacentAngle theY theX atan def
/oppositeAngle 90 adjacentAngle sub def
/arcRad 0.3 def
}%
\psline(0,0)(! theX 0)(! theX theY)(0,0)%
\fcPerpendicular{[theX theY]}{[1 0]}{0.2}%
\rput[t](! theX  2 div -0.1){$2$}%
\rput[l](! theX 0.1 add theY 2 div ){$3$}%
%\rput[br](! theX 2 div theY 2 div 0.1 add){$\sqrt{13}$}%
\parametricplot[linecolor=red]{0}{adjacentAngle}{t cos arcRad mul t sin arcRad mul}%
\rput(! adjacentAngle 2 div cos arcRad 0.2 add mul adjacentAngle 2 div sin arcRad 0.2 add mul){$\theta$}%
%\parametricplot[linecolor=red]{-90}{-90 oppositeAngle sub}{t cos arcRad mul theX add t sin arcRad mul theY add}%
%\rput(! -90 oppositeAngle  2 div sub cos arcRad 0.2 add mul theX add -90 oppositeAngle  2 div sub sin arcRad 0.2 add mul theY add){$\theta$}%
\pstVerb{end}%
\end{pspicture}

\answer{$\sin \theta = \frac{3}{13}\sqrt{13}$, $\cos \theta=\frac{2}{13} \sqrt{13}$, $\tan \theta= \frac{3}{2}$, $\cot \theta=\frac{2}{3}$, $\sec \theta=\frac{\sqrt{13}}{2}$, $\csc \theta=\frac{\sqrt{13}}{3}$  }
\item ~ \begin{pspicture}(-1,-1)(2.5,3.5)
\pstVerb{20 dict begin
/theX 2 def
/theY 1 def
/adjacentAngle theY theX atan def
/oppositeAngle 90 adjacentAngle sub def
/arcRad 0.6 def
}%
\psline(0,0)(! theX 0)(! theX theY)(0,0)%
\fcPerpendicular{[theX theY]}{[1 0]}{0.2}%
%\rput[t](! theX  2 div -0.1){$1$}%
\rput[l](! theX 0.1 add theY 2 div ){$1$}%
\rput[br](! theX 2 div theY 2 div 0.1 add){$\sqrt{5}$}%
\parametricplot[linecolor=red]{0}{adjacentAngle}{t cos arcRad mul t sin arcRad mul}%
\rput(! adjacentAngle 2 div cos arcRad 0.2 add mul adjacentAngle 2 div sin arcRad 0.2 add mul){$\theta$}%
%\parametricplot[linecolor=red]{-90}{-90 oppositeAngle sub}{t cos arcRad mul theX add t sin arcRad mul theY add}%
%\rput(! -90 oppositeAngle  2 div sub cos arcRad 0.2 add mul theX add -90 oppositeAngle  2 div sub sin arcRad 0.2 add mul theY add){$\theta$}%
\pstVerb{end}%
\end{pspicture}

\answer{$\sin \theta = \frac{\sqrt{5}}{5}$, $\cos \theta=\frac{ 2\sqrt{5}}{5}$, $\tan \theta= \frac{1}{2}$, $\cot \theta=2$, $\sec \theta= \frac{\sqrt{5}}{2}$, $\csc \theta=\sqrt{5}$  }
\item ~\psset{xunit=0.5cm, yunit=0.5cm} \begin{pspicture}(-1,-1)(5.5,2.5)
\pstVerb{20 dict begin
/theX 5 def
/theY 2 def
/adjacentAngle theY theX atan def
/oppositeAngle 90 adjacentAngle sub def
/arcRad 0.6 def
}%
\psline(0,0)(! theX 0)(! theX theY)(0,0)%
\fcPerpendicular{[theX theY]}{[1 0]}{0.2}%
\rput[t](! theX  2 div -0.1){$5$}%
\rput[l](! theX 0.1 add theY 2 div ){$2$}%
%\rput[br](! theX 2 div theY 2 div 0.1 add){$6$}%
%\parametricplot[linecolor=red]{0}{adjacentAngle}{t cos arcRad mul t sin arcRad mul}%
%\rput(! adjacentAngle 2 div cos arcRad 0.2 add mul adjacentAngle 2 div sin arcRad 0.2 add mul){$\theta$}%
\parametricplot[linecolor=red]{-90}{-90 oppositeAngle sub}{t cos arcRad mul theX add t sin arcRad mul theY add}%
\rput(! -90 oppositeAngle  2 div sub cos arcRad 0.2 add mul theX add -90 oppositeAngle  2 div sub sin arcRad 0.2 add mul theY add){$\theta$}%
\pstVerb{end}%
\end{pspicture}

\answer{$\sin \theta =\frac{5}{\sqrt{29}}= \frac{5\sqrt{29}}{29}$, $\cos \theta= \frac{2}{\sqrt{29}} =\frac{2\sqrt{29}}{29}$, $\tan \theta=\frac{2}{5} $, $\cot \theta=\frac{5}{2}$, $\sec \theta=\frac{\sqrt{29}}{5}$, $\csc \theta=\frac{\sqrt{29}}{2}$  }
\item ~ 

\psset{xunit=0.5cm, yunit=0.5cm} 
\begin{pspicture}(-1,-1)(5,6)
\pstVerb{20 dict begin
/theX 11 sqrt def
/theY 5 def
/adjacentAngle theY theX atan def
/oppositeAngle 90 adjacentAngle sub def
/arcRad 0.6 def
}%
\psline(0,0)(! theX 0)(! theX theY)(0,0)%
\fcPerpendicular{[theX theY]}{[1 0]}{0.2}%
%\rput[t](! theX  2 div -0.1){$1$}%
\rput[l](! theX 0.1 add theY 2 div ){$5$}%
\rput[br](! theX 2 div theY 2 div 0.1 add){$6$}%
%\parametricplot[linecolor=red]{0}{adjacentAngle}{t cos arcRad mul t sin arcRad mul}%
%\rput(! adjacentAngle 2 div cos arcRad 0.2 add mul adjacentAngle 2 div sin arcRad 0.2 add mul){$\theta$}%
\parametricplot[linecolor=red]{-90}{-90 oppositeAngle sub}{t cos arcRad mul theX add t sin arcRad mul theY add}%
\rput(! -90 oppositeAngle  2 div sub cos arcRad 0.2 add mul theX add -90 oppositeAngle  2 div sub sin arcRad 0.2 add mul theY add){$\theta$}%
\pstVerb{end}%
\end{pspicture}

\answer{$\sin \theta = \frac{\sqrt{11}}{6}$, $\cos \theta=\frac{5}{6}$, $\tan \theta= \frac{\sqrt{11}}{5}$, $\cot \theta=\frac{5}{\sqrt{11}}= \frac{5\sqrt{11}}{11}$, $\sec \theta= \frac{6}{5}$, $\csc \theta=\frac{6}{\sqrt{11}}=\frac{6\sqrt{11}}{11}$  }

\end{enumerate}
\end{problem}

\begin{problem}
Find all solutions of the equation in the interval $[0,2\pi)$.

\begin{enumerate}[ref={\fcProblemRef}]
\item $\sin x = -\frac{\sqrt{2}}{2}$.

\answer{$x=$}
\item $\cos x = \frac{\sqrt{3}}{2}$.

\answer{$x=$}
\item $\sin (3x) = \frac{1}{2}$.

\answer{$x=$}
\item $\cos (7x) = 0$.

\answer{$x=$}
\end{enumerate}
\end{problem}
%\vskip 18cm
%\hfill \begin{tabular}{c|c|c|c|c|c|c||c}
%Problem&1 &2&3&4&5&6& $\sum$\\ \hline
%Score&&&&&&&\\ \hline
%Max&17&17&17&17&17&17&102
%\end{tabular} 


\end{document}