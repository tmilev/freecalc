\documentclass{article}
\usepackage{../homework-problems-UMB}

\toggletrue{solutions}
\toggletrue{answers}
%\togglefalse{solutions}

\renewcommand{\course}{Math 140}
\newcommand{\tooeasy}{ ~[The problem is too easy to appear on a quiz or test.]}

\begin{comment}
\homeworkStart{on Lectures 1 and 2 \\Will be quizzed Monday February 3}{}
\item Find the implied domain of the function.
\begin{multicols}{2}
\begin{enumerate}[ref={\fcProblemRef}]
\item $\displaystyle f(x)=\frac{x+4}{x^2-4}$. 

\answer{\begin{tabular}{l} $x\neq \pm 2$, \\alternatively:\\ $x\in (-\infty, -2)\cup (-2,2)\cup (2,\infty)$\end{tabular} }
\item $\displaystyle f(x)=\frac{2x^3-5}{x^2+5x+6}$.

\answer{\begin{tabular}{l} $x\neq -2,-3$, \\alternatively:\\ $x\in (-\infty, -3)\cup (-3,-2)\cup (-2,\infty)$\end{tabular} }
\item $\displaystyle f(t)=\sqrt[3]{3t-1}$.

\answer{$x\in \mathbb R$ (the domain is all real numbers) }
\item $\displaystyle g(t)=\sqrt{5-t}-\sqrt{1+t}$.

\answer{$x\in [-1,5]$.}
\item $\displaystyle h(x)=\frac{1}{\sqrt[6]{x^2-7x}}$.

\answer{$x\in (-\infty, 0)\cup (7,\infty)$.}
\item $f\displaystyle (u)=\frac{u+1}{1+\frac{1}{u+1}}$.

\answer{
\begin{tabular}{l}
$u\neq -1, -2$ or \\
$u\in (-\infty, -2)\cup (-2, -1)\cup (-1,\infty).$.
\end{tabular}
}
\item $\displaystyle F(x)=\sqrt{10-{\sqrt{x}}}$.

\answer{$x\in [0,100]$}
\end{enumerate}
\end{multicols}
\item Find the functions $f\circ g$, $g\circ f$, $f\circ f$ and $g\circ g$ and their implied domains.

\begin{enumerate}
\item $f(x)=x^2+1$, $g(x)=x+1$. 
\answer{\begin{tabular}{l} Domain, all 4 cases: $x\in \mathbb R$ (all reals)\\ in some order: $(1+x)^{2}+1, (x)^{2}+2, ((x)^{2}+1)^{2}+1, 2+x$\end{tabular} }
\item $f(x)=\sqrt{x+1}$, $g(x)=x+1$. 
\answer{\begin{tabular}{l} Domain of $f\circ g$ is $x\geq -2$. Domain of $g\circ f$ is $x\geq -1$ \\Domain of $f\circ f$ is $ x\geq -1$. Domain of $g\circ g$ is all reals ($x\in \mathbb R$). \\ in some order:$\sqrt{2+x}, 1+\sqrt{1+x}, \sqrt{1+\sqrt{1+x}}, 2+x$\end{tabular}}
\item $f(x)= 2x$, $g(x)= \tan x$. 

You are not required to find the domain.
\answer{\begin{tabular}{l}
Domain  $f\circ f$: all reals ($x\in \mathbb R$). Domain $g\circ f$: $x\neq (2k+1)\frac{\pi}{2}$ for all $k\in \mathbb Z$\\ Domain $ f\circ g$: $x\neq (4k+1)\frac{\pi}{4}$, $x\neq (4k+3)\frac{\pi}{4}$ for all $k\in \mathbb Z$\\
Domain $g\circ g$: $x\neq (2k+1)\frac{\pi}{2}$ and $x\neq k\pi+ \arctan \left(\frac{\pi}{2}\right)$ for all $k\in \mathbb Z$
\\
in some order:$2 \tan{}x, \tan{}(2 x), 4 x, \tan{}(\tan{}x) $
\end{tabular}
}
\item $f(x)=\frac{x+1}{x-1}$, $g(x)=\frac{x-1}{x+1}$.
\answer{ 
\begin{tabular}{l}
Domain $ f\circ f$: $x\neq 1$. Domain $g\circ g$: $x\neq 0$, $x\neq -1$\\
Domain $f\circ g$: $x \neq -1$. Domain $g\circ f$: $x\neq 0$, $x\neq 1$\\
in some order: $- x, \frac{1}{x}, x, -\frac{1}{x} $
\end{tabular}
}
\end{enumerate}

\item (Stewart, 7ed., page 21, problems 27-30)
Evaluate the difference and simplify your answer.
\begin{multicols}{2}
\begin{enumerate}
\item $\frac{f(3+h)-f(3)}{h}$, where $f(x)=4+3x-x^2$.
\answer{$-3-h$}
\item $\frac{f(a+h)-f(a)}{h}$, where $f(x)= x^3$.
\answer{$ 3a^2+3ah+h^2$}
\item $\frac{f(x)-f(a)}{x-a}$, where $f(x)=\frac{1}{x}$.
\answer{$-\frac{1}{ax}$.}
\item $\frac{f(x)-f(1)}{x-1}$, where $f(x)=\frac{x+3}{x+1}$.
\answer{$-\frac{1}{x+1}$.}
\end{enumerate}
\end{multicols}

\item  Compute the composite functions $(f\circ g)(x)$, $(g\circ f)(x)$. Simplify your answer to a single fraction. Find the domain of the composite function. The answer key has not been fully proofread, use with caution. 

\begin{enumerate}[ref={\fcProblemRef}]
\item $\displaystyle f{}({{x}})=\frac{x+2}{x-2},
g{}({{x}})=\frac{x-1}{x+2}$.

\answer{
\begin{tabular}{rl}
$(f\circ g)(x)= \frac{3+3 x}{-5- x}$& $x\neq -2, -5$\\ 
$(g\circ f)(x)=\frac{4}{-2+3 x}$& $x\neq 2, \frac{2}{3}$ 
\end{tabular}
}
\item 
$\displaystyle f{}({{x}})=\frac{x+1}{3x-2}, g{}({{x}})= \frac{x-2}{x-1}
$.

\answer{
\begin{tabular}{rl}
$(f\circ g)(x)= \frac{-3+2 x}{-4+x}
$ & $x\neq 4, 1$ \\
$(g\circ f)(x)=\frac{5-5 x}{3-2 x}$
& $x\neq \frac{2}{3}, \frac{3}{2}$
\end{tabular}
}
\item 
$\displaystyle f{}({{x}})=\frac{2x+1}{3x-1},
g{}({{x}})=\frac{x-2}{2x-1}
$.

\answer{
\begin{tabular}{rl}
$(f\circ g)(x)=\frac{-5+4 x}{-5+x}
$ &$x\neq 5, \frac{1}{2}$ \\
$(g\circ f)(x)=\frac{3-4 x}{3+x}
$ &$x\neq -3, \frac{1}{3}$
\end{tabular}
}
\item 
$\displaystyle f{}({{x}})=\frac{x+1}{x-2},
g{}({{x}})=\frac{x+2}{2x-1}
$.

\answer{
\begin{tabular}{rl}
$(f\circ g)(x)= \frac{1+3 x}{4-3 x}
$&$x\neq \frac{4}{3}, \frac{1}{2}$\\ 
$(g\circ f)(x)=\frac{-3+3 x}{4+x}
$&$x\neq -4, 2$
\end{tabular}
}
\item 
$\displaystyle f{}({{x}})=\frac{5x+1}{4x-1},
g{}({{x}})=\frac{4x-1}{3x+1}
$.

\answer{
\begin{tabular}{rl}
$(f\circ g)(x)= \frac{-4+23 x}{-5+13 x}
$&$x\neq \frac{5}{13}, -\frac{1}{3 }$\\
$(g\circ f)(x)=\frac{5+16 x}{2+19 x}
$&$x\neq -\frac{2}{19}, \frac{ 1}{4}$
\end{tabular}
}
\item $\displaystyle  f(x)= \frac{3x-5}{x-2}$, $\displaystyle g(y)=\frac{y-2 }{y-4} $. 

\answer{ 
\begin{tabular}{rl}
$(f\circ g)(x)=\frac{-2 x+14}{- x+6}$&$x\neq 6, 4$\\
$(g\circ f)(x)=\frac{x-1}{- x+3}$&$x\neq 3,2$
\end{tabular}
}
\item $\displaystyle  f(x)= \frac{x-3}{x+2}$, $\displaystyle g(y)=\frac{y+3 }{y-4} $. 

\answer{ 
\begin{tabular}{rl}
$(f\circ g)(x)=\frac{-2x+15}{3 x-5}$&$x\neq \frac{5}{3}, 4$\\ 
$(g\circ f)(x)=\frac{4 x+3}{-3 x-11}$&$x\neq -\frac{11}{3}, -2$
\end{tabular}
}
\end{enumerate}

\item  (Stewart, 7ed., page 21, problems 45, 46, 49)
Plot the piecewise defined functions.
\begin{multicols}{2}
\begin{enumerate}
\item $G(x)=\frac{3x+|x|}x$.
\item $g(x)=|x|-x$.
\item $f(x)=\doublebrace{x+2}{x\leq -1}{x^2}{x\geq -1}$.
\end{enumerate}
\end{multicols}

\item (Stewart, 7ed, page 21, problems 55-56)
Write down formulas for function whose graphs are as follows. The graphs are up to scale. The arc is a part of a circle.
\begin{multicols}{2}
\begin{enumerate}
\psset{xunit=0.4cm, yunit=0.4cm}
\item
\tiny
\begin{pspicture}(-1,-1)(6,5)
\psaxes{->}(0,0)(-1,-1)(6,5)
\psline[linecolor=red](0,3)(3, 0)(5, 4)
\fcFullDot{5}{4}
\rput[r](4.9, 4){$(5, 4)$}
\rput[b](6,0.1 ){$x$}
\rput[l](0.1,5 ){$y$}
\end{pspicture}
\normalsize
\item
\tiny
\psset{xunit=0.4cm, yunit=0.4cm}
\begin{pspicture}(-4,-1)(4,5)
\psaxes{->}(0,0)(-4.5,-1)(4.5,4)
\psplot[linecolor=red]{-2}{2}{4 x x mul sub sqrt }
\psline[linecolor=red](2,0)(4,3)
\psline[linecolor=red](-2,0)(-4,3)
\rput[b](4.5,0.1 ){$x$}
\rput[l](0.1,5 ){$y$}
\fcFullDot{4}{3}
\rput[r](3.9, 3){$(4, 3)$}
\fcFullDot{-4}{3}
\rput[l](-3.9, 3){$(-4, 3)$}

\end{pspicture}
\normalsize
\end{enumerate}
\end{multicols}

%\begin{problem}Graph the functions by hand, by applying consecutively the transformations learned in class.
\begin{multicols}{2}
\begin{enumerate}
\item $y=\frac{1}{x}$.
\item $y=\frac{1}{x+1}$.
\item $y=\frac{1}{2x+1}$.
\item $y=\frac{3}{2x+1}$.
\item $y=\frac{3+x}{2x+1}$.
\item $y=\left|\frac{3+x}{2x+1}\right|$.
\end{enumerate}
\end{multicols}
\end{problem}
\homeworkEnd
\end{comment}
\begin{comment}
\homeworkStart{on Lectures 3 \\Will be quizzed: date to be announced}{}
\item \tooeasy Convert from degrees to radians.
\begin{multicols}{3}
\begin{enumerate}
\item $15^\circ$.
\item $30^\circ$.
\item $36^\circ$.
\item $45^\circ$.
\item $60^\circ$.
\item $75^\circ$.
\item $90^\circ$.
\item $120^\circ$.
\item $135^\circ$.
\item $150^\circ$.
\item $180^\circ$.
\item $225^\circ$.
\item $270^\circ$.
\item $305^\circ$.
\item $360^\circ$.
\item $405^\circ$.
\item $1200^\circ$.
\item $-900^\circ$.
\item $-2014^\circ$.
\end{enumerate}
\end{multicols}

\item \tooeasy (Textbook page A32, problems 7-12). 
Convert from radians to degrees.
\begin{multicols}{3}
\begin{enumerate}
\item $4\pi$.
\item $-7/2\pi$.
\item $5/12\pi$.
\item $8/3\pi$.
\item $-3/8\pi$.
\item $5$.
\end{enumerate}
\end{multicols}
\item \begin{problem}
(Textbook page A32-, problems 45, 46, 47, 48, 49, 50, 51, 52, 56, 57, 58).
\begin{multicols}{3}
\begin{enumerate}
\item $\sin \theta\cot \theta =\cos \theta$.
\item $(\sin x +\cos x)^2=1+\sin(2x)$.
\item $\sec y - \cos y= \tan y \sin y$.
\item $\tan^2 \alpha-\sin^2 \alpha=\tan^2\alpha\sin^2\alpha$.
\item $\cot^2\theta+\sec^2\theta=\tan^2\theta+\csc^2\theta$.
\item $2\csc 2t= \sec t \csc t$.
\item $\tan 2\theta =\frac{2\tan \theta}{1-\tan^2\theta} $.
\item $\frac{1}{1-\sin \theta}+ \frac{1}{1+\sin \theta}=2\sec^2\theta$.
\item $\tan x + \tan y = \frac{\sin (x+y)}{\cos x \cos y}$.
\item $\sin 3\theta +\sin \theta = 2 \sin 2\theta \cos \theta $.
\item $\cos 3\theta = 4\cos^3\theta-3\cos \theta $.
\end{enumerate} 
\end{multicols}
\end{problem}
\item \begin{problem}(Textbook page A33, problems 65-72).
Find all values of $x$ in the interval $[0,2\pi]$ that satisfy the 
equation.
\begin{multicols}{3}
\begin{enumerate}
\item $2\cos x - 1=0$.
\item $3\cot^2 x= 1$.
\item $2\sin^2 x= 1$.
\item $|\tan x|=1 $.
\item $\sin 2x = \cos x $.
\item $2\cos x +\sin 2x=0$.
\item $\sin x =\tan x$.
\item $2+\cos 2x = 3 \cos x$.
\end{enumerate}
\end{multicols}
\end{problem}

\homeworkEnd
\end{comment}
\begin{comment}
\homeworkStart{on Lectures 4\\Will be quizzed: Wednesday February 19}{}

\item  \tooeasy (Textbook page 69, problems 3-9). 
Evaluate the limits. Justify your computations.
\begin{multicols}{3}
\begin{enumerate}
\item $\displaystyle\lim\limits_{x\to 3} 5x^3-3x^2+x-6$.
\answer{$105$}
\item $\displaystyle\lim\limits_{x\to -1} (x^4-3x)(x^2+5x+3)$.
\answer{$-4$}
\item $\displaystyle\lim\limits_{t\to -2} \frac{t^4- 2}{2t^2 -3t +2} $.
\answer{$\frac78$}
\item $\displaystyle\lim\limits_{u\to -2}\sqrt{u^4+3u +6}$.
\answer{$4$}
\item $\displaystyle\lim\limits_{x \to 8}(1+ \sqrt[3]{x} )(2- 6x^2 + x^3)$.
\answer{$390$}
\item $\displaystyle\lim\limits_{t \to 2}\left( \frac{t^2 - 2 }{ t^3-3t+5} \right)^2$.
\answer{$\frac{4}{49}$}
\item $\displaystyle\lim\limits_{x\to 2}\sqrt{ \frac{2x^2 + 1}{ 3x-2}}$.
\answer{$\frac32$}
\end{enumerate}
\end{multicols}

\item Evaluate the limit if it exists.
\begin{multicols}{2}
\begin{enumerate}[ref={\fcProblemRef}]
\item \label{problemlim(xto2)(x^2-5x+6)/(x-2)}
$\displaystyle\lim\limits_{x\to 2}\frac{x^2-5x+6}{x-2} $. 

\answer{$-1$}
\item $\displaystyle\lim\limits_{x\to 3}\frac{x^2-3x}{x^2-2x-3} $.

\answer{$\frac{3}{4}$}
\item \label{problemlimxto-2(2x^2+x-6)/(x^2-4)}

$\displaystyle \lim_{x\to -2} \frac{2x^2+x-6}{x^2-4}$
\answer{$\frac{7}{4}$}
\item $\displaystyle\lim\limits_{x\to 2}\frac{x^2-5x-6}{x-2} $.

\answer{DNE}
\item $\displaystyle\lim\limits_{x\to -1}\frac{x^2-3x}{x^{2}-2x-3} $.

\answer{DNE}

\item $\displaystyle\lim\limits_{x\to -2}\frac{x^2-4}{2x^2+5x+2} $.

\answer{$\frac{4}{3}$}

\item $\displaystyle\lim\limits_{x\to -1}\frac{2x^2+3x+1}{3x^2-2x-5} $.

\answer{$\frac{1}{8}$}

\item \label{problemlimxto-4(x^2+7x+12)/(x^2+6x+8)}
$\displaystyle \lim_{x \to -4}\frac{x^{2}+7 x+12}{x^{2}+6 x+8}$.

\answer{$\frac{1}{2}$}


\item $\displaystyle\lim\limits_{h\to 0}\frac{(-3+h)^2-9}{h} $.

\answer{$-6$}

\item $\displaystyle\lim\limits_{h\to 0}\frac{(-2+h)^3+8}{h} $.

\answer{$12$}
\item $\displaystyle\lim\limits_{x\to -3}\frac{x+3}{x^3+27} $.

\answer{$\frac{1}{27}$}

\item $\displaystyle\lim\limits_{x\to 1}\frac{x^4-1}{x^3-1} $.

\answer{$\frac{4}{3}$}
\item $\displaystyle\lim\limits_{h\to 0}\frac{\sqrt{4+h}-2}{h} $.

\answer{$\frac{1}{4}$}
\item $\displaystyle\lim\limits_{x\to 3} \frac{\sqrt{5x+1}-4}{x-3}$.

\answer{$\frac{5}{8}$}

\item $\displaystyle\lim\limits_{x\to -3} \frac{\sqrt{x^2+16}-5}{x+3}$.

\answer{$-\frac{3}{5}$}

\item $\displaystyle\lim\limits_{x\to -3} \frac{\frac{1}{3}+ \frac{1}{x}} {3+x}$.

\answer{$-\frac{1}{9}$}
\item $\displaystyle\lim\limits_{x\to -2} \frac{x^2+4x+4}{x^4-16}$.

\answer{$0$}
\item $\displaystyle\lim\limits_{x\to 0} \frac{\sqrt{1+x}- \sqrt{1-x}}{x}$.

\answer{$1$}
\item $\displaystyle\lim\limits_{x\to 0}\left(\frac{1}x -\frac{1}{x^2+x}\right)$.

\answer{$1$}
\item $\displaystyle\lim\limits_{x\to 9} \frac{3-\sqrt{x}}{9x-x^2}$.

\answer{$\frac{1}{54}$}
\item $\displaystyle\lim\limits_{h \to 0}\frac{(2+h)^{-1}-2^{-1}}{h} $.

\answer{$-\frac{1}{4}$}

\item $\displaystyle\lim\limits_{x\to 0} \left(\frac{1}{x\sqrt{1+x}}-\frac{1}{x} \right)$.

\answer{$-\frac{1}{2}$}

\item 
$\displaystyle\lim\limits_{h\to 0}\frac{(x+h)^3-x^3}{h} $.

\answer{$3x^2$}

\item 
\label{problemlim_hto0_(1/(x+h)^2-1/x^2)/h} $\displaystyle\lim\limits_{h\to 0}\frac{\frac{1}{(x+h)^2}-\frac{1}{x^2}}{h} $.

\answer{$-\frac{2}{x^3}$}

\item 
\label{problemlimhto0(1/(2+h)^2-1/4)/h}
$ \displaystyle \lim_{h\to 0} \frac{\frac{1}{(2+h)^2}-\frac{1}{4}}{h}  $.

\answer{$-\frac{1}{4}$}
\item 
\label{problemlimhto0(1/(1+h)^2-1)/h}
$ \displaystyle \lim_{h\to 0} \frac{\frac{1}{(1+h)^2}-1}{h}  $.

\answer{$-2$}
\end{enumerate}
\end{multicols}

\homeworkEnd
\end{comment}
\begin{comment}
\homeworkStart{on Lectures 5 and 6. \\ Problem types will be included on the Test}{}
\item Show the following limits do not exist and compute whether they evaluate to $\infty $, $-\infty$, or neither. 
\begin{multicols}{3}
\begin{enumerate}
\item $\displaystyle\lim_{x\to 3^+} \frac{x^{2}+x-1}{x^2-2x-3} $.
\answer{$\infty$.}
\item $\displaystyle\lim_{x\to 3^-} \frac{x^{2}+x-1}{x^2-2x-3} $.
\answer{$-\infty$.}
\item $\displaystyle\lim_{x\to 1^+} \frac{x^2+1}{\sqrt{x^2+3 }-2} $.
\answer{$\infty$.}
\item $\displaystyle\lim_{x\to 1^-} \frac{x^2+1}{\sqrt{x^2+3 }-2} $.
\answer{$-\infty$.}
\item $\displaystyle\lim_{x\to 2^+} \frac{\sqrt{x^3-8}}{ -x^2+x+2} $.
\answer{$-\infty$.}
\item $\displaystyle\lim_{x\to -1^+} \frac{\sqrt[3]{x^2+2x+1}}{ x^2-2x-3} $.
\answer{$-\infty$.}

\end{enumerate}
\end{multicols}

\item Find the horizontal and vertical asymptotes of the graph of the function. Check your work by plotting the function.

\begin{multicols}{2}
\begin{enumerate}[ref={\fcProblemRef}]
\item 
\label{problemAsymptotesy=(2x/(sqrt(x^2+x+3)-3))} $\displaystyle y=\frac{2x}{\sqrt{x^2+x+3}-3}$. 

\answer{Vertical: $x=2, x=-3$, horizontal: $y=2, y=-2$}

\item $\displaystyle y=\frac{3x^2}{\sqrt{x^2+2x+10}-5}$. 

\answer{Vertical: $x=3, x=-5$, horizontal: none.}
\item $\displaystyle y=\frac{3x+1}{x-2}$.

\answer{vertical: $x=2$, horizontal: $y=3$}
\item \label{problemAsymptotesy=(x^2-1)/(2x^2-3x-2)}
$\displaystyle y=\frac{x^2-1}{2x^2-3x-2}$.

\answer{vertical: $x=2, x=-\frac{1}{2}$, horizontal: $y=\frac{1}{2}$}

\item $\displaystyle y=\frac{2x^2-2x-1}{x^2+x-2}$.

\answer{vertical: $x=1, x=-2$, horizontal: $y=2$}

\item \label{problemAsymptotesy=(-5x^2-3x+5)/(x^2-2x-3)}
$\displaystyle 
f(x)=\frac{-5 x^{2}-3 x+5}{x^{2}-2 x-3}
$

\answer{vertical: $x=-1$, $x=3$, horizontal: $y=-5$}



\item $\displaystyle y=\frac{1+x^4}{x^2-x^4}$.

\answer{vertical: $x=0, x=1, x=-1$, horizontal: $y=-1$}
\item $\displaystyle y=\frac{x^3-x}{x^2-7x+6}$.

\answer{vertical: $x=6$, no horizontal asymptote}
\item $\displaystyle y=\frac{x-9}{\sqrt{4x^2+3x+3}}$.

\answer{no vertical asymptote, horizontal: $y=\pm\frac12$}

\item $\displaystyle y=\frac{\sqrt{x^2+1}- x }{x}$.

\answer{vertical: $x=0$, horizontal: $y=0$, $y=-2$}
\item \label{problemAsymptotesy=x/(sqrt(x^2+3) -2x)}
$\displaystyle 
f(x)= \frac{x}{\sqrt{x^2+3} -2x}
$

\answer{vertical: $x=1$, horizontal: $y=-\frac{1}{3}$, $y=-1$}

\end{enumerate}
\end{multicols}
\item Find the limit or show that it does not exist. If the limit does not exist, indicate whether it is $\pm\infty$, or neither. The answer key has not been proofread, use with caution.
\begin{multicols}{3}
\begin{enumerate}[ref={\fcProblemRef}]
\item $\displaystyle \lim\limits_{x\to\infty }\frac{x-2}{2x+1}$.

\answer{$\frac12$}
\item $\displaystyle \lim\limits_{x\to\infty }\frac{1-x^2}{x^3-x-1}$.

\answer{$ 0$}
\item $\displaystyle \lim\limits_{x\to-\infty }\frac{x-2}{x^2+5}$.

\answer{$ 0$}
\item \label{problemlimxtominusinfty(3x^3+2)/(2x^3-4x+5)} $\displaystyle \lim\limits_{x\to-\infty }\frac{3x^3+2}{2x^3-4x+5}$.

\answer{$ \frac{3}{2}$}
\item $\displaystyle \lim\limits_{x\to\infty }\frac{\sqrt{x}+x^2}{\sqrt{x}-x^2}$.

\answer{$-1$}
\item $\displaystyle \lim\limits_{x\to\infty }\frac{3-x\sqrt{x}}{2x^{\frac{3}{2}}-2}$.

\answer{$-\frac12$}
\item $\displaystyle \lim\limits_{x\to\infty }\frac{(2x^2+3)^2}{(x-1)^2(x^2+1)}$.

\answer{$ 4$}
\item $\displaystyle \lim\limits_{x\to\infty }\frac{x^2-3}{\sqrt{x^4+3}}$.

\answer{$1$}
\item $\displaystyle \lim\limits_{x\to\infty }\frac{\sqrt{16x^6-3x}}{x^3+2}$.

\answer{$4$}
\item $\displaystyle \lim\limits_{x\to-\infty }\frac{\sqrt{16x^6-3x}}{x^3+2}$.

\answer{$-4$}
\item $\displaystyle \lim\limits_{x\to\infty}\sqrt{4x^2+x}-2x$.

\answer{$\frac{1}{4}$}
\item $\displaystyle \lim\limits_{x\to-\infty} x+\sqrt{x^2+3x} $.

\answer{$-\frac{3}{2} $}
\item $\displaystyle \lim\limits_{x\to\infty}\sqrt{x^2+ax}-\sqrt{x^2+bx}$.

\answer{$\frac{a-b}2$}
\item $\displaystyle \lim\limits_{x\to\infty}\cos x$.

\answer{DNE}
\item $\displaystyle \lim\limits_{x\to\infty}\frac{x^4+x}{x^3-x+2}$.

\answer{$\infty$}
\item $\displaystyle \lim\limits_{x\to\infty}\sqrt{x^2+1}$.

\answer{$\infty$}
\item $\displaystyle \lim\limits_{x\to-\infty}(x^4+x^5)$.

\answer{$-\infty$}
\item $\displaystyle \lim\limits_{x\to-\infty}\frac{\sqrt{1+x^6}}{1+x^2}$.

\answer{$\infty$}
\item $\displaystyle \lim\limits_{x\to\infty}(x-\sqrt{x})$.

\answer{$\infty$}
\item $\displaystyle \lim\limits_{x\to\infty}(x^2-x^3)$.

\answer{$-\infty$}
\item $\displaystyle \lim\limits_{x\to\infty}x\sin x$.

\answer{DNE}
\item $\displaystyle \lim\limits_{x\to\infty}\sqrt{x}\sin x$.

\answer{DNE}
\end{enumerate}
\end{multicols}
\item Use the Intermediate Value Theorem to show that there is a real number solution of the given equation in the specified interval. 


\begin{multicols}{2}
\begin{enumerate}[ref={\fcProblemRef}]
\item 
$x^5+x-3=0$ where $x\in (1,2)$.

\item $\sqrt[4]{x}=1-x$ where $x\in \mathbb R$ (i.e., $x$ is an arbitrary real number).
\item $\cos x=2x$, where $x\in (0,1)$.
\item 
$\sin x=x^2-x-1$, where $x\in \mathbb R$ (i.e., $x$ is an arbitrary real number).

\item 
$\cos x=x^4$, where $x\in \mathbb R$ (i.e., $x$ is an arbitrary real number).

\item $x^5-x^2+x+3=0$, where $x\in \mathbb R$.



\end{enumerate}
\end{multicols}

\homeworkEnd
\end{comment}
\begin{comment}
\homeworkStart{Homework Math 140, Lectures 6 and 7. \\ Will be quizzed Wednesday September 25}{}
Find the (implied) domain of $f(x)$. Extend the definition of $f$ at $x=3$ to make $f$ continuous at $3$.
\begin{multicols}{2}
\begin{enumerate}
\item $f(x)=\frac{x^2-x-6}{x-3}$.

\answer{
\begin{tabular}{l}
Implied domain: $x\in (-\infty, 3)\cup (3,\infty)$. \\
Extend $f(x)$ to $\bar f(x)=x+2$.
\end{tabular}
}
\item $f(x)=\frac{x^3-27}{x^2-9}$.

\answer{
\begin{tabular}{l}
Implied domain: $x\in (-\infty, -3)\cup(-3,3)\cup (3,\infty)$. \\
Extend $f(x)$ to $\bar f(x)=\frac{x^2+3x+9}{x+3}$ with domain $x\in (-\infty, -3)\cup(-3,\infty)$.
\end{tabular}
}
\end{enumerate}
\end{multicols}
\begin{problem}(Textbook, page 93, problem 63-64) 
For which values of $x$ is $f$ continuous?
\begin{itemize}
\item $f(x)=\doublebrace{0}{\mathrm{if~} x\mathrm{~is~rational}}{1}{\mathrm{if~}x~\mathrm{is~irrational}}$
\item $f(x)=\doublebrace{0}{\mathrm{if~} x\mathrm{~is~rational}}{x}{\mathrm{if~}x~\mathrm{is~irrational}}$
\end{itemize}
\end{problem}
\begin{problem} This problem is of higher difficulty and will not be used for tests. For which values of $x$ is $f$ continuous?
\[f(x)=\doublebrace{\frac{1}{q^2}}{\mathrm{if~}x\mathrm{~is~rational~and~} x=\frac{p}{q} }{0}{\mathrm{if~}x~\mathrm{is~irrational}}\]
where in the first item $p,q$ are relatively prime integers (i.e., integers without a common divisor).
\end{problem}
Match the graph of each the following functions
\begin{multicols}{2}
\begin{enumerate}[ref={\fcProblemRef}]
\item ~{}
\psset{xunit=1cm, yunit=1cm}
\begin{pspicture}(-2.5, -2.5)(2.5,2.5)
\psframe*[linecolor=white](-5,-5)(5,5)
\psaxes[ticks=none, labels=none]{<->}(0,0)(-2.5,-2.5)(2.5,2.5)
%Function formula: -8 ((x) ((x) (x)))+2 (x)
\psplot[linecolor=red, plotpoints=1000]{-0.8}{0.8}{x 2 mul x x mul x mul -8 mul add }
\end{pspicture}
\item ~{}
\psset{xunit=1cm, yunit=1cm}
\begin{pspicture}(-2.5,-2.5)(2.5,2.5)
\psframe*[linecolor=white](-2.5,-2.5)(2.5,2.5)
\psaxes[ticks=none, labels=none]{<->}(0,0)(-2.5,-2.5)(2.5,2.5)
%Function formula: -2+x
\psplot[linecolor=red, plotpoints=1000]{1}{2.5}{x -2 add } %Function formula: - (x)
\psplot[linecolor=red, plotpoints=1000]{-1}{1}{x -1 mul } %Function formula: 2+x
\psplot[linecolor=red, plotpoints=1000]{-2.5}{-1}{x 2 add }
\end{pspicture}
\item ~{}
\psset{xunit=1cm, yunit=1cm}
\begin{pspicture}(-2.5,-2.5)(2.5,2.5)
\psframe*[linecolor=white](-2.5,-2.5)(2.5,2.5)
\psaxes[ticks=none, labels=none]{<->}(0,0)(-2.5,-2.5)(2.5,2.5)
%Function formula: - ((1)/((x)^{2}+1))
\psplot[linecolor=red, plotpoints=1000]{-2.5}{2.5}{1 1 x 2 exp add div -1 mul }
\end{pspicture}
\item ~{}
\psset{xunit=1cm, yunit=1cm}
\begin{pspicture}(-2.5,-2.5)(2.5,2.5)
\psframe*[linecolor=white](-2.5,-2.5)(2.5,2.5)
\psaxes[ticks=none, labels=none]{<->}(0,0)(-2.5,-2.5)(2.5,2.5)
%Function formula: - (((x)^{2}) ((x) (x)))+(x)^{2}
\psplot[linecolor=red, plotpoints=1000]{-1.46}{1.46}{x 2 exp x x mul x 2 exp mul -1 mul add }
\end{pspicture}
\end{enumerate}
\end{multicols}
to the graph of its derivative among the graphs below
\begin{multicols}{2}
\begin{enumerate}
\item ~{}
\psset{xunit=1cm, yunit=1cm}
\begin{pspicture}(-2.5,-2.5)(2.5,2.5)
\psframe*[linecolor=white](-2.5,-2.5)(2.5,2.5)
\psaxes[ticks=none, labels=none]{<->}(0,0)(-2.5,-2.5)(2.5,2.5)
%Function formula: (x)/(((x)^{2}+1)^{2})
\psplot[linecolor=blue, plotpoints=1000]{-2.5}{2.5}{x 1 x 2 exp add 2 exp div }
\end{pspicture}

\item ~{}\psset{xunit=1cm, yunit=1cm}
\begin{pspicture}(-2.5,-2.5)(2.5,2.5)
\psframe*[linecolor=white](-2.5,-2.5)(2.5,2.5)
\psaxes[ticks=none, labels=none]{<->}(0,0)(-2.5,-2.5)(2.5,2.5)
%Function formula: -24 ((x) (x))+2
\psplot[linecolor=blue, plotpoints=1000]{-0.427}{0.427}{2 x x mul -24 mul add }
\end{pspicture}
\item ~{}
\psset{xunit=1cm, yunit=1cm}
\begin{pspicture}(-2.5,-2.5)(2.5,2.5)
\psframe*[linecolor=white](-2.5,-2.5)(2.5,2.5)
\psaxes[ticks=none, labels=none]{<->}(0,0)(-2.5,-2.5)(2.5,2.5)
%Function formula: -4 ((x)^{3})+2 (x)
\psplot[linecolor=blue, plotpoints=1000]{-1.045}{1.045}{x 2 mul x 3 exp -4 mul add }
\end{pspicture}

\item ~{}
\psset{xunit=1cm, yunit=1cm}
\begin{pspicture}(-2.5,-2.5)(2.5,2.5)
\psframe*[linecolor=white](-2.5,-2.5)(2.5,2.5)
\psaxes[ticks=none, labels=none]{<->}(0,0)(-2.5,-2.5)(2.5,2.5)
%Function formula: 1
\psplot[linecolor=blue, plotpoints=1000]{1}{2.5}{1}
\fcHollowDotBlue{-1}{1}
%Function formula: -1
\fcHollowDotBlue{-1}{-1}
\psplot[linecolor=blue, plotpoints=1000]{-1}{1}{-1}
\fcHollowDotBlue{1}{-1}
%Function formula: 1
\fcHollowDotBlue{1}{1}
\psplot[linecolor=blue, plotpoints=1000]{-2.5}{-1}{1}
\end{pspicture}

\end{enumerate}
\end{multicols}
Give reasons for your choices. Can you guess formulas that would give a similar (or precisely the same) graph, and confirm visually your guess using a graphing device?

\homeworkEnd
\end{comment}
\begin{comment}
\homeworkStart{on Lecture 7. \\ Will be quizzed Wednesday March 5 \\ Problem 2(d) uses material from Lecture 8 and \\will not appear on the quiz}{}
\item % begin homework inverse-functions3
Find the inverse function. You are asked to do the algebra only; you are not asked to determine the domain or range of the function or its inverse. 
\begin{enumerate} [ref={\fcProblemRef}]
\item $f(x)= 3x^2+4x-7$, where $x\geq -\frac{2}{3}$.
\answer{$f^{-1}(x)= -\frac{2}3+\frac{\sqrt{25+3x}}{3}, \quad x\geq -\frac{25}{3}$}

\item $f(x)= 2x^2+3x-5$, where $x\geq -\frac{3}{4}$.
\answer{$f^{-1}(x)=-\frac{3}{4}+\frac{\sqrt{49+8x}}{4}, \quad x\geq -\frac{49}{8}$}

\item $\displaystyle f(x)= \frac{2x+5}{x-4}$, where $x\neq 4$.
\answer{$f^{-1}(x)=\frac{4x+5}{x-2}, \quad x\neq 2$}

\pointsii{3} \label{problemFindInversef=(3x+5)/(2x-4)} $\displaystyle f(x)= \frac{3x+5}{2x-4}$, where $x\neq 2$.

\hiddenanswer{$\displaystyle f^{-1}(x) = \frac{ 4 x +5}{2x-3}, \quad x\neq \frac{3}{2}$}
\item \label{problemFindIversef=(5x+6)/(4x+5)}  $\displaystyle f(x)= \frac{5x+6}{4x+5}$.

\answer{$f^{-1}(x)= \frac{-5x+6}{4x-5}$, $x\neq \frac{5}{4}$}
\item  $\displaystyle f(x)= \frac{2x-3}{-3x+4}$.

\answer{$f^{-1}(x)=\frac{4x+3}{3x+2}  $, $x\neq -\frac{2}{3}$}
\item $f(x)=2^{2x}+2^{x}-2$.
\answer{$f^{-1}(x) =\log_2\frac{-1+\sqrt{9+4x}}{2}, \quad x\geq -2$}

\end{enumerate}
% end homework inverse-functions3

\homeworkEnd
\end{comment}

\begin{comment}
\homeworkStart{on Lecture 8. \\ Will be quizzed Friday March 7}{}
\item \tooeasy 
\begin{problem}(Textbook page 408, problems 3-8). Find the exact value of each expression.
\begin{multicols}{3}
\begin{enumerate}
\item $\log_5 125$. 

\answer{$3$}
\item $\log_3 \frac{1}{27}$. 

\answer{$-3$}
\item $\ln \left(\frac{1}{e}\right) $. 

\answer{$-1$}
\item $\log_{10}\sqrt{10}$. 

\answer{$\frac12$}
\item $e^{\ln 4.5}$.  

\answer{$4.5$}
\item $\log_{10} 0.0001 $.  

\answer{$4$}
\item $\log_{1.5}2.25$.  

\answer{$2$}
\item $\log_5 4- \log_5 500$.  

\answer{$-3$}
\item $\log_2 6 - \log_2 15 +\log_2 20$. 

 \answer{$3$}
\item $\log_3 100- \log_3 18 - \log _3 50 $.  

\answer{$-2$}
\item $e^{-2\ln 5}$.  

\answer{$\frac{1}{25}$}
\item $\ln \left(\ln e^{e^{10}}\right)$. 

 \answer{$10$}
\end{enumerate}
\end{multicols}
\end{problem}
\item \tooeasy % begin homework logarithms-basic2
Use the definition of a logarithm to evaluate each of the following without using a calculator.  

\begin{enumerate}
\item   $\log_2 16$

\item   $\log_3 27$

\item   $\log_3 (1/9)$

\item   $\log_{10} 1000$

\item   $\log_{6} 36^{-2/3}$

\item   $\log_{2} (8\sqrt{2})$

\pointsii{2} $\log_7(49^x/343^y)$

\solution{%
\begin{align*}
\log_7(49^x/343^y) & = \log_749^x - \log_7343^y \\
 & = x\log_749 - y\log_7343 \\
\intertext{But $49 = 7^2$ and $343=7^3$, therefore}
\log_7(49^x/343^y) & = 2x-3y.
\end{align*}
}%


\end{enumerate}
% end homework logarithms-basic2

\item \tooeasy % begin homework logarithms-combine
Express each of the following as a single logarithm.  

\begin{enumerate}
\item   $\ln 4 + \ln 6 - \ln 5$

\pointsii{2} $2\ln 2 - 3\ln 3 + 4\ln 4$

\solution{%
\begin{align*}
2\ln 2 - 3\ln 3 + 4\ln 4 & = \ln 2^2 - \ln 3^3 + \ln 4^4 \\
 & = \ln 4 - \ln 27 + \ln 256 \\
 & = \ln \Big( \frac{4}{27}\Big) + \ln 256 \\
 & = \ln \Big( \frac{4\cdot 256}{27}\Big) \\
 & = \ln \Big( \frac{1024}{27}\Big).
\end{align*}
}%

\item   $\ln 36 - 2\ln 3 - 3\ln 2$

\end{enumerate}
% end homework logarithms-combine

\item % begin homework inverse-functions2
Find the inverse function and its domain. 
\begin{enumerate}
\item  $y=\ln (x+3)$.
\answer{$f^{-1}(x)=e^x-3$}

\solution{%
\begin{align*}
y & = \ln (x+3) \\
e^y & = e^{\ln (x+3)} \\
e^y & = x + 3 \\
e^y - 3 & = x \\
\text{Therefore} \quad f^{-1}(y) & = e^y - 3.
\end{align*}
The domain of $e^y$ is all real numbers, so the domain of $f^{-1}$ is all real numbers.  
}%

\item $f(x)=e^{x^3}$.
\answer{$f^{-1}(x)=\sqrt[3]{\ln x}, \quad x>0$}

\item $y=(\ln x)^2$, $x\geq 1$.
\solution{ ~\\
$ \begin{array}{rcll|l}y&=&(\ln x)^2 &&\mathrm{take~ \sqrt{~} ~on~both~sides,~ } y\geq 0 \\ \sqrt{y}&=&\ln x&&\mathrm{ ~exponentiate} \\ e^{\sqrt{y}}&=&e^{\ln x}=x \\ f^{-1}(y)&=&e^{\sqrt{y}} \\f^{-1}(x)&=&e^{\sqrt{x}} \end{array}$
}
\answer{$f^{-1}(x)=e^{\sqrt{x}}, \quad x\geq 0 $}

\pointsii{5}  $y=\frac{e^x}{1+2e^x}$.
\hiddenanswer{$f^{-1}(x)= \ln \left(\frac{x}{1-2x}\right) $, \quad $x\in (0, \frac12) $}

\solution{%
\begin{align*}
y & = \frac{e^x}{1+2e^x} \\
y(1+2e^x) & = e^x \\
y & = e^x(1-2y) \\
\frac{y}{1-2y} & = e^x \\
\ln\frac{y}{1-2y} & = \ln e^x \\
\ln\frac{y}{1-2y} & = x \\
\text{Therefore} \quad f^{-1}(y) & = \ln\frac{y}{1-2y}.
\end{align*}
The natural logarithm function is only defined for positive input values.  
Therefore the domain is the set of all $y$ for which 
\begin{align*}
\frac{y}{1-2y} & > 0.
\end{align*}
This inequality holds if the numerator and denominator are both positive or both negative.  
This happens if either
\begin{enumerate}
\item  $y > 0$ and $y < 1/2$, or 
\item  $y < 0$ and $y > 1/2$.
\end{enumerate}
The latter option is impossible, so the domain is $\{ y \in \mathbb{R} \ | \ 0 < y < 1/2\}$.  
}%

\end{enumerate}
% end homework inverse-functions2

\item Solve each equation for $x$. If available, use a calculator to give an ($\approx$) answer in decimal notation. If available, use a calculator to verify your approximate solutions.
\begin{multicols}{2}
\begin{enumerate}[ref={\fcProblemRef}]
\item $e^{7-4x}=7$.

\answer{$\frac{7-\ln 7 }{4}\approx 1.263522 $}
\item $\ln (2x-9)=2$.

\answer{$\frac{e^2+9}{2}\approx 8.194528 $}
\item $\ln (x^2-2)=3$.

\answer{$\pm \sqrt{e^3+2}\approx \pm 4.699525 $}
\item  \label{problem2^(x-3)=5} $2^{x-3}=5$.

\answer{$\log_2 5+3= \frac{\ln 5}{\ln 2}+3 \approx 5.321928 $}
\item \label{problemlnx+ln(x-1)=1} $\ln x+\ln (x-1)=1$.

\answer{$\frac{1}{2}\left(1+\sqrt{1+4e}\right)\approx 2.223$}
\item $e^{2x+1}=t$.

\answer{$\frac{\ln t-1}{2}$}
\item $\log_2(m x)=c$.

\answer{$\frac{2^c}{m}$}
\item \label{probleme-e^(-2x)=1} $e- e^{-2x}=1$.

\answer{$-\frac12\ln (e-1)\approx -0.271$}
\item $8(1+e^{-x})^{-1}=3$.

\answer{$-\ln \frac53 =\ln \frac35 \approx -0.510826 $}
\item $\ln (\ln x)=1$.

\answer{$e^e\approx 15.154$}
\item $e^{e^x}=10$.

\answer{$\ln (\ln 10)\approx 0.834$}
\item $\ln(2x+1)=3-\ln x$.

\answer{$\frac{-1+\sqrt{1+8e^3}}{4}\approx 2.928878 $}
\item $e^{2x}-4e^x+3=0$.

\answer{$x=\ln 3\approx 1.098612, ~~~, x=0$}


\item $e^{4x}+3e^{2x}-4=0$. 

\answer{$x=0$}
\item $e^{2x}-e^x-6=0$.

\answer{$x=\ln 3$}
\item 
\label{problemSolve4^(3x)-2^(3x+2)-5}

$4^{3x}-2^{3x+2}-5=0$. 

\answer{$x=\frac{\log_{2}5}{3}$}


\item \label{problemSolve32^x+2(1/2)^(x-1)-7=0}
$3\cdot 2^{x}+2 \left(\frac{1}{2}\right)^{x-1}-7=0$. 

\answer{$x=0 \text{ or } 2-\log_2  3= 2-\frac{\ln 3}{\ln 2}$}



\end{enumerate}
\end{multicols}


\item Differentiate.
\begin{multicols}{2}
\begin{enumerate}
\item $10^{x^3}$. \answer{$3(\ln 10) x^{2} (10)^{x^{3}}$}
\item $2^{\tan x}$. \answer{ $(\ln 2) 2^{\tan x}  \sec^2 x $  }
\item $x^x $. \answer{$x^x(\log{}(x) +1)$}
\item $x^{x^x}$. \answer{$(\ln(x))^{2}  x^{x^{x}+x}+x^{x^{x}+x-1}+(\ln x) x^{x^{x}+x}$}
\item $(\sin x)^{\cos x}$. \answer{$\frac{- \ln(\sin{}x)  (\sin{}x)^{\cos{}x+2} +(\sin{}x)^{\cos{}x} \cos^{2}{}x}{\sin{}x}$}
\item $(\ln x)^{\ln x}$. \answer{$\ln{}(\ln{}(x)) x^{-1} (\ln{}(x))^{\ln{}(x)}+x^{-1} (\ln{}(x))^{\ln{}(x)}$}
\end{enumerate}
\end{multicols}
\item Find the limit.
\begin{multicols}{2}
\begin{enumerate}
\item $\displaystyle \lim\limits_{x\to \infty} \left(1-\frac{2}{x} \right)^x$. \answer{$e^{-2}$}
\item $\displaystyle \lim\limits_{x\to 0} \left(1-x\right)^{\frac{1}{x}}$.
\answer{ $e^{-1}$}
\item $\displaystyle \lim\limits_{x\to \infty} \left(\frac{x}{x-5}\right)^{x}$.
\answer{$e^5$}
\item $\displaystyle \lim\limits_{x\to \infty} \left(\frac{x}{x-2}\right)^{3x+2}$.
\answer{$e^6$}
\end{enumerate}
\end{multicols}
\homeworkEnd
\end{comment}

\begin{comment}
\homeworkStart{on Lectures 9,10. \\ Will be quizzed Wednesday March 12.}{}

\item \tooeasy (Textbook, page 136, 1-44).
Compute the derivative.
\begin{multicols}{2}
\begin{enumerate}
\item $f(x)=2^{40}$.

\answer{$0$}
\item $f(x)=\pi^2$.

\answer{$0$}
\item $f(t)=2-\frac{2}{3}t$.

\answer{$-\frac{2}{3}$}
\item $F(x)=\frac{3}{4}x^8$.

\answer{$6 x^{7}$}
\item $f(x)=x^3-4x+6$.

\answer{$-4+3 x^{2} $}
\item $f(t)=\frac{1}{2}t^6-3t^4+t$.

\answer{$ 3 t^{5}-12 t^{3}+1$}
\item $g(x)=x^2(1-2x)$. 

\solution{
There are two approaches: 
1. Uncover the parenthesis, and then differentiate:

$
\left(x^2(1-2x)\right)'= \left(x^2-2x^3\right)'=2x-6x^2
$

2. Use first the product rule and then simplify:
$
\begin{array}{rcl}
\left(x^2(1-2x)\right)'&=& (x^2)'(1-2x)+x^2(1-2x)'\\
&=&2x(1-2x)+x^2(-2)\\
&=& 2x-4x^2-2x^2\\
&=& 2x-6x^2.
\end{array}
$

Of course, both approaches lead to the same answer.
}


\answer{$ 2 x-6 x^{2}$}
\item $h(x)=(x-2)(2x+3)$.

\answer{$ 4x-1$}
\item $g(t)=2t^{-3/4}$.

\answer{$-\frac{3}{2} t^{-\frac{7}{4}} $}
\item $B(y)=c y^{-6}$.

\answer{$-6 c y^{-7} $}
\item $A(s)=-\frac{12}{s^5}$.
\answer{$60 s^{-6}$}

\end{enumerate}
\end{multicols}

\item  (Textbook, page 136, 1-44). Compute the derivative.
\begin{multicols}{2}
\begin{enumerate}
\item $y=x^{\frac53}-x^{\frac23}$.

\answer{$ \frac53 x^{\frac23}-2/3 x^{-\frac13}$}
\item $S(p)=\sqrt{p}-p$.

\answer{$-1+\frac{1}{2} p^{-\frac{1}{2}} $}
\item $y=\sqrt{x}(x-1)$.

\answer{$ \frac{3}{2} x^{\frac{1}{2}}- \frac{1}{2}  x^{-\frac{1}{2}}$}
\item $R(a)=(3a+1)^2$.

\answer{$6+18 a $}
\item $S(R)=4\pi R^2$.

\answer{$8 \pi R$}
\item $y=\frac{ x^2+4x+3}{\sqrt{x}}$.

\answer{$ 2 x^{-\frac{1}{2}}+\frac{3}{2} x^{\frac{1}{2}}-\frac{3}{2} x^{-\frac{3}{2}}$}
\item $y=\frac{\sqrt{x}+x}{x^2}$.

\answer{$- x^{-2}-\frac{3}{2} x^{-\frac{5}{2}} $}
\item $H(x)=(x+x^{-1})^3$.

\answer{$3x^{2}+3-3x^{-2}-3x^{-4} $}
\item $g(u)=\sqrt 2 u +\sqrt{3u}$.

\answer{$ \sqrt{2}+\frac{\sqrt3}{2}  u^{-\frac{1}{2}}$}
\item $u=\sqrt[5]t+4\sqrt{t^5}$.

\answer{$10 t^{\frac{3}{2}}+\frac{1}{5} t^{-\frac{4}{5}} $}
\item $v=\left(\sqrt{x}+\frac{1}{\sqrt[3]{x}}\right)^2$.


\answer{$1+\frac{1}{3} x^{-\frac{5}{6}}-\frac{2}{3} x^{-\frac{5}{3}} $}
\item $f(x)=(1+2x^2)(x-x^2)$.

\answer{$1-2 x+6 x^{2}-8 x^{3}$}
\item $f(x)=\frac{x^4-5x^3+\sqrt{x}}{x^2}$.

\answer{$-5+2 x-\frac{3}{2} x^{-\frac{5}{2}} $}
\item $V(x)=(2x^3+3)(x^4-2x)$.

\answer{$-6-4 x^{3}+14 x^{6}$}
\item $L(x)=(1+x+x^2)(2-x^4)$.

\answer{$ 2+4 x-4 x^{3}-5 x^{4}-6 x^{5}$}
\item $F(y)=\left(\frac{1}{y^2}-\frac{3}{y^4} \right)(y+5y^3)$.

\answer{$5+9 y^{-4}+14 y^{-2} $}
\item $J(v)=(v^3-2v)(v^{-4}+v^{-2})$.

\answer{$1+6 v^{-4}+v^{-2}$}
\item $g(x)=\frac{1+2x}{3-4x}$.

\answer{$ 10 (3-4 x)^{-2}$}
\end{enumerate}
\end{multicols}
\item (Textbook, page 136, 1-44). Compute the derivative.
\begin{multicols}{2}
\begin{enumerate}

\item $f(x)=\frac{x-3}{x+3}$.

\answer{$6 (3+x)^{-2} $}
\item $y=\frac{x^3}{1-x^2}$.

\answer{$ \frac{3 x^{2}- x^{4}}{(1- x^{2})^{2}}$}
\item $y=\frac{x+1}{x^3+x-2}$.

\answer{$\frac{-3-3 x^{2}-2 x^{3}}{(-2+x+x^{3})^{2}} $}
\item $y=\frac{v^3-2v\sqrt{v}}{v}$.

\answer{$2 v- v^{-\frac{1}{2}}$}
\item $y=\frac{t}{(t-1)^2}$.

\answer{$-\frac{t +1}{(t-1)^3} $}
\item $y=\frac{t^2+2}{t^4-3t^2+1}$.

\answer{$\frac{14 t-8 t^{3}-2 t^{5}}{(1-3 t^{2}+t^{4})^{2}} $}
\item $g(t)=\frac{t-\sqrt{t}}{t^{1/3}}$.

\answer{$-\frac{1}{6} t^{-\frac{5}{6}}+\frac{2}{3} t^{-\frac{1}{3}} $}
\item $y=a x^2+b x + c$.

\answer{$ b+2 a x$}
\item $y=A+\frac{B}x +\frac{C}{x^2}$.

\answer{$\frac{- B-2 C x^{-1}}{x^{2}}$}
\item $f(t)=\frac{2t}{2+\sqrt{t}}$.

\answer{$\frac{4+t^{\frac{1}{2}}}{(2+t^{\frac{1}{2}})^{2}} $}
\item $y=\frac{c x}{1+c x}$.

\answer{$ c (1+c x)^{-2}$}
\item $y=\sqrt[3]{t}(t^2+t+t^{-1}) $.

\answer{$-\frac{2}{3} t^{-\frac{5}{3}}+\frac{4}{3} t^{\frac{1}{3}}+\frac{7}{3} t^{\frac{4}{3}} $}
\item $y=\frac{u^6-2u^3+5}{u^2}$.

\answer{$-2-10 u^{-3}+4 u^{3} $}
\item $f(x)=\frac{x}{x+\frac{c}{x}}$.

\answer{$ \frac{2 x c}{(c+x^{2})^{2}}$}
\item $f(x)=\frac{a x+b}{c x+ d}$.

\answer{$\frac{a d- b c}{(d+c x)^{2}}$}
\end{enumerate}
\end{multicols}


\homeworkEnd
\end{comment}

\begin{comment}
\homeworkStart{on Lecture 11. \\ Will be quizzed the Monday after spring break.}{}
\item 
Compute the derivative.
\begin{multicols}{2}
\begin{enumerate}
\item $\displaystyle f(x)= 2x^3 -3 \cos x$.

\answer{$ 6 x^2 +3 \sin x$}
\item $\displaystyle f(x)=\sqrt{x}\cos x$.

\answer{$ -x^{\frac{1}{2}}\sin x +\frac{1}{2}x^{-\frac{1}{2}} \cos x $}


\item $\displaystyle f(x)=\sin x +\frac{1}{3}\cot x$.

\answer{$\frac{-\frac{1}{3}+\cos x \sin^2x}{\sin^2x} = \cos x- \frac{1}{3} \csc^2 x $}
\item $\displaystyle y=2\sec x - \csc x$.

\answer{$ \frac{\cos^3 x+2 \sin^3x}{(\cos x \sin x)^{2}}$}
\item $\displaystyle y=\frac{1+\sin^2\theta}{\cos^3\theta}$.

\answer{$ y'=\displaystyle \frac{2 \sin{}\theta \cos^{2}{}\theta+3 \sin^{3}{}\theta+3 \sin{}\theta}{\cos^{4}{}\theta} $}
\item $\displaystyle g(t)=4 \sec t + \tan t-\csc t +3\cot t  $.

\answer{$4\sec t \tan t +\sec^2t +\csc t \cot t-3\csc^2 t $}

\item $\displaystyle y= c\cos t + t^2\sin t$.

\answer{$ - c \sin t+2 t \sin t+ t^{2}\cos t$}
\item $\displaystyle y=u(a\cos u + b \cot u)$.

\answer{$ \frac{- a u \sin^3 u+a \cos u \sin^2u- b u +b \cos u \sin u}{\sin^2u}$}
\item $\displaystyle y=\frac{x}{2-\tan x}$.

\answer{$ \frac{x - \cos x \sin x+2 \cos^2 x}{(2 \cos x- \sin x)^{2}}$}
\item $\displaystyle y=\sin \theta \cos \theta$.

\answer{$\cos (2\theta)= \cos^2\theta- \sin^2\theta$}
\item $\displaystyle f(\theta)=\frac{\sec \theta}{1+\sec \theta}$.

\answer{$\frac{\sin\theta}{(1+\cos\theta)^{2}} $}
\item $\displaystyle y=\frac{\cos x}{1-\sin x}$.

\answer{$\frac{1}{1- \sin x} $}
\item $\displaystyle y=\frac{t\sin t}{1+t}$.

\answer{$ \frac{\sin t+t \cos t+t^{2}\cos t }{(1+t)^{2}}$}
\item 
$\displaystyle y=\frac{1-\sec x}{\tan x}$.

\answer{$\frac{\cos x- 1}{\sin^2x} $}

\item $\displaystyle h(\theta)=\theta \csc \theta -\cot \theta$.

\answer{$\frac{1+\sin\theta- \theta \cos\theta}{\sin^2\theta}$}
\item $\displaystyle y=x^2\sin x\tan x$.

\answer{$\frac{2 x \cos{}x \sin^2{}x+2 x^{2} \sin{}x  \cos^2x+x^{2} \sin^3{}(x)}{\cos^2{}x} $}
\end{enumerate}
\end{multicols}

\item Differentiate.

\begin{multicols}{2}
\begin{enumerate}
\item $\tan x$.

\answer{$\sec^2 x$}
\item $\cot x$.

\answer{$-\csc^2 x$}
\item $\sec x$.

\answer{$\sec x \tan x= \frac{\sin x}{\cos^2 x}$}
\item $\csc x$.

\answer{$-\csc x \cot x= -\frac{\cos x }{\sin^2x} $}
\item $\sec x\tan x$.

\answer{$\sec x \tan^2 x+\sec^3 x$}
\item $\sec x+\tan x$.

\answer{$\sec x(\tan x +\sec x) $}
\item $\sec^2 x$.

\answer{$2\tan x\sec^2 x$}
\item $\csc^2 x$.

\answer{$ -2\cot x\csc^2 x$}

\item \label{problemd/dx((secx)e^x)}
$\displaystyle f(x)=(\sec x )e^{x}$.

\answer{$\sec x\tan x e^x + \sec x e^x=(\tan x +1)(\sec x) e^x$}
\item \label{problemd/dx((tanx)e^x)}
$\displaystyle f(x)=(\tan x )e^{x}$.

\answer{$\sec^2x e^x + \tan x e^x$}

\item $\displaystyle \frac{\sin x}{x}$.

\answer{$\frac{x \cos{}x- \sin{}x}{x^{2}}$}

\item $\displaystyle \frac{\sin x}{e^x}$.

\answer{$\frac{\cos x -\sin x}{e^x}$}
\item $\displaystyle x(\cos x) e^x$.

\answer{$  \begin{array}{l}e^x( x \cos{}x-x  \sin{}x+ \cos{}x) \end{array}$}

\item $\displaystyle \frac{e^x}{\tan x}$.

\answer{$  e^x\left(\cot x-\csc^2 x\right)$}
\item $\displaystyle \frac{e^x}{\sec x} +\sec x$.

\answer{$e^x(\cos x-\sin x)+ \sec x \tan x $}

\end{enumerate}

\end{multicols}
\homeworkEnd
\end{comment}

\begin{comment}
\homeworkStart{on Lecture 12. \\ Quiz date to be announced.}{}
\item % begin homework chain-rule1
Compute the derivative using the chain rule.
\begin{multicols}{2}
\begin{enumerate}[ref={\fcProblemRef}]
\item $\displaystyle f(x)=\sqrt{1+x^2}$

\answer{$x (x^{2}+1)^{-\frac{1}{2}}  $}
\item \label{problemd/dx(sqrt(3x^2-x+2))}
$\displaystyle f(x)=\sqrt{3 {{x}}^{2}-{{x}}+2}$.

\answer{$\frac{6x-1}{2\sqrt{3x^2-x+2}}$}


\item \label{problemDifferentialtexDivsqrt(1+2divx^2)}  $\displaystyle  f(x)=\frac{x }{\sqrt{1+\frac{2}{x^2}}}$.

\answer{$\frac{\pm x^2}{\sqrt{x^2+2}} $}

\item \label{problemd/dxsqrt(1-sqrt(x))}

$f(x)=\sqrt{1-\sqrt{x}}$.

\answer{$-\frac{1}{4} x^{-\frac{1}{2}} \left(- \sqrt{x}+1\right)^{-\frac{1}{2}} $}

\pointsii{3} \label{problemd/dx((cosx)^(1/2))} $y = (\cos x)^{\frac{1}{2}}$

\answer{$\frac{1}{2} x^{-\frac{1}{2}} \cos{}\left(\sqrt{x}\right)  = \frac{\cos \left(\sqrt{x}\right)}{2\sqrt{x}}$}

\item $\displaystyle f(x)=\sin^3 x$.

\answer{$ 3 \cos{}x \sin^{2}{}x$}
\pointsii{3} \label{problemd/dx((1+cosx)^2)}  $y = (1+\cos x)^2$.


\answer{$ -2 \cos{}x \sin{}x-2 \sin{}x =-\sin(2x)-2\sin x$}
\item   $\displaystyle f(x)=\frac{1}{\sin^3x}$.

\answer{$  -\frac{3 \cos{}x}{\sin^{4}{}x} $}
\item  $\displaystyle f(x)= \sqrt[3]{4+3\tan x}$.

\answer{$  (4+3\tan x)^{-\frac{2}{3}}\sec^2x $}
\item  $f(x)=(\cos x + 3\sin x)^4$.

\answer{$4(\cos x + 3\sin x)^3 (3\cos x-\sin x) $}

\pointsii{4} \label{problemd/dx(sin(sqrt(x)))}  $\displaystyle y = \sin \left( \sqrt{x}\right)$

\answer{$\frac{\cos\sqrt{x}}{2\sqrt{x}}$}
\item  $y = \cos\left( 4x\right)$

\answer{$-4 \sin{}\left(4 x\right)  $}

\item $\sec^2 (3x^2)$. 

\answer{$12 \frac{x\sin{}(3 x^{2}) }{\left(\cos{}\left(3 x^{2}\right)\right)^{3}}$}

\item $\csc^2 (3x^2)$. 

\answer{$-12 \frac{ x  \cos{}\left(3 x^{2}\right) }{\left(\sin{}\left(3 x^{2}\right)\right)^{3}}$}
\item $e^{2x}$.

\answer{$2e^{2x}$}

\item $e^{-x^2}$

\answer{$-2xe^{-x^2}$}

\item $e^{\sqrt{x}}$

\answer{$\frac{1}{2\sqrt{x}}e^{\sqrt{x}}$}

\item \label{problemd/dx(e^(-1/x))}

$\displaystyle f(x)=e^{-\frac{1}{x}}$.

\answer{$ \frac{e^{-\frac{1}{x}}}{x^2}$}

\item $5^{x}$.

\answer{$(\ln 5)5^x $}
\item $e^{2^x}$.

\answer{$e^{2^x}2^x (\ln 2) $}

\item $2^{3^x}$.

\answer{$2^{3^{x}} 3^{x} (\ln{}2)  (\ln{}3) $}
\item $3^{2^x}$.

\answer{$ 3^{2^{x}} 2^{x}(\ln{}2)(\ln{}3)$}
\pointsii{5} \label{problemd/dx(sqrt(sec(4x)))}  $y = \sqrt{\sec (4x)}$

\answer{ $  2\sqrt{\sec (4x)}\tan (4x) =  2(\sec (4x))^{\frac{3}{2}}\sin (4x). $}
\item \label{problemd/dx(x^2tan(5x))} $y = x^2\tan (5x)$

\answer{$2x\tan (5x) - 5x^2\sec^2 (5x)$}
\pointsii{5} \label{problemd/dx((1+sin(x^2))/(1+cos(x^2)))}  $\displaystyle y = \frac{1+\sin \left(x^2\right)}{1+\cos \left(x^2\right)}$.

\answer{ $ \frac{2x\left(1 + \cos \left(x^2\right) + \sin \left(x^2\right)\right)}{\left(1+\cos \left(x^2\right)\right)^2}$}
\end{enumerate}
\end{multicols}
% end homework chain-rule1

\item Differentiate. The answer key has not been proofread, use with caution.
\begin{multicols}{3}
\begin{enumerate}[ref={\fcProblemRef}]
\item $\displaystyle f(x)=\sin (-5x)$. 

\answer{$ -5 \cos(-5 x)=-5\cos (5x)$}
\item $\displaystyle f(x)=\cot (2x)$. 

\answer{$-2\csc^2 (2x)$}
\item $\displaystyle f(x)=e^{-3x}$. 

\answer{$-3 e^{-3 x} $}
\item $\displaystyle f(x)=e^{\frac{1}x}$. 

\answer{$ -\frac{e^{\frac{1}{x}}}{x^2}$}
\item $\displaystyle f(x)=e^{\sqrt{x}}$. 

\answer{$ \frac{1}{2\sqrt{x}}e^{\sqrt{x}} $}
\item $\displaystyle f(x)=\ln (1+x) $

\answer{$\frac{1}{1+x} $}
\item $\displaystyle f(x)=\ln(1+x^3) $

\answer{$\frac{3x^2}{1+x^3}$}
\item $\displaystyle f(x)=\frac{1}{2}\ln\left(\frac{1+x}{1-x}\right) $

\answer{$\frac{1}{1-x^{2}}$}
\end{enumerate}
\end{multicols}

\item Compute the derivative.
\begin{multicols}{2}
\begin{enumerate}
\item   $\displaystyle f(x)=\sqrt{1+x^2}$

\answer{$x (x^{2}+1)^{-\frac{1}{2}}  $}
\item \label{problemd/dx(cos(x))^(1/2)}  $\displaystyle f(x)=(\cos x)^{\frac{1}{ 2}}$

\answer{$-\frac{1}{2} \sin{}x (\cos{}x)^{-\frac{1}{2}}  $}
\item $\displaystyle f(x)=\sin^3 x$

\answer{$ 3 \cos{}x \sin^{2}{}x$}
\item \label{problemd/dx(1+cos(x))^2} $\displaystyle f(x)=(1+\cos x)^2$

\answer{$ -2 \cos{}x \sin{}x-2 \sin{}x =-\sin(2x)-2\sin x$}
\item   $\displaystyle f(x)=\frac{1}{\sin^3x}$

\answer{$  -\frac{3 \cos{}x}{\sin^{4}{}x} $}
\item  $\displaystyle f(x)= \sqrt[3]{4+3\tan x}$

\answer{$  (4+3\tan x)^{-\frac{2}{3}}\sec^2x $}
\item  $f(x)=(\cos x + 3\sin x)^4$

\answer{$4(\cos x + 3\sin x)^3 (3\cos x-\sin x) $}
\item \label{problemd/dx(sin(sqrt(x)))}  $f(x)=\sin\left(\sqrt{x}\right)$

\answer{$\frac{1}{2} x^{-\frac{1}{2}} \cos{}\left(\sqrt{x}\right)  = \frac{\cos \left(\sqrt{x}\right)}{2\sqrt{x}}$}
\item  $f(x)=\cos(4x)$

\answer{$-4 \sin{}\left(4 x\right)  $}
\item $\displaystyle f(x)= (x^4+3x^2-2)^5$.

\answer{$ \left(30 x +20 x^{3}\right) \left(-2+3 x^{2}+x^{4}\right)^{4}$}
\item $\displaystyle f(x)= (4x-x^2)^{100}$.

\answer{$(-200 x+400) \left(4 x- x^{2}\right)^{99}$}
\item $\displaystyle f(x)= \sqrt{1-2x}$.

\answer{$- (1-2 x)^{-\frac{1}{2}}$}
\item $\displaystyle f(x)= \frac{1}{(1+\sec x)^2}$.

\answer{$\frac{-2 \cos{}(x) \sin{}(x)}{(1+\cos{}(x))^{3}} =\frac{- \sin{}(2x)} {(1+\cos{}(x))^{3}} $}
\item $\displaystyle f(x)=\frac{1}{1+x^2} $.

\answer{$\frac{-2 x}{(1+x^{2})^{2}} $}
\item $\displaystyle f(x)= \sqrt[3]{1+\tan x}$.

\answer{$ 
\frac{1}{3}\left(1+\tan x \right)^{-\frac{2}{3}} \sec^2 x
$}
\item $\displaystyle f(x)=\cos (a^3+x^3) $.

\answer{$ -3 x^{2}\sin{}(a^{3}+x^{3}) $}
\item $\displaystyle f(x)= a^3+\cos^3 x$.

\answer{$ -3 \sin{}(x) (\cos{}(x))^{2}$}
\item $\displaystyle f(x)= x\sec (k x) $.

\answer{$\frac{\cos{}(k x)+k x \sin{}(k x) }{(\cos{}(k x))^{2}} $}
\item $\displaystyle f(\theta)= 3\cot (n\theta)$.

\answer{$ \frac{-3 n}{(\sin{}(n \theta))^{2}}$}
\item $\displaystyle f(x)= (2x - 3)^4 (x^2 + x + 1)^5$.

\answer{$ \left(-7-12 x+28 x^{2}\right)\left(-3+2 x\right)^{3} \left(1+x +x^{2} \right)^{4}$}
\item $\displaystyle f(x)= (x^2+1)^3(x^2+2)^6$.

\answer{$\left(24 x+18 x^{3}\right)\left(1+x^{2}\right)^{2} \left(2+x^{2}\right)^{5} $}
\item $\displaystyle f(t)= (t+1)^{\frac{2}{3}}(2t^2-1)^3$.

\answer{$ \left(\frac{40}{3} t^{2}+12 t-\frac{2}{3}\right)\left(2 t^{2}-1\right)^{2}\left(t+1\right)^{-\frac{1}{3}}$}
\item $\displaystyle f(t)= (3t-1)^4(2t+1)^{-3}$.

\answer{$(3 t-1)^{3}\frac{6 t+18}{(2 t+1)^{4}}$}
\item $\displaystyle f(x)=\left(\frac{x^2+1}{x^2-1} \right)^3 $.

\answer{$\frac{-12 x}{\left(x^{2}-1\right)^{2}} \left(\frac{x^{2}+1}{x^{2}-1}\right)^{2} $}
\item $\displaystyle f(s)= \sqrt{\frac{s^2+1}{s^2+4}}$.

\answer{$\frac{3 s}{\left(s^{2}+4\right)^{2}} \left(\frac{s^{2}+1}{s^{2}+4}\right)^{-\frac{1}{2}} $}

\end{enumerate}
\end{multicols}


\item Differentiate. 
\begin{multicols}{2}
\begin{enumerate}
\item $\displaystyle f(x)=\sin (\tan (2x)) $.

\answer{$2\sec^2(2x) \cos (\tan 2x) $}
\item $\displaystyle f(x)=\sec^2(m x) $.


\answer{$ \frac{2 m \sin{}(m x) }{(\cos{}(m x))^{3}} $}
\item $\displaystyle f(x)= \sec^2 x+\tan^2 x$.

\answer{$\frac{4 \sin{}x}{\cos^{3}{}x} $}
\item $\displaystyle f(x)=x\sin\left( \frac{1}{x}\right) $.

\answer{$- x^{-1}\cos{}(x^{-1}) +\sin{}(x^{-1})$}
\item $\displaystyle f(x)= \left(\frac{1-\cos (2x)}{1+\cos (2x)}\right)^4$.

\answer{$\frac{16 \sin{}(2 x)}{(\cos{}(2 x)+1)^{2}} \left(\frac{- \cos{}(2 x)+1}{\cos{}(2 x)+1}\right)^{3} $}
\item $\displaystyle f(x)=\sqrt{\frac{x}{x^2+4}} $.

\answer{$ \frac{-\frac{1}{2} x^{2}+2}{(x^{2}+4)^{2}} \left(\frac{x}{x^{2}+4}\right)^{-\frac{1}{2}}$}
\item $\displaystyle f(t)= \cot^2(\sin t)$.

\answer{$ \frac{-2 \cos{}t \cos{}(\sin{}t)}{\sin^{3}{}(\sin{}t)}$}
\item $\displaystyle f(x)= \left(a x+\sqrt{x^2+b^2}\right)^{-2}$.

\answer{$\frac{-2 x\left(x^{2}+b^{2} \right)^{-\frac{ 1 }{2}} -2 a}{\left(\left(x^{ 2}+b^{2}\right)^{ \frac{1}{2}} +a x\right)^{3}} $}
\item $\displaystyle f(x)= \left(x^2+(1-3x)^5 \right)^3$.

\answer{
\begin{tabular}{l}
$\left(-45 (-3 x+1)^{4} +6 x \right) \left((-3 x+1)^{5}+x^{2}\right)^{2}$
\\
Using computer algebra:
\\
$(-3645x^{4}+4860x^{3}-2430x^{2}+546x -45)\left((-3 x+1)^{5}+x^{2}\right)^{2}$ \\
Using computer algebra full expansion:
\\
$\begin{array}{l}
-215233605x^{14}+1004423490x^{13}-2176250895x^{12}\\
+2903793624x^{11} -2666357595x^{10}+1782098820x^{9}\\
-893713176x^{8} +341444160x^{7} -99805041x^{6} \\ +22199676x^{5}-3697470x^{4}  +447132x^{3}\\
-37125x^{2}+1896x -45
\end{array}
$
\end{tabular} 
}
\item $\displaystyle f(x)=\sin (\sin (\sin x))$.

\answer{$\cos{}x \cos{}(\sin{}x) \cos{}(\sin{}(\sin{}x)) $}
\item $\displaystyle f(x)= \sqrt{x+\sqrt{x}}$.

\answer{$ \left(\frac{1}{2} +\frac{1}{4} x^{-\frac{1}{2}}\right) \left(x^{\frac{1}{2}}+x\right)^{-\frac{1}{2}}$}

\item $\displaystyle f(x)= \sqrt{x+\sqrt{x+\sqrt{x}}}$.

\answer{$\frac{1}{2} \left(\left(x^{\frac{1}{2}}+x\right)^{\frac{1}{2}}+x\right)^{-\frac{1}{2}} \left(\frac{1}{2} \left(x^{\frac{1}{2}}+x\right)^{-\frac{1}{2}} \left(\frac{1}{2} x^{-\frac{1}{2}}+1\right)+1\right) $}
\item $\displaystyle f(x)=(2r \sin (r x)+n)^p $.

\answer{$ p r(2 r \sin{}(r x)+n)^{p-1} \cos{}(r x) $}
\item $\displaystyle f(x)=\cos^4(\sin^3 x) $.

\answer{$-12 \cos{}x \sin^{2}{}x \sin{}(\sin^{3}{}x) \cos^{3}{}(\sin^{3}{}x) $}
\item $\displaystyle f(x)=\cos \sqrt{\sin (\tan (\pi x))} $.

\answer{$ \frac{-\frac{1}{2} \pi \cos{}(\tan{}(\pi x))  \sin{}\left(\sqrt{\sin (\tan{}(\pi x) )} \right)}{\sqrt{\sin{}(\tan{}(\pi x))} \cos^{2}{}(\pi x) }$}
\item $\displaystyle f(x)=\left(x+(x+\sin^2 x)^3 \right)^4 $.

\answer{$4 ((\sin^{2}{}x+x)^{3}+x)^{3} (3 (\sin^{2}{}x+x)^{2} (2 \sin{}x \cos{}x+1)+1) $}
\end{enumerate}
\end{multicols}
\item Compute the second derivative.
\begin{multicols}{3}
\begin{enumerate}
\item $\displaystyle f(x)=\sin (-5x)$. 

\answer{$ 25 \sin{}(5 x)$}
\item $\displaystyle f(x)=e^{-3x}$. 

\answer{$9 e^{-3 x} $}
\item $\displaystyle f(x)=e^{\frac{1}x}$. 

\answer{$ 2 e^{x^{-1}} x^{-3}+e^{x^{-1}} x^{-4}$}
\item $\displaystyle f(x)=e^{\sqrt{x}}$. 

\answer{$ e^{x^{\frac{1}{2}}} x^{-\frac{3}{2}}+\frac{1}{4} e^{x^{\frac{1}{2}}} x^{-1}$}
\item $\displaystyle f(x)=\frac{e^{x}-e^{-x}}{e^x+e^{-x}} $

\answer{$\frac{-8 \left(- e^{- x}+e^{x}\right)}{\left(e^{- x}+e^{x}\right)^{3}} $}
\item $\displaystyle f(x)=\frac{1}2\ln \left(\frac{1+x}{1-x}\right) $

\answer{$\frac12\left(-\frac{1}{(x+1)^{2}}+\frac{1}{(- x+1)^{2}}\right)= \frac{ 2x}{\left(1-x^2\right)^2} $}
\end{enumerate}
\end{multicols}

\homeworkEnd
\end{comment}
\begin{comment}
\homeworkStart{on Lecture 14. \\ Will be quizzed Wednesday April 2}{}

\item Express $\frac{\diff y}{\diff x}$ as a function of $x$ and $y$ by implicit differentiation. The answer key has not been proofread, use with caution.
\begin{multicols}{2}
\begin{enumerate}
\item $x^3+y^3=1$.

\answer{$\frac{\diff y}{\diff x}=-\frac{x^2}{y^2}$}
\item $ 2\sqrt x+\sqrt y=3$.

\answer{$\frac{\diff y}{\diff x}=-2\sqrt{\frac{ y}{x}}$}
\item $ x^2+x y-y^2=4$.

\answer{$\frac{\diff y}{\diff x}=\frac{-2x-y}{x-2y}$}
\item $ 2x^3+x^2y-x y^3=2$.

\answer{$\frac{\diff y}{\diff x}=\frac{y^{3}-6 x^{2}-2 x y}{-3 x y^{2}+x^{2}}$}
\item $ x^4(x+y)=y^2(3x-y)$.

\answer{$\frac{\diff y}{\diff x}= \frac{ -5x^4 -4x^3y +3y^2}{x^4- 6xy - 3y^2}$}
\item $ y^5+x^2y^3=1+x^4y $.

\answer{$\frac{\diff y}{\diff x}=\frac{ 4x^3y-2xy^3}{5y^4+3x^2y^2- x^4 }$}
\item $ y\cos x=x^2+y^2 $.

\answer{$\frac{\diff y}{\diff x}= \frac{ y\sin x+2x}{\cos x-2y}$}
\item $ \cos (x y)=1+\sin y$.

\answer{$\frac{\diff y}{\diff x}= -\frac{y\sin (xy)}{\cos y+x\sin(xy) }$}
\item $ 4\cos x\sin y=1$.

\answer{$\frac{\diff y}{\diff x}=\tan x \tan y$}
\item $ y\sin \left(x^2\right)=x\sin \left(y^2\right)$.

\answer{$\frac{\diff y}{\diff x}=\frac{-2xy\cos\left(x^2\right)+\sin \left(y^2\right)}{ - 2 x y \cos\left( y^2 \right)+\sin \left(x^2 \right) }$}
\item $ \tan \left(\frac{x}{y}\right)=x+y$.

\answer{$\frac{\diff y}{\diff x}=\frac{-y^2+y\sec^2 \left(\frac{y}{x}\right) }{y^2 + x\sec^2\left(\frac{x}{y}\right) }$}
\item $ \sqrt{x+y}=1+x^2y^2$.

\answer{$\frac{\diff y}{\diff x}= \frac{-\frac{1}{2} (y+x)^{-\frac{1}{2}}+8 x y^{2}}{\frac{1}{2} (y+x)^{-\frac{1}{2}}-8 x^{2} y} $}
\item $ \sqrt{xy}=1+x^2 y$.

\answer{$\frac{\diff y}{\diff x}=\frac{-\frac{1}{2} x^{-\frac{1}{2}} y^{\frac{1}{2}} -2 x y}{\frac{1}{2} x^{\frac12} y^{-\frac{1}{2}} +x^{2}} $}
\item $ x\sin y+y\sin x=1$.

\answer{$\frac{\diff y}{\diff x}=\frac{- y \cos{}x- \sin{}y}{x \cos{}y+\sin{}x}$}
\item $ y\cos x=1+\sin (x y)$.

\answer{$\frac{\diff y}{\diff x}=\frac{\cos{}(x y) y+y \sin{}x}{- \cos{}(x y) x+\cos{}x}$}
\item $ \tan (x-y)=\frac{y}{1+x^2}$.

\answer{$\frac{\diff y}{\diff x}=\frac{- (\sec{}(- y+x))^{2} x^{4}-2 (\sec{}(- y+x))^{2} x^{2}- (\sec{}(- y+x))^{2}-2 y x}{- (\sec{}(- y+x))^{2} x^{4}-2 (\sec{}(- y+x))^{2} x^{2}- (\sec{}(- y+x))^{2}- x^{2}-1}$}
\end{enumerate}
\end{multicols}
\item Verify that the coordinates of the given point satisfy the given equation. Use implicit differentiation to find an equation of the tangent line to the curve at the given point. 
\begin{multicols}{2}
\begin{enumerate}[ref={\fcProblemRef}]
\item 
\label{problemImplicitTangentysin(2x)=xcos(2y)point(pi/2,pi/4)} $y\sin (2x)=x\cos (2y) $, $\left(\frac{\pi}{2}, \frac{\pi}{4}\right)$. 

\psset{xunit=0.3cm, yunit=0.3cm}
\begin{pspicture}(-5.8,-5.8)(5.8,5.8)
\fcAxesStandard{-5.5}{-5.5}{5.5}{5.5}
\fcLabels{5.5}{5.5}
\fcXTickWithLabel{1}{$1$}
\fcImplicitIId[linestyle=solid, linecolor=red, linewidth-=0.05, showGridImplicitIId=false]{-5}{-5}{1000}{1000}{0.01}{0.01}{2 x mul 180 mul 3.141592654 div sin y mul 2 y mul 180 mul 3.141592654 div cos x mul sub} 

\fcFullDot[linecolor=blue]{3.141592654 2 div}{3.141592654 4 div}
\end{pspicture}

\answer{$y=\frac{1}{2}x$}

\item $ \sin (x+y)=2x-2y$, $(\pi,\pi)$ . 

\answer{$\frac{1}{3} x+\frac{2}{3} \pi $}
\item 
$x^2+x y+y^2=3 $, $(1,-2)$ (ellipse). 

\psset{xunit=0.5cm, yunit=0.5cm}
\begin{pspicture}(-3.6,-3.8)(3.6,3.6)
\tiny
\fcAxesStandard{-3.5}{-3.5}{3.5}{3.5}
\fcLabels{3.5}{3.5}
\fcXTickWithLabel{1}{$1$}
\fcImplicitIId[linestyle=solid, linecolor=red, linewidth-=0.05, showGridImplicitIId=false]{-2}{-2}{400}{400}{0.01}{0.01}{x x mul x y mul add y y mul add 3 sub}
%\psline[linecolor=blue](-3.5,-2)(3.5,-2)
\fcFullDot[linecolor=blue]{1}{-2}
\end{pspicture}


\answer{$y=-2 $}

\item  $x^2+2x y-y^2+x=2 $, $(1,2)$ (hyperbola). 

\psset{xunit=0.4cm, yunit=0.4cm}
\begin{pspicture}(-5.6,-5.6)(5.6,5.6)
\tiny
\fcAxesStandard{-5.5}{-5.5}{5.5}{5.5}
\fcLabels{5.5}{5.5}
\fcXTickWithLabel{1}{$1$}
\fcImplicitIId[linestyle=solid, linecolor=red, linewidth-=0.05, showGridImplicitIId=false]{-5}{-5}{500}{500}{0.02}{0.02}{x x mul x y mul 2 mul add y y mul sub x add 2 sub}
%\psline[linecolor=blue](-3.5,-2)(3.5,-2)
\fcFullDot[linecolor=blue]{1}{2}
\end{pspicture}

\answer{$y= \frac{7}{2} x-\frac{3}{2}$}


\item  \label{problemImplicitTangenty^3+x^3+4xy=3/4}
$\displaystyle y^{3}+x^{3}+4 x y=\frac{3}{4}$, $\left(-\frac{1}{2},-\frac{1}{2}\right)$
\psset{xunit=0.7cm, yunit=0.7cm}
\begin{pspicture}(-2.4,-2.4)(2.4,2.4)
\fcAxesStandard{-2}{-2}{2}{2}
\fcImplicitIId[linecolor=red, linestyle=dashed, dashes={[1 1] 0}, showGridImplicitIId=false, useMidpointImplicitPlots=false]{-3}{-3}{120}{120}{0.05}{0.05}{y y y mul mul x x x mul mul add 4 x y mul mul add 0.75 sub} 
\fcFullDot{-0.5}{-0.5}
\end{pspicture}


\answer{$y=-x-1$}
\item  \label{problemImplicitTangenty^3+x^3+4xy=-4at(1,-1)}
$\displaystyle y^{3}+x^{3}+4 x y=-4$, $(1,-1)$

\psset{xunit=0.7cm, yunit=0.7cm}
\begin{pspicture}(-3.1,-3.1)(3.1,3.1)
\fcAxesStandard{-2}{-2}{2}{2}
\fcImplicitIId[linecolor=red, linestyle=dashed, dashes={[1 1] 0}, showGridImplicitIId=false, useMidpointImplicitPlots=false]{-3}{-3}{300}{300}{0.02}{0.02}{y y y mul mul x x x mul mul add 4 x y mul mul add 4 add} 
\fcFullDot{1}{-1}
\end{pspicture}

\answer{$ $}

\item $x^2+y^2=(2x^2+2y^2-x)^2 $, $(0,\frac{1}{2})$. 

\answer{$y= x+\frac{1}{2}$}
\item $x^{\frac{2}{3}}+y^{\frac{2}{3}}=4$, $(-3\sqrt{3},1)$. 

\answer{$y=\frac{1}{\sqrt{3}}x+4 $}
\item $2(x^2+y^2)^2 =25(x^2-y^2)$, $(3,1)$. 

\answer{$y= -\frac{9}{13} x+\frac{40}{13}$}
\item $y^2(y^2-4)=x^2(x^2-5) $, $(0,-2)$. 

\answer{$y=-2 $}
\item $x^{\frac{4}{3}}+y^{\frac{4}{3}}=10$ at $(-3\sqrt{3}, 1)$. 

\answer{$y=\sqrt{3}x+10$ }
\item $x^2y^3+x^3-y^2=1$ at $(1,1)$. 

\answer{$y=-5 x+6$}
\end{enumerate}
\end{multicols}

\homeworkEnd

\end{comment}
\begin{comment}
\homeworkStart{on Lecture 14, related rates \\Quiz time to be announced}{}

\item \begin{problem} (page 180)
If $V$ is the volume of a cube with edge length $x$ and the cube expands as time passes, find $\frac{dV}{dt}$ in terms of $\frac{dx}{dt}$.
\end{problem}
\begin{problem} (page 180)
Each side of a square is increasing at a rate of 6cm/s. At what rate is the area of the square increasing when the area of the square is 16 $cm^2$?
\end{problem}
\begin{problem}(page 180)
The radius of a ball is increasing at a rate of 4 mm/s. How fast is the volume increasing when the diameter is 80mm?
\end{problem}
\begin{problem}(page 181)
A street light is mounted at the top of a 4.5m tall pole. A man 180 cm tall walks away from the pole at a speed of 5km/h along a straight path. How fast is the tip of his shadow moving when he is 12m from the pole?
\end{problem}
\begin{problem}(page 181)
A boat is pulled into a dock by a rope attached to the bow of the boat and passing through a pulley on the dock that is 1m higher than the bow of the boat. If the rope is pulled in at a rate of 1m/s, how fast is the boat approaching the dock when it is 8m from the dock?
\end{problem}
\begin{problem}(page 182)
A Ferris wheel with a radius of 10m is rotating at a rate of one revolution every 2 minutes. How fast is a riding rising when his seat is 16 m above ground level?
\end{problem}
\begin{problem}(page 183)
The minute hand on a watch is 8mm long and the hour hand is 4mm long. How fast is the distance between the tips of the hands changing at one' clock?
\end{problem}
\homeworkEnd

\end{comment}

\begin{comment}
\homeworkStart{on Lecture 15. \\ Problem types will appear on the Test but will not be quizzed}{}
\item Find the maximum and minimum values of $f$ on the given interval and the values of $x$ for which they are attained.
\begin{multicols}{2}
\begin{enumerate}[ref={\fcProblemRef}]
\item $\displaystyle f(x)=9+3x-x^2$, $x\in [0,4]$.

\answer{$f_{max}=f\left(\frac{3}{2}\right)=\frac{45}{4} $, $f_{min}=f\left(4\right)=5$}
\item $\displaystyle f(x)=5+4x-2x^3$, $x\in[-1,1] $.

\answer{$f_{max}=f\left(\frac{\sqrt{6}}{3} \right)= \frac{8}{9} \sqrt{6} +5  $, $f_{min} =f\left(-\frac{ \sqrt{6} }{3} \right)=5 - \frac{8 }{9}\sqrt{6} $}

\item $\displaystyle f(x)=2x^3-x^2-20x+1$, $x\in [-4,3]$.

\answer{$f_{max}=f\left( -\frac{5}{3}\right)=\frac{602}{27}$, $f_{min}=f\left(-4\right)=-63$}

\item $\displaystyle f(x)=3x^4-4x^3-12x^2+1$, $x\in [-2, 3]$.

\answer{$f_{max}=f\left(-2\right)=33$, $f_{min}=f\left(2\right)=-31 $}

\item \label{problemmaxminx^3-x^2-x+1over[-1,1]}
$f(x)=x^3-x^2-x+1,$  $x\in [-1,1]$.

\psset{xunit=2cm, yunit=2cm}
\begin{pspicture}(-1.5,-0.5)(1.5,1.5)
\tiny
\fcAxesStandard{-1.5}{-0.5}{1.5}{1.5}
\psplot[linecolor=red]{-1}{1}{x x x mul mul x x -1 mul mul -1 x mul 1 add add add}
\fcXTickWithLabel{1}{$1$}
\fcYTickWithLabel{1}{$1$}
\end{pspicture}
\item \label{problemmaxminx^3-x+1over[-2,1]}
$f(x)=x^3-x+1$,  $x\in[-2,1]$.

\answer{$f_{max}=f\left(-\frac{\sqrt{3}}{3}\right)= \frac{2}{9}\sqrt{3}+1, \quad f_{min}=f\left(-2\right)= -5 $}
\item $\displaystyle f(x)=(x^2-1)^3$, $x\in [-1, 2]$.

\answer{$f_{max}=f\left(2\right)=27$, $f_{min}=f\left(0\right)=-1$}
\item $\displaystyle f(x)=x+\frac{1}{x}$, $x\in [0.2,4 ]$.

\answer{$f_{max}=f\left(0.2\right)=\frac{26}{5}=5.2$, $f_{min}=f\left(1\right)=2$}
\item $\displaystyle f(x)=\frac{x}{x^2-x+1}$, $x\in [0,3 ]$.

\answer{$f_{max}=f\left(1\right)=1$, $f_{min}=f\left(-1\right)=-\frac{1}{3}$}
\item $\displaystyle f(t)=t\sqrt{4-t^2}$, $x\in [-1,2 ]$.

\answer{$f_{max}=f\left(\sqrt{2}\right)=2$, $f_{min}=f\left(-\sqrt{2}\right)=-2$}
\item $\displaystyle f(t)=\sqrt[3]{t}(8-t) $, $x\in [0,8 ]$.

\answer{$f_{max}=f\left(2\right)=6\sqrt[3]{2}$, $f_{min}=f\left(0\right)=f(8)=0$}
\item $\displaystyle f(t)=2\cos t+\sin (2t)$, $x\in [0,\frac{\pi}{2} ]$.

\answer{$f_{max}=f\left(\frac{\pi}{6}\right)=\frac{3}{2}\sqrt{3}$, $f_{min}=f\left(\frac{\pi}{2}\right)=0$}
\item $\displaystyle f(t)=t+\cot \left(\frac{t}{2}\right) $, $x\in [\frac{\pi}{4},\frac{7\pi}{4} ]$.

\answer{$f_{max}=f\left(\frac{3\pi}{2}\right)=\frac{3\pi}{2}-1$, $f_{min}=f\left(\frac{\pi}{2}\right)=\frac{\pi}{2}+1$}

\item $\displaystyle f(t)=t+\cot \left(\frac{t}{2}\right) $, $x\in [\frac{\pi}{4},\frac{7\pi}{4} ]$.
\item $\displaystyle f(x)=x e^{3 x}$, $x\in \left[-3, \frac{1}{6}\right]$.

\answer{$f_{max}=f\left( \frac{1}{6}\right)=\frac{e^{\frac{1}{2}}}{6}\approx 0.274787 $, $f_{min}=f\left( -\frac{1}{3}\right)= -\frac{1}{3e}\approx -0.122626 $}
\item $\displaystyle f(x)=\left(x-2\right) \left(x+1\right) e^{x} $, $x\in \left[-5,2\right]$.

\answer{$\begin{array}{l}
f_{max}=f\left(-\frac{\sqrt{13}}{2}-\frac{1}{2}
\right)= \left(\sqrt{13}+2\right) e^{\left(-\frac{\sqrt{13}}{2}-\frac{1}{2}\right)}\approx 0.560448\\
f_{min}=f\left(\frac{\sqrt{13}}{2}-\frac{1}{2} \right)= \left(-\sqrt{13}+2\right) e^{\left(\frac{\sqrt{13}}{2}-\frac{1}{2}\right)} \approx -5.907619
\end{array}
$}
\item $\displaystyle f(x)=$, $x\in \left[-3,3\right]$.

\answer{$\begin{array}{rcl}
f_{max}&=&f\left( \frac{\sqrt{3}}{2}-\frac{1}{2}
\right) =\left(\frac{\sqrt{3}}{2}+\frac{1}{2}\right) e^{\frac{\sqrt{3}}{2}-1}\approx 1.194743 \\
f_{min}&=&f\left(-\frac{\sqrt{3}}{2}-\frac{1}{2} \right)=\left(-\frac{\sqrt{3}}{2}+\frac{1}{2}\right) e^{-\frac{\sqrt{3}}{2}-1}\approx -0.056638 \end{array}$}
\item \label{problemmaxminxe^(2x)over[-2,1/2]}
$f(x)=x e^{2x}$, $x\in\left[ -2,\frac{1}{2}\right]$.

\answer{$ f_{max}=f\left(\frac{1}{2}\right)= \frac{e}{2}$, $f_{min}=f\left(-\frac{1}{2}\right)=-\frac{1}{2e}$}
\end{enumerate}
\end{multicols}

\item 

\begin{problem}(page 257)
Find the dimensions of a rectangle with area 1000 $m^2$ whose perimeter is as small as possible.
\end{problem}
\begin{problem}(page 257)
Find the dimensions of a rectangle with area 1000 $m^2$ whose perimeter is as small as possible.
\end{problem}
\begin{problem}(pages 256-259)
A box with an open top is to be constructed from a square piece of cardboard, 1m wide, by cutting out a square from each of the four corners and bending up the sides. Find the largest volume that such a box can have.
\end{problem}
\begin{problem}(pages 256-259)
A right circular cylinder is inscribed in a sphere of radius $r$. Find the largest possible volume of such a cylinder.
\end{problem}
\begin{problem}(pages 256-259)
A cone-shaped drinking cup is made from a circular piece of paper of radius $r$ by cutting out a sector and joining the edges $OA$ and $OB$. Find the maximum capacity of such a cup.
\begin{pspicture}(0,0)(1,1)
\pswedge*[linecolor=cyan](0,0){1}{120}{60}
\pswedge[linecolor=blue](0,0){1}{120}{60}
\rput[t] (0,-0.2){$O$}
\rput[b] (0.5,1){$B$}
\rput[b] (0.15,0.4){$r$}
\rput[b] (-0.5,1){$A$}
\end{pspicture}
\end{problem}
\homeworkEnd
\end{comment}
%\begin{comment}
\homeworkStart{on Lecture 17. \\ will be quizzed on Monday April 21}{}
\item 
Find the
\begin{multicols}{2}
\begin{itemize}
\item the implied domain of $f$,
\item $x$ and $y$ intercepts of $f$,
\item horizontal and vertical asymptotes,
\item intervals of increase and decrease,
\item local and global minima, maxima,
\item intervals of concavity,
\item points of inflection.
\end{itemize}
\end{multicols}
Label all relevant points on the graph. Show all of your computations.
\begin{enumerate}[ref={\fcProblemRef}]
\item $\displaystyle f(x)=\frac{x+\frac 1 2}{x^{2}+x+1}$

\psset{xunit=1cm, yunit=1cm}
\begin{pspicture}(-5, -1)(5,2)
\psframe*[linecolor=white](-5,-1)(5,2)
\tiny
\psaxes[ticks=none, labels=none]{<->}(0,0)(-5,-0.5)(5,1.5)
\fcLabels{5}{1.5}
%Function formula: \frac{x+1/2}{x^{2}+x+1}
\psplot[linecolor=\fcColorGraph, plotpoints=1000]{-5}{5}{0.5 x add 1 x add x 2 exp add div }
\end{pspicture}

\answer{
\begin{tabular}{l}
$y$-intercept: $\frac12$. $x$-intercept: $-\frac12$\\
Horizontal asymptote: $y=0$, vertical: none \\
local and global min at $x=\frac{ -1-\sqrt{3}}{2}$, local and global max at $x=\frac{ -1+\sqrt{3}}{2}$\\
Intervals of decrease: $ \left(-\infty, \frac{-1 -\sqrt{3} }{2}\right)\cup \left(\frac{-1 +\sqrt{3} }{2}, \infty\right) $, intervals of decrease $\left( \frac{ -1-\sqrt{3}}{2}, \frac{-1+ \sqrt{3}}{2}\right)$ \\
Concave down on $(-\infty, -2)\cup \left(-\frac12, 1\right)$, concave up on $\left(-2, -\frac12\right)\cup (1,\infty)$\\
Inflection points at: $x=-2$, $x= -\frac12$, $x=1 $ \\
\end{tabular}
}

\item \label{problemSketchCurve(2x^2-5x+9/2)/(x^2-3 x+3)} $\displaystyle f(x)=\frac{2 x^{2}-5 x+\frac{9}{2}}{x^{2}-3 x+3}$. \textbf{For this problem, indicate only the $x$-coordinates of the local maxima/minima and inflection points; you do not need to compute the $y$-coordinates of those points.} 

Computation shows that 
$\displaystyle
f'(x)=\frac{- x^{2}+3 x-\frac{3}{2} }{ \left(x^2- 3 x+3\right)^2}
$
and that 
$\displaystyle f''(x)=\frac{(2x-3)x(x-3)}{\left(x^2- 3 x+3\right)^3}$; you may use those computations without further justification. 
 
\psset{xunit=0.6cm, yunit=0.6cm}
\begin{pspicture}(-5, -1)(5,4)
\psframe*[linecolor=white](-5,-1)(5,4)
\tiny
\psaxes[ticks=none, labels=none]{<->}(0,0) (-5,-0.5) (5, 3.5)
\fcLabels{5}{3.5}
%Function formula: \frac{2 x^{2}-5 x+9/2}{x^{2}-3 x+3}
\psplot[linecolor=\fcColorGraph, plotpoints=1000]{-5}{5 } {4.5 x -5 mul add x 2 exp 2 mul add 3 x -3 mul add x 2 exp add div }
\end{pspicture}

\answer{
\begin{tabular}{l}
$y$-intercept: $\frac32$\\
horizontal asymptote: $y=2$, vertical: none\\
increasing on
$\left(\frac{3-\sqrt{3}}2, \frac{3+ \sqrt{3}}2 \right) $, decreasing on $\left(-\infty, \frac{3-\sqrt{3}}2\right)\cup \left(\frac{3+\sqrt{3}}2, \infty\right) $\\
local and global min at $x=\frac{3-\sqrt{3}}2$, local and global max at $x=\frac{3+\sqrt{3}}2$\\
concave up on $\left(0, \frac32\right)\cup \left(3, \infty \right)$, concave down $\left(-\infty, 0\right)\cup \left(\frac32, 3\right)$\\
inflection points at $x=0,x=\frac32, x=3$
\end{tabular}
}


\item $\displaystyle f(x)=\frac{2 \sqrt{- x^{2}+1}+ 1} {\sqrt{- x^{2}+1}+1}$,  $\displaystyle f(x)=\frac{1}{\sqrt{- x^{2} +1}+1}$

\psset{xunit=1cm, yunit=1cm}
\begin{pspicture}(-1, -1)(1.2,3)
\psframe*[linecolor=white](-1,-5)(1,5)
\tiny
\psaxes[ticks=none, labels=none]{<->}(0,0)(-1,-0.5)(1,2.5)
\fcLabels{1}{2.5}
%Function formula: \frac{2 (- x^{2}+1)^{1/2}+1}{(- x^{2}+1)^{1/2}+1}
\rput(1,3){}
\psplot[linecolor=brown, plotpoints=1000]{-1}{1}{1 1 x 2 exp -1 mul add 0.5 exp 2 mul add 1 1 x 2 exp -1 mul add 0.5 exp add div }
%Function formula: \frac{1}{(- x^{2}+1)^{1/2}+1}
\rput(1,3){}
\psplot[linecolor=\fcColorGraph, plotpoints=1000]{-1}{1}{1 1 1 x 2 exp -1 mul add 0.5 exp add div }
\end{pspicture}

The two functions are plotted simultaneously in the $x,y$-plane. Indicate which part of the graph is the graph of which function.

\answer{
\begin{tabular}{l}
For $f(x)=\frac{2 \sqrt{- x^{2}+1}+1}{ \sqrt{- x^{2}+1}+1}$: \\
$y$-intercept: $x=\frac{3}2$, no $x$ intercept\\
no asymptotes\\
increasing on $[-1, 0]$, decreasing on $[0, 1]$ \\
global and local max at $x=0$, global and local min at $x=\pm 1$.\\
concave down on $[-1,1]$\\
no inflection points
\end{tabular}
}
\answer{
\begin{tabular}{l}
For $f(x)=\frac{1}{\sqrt{- x^{2}+1}+1}$: \\
$y$-intercept: $x=\frac{1}2$, no $x$ intercept\\
no asymptotes\\
decreasing on $[-1, 0]$, increasing on $[0, 1]$ \\
global and local min at $x=0$, global and local max at $x=\pm 1$.\\
concave up on $[-1,1]$\\
no inflection points
\end{tabular}
}
\item $\displaystyle f(x)=\frac{e^x+e^{-x}}{e^x-e^{-x}}$

\psset{xunit=0.5cm, yunit=0.5cm}
\begin{pspicture}(-4, -5)(4,5)
\psframe*[linecolor=white](-4,-5)(4,5)
\tiny
\psaxes[ticks=none, labels=none]{<->}(0,0)(-4,-4.5)(4,4.5)
\fcLabels{4}{5}
%Function formula: \frac{e^{- x}+e^{x}}{- e^{- x}+e^{x}}
\psplot[linecolor=\fcColorGraph, plotpoints=1000]{0.2}{4}{2.718281828 x exp 2.718281828 x -1 mul exp add 2.718281828 x exp 2.718281828 x -1 mul exp -1 mul add div }
%Function formula: \frac{e^{- x}+e^{x}}{- e^{- x}+e^{x}}
\psplot[linecolor=\fcColorGraph, plotpoints=1000]{-4}{-0.2}{2.718281828 x exp 2.718281828 x -1 mul exp add 2.718281828 x exp 2.718281828 x -1 mul exp -1 mul add div }
\end{pspicture}
\item $\displaystyle f(x)=\frac{- e^{- x}+e^{x}}{e^{- x}+e^{x}}$

\psset{xunit=1cm, yunit=1cm}
\begin{pspicture}(-4, -1.5)(4,1.5)
\psframe*[linecolor=white](-4,-1.5)(4,1.5)
\tiny
\psaxes[ticks=none, labels=none]{<->}(0,0)(-4,-1.1)(4,1.1)
\fcLabels{4}{1.1}
%Function formula: \frac{- e^{- x}+e^{x}}{e^{- x}+e^{x}}
\psplot[linecolor=\fcColorGraph, plotpoints=1000]{-4}{4}{2.718281828 x exp 2.718281828 x -1 mul exp -1 mul add 2.718281828 x exp 2.718281828 x -1 mul exp add div }
\end{pspicture}
\item $\displaystyle f(x)=\ln{}\left(\frac{{{x}}+1}{- {{x}}+1}\right)$

\psset{xunit=1cm, yunit=1cm}
\begin{pspicture}(-1, -4.2)(1,4.2)
\psframe*[linecolor=white](-1,-5)(1,5)
\tiny
\psaxes[ticks=none, labels=none]{<->}(0,0)(-1.3,-4)(1.3,4)
\fcLabels{1.3}{4}
%Function formula: \log{}(\frac{x+1}{- x+1})
\psplot[linecolor=\fcColorGraph, plotpoints=1000]{-0.94}{0.94}{1 x add 1 x -1 mul add div ln }

\end{pspicture}

\item $\displaystyle f(x)=\frac{x^{2}+3 x+1}{x^{2}+2 x}$. \textbf{For this problem, indicate only the $x$-coordinates of the local maxima/minima and inflection points; you do not need to compute the $y$-coordinates of those points.} 

Computation shows that 
$\displaystyle
f'(x)= \frac{- x^{2}-2 x-2}{\left(x^{2}+2 x\right)^{2}} 
$
and that 
$\displaystyle f''(x)= \frac{2 x^{3}+6 x^{2}+12 x+8}{\left(x^{2}+2 x\right)^{3}} =\frac{\left(x+1\right) \left(2 x^{2}+4 x+8\right)}{\left(x^{2}+2 x\right)^{3}} $; you may use those computations without further justification. 

\psset{xunit=0.7cm, yunit=0.7cm}
\begin{pspicture}(-5, -4.7)(5,5)
\psframe*[linecolor=white](-5,-4.7)(5,5)
\tiny
\psaxes[ticks=none, labels=none]{<->}(0,0)(-5,-4.5)(5,4.5)
\fcLabels{5}{5}
%Function formula: \frac{x^{2}+3 x+1}{x^{2}+2 x}
\psplot[linecolor=\fcColorGraph, plotpoints=1000]{0.1}{5}{1 x 3 mul add x 2 exp add x 2 mul x 2 exp add div }
%Function formula: \frac{x^{2}+3 x+1}{x^{2}+2 x}
\psplot[linecolor=\fcColorGraph, plotpoints=1000]{-1.9}{-0.1}{1 x 3 mul add x 2 exp add x 2 mul x 2 exp add div }
%Function formula: \frac{x^{2}+3 x+1}{x^{2}+2 x}
\psplot[linecolor=\fcColorGraph, plotpoints=1000]{-5}{-2.1}{1 x 3 mul add x 2 exp add x 2 mul x 2 exp add div }
\end{pspicture}

\answer{
\begin{tabular}{l}
$y$-intercept: none, $x$-intercepts: $\frac{-3\mp\sqrt{5}}2$ \\
horizontal asymptote: $y=1$, vertical: $x=-2$ and $x=0$\\
always decreasing\\
no local/global minima/maxima\\
concave down on $\left(-\infty,-2\right)\cup \left(-1,0 \right)$, concave up on $\left(-2, -1\right)\cup \left(0, \infty\right)$\\
inflection point at $x=-1$
\end{tabular}
}


\item \label{problemSketch(x+1)/(x^2+2x+4)} $\displaystyle f(x)=\frac{x+1}{x^2+2x+4}$
\psset{xunit=1.4cm, yunit=1.4cm}
\begin{pspicture}(-5, -1.2)(3.4,1.2)
\psframe*[linecolor=white](-4.5,-1)(4.5,1)
\tiny
\fcAxesStandard{-5}{-1}{3}{1} %Function formula: \frac{x+1}{x^{2}+2 x+4}
\psplot[linecolor=\fcColorGraph, plotpoints=1000]{-5}{3}{1 x add 4 x 2 mul add x 2 exp add div }
\end{pspicture}

\answer{
\begin{tabular}{l}
$y$-intercept: $\frac14$, $x$-intercept: $-1$\\
horizontal asymptote: $y=0$, vertical: none\\
increasing on
$\left(-1-\sqrt{3}, -1+\sqrt{3}  \right) $, decreasing on $\left(-\infty, -1-\sqrt{3}\right)\cup \left(-1+\sqrt{3}, \infty\right) $\\
local and global min at $x=-1-\sqrt{3}$, local and global max at $x=-1+\sqrt{3}$\\
concave up on $\left(-4, -1\right)\cup \left(2, \infty \right)$, concave down $\left(-\infty, -4\right)\cup \left(-1, 2\right)$\\
inflection points at $x=-4,x=-1, x=2$
\end{tabular}
}
\item $\displaystyle f(x)= \frac{3 x^{3}-30 x^{2}+97 x-99}{x^{2}-6 x+8} $. \textbf{For this problem, do not find the $x$-intercepts of the function. Indicate only the $x$-coordinates of the local maxima/minima and inflection points; you do not need to compute the $y$-coordinates of those points.} 

Computation shows that 
$\displaystyle
f'(x)=\frac{3 x^{4}-36 x^{3}+155 x^{2}-282 x+182}{\left(x^{2}-6 x+8\right)^{2}}=\frac{\left(x^{2}-6 x+7\right) \left(3 x^{2}-18 x+26\right)}{\left(x^{2}-6 x+8\right)^{2}}
$
and that 
$\displaystyle f''(x)=\frac{2 x^{3}-18 x^{2}+60 x-72}{\left(x^{2}-6 x+8\right)^{3}}=\frac{\left(x-3\right) \left(2 x^{2}-12 x+24\right)}{\left(x^{2}-6 x+8\right)^{3}}$; you may use those computations without further justification. 

\psset{xunit=0.15cm, yunit=0.15cm}
\begin{pspicture}(-11,-21)(12,11)
\fcAxesStandard{-10}{-20}{10}{10}
\newcommand{\theFun}{3 x x x mul mul mul -30 x x mul mul 97 x mul -99 add add add x x mul -6 x mul 8 add add div}
\psplot[plotpoints=500, linecolor=\fcColorGraph]{-7}{1.97}{\theFun}
\psplot[plotpoints=500, linecolor=\fcColorGraph]{2.02}{3.98}{\theFun}
\psplot[plotpoints=500, linecolor=\fcColorGraph]{4.03}{10}{\theFun}
\end{pspicture}



\answer{
\begin{tabular}{l}
$y$-intercept $\frac{-99}{8}$, $x$-intercept: not requested \\
horizontal asymptote: none, vertical: $x=2$ and $x=4$\\
increasing on $\left(- \infty, -\sqrt{2}+3\right)\cup \left(-\frac{\sqrt{3}}{3}+3, \frac{\sqrt{3}}{3}+3\right)\cup \left(\sqrt{2}+3, \infty\right)
$, decreasing on the complement intervals\\
local maxima at $x=-\sqrt{2}+3$, $ x=\frac{\sqrt{3}}{3}+3$\\
local minima at $x=-\frac{\sqrt{3}}{3}+3$, $ x=\sqrt{2}+3$\\
concave up on $\left(\left(2, 3\right)\cup \left(4, \infty\right)\right)$\\ 
concave down on $\left(-\infty, 2\right)\cup \left(3, 4\right)$\\
inflection point at $x=3$
\end{tabular}
}

\end{enumerate}

\solution{\ref{problemSketchCurve(2x^2-5x+9/2)/(x^2-3 x+3)}
We have that $f$ is not defined only when we have division by zero, i.e.,  if $x^2-3x+3$ equals zero. However, the roots of $x^{2}-3x+3$ are not real numbers: they are $\frac{3\pm \sqrt{3^2-4\cdot 3 }}{2}= \frac{3\pm \sqrt{-3}}{2}$, and therefore $x^2-3x+3$ can never equal zero. Alternatively, completing the square shows that the denominator is always positive:
\[
x^2-3x+3=x^2-2\cdot \frac{3}{2} x+\frac{9}{4}-\frac{9}{4}+3=\left(x-\frac{3}{2}\right)^2+\frac{3}{4} >0 
\]
Therefore the domain of $f$ is all real numbers.\textbf{To be continued.}
}
\homeworkEnd
%\end{comment}
\begin{comment}
\homeworkStart{on Lecture 19. \\ will be quizzed on Monday April 28}{}
\item ~\begin{enumerate}[ref={\fcProblemRef}]
\item Find the linearization of $f(x) = \sqrt{x}$ at $a = 100$ and use it to approximate
$\sqrt{99.8}$.

\answer{$L(x) = 10 + 0.05(x-100)$. Therefore $\sqrt{99.8} \approx L(99.8) = 9.99$.}
 
\item Find the linearization of $f(x)=\sqrt{8+x}$ at $a=1$ and use it to approximate $\sqrt{9.02}$.

\answer{ $f(x)\approx 3+ \frac16 (x-1)=\frac{1}{6} x+\frac{17}{6}$. Therefore $\sqrt{9.02}\approx \frac{901}{300} \approx 3.003333$}
\item Find the linearization of $f(x)=\sqrt[3]{8+x}$ at $a=0$ and use it to approximate $\sqrt[3] {7.97}$.

\answer{ $\sqrt[3]{8+x}\approx \frac{1}{12}x+2$. Therefore $\sqrt[3]{7.97}\simeq \frac{799}{400} =1.9975$}

\item Find the linearization of $f(x)=\ln x$ at $a=1$ and use it to approximate $\ln 1.01$.

\answer{ $f(x)\approx f(1)+f'(1)(x-1)=x-1 $, $\ln 1.01\approx 0.01$. }
\item Use a linear approximation to estimate $(1.001)^9$. 

\answer{$(1.001)^9 \approx 1.009$.}
\item \label{problem-linearization-estimate0.9999power2014} Use a linear approximation to estimate $(0.9999)^{2014}$. 

\answer{$(0.9999)^{2014} \approx 0.7986$.}

\end{enumerate}

\homeworkEnd
\end{comment}
\begin{comment}
\homeworkStart{Homework Math 140, Lectures 20-21. \\ Will be quizzed Monday December 2}{}
Find all antiderivatives of the functions.
\begin{multicols}{3}
\begin{enumerate}
\item $\displaystyle f(x)=\sqrt {3}+\pi^2$.

\answer{$\displaystyle x\left(\pi^{2} +\sqrt{3}\right)+C$}
\item $\displaystyle f(x)=x-5$.

\answer{$\displaystyle \frac{x}{2}-5x+C$}

\item $\displaystyle f(x)= x^2-2x+6$.

\answer{ $\frac{x^3}{3} -x^2+6x+C$ }

\item $\displaystyle f(x)=\frac{x(x+1)}{2} $.

\answer{$\frac{1}{6} x^{3}+\frac{1}{4} x^{2}+C$}
\item $\displaystyle f(x)=x(x+1)(2x+1)$.

\answer{$\frac{1}{2} x^{4}+x^{3}+\frac{1}{2} x^{2}+C$}
\item $\displaystyle f(x)=7x^{\frac{2}{7}}+x^{-\frac{4}{7}}$.

\answer{$\frac{49}{9} x^{\frac{9}{7}}+\frac{7}{3} x^{\frac{3}{7}}+ C$}
\item $\displaystyle f(x)=x^{2.4}-2x^{\sqrt{3}-1}$.

\answer{$\frac{5}{17} x^{\frac{17}{5}}-\frac{2\sqrt{3} x^{\sqrt{3}}}{3} +C$}
\item $\displaystyle f(x)=\frac{8}{x^7}$.

\answer{$-\frac{4}{3} x^{-6}+C$}
\item $\displaystyle f(x)=\frac{x+1}{x^3}$.

\answer{$- x^{-1}-\frac{1}{2} x^{-2}+C$}
\item $\displaystyle f(x)=\frac{1}{x}$.

\answer{$\ln |x|+C$}
\item $\displaystyle f(x)=\frac{x^2+1}{x}$.

\answer{$\frac{1}{2} x^{2}+\ln|x|+C $}
\item $\displaystyle f(x)=\frac{5-4x^3+2x^6}{x^4}$.

\answer{$\frac{2}{3} x^{3}-\frac{5}{3} x^{-3}-4 \ln|x|+C $}
\item $\displaystyle g(x)=\frac{1+\sqrt{x}+x}{\sqrt{x^3}}$.

\answer{$2 x^{\frac{1}{2}}-2 x^{-\frac{1}{2}}+\ln|x|+C $}
\item $\displaystyle f(x)=3\sin t-4\cos t$.

\answer{$-3\cos t -4\sin t +C $}
\item $\displaystyle f(\theta)=\sec^2\theta$.

\answer{$\tan \theta +C $}
\item $\displaystyle f(\theta)=\csc^2\theta$.

\answer{$\tan \theta +C $}

\item $\displaystyle f(t)=\sec t \tan t +\csc t \cot t$.

\answer{$\sec t-\csc t +C $}
\item $\displaystyle f(x)=\frac{2+x\cos x}{x}$.

\answer{$2\ln |x|+\sin x $}
\end{enumerate}
\end{multicols}
\begin{problem}
Verify by differentiation that the formula is correct.
\begin{multicols}{2}
\begin{enumerate}
\item $\displaystyle \int\frac{1}{x^2\sqrt{1+x^2}}dx=-\frac{\sqrt{1+x^2}}{x}+C$.
\item $\displaystyle \int cos^2x ~d x= \frac{1}{2}x +\frac{1}{4}\sin (2x)+C$.
\item $\displaystyle \int \cos^3 x ~dx =\sin x - \frac{1}{3}\sin^3 (x)+C$.
\item $\displaystyle \int \frac{x}{\sqrt{a+bx}}dx= \frac{2}{3b^2}(bx-2a)\sqrt{a+bx}+C$
\end{enumerate}
\end{multicols}
\end{problem}
\begin{problem}
Evaluate the definite integral.
\begin{multicols}{3}
\begin{enumerate}
\item $\displaystyle \int\limits_{-2}^{3} (x^2-3) dx$.
\item $\displaystyle \int\limits_{1}^{2} (4x^3-3x^2+2x) dx$.
\item $\displaystyle \int\limits_{-2}^{0} \left(\frac{1}{2}t^4+\frac{1}{4}t^3-t \right) dt$.
\item $\displaystyle \int\limits_{0}^{3}(1+6w^2-10w^4) dw$.
\item $\displaystyle \int\limits_{0}^{2}(2x-3)(4x^2+1) dx$.
\item $\displaystyle \int\limits_{-1}^{1} t(1-t)^2 dt$.
\item $\displaystyle \int\limits_{0}^{\pi} (4\sin \theta -3 \cos \theta)d\theta$.
\item $\displaystyle \int\limits_{1}^{2}\left(\frac{1}{x^2}-\frac{4}{x^3}\right) dx$.
\item $\displaystyle \int\limits_{1}^{4}\left(\frac{4+6u}{\sqrt{u}}\right) du$.
\item $\displaystyle \int\limits_{1}^{2}\left(x+\frac{1}{x}\right)^2 dx$.
\item $\displaystyle \int\limits_{1}^{4}\sqrt{\frac{5}{x}} dx$.
\item $\displaystyle \int\limits_{1}^{9}\frac{3x-2}{\sqrt{x}} dx$.
\item $\displaystyle \int\limits_{1}^{4}\sqrt{t}(1+t) dt$.
\item $\displaystyle \int\limits_{\frac{\pi}{4}}^{\frac{\pi}{4}} \csc^2\theta ~d\theta$.
\item $\displaystyle \int\limits_{0}^{\frac{\pi}{4}}\frac{1+\cos^2\theta}{\cos^2\theta} d\theta$.
\item $\displaystyle \int\limits_{0}^{\frac{\pi}{3}} \frac{\sin \theta +\sin \theta \tan^2\theta}{\sec^2\theta}d\theta$.
\item $\displaystyle \int\limits_{0}^1 \frac{1+\sqrt[3]{x}}{\sqrt{x}}dx$.
\item $\displaystyle \int\limits_{1}^{8}\frac{x-1}{\sqrt[3]{x^2}} dx$.
\item $\displaystyle \int\limits_0^{1} (\sqrt[4]{x^5}+\sqrt[5]{x^4})dx $.
\item $\displaystyle \int\limits_{0}^{1}(1+x^2)^3 dx$.
\item $\displaystyle \int\limits_{2}^{5}|x-3| dx$.
\item $\displaystyle \int\limits_{0}^{2} |2x-1| dx$.
\item $\displaystyle \int\limits_{-1}^{2}(x-2|x|) dx$.
\item $\displaystyle \int\limits_{0}^{\frac{3\pi}{2}}|\sin x| dx$.
\end{enumerate}
\end{multicols}
\end{problem}
\homeworkEnd
\end{comment}
\begin{comment}
\homeworkStart{Homework Math 140, Lecture 22. \\ Will be quizzed Monday December 9}{}
\begin{problem}
Evaluate the indefinite integral.
\begin{multicols}{3}
\begin{enumerate}
\item $\displaystyle\int x \sin(x^2)~dx $.
\item $\displaystyle\int x^2\cos(x^3)~dx $.
\item $\displaystyle\int (1-2x)^9 ~dx $.
\item $\displaystyle\int (3t+2)^{2.4}~dt $.
\item $\displaystyle\int (x+1)\sqrt{2x+x^2} ~dx $.
\item $\displaystyle\int \sec^2(2\theta) ~d\theta $.
\item $\displaystyle\int \sec (3t) \tan (3t)~dt $.
\item $\displaystyle\int u\sqrt{1-u^2} ~du $.
\item $\displaystyle\int \frac{a+bx^2}{\sqrt{3ax+bx^3}}~dx $.
\item $\displaystyle\int \frac{\sin \sqrt{x}}{\sqrt{x}} ~dx $.
\item $\displaystyle\int \sec^2\theta \tan^3\theta  ~d\theta $.
\item $\displaystyle\int \cos^4\theta\sin \theta ~d\theta $.
\item $\displaystyle\int (x^2+1)(x^3+3x)^4 ~dx $.
\item $\displaystyle\int \sqrt{x}\sin (1+x^{\frac{3}{2}}) ~dx $.
\item $\displaystyle\int \frac{\cos x}{\sin^2 x} ~dx $.
\item $\displaystyle\int \frac{\cos\left(\frac{\pi}{x}\right)}{x^2} ~dx $.
\item $\displaystyle\int \frac{z^2}{\sqrt[3]{1+z^3}} ~dz $.
\item $\displaystyle\int \frac{dt}{\cos^2 t\sqrt{1+\tan t}} $.
\item $\displaystyle\int \sqrt{\cot x} \csc^2 x~dx $.
\item $\displaystyle\int \sin t \sec^2(\cos t)~dt $.
\item $\displaystyle\int \sec^3x \tan x~dx $.
\item $\displaystyle\int x^2\sqrt{2+x}~dx $.
\item $\displaystyle\int x(2x+5)^8 ~dx $.
\item $\displaystyle\int x^3\sqrt{x^2+1} ~dx $.
\end{enumerate}
\end{multicols}
\end{problem}
\begin{problem}
Evaluate the integral. You may use the formula $\int \frac{1}{1+x^2}dx=\arctan x+C $. The function $\arctan x$ is the arctangent function (sometimes written as $\tan^{-1}x$).
\begin{multicols}{3}
\begin{enumerate}
\item $\displaystyle\int \frac{dx}{5-3x}$.
\item $\displaystyle\int e^x\sin (e^x) dx$.
\item $\displaystyle\int \frac{(\ln x)^2}{x} dx$.
\item $\displaystyle\int \frac{dx}{ax+b} dx$, $a\neq 0$.
\item $\displaystyle\int e^x\sqrt{1+e^x} dx$.
\item $\displaystyle\int e^{\cos t }\sin t dt$.
\item $\displaystyle\int e^{\tan x}\sec^2x dx$.
\item $\displaystyle\int \frac{\arctan^{-1}x}{1+x^2} dx$. 
\item $\displaystyle\int \frac{1+x}{1+x^2} dx$. 
\item $\displaystyle\int \frac{\sin \ln x}{x} dx$.
\item $\displaystyle\int \frac{\sin (2x)}{1+\cos^2x}dx$.
\item $\displaystyle\int \frac{\sin x}{1+\cos^2 x} dx$.
\item $\displaystyle\int \cot x dx$.
\item $\displaystyle\int \frac{x}{1+x^4}dx$.
\item $\displaystyle\int\limits_{e}^{e^4}\frac{dx}{x\sqrt{\ln x}} dx$.
\item $\displaystyle\int\limits_{0}^{1}xe^{-x^2} dx$.
\item $\displaystyle\int\limits_{0}^{1}\frac{e^z+1}{e^z+z} dz$.
%\item $\displaystyle\int\limits_{0}^{1/2}\frac{\arcsin x}{\sqrt{1-x^2}} dx$.
\end{enumerate}
\end{multicols}
\end{problem}
\homeworkEnd
\end{comment}