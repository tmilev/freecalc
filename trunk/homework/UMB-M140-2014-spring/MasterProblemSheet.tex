\documentclass{article}
\ProvidesPackage{homework-problems-UMB}
\addtolength{\hoffset}{-3.5cm}
\addtolength{\textwidth}{6.8cm}
\addtolength{\voffset}{-3.3cm}
\addtolength{\textheight}{6.3cm}
\usepackage{../homework-problems} %warnign folder paths are relative to the file that uses the includepackage

\renewcommand{\answer}[1]{\iftoggle{answers}{ \hfill{~} \rotatebox{180}{\tiny answer: #1}}{} }
\renewcommand{\pointsii}[1]{}


\toggletrue{solutions}
%\togglefalse{solutions}
\toggletrue{answers}
\newtheorem{problem}{Problem}

\newcommand{\hide}[1]{}

\begin{document}
\begin{center}
\Large
Master Problem Sheet \\ Math 140 Calculus I \\ 
\end{center}

%\noindent \textbf{Name:} \hfill{~}
%\begin{tabular}{c|c|c|c|c|c|c|c|c||c}
%Problem&1 &2&3&4&5&6&7&8& $\sum$\\ \hline
%Score&&&&&&&&&\\ \hline
%Max&20&20&20&20&20&10&20&20&150
%\end{tabular}




This master problem sheet contains all freecalc problems on the topics studied in Calculus II. The latest \LaTeX{} source of this file (complete with typo and error fixes) can be downloaded from the freecalc project page below. 

\url{https://sourceforge.net/p/freecalculus/code/HEAD/tree/}

A list of contributors/authors of the freecalc project (and in particular, the present problem collection) see the following file.
\url{https://sourceforge.net/p/freecalculus/code/HEAD/tree/trunk/contributors.tex}

\section{Functions, Basic Facts}
\begin{problem}
(Stewart, 7ed., page 21, problems 27-30)
Evaluate the difference and simplify your answer.
\begin{multicols}{2}
\begin{enumerate}
\item $\frac{f(3+h)-f(3)}{h}$, where $f(x)=4+3x-x^2$.
\answer{$-3-h$}
\item $\frac{f(a+h)-f(a)}{h}$, where $f(x)= x^3$.
\answer{$ 3a^2+3ah+h^2$}
\item $\frac{f(x)-f(a)}{x-a}$, where $f(x)=\frac{1}{x}$.
\answer{$-\frac{1}{ax}$.}
\item $\frac{f(x)-f(1)}{x-1}$, where $f(x)=\frac{x+3}{x+1}$.
\answer{$-\frac{1}{x+1}$.}
\end{enumerate}
\end{multicols}

\end{problem}
\subsection{Function composition}
\begin{problem}
Compute the expressions $(f\circ g)(x)$, $(g\circ f)(x)$ and simplify to a single fraction. 

\begin{enumerate}
\item $\displaystyle f{}({{x}})=\frac{x+2}{x-2},
g{}({{x}})=\frac{x-1}{x+2}$.

\answer{$(f\circ g)(x)= \frac{3+3 x}{-5- x}$, $(g\circ f)(x)=\frac{4}{-2+3 x}$  }
\item 
$\displaystyle f{}({{x}})=\frac{x+1}{3x-2},
g{}({{x}})=\frac{x-2}{x-1}
$.

\answer{
$(f\circ g)(x)= \frac{-3+2 x}{-4+x}
$, 
$(g\circ f)(x)=\frac{5-5 x}{3-2 x}
$  }
\item 
$\displaystyle f{}({{x}})=\frac{2x+1}{3x-1},
g{}({{x}})=\frac{x-2}{2x-1}
$.

\answer{
$(f\circ g)(x)=\frac{-5+4 x}{-5+x}
$, 
$(g\circ f)(x)=\frac{3-4 x}{3+x}
$  }
\item 
$\displaystyle f{}({{x}})=\frac{x+1}{x-2},
g{}({{x}})=\frac{x+2}{2x-1}
$.

\answer{
$(f\circ g)(x)= \frac{1+3 x}{4-3 x}
$, 
$(g\circ f)(x)=\frac{-3+3 x}{4+x}
$  }
\item 
$\displaystyle f{}({{x}})=\frac{5x+1}{4x-1},
g{}({{x}})=\frac{4x-1}{3x+1}
$.

\answer{
$(f\circ g)(x)= \frac{-4+23 x}{-5+13 x}
$, 
$(g\circ f)(x)=\frac{5+16 x}{2+19 x}
$  }
\item $\displaystyle  f(x)= \frac{3x-5}{x-2}$, $\displaystyle g(y)=\frac{y-2 }{y-4} $. 

\answer{ $(f\circ g)(x)=\frac{-2 x+14}{- x+6}$, $(g\circ f)(x)=\frac{x-1}{- x+3}$}
\item $\displaystyle  f(x)= \frac{x-3}{x+2}$, $\displaystyle g(y)=\frac{y+3 }{y-4} $. 

\answer{ $(f\circ g)(x)=\frac{-2 (x)+15}{3 x-5}$, $(g\circ f)(x)=\frac{4 x+3}{-3 x-11}$}
\end{enumerate}

\end{problem}
\begin{problem}
Find an expression for the function $(f\circ g)(x)$ and $(g\circ f)(x)$. Simplify your answer to a single fraction. 
\begin{enumerate}
\item $\displaystyle  f(x)= \frac{3x-5}{x-2}$, $\displaystyle g(y)=\frac{y-2 }{y-4} $. \answer{ $(f\circ g)(x)=\frac{-2 x+14}{- x+6}$, $(g\circ f)(x)=\frac{x-1}{- x+3}$}
\item $\displaystyle  f(x)= \frac{x-3}{x+2}$, $\displaystyle g(y)=\frac{y+3 }{y-4} $. \answer{ $(f\circ g)(x)=\frac{-2 (x)+15}{3 x-5}$, $(g\circ f)(x)=\frac{4 x+3}{-3 x-11}$}
\end{enumerate}

\end{problem}
\subsection{Domains and ranges}
\begin{problem}
(Stewart, 7th ed., page 21, 31-37) Find the implied domain of the function
\begin{multicols}{2}
\begin{enumerate}
\item $f(x)=\frac{x+4}{x^2-9}$.
\item $f(x)=\frac{2x^3-5}{x^2+x-6}$.
\item $f(t)=\sqrt[3]{2t-1}$.
\item $g(t)=\sqrt{3-t}-\sqrt{2+t}$.
\item $h(x)=\frac{1}{\sqrt[4]{x^2-5x}}$.
\item $f(u)=\frac{u+1}{1+\frac{1}{u+1}}$.
\item $F(p)=\sqrt{2-{\sqrt{p}}}$.
\end{enumerate}
\end{multicols}
\end{problem}
\begin{problem}
Find the functions $f\circ g$, $g\circ f$, $f\circ f$ and $g\circ g$ and their implied domains.

\begin{enumerate}
\item $f(x)=x^2+1$, $g(x)=x+1$. \answer{ in some order: $(1+x)^{2}+1, (x)^{2}+2, ((x)^{2}+1)^{2}+1, 2+x$ }
\item $f(x)=\sqrt{x+1}$, $g(x)=x+1$. \answer{ in some order:$\sqrt{2+x}, 1+\sqrt{1+x}, \sqrt{1+\sqrt{1+x}}, 2+x$}
\item $f(x)= 2x$, $g(x)= \tan x$.
\answer{ in some order:$2 \tan{}x, \tan{}(2 x), 4 x, \tan{}(\tan{}x) $}
\item $f(x)=\frac{x+1}{x-1}$, $g(x)=\frac{x-1}{x+1}$.
\answer{ in some order:$- x, \frac{1}{x}, x, -\frac{1}{x} $}
\end{enumerate}

\end{problem}
\subsection{Linear Transformations and Graphs of Functions}
\begin{problem}
\begin{problem}Graph the functions by hand, by applying consecutively the transformations learned in class.
\begin{multicols}{2}
\begin{enumerate}
\item $y=\frac{1}{x}$.
\item $y=\frac{1}{x+1}$.
\item $y=\frac{1}{2x+1}$.
\item $y=\frac{3}{2x+1}$.
\item $y=\frac{3+x}{2x+1}$.
\item $y=\left|\frac{3+x}{2x+1}\right|$.
\end{enumerate}
\end{multicols}
\end{problem}
\end{problem}
\subsection{Piecewise Defined Functions}
\begin{problem}
Write down formulas for function whose graphs are as follows. The graphs are up to scale. The arc is a part of a circle.
\begin{multicols}{2}
\begin{enumerate}
\psset{xunit=0.4cm, yunit=0.4cm}
\item
\tiny
\begin{pspicture}(-1,-1)(6,5)
\psaxes{->}(0,0)(-1,-1)(6,5)
\psline[linecolor=red](0,3)(2, 0)(5, 4)
\fcFullDot{5}{4}
\rput[r](4.9, 4){$(5, 4)$}
\rput[b](6,0.1 ){$x$}
\rput[l](0.1,5 ){$y$}
\end{pspicture}
\normalsize
\item
\tiny
\psset{xunit=0.4cm, yunit=0.4cm}
\begin{pspicture}(-4,-1)(4,5)
\psaxes{->}(0,0)(-4.5,-1)(4.5,4)
\psplot[linecolor=red]{-2}{2}{4 x x mul sub sqrt }
\psline[linecolor=red](2,0)(4,1)
\psline[linecolor=red](-2,0)(-4,1)
\rput[b](4.5,0.1 ){$x$}
\rput[l](0.1,5 ){$y$}
\fcFullDot{4}{1}
\rput[b](4, 1.1){$(4, 1)$}
\fcFullDot{-4}{1}
\rput[b](-4, 1.1){$(-4, 1)$}

\end{pspicture}
\normalsize
\end{enumerate}
\end{multicols}

\end{problem}
\begin{problem}
Plot the piecewise defined functions by hand. Compare your answer to the plot of a computer algebra system.
\begin{multicols}{2}
\begin{enumerate}
\item $G(x)=\frac{x+|x|}{2x}$.
\item $g(x)=|x|-x$.
\item $f(x)=\doublebrace{x}{x\leq 1}{x^2}{x\geq 1}$.
\end{enumerate}
\end{multicols}

\end{problem}

\section{Trigonometry}
\subsection{Angle conversion}
\begin{problem}
Convert from degrees to radians.
\begin{multicols}{3}
\begin{enumerate}
\item $15^\circ$.
\item $30^\circ$.
\item $36^\circ$.
\item $45^\circ$.
\item $60^\circ$.
\item $75^\circ$.
\item $90^\circ$.
\item $120^\circ$.
\item $135^\circ$.
\item $150^\circ$.
\item $180^\circ$.
\item $225^\circ$.
\item $270^\circ$.
\item $305^\circ$.
\item $360^\circ$.
\item $405^\circ$.
\item $1200^\circ$.
\item $-900^\circ$.
\item $-2014^\circ$.
\end{enumerate}
\end{multicols}

\end{problem}
\begin{problem}
Convert from radians to degrees. The answer key has not been proofread, use with caution.
\begin{multicols}{3}
\begin{enumerate}
\item $4\pi$.

\answer{$720^{\circ}$}
\item $-\frac{7}{6}\pi$.

\answer{$-210^{\circ}$}
\item $\frac{7}{12}\pi$.

\answer{$105^{\circ}$}
\item $\frac{4}{3}\pi$.

\answer{$240^{\circ}$}
\item $-\frac{3}{8}\pi$.

\answer{$-67.5^{\circ}$}
\item $2014\pi$.

\answer{$362520^{\circ}$}
\item $5$.

\answer{$\left(\frac{900}{\pi}\right)^{\circ}\approx 286^\circ$}
\item $-2014$.

\answer{$-362520 ^{\circ}$}
\end{enumerate}
\end{multicols}
\end{problem}
\subsection{Trigonometry identities}
\begin{problem}
(Textbook appendix page 32-, problems 45, 46, 47, 48, 49, 50, 51, 52, 56, 57, 58).
Derive the trigonometry identities.
\begin{multicols}{3}
\begin{enumerate}
\item $\sin \theta\cot \theta =\cos \theta$.
\item $(\sin x +\cos x)^2=1+\sin(2x)$.
\item $\sec y - \cos y= \tan y \sin y$.
\item $\tan^2 \alpha-\sin^2 \alpha=\tan^2\alpha\sin^2\alpha$.
\item $\cot^2\theta+\sec^2\theta=\tan^2\theta+\csc^2\theta$.
\item $2\csc 2t= \sec t \csc t$.
\item $\tan 2\theta =\frac{2\tan \theta}{1-\tan^2\theta} $.
\item $\frac{1}{1-\sin \theta}+ \frac{1}{1+\sin \theta}=2\sec^2\theta$.
\item $\tan x + \tan y = \frac{\sin (x+y)}{\cos x \cos y}$.
\item $\sin 3\theta +\sin \theta = 2 \sin 2\theta \cos \theta $.
\item $\cos 3\theta = 4\cos^3\theta-3\cos \theta $.
\end{enumerate} 
\end{multicols}

\end{problem}

\subsection{Trigonometry equations}
\begin{problem}
\begin{problem}(Textbook page A33, problems 65-72).
Find all values of $x$ in the interval $[0,2\pi]$ that satisfy the 
equation.
\begin{multicols}{3}
\begin{enumerate}
\item $2\cos x - 1=0$.
\item $3\cot^2 x= 1$.
\item $2\sin^2 x= 1$.
\item $|\tan x|=1 $.
\item $\sin 2x = \cos x $.
\item $2\cos x +\sin 2x=0$.
\item $\sin x =\tan x$.
\item $2+\cos 2x = 3 \cos x$.
\end{enumerate}
\end{multicols}
\end{problem}

\end{problem}
\solution{ \ref{problemSolve2cos^2x-(1+sqrt(2))cosx+sqrt(2)/2=0}
Set $\cos x=u$. Then 
\[
2\cos^2 x- (1+\sqrt{2})\cos x+\frac{\sqrt 2}2=0 
\] 
becomes 
\[2u^2-(1+\sqrt{2})u+\frac{\sqrt{2}}{2}=0.
\] 
This is a quadratic equation in $u$ and therefore has solutions
\[
\begin{array}{rcl}
u_1, u_2\displaystyle &=& \displaystyle \frac{ 1+\sqrt{2}\pm\sqrt{ (1+\sqrt{2})^2-4 \sqrt{2} } }4\\
&=&\displaystyle \frac{1+\sqrt{2}\pm\sqrt{1-2\sqrt{2}+2} }4\\
&=&\displaystyle \frac{1+\sqrt{2}\pm \sqrt{(1-\sqrt{2})^2}}4\\
&=&\displaystyle \frac{1+\sqrt{2}\pm (1-\sqrt{2}) }4=\doublebrace{\frac{1}2 }{ \mathrm{or}} {\frac{\sqrt{2}}{2}}{}
\end{array}
\]
Therefore $u=\cos x= \frac12$ or $u=\cos x=\frac{\sqrt{2}}2$, and, as $x$ is in the interval $[0,2\pi]$, we get $x=\frac{\pi}{3}, \frac{5\pi}{3}$ (for $\cos x=\frac12$) or $x=\frac{\pi }4 ,\frac{7\pi}4$ (for $\cos x=\frac{\sqrt{2}}{2}$).
} % end solution

\section{Limits}
\begin{problem}
(Problem contributed by Gabe Cunningham) Find the following limits, or show that they do not exist:
\begin{multicols}{2}
\begin{enumerate}
\item ${\displaystyle \lim_{x \to 2} \frac{x^2-4}{x^2-x-2}}$
\answer{$\frac43$}
\item ${\displaystyle \lim_{x \to -\infty} \frac{5x^3+x-1}{2x^3-7}}$
\answer{$5/2$}
\item ${\displaystyle \lim_{x \to 1^{+}} \frac{x-3}{x-1}}$
\answer{$-\infty$}
\item ${\displaystyle \lim_{h \to 0} \frac{2(x+h)^3 - 2x^3}{h}}$
\answer{$6x^2$}
\item ${\displaystyle \lim_{x \to \infty} \frac{\sqrt{9x^2-2}}{x+4}}$
\answer{$3$}
\item ${\displaystyle \lim_{x \to -1} \frac{2x+3}{x+1}}$
\answer{Does not exist}
\end{enumerate}
\end{multicols}

\end{problem}
\begin{problem}
(Textbook page 70, problems 11-32). 
Evaluate the limit if it exists.
\begin{multicols}{2}
\begin{enumerate}
\item $\displaystyle\lim\limits_{x\to 5}\frac{x^2-6x+5}{x-5} $. 
\answer{4}
\item $\displaystyle\lim\limits_{x\to 4}\frac{x^2-4x}{x^2-3x-4} $.
\answer{$\frac{4}5$}
\item $\displaystyle\lim\limits_{x\to 5}\frac{x^2-5x+6}{x-5} $.
\answer{DNE}
\item $\displaystyle\lim\limits_{x\to -1}\frac{x^2-4x}{x^{2}-3x-4} $.
\answer{DNE}
\item $\displaystyle\lim\limits_{t\to -3}\frac{t^2-9}{2t^2+7t+3} $.
\answer{$\frac{6}{5}$}
\item $\displaystyle\lim\limits_{x\to -1}\frac{2x^2+3x+1}{x^2-2x-3} $.
\answer{$\frac{1}{4}$}
\item $\displaystyle\lim\limits_{h\to 0}\frac{(-5+h)^2-25}{h} $.
\answer{$-10$}
\item $\displaystyle\lim\limits_{h\to 0}\frac{(2+h)^3-8}{h} $.
\answer{$12$}
\item $\displaystyle\lim\limits_{x\to -2}\frac{x+2}{x^3+8} $.
\answer{$\frac{1}{12}$}
\item $\displaystyle\lim\limits_{t\to 1}\frac{t^4-1}{t^3-1} $.
\answer{$\frac{4}{3}$}
\item $\displaystyle\lim\limits_{h\to 0}\frac{\sqrt{9+h}-3}{h} $.
\answer{$\frac{1}{6}$}
\item $\displaystyle\lim\limits_{u\to 2} \frac{\sqrt{4u+1}-3}{u-2}$.
\answer{$\frac{2}{3}$}
\item $\displaystyle\lim\limits_{x\to -4} \frac{\frac{1}{4}+ \frac{1}{x}} {4+x}$.
\answer{$-\frac{1}{16}$}
\item $\displaystyle\lim\limits_{x\to -1} \frac{x^2+2x+1}{x^4-1}$.
\answer{$0$}
\item $\displaystyle\lim\limits_{t\to 0} \frac{\sqrt{1+t}- \sqrt{1-t}}{t}$.
\answer{$1$}
\item $\displaystyle\lim\limits_{t\to 0}\left(\frac{1}t -\frac{1}{t^2+t}\right)$.
\answer{$1$}
\item $\displaystyle\lim\limits_{x\to 16} \frac{4-\sqrt{x}}{16x-x^2}$.
\answer{$\frac{1}{128}$}
\item $\displaystyle\lim\limits_{h \to 0}\frac{(3+h)^{-1}-3^{-1}}{h} $.
\answer{$-\frac{1}{9}$}
\item $\displaystyle\lim\limits_{t\to 0} \left(\frac{1}{t\sqrt{1+t}}-\frac{1}{t} \right)$.
\answer{$-\frac{1}{2}$}
\item $\displaystyle\lim\limits_{x\to -4} \frac{\sqrt{x^2+9}-5}{x+4}$.
\answer{$-\frac{4}{5}$}
\item $\displaystyle\lim\limits_{h\to 0}\frac{(x+h)^3-x^3}{h} $.
\answer{$3x^2$}
\item $\displaystyle\lim\limits_{h\to 0}\frac{\frac{1}{(x+h)^2}-\frac{1}{x^2}}{h} $.
\answer{$-\frac{2}{x^3}$}
\solution{ 
~\\
$
\begin{array}{rcl}
\lim\limits_{h\to 0}\frac{\frac{1}{(x+h)^2}-\frac{1}{x^2}}{h}&=&\lim\limits_{h\to 0}\frac{x^2-(x+h)^2}{hx^2(x+h)^2}=\lim\limits_{h\to 0} \frac{x^2-(x^2+2xh+h^2)}{hx^2(x+h)^2}\\
&=&\lim\limits_{h\to 0}\frac{\cancel{h}(-2x+h)}{\cancel{h}x^2(x+h)^2}= \frac{-2x+0}{x^2(x+0)^2}=-\frac{2}{x^3}
\end{array}
$
}
\end{enumerate}
\end{multicols}

\end{problem}
\solution{\ref{problemlim(xto2)(x^2-5x+6)/(x-2)}

$
\begin{array}{rcll|l}
\displaystyle 
\displaystyle \lim\limits_{x\to 2}\frac{x^2-5x+6}{x-2} &=&\displaystyle \lim\limits_{x\to 2}\frac{(x-3)\cancel{(x-2)}}{\cancel{x-2}} &&\text{factor and cancel}\\
&=&\displaystyle 2-3=-1
\end{array}
$
}
\solution{\ref{problemlimxto-2(2x^2+x-6)/(x^2-4)}

$\begin{array}{rcll|l}
\displaystyle \lim\limits_{x\to -2} \frac{2x^2+x-6}{x^2-4}&=&  \displaystyle \lim\limits_{x\to -2}\frac{ (2x -3)\cancel{( x+ 2 ) }}{ (x-2)\cancel{(x+2)}} &&\text{factor and cancel}\\ 
&=&\displaystyle  \frac{(2(-2)-3)}{-2-2} &&\text{substitute}\\
&=&\displaystyle \frac{7}{4}
\end{array}
$

}
\solution{\ref{limproblem(xto-2)(x^2-4)/(2x^2+5x+2)}

$
\begin{array}{rcll|l}
\displaystyle 
\displaystyle \lim\limits_{x\to 2}\frac{x^2-4}{2x^2+5x+2} &=&\displaystyle \lim\limits_{x\to -2} \frac{(x-2)\cancel{(x+2)}}{(2x+1) \cancel{(x+2)}} &&\text{factor and cancel}\\
&=&\displaystyle \frac{(-2)-2}{2(-2)+1}=\frac{4}{3}
\end{array}
$
}
\solution{
\ref{problemlim(xto-1)(2x^2+3x+1)/(3x^2-2x-5)}

$
\begin{array}{rcll|l}
\displaystyle \lim\limits_{x\to-1}\frac{2x^2+3x+1}{3x^2-2x-5} &=&\displaystyle \lim\limits_{x\to -1}\frac{(2x+1)\cancel{(x+1)}}{(3x-5)\cancel{(x+1)}}&&\text{factor and cancel}\\
&=&\displaystyle \frac{2(-1)+1}{3(-1)-5} =\frac{1}{8}.
\end{array}
$
}
\solution{\ref{problemlimxto-4(x^2+7x+12)/(x^2+6x+8)}.

$\begin{array}{rcll|l}
\displaystyle \lim_{x \to -4}\frac{x^{2}+7 x+12}{x^{2}+6 x+8}&=& \displaystyle \lim_{x \to -4}\frac{(x+3)(\cancel{x+4})}{(x+2)(\cancel{x+4})} &&\text{factor}\\
&=&\displaystyle \frac{-4+3}{-4+2}=-\frac{1}{2}.
\end{array}
$

}
\input{\freecalcBaseFolder/modules/limits/homework/limit-x-tends-to-a-difference-quotient-subproblem-22-solution}

\solution{\ref{problemlimhto0(1/(2+h)^2-1/4)/h}.

\textbf{Variant I.}

$\begin{array}{rcll|l}
\displaystyle \lim_{h\to 0} \frac{\frac{1}{(2+h)^2}-\frac{1}{4}}{h}&=&\displaystyle \lim_{h\to 0}\frac{\frac{4-(2+h)^2}{4(2+h)^2}}{h}\\
&=&\displaystyle \lim_{h\to 0} \frac{4- (4+4h+h^2)}{4h(2+h)^2}\\
&=&\displaystyle \lim_{h\to 0} \frac{-4h-h^2}{4h(2+h)^2}\\
&=&\displaystyle \lim_{h\to 0} \frac{\cancel{h}(-4-h) }{4\cancel{h}(2+h)^2}&&\text{substitute }h=0\\
&=&\displaystyle \frac{-4-0}{4(2+0)^2}\\
&=&\displaystyle -\frac{1}{4}
\end{array}
$

\textbf{Variant II.}

$\begin{array}{rcll|l}
\displaystyle \lim_{h\to 0} \frac{\frac{1}{(2+h)^2}-\frac{1}{4}}{h}&=&\displaystyle \frac{\diff }{\diff x}\left(\frac{1}{x^2}\right)_{|x=2}\\
&=&\displaystyle \left(\frac{-2}{x^3}\right)_{|x=2}\\
&=&\displaystyle -\frac{1}{4}
\end{array}
$

}


\solution{\ref{problemlimhto0(1/(1+h)^2-1)/h}.

\textbf{Variant I.}

$\begin{array}{rcll|l}
\displaystyle \lim_{h\to 0} \frac{\frac{1}{(1+h)^2}-1}{h}&=&\displaystyle \lim_{h\to 0}\frac{\frac{1-(1+h)^2}{ (1+h)^2}}{h}\\
&=&\displaystyle \lim_{h\to 0} \frac{1- (1+2h+h^2)}{h(1+h)^2}\\
&=&\displaystyle \lim_{h\to 0} \frac{-2h-h^2}{h(1+h)^2}\\
&=&\displaystyle \lim_{h\to 0} \frac{\cancel{h}(-2-h) }{\cancel{h}(1+h)^2}&&\text{substitute }h=0\\
&=&\displaystyle \frac{-2-0}{(1+0)^2}\\
&=&\displaystyle -2.
\end{array}
$

\textbf{Variant II.}

$\begin{array}{rcll|l}
\displaystyle \lim_{h\to 0} \frac{\frac{1}{(1+h)^2}-1}{h}&=&\displaystyle \frac{\diff }{\diff x}\left(\frac{1}{x^2}\right)_{|x=1}&&\text{derivative definition}\\
&=&\displaystyle \left(\frac{-2}{x^3}\right)_{|x=1}\\
&=&\displaystyle -2.
\end{array}
$

}
\begin{problem}
Evaluate the limit if it exists.
\begin{enumerate}
\item $\lim\limits_{x\to 1} \frac{3x^2+4x-7}{x^3-x}$ \answer{$5 $.}
\item $\lim\limits_{x\to -1} \frac{2x^2-3x-5}{x^3+1}$ \answer{$ -\frac{7}{3}$.}
\end{enumerate}

\end{problem}
\begin{problem}
Evaluate the limits. Justify your computations.
\begin{multicols}{3}
\begin{enumerate}
\item $\displaystyle\lim\limits_{x\to 2} 2x^2-3x-6$.
\answer{$-4$}
\item $\displaystyle\lim\limits_{x\to -1} \frac{x^4-x}{x^2+2x+3}$.
\answer{$1$}
\item $\displaystyle\lim\limits_{x\to -1} \frac{1}{x^2 -3x +2} $.
\answer{$\frac{1}{6}$}
\item $\displaystyle\lim\limits_{x\to -2}\sqrt{x^4+16}$.
\answer{$\sqrt{32}$}
\item $\displaystyle\lim\limits_{x \to 8}(1+ \sqrt[3]{x} )(2-  x)$.
\answer{$-18$}
\end{enumerate}
\end{multicols}

\end{problem}
\begin{problem}
Find the limit or show that it does not exist. If the limit does not exist, indicate whether it is $\pm\infty$, or neither. The answer key has not been proofread, use with caution.
\begin{multicols}{3}
\begin{enumerate}
\item $\displaystyle \lim\limits_{x\to\infty }\frac{x-2}{2x+1}$.

\answer{$\frac12$}
\item $\displaystyle \lim\limits_{x\to\infty }\frac{1-x^2}{x^3-x-1}$.

\answer{$ 0$}
\item $\displaystyle \lim\limits_{x\to-\infty }\frac{x-2}{x^2+5}$.

\answer{$ 0$}
\item $\displaystyle \lim\limits_{x\to-\infty }\frac{3x^3+2}{2x^3-4x+5}$.

\answer{$ \frac{3}{2}$}
\item $\displaystyle \lim\limits_{x\to\infty }\frac{\sqrt{x}+x^2}{\sqrt{x}-x^2}$.

\answer{$-1$}
\item $\displaystyle \lim\limits_{x\to\infty }\frac{3-x\sqrt{t}}{2x^{\frac{3}{2}}-2}$.

\answer{$-\frac12$}
\item $\displaystyle \lim\limits_{x\to\infty }\frac{(2x^2+3)^2}{(x-1)^2(x^2+1)}$.

\answer{$ 4$}
\item $\displaystyle \lim\limits_{x\to\infty }\frac{x^2-3}{\sqrt{x^4+3}}$.

\answer{$1$}
\item $\displaystyle \lim\limits_{x\to\infty }\frac{\sqrt{16x^6-3x}}{x^3+2}$.

\answer{$3$}
\item $\displaystyle \lim\limits_{x\to-\infty }\frac{\sqrt{16x^6-3x}}{x^3+2}$.

\answer{$-3$}
\item $\displaystyle \lim\limits_{x\to\infty}\sqrt{4x^2+x}-2x$.

\answer{$\frac{1}{4}$}
\item $\displaystyle \lim\limits_{x\to-\infty} x+\sqrt{x^2+3x} $.

\answer{$-\frac{3}{2} $}
\item $\displaystyle \lim\limits_{x\to\infty}\sqrt{x^2+ax}-\sqrt{x^2+bx}$.

\answer{$\frac{a-b}2$}
\item $\displaystyle \lim\limits_{x\to\infty}\cos x$.

\answer{DNE}
\item $\displaystyle \lim\limits_{x\to\infty}\frac{x^4+x}{x^3-x+2}$.

\answer{$\infty$}
\item $\displaystyle \lim\limits_{x\to\infty}\sqrt{x^2+1}$.

\answer{$\infty$}
\item $\displaystyle \lim\limits_{x\to-\infty}(x^4+x^5)$.

\answer{$-\infty$}
\item $\displaystyle \lim\limits_{x\to-\infty}\frac{\sqrt{1+x^6}}{1+x^2}$.

\answer{$\infty$}
\item $\displaystyle \lim\limits_{x\to\infty}(x-\sqrt{x})$.

\answer{$\infty$}
\item $\displaystyle \lim\limits_{x\to\infty}(x^2-x^3)$.

\answer{$-\infty$}
\item $\displaystyle \lim\limits_{x\to\infty}x\sin x$.

\answer{DNE}
\item $\displaystyle \lim\limits_{x\to\infty}\sqrt{x}\sin x$.

\answer{DNE}
\end{enumerate}
\end{multicols}
\end{problem}




\begin{problem}
Find the derivative of the following functions.
\begin{multicols}{2}
\begin{enumerate}
\item ${\displaystyle \frac{\sin x}{x^2}}$

\answer{${\displaystyle \frac{x \cos x - 2 \sin x}{x^3}}$}
\item ${\displaystyle e^{\sqrt{x^2 + 1}}}$

\answer{
$\begin{array}{l}\displaystyle e^{\sqrt{x^2 + 1}} \cdot \frac{1}{2}\left(x^2+ 1\right)^{ -\frac{ 1}{2}} \cdot 2x \\~\\
= \frac{x e^{\sqrt{x^2+1}} }{ \sqrt{ x^2 +1} } \end{array}$
} 
\item ${\displaystyle \ln \left(x-\frac{1}{x} \right)}$

\answer{${\displaystyle \frac{1}{x-\frac{1}{x}} \cdot \left(1 + \frac{1}{x^2}\right)}$} 
\item ${\displaystyle \sqrt[3]{x} \ln x}$

\answer{${\displaystyle \frac{1}{3} \frac{1}{\sqrt[3]{x^2}} \ln x + \frac{1}{\sqrt[3]{x^2}} }$} 
\item ${\displaystyle \cos(e^x)}$

\answer{${\displaystyle -\sin\left(e^x\right) \cdot e^x}$} 
\item ${\displaystyle \sin^3(2x)}$

\answer{${\displaystyle 3 \sin^2(2x) \cdot \cos(2x) \cdot 2 = 6 \sin^2(2x) \cos(2x)}$} 
\item ${\displaystyle f(x) = \int_x^1 (2+t^4)^5 \; \diff t}$

\answer{${\displaystyle -\left(2+x^4\right)^5}$} 
\item ${\displaystyle g(x) = \int_{0}^{x^3} \cos^2 t \; \diff t}$

\answer{${\displaystyle 3x^2 \cos^2\left(x^3\right)}$} 
\item Find $y'$ if $2x^2 + x + xy = 1$.

\answer{${\displaystyle y' = \frac{-4x-1-y}{x}}$} 
\item Find $y'$ if $x \sin y + y \sin x = 4$.

\answer{${\displaystyle y' = \frac{-\sin y - y \cos x}{\sin x + x \cos y}}$} 
\end{enumerate}
\end{multicols}

\end{problem}
\solution{\ref{problemDifferentiateFTC1int_x^1(2+t^4)^5dt} %(Contributed by student Anamaria Ronayne)

We recall that the Fundamental Theorem of Calculus part 1 states that $\frac{\diff}{\diff x}\left(\int_{a}^{x}h(t)dt\right)=h(x)$
where $a$ is a constant. We can rewrite the integral so it has $x$ as the upper limit:
\[
f(x)=\int_{x}^{1}(2+1^4)^5dt =-\int_{1}^{x}(2+1^4)^5dt\quad.
\]
Therefore
\[
\frac{\diff}{\diff x}\left( -\int_{1}^{x}(2+t^4)^5 \diff t\right)=- \frac{\diff }{\diff x}\left(\int_{1}^{x}(2+t^4)^5\diff t\right)\stackrel{\text{FTC part 1}}{=}
-(2+x^4)^5\quad .
\]

}

\begin{problem}
Evaluate the indefinite integrals.
\begin{multicols}{2}
\begin{enumerate}[ref={\fcProblemRef}]
\item \label{probleminte^x(sqrt(e^x+1))dx} ${\displaystyle \int e^x \left(\sqrt{e^x + 1}\right) \diff x}$

\answer{${\displaystyle \frac23 (e^x + 1)^{\frac{3}{2}} + C}$}

\item ${\displaystyle \int \frac{x^4 + 3x}{x^2} \diff x}$

\answer{${\displaystyle \frac{x^3}{3} + 3 \ln |x| + C}$}

\item ${\displaystyle \int x^2 e^{x^3} \diff x}$
\answer{${\displaystyle \frac{1}{3}e^{x^3} + C}$}

\item ${\displaystyle \int \frac{\cos x}{\sin x} \diff x}$
\answer{${\displaystyle \ln |\sin x| + C}$}

\item $\displaystyle\int \tan (2x) ~\diff x$.

\answer{$-\frac{1}{2}\ln\left|\cos (2x) \right| +C$}

\item $\displaystyle \int \cot \left(\frac{x}{2}\right)~\diff x$

\answer{$2\ln\left(\sin \left(\frac{x}{2}\right) \right)$}

\end{enumerate}
\end{multicols}

\end{problem}
\solution{\ref{probleminte^x(sqrt(e^x+1))dx}
\[
\begin{array}{rcll|l}
\displaystyle\int e^x\sqrt{e^x+1} ~ \diff x&=& \displaystyle \int \sqrt{e^x+1}\diff \left(e^x\right)\\
&=&\displaystyle \int \sqrt{e^x+1}\diff \left(e^x+1\right)
\displaystyle &&\text{Set }u=e^x+1\\
&=&\displaystyle \int \sqrt{u}\diff u\\
&=&\displaystyle \frac{2}{3}u^{\frac{3}{2}}+C\\
&=&\displaystyle \frac{2}{3}\left(e^x+1\right)^{\frac{3}{2}}+C
\end{array}
\]
}

\begin{problem}
Evaluate the definite integrals.
\begin{multicols}{2}
\begin{enumerate}
\item $\displaystyle\int\limits_{1}^{2} \frac{x}{2x^2+1 }  ~dx$.

\answer{$\frac14 \ln 3$}
\item $\displaystyle\int\limits_{0}^{\frac{1}4}\frac{x }{\sqrt{1-3x^2}}dx$.

\answer{$\frac{1}3\left(1-\sqrt{\frac{13}{16}} \right)$}

\item $\displaystyle\int\limits_{1}^{8} \frac{t-t^{\frac{1}{3}}+2}{t^{\frac{4}{3}}} dt\quad .$

\answer{$- \ln8+\frac{15}{2}$}
\item $\displaystyle\int\limits_{1}^{4} (x+\sqrt{x})^2 dx\quad .$

\answer{$\frac{119}{2}$}

\end{enumerate}
\end{multicols}

\end{problem}

\begin{problem}
\input{../../modules/antiderivatives/homework/antiderivatives-basic-integrals-initial-condition-1}
\end{problem}

\begin{problem}
\begin{enumerate}
\item Sketch the graph of $y = x^4 - 8x^2 + 8$ by determining the intervals of increase and decrease, finding the local mins and maxes, determining where the graph is concave up and concave down, and plotting a few key points.

\answer{
\begin{tabular}{l}
Check your graph with a calculator or online graphing program. \\
Local max at 0, local mins at 2 and -2. Concave down between $-\sqrt{4/3}$ and $\sqrt{4/3}$, and concave up otherwise.
\end{tabular}
}

\item Sketch the graph of $y = \frac{x-1}{x^2-9}$ by graphing any vertical and horizontal asymptotes, finding the $x$- and $y$-intercepts, and then sketching a graph that fits this information.

\answer{
\begin{tabular}{l}
Check your graph with a calculator or online graphing program. \\
Vertical asymptotes at $x = 3$ and $x = -3$. \\ Horizontal
asymptote at $y = 0$. \\
$y$-intercept of $\frac{1}{9}$; $x$-intercept of $1$.
\end{tabular}
}
\end{enumerate}

\end{problem}

\begin{problem}
A computer generated plot of $y=f(x)$ is included below. Find the 
\begin{multicols}{2}
\begin{itemize}
\item implied domain of $f$
\item $x$ and $y$ intercepts of $f$.
\item horizontal and vertical asymptotes.
\item intervals of increase and decrease
\item local and global minima, maxima,
\item intervals of concavity 
\item points of inflection
\end{itemize}
\end{multicols}
Label all relevant points on the graph. 
\begin{enumerate}
\item \label{problemSketch(x+1)/(x^2+2x+4)}  $\displaystyle f(x)=\frac{x+1}{x^2+2x+4}$
\psset{xunit=1cm, yunit=1cm}
\begin{pspicture}(-4.500000, -5)(4.500000,5) 
\psframe*[linecolor=white](-4.500000,-1)(4.500000,1) 
\tiny 
\psaxesStandard{-5.000000}{-1}{3.000000}{1} %Function formula: \frac{x+1}{x^{2}+2 x+4} 
\psplot[linecolor=\psColorGraph, plotpoints=1000]{-5.000000}{3.000000}{1 x add 4 x 2 mul add x 2 exp add div }
\end{pspicture} 

\answer{
\begin{tabular}{l}
$y$-intercept: $\frac14$, $x$-intercept: $-1$\\
horizontal asymptote: $y=0$, vertical: none\\
increasing on 
$\left(-1-\sqrt{3}, -1+\sqrt{3}  \right) $, decreasing on $\left(-\infty, -1-\sqrt{3}\right)\cup \left(-1+\sqrt{3}, \infty\right) $\\
local and global min at $x=-1-\sqrt{3}$, local and global max at $x=-1+\sqrt{3}$\\
concave up on $\left(-4, -1\right)\cup \left(2, \infty \right)$, concave down $\left(-\infty, -4\right)\cup \left(-1, 2\right)$\\
inflection points at $x=-4,x=-1, x=2$
\end{tabular}
}
\item $f(x)=\frac{x^{2}+3 x+1}{x^{2}+2 x}$
\psset{xunit=0.5cm, yunit=0.5cm}
\begin{pspicture}(-5.4, -4.118006)(5.4,6.738095) 
\tiny 
\psaxesStandard{-5.15}{-3.868006}{5.15}{6.388095}

%Function formula: \frac{x^{2}+3 x+1}{x^{2}+2 x} 
\psplot[linecolor=\psColorGraph, plotpoints=1000]{0.1}{5}{ 1 x 3 mul add x 2 exp add x 2 mul x 2 exp add div }

%Function formula: \frac{x^{2}+3 x+1}{x^{2}+2 x} 
\psplot[linecolor=\psColorGraph, plotpoints=1000]{-1.9}{-0.1}{ 1 x 3 mul add x 2 exp add x 2 mul x 2 exp add div }

%Function formula: \frac{x^{2}+3 x+1}{x^{2}+2 x} 
\psplot[linecolor=\psColorGraph, plotpoints=1000]{-5}{-2.1}{ 1 x 3 mul add x 2 exp add x 2 mul x 2 exp add div }
\end{pspicture}

\answer{
\begin{tabular}{l}
$y$-intercept: none, $x$-intercepts: $\frac{-3\pm \sqrt{5}}2$ \\
horizontal asymptote: $y=1$, vertical: $x=0$ and $x=-2$\\
always decreasing\\
no local/global minima/maxima\\
concave down on $\left(-\infty,-2\right)\cup \left(-1,0 \right)$, concave up on $\left(-2, -1\right)\cup \left(0, \infty\right)$\\
inflection point at $x=-1$
\end{tabular}
}
\end{enumerate}


\end{problem}
\input{../../modules/curve-sketching/homework/sketch-graph-with-all-the-details-3-solutions}

\begin{problem}
\begin{enumerate}[ref={\fcProblemRef}]
\item Find the linearization of the function $f(x) = \sqrt{x}$ at $a = 100$, and then use your new function to approximate
$\sqrt{99.8}$.

\answer{$L(x) = 10 + 0.05(x-100)$. Therefore $\sqrt{99.8} \approx L(99.8) = 9.99$.}

\item Use a linear approximation to estimate $(1.001)^9$. 

\answer{$(1.001)^9 \approx 1.009$.}

\item $f(x)=\sqrt{8+x}$ at $a=1$ and use it to approximate $\sqrt{9.02}$.

\answer{ $f(x)\approx 3+ \frac16 (x-1)=\frac{1}{6} x+\frac{17}{6}$. Therefore $\sqrt{9.02}\approx \frac{901}{300} \approx 3.003333$}
\item $f(x)=\sqrt[3]{8+x}$ at $a=0$ and use it to approximate $\sqrt[3] {7.97}$.

\answer{ $\sqrt[3]{8+x}\approx \frac{1}{12}x+2$. Therefore $\sqrt[3]{7.97}\simeq \frac{799}{400} =1.9975$}

\item \label{problem-linearization-estimate0.9999power2014} Use a linear approximation to estimate $(0.9999)^{2014}$. 

\answer{$(0.9999)^{2014} \approx 0.7986$.}

\item Find the linearization of $f(x)=\ln x$ at $a=1$ and use it to approximate $\ln 1.01$.

\answer{ $f(x)\approx f(1)+f'(1)(x-1)=x-1 $, $\ln 1.01\approx 0.01$. }

\end{enumerate}

\end{problem}

\begin{problem}
Estimate the integral using a Riemann sum using the indicated sample points and interval length.
\begin{enumerate}[ref={\fcProblemRef}]
% Riemann sums
\item\label{problemRiemannSum-sqrt(8x+1)} $\displaystyle \int_0^4 \left(\sqrt{8x+1}\right)\diff x$. Use four intervals of equal width, choose the sample point to be the left endpoint of each interval. 

\answer{ $\Delta x = 1$ and $f(x) = \sqrt{8x+1}$. Thus ${\displaystyle \int_0^4 f(x) \diff x \approx 9 + \sqrt{17}}$.}

\item $\displaystyle \int_0^6 \frac{1}{x^2+1} \diff x$. Use three intervals of equal width, choose the sample point to be the left endpoint. 

\answer{ $\Delta x = 2$ and $f(x) = \frac{1}{x^2+1}$. Thus ${\displaystyle \int_0^6 f(x) \diff x \approx \frac{214}{85}}$.}
\item\label{problemRiemannSum-1div1plusxsquared} $\displaystyle \int\limits_{-3.5}^{-0.5} \frac{\diff x}{x^2+1} $. Use three intervals of equal width, choose the sample point to be the midpoint of each interval. 

\answer{ $\Delta x = 1$ and $f(x) = \frac{1}{x^2+1}$. Thus $\displaystyle \int \limits_{-3.5}^{-0.5} f(x) \diff x  \approx \Delta x\left(f{} \left(-3 \right)+ f{}\left( -2\right)+f{}\left(-1\right)\right)=\frac{4}{5}=0.8$.}

\item $\displaystyle\int_{0}^2 \frac{\diff x}{1+x+x^3}$. Use $\Delta x=\frac{1}2 $ and right endpoint sampling points.

\answer{$ \frac{1}{2}\left(\frac{8}{13}+\frac{1}{3}+\frac{8}{47}+\frac{1}{11}\right)=\frac{12197}{20163}\approx 0.604920$}
\item $\displaystyle\int_{-2}^{0} \frac{\diff x}{1+x+x^2}$. Use $\Delta x=\frac23 $ and left endpoint sampling points.

\answer{$\frac23\left(\frac{1}{3}+\frac{9}{13}+\frac{9}{7}\right)=\frac{1262}{819}\approx 1.540904$}

\item $\displaystyle \int\limits_0^2 \frac{\diff x}{1+x^3}$. Use four intervals of equal width, choose the sample point to be the left endpoint of each interval. 

\answer{ $\Delta x = 0.5$ and $f(x) = \frac{1}{1+x^3}$. Thus $\displaystyle \int\limits_0^2 f(x) \diff x  \approx \Delta x\left(f{}\left(0\right)+f{}\left(1\right)+f{}\left(\frac{1}{2}\right)+f{}\left(\frac{3}{2}\right)\right)=\frac{1649}{1260}\approx 1.30873$.}

\item $\displaystyle \int\limits_{-2}^{0} \frac{\diff x}{x^4+1} $. Use four intervals of equal width, choose the sample point to be the right endpoint. 

\answer{ $\Delta x = 0.5$ and $f(x) = \frac{1}{1+x^3}$. Thus $\displaystyle \int\limits_0^2 f(x) \diff x  \approx \Delta x\left(f{}\left(-\frac{3}{2}\right)+f{}\left(-1\right)+f{}\left(-\frac{1}{2}\right)+f{}\left(0\right)\right)=\frac{8595}{6596}\approx 1.303062$.}

\end{enumerate}



\end{problem}
\input{\freecalcBaseFolder/modules/integration/homework/riemann-sum-problems-1-problem-1-solution}
\input{\freecalcBaseFolder/modules/integration/homework/riemann-sum-problems-1-problem-3-solution}
\solution{\ref{problemRiemannSum1/(3x^2+1)from-1to0with3intervalsLeftEndpt}

$\Delta x = \frac{1}{3}$ and $f(x) =\frac{1}{3 {{x}}^{2}+1}$. Thus $\displaystyle \int\limits_{-1}^0 f(x) \diff x$  is approximated by $\Delta x \left(f{}\left(-1\right)+f{}\left(-\frac{2}{3}\right)+f{}\left(-\frac{1}{3}\right)\right)=\frac{10}{21}$.

}


\begin{problem}
\begin{enumerate}[ref={\fcProblemRef}]
% Optimization
\item \label{problemponthyperbolax^2-4y^2closestTo1,1} What is the $x$-coordinate of the point on the hyperbola $x^2 - 4y^2 = 16$ that is closest to the point $(1, 0)$?

\psset{xunit=0.3cm, yunit=0.3cm}
\begin{pspicture}(-10.500000, -5)(10.500000,5)
\psframe*[linecolor=white](-10.500000,-5)(10.500000,5)
\tiny
\fcAxesStandard{-10.000000}{-4.5}{10.000000}{4.5} %Function formula: - (1/4 x^{2}-4)^{1/2}
\psplot[linecolor=\fcColorGraph, plotpoints=1000]{-10.000000}{-4.000000}{-4 x 2 exp 0.25 mul add 0.5 exp -1 mul }
%Function formula: (1/4 x^{2}-4)^{1/2}
\psplot[linecolor=\fcColorGraph, plotpoints=1000]{-10.000000}{-4.000000}{-4 x 2 exp 0.25 mul add 0.5 exp }
%Function formula: - (1/4 x^{2}-4)^{1/2}
\psplot[linecolor=\fcColorGraph, plotpoints=1000]{4.000000}{10.000000}{-4 x 2 exp 0.25 mul add 0.5 exp -1 mul }
%Function formula: (1/4 x^{2}-4)^{1/2}
\psplot[linecolor=\fcColorGraph, plotpoints=1000]{4.000000}{10.000000}{-4 x 2 exp 0.25 mul add 0.5 exp }
\fcFullDot{1}{0}
\fcFullDot{4}{0}
\pscircle[linestyle=dotted](1,0){0.9}
\end{pspicture}

\answer{$x = 4$}

\item What is the $x$-coordinate of the point on the ellipse $x^2+4y^2=16$ closest to the point $(1,0)$?

\psset{xunit=0.3cm, yunit=0.3cm}
\begin{pspicture}(-4.500000, -5)(4.500000,5)
\psframe*[linecolor=white](-4.500000,-5)(4.500000,5)
\tiny
\psline[linecolor=red!1](-10,0)(-9.99,0)
\fcAxesStandard{-4.000000}{-4.5}{4.000000}{4.5} %Function formula: - (-1/4 x^{2}+4)^{1/2}
\psplot[linecolor=\fcColorGraph, plotpoints=1000]{-4.000000}{4.000000}{4 x 2 exp -0.25 mul add 0.5 exp -1 mul }
%Function formula: (-1/4 x^{2}+4)^{1/2}
\psplot[linecolor=\fcColorGraph, plotpoints=1000]{-4.000000}{4.000000}{4 x 2 exp -0.25 mul add 0.5 exp }
\pscircle[linestyle=dotted](1,0){0.574456}
\fcFullDot{1}{0}
\fcFullDot{1.333333}{1.885618}
\fcFullDot{1.333333}{-1.885618}
\end{pspicture}
\answer{$x=\frac43$}
\item \label{problemMaxVolumeBoxFixedAreaDoubleBottomNoLid} You want to build a rectangular box with a square base out of sheet metal. You are going to use 2 pieces of sheet metal for the bottom of the box to reinforce it, and only a single piece of sheet metal for all of the sides and the top. If you want to use no more than $36$ sq. ft. of material, what is the largest possible volume you can enclose?

\answer{12 cubic feet.}
\end{enumerate}

\end{problem}
\solution{\ref{problemponthyperbolax^2-4y^2closestTo1,1}

The distance function between an arbitrary point $(x,y)$ and the point $(1,0)$ is $d=\sqrt{(x-1)^2+(y-0)^2}$. On the other hand, when the point $(x,y)$ lies on the hyperbola we have $y^2= \frac{x^2 -16 }{4 }$. In this way, the problem becomes that of minimizing the distance function

\[
dist(x)=\sqrt{(x-1)^2+y^2}=\sqrt{(x-1)^2+\frac{x^2-16}{4}} \quad .
\]
This is a standard optimization problem: we need to find the critical endpoints, i.e., the points where $dist'=0$. As the square root function is an increasing function, the function $\displaystyle \sqrt{(x-1)^2+\frac{x^2-16}{4}}$ achieves its minimum when the function 
\[
l=dist^2=(x-1)^2+\frac{x^2-16}{4}\quad 
\]
does. $l$ is a quadratic function of $x$ and we can directly determine its minimimum via elementary methods. Alternatively, we find the critical points of $l$:
\[
\begin{array}{rcl}
\displaystyle l'&=&\displaystyle 0\\
\displaystyle 2(x-1)+\frac{x}{2} &=&\displaystyle 0\\
\displaystyle \frac{5}{2}x-2&=&0\\
\displaystyle x&=&\displaystyle \frac{4}{5}\quad .
\end{array}
\]
On the other hand, $x^2=16+4y^2$ and therefore $|x|\geq \sqrt{16} = 4$. Therefore $x\in (-\infty, -4]\cup [4,\infty)$. As $x= \frac{4}{5 }$ is outside of the allowed range, it follows that our either function attains its minimum at one of the endpoints $\pm 4$ or the function has no minimum at all. It is clear however that as $x$ tends to  $\infty$, so does $dist$. Therefore $dist$ attains its maximum for $x=4$ or $-4$ and $y=\pm\sqrt{(\pm4)^2-16}=0$. Direct check shows that $dist_{|x=4} =\sqrt{(4-1)^2 +\frac{4^2- 16}{4 }}=3$ and $dist_{|x=-4}=\sqrt{(-4-1)^2+\frac{4^2-16}{4}}=5$  so our function $dist$ has a minimal value of $3$ achieved when $x=4$, which is our final answer. Notice that this answer can be immediately given without computation by looking at the figure drawn for \ref{problemponthyperbolax^2-4y^2closestTo1,1}. Indeed, it is clear that there are no points from the hyperbola lying inside the dotted circle centered at $(1,0)$. Therefore the point where this circle touches the hyperbola must have the shortest distance to the center of the circle.
}

\begin{problem}
Use the Intermediate Value theorem and the Mean Value Theorem/Rolle's Theorem to prove that the function has \textbf{exactly one} real root.
\begin{enumerate}
\item \label{problemIVTandMVTx^3+4x+7} $f(x)=x^3+4x+7$.
\item $f(x)= x^3 +x^2+x+1$.
\item \label{problemIVTandMVTcos3xdiv3+sinx-3x} $f(x)=\cos^3 \left({\frac{x}{3}}\right) +\sin x-  3x$.
\end{enumerate}

\end{problem}
\textbf{Solution \ref{problemIVTandMVTx^3+4x+7}.}  $f(-2) = -9$ and $f(1) = 12$. Since $f(x)$ is continuous and has both negative and positive outputs, it must have a zero. In other words, for some $c$ between $-2$ and $1$, $f(c) = 0$. If there were solutions $x = a$ and $x = b$,  then we would have $f(a) = f(b)$, and Rolle's Theorem would guarantee that for some $x$-value, $f'(x) = 0$. However, $f'(x) = 3x^2 + 4$, which always positive and therefore is never 0. Therefore there cannot be 2 or more solutions. 

The above can be stated informally as follows. Note that $f'(x) = 3x^2 + 4$, which is always positive. Therefore, the graph of $f$ is increasing from left to right. So once the graph crosses the $x$-axis, it can never turn around and cross again, so there can only be a single zero (that is, a single solution to $f(x) = 0$).


\textbf{Solution \ref{problemIVTandMVTcos3xdiv3+sinx-3x}.} $f(5)= \cos^3 \left( \frac{5}{ 3} \right) +\sin 5-15 \leq 2-15=-13<0 $ (because $\cos a, \sin b\in [-1,1]$ for arbitrary $a, b$). Similarly $f(-5)=\cos^3\left(-\frac{5}{3}\right) +\sin (-5)+15 \geq 15-2>0$. Therefore by the Intermediate Value Theorem $f(x)=0$ has at least one solution in the interval $[-5,5]$.

Suppose on the contrary to what we are trying to prove, $f(x)=0$ has two or more solutions; call the first 2 solutions $a,b$. That means that $f(a)=f(b)=0$, so by the Mean value theorem, there exists a $c\in (a,b)$ such that $f'(c)=(f(a)-f(b))/(a-b)=(0-0)/(a-b)=0$. On the other hand we may compute:
\[ 
f'(x)=-3+\cos x-\cos^{2}\left(\frac{x}3\right)\sin\left(\frac{x}{3}\right) \leq -1<0,
\] 
where the first inequality follows from the fact that $\sin x,\cos x\in [-1,1]$. So we got that $f'(c)=0$ for some $c$ but at the same time $f'(x)<0$ for all $x$, which is a contradiction. Therefore $f(x)=0$ has exactly one solution. 


\begin{problem}
\begin{enumerate}[ref={\fcProblemRef}]
% Related Rates
\item \label{problemRelatedRatesFinddr/dtIfds/dtIsKnownS-sphere} A spherical soap bubble is slowly shrinking. If its surface area is decreasing at a rate of 50 square millimeters per
second, how quickly is the radius decreasing when the surface area is 1000 square millimeters?

\answer{$\frac{\diff r}{\diff t} = - \frac{ 50}{ 8 \pi \sqrt{\frac{250}{\pi}}} =- \frac{ \sqrt{10 } }{8 \sqrt{\pi}}$.}

\item \label{problemRelatedRatesCarAlongEllipticalTrack}A car drives along an elliptical track. The track can be modeled by the equation $x^2 + 5y^2 = 14$, where $x$ and $y$ are 
measured in kilometres of distance from the center of the track. As the car passes the point $(3, 1)$, the $x$-coordinate is 
increasing at a rate of $1.5$ km/min. How quickly is the $y$-coordinate changing at that point?

\answer{$\frac{\diff y}{\diff t} = -\frac{9}{10}$ km/min.}
\item \label{problemRelatedRatesGravelCone} Gravel is being dumped from a conveyor belt at a constant rate of $ 500 $ litres per minute. The gravel pile forms a cone with circular base, the diameter of which remains equal to the hight of the cone at any given moment. Use related rates to approximate how fast is the height of the pile increasing when it is 2 meters tall?
\answer{$\frac{\diff h}{\diff t} =  \frac{ 1}{ 2 \pi}$ m/min.} 
\end{enumerate}
\end{problem}
\solution{\ref{problemRelatedRatesFinddr/dtIfds/dtIsKnownS-sphere}
Let $t$ denote time. Let $R$ be the radius of the sphere. Let $S$ be the surface area of the sphere. We are given that 
\[
\frac{\diff S}{\diff t}=-50\quad ,
\] 
where the sign is negative because the bubble is decreasing. We are asked to find 
\[
\frac{\diff R}{\diff t} = ?
\]
$R$ and $S$ are related via
\[
\begin{array}{rcll|l}
\displaystyle S&=&\displaystyle 4\pi R^2 &&\text{differentiate with respect to time}\\
\displaystyle \frac{\diff S}{\diff t}&=&\displaystyle  4\pi *2R*\frac{\diff R}{\diff t}\\
\displaystyle \frac{\diff R}{\diff t}&=&\displaystyle \frac{1}{8\pi R }\frac{\diff S}{\diff t}\quad .
\end{array}
\]
When $S= 1000$, we can find the corresponding value of $R$:
\[
\begin{array}{rcl}
1000&=&\displaystyle  4\pi R^2\\
R^2&=&\displaystyle \frac{250}{\pi }\\
R&=&\displaystyle  \sqrt{\frac{250}{\pi }}\quad .
\end{array}
\]
Finally, we can compute:
\[
\frac{\diff R}{\diff t}_{|S=1000}= \frac{1}{8\pi \underbrace{\sqrt{\frac{250}{\pi }}}_{=R \text{ when }S=1000} }*(\underbrace{-50}_{=\frac{\diff S}{\diff t}})= -\frac{\sqrt{10\pi}}{ 8\pi}\quad.
\]
}

\solution{\ref{problemRelatedRatesCarAlongEllipticalTrack}. 
Let $t$ denote time, and let $t_0$ be the point in time in which the measurements take place. We are given that $\frac{\diff x}{\diff t}_{|t=t_0}=1.5 km/min$,  $x_{|t=t_0}=3$, $y_{|t=t_0}=1$. The problem asks us to find $\frac{\diff y}{\diff t}_{|t=t_0}$. 

Compute:
\[
\begin{array}{rcll|l}
\displaystyle x^2+5y^2&=&14&&\displaystyle \text{apply }\frac{\diff }{\diff t}\\
\displaystyle 2x \frac{\diff x}{\diff t}+10 y \frac{\diff y}{\diff t}&=&0 &&\displaystyle \text{fix time }t=t_0 \\
\displaystyle 2 \cdot 3 \cdot 1.5 +10 \cdot 1\cdot \frac{\diff y}{\diff t}_{|t=t_0}&=&0\\
\displaystyle 10\frac{\diff y}{\diff t}_{|t=t_0}&=&-9\\
\displaystyle \frac{\diff y}{\diff t}_{|t=t_0}&=&\displaystyle -\frac{9}{10}\quad .\\
\end{array}
\]
The measurement unit of $\diff y$ is $km$, the measurement unit of  $t$ is minutes, therefore the answer is $-\frac{9}{10}$km/min.
}


\solution{\ref{problemRelatedRatesGravelCone}. 
\begin{minipage}[t][][b]{0.6\textwidth}
At  time $ t $ (minutes) let $ h $ be the height of the pile (m), let $ r $ be the radius of the base (m), and let $ V $ be the volume of the pile (m$ ^3 $). \\
We are given that $ \frac{dV}{dt}=500 $L/min $ =\frac12 $ m$ ^3 $/min.
We are also given the diameter is equal to the height, so $ 2r=h$.\\ 
We are asked to find $ \frac{dh}{dt} $ when $ h=2$.\\
The formula for the volume of a cone is 
\[  
V=\frac13\pi r^2 h. 
\] 
Since $ r=\frac{h}{2} $ we obtain 
\[
V=\frac13 \pi \frac{h^2}{4}h = \frac{1}{12}\pi h^3
\]
Differentiating with respect to $ t $ then gives
\[
\frac{dV}{dt} = \frac14 \pi h^2 \frac{dh}{dt} \;\; \Rightarrow \;\;\frac{dh}{dt}=\frac{4}{h^2\pi}\frac{dV}{dt}
\]
Substituting $ \frac{dV}{dt}= \frac12 $m$ ^3$/min, and $ h=2 $ we obtain
\[
 \frac{dh}{dt}=\frac{4}{4\pi}\cdot\frac12=\frac{1}{2\pi}\textrm{ m/min}
\]
\end{minipage} 
\begin{minipage}[t][][c]{0.4\textwidth}
\begin{tikzpicture}[scale=1]
    \draw[dashed] (0,0) arc (170:10:2cm and 0.4cm)coordinate[pos=0] (a);
    \draw (0,0) arc (-170:-10:2cm and 0.4cm)coordinate (b);
    \draw[densely dashed] ([yshift=4cm]$(a)!0.5!(b)$) -- node[right,font=\footnotesize] {$h$}coordinate[pos=0.95] (aa)($(a)!0.5!(b)$)
                            -- node[above,font=\footnotesize] {$r$}coordinate[pos=0.1] (bb) (b);
    \draw (aa) -| (bb);
    \draw (a) -- ([yshift=4cm]$(a)!0.5!(b)$) -- (b);
  \end{tikzpicture}
  \end{minipage} 
} %  

\begin{problem}
\begin{enumerate}[ref={\fcProblemRef}]
% Area problems
\item 
\label{problemAreaBetweeny=2x^2,y=4+x^2} Find the area of the region bounded by the curves $y = 2x^2$ and $y = 4 + x^2$.

\answer{$\frac{32}{3}$}
\item \label{problemAreaBetweeny=2-x,x=4-y^2} Find the area of the region bounded by the curves $x = 4 - y^2$ and $y = 2 - x$.

\answer{$\frac92$}
\item \label{problemAreaBetweeny=x^2} Find the area of the region bounded by the curves $y=x^2$ and $y=2x^2+x-2$.

\answer{$\frac{9}{2}$}


\item \label{problemareabetweeny=x^2andy=2x^2+x-2}
\begin{itemize}
\item Sketch the region bounded by the curves $y=x^2$ and $y=2x^2+x-2$.

\psset{xunit=0.5cm, yunit=0.5cm}
\begin{pspicture}(-3.4,-3.6)(3,5.7)
\fcAxesStandardNoFrame{-3.5}{-3.5}{2.5}{5.5}
\fcGrid[linestyle=dashed, linewidth=0.5, linecolor=gray]{-3}{-3}{5}{8}{1}{1}{}
\rput[t](0.9,-0.2){$1$}
\fcLabels{3.5}{5.5}
%\psplot{-3}{2}{x x mul}
%\psplot{-3}{2}{x x mul 2 mul x -2 add add}
\end{pspicture}

\item Find the area of the region.

\answer{$\frac{9}{2}$}
\end{itemize}
\item \label{problemAreaBetween-x^2+2x-1and-2x^2+3x+1}
~
\begin{itemize}
\item Sketch the region bounded by the curves $y=- x^{2}+2 x-1$ and $y=-2 x^{2}+3 x+1$. Make sure to indicate the points where the curves intersect.

\psset{xunit=0.5cm, yunit=0.5cm}
\begin{pspicture}(-3.5,-8.8)(3.7,5.7)
\fcAxesStandard{-3.5}{-8.4}{3.5}{5.5}
\fcGrid[linestyle=dashed, linewidth=0.5, linecolor=gray]{-2}{-8}{5}{13}{1}{1}{}
\rput[t](0.9,-0.2){$1$}
\fcLabels{3.5}{5.5}
%\psplot[linecolor=\fcColorGraph]{-1.3}{2.7}{x x -1 mul mul 2 x mul -1 add add}
%\psplot[linecolor=\fcColorGraph]{-1.3}{2.7}{x x -2 mul mul 3 x mul 1 add add}
\end{pspicture}
\item Find the area of the region.
\end{itemize}
\end{enumerate}

\end{problem}
\input{../../modules/area-between-curves/homework/area-between-curves-problems-1-problem-2-solution}

\solution{\ref{problemareabetweeny=x^2andy=2x^2+x-2}

\textbf{Region plot.}

\psset{xunit=0.5cm, yunit=0.5cm}
\begin{pspicture}(-3.5,-3.5)(3,5.7)
\pscustom*[linecolor=\fcColorAreaUnderGraph]{%
\psplot{-2}{1}{x x mul}%
\psplot{1}{-2}{x x mul 2 mul x -2 add add}%
}%
\fcAxesStandardNoFrame{-3.5}{-3.5}{2.5}{5.5}
\psplot{-3}{2}{x x mul}
\psplot{-3}{2}{x x mul 2 mul x -2 add add}
\fcGrid[linestyle=dashed, linewidth=0.5, linecolor=gray]{-3}{-3}{5}{8}{1}{1}{}
\rput[t](0.9,-0.2){$1$}
\fcLabels{3.5}{5.5}
\end{pspicture}

The intersection between the two parabolas are found via
\[
\begin{array}{rcl}
x^2&=&2x^2+x-2\\
x^2+x-2&=&0\\
(x-1)(x+2)&=&0\\
x=1&& x=-2\\
y=1&&y=4.
\end{array}
\]

\textbf{Area of the region.} 
\[
\begin{array}{rcll|l}
A&=&\displaystyle\int_{1}^{-2}\left|x^2-(2x^2+x-2) \right|\diff x&&x^2>(2x^2+x-2) \text{ for }x\in [-2,1] \text{ (from plot)}\\
&=&\displaystyle\int_{1}^{-2}\left(x^2-(2x^2+x-2) \right)\diff x\\
&=&\displaystyle \left[-\frac{1}{3} x^{3}-\frac{1}{2} x^{2}+2 x \right]_{-2}^1\\
&=&\displaystyle \frac{9}{2}.
\end{array}
\]
}





\begin{problem}
\begin{enumerate}[ref={\fcProblemRef}]
% Volume problems
\item 
\label{problemVolumeRegionBoundedByy=2x^2-x+1,y=x^2+1rotatedAroundx=0} Consider the region bounded by the curves $y = 2x^2-x+1$ and $y =x^2+1$. What is the volume of the solid obtained by rotating this region about the line $x = 0$?

\answer{$\frac{2}{5}\pi$.} 
\item Consider the region bounded by the curves $y = 1-x^2$ and $y =0$. What is the volume of the solid obtained by
rotating this region about the line $y = 0$?

\answer{$\frac{16 \pi}{15}$}
 
\item Consider the region bounded by the curves $y = x^2$ and $x = y^2$. What is the volume of the solid obtained by
rotating this region about the line $x = 2$?

\answer{ $\frac{31 \pi}{30}$}
\item \label{problemVolumeAreay=-x^2+2andy=0rotatedAroundy=0andy=-3}
Set up \textsc{but do not evaluate} an integral to calculate the volume of the solid obtained by rotating the region bounded by $y=-x^2+2$ and $y=0$ about the given line. 

\begin{itemize}
\item The $x$ axis.
\item The line $y=-3$.
\end{itemize}

\item \label{problemVolumeRevolution-x^2+1aroundy=0andy=-4}
Set up \textsc{but do not evaluate} an integral to calculate the volume of the solid obtained by rotating the region bounded by $y=-x^2+1$ and $y=0$ about the given line. 
\begin{itemize}
\item The $x$ axis.
\item The line $y=-4$.
\end{itemize}


\end{enumerate}

\end{problem}

\begin{problem}
State the Fundamental Theorem of Calculus (both parts).
\end{problem}
\end{document}
