\documentclass{article}
\ProvidesPackage{homework-problems-UMB}
\addtolength{\hoffset}{-3.5cm}
\addtolength{\textwidth}{6.8cm}
\addtolength{\voffset}{-3cm}
\addtolength{\textheight}{6cm}
\usepackage{../homework-problems} %warning folder paths are relative to the file that uses the includepackage

\renewcommand{\answer}[1]{\iftoggle{answers}{ \hfill{~} \rotatebox{180}{\tiny answer: #1}}{} }
\renewcommand{\pointsii}[1]{}
\renewcommand{\hiddenanswer}{\answer}
\renewcommand{\points}[1]{\item}
\renewcommand{\pointsii}[1]{\item}
\renewcommand{\Arctan}{\arctan}
\renewcommand{\Arcsin}{\arcsin}
\renewcommand{\Arccot}{\operatorname{arccot}}

\toggletrue{solutions}
%\togglefalse{solutions}
\toggletrue{answers}
\newtheorem{problem}{Problem}

\newcommand{\hide}[1]{}

\begin{document}
\begin{center}
\Large
Master Problem Sheet \\ Math 140 Calculus I \\ 
\end{center}

%\noindent \textbf{Name:} \hfill{~}
%\begin{tabular}{c|c|c|c|c|c|c|c|c||c}
%Problem&1 &2&3&4&5&6&7&8& $\sum$\\ \hline
%Score&&&&&&&&&\\ \hline
%Max&20&20&20&20&20&10&20&20&150
%\end{tabular}




This master problem sheet contains all freecalc problems on the topics studied in Calculus II. The latest \LaTeX{} source of this file (complete with typo and error fixes) can be downloaded from the freecalc project page below. 

\url{https://sourceforge.net/p/freecalculus/code/HEAD/tree/}

A list of contributors/authors of the freecalc project (and in particular, the present problem collection) see the following file.
\url{https://sourceforge.net/p/freecalculus/code/HEAD/tree/trunk/contributors.tex}

\section{Functions, Basic Facts}
\begin{problem}
(Stewart, 7ed., page 21, problems 27-30)
Evaluate the difference and simplify your answer.
\begin{multicols}{2}
\begin{enumerate}
\item $\frac{f(3+h)-f(3)}{h}$, where $f(x)=4+3x-x^2$.
\answer{$-3-h$}
\item $\frac{f(a+h)-f(a)}{h}$, where $f(x)= x^3$.
\answer{$ 3a^2+3ah+h^2$}
\item $\frac{f(x)-f(a)}{x-a}$, where $f(x)=\frac{1}{x}$.
\answer{$-\frac{1}{ax}$.}
\item $\frac{f(x)-f(1)}{x-1}$, where $f(x)=\frac{x+3}{x+1}$.
\answer{$-\frac{1}{x+1}$.}
\end{enumerate}
\end{multicols}

\end{problem}
\subsection{Function composition}
\begin{problem}
Compute the composite functions $(f\circ g)(x)$, $(g\circ f)(x)$. Simplify your answer to a single fraction. Find the domain of the composite function. The answer key has not been fully proofread, use with caution. 

\begin{enumerate}[ref={\fcProblemRef}]
\item $\displaystyle f{}({{x}})=\frac{x+2}{x-2},
g{}({{x}})=\frac{x-1}{x+2}$.

\answer{
\begin{tabular}{rl}
$(f\circ g)(x)= \frac{3+3 x}{-5- x}$& $x\neq -2, -5$\\ 
$(g\circ f)(x)=\frac{4}{-2+3 x}$& $x\neq 2, \frac{2}{3}$ 
\end{tabular}
}
\item 
$\displaystyle f{}({{x}})=\frac{x+1}{3x-2}, g{}({{x}})= \frac{x-2}{x-1}
$.

\answer{
\begin{tabular}{rl}
$(f\circ g)(x)= \frac{-3+2 x}{-4+x}
$ & $x\neq 4, 1$ \\
$(g\circ f)(x)=\frac{5-5 x}{3-2 x}$
& $x\neq \frac{2}{3}, \frac{3}{2}$
\end{tabular}
}
\item 
$\displaystyle f{}({{x}})=\frac{2x+1}{3x-1},
g{}({{x}})=\frac{x-2}{2x-1}
$.

\answer{
\begin{tabular}{rl}
$(f\circ g)(x)=\frac{-5+4 x}{-5+x}
$ &$x\neq 5, \frac{1}{2}$ \\
$(g\circ f)(x)=\frac{3-4 x}{3+x}
$ &$x\neq -3, \frac{1}{3}$
\end{tabular}
}
\item 
$\displaystyle f{}({{x}})=\frac{x+1}{x-2},
g{}({{x}})=\frac{x+2}{2x-1}
$.

\answer{
\begin{tabular}{rl}
$(f\circ g)(x)= \frac{1+3 x}{4-3 x}
$&$x\neq \frac{4}{3}, \frac{1}{2}$\\ 
$(g\circ f)(x)=\frac{-3+3 x}{4+x}
$&$x\neq -4, 2$
\end{tabular}
}
\item 
$\displaystyle f{}({{x}})=\frac{5x+1}{4x-1},
g{}({{x}})=\frac{4x-1}{3x+1}
$.

\answer{
\begin{tabular}{rl}
$(f\circ g)(x)= \frac{-4+23 x}{-5+13 x}
$&$x\neq \frac{5}{13}, -\frac{1}{3 }$\\
$(g\circ f)(x)=\frac{5+16 x}{2+19 x}
$&$x\neq -\frac{2}{19}, \frac{ 1}{4}$
\end{tabular}
}
\item $\displaystyle  f(x)= \frac{3x-5}{x-2}$, $\displaystyle g(y)=\frac{y-2 }{y-4} $. 

\answer{ 
\begin{tabular}{rl}
$(f\circ g)(x)=\frac{-2 x+14}{- x+6}$&$x\neq 6, 4$\\
$(g\circ f)(x)=\frac{x-1}{- x+3}$&$x\neq 3,2$
\end{tabular}
}
\item $\displaystyle  f(x)= \frac{x-3}{x+2}$, $\displaystyle g(y)=\frac{y+3 }{y-4} $. 

\answer{ 
\begin{tabular}{rl}
$(f\circ g)(x)=\frac{-2x+15}{3 x-5}$&$x\neq \frac{5}{3}, 4$\\ 
$(g\circ f)(x)=\frac{4 x+3}{-3 x-11}$&$x\neq -\frac{11}{3}, -2$
\end{tabular}
}
\end{enumerate}

\end{problem}
\begin{problem}
Find the expressions $(f\circ g)(x)$ and $(g\circ f)(x)$. Simplify your answer to a single fraction. 
\begin{enumerate}
\item $\displaystyle  f(x)= \frac{3x-5}{x-2}$, $\displaystyle g(y)=\frac{y-2 }{y-4} $. \answer{ $(f\circ g)(x)=\frac{-2 x+14}{- x+6}$, $(g\circ f)(x)=\frac{x-1}{- x+3}$}
\item $\displaystyle  f(x)= \frac{x-3}{x+2}$, $\displaystyle g(y)=\frac{y+3 }{y-4} $. \answer{ $(f\circ g)(x)=\frac{-2 (x)+15}{3 x-5}$, $(g\circ f)(x)=\frac{4 x+3}{-3 x-11}$}
\end{enumerate}

\end{problem}
\subsection{Domains and ranges}
\begin{problem}
Find the implied domain of the function.
\begin{multicols}{2}
\begin{enumerate}[ref={\fcProblemRef}]
\item $\displaystyle f(x)=\frac{x+4}{x^2-4}$. 

\answer{\begin{tabular}{l} $x\neq \pm 2$, \\alternatively:\\ $x\in (-\infty, -2)\cup (-2,2)\cup (2,\infty)$\end{tabular} }
\item $\displaystyle f(x)=\frac{2x^3-5}{x^2+5x+6}$.

\answer{\begin{tabular}{l} $x\neq -2,-3$, \\alternatively:\\ $x\in (-\infty, -3)\cup (-3,-2)\cup (-2,\infty)$\end{tabular} }
\item $\displaystyle f(t)=\sqrt[3]{3t-1}$.

\answer{$x\in \mathbb R$ (the domain is all real numbers) }
\item $\displaystyle g(t)=\sqrt{5-t}-\sqrt{1+t}$.

\answer{$x\in [-1,5]$.}
\item $\displaystyle h(x)=\frac{1}{\sqrt[6]{x^2-7x}}$.

\answer{$x\in (-\infty, 0)\cup (7,\infty)$.}
\item $f\displaystyle (u)=\frac{u+1}{1+\frac{1}{u+1}}$.

\answer{
\begin{tabular}{l}
$u\neq -1, -2$ or \\
$u\in (-\infty, -2)\cup (-2, -1)\cup (-1,\infty).$.
\end{tabular}
}
\item $\displaystyle F(x)=\sqrt{10-{\sqrt{x}}}$.

\answer{$x\in [0,100]$}
\end{enumerate}
\end{multicols}
\end{problem}
\begin{problem}
Find the functions $f\circ g$, $g\circ f$, $f\circ f$ and $g\circ g$ and their implied domains.

\begin{enumerate}
\item $f(x)=x^2+1$, $g(x)=x+1$. 
\answer{\begin{tabular}{l} Domain, all 4 cases: $x\in \mathbb R$ (all reals)\\ in some order: $(1+x)^{2}+1, (x)^{2}+2, ((x)^{2}+1)^{2}+1, 2+x$\end{tabular} }
\item $f(x)=\sqrt{x+1}$, $g(x)=x+1$. 
\answer{\begin{tabular}{l} Domain of $f\circ g$ is $x\geq -2$. Domain of $g\circ f$ is $x\geq -1$ \\Domain of $f\circ f$ is $ x\geq -1$. Domain of $g\circ g$ is all reals ($x\in \mathbb R$). \\ in some order:$\sqrt{2+x}, 1+\sqrt{1+x}, \sqrt{1+\sqrt{1+x}}, 2+x$\end{tabular}}
\item $f(x)= 2x$, $g(x)= \tan x$. 

You are not required to find the domain.
\answer{\begin{tabular}{l}
Domain  $f\circ f$: all reals ($x\in \mathbb R$). Domain $g\circ f$: $x\neq (2k+1)\frac{\pi}{2}$ for all $k\in \mathbb Z$\\ Domain $ f\circ g$: $x\neq (4k+1)\frac{\pi}{4}$, $x\neq (4k+3)\frac{\pi}{4}$ for all $k\in \mathbb Z$\\
Domain $g\circ g$: $x\neq (2k+1)\frac{\pi}{2}$ and $x\neq k\pi+ \arctan \left(\frac{\pi}{2}\right)$ for all $k\in \mathbb Z$
\\
in some order:$2 \tan{}x, \tan{}(2 x), 4 x, \tan{}(\tan{}x) $
\end{tabular}
}
\item $f(x)=\frac{x+1}{x-1}$, $g(x)=\frac{x-1}{x+1}$.
\answer{ 
\begin{tabular}{l}
Domain $ f\circ f$: $x\neq 1$. Domain $g\circ g$: $x\neq 0$, $x\neq -1$\\
Domain $f\circ g$: $x \neq -1$. Domain $g\circ f$: $x\neq 0$, $x\neq 1$\\
in some order: $- x, \frac{1}{x}, x, -\frac{1}{x} $
\end{tabular}
}
\end{enumerate}

\end{problem}
\subsection{Linear Transformations and Graphs of Functions}
\begin{problem}
\begin{problem}Graph the functions by hand, by applying consecutively the transformations learned in class.
\begin{multicols}{2}
\begin{enumerate}
\item $y=\frac{1}{x}$.
\item $y=\frac{1}{x+1}$.
\item $y=\frac{1}{2x+1}$.
\item $y=\frac{3}{2x+1}$.
\item $y=\frac{3+x}{2x+1}$.
\item $y=\left|\frac{3+x}{2x+1}\right|$.
\end{enumerate}
\end{multicols}
\end{problem}
\end{problem}
\subsection{Piecewise Defined Functions}
\begin{problem}
(Stewart, 7ed, page 21, problems 55-56)
Write down formulas for function whose graphs are as follows. The graphs are up to scale. The arc is a part of a circle.
\begin{multicols}{2}
\begin{enumerate}
\psset{xunit=0.4cm, yunit=0.4cm}
\item
\tiny
\begin{pspicture}(-1,-1)(6,5)
\psaxes{->}(0,0)(-1,-1)(6,5)
\psline[linecolor=red](0,3)(3, 0)(5, 4)
\fcFullDot{5}{4}
\rput[r](4.9, 4){$(5, 4)$}
\rput[b](6,0.1 ){$x$}
\rput[l](0.1,5 ){$y$}
\end{pspicture}
\normalsize
\item
\tiny
\psset{xunit=0.4cm, yunit=0.4cm}
\begin{pspicture}(-4,-1)(4,5)
\psaxes{->}(0,0)(-4.5,-1)(4.5,4)
\psplot[linecolor=red]{-2}{2}{4 x x mul sub sqrt }
\psline[linecolor=red](2,0)(4,3)
\psline[linecolor=red](-2,0)(-4,3)
\rput[b](4.5,0.1 ){$x$}
\rput[l](0.1,5 ){$y$}
\fcFullDot{4}{3}
\rput[r](3.9, 3){$(4, 3)$}
\fcFullDot{-4}{3}
\rput[l](-3.9, 3){$(-4, 3)$}

\end{pspicture}
\normalsize
\end{enumerate}
\end{multicols}

\end{problem}
\begin{problem}
(Stewart, 7ed., page 21, problems 45, 46, 49)
Plot the piecewise defined functions.
\begin{multicols}{2}
\begin{enumerate}
\item $G(x)=\frac{3x+|x|}x$.
\item $g(x)=|x|-x$.
\item $f(x)=\doublebrace{x+2}{x\leq -1}{x^2}{x\geq -1}$.
\end{enumerate}
\end{multicols}

\end{problem}

\section{Trigonometry}
\subsection{Angle conversion}
\begin{problem}
Convert from degrees to radians.
\begin{multicols}{3}
\begin{enumerate}
\item $15^\circ$.
\item $30^\circ$.
\item $36^\circ$.
\item $45^\circ$.
\item $60^\circ$.
\item $75^\circ$.
\item $90^\circ$.
\item $120^\circ$.
\item $135^\circ$.
\item $150^\circ$.
\item $180^\circ$.
\item $225^\circ$.
\item $270^\circ$.
\item $305^\circ$.
\item $360^\circ$.
\item $405^\circ$.
\item $1200^\circ$.
\item $-900^\circ$.
\item $-2014^\circ$.
\end{enumerate}
\end{multicols}

\end{problem}
\begin{problem}
(Textbook page A32, problems 7-12). 
Convert from radians to degrees.
\begin{multicols}{3}
\begin{enumerate}
\item $4\pi$.
\item $-7/2\pi$.
\item $5/12\pi$.
\item $8/3\pi$.
\item $-3/8\pi$.
\item $5$.
\end{enumerate}
\end{multicols}
\end{problem}
\subsection{Trigonometry identities}
\begin{problem}
\begin{problem}
(Textbook page A32-, problems 45, 46, 47, 48, 49, 50, 51, 52, 56, 57, 58).
\begin{multicols}{3}
\begin{enumerate}
\item $\sin \theta\cot \theta =\cos \theta$.
\item $(\sin x +\cos x)^2=1+\sin(2x)$.
\item $\sec y - \cos y= \tan y \sin y$.
\item $\tan^2 \alpha-\sin^2 \alpha=\tan^2\alpha\sin^2\alpha$.
\item $\cot^2\theta+\sec^2\theta=\tan^2\theta+\csc^2\theta$.
\item $2\csc 2t= \sec t \csc t$.
\item $\tan 2\theta =\frac{2\tan \theta}{1-\tan^2\theta} $.
\item $\frac{1}{1-\sin \theta}+ \frac{1}{1+\sin \theta}=2\sec^2\theta$.
\item $\tan x + \tan y = \frac{\sin (x+y)}{\cos x \cos y}$.
\item $\sin 3\theta +\sin \theta = 2 \sin 2\theta \cos \theta $.
\item $\cos 3\theta = 4\cos^3\theta-3\cos \theta $.
\end{enumerate} 
\end{multicols}
\end{problem}
\end{problem}

\subsection{Trigonometry equations}
\begin{problem}
\begin{problem}(Textbook page A33, problems 65-72).
Find all values of $x$ in the interval $[0,2\pi]$ that satisfy the 
equation.
\begin{multicols}{3}
\begin{enumerate}
\item $2\cos x - 1=0$.
\item $3\cot^2 x= 1$.
\item $2\sin^2 x= 1$.
\item $|\tan x|=1 $.
\item $\sin 2x = \cos x $.
\item $2\cos x +\sin 2x=0$.
\item $\sin x =\tan x$.
\item $2+\cos 2x = 3 \cos x$.
\end{enumerate}
\end{multicols}
\end{problem}

\end{problem}
\solution{ \ref{problemSolve2cos^2x-(1+sqrt(2))cosx+sqrt(2)/2=0}
Set $\cos x=u$. Then 
\[
2\cos^2 x- (1+\sqrt{2})\cos x+\frac{\sqrt 2}2=0 
\] 
becomes 
\[2u^2-(1+\sqrt{2})u+\frac{\sqrt{2}}{2}=0.
\] 
This is a quadratic equation in $u$ and therefore has solutions
\[
\begin{array}{rcl}
u_1, u_2\displaystyle &=& \displaystyle \frac{ 1+\sqrt{2}\pm\sqrt{ (1+\sqrt{2})^2-4 \sqrt{2} } }4\\
&=&\displaystyle \frac{1+\sqrt{2}\pm\sqrt{1-2\sqrt{2}+2} }4\\
&=&\displaystyle \frac{1+\sqrt{2}\pm \sqrt{(1-\sqrt{2})^2}}4\\
&=&\displaystyle \frac{1+\sqrt{2}\pm (1-\sqrt{2}) }4=\doublebrace{\frac{1}2 }{ \mathrm{or}} {\frac{\sqrt{2}}{2}}{}
\end{array}
\]
Therefore $u=\cos x= \frac12$ or $u=\cos x=\frac{\sqrt{2}}2$, and, as $x$ is in the interval $[0,2\pi]$, we get $x=\frac{\pi}{3}, \frac{5\pi}{3}$ (for $\cos x=\frac12$) or $x=\frac{\pi }4 ,\frac{7\pi}4$ (for $\cos x=\frac{\sqrt{2}}{2}$).
} % end solution

\section{Limits}
\begin{problem}
%(Problem contributed by Gabe Cunningham) 
Find the following limits, or show that they do not exist:
\begin{multicols}{2}
\begin{enumerate}
\item ${\displaystyle \lim_{x \to 2} \frac{x^2-4}{x^2-x-2}}$
\answer{$\frac43$}
\item ${\displaystyle \lim_{x \to -\infty} \frac{5x^3+x-1}{2x^3-7}}$
\answer{$\frac{5}{2}$}
\item ${\displaystyle \lim_{x \to 1^{+}} \frac{x-3}{x-1}}$
\answer{$-\infty$}
\item ${\displaystyle \lim_{h \to 0} \frac{2(x+h)^3 - 2x^3}{h}}$
\answer{$6x^2$}
\item ${\displaystyle \lim_{x \to \infty} \frac{\sqrt{9x^2-2}}{x+4}}$
\answer{$3$}
\item ${\displaystyle \lim_{x \to -1} \frac{2x+3}{x+1}}$
\answer{Does not exist}
\end{enumerate}
\end{multicols}

\end{problem}
\begin{problem}
Evaluate the limit if it exists.
\begin{multicols}{2}
\begin{enumerate}[ref={\fcProblemRef}]
\item \label{problemlim(xto2)(x^2-5x+6)/(x-2)}
$\displaystyle\lim\limits_{x\to 2}\frac{x^2-5x+6}{x-2} $. 

\answer{$-1$}
\item $\displaystyle\lim\limits_{x\to 3}\frac{x^2-3x}{x^2-2x-3} $.

\answer{$\frac{3}{4}$}
\item \label{problemlimxto-2(2x^2+x-6)/(x^2-4)}

$\displaystyle \lim_{x\to -2} \frac{2x^2+x-6}{x^2-4}$
\answer{$\frac{7}{4}$}
\item $\displaystyle\lim\limits_{x\to 2}\frac{x^2-5x-6}{x-2} $.

\answer{DNE}
\item $\displaystyle\lim\limits_{x\to -1}\frac{x^2-3x}{x^{2}-2x-3} $.

\answer{DNE}

\item $\displaystyle\lim\limits_{x\to -2}\frac{x^2-4}{2x^2+5x+2} $.

\answer{$\frac{4}{3}$}

\item $\displaystyle\lim\limits_{x\to -1}\frac{2x^2+3x+1}{3x^2-2x-5} $.

\answer{$\frac{1}{8}$}

\item \label{problemlimxto-4(x^2+7x+12)/(x^2+6x+8)}
$\displaystyle \lim_{x \to -4}\frac{x^{2}+7 x+12}{x^{2}+6 x+8}$.

\answer{$\frac{1}{2}$}


\item $\displaystyle\lim\limits_{h\to 0}\frac{(-3+h)^2-9}{h} $.

\answer{$-6$}

\item $\displaystyle\lim\limits_{h\to 0}\frac{(-2+h)^3+8}{h} $.

\answer{$12$}
\item $\displaystyle\lim\limits_{x\to -3}\frac{x+3}{x^3+27} $.

\answer{$\frac{1}{27}$}

\item $\displaystyle\lim\limits_{x\to 1}\frac{x^4-1}{x^3-1} $.

\answer{$\frac{4}{3}$}
\item $\displaystyle\lim\limits_{h\to 0}\frac{\sqrt{4+h}-2}{h} $.

\answer{$\frac{1}{4}$}
\item $\displaystyle\lim\limits_{x\to 3} \frac{\sqrt{5x+1}-4}{x-3}$.

\answer{$\frac{5}{8}$}

\item $\displaystyle\lim\limits_{x\to -3} \frac{\sqrt{x^2+16}-5}{x+3}$.

\answer{$-\frac{3}{5}$}

\item $\displaystyle\lim\limits_{x\to -3} \frac{\frac{1}{3}+ \frac{1}{x}} {3+x}$.

\answer{$-\frac{1}{9}$}
\item $\displaystyle\lim\limits_{x\to -2} \frac{x^2+4x+4}{x^4-16}$.

\answer{$0$}
\item $\displaystyle\lim\limits_{x\to 0} \frac{\sqrt{1+x}- \sqrt{1-x}}{x}$.

\answer{$1$}
\item $\displaystyle\lim\limits_{x\to 0}\left(\frac{1}x -\frac{1}{x^2+x}\right)$.

\answer{$1$}
\item $\displaystyle\lim\limits_{x\to 9} \frac{3-\sqrt{x}}{9x-x^2}$.

\answer{$\frac{1}{54}$}
\item $\displaystyle\lim\limits_{h \to 0}\frac{(2+h)^{-1}-2^{-1}}{h} $.

\answer{$-\frac{1}{4}$}

\item $\displaystyle\lim\limits_{x\to 0} \left(\frac{1}{x\sqrt{1+x}}-\frac{1}{x} \right)$.

\answer{$-\frac{1}{2}$}

\item 
$\displaystyle\lim\limits_{h\to 0}\frac{(x+h)^3-x^3}{h} $.

\answer{$3x^2$}

\item 
\label{problemlim_hto0_(1/(x+h)^2-1/x^2)/h} $\displaystyle\lim\limits_{h\to 0}\frac{\frac{1}{(x+h)^2}-\frac{1}{x^2}}{h} $.

\answer{$-\frac{2}{x^3}$}

\item 
\label{problemlimhto0(1/(2+h)^2-1/4)/h}
$ \displaystyle \lim_{h\to 0} \frac{\frac{1}{(2+h)^2}-\frac{1}{4}}{h}  $.

\answer{$-\frac{1}{4}$}
\item 
\label{problemlimhto0(1/(1+h)^2-1)/h}
$ \displaystyle \lim_{h\to 0} \frac{\frac{1}{(1+h)^2}-1}{h}  $.

\answer{$-2$}
\end{enumerate}
\end{multicols}

\end{problem}
\solution{\ref{problemlim(xto2)(x^2-5x+6)/(x-2)}

$
\begin{array}{rcll|l}
\displaystyle 
\displaystyle \lim\limits_{x\to 2}\frac{x^2-5x+6}{x-2} &=&\displaystyle \lim\limits_{x\to 2}\frac{(x-3)\cancel{(x-2)}}{\cancel{x-2}} &&\text{factor and cancel}\\
&=&\displaystyle 2-3=-1
\end{array}
$
}
\solution{\ref{problemlimxto-2(2x^2+x-6)/(x^2-4)}

$\begin{array}{rcll|l}
\displaystyle \lim\limits_{x\to -2} \frac{2x^2+x-6}{x^2-4}&=&  \displaystyle \lim\limits_{x\to -2}\frac{ (2x -3)\cancel{( x+ 2 ) }}{ (x-2)\cancel{(x+2)}} &&\text{factor and cancel}\\ 
&=&\displaystyle  \frac{(2(-2)-3)}{-2-2} &&\text{substitute}\\
&=&\displaystyle \frac{7}{4}
\end{array}
$

}
\solution{\ref{limproblem(xto-2)(x^2-4)/(2x^2+5x+2)}

$
\begin{array}{rcll|l}
\displaystyle 
\displaystyle \lim\limits_{x\to 2}\frac{x^2-4}{2x^2+5x+2} &=&\displaystyle \lim\limits_{x\to -2} \frac{(x-2)\cancel{(x+2)}}{(2x+1) \cancel{(x+2)}} &&\text{factor and cancel}\\
&=&\displaystyle \frac{(-2)-2}{2(-2)+1}=\frac{4}{3}.
\end{array}
$
}
\solution{
\ref{problemlim(xto-1)(2x^2+3x+1)/(3x^2-2x-5)}

$
\begin{array}{rcll|l}
\displaystyle \lim\limits_{x\to-1}\frac{2x^2+3x+1}{3x^2-2x-5} &=&\displaystyle \lim\limits_{x\to -1}\frac{(2x+1)\cancel{(x+1)}}{(3x-5)\cancel{(x+1)}}&&\text{factor and cancel}\\
&=&\displaystyle \frac{2(-1)+1}{3(-1)-5} =\frac{1}{8} 
\end{array}
$
}
\solution{\ref{problemlimxto-4(x^2+7x+12)/(x^2+6x+8)}.

$\begin{array}{rcll|l}
\displaystyle \lim_{x \to -4}\frac{x^{2}+7 x+12}{x^{2}+6 x+8}&=& \displaystyle \lim_{x \to -4}\frac{(x+3)(\cancel{x+4})}{(x+2)(\cancel{x+4})} &&\text{factor}\\
&=&\displaystyle \frac{-4+3}{-4+2}=-\frac{1}{2}.
\end{array}
$

}
\solution{ \ref{problemlim_hto0_(1/(x+h)^2-1/x^2)/h}

$
\begin{array}{rcl}
\displaystyle \lim\limits_{h\to 0}\frac{\frac{1}{(x+h)^2}-\frac{1}{x^2}}{h} &=&\displaystyle \lim\limits_{h\to 0}\frac{x^2-(x+h)^2}{hx^2(x+h)^2}=\lim\limits_{h\to 0} \frac{x^2-(x^2+2xh+h^2)}{hx^2(x+h)^2}\\
&=&\displaystyle \lim\limits_{h\to 0}\frac{\cancel{h}(-2x+h)}{\cancel{h}x^2(x+h)^2}= \frac{-2x+0}{x^2(x+0)^2}=-\frac{2}{x^3}.
\end{array}
$
}

\solution{\ref{problemlimhto0(1/(2+h)^2-1/4)/h}.

\textbf{Variant I.}

$\begin{array}{rcll|l}
\displaystyle \lim_{h\to 0} \frac{\frac{1}{(2+h)^2}-\frac{1}{4}}{h}&=&\displaystyle \lim_{h\to 0}\frac{\frac{4-(2+h)^2}{4(2+h)^2}}{h}\\
&=&\displaystyle \lim_{h\to 0} \frac{4- (4+4h+h^2)}{4h(2+h)^2}\\
&=&\displaystyle \lim_{h\to 0} \frac{-4h-h^2}{4h(2+h)^2}\\
&=&\displaystyle \lim_{h\to 0} \frac{\cancel{h}(-4-h) }{4\cancel{h}(2+h)^2}&&\text{substitute }h=0\\
&=&\displaystyle \frac{-4-0}{4(2+0)^2}\\
&=&\displaystyle -\frac{1}{4}
\end{array}
$

\textbf{Variant II.}

$\begin{array}{rcll|l}
\displaystyle \lim_{h\to 0} \frac{\frac{1}{(2+h)^2}-\frac{1}{4}}{h}&=&\displaystyle \frac{\diff }{\diff x}\left(\frac{1}{x^2}\right)_{|x=2}\\
&=&\displaystyle \left(\frac{-2}{x^3}\right)_{|x=2}\\
&=&\displaystyle -\frac{1}{4}
\end{array}
$

}


\solution{\ref{problemlimhto0(1/(1+h)^2-1)/h}.

\textbf{Variant I.}

$\begin{array}{rcll|l}
\displaystyle \lim_{h\to 0} \frac{\frac{1}{(1+h)^2}-1}{h}&=&\displaystyle \lim_{h\to 0}\frac{\frac{1-(1+h)^2}{ (1+h)^2}}{h}\\
&=&\displaystyle \lim_{h\to 0} \frac{1- (1+2h+h^2)}{h(1+h)^2}\\
&=&\displaystyle \lim_{h\to 0} \frac{-2h-h^2}{h(1+h)^2}\\
&=&\displaystyle \lim_{h\to 0} \frac{\cancel{h}(-2-h) }{\cancel{h}(1+h)^2}&&\text{substitute }h=0\\
&=&\displaystyle \frac{-2-0}{(1+0)^2}\\
&=&\displaystyle -2.
\end{array}
$

\textbf{Variant II.}

$\begin{array}{rcll|l}
\displaystyle \lim_{h\to 0} \frac{\frac{1}{(1+h)^2}-1}{h}&=&\displaystyle \frac{\diff }{\diff x}\left(\frac{1}{x^2}\right)_{|x=1}&&\text{derivative definition}\\
&=&\displaystyle \left(\frac{-2}{x^3}\right)_{|x=1}\\
&=&\displaystyle -2.
\end{array}
$

}
\begin{problem}
Evaluate the limit if it exists.
\begin{enumerate}
\item $\displaystyle \lim\limits_{x\to 1} \frac{3x^2+4x-7}{x^3-x}$ \answer{$5 $.}
\item $\displaystyle \lim\limits_{x\to -1} \frac{2x^2-3x-5}{x^3+1}$ \answer{$ -\frac{7}{3}$.}
\end{enumerate}

\end{problem}
\begin{problem}
(Textbook page 69, problems 3-9). 
Evaluate the limits. Justify your computations.
\begin{multicols}{3}
\begin{enumerate}
\item $\displaystyle\lim\limits_{x\to 3} 5x^3-3x^2+x-6$.
\answer{$105$}
\item $\displaystyle\lim\limits_{x\to -1} (x^4-3x)(x^2+5x+3)$.
\answer{$-4$}
\item $\displaystyle\lim\limits_{t\to -2} \frac{t^4- 2}{2t^2 -3t +2} $.
\answer{$\frac78$}
\item $\displaystyle\lim\limits_{u\to -2}\sqrt{u^4+3u +6}$.
\answer{$4$}
\item $\displaystyle\lim\limits_{x \to 8}(1+ \sqrt[3]{x} )(2- 6x^2 + x^3)$.
\answer{$390$}
\item $\displaystyle\lim\limits_{t \to 2}\left( \frac{t^2 - 2 }{ t^3-3t+5} \right)^2$.
\answer{$\frac{4}{49}$}
\item $\displaystyle\lim\limits_{x\to 2}\sqrt{ \frac{2x^2 + 1}{ 3x-2}}$.
\answer{$\frac32$}
\end{enumerate}
\end{multicols}

\end{problem}
\begin{problem}
Find the limit or show that it does not exist. If the limit does not exist, indicate whether it is $\pm\infty$, or neither. The answer key has not been proofread, use with caution.
\begin{multicols}{3}
\begin{enumerate}[ref={\fcProblemRef}]
\item $\displaystyle \lim\limits_{x\to\infty }\frac{x-2}{2x+1}$.

\answer{$\frac12$}
\item $\displaystyle \lim\limits_{x\to\infty }\frac{1-x^2}{x^3-x-1}$.

\answer{$ 0$}
\item $\displaystyle \lim\limits_{x\to-\infty }\frac{x-2}{x^2+5}$.

\answer{$ 0$}
\item \label{problemlimxtominusinfty(3x^3+2)/(2x^3-4x+5)} $\displaystyle \lim\limits_{x\to-\infty }\frac{3x^3+2}{2x^3-4x+5}$.

\answer{$ \frac{3}{2}$}
\item $\displaystyle \lim\limits_{x\to\infty }\frac{\sqrt{x}+x^2}{\sqrt{x}-x^2}$.

\answer{$-1$}
\item $\displaystyle \lim\limits_{x\to\infty }\frac{3-x\sqrt{x}}{2x^{\frac{3}{2}}-2}$.

\answer{$-\frac12$}
\item $\displaystyle \lim\limits_{x\to\infty }\frac{(2x^2+3)^2}{(x-1)^2(x^2+1)}$.

\answer{$ 4$}
\item $\displaystyle \lim\limits_{x\to\infty }\frac{x^2-3}{\sqrt{x^4+3}}$.

\answer{$1$}
\item $\displaystyle \lim\limits_{x\to\infty }\frac{\sqrt{16x^6-3x}}{x^3+2}$.

\answer{$4$}
\item $\displaystyle \lim\limits_{x\to-\infty }\frac{\sqrt{16x^6-3x}}{x^3+2}$.

\answer{$-4$}
\item $\displaystyle \lim\limits_{x\to\infty}\sqrt{4x^2+x}-2x$.

\answer{$\frac{1}{4}$}
\item $\displaystyle \lim\limits_{x\to-\infty} x+\sqrt{x^2+3x} $.

\answer{$-\frac{3}{2} $}
\item $\displaystyle \lim\limits_{x\to\infty}\sqrt{x^2+ax}-\sqrt{x^2+bx}$.

\answer{$\frac{a-b}2$}
\item $\displaystyle \lim\limits_{x\to\infty}\cos x$.

\answer{DNE}
\item $\displaystyle \lim\limits_{x\to\infty}\frac{x^4+x}{x^3-x+2}$.

\answer{$\infty$}
\item $\displaystyle \lim\limits_{x\to\infty}\sqrt{x^2+1}$.

\answer{$\infty$}
\item $\displaystyle \lim\limits_{x\to-\infty}(x^4+x^5)$.

\answer{$-\infty$}
\item $\displaystyle \lim\limits_{x\to-\infty}\frac{\sqrt{1+x^6}}{1+x^2}$.

\answer{$\infty$}
\item $\displaystyle \lim\limits_{x\to\infty}(x-\sqrt{x})$.

\answer{$\infty$}
\item $\displaystyle \lim\limits_{x\to\infty}(x^2-x^3)$.

\answer{$-\infty$}
\item $\displaystyle \lim\limits_{x\to\infty}x\sin x$.

\answer{DNE}
\item $\displaystyle \lim\limits_{x\to\infty}\sqrt{x}\sin x$.

\answer{DNE}
\end{enumerate}
\end{multicols}
\end{problem}




\begin{problem}
%(Problem contributed by Gabe Cunningham) 

Find the derivative of the following functions.
\begin{multicols}{2}
\begin{enumerate}
\item ${\displaystyle \frac{\sin x}{x^2}}$

\answer{${\displaystyle \frac{x \cos x - 2 \sin x}{x^3}}$}
\item ${\displaystyle e^{\sqrt{x^2 + 1}}}$

\answer{
$\begin{array}{l}\displaystyle e^{\sqrt{x^2 + 1}} \cdot \frac{1}{2}\left(x^2+ 1\right)^{ -\frac{ 1}{2}} \cdot 2x \\~\\
= \frac{x e^{\sqrt{x^2+1}} }{ \sqrt{ x^2 +1} } \end{array}$
} 
\item ${\displaystyle \ln \left(x-\frac{1}{x} \right)}$

\answer{${\displaystyle \frac{1}{x-\frac{1}{x}} \cdot \left(1 + \frac{1}{x^2}\right)}$} 
\item ${\displaystyle \sqrt[3]{x} \ln x}$

\answer{${\displaystyle \frac{1}{3} \frac{1}{\sqrt[3]{x^2}} \ln x + \frac{1}{\sqrt[3]{x^2}} }$} 
\item ${\displaystyle \cos(e^x)}$

\answer{${\displaystyle -\sin\left(e^x\right) \cdot e^x}$} 
\item ${\displaystyle \sin^3(2x)}$

\answer{${\displaystyle 3 \sin^2(2x) \cdot \cos(2x) \cdot 2 = 6 \sin^2(2x) \cos(2x)}$} 
\item ${\displaystyle f(x) = \int_x^1 (2+t^4)^5 \; \diff t}$

\answer{${\displaystyle -\left(2+x^4\right)^5}$} 
\item ${\displaystyle g(x) = \int_{0}^{x^3} \cos^2 t \; \diff t}$

\answer{${\displaystyle 3x^2 \cos^2\left(x^3\right)}$} 
\item Find $y'$ if $2x^2 + x + xy = 1$.

\answer{${\displaystyle y' = \frac{-4x-1-y}{x}}$} 
\item Find $y'$ if $x \sin y + y \sin x = 4$.

\answer{${\displaystyle y' = \frac{-\sin y - y \cos x}{\sin x + x \cos y}}$} 
\end{enumerate}
\end{multicols}

\end{problem}
\solution{\ref{problemDifferentiateFTC1int_x^1(2+t^4)^5dt} %(Contributed by student Anamaria Ronayne)
We recall that the Fundamental Theorem of Calculus part 1 states that $\frac{\diff}{\diff x}\left(\int_{a}^{x}h(t)dt\right)=h(x)$
where $a$ is a constant. We can rewrite the integral so it has $x$ as the upper limit:
\[
f(x)=\int_{x}^{1}(2+1^4)^5\diff t =-\int_{1}^{x}(2+1^4)^5\diff t\quad.
\]
Therefore
\[
\frac{\diff}{\diff x}\left( -\int_{1}^{x}(2+t^4)^5 \diff t\right)=- \frac{\diff }{\diff x}\left(\int_{1}^{x}(2+t^4)^5\diff t \right)\stackrel{\text{FTC part 1}}{=}
-(2+x^4)^5\quad .
\]

}

\begin{problem}
Evaluate the indefinite integrals.
\begin{multicols}{2}
\begin{enumerate}
\item ${\displaystyle \int e^x \left(\sqrt{e^x + 1}\right) \diff x}$

\answer{${\displaystyle \frac23 (e^x + 1)^{\frac{3}{2}} + C}$}

\item ${\displaystyle \int \frac{x^4 + 3x}{x^2} \diff x}$

\answer{${\displaystyle \frac{x^3}{3} + 3 \ln |x| + C}$}

\item ${\displaystyle \int x^2 e^{x^3} \diff x}$
\answer{${\displaystyle \frac{1}{3}e^{x^3} + C}$}

\item ${\displaystyle \int \frac{\cos x}{\sin x} \diff x}$
\answer{${\displaystyle \ln |\sin x| + C}$}

\item $\displaystyle\int \tan (2x) ~\diff x$.

\answer{$-\frac{1}{2}\ln\left|\cos (2x) \right| +C$}

\item $\displaystyle \int \cot \left(\frac{x}{2}\right)~\diff x$

\answer{$2\ln\left(\sin \left(\frac{x}{2}\right) \right)$}

\end{enumerate}
\end{multicols}

\end{problem}
\solution{\ref{probleminte^x(sqrt(e^x+1))dx}
\[
\begin{array}{rcll|l}
\displaystyle\int e^x\sqrt{e^x+1} ~ \diff x&=& \displaystyle \int \sqrt{e^x+1}\diff \left(e^x\right)\\
&=&\displaystyle \int \sqrt{e^x+1}\diff \left(e^x+1\right)
\displaystyle &&\text{Set }u=e^x+1\\
&=&\displaystyle \int \sqrt{u}\diff u\\
&=&\displaystyle \frac{2}{3}u^{\frac{3}{2}}+C\\
&=&\displaystyle \frac{2}{3}\left(e^x+1\right)^{\frac{3}{2}}+C
\end{array}
\]
}

\begin{problem}
Evaluate the definite integrals.
\begin{multicols}{2}
\begin{enumerate}[ref={\fcProblemRef}]
\item $\displaystyle\int\limits_{1}^{2} \frac{x}{2x^2+1 }  ~\diff x$.

\answer{$\frac14 \ln 3$}
\item $\displaystyle\int\limits_{0}^{\frac{1}4}\frac{x }{\sqrt{1-3x^2}}\diff x$.

\answer{$\frac{1}3\left(1-\sqrt{\frac{13}{16}} \right)$}

\item $\displaystyle\int\limits_{1}^{8} \frac{t-t^{\frac{1}{3}} +2}{t^{ \frac{4}{3 } } } \diff t\quad .$

\answer{$- \ln8+\frac{15}{2}$}
\item $\displaystyle\int\limits_{1}^{4} (x+\sqrt{x})^2 \diff x\quad .$

\answer{$\frac{119}{2}$}

\end{enumerate}
\end{multicols}

\end{problem}

\begin{problem}
\begin{enumerate}
\item Find $f(x)$ if $f'(x) = 3 + \frac{1}{x}$ and $f(1) = 2$.

\answer{$f(x) = 3x + \ln |x| - 1$}
\item Find $f(x)$ if $f'(x) = x - \sin x$ and $f(0) = 7$.

\answer{$f(x) = \frac{x^2}{2} + \cos x + 6$}
\end{enumerate}

\end{problem}

\begin{problem}
\begin{enumerate}
\item Sketch the graph of $y = x^4 - 8x^2 + 8$ by determining the intervals of increase and decrease, finding the local mins and maxes, determining where the graph is concave up and concave down, and plotting a few key points.

\answer{
\begin{tabular}{l}
Check your graph with a calculator or online graphing program. \\
Local max at 0, local mins at 2 and -2. Concave down between $-\sqrt{4/3}$ and $\sqrt{4/3}$, and concave up otherwise.
\end{tabular}
}

\item Sketch the graph of $y = \frac{x-1}{x^2-9}$ by graphing any vertical and horizontal asymptotes, finding the $x$- and $y$-intercepts, and then sketching a graph that fits this information.

\answer{
\begin{tabular}{l}
Check your graph with a calculator or online graphing program. \\
Vertical asymptotes at $x = 3$ and $x = -3$. \\ Horizontal
asymptote at $y = 0$. \\
$y$-intercept of $\frac{1}{9}$; $x$-intercept of $1$.
\end{tabular}
}
\item \label{problemSketch(4x^2+10x+5)/(2x+1)}

Consider the function $\displaystyle f(x)=\frac{4 x^{2}+10 x+5}{2 x+1} $. Computation shows that $\displaystyle f'(x)=\frac{8 x^{2}+8 x}{\left(2 x+1\right)^{2}}$ and $\displaystyle f''(x)=\frac{8}{\left(2 x+1\right)^{3}} $.

\begin{itemize}
\item Find the intervals of increase and intervals of decrease of $f$.

\item Find the local maxima and minima of $f$. 
\item Find where the function is concave up and where it is concave down.
\item Sketch the function $f(x)$ roughly by hand. Make sure that your plot matches your computations from the preceding parts of the problem.

You may use the provided grid and coordinate system. From the previous page, we recall that $\displaystyle f(x)=\frac{4 x^{2}+10 x+5}{2 x+1} $, $\displaystyle f'(x)=\frac{8 x^{2}+8 x}{\left(2 x+1\right)^{2}}$ and $\displaystyle f''(x)=\frac{8}{\left(2 x+1\right)^{3}} $.

The 4 points plotted on the grid are known to lie on the curve.

\psset{xunit=1cm, yunit=1cm}
\begin{pspicture}(-3.1,-3)(3.1,12.1)
\fcAxesStandard{-3}{-3}{3}{12}
\fcGrid[linestyle=dashed, linewidth=0.5, linecolor=gray]{-3}{-3}{6}{15}{1}{1}{}
\rput[t](0.9,-0.2){$1$}
\fcLabels{3}{12}
\fcFullDot{-3}{1 dict begin /x -3 def x x mul 4 mul 10 x mul 5 add add 2 x mul 1 add div end}
\fcFullDot{-0.59}{1 dict begin /x -0.59 def x x mul 4 mul 10 x mul 5 add add 2 x mul 1 add div end}
\fcFullDot{-0.44}{1 dict begin /x -0.44 def x x mul 4 mul 10 x mul 5 add add 2 x mul 1 add div end}
\fcFullDot{3}{1 dict begin /x 3 def x x mul 4 mul 10 x mul 5 add add 2 x mul 1 add div end}

%\psplot{-3}{-0.59}{x x mul 4 mul 10 x mul 5 add add 2 x mul 1 add div}
%\psplot{-0.44}{3}{x x mul 4 mul 10 x mul 5 add add 2 x mul 1 add div}
\end{pspicture}
\end{itemize}
\item \label{problemSketch(2 x^2-4 x+2)/(x^2-2 x)}
Consider the function $\displaystyle f(x)=\frac{2 x^{2}-4 x+2}{x^{2}-2 x} $.

\begin{itemize}
\item Find the vertical asymptotes of $f$. \textbf{For this particular sub-question, and for this sub-question alone, no justification is required (just write the answer).}

\item Computation shows that $\displaystyle f'(x)=\frac{-4 x+4}{\left(x^{2}-2 x\right)^{2}}$. Find the intervals of increase and decrease of $f$.
\item Find the local maxima and minima of $f$. 
\item Computation shows that $\displaystyle f''(x)=\frac{12 x^{2}-24 x+16}{\left(x^{2}-2 x\right)^{3}} $. Find where the function is concave up and where it is concave down.
\item Sketch the function $f(x)$ roughly by hand. Make sure that your plot matches your computations from the preceding parts of the problem.


You may use the provided grid and coordinate system. We recall that \\
$\displaystyle f(x)=\frac{2 x^{2}-4 x+2}{x^{2}-2 x} $,\\ 
$\displaystyle f'(x)=\frac{-4 x+4}{\left(x^{2}-2 x\right)^{2}}$,\\ $\displaystyle f''(x)=\frac{12 x^{2}-24 x+16}{\left(x^{2}-2 x\right)^{3}}$.

The points plotted below are known to lie on the curve.

\psset{xunit=0.8cm, yunit=0.8cm}
\begin{pspicture}(-3.2,-9.2)(5.2,13.2)
\fcAxesStandard{-3}{-8}{5}{13}
\fcGrid[linestyle=dashed, linewidth=0.5, linecolor=gray]{-3}{-9}{8}{22}{1}{1}{}
\rput[t](0.9,-0.2){$1$}
\fcLabels{5}{13}
\newcommand{\theFun}{2 x x mul mul -4 x mul 2 add add x x mul -2 x mul add div\space}
\fcFullDot{-3}{1 dict begin /x -3 def \theFun end}
\fcFullDot{-0.1}{1 dict begin /x -0.1 def \theFun end}
\fcFullDot{0.1}{1 dict begin /x 0.1 def \theFun end}
\fcFullDot{1.9}{1 dict begin /x 1.9 def \theFun end}
\fcFullDot{2.1}{1 dict begin /x 2.1 def \theFun end}
\fcFullDot{5}{1 dict begin /x 5 def \theFun end}

%\psplot{-3}{-0.1}{\theFun}
%\psplot{0.1}{1.9}{\theFun}
%\psplot{2.1}{5}{\theFun}
\end{pspicture}

\end{itemize}
\end{enumerate}

\end{problem}

\begin{problem}
A computer generated plot of $y=f(x)$ is included below. Find the
\begin{multicols}{2}
\begin{itemize}
\item implied domain of $f$
\item $x$ and $y$ intercepts of $f$.
\item horizontal and vertical asymptotes.
\item intervals of increase and decrease
\item local and global minima, maxima,
\item intervals of concavity
\item points of inflection
\end{itemize}
\end{multicols}
Label all relevant points on the graph.
\begin{enumerate}
\item \label{problemSketch(x+1)/(x^2+2x+4)}  $\displaystyle f(x)=\frac{x+1}{x^2+2x+4}$
\psset{xunit=1cm, yunit=1cm}
\begin{pspicture}(-4.500000, -5)(4.500000,5)
\psframe*[linecolor=white](-4.500000,-1)(4.500000,1)
\tiny
\fcAxesStandard{-5.000000}{-1}{3.000000}{1} %Function formula: \frac{x+1}{x^{2}+2 x+4}
\psplot[linecolor=\fcColorGraph, plotpoints=1000]{-5.000000}{3.000000}{1 x add 4 x 2 mul add x 2 exp add div }
\end{pspicture}

\answer{
\begin{tabular}{l}
$y$-intercept: $\frac14$, $x$-intercept: $-1$\\
horizontal asymptote: $y=0$, vertical: none\\
increasing on
$\left(-1-\sqrt{3}, -1+\sqrt{3}  \right) $, decreasing on $\left(-\infty, -1-\sqrt{3}\right)\cup \left(-1+\sqrt{3}, \infty\right) $\\
local and global min at $x=-1-\sqrt{3}$, local and global max at $x=-1+\sqrt{3}$\\
concave up on $\left(-4, -1\right)\cup \left(2, \infty \right)$, concave down $\left(-\infty, -4\right)\cup \left(-1, 2\right)$\\
inflection points at $x=-4,x=-1, x=2$
\end{tabular}
}
\item $f(x)=\frac{x^{2}+3 x+1}{x^{2}+2 x}$
\psset{xunit=0.5cm, yunit=0.5cm}
\begin{pspicture}(-5.4, -4.118006)(5.4,6.738095)
\tiny
\fcAxesStandard{-5.15}{-3.868006}{5.15}{6.388095}

%Function formula: \frac{x^{2}+3 x+1}{x^{2}+2 x}
\psplot[linecolor=\fcColorGraph, plotpoints=1000]{0.1}{5}{ 1 x 3 mul add x 2 exp add x 2 mul x 2 exp add div }

%Function formula: \frac{x^{2}+3 x+1}{x^{2}+2 x}
\psplot[linecolor=\fcColorGraph, plotpoints=1000]{-1.9}{-0.1}{ 1 x 3 mul add x 2 exp add x 2 mul x 2 exp add div }

%Function formula: \frac{x^{2}+3 x+1}{x^{2}+2 x}
\psplot[linecolor=\fcColorGraph, plotpoints=1000]{-5}{-2.1}{ 1 x 3 mul add x 2 exp add x 2 mul x 2 exp add div }
\end{pspicture}

\answer{
\begin{tabular}{l}
$y$-intercept: none, $x$-intercepts: $\frac{-3\pm \sqrt{5}}2$ \\
horizontal asymptote: $y=1$, vertical: $x=0$ and $x=-2$\\
always decreasing\\
no local/global minima/maxima\\
concave down on $\left(-\infty,-2\right)\cup \left(-1,0 \right)$, concave up on $\left(-2, -1\right)\cup \left(0, \infty\right)$\\
inflection point at $x=-1$
\end{tabular}
}
\end{enumerate}


\end{problem}
\solution{\ref{problemSketch(x+1)/(x^2+2x+4)}
\textbf{Domain.} As $f$ is a quotient of two polynomials (rational function), its implied domain is all $x$ except those for which we get division by zero for $f$. Consequently the domain of $f$ is all $x$ for which $x ^2+2x+4=0$. However, the polynomial $x^2+2x+4$ has no real roots - its roots are $\displaystyle \frac{-2\pm \sqrt{4-16} }{2}=-1\pm \sqrt{-3}$, and therefore the domain of $f$ is all real numbers. Alternatively, we can complete the square: $x^2+2x+4=(x+1)^2+3$ and so $x^2+2x+4$ is positive for all values of $x$. 

The $y$-intercept of $f$ equals by definition $\displaystyle f(0)= \frac{ 0+ 1}{0^2+2\cdot 0 + 4}=\frac{1}{4}$. The $x$ intercept of $f$ is those values of $x$ for which $f(x)=0$. We compute

\[
\begin{array}{rcl}
\displaystyle f(x)&=&0\\
\displaystyle \frac{x+1}{x^2+2x+4}&=&0\\
x+1&=&0\\
x&=&-1\quad ,
\end{array}
\]
and the $x$-intercept of $f$ is $x=-1$. 

\textbf{Asymptotes.} The line $x=a$ is a vertical asymptote when $\lim\limits_{x\to a^{\pm}}f(x)=\pm \infty$; as $f$ is defined for all real numbers, this implies that there are no vertical asymptotes. 
 
The line $y=L$ is a horizontal asymptote if $\lim\limits_{x\to\pm \infty}f(x)$ exists and equals $L$. We compute:
\[
\lim\limits_{x\to \infty} f(x)=\lim\limits_{x\to \infty} \frac{(x+1)\frac{1}{x^2}}{ (x^2+ 2x +4)\frac{1}{x^2}} = \lim\limits_{x\to \infty}\frac{\frac{1}{x}+\frac{1}{x^2}}{ 1+ \frac{2}{x} +\frac{4}{x^2}}=\frac{0+0}{1+0+0}=0
\]
Therefore $y=0$ is a horizontal asymptote for $f$. An analogous computation shows that $\lim\limits_{x\to\pm \infty}f(x)=0$ and so $y=0$ is the only horizontal asymptote of $f$.

\textbf{Intervals of increase and decrease.} 
The intervals of increase and decrease of $f$ are governed by the sign of $f'$. We compute:
\[
\begin{array}{rcll|l}
f'(x)&=&\displaystyle \left(\frac{x+1}{x^2+2x+4}\right)' &&\text{qutotient rule}\\
&=&\displaystyle \frac{(x+1)'\left(x^2+ 2x+4\right)- (x+1)\left( x^2 +2 x+4\right)'}{\left(x^2+2x+4 \right)^2}\\
&=&\displaystyle\frac{ x^2+2x+4-(x+1)(2x+2)}{\left(x^2+2x+4 \right)^2}\\
&=&\displaystyle \frac{x^2+2x+4-\left( 2x^2+ 4x+ 2 \right)}{ \left( x^2 +2x+4 \right)^2}\\
&=&\displaystyle \frac{-x^2-2x+2}{\left(x^2+2x+4 \right)^2}
\end{array}
\] 
As $x^2+2x+4$ is positive, the sign of $f'$ is governed by the sign of $-x^2+2x+2$. To find out where $-x^2+2x+2$ changes sign, we compute the zeroes of this expression:
\[\begin{array}{rcll|l}
-x^2-2x+2&=&0\\
x^2+2x-2&=&0 &&\text{use the quadratic formula}\\
x_1, x_2&=& -1\pm \sqrt{3}\quad .
\end{array}
\]
Therefore the quadratic $-x^2+2x+2$ factors as 
\begin{equation}
\label{eq1problemSketch(x+1)/(x^2+2x+4)}
-(x-x_1)(x-x_2)=-\left(x-\left(-1-\sqrt{3}\right)\right)\left(x-\left(-1+\sqrt{3}\right)\right)
\end{equation} 
The points $x_1, x_2$ split the real line into three intervals: $(-\infty, -1-\sqrt{3})$, $(-1-\sqrt{3}, -1+\sqrt{3})$ and $(-1+ \sqrt{3}, \infty )$, and each of the factors of \eqref{eq1problemSketch(x+1)/(x^2+2x+4)} has constant sign inside each of the intervals. If we chooose $x$ to be a very negative number, it follows that $-(x-x_1)(x-x_2)$ is a negative, and therefore $ f'(x)$ is negative for $x\in(-\infty, -1-\sqrt{3})$. For $x\in (-1-\sqrt{3}, -1+\sqrt{3})$, exactly one factor of $f'$ changes sign and therefore $f'(x)$ is positive in that interval; finally only one factor of $f'(x)$ changes sign in the last interval so $f'(x)$ is negative on $(-1+ \sqrt{3}, \infty )$.

Our computations can be summarized in the following table. 

\begin{tabular}{|lll|}\hline
Interval & $f'(x)$ & $f(x)$   \\\hline
$(-\infty, -1-\sqrt{3})$ & $-$& $\searrow $ \\\hline
$(-1-\sqrt{3}, -1+\sqrt{3})$ &$+$&$\nearrow$\\\hline
$( -1+\sqrt{3}, \infty)$&$-$&$\searrow$ \\\hline
\end{tabular}

\textbf{Local and global minima and maxima. } The table above shows that $f(x)$ changes from decreasing to increasing at $x=x_1=-1-\sqrt{3}$ and therefore $f$ has a local minimum at that point. The table also shows that $f(x)$ changes from increasing to decreasing at $ x=x_2=-1+\sqrt{3}$ and therefore $f$ has a local maximum at that point. The so found local maximum and local minimum turn out to be global: indeed, no other finite point is critical and thus cannot be maximum or minimum; on the other hand $\lim\limits_{x\to\pm \infty}f(x)=1$ and this implies that all $x$ sufficiently far away from $x=0$ have that $f(x)$ is close to $0$, and therefore $f(x)$ is larger than $f(x_1)$ and smaller than $f(x_2)$ for all $x$.

\textbf{Intervals of concavity. } 
The intervals of concavity of $f$ are governed by the sign of $f''$. The second derivative of $f$ is:
\[
\begin{array}{rcll|l}
f''(x)&=&\displaystyle (f'(x))'= \left(\frac{-x^2-2x+2}{\left(x^2+2x+4 \right)^2}\right)'\\
&=&\displaystyle (-x^2-2x+2)'\left(\frac{1}{(x^2+2x+4)^2}\right)+(-x^2-2x+2)\left(\frac{1}{(x^2+2x+4)^2}\right)' &&\begin{array}{l}\text{use chain rule }\\\text{for second differentiation}\end{array}\\
&=&\displaystyle (-2x-2)\left(\frac{1}{(x^2+2x+4)^2}\right)+(-x^2-2x+2)(-2)\frac{(x^2+2x+4)'}{(x^2+2x+4)^{3}}\\
&=&\displaystyle -(2x+2)\left(\frac{1}{(x^2+2x+4)^2}\right) +(2x^2+4x-4)\frac{(2x+2)}{(x^2+2x+4)^{3}}&&\text{factor out }\frac{(2x+2)}{(x^2+2x+4)^2}\\
&=&\displaystyle \frac{(2x+2)}{(x^2+2x+4)^2}\left(-1+\frac{(2x^2+4x-4)}{(x^2+2x+4)}\right)\\
&=&\displaystyle \frac{(2x+2)}{(x^2+2x+4)^2}\left(\frac{-(x^2+2x+4)+(2x^2+4x-4)}{(x^2+2x+4)} \right)\\
&=&\displaystyle \frac{(2x+2)(x^{2}+2 x-8)}{(x^2+2x+4)^3}&& \text{factor } (x^2+2x-8)\\
&=&\displaystyle \frac{(2x+2)(x+4)(x-2)}{(x^2+2x+4)^3}
\end{array}
\]
As we previously established, the denominator of the above expression is always positive. Therefore the expression above changes sign when the terms in the numerator change sign, namely, at $x=-1$, $x=-4$ and $x=2$. 

Our computations can be summarized in the following table. 

\begin{tabular}{|lll|}\hline
Interval & $f''(x)$ & $f(x)$   \\\hline
$(-\infty, -4)$ & $-$& $\cap$ \\\hline
$(-4, -1)$ &$+$&$\cup$\\\hline
$(-1, 2)$&$-$&$\cap$ \\\hline
$(2, \infty)$&$+$&$\cup$ \\\hline
\end{tabular}

\textbf{Points of inflection.} The preceding table shows that $f''(x)$ changes sign at $-4, -1, 2$ and therefore the points of inflection are located at $x=-4, x=-1$ and $x=2$, i.e., the points of inflection are $\left(-4, -\frac{1}{4}\right)$, $\left(-1, 0\right)$, $\left(2, \frac{1}{4}\right)$.

}

\begin{problem}
~\begin{enumerate}[ref={\fcProblemRef}]
\item Find the linearization of $f(x) = \sqrt{x}$ at $a = 100$ and use it to approximate
$\sqrt{99.8}$.

\answer{$L(x) = 10 + 0.05(x-100)$. Therefore $\sqrt{99.8} \approx L(99.8) = 9.99$.}
 
\item Find the linearization of $f(x)=\sqrt{8+x}$ at $a=1$ and use it to approximate $\sqrt{9.02}$.

\answer{ $f(x)\approx 3+ \frac16 (x-1)=\frac{1}{6} x+\frac{17}{6}$. Therefore $\sqrt{9.02}\approx \frac{901}{300} \approx 3.003333$}
\item Find the linearization of $f(x)=\sqrt[3]{8+x}$ at $a=0$ and use it to approximate $\sqrt[3] {7.97}$.

\answer{ $\sqrt[3]{8+x}\approx \frac{1}{12}x+2$. Therefore $\sqrt[3]{7.97}\simeq \frac{799}{400} =1.9975$}

\item Find the linearization of $f(x)=\ln x$ at $a=1$ and use it to approximate $\ln 1.01$.

\answer{ $f(x)\approx f(1)+f'(1)(x-1)=x-1 $, $\ln 1.01\approx 0.01$. }
\item Use a linear approximation to estimate $(1.001)^9$. 

\answer{$(1.001)^9 \approx 1.009$.}
\item \label{problem-linearization-estimate0.9999power2014} Use a linear approximation to estimate $(0.9999)^{2014}$. 

\answer{$(0.9999)^{2014} \approx 0.7986$.}

\end{enumerate}

\end{problem}

\begin{problem}
Estimate the integral using a Riemann sum using the indicated sample points and interval length.
\begin{enumerate}[ref={\fcProblemRef}]
% Riemann sums
\item \label{problemRiemannSum-sqrt(8x+1)} $\displaystyle \int_0^4 \left(\sqrt{8x+1}\right)\diff x$. Use four intervals of equal width, choose the sample point to be the left endpoint of each interval. 

\answer{ $\Delta x = 1$ and $f(x) = \sqrt{8x+1}$. Thus ${\displaystyle \int_0^4 f(x) \diff x \approx 9 + \sqrt{17}}$.}

\item $\displaystyle \int_0^6 \frac{1}{x^2+1} \diff x$. Use three intervals of equal width, choose the sample point to be the left endpoint. 

\answer{ $\Delta x = 2$ and $f(x) = \frac{1}{x^2+1}$. Thus ${\displaystyle \int_0^6 f(x) \diff x \approx \frac{214}{85}}$.}
\item \label{problemRiemannSum-1div1plusxsquared} $\displaystyle \int\limits_{-3.5}^{-0.5} \frac{\diff x}{x^2+1} $. Use three intervals of equal width, choose the sample point to be the midpoint of each interval. 

\answer{ $\Delta x = 1$ and $f(x) = \frac{1}{x^2+1}$. Thus $\displaystyle \int \limits_{-3.5}^{-0.5} f(x) \diff x  \approx \Delta x\left(f{} \left(-3 \right)+ f{}\left( -2\right)+f{}\left(-1\right)\right)=\frac{4}{5}=0.8$.}
\item $\displaystyle\int_{0}^2 \frac{\diff x}{1+x+x^3}$. Use $\Delta x=\frac{1}2 $ and right endpoint sampling points.

\answer{$ \frac{1}{2}\left(\frac{8}{13}+\frac{1}{3}+\frac{8}{47}+\frac{1}{11}\right)=\frac{12197}{20163}\approx 0.604920$}

\item $\displaystyle\int_{-2}^{0} \frac{\diff x}{1+x+x^2}$. Use $\Delta x=\frac23 $ and left endpoint sampling points.

\answer{$\frac23\left(\frac{1}{3}+\frac{9}{13}+\frac{9}{7}\right)=\frac{1262}{819}\approx 1.540904$}

\item $\displaystyle \int\limits_0^2 \frac{\diff x}{1+x^3}$. Use four intervals of equal width, choose the sample point to be the left endpoint of each interval. 

\answer{ $\Delta x = 0.5$ and $f(x) = \frac{1}{1+x^3}$. Thus $\displaystyle \int\limits_0^2 f(x) \diff x  \approx \Delta x\left(f{}\left(0\right)+f{}\left(1\right)+f{}\left(\frac{1}{2}\right)+f{}\left(\frac{3}{2}\right)\right)=\frac{1649}{1260}\approx 1.30873$.}

\item $\displaystyle \int\limits_{-2}^{0} \frac{\diff x}{x^4+1} $. Use four intervals of equal width, choose the sample point to be the right endpoint. 

\answer{ $\Delta x = 0.5$ and $f(x) = \frac{1}{1+x^3}$. Thus $\displaystyle \int\limits_0^2 f(x) \diff x  \approx \Delta x\left(f{}\left(-\frac{3}{2}\right)+f{}\left(-1\right)+f{}\left(-\frac{1}{2}\right)+f{}\left(0\right)\right)=\frac{8595}{6596}\approx 1.303062$.}

\item  \label{problemRiemannSum1/(3x^2+1)from-1to0with3intervalsLeftEndpt}
$\displaystyle\int_{-1}^0\frac{1}{3 {{x}}^{2}+1}\diff x
$. Use \textbf{$3$ intervals} of equal width, choose the sampling points to be the \textbf{left endpoints} of each interval. 
Simplify your answer to a rational number (single fraction of two integers).

\answer{ $\Delta x = \frac{1}{3}$ and $f(x) =\frac{1}{3 {{x}}^{2}+1}$. Thus $\displaystyle \int\limits_{-1}^0 f(x) \diff x$  is approximated by $\Delta x \left(f{}\left(-1\right)+f{}\left( -\frac{2}{3}\right)+f{}\left( -\frac{1}{3}\right)\right)=\frac{10}{21}$.}

\end{enumerate}



\end{problem}
\solution{\ref{problemRiemannSum-sqrt(8x+1)}. The interval $[0,4]$ is subdivided into $n=4$ intervals, therefore the length of each is $\Delta x=1$. The intervals are therefore
\[
[0,1], [1,2], [2,3], [3,4]\quad .
\]
The problem asks us to use the left endpoints of each interval as sampling points. Therefore our sampling points are $0,1,2,3$. Therefore the Riemann sum we are looking for is
\[
\Delta x\left(f(0)+f(1)+f(2)+f(3) \right)=1\cdot \left(\sqrt{8\cdot 0+1}+\sqrt{8\cdot 1+1}+\sqrt{8\cdot 2+1}+\sqrt{8\cdot 3+1}\right)= 9+\sqrt{17}\approx 13.1231
\]

\hfil \hfil \psset{xunit=1cm, yunit=1cm}
\begin{pspicture}(-0.9, -0.9)(4.4,6.233433)
\tiny
\psline*[linecolor=\fcColorAreaUnderGraph, linewidth=0.1pt](0.000000, 0.000000)(0.000000, 1.000000)(1.000000, 1.000000)(1.000000, 0.000000)(0.000000, 0.000000)
\psline*[linecolor=\fcColorAreaUnderGraph, linewidth=0.1pt](1.000000, 0.000000)(1.000000, 3.000000)(2.000000, 3.000000)(2.000000, 0.000000)(1.000000, 0.000000)
\psline*[linecolor=\fcColorAreaUnderGraph, linewidth=0.1pt](2.000000, 0.000000)(2.000000, 4.123106)(3.000000, 4.123106)(3.000000, 0.000000)(2.000000, 0.000000)
\psline*[linecolor=\fcColorAreaUnderGraph, linewidth=0.1pt](3.000000, 0.000000)(3.000000, 5.000000)(4.000000, 5.000000)(4.000000, 0.000000)(3.000000, 0.000000)
\psline[linecolor=blue, linewidth=0.1pt](0.000000, 0.000000)(0.000000, 1.000000)(1.000000, 1.000000)(1.000000, 0.000000)(0.000000, 0.000000)
\psline[linecolor=blue, linewidth=0.1pt](1.000000, 0.000000)(1.000000, 3.000000)(2.000000, 3.000000)(2.000000, 0.000000)(1.000000, 0.000000)
\psline[linecolor=blue, linewidth=0.1pt](2.000000, 0.000000)(2.000000, 4.123106)(3.000000, 4.123106)(3.000000, 0.000000)(2.000000, 0.000000)
\psline[linecolor=blue, linewidth=0.1pt](3.000000, 0.000000)(3.000000, 5.000000)(4.000000, 5.000000)(4.000000, 0.000000)(3.000000, 0.000000)
%Function formula: (8 x+1)^{1/2}
\psplot[linecolor=\fcColorGraph, plotpoints=1000]{0}{4}{ 1 x 8 mul add 0.5 exp }
\psaxes(0,0)(-0.65,-0.65)(4.15,5.883433)
\end{pspicture}
}

\solution{
\ref{problemRiemannSum-1div1plusxsquared}. The interval $[-3.5,-0.5]$ is subdivided into $n=3$ intervals, therefore the length of each is $\Delta x=\frac{-0.5-(-3.5)}{3}=\frac{3}{3}= 1$. The intervals are therefore
\[
[-3.5,-2.5], [-2.5,-1.5], [-1.5,-0.5]\quad .
\]
The problem asks us to use the midpoint of each interval as a sampling point. Therefore our sampling points are $-3,-2,-1$. Therefore the Riemann sum we are looking for is
\[
\Delta x\left(f(-3)+f(-2)+f(-1) \right)=1\cdot \left( \frac{1}{10}+\frac{1}{5}+\frac{1}{2}\right)= 0.8\quad .
\]

\hfil \hfil 
\psset{xunit=1cm, yunit=1cm}
\begin{pspicture}(-3.9, -0.9)(1.4,1.499857)
\tiny

\psline*[linecolor=\fcColorAreaUnderGraph, linewidth=0.1pt](-3.500000, 0.000000)(-3.500000, 0.100000)(-2.500000, 0.100000)(-2.500000, 0.000000)(-3.500000, 0.000000)
\psline*[linecolor=\fcColorAreaUnderGraph, linewidth=0.1pt](-2.500000, 0.000000)(-2.500000, 0.200000)(-1.500000, 0.200000)(-1.500000, 0.000000)(-2.500000, 0.000000)
\psline*[linecolor=\fcColorAreaUnderGraph, linewidth=0.1pt](-1.500000, 0.000000)(-1.500000, 0.500000)(-0.500000, 0.500000)(-0.500000, 0.000000)(-1.500000, 0.000000)
\psline[linecolor=blue, linewidth=0.1pt](-3.500000, 0.000000)(-3.500000, 0.100000)(-2.500000, 0.100000)(-2.500000, 0.000000)(-3.500000, 0.000000)
\psline[linecolor=blue, linewidth=0.1pt](-2.500000, 0.000000)(-2.500000, 0.200000)(-1.500000, 0.200000)(-1.500000, 0.000000)(-2.500000, 0.000000)
\psline[linecolor=blue, linewidth=0.1pt](-1.500000, 0.000000)(-1.500000, 0.500000)(-0.500000, 0.500000)(-0.500000, 0.000000)(-1.500000, 0.000000)
\rput[t](-3.500000,-0.03){$-\frac{7}{2}$}\rput[t](-2.500000,-0.03){$-\frac{5}{2}$}\rput[t](-1.500000,-0.03){$-\frac{3}{2}$}\rput[t](-0.500000,-0.03){$-\frac{1}{2}$}
%Function formula: (x^{2}+1)^{-1}
\psplot[linecolor=\fcColorGraph, plotpoints=1000]{-3.5}{1}{ 1 x 2 exp add -1 exp }
\psaxes[ticks=none, labels=none, arrows=<-> ](0,0)(-3.65,-0.65)(1.15,1.149857)
\fcLabels{1.15}{1.149857}
\end{pspicture}
}
\solution{\ref{problemRiemannSum1/(3x^2+1)from-1to0with3intervalsLeftEndpt}

$\Delta x = \frac{1}{3}$ and $f(x) =\frac{1}{3 {{x}}^{2}+1}$. Thus $\displaystyle \int\limits_{-1}^0 f(x) \diff x$  is approximated by $\Delta x \left(f{}\left(-1\right)+f{}\left(-\frac{2}{3}\right)+f{}\left(-\frac{1}{3}\right)\right)=\frac{10}{21}$.

}


\begin{problem}
\begin{enumerate}[ref={\fcProblemRef}]
% Optimization
\item \label{problemponthyperbolax^2-4y^2closestTo1,1} What is the $x$-coordinate of the point on the hyperbola $x^2 - 4y^2 = 16$ that is closest to the point $(1, 0)$?

\psset{xunit=0.3cm, yunit=0.3cm}
\begin{pspicture}(-10.500000, -5)(10.500000,5)
\psframe*[linecolor=white](-10.500000,-5)(10.500000,5)
\tiny
\fcAxesStandard{-10.000000}{-4.5}{10.000000}{4.5} %Function formula: - (1/4 x^{2}-4)^{1/2}
\psplot[linecolor=\fcColorGraph, plotpoints=1000]{-10.000000}{-4.000000}{-4 x 2 exp 0.25 mul add 0.5 exp -1 mul }
%Function formula: (1/4 x^{2}-4)^{1/2}
\psplot[linecolor=\fcColorGraph, plotpoints=1000]{-10.000000}{-4.000000}{-4 x 2 exp 0.25 mul add 0.5 exp }
%Function formula: - (1/4 x^{2}-4)^{1/2}
\psplot[linecolor=\fcColorGraph, plotpoints=1000]{4.000000}{10.000000}{-4 x 2 exp 0.25 mul add 0.5 exp -1 mul }
%Function formula: (1/4 x^{2}-4)^{1/2}
\psplot[linecolor=\fcColorGraph, plotpoints=1000]{4.000000}{10.000000}{-4 x 2 exp 0.25 mul add 0.5 exp }
\fcFullDot{1}{0}
\fcFullDot{4}{0}
\pscircle[linestyle=dotted](1,0){0.9}
\end{pspicture}

\answer{$x = 4$}

\item What is the $x$-coordinate of the point on the ellipse $x^2+4y^2=16$ closest to the point $(1,0)$?

\psset{xunit=0.3cm, yunit=0.3cm}
\begin{pspicture}(-4.500000, -5)(4.500000,5)
\psframe*[linecolor=white](-4.500000,-5)(4.500000,5)
\tiny
\psline[linecolor=red!1](-10,0)(-9.99,0)
\fcAxesStandard{-4.000000}{-4.5}{4.000000}{4.5} %Function formula: - (-1/4 x^{2}+4)^{1/2}
\psplot[linecolor=\fcColorGraph, plotpoints=1000]{-4.000000}{4.000000}{4 x 2 exp -0.25 mul add 0.5 exp -1 mul }
%Function formula: (-1/4 x^{2}+4)^{1/2}
\psplot[linecolor=\fcColorGraph, plotpoints=1000]{-4.000000}{4.000000}{4 x 2 exp -0.25 mul add 0.5 exp }
\pscircle[linestyle=dotted](1,0){0.574456}
\fcFullDot{1}{0}
\fcFullDot{1.333333}{1.885618}
\fcFullDot{1.333333}{-1.885618}
\end{pspicture}
\answer{$x=\frac43$}
\item \label{problemMaxVolumeBoxFixedAreaDoubleBottomNoLid} You want to build a rectangular box with a square base out of sheet metal. You are going to use 2 pieces of sheet metal for the bottom of the box to reinforce it, and only a single piece of sheet metal for all of the sides and the top. If you want to use no more than $36$ sq. ft. of material, what is the largest possible volume you can enclose?

\answer{12 cubic feet.}
\end{enumerate}

\end{problem}
\solution{\ref{problemponthyperbolax^2-4y^2closestTo1,1}

The distance function between an arbitrary point $(x,y)$ and the point $(1,0)$ is $d=\sqrt{(x-1)^2+(y-0)^2}$. On the other hand, when the point $(x,y)$ lies on the hyperbola we have $y^2= \frac{x^2 -16 }{4 }$. In this way, the problem becomes that of minimizing the distance function

\[
dist(x)=\sqrt{(x-1)^2+y^2}=\sqrt{(x-1)^2+\frac{x^2-16}{4}} \quad .
\]
This is a standard optimization problem: we need to find the critical endpoints, i.e., the points where $dist'=0$. As the square root function is an increasing function, the function $\displaystyle \sqrt{(x-1)^2+\frac{x^2-16}{4}}$ achieves its minimum when the function 
\[
l=dist^2=(x-1)^2+\frac{x^2-16}{4}\quad 
\]
does. $l$ is a quadratic function of $x$ and we can directly determine its minimimum via elementary methods. Alternatively, we find the critical points of $l$:
\[
\begin{array}{rcl}
\displaystyle l'&=&\displaystyle 0\\
\displaystyle 2(x-1)+\frac{x}{2} &=&\displaystyle 0\\
\displaystyle \frac{5}{2}x-2&=&0\\
\displaystyle x&=&\displaystyle \frac{4}{5}\quad .
\end{array}
\]
On the other hand, $x^2=16+4y^2$ and therefore $|x|\geq \sqrt{16} = 4$. Therefore $x\in (-\infty, -4]\cup [4,\infty)$. As $x= \frac{4}{5 }$ is outside of the allowed range, it follows that our either function attains its minimum at one of the endpoints $\pm 4$ or the function has no minimum at all. It is clear however that as $x$ tends to  $\infty$, so does $dist$. Therefore $dist$ attains its maximum for $x=4$ or $-4$ and $y=\pm\sqrt{(\pm4)^2-16}=0$. Direct check shows that $dist_{|x=4} =\sqrt{(4-1)^2 +\frac{4^2- 16}{4 }}=3$ and $dist_{|x=-4}=\sqrt{(-4-1)^2+\frac{4^2-16}{4}}=5$  so our function $dist$ has a minimal value of $3$ achieved when $x=4$, which is our final answer. Notice that this answer can be immediately given without computation by looking at the figure drawn for \ref{problemponthyperbolax^2-4y^2closestTo1,1}. Indeed, it is clear that there are no points from the hyperbola lying inside the dotted circle centered at $(1,0)$. Therefore the point where this circle touches the hyperbola must have the shortest distance to the center of the circle.
}

\solution{\ref{problemMaxVolumeBoxFixedAreaDoubleBottomNoLid}
Let $B$ denote the area of the base of the box, equal to the area of the top. Let $W$ denote the area of the four walls of the box (the four walls are all equal because the base of the box is a square). Then the surface area $S$ of material used will be 
\[
S=\underbrace{2B}_{\text{two pieces for the bottom}}+\underbrace{4W}_{4 \text{ walls}} +\underbrace{B}_{\text{the box lid}}=3B+4W\quad.
\] 
Let $x$ denote the length of the side of the square base and let $y$ denote the height of the box.  Then  
\[
B=x^2
\]
and 
\[
W=xy\quad .
\]
As the surface area $S$ is fixed to be $36$ square feet, we have that
\[
S=3B+4W=36= 3x^2 + 4xy\quad .
\]
As $y$ is positive, the above formula shows that $3x^2\leq 36$ and so $x\leq \sqrt{12}$. Let us now express $y$ in terms of $x$:
\[
\begin{array}{rcl}
3x^2+4xy&=&36\\
4xy&=&36-x^2\\
y&=&\displaystyle\frac{36-x^2}{4x}\quad .
\end{array}
\]
The problem asks us to maximize the volume $V$ of the box. The volume of the box equals the area of the base times the height of the box: 
\[
V=B\cdot y=yx^2 = \frac{(36-3x^2)}{4x}x^2=\frac{36x-3x^3}{4}\quad .
\]
As $x$ is non-negative, it follows that the domain for $x$ is:
\[
x\in [0, \sqrt{12}]\quad .
\]
To maximize the volume we find the critical points, i.e., the values of $x$ for which  $V'$ vanishes:

\[\begin{array}{rcl}
0&=&V' = \displaystyle\left(\frac{36x-3x^3}{4}\right)'\\
0&=&\displaystyle \frac{36- 9x^2}{4}\\
9x^2&=&36\\
x^2&=&4\\
x&=&\pm 2
\end{array}
\]
As $x$ measures length, $x=-2$ is not possible (outside of the domain for $x$). Therefore the only critical point is $x=2$. Direct check shows that at the endpoints $x=0$ and $x=\sqrt{12}$, we have that $V=0$. Therefore the maximal volume is achieved when $x=2$:
 
\[
V_{max}=V_{|x=2}= \frac{36(2)-3(2)^3}{4} =12\quad .
\]
 
}



\begin{problem}
Use the Intermediate Value theorem and the Mean Value Theorem/Rolle's Theorem to prove that the function has \textbf{exactly one} real root.
\begin{enumerate}
\item \label{problemIVTandMVTx^3+4x+7} $f(x)=x^3+4x+7$.
\item $f(x)= x^3 +x^2+x+1$.
\item \label{problemIVTandMVTcos3xdiv3+sinx-3x} $f(x)=\cos^3 \left({\frac{x}{3}}\right) +\sin x-  3x$.
\end{enumerate}

\end{problem}
\textbf{Solution \ref{problemIVTandMVTx^3+4x+7}.}  $f(-2) = -9$ and $f(1) = 12$. Since $f(x)$ is continuous and has both negative and positive outputs, it must have a zero. In other words, for some $c$ between $-2$ and $1$, $f(c) = 0$. If there were solutions $x = a$ and $x = b$,  then we would have $f(a) = f(b)$, and Rolle's Theorem would guarantee that for some $x$-value, $f'(x) = 0$. However, $f'(x) = 3x^2 + 4$, which always positive and therefore is never 0. Therefore there cannot be 2 or more solutions. 

The above can be stated informally as follows. Note that $f'(x) = 3x^2 + 4$, which is always positive. Therefore, the graph of $f$ is increasing from left to right. So once the graph crosses the $x$-axis, it can never turn around and cross again, so there can only be a single zero (that is, a single solution to $f(x) = 0$).


\textbf{Solution \ref{problemIVTandMVTcos3xdiv3+sinx-3x}.} $f(5)= \cos^3 \left( \frac{5}{ 3} \right) +\sin 5-15 \leq 2-15=-13<0 $ (because $\cos a, \sin b\in [-1,1]$ for arbitrary $a, b$). Similarly $f(-5)=\cos^3\left(-\frac{5}{3}\right) +\sin (-5)+15 \geq 15-2>0$. Therefore by the Intermediate Value Theorem $f(x)=0$ has at least one solution in the interval $[-5,5]$.

Suppose on the contrary to what we are trying to prove, $f(x)=0$ has two or more solutions; call the first 2 solutions $a,b$. That means that $f(a)=f(b)=0$, so by the Mean value theorem, there exists a $c\in (a,b)$ such that $f'(c)=(f(a)-f(b))/(a-b)=(0-0)/(a-b)=0$. On the other hand we may compute:
\[ 
f'(x)=-3+\cos x-\cos^{2}\left(\frac{x}3\right)\sin\left(\frac{x}{3}\right) \leq -1<0,
\] 
where the first inequality follows from the fact that $\sin x,\cos x\in [-1,1]$. So we got that $f'(c)=0$ for some $c$ but at the same time $f'(x)<0$ for all $x$, which is a contradiction. Therefore $f(x)=0$ has exactly one solution. 


\begin{problem}
\begin{enumerate}[ref={\fcProblemRef}]
% Related Rates
\item \label{problemRelatedRatesFinddr/dtIfds/dtIsKnownS-sphere} A spherical soap bubble is slowly shrinking. If its surface area is decreasing at a rate of 50 square millimeters per
second, how quickly is the radius decreasing when the surface area is 1000 square millimeters?

\answer{$\frac{\diff r}{\diff t} = - \frac{ 50}{ 8 \pi \sqrt{\frac{250}{\pi}}} =- \frac{ \sqrt{10 } }{8 \sqrt{\pi}}$.}

\item \label{problemRelatedRatesCarAlongEllipticalTrack}A car drives along an elliptical track. The track can be modeled by the equation $x^2 + 5y^2 = 14$, where $x$ and $y$ are 
measured in kilometers of distance from the center of the track. As the car passes the point $(3, 1)$, the $x$-coordinate is 
increasing at a rate of $1.5$ km/min. How quickly is the $y$-coordinate changing at that point?

\answer{$\frac{\diff y}{\diff t} = -\frac{9}{10}$ km/min.}
\item Gravel is being dumped so that the it forms a cone with circular base. The diameter of the base of the cone equals the hight of the cone at any given moment. Use related rates to approximate how fast is the height of the pile increasing when it is 2 meters tall? 
\end{enumerate}
\end{problem}
\solution{\ref{problemRelatedRatesCarAlongEllipticalTrack}. 
Let $t$ denote time, and let $t_0$ be the point in time in which the measurements take place. We are given that $\frac{\diff x}{\diff t}_{|t=t_0}=1.5 km/min$,  $x_{|t=t_0}=3$, $y_{|t=t_0}=1$. The problem asks us to find $\frac{\diff y}{\diff t}_{|t=t_0}$. 

Compute:
\[
\begin{array}{rcll|l}
\displaystyle x^2+5y^2&=&14&&\displaystyle \text{apply }\frac{\diff }{\diff t}\\
\displaystyle 2x \frac{\diff x}{\diff t}+10 y \frac{\diff y}{\diff t}&=&0 &&\displaystyle \text{fix time }t=t_0 \\
\displaystyle 2 \cdot 3 \cdot 1.5 +10 \cdot 1\cdot \frac{\diff y}{\diff t}_{|t=t_0}&=&0\\
\displaystyle 10\frac{\diff y}{\diff t}_{|t=t_0}&=&-9\\
\displaystyle \frac{\diff y}{\diff t}_{|t=t_0}&=&\displaystyle -\frac{9}{10}\quad .\\
\end{array}
\]
The measurement unit of $\diff y$ is $km$, the measurement unit of  $t$ is minutes, therefore the answer is $-\frac{9}{10}$km/min.

}

\begin{problem}
\begin{enumerate}[ref={\fcProblemRef}]
% Area problems
\item 
\label{problemAreaBetweeny=2x^2,y=4+x^2} Find the area of the region bounded by the curves $y = 2x^2$ and $y = 4 + x^2$.

\answer{$\frac{32}{3}$}
\item \label{problemAreaBetweeny=2-x,x=4-y^2} Find the area of the region bounded by the curves $x = 4 - y^2$ and $y = 2 - x$.

\answer{$\frac92$}
\item \label{problemAreaBetweeny=x^2} Find the area of the region bounded by the curves $y=x^2$ and $y=2x^2+x-2$.

\answer{$\frac{9}{2}$}


\item \label{problemareabetweeny=x^2andy=2x^2+x-2}
\begin{itemize}
\item Sketch the region bounded by the curves $y=x^2$ and $y=2x^2+x-2$.

\psset{xunit=0.5cm, yunit=0.5cm}
\begin{pspicture}(-3.4,-3.4)(3,5.7)
\fcAxesStandardNoFrame{-3.5}{-3.5}{2.5}{5.5}
\fcGrid[linestyle=dashed, linewidth=0.5, linecolor=gray]{-3}{-3}{5}{8}{1}{1}{}
\rput[t](0.9,-0.2){$1$}
\fcLabels{3.5}{5.5}
%\psplot{-3}{2}{x x mul}
%\psplot{-3}{2}{x x mul 2 mul x -2 add add}
\end{pspicture}

\vskip 2cm


\item Find the area of the region.

\answer{$\frac{9}{2}$}
\end{itemize}
\item \label{problemAreaBetween-x^2+2x-1and-2x^2+3x+1}
~
\begin{itemize}
\item Sketch the region bounded by the curves $y=- x^{2}+2 x-1$ and $y=-2 x^{2}+3 x+1$. Make sure to indicate the points where the curves intersect.

\psset{xunit=0.5cm, yunit=0.5cm}
\begin{pspicture}(-3.5,-8.8)(3.7,5.7)
\fcAxesStandard{-3.5}{-8.4}{3.5}{5.5}
\fcGrid[linestyle=dashed, linewidth=0.5, linecolor=gray]{-2}{-8}{5}{13}{1}{1}{}
\rput[t](0.9,-0.2){$1$}
\fcLabels{3.5}{5.5}
%\psplot[linecolor=\fcColorGraph]{-1.3}{2.7}{x x -1 mul mul 2 x mul -1 add add}
%\psplot[linecolor=\fcColorGraph]{-1.3}{2.7}{x x -2 mul mul 3 x mul 1 add add}
\end{pspicture}
\item Find the area of the region.
\end{itemize}
\end{enumerate}

\end{problem}
\solution{\noindent \ref{problemAreaBetweeny=2-x,x=4-y^2}.
$x=4-y^2$ is a parabola (here we consider $x$ as a function of $y$). $y=-x+2$ implies that $x=2-y$ and so the two curves intersect when
\[
\begin{array}{rcl}
4-y^2&=&2-y\\
-y^2+y+2&=&0\\
-(y+1)(y-2)&=&0\\
y&=& -1\text{~or~}2\quad \quad .
\end{array}
\]
As $x=2-y$, this implies that $x=0$ when $y=2$ and $x=3$ when $y=-1$, or in other words the points of intersection are $(0,2)$ and $(3, -1)$. Therefore we the region is the one plotted below. Integrating with respect to $y$, we get that the area is
\[
\begin{array}{rcl}
A&=&\displaystyle \int\limits_{-1}^{2} \left|4-x^2-(-x+2) \right| \diff y = \int\limits_{-1}^2 \left(-y^2+y+2\right)\diff y \\
&=& \displaystyle \left[- \frac{y^3}3 +\frac{y^2}{2}+ 2y\right]_{-1}^2
=-\frac{8}{3}+2+4 -\left(-\frac{(-1)^3}{3} +\frac{ (-1)^2}{2}-2 \right)\\
&=&\displaystyle \frac{9}{2}\quad .
\end{array}
\]
\psset{xunit=0.5cm, yunit=0.5cm}
\begin{pspicture}(-3.500000, -5)(4.500000,5.5)
\psframe*[linecolor=white](-3.500000,-5)(4.500000,5)
\tiny
\pscustom*[linecolor=cyan]{
\psplot[linecolor=\fcColorGraph, plotpoints=1000]{0}{4}{4 x -1 mul add 0.5 exp }
\psplot[linecolor=\fcColorGraph, plotpoints=1000]{4}{3}{4 x -1 mul add 0.5 exp -1 mul }
}
\rput(-1.5,5){$y=- x+2$}
\psplot[linecolor=\fcColorGraph, plotpoints=1000]{-3.000000}{4.000000}{2 x -1 mul add }
%Function formula: - (- x+4)^{1/2}
\psplot[linecolor=\fcColorGraph, plotpoints=1000]{-3.000000}{4.000000}{4 x -1 mul add 0.5 exp -1 mul }
%Function formula: (- x+4)^{1/2}
\rput(2,2){$x=4-y^2$}
\psplot[linecolor=\fcColorGraph, plotpoints=1000]{-3.000000}{4.000000}{4 x -1 mul add 0.5 exp }
\psaxes[arrows=<->, ticks=none, labels=none](0,0) (-3.000000,-4.5)(4.5,4.5) %Function formula: - x+2
\end{pspicture}
}
\solution{\ref{problemareabetweeny=x^2andy=2x^2+x-2}

\textbf{Region plot.}
\psset{xunit=0.5cm, yunit=0.5cm}
\begin{pspicture}(-3,-3)(3,5.7)
\fcAxesStandard{-3.5}{-3.5}{2.5}{5.5}
\pscustom*[linecolor=\fcColorAreaUnderGraph]{%
\psplot{-2}{1}{x x mul}%
\psplot{1}{-2}{x x mul 2 mul x -2 add add}%
}%
\psplot{-3}{2}{x x mul}
\psplot{-3}{2}{x x mul 2 mul x -2 add add}
\fcGrid[linestyle=dashed, linewidth=0.5, linecolor=gray]{-3}{-3}{5}{8}{1}{1}{}
\rput[t](0.9,-0.2){$1$}
\fcLabels{3.5}{5.5}
\end{pspicture}

The intersection between the two parabolas are found via
\[
\begin{array}{rcl}
x^2&=&2x^2+x-2\\
x^2+x-2&=&0\\
(x-1)(x+2)&=&0\\
x=1&& x=-2\\
y=1&&y=4.
\end{array}
\]

\textbf{Area of the region.} 
\[
\begin{array}{rcll|l}
A&=&\displaystyle\int_{1}^{-2}\left|x^2-(2x^2+x-2) \right|\diff x&&x^2>(2x^2+x-2) \text{ for }x\in [-2,1] \text{ (from plot)}\\
&=&\displaystyle\int_{1}^{-2}\left(x^2-(2x^2+x-2) \right)\diff x\\
&=&\displaystyle \left[-\frac{1}{3} x^{3}-\frac{1}{2} x^{2}+2 x \right]_{-2}^1\\
&=&\displaystyle \frac{9}{2}.
\end{array}
\]
}
\solution{\ref{problemAreaBetween-x^2+2x-1and-2x^2+3x+1}

\textbf{Region plot.}

\psset{xunit=0.5cm, yunit=0.5cm}
\begin{pspicture}(-3.5,-8.8)(3.7,5.7)
\fcAxesStandard{-3.5}{-8.4}{3.5}{5.5}
\pscustom*[linecolor=cyan]{
\psplot{-1}{2}{x x -1 mul mul 2 x mul -1 add add}
\psplot{2}{-1}{x x -2 mul mul 3 x mul 1 add add}
}
\fcGrid[linestyle=dashed, linewidth=0.5, linecolor=gray]{-2}{-8}{5}{13}{1}{1}{}
\rput[t](0.9,-0.2){$1$}
\fcLabels{3.5}{7.5}
\psplot[linecolor=\fcColorGraph]{-1.3}{2.7}{x x -1 mul mul 2 x mul -1 add add}
\psplot[linecolor=\fcColorGraph]{-1.3}{2.7}{x x -2 mul mul 3 x mul 1 add add}
\end{pspicture}

The intersections between the two parabolas are found via
\[
\begin{array}{rcl}
-2x^2+3x+1&=&-x^2+2x-1\\
-x^2+x+2&=&0\\
-(x+1)(x-2)&=&0\\
x=-1&\text{or}& x=2\\
y=-4&&y=-1.
\end{array}
\]

\textbf{Area of the region.} 
\[
\begin{array}{rcll|l}
A&=&\displaystyle\int_{-1}^{2}\left|-2x^2+3x+1-(-x^2+2x-1) \right|\diff x&& \begin{array}{l} -2x^2+3x+1>-x^2+2x-1 \\ \text{ for }x\in [-1,2] \text{ (from plot)}\end{array}\\
&=&\displaystyle\int_{-1}^{2}\left(-2x^2+3x+1-(-x^2+2x-1) \right)\diff x\\
&=&\displaystyle\int_{-1}^{2}\left(-x^2+x+2 \right)\diff x\\
&=&\displaystyle \left[-\frac{1}{3} x^{3}+\frac{1}{2} x^{2}+2 x \right]_{-1}^2\\
&=&\displaystyle \left(-\frac{1}{3} 2^{3}+\frac{1}{2} 2^{2}+2 \cdot 2 \right)-\left( -\frac{1}{3} (-1)^{3}+\frac{1}{2} (-1)^{2}+2 (-1) \right)\\
&=&\displaystyle \frac{9}{2}.
\end{array}
\]
}




\begin{problem}
\begin{enumerate}[ref={\fcProblemRef}]
% Volume problems
\item 
\label{problemVolumeRegionBoundedByy=2x^2-x+1,y=x^2+1rotatedAroundx=0} Consider the region bounded by the curves $y = 2x^2-x+1$ and $y =x^2+1$. What is the volume of the solid obtained by rotating this region about the line $x = 0$?

\answer{$\frac{2}{5}\pi$.} 
\item Consider the region bounded by the curves $y = 1-x^2$ and $y =0$. What is the volume of the solid obtained by
rotating this region about the line $y = 0$?

\answer{$\frac{16 \pi}{15}$}
 
\item Consider the region bounded by the curves $y = x^2$ and $x = y^2$. What is the volume of the solid obtained by
rotating this region about the line $x = 2$?

\answer{ $\frac{31 \pi}{30}$}
\item \label{problemVolumeAreay=-x^2+2andy=0rotatedAroundy=0andy=-3}
Set up \textsc{but do not evaluate} an integral to calculate the volume of the solid obtained by rotating the region bounded by $y=-x^2+2$ and $y=0$ about the given line. 

\begin{itemize}
\item The $x$ axis.
\item The line $y=-3$.
\end{itemize}

\item \label{problemVolumeRevolution-x^2+1aroundy=0andy=-4}
Set up \textsc{but do not evaluate} an integral to calculate the volume of the solid obtained by rotating the region bounded by $y=-x^2+1$ and $y=0$ about the given line. 
\begin{itemize}
\item The $x$ axis.
\item The line $y=-4$.
\end{itemize}


\end{enumerate}

\end{problem}

\begin{problem}
State the Fundamental Theorem of Calculus (both parts).
\end{problem}
\end{document}
