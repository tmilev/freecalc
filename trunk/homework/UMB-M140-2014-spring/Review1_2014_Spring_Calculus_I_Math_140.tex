\documentclass{article}
\usepackage{../homework-problems-UMB}

\toggletrue{solutions}
%\togglefalse{solutions}
\toggletrue{answers}

\begin{document}
\begin{center}
\Large
Review problems Test 1\\ Math 140 Calculus I \\ \normalsize Instructor: Todor Milev
\end{center}

%\noindent \textbf{Name:} \hfill{~}
%\begin{tabular}{c|c|c|c|c|c|c|c|c||c}
%Problem&1 &2&3&4&5&6&7&8& $\sum$\\ \hline
%Score&&&&&&&&&\\ \hline
%Max&20&20&20&20&20&10&20&20&150
%\end{tabular} 

\noindent The exam is closed books, no calculators will be allowed. The time for work will be 50 minutes. The problems on the exam will be similar to the problems in the review sheet. You will be asked a theoretical question (see the last two problems).

\begin{enumerate}
\item Find the expressions $(f\circ g)(x)$ and $(g\circ f)(x)$. Simplify your answer to a single fraction. 
\begin{enumerate}
\item $\displaystyle  f(x)= \frac{3x-5}{x-2}$, $\displaystyle g(y)=\frac{y-2 }{y-4} $. \answer{ $(f\circ g)(x)=\frac{-2 x+14}{- x+6}$, $(g\circ f)(x)=\frac{x-1}{- x+3}$}
\item $\displaystyle  f(x)= \frac{x-3}{x+2}$, $\displaystyle g(y)=\frac{y+3 }{y-4} $. \answer{ $(f\circ g)(x)=\frac{-2 (x)+15}{3 x-5}$, $(g\circ f)(x)=\frac{4 x+3}{-3 x-11}$}
\end{enumerate}

\item Find all solutions in the interval $[0,2\pi]$ of the equation.
\begin{enumerate}
\item $2\cos^{2}x-(1+\sqrt{2})\cos x+\frac{\sqrt{2}}{2}=0$. 

\answer{ $x=\frac{\pi}{4}, \frac{\pi}{3},\frac{5\pi}{3}, \frac{7\pi}4 $}.

\solution{ %
Set $\cos x=u$. Then 
\[
2\cos^2 x- (1+\sqrt{2})\cos x+\frac{\sqrt 2}2=0 
\] 
becomes 
\[2u^2-(1+\sqrt{2})u+\frac{\sqrt{2}}{2}=0.
\] 
This is a quadratic equation in $u$ and therefore has solutions
\[
\begin{array}{rcl}
u_1, u_2&=& \displaystyle \frac{ 1+\sqrt{2}\pm\sqrt{ (1+\sqrt{2})^2-4 \sqrt{2} } }4\\
&=&\frac{1+\sqrt{2}\pm\sqrt{1-2\sqrt{2}+2} }4\\
&=&\frac{1+\sqrt{2}\pm \sqrt{(1-\sqrt{2})^2}}4\\
&=&\frac{1+\sqrt{2}\pm (1-\sqrt{2}) }4=\doublebrace{\frac{1}2 }{ \mathrm{or}} {\frac{\sqrt{2}}{2}}{}
\end{array}
\]
Therefore $u=\cos x= \frac12$ or $u=\cos x=\frac{\sqrt{2}}2$, and, as $x$ is in the interval $[0,2\pi]$, we get $x=\frac{\pi}{3}, \frac{5\pi}{3}$ (for $\cos x=\frac12$) or $x=\frac{\pi }4 ,\frac{7\pi}4$ (for $\cos x=\frac{\sqrt{2}}{2}$).
} % end solution
\item $\sqrt {3} \sin x= \sin (2x)$.  \answer{$x=\frac{\pi}{6}, \frac{11\pi}{6}, 0, \pi, 2\pi $}.
\end{enumerate}

\item Evaluate the limit if it exists.
\begin{enumerate}
\item $\displaystyle \lim\limits_{x\to 1} \frac{3x^2+4x-7}{x^3-x}$ \answer{$5 $.}
\item $\displaystyle \lim\limits_{x\to -1} \frac{2x^2-3x-5}{x^3+1}$ \answer{$ -\frac{7}{3}$.}
\end{enumerate}

\item Evaluate the limit if it exists.
\begin{enumerate}
\item $\displaystyle \lim\limits_{x\to 3^+} \frac{\sqrt{\frac{x^2}{9}-1 }}{2x^2 -3x-9 }$. \answer{$ \infty$.}
\solution{ 
We have that 
\[
\begin{array}{rcl}
\displaystyle \lim\limits_{x\to 3^+}\frac{\sqrt{\frac{x^2}9-1} }{2x^2-3x-9}&=&\displaystyle  \lim\limits_{x\to 3^+} \frac{\sqrt{(\frac{x}3-1)(\frac{x}3+1)} }{2(x+\frac{3}2)(x-3)}= \lim\limits_{x\to 3^+}
\frac{\left(\frac{1}3 (x-3)(\frac{x}3+1) \right)^{ \frac{1}2}}{ 2(x+\frac{3}2)(x-3)} \\~\\
&=&\displaystyle  \lim\limits_{x\to 3^+} \frac{\sqrt{\frac{1}3\left(\frac{x}3+1\right)} }{2\left(x+\frac{3}2\right)(x-3)^{\frac12}}= \lim\limits_{x\to 3^+} \frac{\sqrt{\underbrace{ \frac{1}3\left(\frac{x}3+1\right)}_{\to\frac23 }} }{\underbrace{2\left(x+\frac{3}2\right)}_{\to 9}\underbrace{(x-3)^{\frac12}}_{\to 0^+}}=\infty,
\end{array}
\]
where the latest term is $+\infty$ because it is of the form $\frac{(+)}{(+)(+)}$.
}
\item $\displaystyle \lim\limits_{x\to -2^-} \frac{\sqrt{\frac{x^2}{4}-1 }}{2x^2 +3x-2 }$. \answer{$ \infty$.}
\end{enumerate}


\item ~
\begin{enumerate}[ref={\fcProblemRef}]
\item \label{problemIVTtoshowx^2+13x+14=sinx-has-solutions} (1) Solve the equation $x^2+13x+41=1$.  (2) Use the intermediate value theorem to prove that the equation $x^2+13x+41=\sin  x$ has at least two solutions, lying between the two numbers found in (1).
\item (1) Solve the equation $x^2-15x+55=1$.  (2) Use the intermediate value theorem to prove that the equation $x^2-15x+55=\cos  x$ has at least two solutions, lying between the two numbers found in (1).
\end{enumerate}

\item Evaluate the limit if it exists.
\begin{enumerate}
\item $\lim\limits_{x\to-\infty}\sqrt{x^2+x}-\sqrt{x^2-x}$. \answer{$-1 $.}
\item $\lim\limits_{x\to\infty}\sqrt{x^2+2x}-\sqrt{x^2-2x} $. \answer{ $2 $.}
\end{enumerate}

\solution{
\[ \begin{array}{rcl}
\displaystyle\lim_{x\to -\infty} \sqrt{x^2+x}-\sqrt{x^2-x} &=&\displaystyle\lim_{x\to -\infty} \left(\sqrt{x^2+x}-\sqrt{x^2-x}\right) \frac{ \left(\sqrt{x^2+x}+\sqrt{x^2-x}\right) }{\left( \sqrt{x^2+x}+\sqrt{x^2-x}\right)}
\\
&=&\displaystyle \lim_{x\to -\infty} \frac{x^2+x-(x^2-x) }{\sqrt{x^2+x}+\sqrt{x^2-x} } = \lim_{x\to -\infty} \frac{2x \frac{1}{x} }{\left(\sqrt{x^2+x}+\sqrt{x^2-x} \right)\frac{1}{x}} 
\\&=&\displaystyle \lim_{x\to -\infty} \frac{2}{\frac{\sqrt{x^2+x}}x+\frac{\sqrt{x^2-x}}x }= \lim_{x\to -\infty} \frac{2}{ {\color{red}-} \sqrt{\frac{x^2+x}{x^2}} {\color{red}-} \sqrt{\frac{x^2-x}{x^2}} }\\
&=&\displaystyle \lim_{x\to -\infty} \frac{2}{ - \sqrt{1+\frac1x} - \sqrt{1-\frac1x} }=\frac{2}{-\sqrt{1+0}-\sqrt{1-0}}=-1
\end{array}
\]
The sign highlighted in red arises from the fact that, for negative $x$, we have that $ x={\color{red}-}\sqrt{x^2}$.
}
\item Find the horizontal and vertical asymptotes of the curve
\begin{enumerate}
\item $y=\frac{2x}{\sqrt{x^2+x+3}-3}$. \answer{Vertical: $x=2, x=-3$, horizontal: $y=2, y=-2$}
\solution{
\textbf{Vertical asymptotes.} A function $f(x)$ has a vertical asymptote at $x=a$ if $\lim\limits_{x\to a} f(x)=\pm \infty$. 

The function is algebraic, and therefore, if it is defined, has a finite limit (i.e., no asymptote). Therefore the function can have vertical asymptotes only for those $x$ for which $f(x)$ is not defined. The function is not defined for $\sqrt{x^2+x+3}-3=0$, which has two solutions, $x=2$ and $x=-3$. These are precisely the vertical asymptotes: indeed, 
\[
\lim\limits_{x\to 2^+} \frac{2x}{\sqrt{x^2+x+3}-3}=\infty \quad \quad \quad 
\lim\limits_{x\to 2^-} \frac{2x}{\sqrt{x^2+x+3}-3}=-\infty 
\]
and
\[
\lim\limits_{x\to -3^+} \frac{2x}{\sqrt{x^2+x+3}-3}=\infty \quad \quad \quad 
\lim\limits_{x\to -3^-} \frac{2x}{\sqrt{x^2+x+3}-3}=-\infty 
\]

\textbf{Horizontal asymptotes.} A function $f(x)$ has a horizontal asymptote if $\lim\limits_{x\to \pm\infty} f(x)$ exists. If that limit exists, and is some number, say, $N$, then $y=N$ is the equation of the corresponding asymptote.

Consider the limit $x\to -\infty$. We have that 
\[
\lim\limits_{x\to -\infty} \frac{2x}{\sqrt{x^2+3x+3}-3}= \lim\limits_{x\to \pm \infty} \frac{2}{\frac{\sqrt{x^2+x+3}}x-\frac3x}=\lim\limits_{x\to \pm \infty} \frac{2}{-\sqrt{\frac{x^2+3x+3}{x^2}}-\frac3x} =-2\quad . 
\]
Therefore $y=-2$ is a horizontal asymptote. 

The case $x\to \infty$, is handled similarly and yields that $y=2$ is a horizontal asymptote.
}

\item $y=\frac{3x^2}{\sqrt{x^2+2x+10}-5}$. \answer{Vertical: $x=3, x=-5$, horizontal: none.}
\end{enumerate}
\item 
\begin{enumerate}
\item State the Intermediate Value Theorem. (See Lecture 5).
\item Give the $\varepsilon,\delta$-definition of limit. (You are asked to reproduce the sentence that involves the symbols $\varepsilon,\delta$ from Lecture 4).
\end{enumerate}
\end{enumerate}
\end{document}





\end{document}