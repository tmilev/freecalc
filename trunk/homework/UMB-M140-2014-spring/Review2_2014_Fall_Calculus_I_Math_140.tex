\documentclass{article}
\ProvidesPackage{homework-problems-UMB}
\addtolength{\hoffset}{-3.5cm}
\addtolength{\textwidth}{6.8cm}
\addtolength{\voffset}{-3.3cm}
\addtolength{\textheight}{6.3cm}
\usepackage{../homework-problems} %warnign folder paths are relative to the file that uses the includepackage

\renewcommand{\answer}[1]{\iftoggle{answers}{ \hfill{~} \rotatebox{180}{\tiny answer: #1}}{} }
\renewcommand{\pointsii}[1]{}


\newtheorem{problem}{Problem}
\toggletrue{solutions}
%\togglefalse{solutions}
\toggletrue{answers}

%\title{Exam II\\ Math 140 Calculus I \\Instructor: Todor Milev}
%\date{April 9 2012}
\begin{document}
\begin{center}
\Large
Review problems Test 2\\ Math 140 Calculus I \\ \normalsize Instructor: Todor Milev
\end{center}
%\textbf{Name:}


\noindent The exam is closed books, no calculators will be allowed. The material covered will be the material from Lecture 8 to Lecture 17 (including Lectures 8 and 17). The time for work will be 50 minutes. The problems on the exam will be similar to the problems in the review sheet. You will be asked a theoretical question (see the last problem).

\begin{enumerate}
\item (Lecture 7) Find the inverse function $f^{-1}$. Plot roughly by hand $y=f(x)$. Using the plot of $y=f(x)$, plot roughly by hand $f^{-1}(x)$. Indicate the relationship between the graph of $f(x)$ and $f^{-1}(x)$.
\begin{enumerate}
\item $f(x)= x^2+2x-2$,\quad \quad \quad $ x\geq -1$. 
\answer{
$f^{-1}(x)=\sqrt{x+3}-1$
\psset{xunit=0.2cm, yunit=0.2cm}
\begin{pspicture}(-3, -5)(6,5) 
\psframe*[linecolor=white](-3,-5)(6,5) 
\tiny 
\psaxes[ticks=none, labels=none]{<->}(0,0)(-3,-4.5)(6,4.5)
\psLabels{6}{5}
%Function formula: (x+3)^{1/2}-1 
\psplot[linecolor=\psColorGraph, plotpoints=1000]{-3}{6}{-1 3 x add 0.5 exp add }
%Function formula: x^{2}+2 x-2 
\psplot[linecolor=\psColorGraph, plotpoints=1000]{-1}{2}{-2 x 2 mul add x 2 exp add }
\end{pspicture} 
}
\item $f(x)= x^2+x-2$, \quad \quad \quad $ x\geq -\frac{1}{2}$.

\answer{
$f^{-1}(x)=\frac{ \sqrt{4 x+9}-1}2$
\psset{xunit=0.2cm, yunit=0.2cm}
\begin{pspicture}(-2.25, -5)(4,5) 
\psframe*[linecolor=white](-2.25,-5)(4,5) 
\tiny 
\psaxes[ticks=none, labels=none]{<->}(0,0)(-2.25,-4.5)(4,4.5)
\psLabels{4}{5}
%Function formula: 1/2 (4 x+9)^{1/2}-1/2 
\psplot[linecolor=\psColorGraph, plotpoints=1000]{-2.25}{4}{-0.5 9 x 4 mul add 0.5 exp 0.5 mul add }
%Function formula: x^{2}+x-2 
\psplot[linecolor=\psColorGraph, plotpoints=1000]{-0.5}{2}{-2 x add x 2 exp add }
\end{pspicture} 
}
\end{enumerate}
\item (Lecture 8) Solve the equation.
\begin{enumerate}
\item $e^{4x}+3e^{2x}-4=0$. \answer{$x=0$}
\item $4^{3x}-2^{3x+2}-5=0$. \answer{$x=\frac{\log_25}{3}$}
\end{enumerate}

\item (Lectures 10-14)
Compute the derivative of the function.
\begin{enumerate}[ref={\fcProblemRef}]
\item $\displaystyle f(x)=\frac{1+x }{1+\frac{2}x}$. 

\answer{$\frac{x^{2}+4 x+2}{(2+x)^{2}}$}

\item $\displaystyle f(x)=\frac{1+x }{1+\frac{3}x}$. 

\answer{$\frac{x^{2}+6 x+3}{(3+x)^{2}}$}
\end{enumerate}

\item (Lectures 10-14) Compute the derivative of the function.
\begin{enumerate}
\item $2^{3^x}$.
\answer{$2^{3^{x}} 3^{x} (\ln{}2)  (\ln{}3) $}
\item $3^{2^x}$.
\answer{$ 3^{2^{x}} 2^{x}(\ln{}2)(\ln{}3)$}
\end{enumerate}

\item (Lectures 10-14) Compute the derivative of the function.
\begin{enumerate}
\item $\sec^2 (3x^2)$. \answer{$12 \frac{x\sin{}(3 x^{2}) }{\left(\cos{}\left(3 x^{2}\right)\right)^{3}}$}
\item $\csc^2 (3x^2)$. \answer{$-12 \frac{ x  \cos{}\left(3 x^{2}\right) }{\left(\sin{}\left(3 x^{2}\right)\right)^{3}}$}
\end{enumerate}
\item (Lecture 15) Use implicit differentiation to express $\frac{dy}{dx}$ via $y $ and $x$, where $x$ and $y$ satisfy the following relation.
\begin{enumerate}
\item  $x^4(x+y)=y^2(3x-y)$.
\answer{ $ \frac{dy}{dx}=\frac{-4 x^{3} y+3 y^{2}-5 x^{4}}{3 y^{2}+x^{4}-6 y x}$}
\item $2x^3+x^2y-xy^3=2$.
\answer{$\frac{dy}{dx}=\frac{y^{3}-6 x^{2}-2 x y}{-3 x y^{2}+x^{2}}$}
\end{enumerate}

\item (Lecture 17) 
\begin{problem}(page 257)
Find the dimensions of a rectangle with area 1000 $m^2$ whose perimeter is as small as possible.
\end{problem}
\begin{problem}(pages 256-259)
A box with an open top is to be constructed from a square piece of cardboard, 1m wide, by cutting out a square from each of the four corners and bending up the sides. Find the largest volume that such a box can have.
\end{problem}
\begin{problem}(pages 256-259)
A right circular cylinder is inscribed in a sphere of radius $r$. Find the largest possible volume of such a cylinder.
\end{problem}
\begin{problem}(pages 256-259)
A cone-shaped drinking cup is made from a circular piece of paper of radius $r$ by cutting out a sector and joining the edges $OA$ and $OB$. Find the maximum capacity of such a cup.
\begin{pspicture}(0,0)(1,1)
\pswedge*[linecolor=cyan](0,0){1}{120}{60}
\pswedge[linecolor=blue](0,0){1}{120}{60}
\rput[t] (0,-0.2){$O$}
\rput[b] (0.5,1){$B$}
\rput[b] (0.15,0.4){$r$}
\rput[b] (-0.5,1){$A$}
\end{pspicture}
\end{problem}
\item (Lecture 13)
\begin{enumerate}
\item State the quotient rule for computing the derivative of $\left(\frac{f}{g}\right)'$. Derive the quotient rule
using the chain rule, the negative power rule and the product rule.
\item State the power rule for computing the derivative
$\left(x^r\right)'$ for an arbitrary real number $r$ and $x>0$. Derive the power rule using the chain rule, the rule $\left(e^{x}\right)'=e^x$, the constant multiple derivative rule and the logarithm derivative rule $(\ln x)'=\frac{1}x$.
\end{enumerate}
\end{enumerate}

%\begin{tabular}{c|c|c|c|c|c|c|c|c||c}
%Problem&1 &2&3&4&5&6&7&8& $\sum$\\ \hline
%Score&&&&&&&&&\\ \hline
%Max&20&20&20&20&20&10&20&20&150
%\end{tabular}
\end{document}
