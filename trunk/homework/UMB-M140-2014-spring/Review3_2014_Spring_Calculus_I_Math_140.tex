\documentclass{article}
\usepackage{../homework-problems-UMB}

\newtheorem{problem}{Problem}
\toggletrue{solutions}
%\togglefalse{solutions}
\toggletrue{answers}

%\title{Exam II\\ Math 140 Calculus I \\Instructor: Todor Milev}
%\date{April 9 2012}
\begin{document}
\begin{center}
\Large
Review problems Test 3\\ Math 140 Calculus I \\ \normalsize Instructor: Todor Milev
\end{center}
%\textbf{Name:} 


\noindent The exam is closed books, no calculators will be allowed.

\begin{enumerate}
\item (Lecture 19) Use the Intermediate Value theorem and the Mean Value Theorem/Rolle's Theorem to prove that the function has \textbf{exactly one} real root.
\begin{enumerate}
\item \label{problemIVTandMVTx^3+4x+7} $f(x)=x^3+4x+7$.
\item $f(x)= x^3 +x^2+x+1$.
\item \label{problemIVTandMVTcos3xdiv3+sinx-3x} $f(x)=\cos^3 \left({\frac{x}{3}}\right) +\sin x-  3x$.
\end{enumerate}

\item (Lecture 19) \begin{enumerate}

\item \label{problem-linearization-estimate0.9999power2014} Use a linear approximation to estimate $(0.9999)^{2014}$. 

\answer{$(0.9999)^{2014} \approx 0.7986$.}

\item Find the linearization of $f(x)=\ln x$ at $a=1$ and use it to approximate $\ln 1.01$.

\answer{ $f(x)\approx f(1)+f'(1)(x-1)=x-1 $, $\ln 1.01\approx 0.01$. }
\end{enumerate}
\item (Lecture 20) Estimate the integral using a Riemann sum using the indicated sample points and interval length.
\begin{enumerate}
% Riemann sums
\item $\displaystyle \int\limits_0^2 \frac{\diff x}{1+x^3}$. Use four intervals of equal width, choose the sample point to be the left endpoint of each interval. 

\answer{ $\Delta x = 0.5$ and $f(x) = \frac{1}{1+x^3}$. Thus $\displaystyle \int\limits_0^2 f(x) \diff x  \approx \Delta x\left(f{}\left(0\right)+f{}\left(1\right)+f{}\left(\frac{1}{2}\right)+f{}\left(\frac{3}{2}\right)\right)=\frac{1649}{1260}\approx 1.30873$.}

\item $\displaystyle \int\limits_{-2}^{0} \frac{\diff x}{x^4+1} $. Use four intervals of equal width, choose the sample point to be the right endpoint. 

\answer{ $\Delta x = 0.5$ and $f(x) = \frac{1}{1+x^3}$. Thus $\displaystyle \int\limits_0^2 f(x) \diff x  \approx \Delta x\left(f{}\left(-\frac{3}{2}\right)+f{}\left(-1\right)+f{}\left(-\frac{1}{2}\right)+f{}\left(0\right)\right)=\frac{8595}{6596}\approx 1.303062$.}

\item\label{problemRiemannSum-1div1plusxsquared} $\displaystyle \int\limits_{-3.5}^{-0.5} \frac{\diff x}{x^2+1} $. Use three intervals of equal width, choose the sample point to be the midpoint of each interval. 

\answer{ $\Delta x = 1$ and $f(x) = \frac{1}{x^2+1}$. Thus $\displaystyle \int\limits_{-3.5}^{-0.5} f(x) \diff x  \approx \Delta x\left(f{}\left(-3\right)+f{}\left(-2\right)+f{}\left(-1\right)\right)=\frac{4}{5}=0.8$.}

\end{enumerate}



\item (Lecture 21) \begin{enumerate}
\item Find $f(x)$ if $f'(x) = 3 + \frac{1}{x}$ and $f(1) = 2$.

\answer{$f(x) = 3x + \ln |x| - 1$}
\item Find $f(x)$ if $f'(x) = x - \sin x$ and $f(0) = 7$.

\answer{$f(x) = \frac{x^2}{2} + \cos x + 6$}
\end{enumerate}

\item (Lecture 21) Integrate (definite or indefinite).
\begin{enumerate}
\item $\displaystyle\int\limits_{1}^{8} \frac{t-t^{\frac{1}{3}}+ 2}{ t^{\frac{4}{3}}} \diff t\quad .$

\answer{$- \ln8+\frac{15}{2}$}
\item $\displaystyle\int\limits_{1}^{4} \left(x+\sqrt{x}\right)^2 dx\quad .$

\answer{$\frac{533}{10}$}
\item \label{problemIntegrate(sqrt[3]x-x^(1/2)+1)/xdx}
$\displaystyle\int \frac{\sqrt[3]{x}-x^{\frac{1}{2}}+1 }{x}\diff x$.

\answer{$ -2 \sqrt{x}+3 x^{\frac{1}{3}}+\ln \left|x\right| +C$}

\item \label{problemint(sqrt[3](x)-1)/xdx}

$\displaystyle\int \frac{\sqrt[3]{x}-1 }{x}\diff x$.

\answer{$3x^{\frac{1}{3} }-\ln |x|+C$}
\end{enumerate}

\item (Lecture 22) Evaluate the indefinite integral 
\begin{enumerate}
\item $\displaystyle\int \tan (2x) \diff x$.
\answer{$-\frac{1}{2}\ln|\cos (2x) | +C$}
\item $\displaystyle \int \cot \left(\frac{x}{2}\right) \diff x$
\answer{$2\ln\left|\sin \left(\frac{x}{2}\right) \right|+C$}
\end{enumerate}

\item (Lecture 22) Evaluate the definite integral 
\begin{enumerate}
\item \label{problemIntx/(2x^2+1)} $\displaystyle\int\limits_{1}^{2} \frac{x}{2x^2+1 }  \diff x$.

\answer{$\frac14 \ln 3$}
\item $\displaystyle\int\limits_{0}^{\frac{1}4}\frac{x }{\sqrt{1-3x^2}}\diff x$.

\answer{$\frac{1}3\left(1-\sqrt{\frac{13}{16}} \right)$}
\end{enumerate}


\item (Lecture 22) Differentiate $f(x)$ using the Fundamental Theorem of Calculus part 1.
\begin{multicols}{2}
\begin{enumerate}[ref={\fcProblemRef}]
\item $\displaystyle f(x) = \int\limits_1^x \sin \left(t^2\right)  \diff t$

\answer{$\displaystyle \sin \left(x^2\right)$} 
\item \label{problemd/dxint_1^x(t-sqrt(t))dt}

$\displaystyle f(x)=\int_{1}^x\left(t-\sqrt{t}\right)\diff t $.

\answer{$x-\sqrt{t}$}
\item \label{problemDifferentiateFTC1int_x^1(2+t^4)^5dt}  ${\displaystyle f(x) = \int\limits_x^1 (2+t^4)^5 \; \diff t}$

\answer{${\displaystyle -\left(2+x^4\right)^5}$} 

\item $\displaystyle f(x)=\int\limits_{0}^{x^2} t^2\diff t $.

\answer{$f'(x)=2x^5$ }

\item \label{problemd/dx(int_(ln x)^(e^x)t^3dt)} $\displaystyle f(x)=\int\limits_{\ln x}^{e^x} t^3\diff t $.

\answer{$f'(x)=e^{4x}-\frac{(\ln x)^3}{x}$ }

\item \label{problemd/dxint_1^x(sqrt(t)-t^(1/3))dt}
$\displaystyle f(x)=\int_{1}^x\left(\sqrt{t}- \sqrt[3]{t}\right)\diff t$.

\answer{$ \sqrt{x}-\sqrt[3]{x}$}
\item \label{problemd/dxint_1^(1/(x+1))sin(t^2)dt}

$\displaystyle f(x)=\int_{1}^{\frac{1}{x+1} }\sin \left(  t^2\right) \diff t$.

\answer{}
\item \label{problemd/dxint_1^(1/(1+x))cos(t^2)dt}

$\displaystyle f(x)= \int_{1}^{\frac{1}{x+1} }\cos  \left(  t^2\right) \diff t$.

\answer{$-\frac{1}{(x+1)^2}\cos \left( \frac{1}{(x+1)^2}\right) $}
\item ${\displaystyle f(x) = \int_{0}^{x^3} \cos^2 t \; \diff t}$

\answer{${\displaystyle 3x^2 \cos^2\left(x^3\right)}$} 

\end{enumerate}
\end{multicols}
\item (Lecture 22) State the Fundamental Theorem of Calculus (both parts). (Lecture 22)

\item (Lecture 23) \begin{enumerate}[ref={\fcProblemRef}]
% Area problems
\item 
\label{problemAreaBetweeny=2x^2,y=4+x^2} Find the area of the region bounded by the curves $y = 2x^2$ and $y = 4 + x^2$.

\answer{$\frac{32}{3}$}
\item \label{problemAreaBetweeny=2-x,x=4-y^2} Find the area of the region bounded by the curves $x = 4 - y^2$ and $y = 2 - x$.

\answer{$\frac92$}
\item \label{problemAreaBetweeny=x^2} Find the area of the region bounded by the curves $y=x^2$ and $y=2x^2+x-2$.

\answer{$\frac{9}{2}$}


\item \label{problemareabetweeny=x^2andy=2x^2+x-2}
\begin{itemize}
\item Sketch the region bounded by the curves $y=x^2$ and $y=2x^2+x-2$.

\psset{xunit=0.5cm, yunit=0.5cm}
\begin{pspicture}(-3.4,-3.4)(3,5.7)
\fcAxesStandardNoFrame{-3.5}{-3.5}{2.5}{5.5}
\fcGrid[linestyle=dashed, linewidth=0.5, linecolor=gray]{-3}{-3}{5}{8}{1}{1}{}
\rput[t](0.9,-0.2){$1$}
\fcLabels{3.5}{5.5}
%\psplot{-3}{2}{x x mul}
%\psplot{-3}{2}{x x mul 2 mul x -2 add add}
\end{pspicture}

\vskip 2cm


\item Find the area of the region.

\answer{$\frac{9}{2}$}
\end{itemize}
\item \label{problemAreaBetween-x^2+2x-1and-2x^2+3x+1}
~
\begin{itemize}
\item Sketch the region bounded by the curves $y=- x^{2}+2 x-1$ and $y=-2 x^{2}+3 x+1$. Make sure to indicate the points where the curves intersect.

\psset{xunit=0.5cm, yunit=0.5cm}
\begin{pspicture}(-3.5,-8.8)(3.7,5.7)
\fcAxesStandard{-3.5}{-8.4}{3.5}{5.5}
\fcGrid[linestyle=dashed, linewidth=0.5, linecolor=gray]{-2}{-8}{5}{13}{1}{1}{}
\rput[t](0.9,-0.2){$1$}
\fcLabels{3.5}{5.5}
%\psplot[linecolor=\fcColorGraph]{-1.3}{2.7}{x x -1 mul mul 2 x mul -1 add add}
%\psplot[linecolor=\fcColorGraph]{-1.3}{2.7}{x x -2 mul mul 3 x mul 1 add add}
\end{pspicture}
\item Find the area of the region.
\end{itemize}
\end{enumerate}


\item (Lecture 24)
\begin{enumerate}[ref={\fcProblemRef}]
% Volume problems
\item 
\label{problemVolumeRegionBoundedByy=2x^2-x+1,y=x^2+1rotatedAroundx=0} Consider the region bounded by the curves $y = 2x^2-x+1$ and $y =x^2+1$. What is the volume of the solid obtained by rotating this region about the line $x = 0$?

\answer{$\frac{2}{5}\pi$.} 
\item Consider the region bounded by the curves $y = 1-x^2$ and $y =0$. What is the volume of the solid obtained by
rotating this region about the line $y = 0$?

\answer{$\frac{16 \pi}{15}$}
 
\item Consider the region bounded by the curves $y = x^2$ and $x = y^2$. What is the volume of the solid obtained by
rotating this region about the line $x = 2$?

\answer{ $\frac{31 \pi}{30}$}
\item \label{problemVolumeAreay=-x^2+2andy=0rotatedAroundy=0andy=-3}
Set up \textsc{but do not evaluate} an integral to calculate the volume of the solid obtained by rotating the region bounded by $y=-x^2+2$ and $y=0$ about the given line. 

\begin{itemize}
\item The $x$ axis.
\item The line $y=-3$.
\end{itemize}

\item \label{problemVolumeRevolution-x^2+1aroundy=0andy=-4}
Set up \textsc{but do not evaluate} an integral to calculate the volume of the solid obtained by rotating the region bounded by $y=-x^2+1$ and $y=0$ about the given line. 
\begin{itemize}
\item The $x$ axis.
\item The line $y=-4$.
\end{itemize}


\end{enumerate}



\end{enumerate}



\textbf{Solution \ref{problemIVTandMVTx^3+4x+7}.}  $f(-2) = -9$ and $f(1) = 12$. Since $f(x)$ is continuous and has both negative and positive outputs, it must have a zero. In other words, for some $c$ between $-2$ and $1$, $f(c) = 0$. If there were solutions $x = a$ and $x = b$,  then we would have $f(a) = f(b)$, and Rolle's Theorem would guarantee that for some $x$-value, $f'(x) = 0$. However, $f'(x) = 3x^2 + 4$, which always positive and therefore is never 0. Therefore there cannot be 2 or more solutions. 

The above can be stated informally as follows. Note that $f'(x) = 3x^2 + 4$, which is always positive. Therefore, the graph of $f$ is increasing from left to right. So once the graph crosses the $x$-axis, it can never turn around and cross again, so there can only be a single zero (that is, a single solution to $f(x) = 0$).


\textbf{Solution \ref{problemIVTandMVTcos3xdiv3+sinx-3x}.} $f(5)= \cos^3 \left( \frac{5}{ 3} \right) +\sin 5-15 \leq 2-15=-13<0 $ (because $\cos a, \sin b\in [-1,1]$ for arbitrary $a, b$). Similarly $f(-5)=\cos^3\left(-\frac{5}{3}\right) +\sin (-5)+15 \geq 15-2>0$. Therefore by the Intermediate Value Theorem $f(x)=0$ has at least one solution in the interval $[-5,5]$.

Suppose on the contrary to what we are trying to prove, $f(x)=0$ has two or more solutions; call the first 2 solutions $a,b$. That means that $f(a)=f(b)=0$, so by the Mean value theorem, there exists a $c\in (a,b)$ such that $f'(c)=(f(a)-f(b))/(a-b)=(0-0)/(a-b)=0$. On the other hand we may compute:
\[ 
f'(x)=-3+\cos x-\cos^{2}\left(\frac{x}3\right)\sin\left(\frac{x}{3}\right) \leq -1<0,
\] 
where the first inequality follows from the fact that $\sin x,\cos x\in [-1,1]$. So we got that $f'(c)=0$ for some $c$ but at the same time $f'(x)<0$ for all $x$, which is a contradiction. Therefore $f(x)=0$ has exactly one solution. 

\solution{\ref{problem-linearization-estimate0.9999power2014}
Let $f(x)=x^{2014}$. We are looking to approximate $(0.9999)^{2014}= f(0.9999)$. As $f(1)=1^{2014}=1$ is easy to compute, is makes sense to use linear approximation at $a=1$ to approximate $(0.9999)^{2014}$. We have that 
\[
f'(x)=2014x^{2013} \quad .
\]
Therefore the linear approximation of $f(x)=x^{2014}$ at $a=1$ is:
\[
f(x)\approx f(1) +f'(1)(x-1)= 1^{2014}+2014 \cdot 1^{2013}(x-1)=1+ 2014(x-1)=2014x-2013 \quad .
\]
Therefore 
\[
f(0.9999)\approx 2014\cdot 0.9999 -2013=1\cdot 0.9999 +2013(0.9999-1)=0.9999-2013\cdot 0.0001= 0.9999-0.2013=0.7986
\]


A computation with computer shows that $0.999^{2014}=0.817577\dots $. While our approximation of $0.7986$ is less than perfect, it is within the same order of magnitude. We study techniques for estimating errors in linear approximations later.
}

\solution{
\ref{problemRiemannSum-1div1plusxsquared}. The interval $[-3.5,-0.5]$ is subdivided into $n=3$ intervals, therefore the length of each is $\Delta x=1$. The intervals are therefore
\[
[-3.5,-2.5], [-2.5,-1.5], [-1.5,-0.5]\quad .
\]
The problem asks us to use the midpoint of each interval as a sampling point. Therefore our sampling points are $-3,-2,-1$. Therefore the Riemann sum we are looking for is
\[
\Delta x\left(f(-3)+f(-2)+f(-1) \right)=1\cdot \left( \frac{1}{10}+\frac{1}{5}+\frac{1}{2}\right)= 0.8\quad .
\]
\psset{xunit=1cm, yunit=1cm}
\begin{pspicture}(-3.9, -0.9)(1.4,1.499857)
\tiny

\psline*[linecolor=\fcColorAreaUnderGraph, linewidth=0.1pt](-3.500000, 0.000000)(-3.500000, 0.100000)(-2.500000, 0.100000)(-2.500000, 0.000000)(-3.500000, 0.000000)
\psline*[linecolor=\fcColorAreaUnderGraph, linewidth=0.1pt](-2.500000, 0.000000)(-2.500000, 0.200000)(-1.500000, 0.200000)(-1.500000, 0.000000)(-2.500000, 0.000000)
\psline*[linecolor=\fcColorAreaUnderGraph, linewidth=0.1pt](-1.500000, 0.000000)(-1.500000, 0.500000)(-0.500000, 0.500000)(-0.500000, 0.000000)(-1.500000, 0.000000)
\psline[linecolor=blue, linewidth=0.1pt](-3.500000, 0.000000)(-3.500000, 0.100000)(-2.500000, 0.100000)(-2.500000, 0.000000)(-3.500000, 0.000000)
\psline[linecolor=blue, linewidth=0.1pt](-2.500000, 0.000000)(-2.500000, 0.200000)(-1.500000, 0.200000)(-1.500000, 0.000000)(-2.500000, 0.000000)
\psline[linecolor=blue, linewidth=0.1pt](-1.500000, 0.000000)(-1.500000, 0.500000)(-0.500000, 0.500000)(-0.500000, 0.000000)(-1.500000, 0.000000)
\rput[t](-3.500000,-0.03){$-\frac{7}{2}$}\rput[t](-2.500000,-0.03){$-\frac{5}{2}$}\rput[t](-1.500000,-0.03){$-\frac{3}{2}$}\rput[t](-0.500000,-0.03){$-\frac{1}{2}$}
%Function formula: (x^{2}+1)^{-1}
\psplot[linecolor=\fcColorGraph, plotpoints=1000]{-3.5}{1}{ 1 x 2 exp add -1 exp }
\psaxes[ticks=none, labels=none, arrows=<-> ](0,0)(-3.65,-0.65)(1.15,1.149857)
\fcLabels{1.15}{1.149857}
\end{pspicture}
}


\solution{\ref{problemIntx/(2x^2+1)}
\[
\begin{array}{rcll|l}
\displaystyle\int\limits_{1}^{2} \frac{x}{2x^2+1 }  \diff x&=&\displaystyle \int \limits_{x=1}^{x=2} \frac{\frac{1}{4}\diff (2x^2)}{2x^2+1 }  =
\frac{1}{4}\int \limits_{x=1}^{x=2} \frac{\diff (2x^2+1)}{2x^2+1 } &&\text{Set }u=2x^2+1\\
&=&\displaystyle \frac{1}{4}\int\limits_{\substack{x=1 \\u=3} }^{\substack{ x=2\\u=9}} \frac{\diff u}{u} = \frac{1}{4}\left[ \ln u\right]_{3}^9=\frac{1}{4}\left(\ln 9-\ln 3\right)=\frac{\ln 3}{4}
\end{array}
\]
}

\solution{\noindent \ref{problemAreaBetweeny=2-x,x=4-y^2}.
$x=4-y^2$ is a parabola (here we consider $x$ as a function of $y$). $y=-x+2$ implies that $x=2-y$ and so the two curves intersect when
\[
\begin{array}{rcl}
4-y^2&=&2-y\\
-y^2+y+2&=&0\\
-(y+1)(y-2)&=&0\\
y&=& -1\text{~or~}2\quad \quad .
\end{array}
\]
As $x=2-y$, this implies that $x=0$ when $y=2$ and $x=3$ when $y=-1$, or in other words the points of intersection are $(0,2)$ and $(3, -1)$. Therefore we the region is the one plotted below. Integrating with respect to $y$, we get that the area is
\[
\begin{array}{rcl}
A&=&\displaystyle \int\limits_{-1}^{2} \left|4-x^2-(-x+2) \right| \diff y = \int\limits_{-1}^2 \left(-y^2+y+2\right)\diff y \\
&=& \displaystyle \left[- \frac{y^3}3 +\frac{y^2}{2}+ 2y\right]_{-1}^2
=-\frac{8}{3}+2+4 -\left(-\frac{(-1)^3}{3} +\frac{ (-1)^2}{2}-2 \right)\\
&=&\displaystyle \frac{9}{2}\quad .
\end{array}
\]
\psset{xunit=0.5cm, yunit=0.5cm}
\begin{pspicture}(-3.500000, -5)(4.500000,5.5)
\psframe*[linecolor=white](-3.500000,-5)(4.500000,5)
\tiny
\pscustom*[linecolor=cyan]{
\psplot[linecolor=\fcColorGraph, plotpoints=1000]{0}{4}{4 x -1 mul add 0.5 exp }
\psplot[linecolor=\fcColorGraph, plotpoints=1000]{4}{3}{4 x -1 mul add 0.5 exp -1 mul }
}
\rput(-1.5,5){$y=- x+2$}
\psplot[linecolor=\fcColorGraph, plotpoints=1000]{-3.000000}{4.000000}{2 x -1 mul add }
%Function formula: - (- x+4)^{1/2}
\psplot[linecolor=\fcColorGraph, plotpoints=1000]{-3.000000}{4.000000}{4 x -1 mul add 0.5 exp -1 mul }
%Function formula: (- x+4)^{1/2}
\rput(2,2){$x=4-y^2$}
\psplot[linecolor=\fcColorGraph, plotpoints=1000]{-3.000000}{4.000000}{4 x -1 mul add 0.5 exp }
\psaxes[arrows=<->, ticks=none, labels=none](0,0) (-3.000000,-4.5)(4.5,4.5) %Function formula: - x+2
\end{pspicture}
}
\solution{\ref{problemareabetweeny=x^2andy=2x^2+x-2}

\textbf{Region plot.}
\psset{xunit=0.5cm, yunit=0.5cm}
\begin{pspicture}(-3,-3)(3,5.7)
\fcAxesStandard{-3.5}{-3.5}{2.5}{5.5}
\pscustom*[linecolor=\fcColorAreaUnderGraph]{%
\psplot{-2}{1}{x x mul}%
\psplot{1}{-2}{x x mul 2 mul x -2 add add}%
}%
\psplot{-3}{2}{x x mul}
\psplot{-3}{2}{x x mul 2 mul x -2 add add}
\fcGrid[linestyle=dashed, linewidth=0.5, linecolor=gray]{-3}{-3}{5}{8}{1}{1}{}
\rput[t](0.9,-0.2){$1$}
\fcLabels{3.5}{5.5}
\end{pspicture}

The intersection between the two parabolas are found via
\[
\begin{array}{rcl}
x^2&=&2x^2+x-2\\
x^2+x-2&=&0\\
(x-1)(x+2)&=&0\\
x=1&& x=-2\\
y=1&&y=4.
\end{array}
\]

\textbf{Area of the region.} 
\[
\begin{array}{rcll|l}
A&=&\displaystyle\int_{1}^{-2}\left|x^2-(2x^2+x-2) \right|\diff x&&x^2>(2x^2+x-2) \text{ for }x\in [-2,1] \text{ (from plot)}\\
&=&\displaystyle\int_{1}^{-2}\left(x^2-(2x^2+x-2) \right)\diff x\\
&=&\displaystyle \left[-\frac{1}{3} x^{3}-\frac{1}{2} x^{2}+2 x \right]_{-2}^1\\
&=&\displaystyle \frac{9}{2}.
\end{array}
\]
}
\solution{\ref{problemAreaBetween-x^2+2x-1and-2x^2+3x+1}

\textbf{Region plot.}

\psset{xunit=0.5cm, yunit=0.5cm}
\begin{pspicture}(-3.5,-8.8)(3.7,5.7)
\fcAxesStandard{-3.5}{-8.4}{3.5}{5.5}
\pscustom*[linecolor=cyan]{
\psplot{-1}{2}{x x -1 mul mul 2 x mul -1 add add}
\psplot{2}{-1}{x x -2 mul mul 3 x mul 1 add add}
}
\fcGrid[linestyle=dashed, linewidth=0.5, linecolor=gray]{-2}{-8}{5}{13}{1}{1}{}
\rput[t](0.9,-0.2){$1$}
\fcLabels{3.5}{7.5}
\psplot[linecolor=\fcColorGraph]{-1.3}{2.7}{x x -1 mul mul 2 x mul -1 add add}
\psplot[linecolor=\fcColorGraph]{-1.3}{2.7}{x x -2 mul mul 3 x mul 1 add add}
\end{pspicture}

The intersections between the two parabolas are found via
\[
\begin{array}{rcl}
-2x^2+3x+1&=&-x^2+2x-1\\
-x^2+x+2&=&0\\
-(x+1)(x-2)&=&0\\
x=-1&\text{or}& x=2\\
y=-4&&y=-1.
\end{array}
\]

\textbf{Area of the region.} 
\[
\begin{array}{rcll|l}
A&=&\displaystyle\int_{-1}^{2}\left|-2x^2+3x+1-(-x^2+2x-1) \right|\diff x&& \begin{array}{l} -2x^2+3x+1>-x^2+2x-1 \\ \text{ for }x\in [-1,2] \text{ (from plot)}\end{array}\\
&=&\displaystyle\int_{-1}^{2}\left(-2x^2+3x+1-(-x^2+2x-1) \right)\diff x\\
&=&\displaystyle\int_{-1}^{2}\left(-x^2+x+2 \right)\diff x\\
&=&\displaystyle \left[-\frac{1}{3} x^{3}+\frac{1}{2} x^{2}+2 x \right]_{-1}^2\\
&=&\displaystyle \left(-\frac{1}{3} 2^{3}+\frac{1}{2} 2^{2}+2 \cdot 2 \right)-\left( -\frac{1}{3} (-1)^{3}+\frac{1}{2} (-1)^{2}+2 (-1) \right)\\
&=&\displaystyle \frac{9}{2}.
\end{array}
\]
}




%\begin{tabular}{c|c|c|c|c|c|c|c|c||c}
%Problem&1 &2&3&4&5&6&7&8& $\sum$\\ \hline
%Score&&&&&&&&&\\ \hline
%Max&20&20&20&20&20&10&20&20&150
%\end{tabular} 
\end{document}