\documentclass{article}
\ProvidesPackage{homework-problems-UMB}
\addtolength{\hoffset}{-3.5cm}
\addtolength{\textwidth}{6.8cm}
\addtolength{\voffset}{-3cm}
\addtolength{\textheight}{6cm}
\usepackage{../homework-problems} %warning folder paths are relative to the file that uses the includepackage

\renewcommand{\answer}[1]{\iftoggle{answers}{ \hfill{~} \rotatebox{180}{\tiny answer: #1}}{} }
\renewcommand{\pointsii}[1]{}
\renewcommand{\hiddenanswer}{\answer}
\renewcommand{\points}[1]{\item}
\renewcommand{\pointsii}[1]{\item}
\renewcommand{\Arctan}{\arctan}
\renewcommand{\Arcsin}{\arcsin}
\renewcommand{\Arccot}{\operatorname{arccot}}

\toggletrue{solutions}
\toggletrue{answers}
\renewcommand{\fcProblemRef}{\theproblem.\theenumi}
\renewcommand{\fcSubProblemRef}{\theenumi.\theenumii}


\newcommand{\hide}[1]{}
\newtheorem{problem}{Problem}
\pagestyle{empty}
\begin{document}
\begin{center}
\Large
Review sheet Test 1 \\ Math 140 Calculus I \\ \normalsize Summer 2015 \\ Instructor: Todor Milev
\end{center}
%\noindent \textbf{Name:\underline{~~~~~~~~~~~~~~~~~~~~~~~} } \hfill{~}



\noindent The exam is closed textbook. \textbf{No electronic devices are allowed during the exam. } The exam will contain 6 problems, of the 6 problem types given in this review sheet. %You are allowed one single formula sheet, handwritten by you. No template problem solutions are allowed. The sheet will be collected with the test. Photocopied formula sheets are not allowed. 

\begin{problem}
Compute the composite functions $(f\circ g)(x)$, $(g\circ f)(x)$. Simplify your answer to a single fraction. Find the domain of the composite function. The answer key has not been fully proofread, use with caution. 

\begin{enumerate}[ref={\fcProblemRef}]
\item $\displaystyle f{}({{x}})=\frac{x+2}{x-2},
g{}({{x}})=\frac{x-1}{x+2}$.

\answer{
\begin{tabular}{rl}
$(f\circ g)(x)= \frac{3+3 x}{-5- x}$& $x\neq -2, -5$\\ 
$(g\circ f)(x)=\frac{4}{-2+3 x}$& $x\neq 2, \frac{2}{3}$ 
\end{tabular}
}
\item 
$\displaystyle f{}({{x}})=\frac{x+1}{3x-2}, g{}({{x}})= \frac{x-2}{x-1}
$.

\answer{
\begin{tabular}{rl}
$(f\circ g)(x)= \frac{-3+2 x}{-4+x}
$ & $x\neq 4, 1$ \\
$(g\circ f)(x)=\frac{5-5 x}{3-2 x}$
& $x\neq \frac{2}{3}, \frac{3}{2}$
\end{tabular}
}
\item 
$\displaystyle f{}({{x}})=\frac{2x+1}{3x-1},
g{}({{x}})=\frac{x-2}{2x-1}
$.

\answer{
\begin{tabular}{rl}
$(f\circ g)(x)=\frac{-5+4 x}{-5+x}
$ &$x\neq 5, \frac{1}{2}$ \\
$(g\circ f)(x)=\frac{3-4 x}{3+x}
$ &$x\neq -3, \frac{1}{3}$
\end{tabular}
}
\item 
$\displaystyle f{}({{x}})=\frac{x+1}{x-2},
g{}({{x}})=\frac{x+2}{2x-1}
$.

\answer{
\begin{tabular}{rl}
$(f\circ g)(x)= \frac{1+3 x}{4-3 x}
$&$x\neq \frac{4}{3}, \frac{1}{2}$\\ 
$(g\circ f)(x)=\frac{-3+3 x}{4+x}
$&$x\neq -4, 2$
\end{tabular}
}
\item 
$\displaystyle f{}({{x}})=\frac{5x+1}{4x-1},
g{}({{x}})=\frac{4x-1}{3x+1}
$.

\answer{
\begin{tabular}{rl}
$(f\circ g)(x)= \frac{-4+23 x}{-5+13 x}
$&$x\neq \frac{5}{13}, -\frac{1}{3 }$\\
$(g\circ f)(x)=\frac{5+16 x}{2+19 x}
$&$x\neq -\frac{2}{19}, \frac{ 1}{4}$
\end{tabular}
}


\item 
$\displaystyle  f(x)= \frac{3x-5}{x-2}$, $\displaystyle g(x)=\frac{x-2 }{x-4} $. 

\answer{ 
\begin{tabular}{rl}
$(f\circ g)(x)=\frac{-2 x+14}{- x+6}$&$x\neq 6, 4$\\
$(g\circ f)(x)=\frac{x-1}{- x+3}$&$x\neq 3,2$
\end{tabular}
}

\item 
$\displaystyle  f(x)= \frac{x-3}{x+2}$, $\displaystyle g(y)=\frac{y+3 }{y-4} $. 

\answer{ 
\begin{tabular}{rl}
$(f\circ g)(x)=\frac{-2x+15}{3 x-5}$&$x\neq \frac{5}{3}, 4$\\ 
$(g\circ f)(x)=\frac{4 x+3}{-3 x-11}$&$x\neq -\frac{11}{3}, -2$
\end{tabular}
}

\end{enumerate}

\end{problem}
\input{../../modules/functions-basics/homework/functions-composing-fractional-linear-1-solutions}
\begin{problem}
Evaluate the limit if it exists.
\begin{multicols}{2}
\begin{enumerate}[ref={\fcProblemRef}]
\item \label{problemlim(xto2)(x^2-5x+6)/(x-2)}
$\displaystyle\lim\limits_{x\to 2}\frac{x^2-5x+6}{x-2} $. 

\answer{$-1$}
\item $\displaystyle\lim\limits_{x\to 3}\frac{x^2-3x}{x^2-2x-3} $.

\answer{$\frac{3}{4}$}
\item \label{problemlimxto-2(2x^2+x-6)/(x^2-4)}

$\displaystyle \lim_{x\to -2} \frac{2x^2+x-6}{x^2-4}$
\answer{$\frac{7}{4}$}
\item $\displaystyle\lim\limits_{x\to 2}\frac{x^2-5x-6}{x-2} $.

\answer{DNE}
\item $\displaystyle\lim\limits_{x\to -1}\frac{x^2-3x}{x^{2}-2x-3} $.

\answer{DNE}

\item $\displaystyle\lim\limits_{x\to -2}\frac{x^2-4}{2x^2+5x+2} $.

\answer{$\frac{4}{3}$}

\item $\displaystyle\lim\limits_{x\to -1}\frac{2x^2+3x+1}{3x^2-2x-5} $.

\answer{$\frac{1}{8}$}

\item \label{problemlimxto-4(x^2+7x+12)/(x^2+6x+8)}
$\displaystyle \lim_{x \to -4}\frac{x^{2}+7 x+12}{x^{2}+6 x+8}$.

\answer{$\frac{1}{2}$}


\item $\displaystyle\lim\limits_{h\to 0}\frac{(-3+h)^2-9}{h} $.

\answer{$-6$}

\item $\displaystyle\lim\limits_{h\to 0}\frac{(-2+h)^3+8}{h} $.

\answer{$12$}
\item $\displaystyle\lim\limits_{x\to -3}\frac{x+3}{x^3+27} $.

\answer{$\frac{1}{27}$}

\item $\displaystyle\lim\limits_{x\to 1}\frac{x^4-1}{x^3-1} $.

\answer{$\frac{4}{3}$}
\item $\displaystyle\lim\limits_{h\to 0}\frac{\sqrt{4+h}-2}{h} $.

\answer{$\frac{1}{4}$}
\item $\displaystyle\lim\limits_{x\to 3} \frac{\sqrt{5x+1}-4}{x-3}$.

\answer{$\frac{5}{8}$}

\item $\displaystyle\lim\limits_{x\to -3} \frac{\sqrt{x^2+16}-5}{x+3}$.

\answer{$-\frac{3}{5}$}

\item $\displaystyle\lim\limits_{x\to -3} \frac{\frac{1}{3}+ \frac{1}{x}} {3+x}$.

\answer{$-\frac{1}{9}$}
\item $\displaystyle\lim\limits_{x\to -2} \frac{x^2+4x+4}{x^4-16}$.

\answer{$0$}
\item $\displaystyle\lim\limits_{x\to 0} \frac{\sqrt{1+x}- \sqrt{1-x}}{x}$.

\answer{$1$}
\item $\displaystyle\lim\limits_{x\to 0}\left(\frac{1}x -\frac{1}{x^2+x}\right)$.

\answer{$1$}
\item $\displaystyle\lim\limits_{x\to 9} \frac{3-\sqrt{x}}{9x-x^2}$.

\answer{$\frac{1}{54}$}
\item $\displaystyle\lim\limits_{h \to 0}\frac{(2+h)^{-1}-2^{-1}}{h} $.

\answer{$-\frac{1}{4}$}

\item $\displaystyle\lim\limits_{x\to 0} \left(\frac{1}{x\sqrt{1+x}}-\frac{1}{x} \right)$.

\answer{$-\frac{1}{2}$}

\item 
$\displaystyle\lim\limits_{h\to 0}\frac{(x+h)^3-x^3}{h} $.

\answer{$3x^2$}

\item 
\label{problemlim_hto0_(1/(x+h)^2-1/x^2)/h} $\displaystyle\lim\limits_{h\to 0}\frac{\frac{1}{(x+h)^2}-\frac{1}{x^2}}{h} $.

\answer{$-\frac{2}{x^3}$}

\item 
\label{problemlimhto0(1/(2+h)^2-1/4)/h}
$ \displaystyle \lim_{h\to 0} \frac{\frac{1}{(2+h)^2}-\frac{1}{4}}{h}  $.

\answer{$-\frac{1}{4}$}
\item 
\label{problemlimhto0(1/(1+h)^2-1)/h}
$ \displaystyle \lim_{h\to 0} \frac{\frac{1}{(1+h)^2}-1}{h}  $.

\answer{$-2$}
\end{enumerate}
\end{multicols}

\end{problem}
\begin{problem}
~
\begin{enumerate}[ref={\fcProblemRef}]
\item \label{problemIVTtoshowx^2+13x+14=sinx-has-solutions} (1) Solve the equation $x^2+13x+41=1$.  (2) Use the intermediate value theorem to prove that the equation $x^2+13x+41=\sin  x$ has at least two solutions, lying between the two numbers found in (1).
\item (1) Solve the equation $x^2-15x+55=1$.  (2) Use the intermediate value theorem to prove that the equation $x^2-15x+55=\cos  x$ has at least two solutions, lying between the two numbers found in (1).
\end{enumerate}

\end{problem}
\solution{\ref{problemIVTtoshowx^2+13x+14=sinx-has-solutions}.
\noindent (1)
\begin{eqnarray*}
x^2+13x+41&=&1\\
x^2+13x+40&=&0\\
(x+5)(x+8)&=&0\quad .
\end{eqnarray*}
Therefore the two solutions are $x_1=-5$ and $x_2=-8$.

\noindent (2) Consider the function
\[
f(x)=x^2+13x+41-\sin x\quad.
\]
Our strategy for proving $f(x)=0$ has a solution consists in finding a number $a$ such that $f(a)<0$ and a number $b$ such that $f(b)>0$, and then using the Intermediate Value Theorem (IVT) with $N=0$.

Let
\[
g(x)=x^2+13x+41,
\]
and so $f(x)=g(x)-\sin x$. We have no techniques for evaluating $\sin x$ without calculator, but we do have all knowledge necessary to evaluate $g(x)$. Indeed, from high school we know that the lowest point of the parabola $g(x)$ is located at $x=-\frac{13}2=-6.5$. Then $g(-6.5)= -1.25$. Therefore
\[
f(-6.5)=g(-6.5)-\sin(-6.5)= g(-6.5)+\sin (6.5)=-1.25+\sin 6.5 \leq -0.25,
\]
where for the very last inequality we use the fact that $\sin 6.5< 1 $ (remember $\sin t\leq 1$ for all real values of $t$).

On the other hand,
\[f(-5)= g(-5)-\sin (-5) = 1+\sin 5> 0\]
as $\sin 5 >-1$ (remember $\sin t\geq -1$ for all real values of $t$). Therefore $f(-5)>0$ and $f(-6.5)<0$ and by the Intermediate Value Theorem (IVT) $f(x)=0$ has a solution in the interval $x\in (-6.5, -5)$.

Proving $f(x)=0$ has a solution in the interval $x\in (-8, -6.5)$ is similar and we leave it to the student.

Below is a computer generated plot of the function with the use of which we can visually verify our answer.

\psset{xunit=1cm, yunit=1cm}
\begin{pspicture}(-9, -5)(1,5)
\psframe*[linecolor=white](-9,-5)(1,5)
\tiny
\psaxes[ticks=none, labels=none]{<->}(0,0)(-9,-4.5)(1,4.5)
\fcLabels{1}{5}
\fcXTickWithLabel{-6.5}{$-6.5$}
\fcXTickWithLabel{-8}{$-8$}
\fcXTickWithLabel{-5}{$-5$}
%Function formula: (x)^{2}+40+13 (x)
\rput(-4.5,-2){$y=x^{2}+13 x+40$}
\psplot[linecolor=grey!30, plotpoints=1000]{-9}{-4}{x 13 mul 40 x 2 exp add add }
\rput(-6.5,3){$y=x^{2}+41+13 x- \sin x$}
\psplot[linecolor=\fcColorGraph, plotpoints=1000]{-9}{-4}{x 57.29578 mul sin -1 mul x 13 mul 41 x 2 exp add add add }

\end{pspicture}
}
\begin{problem}
Find the horizontal and vertical asymptotes of the graph of the function. Check your work by plotting the function.

\begin{multicols}{2}
\begin{enumerate}[ref={\fcProblemRef}]
\item 
\label{problemAsymptotesy=(2x/(sqrt(x^2+x+3)-3))} $\displaystyle y=\frac{2x}{\sqrt{x^2+x+3}-3}$. 

\answer{Vertical: $x=2, x=-3$, horizontal: $y=2, y=-2$}

\item $\displaystyle y=\frac{3x^2}{\sqrt{x^2+2x+10}-5}$. 

\answer{Vertical: $x=3, x=-5$, horizontal: none.}
\item $\displaystyle y=\frac{3x+1}{x-2}$.

\answer{vertical: $x=2$, horizontal: $y=3$}
\item \label{problemAsymptotesy=(x^2-1)/(2x^2-3x-2)}
$\displaystyle y=\frac{x^2-1}{2x^2-3x-2}$.

\answer{vertical: $x=2, x=-\frac{1}{2}$, horizontal: $y=\frac{1}{2}$}

\item $\displaystyle y=\frac{2x^2-2x-1}{x^2+x-2}$.

\answer{vertical: $x=1, x=-2$, horizontal: $y=2$}

\item \label{problemAsymptotesy=(-5x^2-3x+5)/(x^2-2x-3)}
$\displaystyle 
f(x)=\frac{-5 x^{2}-3 x+5}{x^{2}-2 x-3}
$

\answer{vertical: $x=-1$, $x=3$, horizontal: $y=-5$}



\item $\displaystyle y=\frac{1+x^4}{x^2-x^4}$.

\answer{vertical: $x=0, x=1, x=-1$, horizontal: $y=-1$}
\item $\displaystyle y=\frac{x^3-x}{x^2-7x+6}$.

\answer{vertical: $x=6$, no horizontal asymptote}
\item $\displaystyle y=\frac{x-9}{\sqrt{4x^2+3x+3}}$.

\answer{no vertical asymptote, horizontal: $y=\pm\frac12$}

\item $\displaystyle y=\frac{\sqrt{x^2+1}- x }{x}$.

\answer{vertical: $x=0$, horizontal: $y=0$, $y=-2$}
\item \label{problemAsymptotesy=x/(sqrt(x^2+3) -2x)}
$\displaystyle 
f(x)= \frac{x}{\sqrt{x^2+3} -2x}
$

\answer{vertical: $x=1$, horizontal: $y=-\frac{1}{3}$, $y=-1$}

\end{enumerate}
\end{multicols}
\end{problem}
\solution{\ref{problemAsymptotesy=(2x/(sqrt(x^2+x+3)-3))}
\textbf{Vertical asymptotes.} A function $f(x)$ has a vertical asymptote at $x=a$ if $\lim\limits_{x\to a} f(x)=\pm \infty$. 

The function is algebraic, and therefore has a finite limit at every point it is defined (i.e., no asymptote). Therefore the function can have vertical asymptotes only for those $x$ for which $f(x)$ is not defined. The function is not defined for $\sqrt{x^2+x+3}-3=0$, which has two solutions, $x=2$ and $x=-3$. These are precisely the vertical asymptotes: indeed, 
\[
\lim\limits_{x\to 2^+} \frac{2x}{\sqrt{x^2+x+3}-3}=\infty \quad \quad \quad 
\lim\limits_{x\to 2^-} \frac{2x}{\sqrt{x^2+x+3}-3}=-\infty 
\]
and
\[
\lim\limits_{x\to -3^+} \frac{2x}{\sqrt{x^2+x+3}-3}=\infty \quad \quad \quad 
\lim\limits_{x\to -3^-} \frac{2x}{\sqrt{x^2+x+3}-3}=-\infty 
\]

\textbf{Horizontal asymptotes.} A function $f(x)$ has a horizontal asymptote if $\lim\limits_{x\to \pm\infty} f(x)$ exists. If that limit exists, and is some number, say, $N$, then $y=N$ is the equation of the corresponding asymptote.

Consider the limit $x\to -\infty$. We have that 
\[
\begin{array}{rcll|l}
\displaystyle \lim\limits_{x\to -\infty} \frac{2x}{ \sqrt{x^2+3x+3} - 3}&=&\displaystyle \lim\limits_{x\to - \infty} \frac{ 2}{ \frac{ \sqrt{ x^2 + x+3}}x-\frac3x}\\
&=&\displaystyle  \lim\limits_{x\to - \infty} \frac{2}{-\sqrt{\frac{ x^2 +3x+3}{x^2}}-\frac3x}  && \frac{1}{x} =- \sqrt{\frac{1}{x^2} } \text{ when } x<0\\
&=&\displaystyle \lim\limits_{x\to - \infty} \frac{2}{-\sqrt{1+ \frac{3}{x} + \frac{3}{x^2}}-\frac3x}\\
&=& \displaystyle \frac{\lim\limits_{x\to - \infty} 2}{-\sqrt{ \lim \limits_{x \to - \infty} 1+\lim\limits_{x\to - \infty} \frac{3}{x} + \lim\limits_{x \to - \infty} \frac{3}{x^2}}-\lim\limits_{x\to - \infty} \frac3x}\\
&=&\displaystyle \frac{2}{-\sqrt{1+0+0}-0}\\
&=&\displaystyle -2\quad . 
\end{array}
\]
Therefore $y=-2$ is a horizontal asymptote. 

The case $x\to \infty$, is handled similarly and yields that $y=2$ is a horizontal asymptote.

A computer generated graph confirms our computations.

\psset{xunit=0.2cm, yunit=0.2cm}
\begin{pspicture}(-16, -20)(16,17)
\tiny
\fcAxesStandard{-15}{-19.32133}{15}{16.190354}
\fcXTick{10}
\rput[t](10, -0.6){$10$}
%Function formula: \frac{2 x}{\sqrt{x^{2}+x+3}-3}
\psplot[linecolor=\fcColorGraph, plotpoints=1000]{2.344}{15}{x  2 mul    -3  3 x add   x  2 exp   add    0.5 exp   add   div  }
%Function formula: \frac{2 x}{\sqrt{x^{2}+x+3}-3}
\psplot[linecolor=\fcColorGraph, plotpoints=1000]{-2.6}{1.78}{x  2 mul    -3  3 x add   x  2 exp   add    0.5 exp   add   div  }
%Function formula: \frac{2 x}{\sqrt{x^{2}+x+3}-3}
\psplot[linecolor=\fcColorGraph, plotpoints=1000]{-15}{-3.4}{x  2 mul    -3  3 x add   x  2 exp   add    0.5 exp   add   div  }
\psline[linestyle=dotted](-3,-19.3)(-3,16.1)
\psline[linestyle=dotted](2,-19.3)(2,16.1)
\psline[linestyle=dashed, linecolor=blue](-15, 2)(15, 2)
\psline[linestyle=dashed, linecolor=blue](-15, -2)(15, -2)
\rput[b](-8, 2.6){$y=2$}
\rput[t](8, -2.6){$y=-2$}
\rput[bl](5,5){$y=\frac{2x}{\sqrt{x^2+x+3}-3}$}
\rput[l](2.6,-8){$x=2$}
\rput[r](-3.6,8){$x=-3$}
\end{pspicture}

}


\solution{\ref{problemAsymptotesy=(x^2-1)/(2x^2-3x-2)}



\textbf{Vertical asymptotes.} A function $f(x)$ has a vertical asymptote at $x=a$ if $\lim\limits_{x\to a} f(x)=\pm \infty$. 

The function is algebraic, and therefore has a finite limit at every point it is defined (i.e., no asymptote). Therefore the function can have vertical asymptotes only for those $x$ for which $f(x)$ is not defined. The function is not defined for $2x^2-3x-2=0$, which has two solutions, $x=2$ and $x=-\frac{1}{2}$. These are precisely the vertical asymptotes: indeed, 
\[
\begin{array}{rcll|l}
\displaystyle 
\lim\limits_{x\to 2^+} \frac{x^2-1}{2x^2-3x-2}&=&\displaystyle  \lim\limits_{x\to 2^+} \frac{x^2-1}{2(x-2) \left(x+\frac{1}{2}\right)} = \infty &&\text{Limit of form }\frac{(+)}{(+)(+)}\\
\displaystyle 
\lim\limits_{x\to 2^-} \frac{x^2-1}{2x^2-3x-2}&=&\displaystyle  \lim\limits_{x\to 2^-} \frac{x^2-1}{2(x-2) \left(x+\frac{1}{2}\right)} = -\infty &&\text{Limit of form }\frac{(+)}{(-)(+)}\\
\end{array}
\]
and
\[
\begin{array}{rcll|l}
\displaystyle 
\lim\limits_{x\to -\frac{1}{2}^+} \frac{x^2-1}{2x^2-3x-2}&=&\displaystyle  \lim\limits_{x\to -\frac{1}{2}^+} \frac{x^2-1}{2(x-2) \left(x+\frac{1}{2}\right)} = \infty &&\text{Limit of form }\frac{(-)}{(+)(-)}\\
\displaystyle 
\lim\limits_{x\to -\frac{1}{2}^-} \frac{x^2-1}{2x^2-3x-2}&=&\displaystyle  \lim\limits_{x\to -\frac{1}{2}^-} \frac{x^2-1}{2(x-2) \left(x+\frac{1}{2}\right)} = -\infty &&\text{Limit of form }\frac{(-)}{(-)(-)}\\
\end{array}
\]

\textbf{Horizontal asymptotes.} A function $f(x)$ has a horizontal asymptote if $\lim\limits_{x\to \pm\infty} f(x)$ exists. If that limit exists, and is some number, say, $N$, then $y=N$ is the equation of the corresponding asymptote.

We have that 
\[
\begin{array}{rcll|l}\renewcommand{\arraystretch}{1.6}
\displaystyle \lim\limits_{x\to \infty} \frac{x^2-1}{2x^2-3x-2} &=&\displaystyle \lim\limits_{x\to \infty} \frac{\left(x^2-1\right)\frac{1}{x^2}}{\left(2x^2-3x-2\right)\frac{1}{x^2}}&&\text{Divide by highest term in den.}\\
&=&\displaystyle  \displaystyle \lim\limits_{x\to \infty} \frac{1-\frac{1}{x^2}}{2-\frac{3}{x}-\frac{2}{x^2}} \\
&=&\displaystyle  \displaystyle  \frac{\lim\limits_{x\to \infty}1-\lim\limits_{x\to \infty}\frac{1}{x^2}}{\lim\limits_{x\to \infty}2-\lim\limits_{x\to \infty}\frac{3}{x}-\lim\limits_{x\to \infty}\frac{2}{x^2}}&&\text{Step may be skipped}\\
&=& \displaystyle \frac{1-0}{2-0-0}\\
&=&\displaystyle \frac{1}{2}\\
\end{array}
\]
A similar computation shows that 
\[
\begin{array}{rcll|l}\renewcommand{\arraystretch}{1.6}
\displaystyle \lim\limits_{x\to -\infty} \frac{x^2-1}{2x^2-3x-2} 
&=&\displaystyle \frac{1}{2}\\
\end{array}
\]

Therefore $y=\frac{1}{2}$ is the only horizontal asymptote, valid in both directions ($x\to \pm \infty$). 


A computer generated graph confirms our computations.

\psset{xunit=0.2cm, yunit=0.2cm}
\begin{pspicture}(-16, -20)(16,17)
\tiny
\fcAxesStandard{-15}{-19.32133}{15}{16.190354}
\fcXTick{10}
\newcommand{\theFun}{x x mul 1 sub 2 x x mul mul -3 x mul -2 add add div }
\psplot[linecolor=\fcColorGraph, plotpoints=1000]{-15}{-0.508}{\theFun}
\psplot[linecolor=\fcColorGraph, plotpoints=1000]{-0.49}{1.969}{\theFun}
\psplot[linecolor=\fcColorGraph, plotpoints=1000]{2.04}{15}{\theFun}
\psline[linestyle=dotted](-0.5,-19.3)(-0.5,16.1)
\psline[linestyle=dotted](2,-19.3)(2,16.1)
\psline[linestyle=dashed, linecolor=blue](-15, 0.5)(15, 0.5)
\rput[b](-8, 0.6){$y=\frac{1}{2}$}
\rput[bl](5,2){$y= \frac{x^2-1}{2x^2-3x-2}$}
\rput[l](2.6,-8){$x=2$}
\rput[r](-0.6,8){$x=-\frac{1}{2}$}
\end{pspicture}
}

\solution{\ref{problemAsymptotesy=(-5x^2-3x+5)/(x^2-2x-3)}

\textbf{Vertical asymptotes.} The function is rational, and therefore has a finite limit (and therefore no vertical asymptote) at every point it its domain. The function is not defined for $x^2-2x-3=0$, which has two solutions, $x=-1$ and $x=3$. These are precisely the vertical asymptotes: indeed, 
\[
\begin{array}{rcll|l}
\displaystyle 
\lim\limits_{x\to -1^+} \frac{-5x^2-3x+5}{x^2-2x-3}&=&\displaystyle  \lim\limits_{x\to -1^+} \frac{-5x^2-3x+5}{(x+1) \left(x-3\right)} = -\infty &&\text{Limit of form }\frac{(+)}{(+)(-)}\\
\displaystyle 
\lim\limits_{x\to -1^-} \frac{-5x^2-3x+5}{x^2-2x-3}&=&\displaystyle  \lim\limits_{x\to -1^-} \frac{-5x^2-3x+5}{(x+1) \left(x-3\right)} = \infty &&\text{Limit of form }\frac{(+)}{(-)(-)}\\
\end{array}
\]
and
\[
\begin{array}{rcll|l}
\displaystyle 
\lim\limits_{x\to 3^+} \frac{-5x^2-3x+5}{x^2-2x-3}&=&\displaystyle  \lim\limits_{x\to 3^+} \frac{-5x^2-3x+5}{(x+1) \left(x-3\right)} = -\infty &&\text{Limit of form }\frac{(-)}{(+)(+)}\\
\displaystyle 
\lim\limits_{x\to 3^-} \frac{-5x^2-3x+5}{x^2-2x-3}&=&\displaystyle  \lim\limits_{x\to 3^-} \frac{-5x^2-3x+5}{(x+1) \left(x-3\right)} = \infty &&\text{Limit of form }\frac{(-)}{(+)(-)}\\
\end{array}
\]

\textbf{Horizontal asymptotes.} 
\[
\begin{array}{rcll|l}\renewcommand{\arraystretch}{1.6}
\displaystyle \lim\limits_{x\to \pm\infty} \frac{-5x^2-3x+5}{x^2-2x-3} &=&\displaystyle \lim\limits_{x\to \pm\infty} \frac{\left(-5x^2-3x+5\right)\frac{1}{x^2}}{\left(x^2-2x-3\right)\frac{1}{x^2}}&&\text{Divide by highest term in den.}\\
&=&\displaystyle  \displaystyle \lim\limits_{x\to \pm\infty} \frac{-5-\frac{3}{x}+\frac{5}{x^2}}{1-\frac{2}{x}-\frac{3}{x^2}} \\
&=&\displaystyle  \displaystyle  \frac{-\lim\limits_{x\to \pm\infty}5-\lim\limits_{x\to \pm\infty}\frac{3}{x}+\lim\limits_{x\to \pm\infty}\frac{5}{x^2}}{\lim\limits_{x\to \pm\infty}1-\lim\limits_{x\to \pm\infty}\frac{2}{x}-\lim\limits_{x\to \pm\infty}\frac{3}{x^2}}&&\text{Step may be skipped}\\
&=& \displaystyle \frac{-5-0+0}{1-0-0}\\
&=&\displaystyle -5.\\
\end{array}
\]

Therefore $y=-5$ is the only horizontal asymptote, valid in both directions ($x\to \pm \infty$). 


A computer generated graph confirms our computations.

\psset{xunit=0.2cm, yunit=0.2cm}
\begin{pspicture}(-16, -20.9)(16,17.2)
\tiny
\fcAxesStandard{-15}{-20.8}{15}{17.5}
\fcXTick{10}
\newcommand{\theFun}{x x -5 mul mul -3 x mul 5 add add x x mul -2 x mul -3 add add div\space}
\psplot[linecolor=\fcColorGraph, plotpoints=1000]{-15}{-1.04}{\theFun}
\psplot[linecolor=\fcColorGraph, plotpoints=1000]{-0.96}{2.95}{\theFun}
\psplot[linecolor=\fcColorGraph, plotpoints=1000]{3.05}{15}{\theFun}
\psline[linestyle=dotted](-1,-19.3)(-1,16.1)
\psline[linestyle=dotted](3,-19.3)(3,16.1)
\psline[linestyle=dashed, linecolor=blue](-15, -5)(15, -5)
\rput[t](-8, -5.2){$y=-5$}
\rput[bl](5,2){$y= \frac{-5x^2-3x+5}{x^2-2x-3}$}
\rput[l](3.2,-8){$x=3$}
\rput[r](-0.6,8){$x=-1$}
\end{pspicture}
}

\solution{\ref{problemAsymptotesy=x/(sqrt(x^2+3) -2x)}

\textbf{Vertical asymptotes.} A function $f(x)$ has a vertical asymptote at $x=a$ if $\lim\limits_{x\to a} f(x)=\pm \infty$. 

The function is algebraic, and therefore has a finite limit at every point it is defined (i.e., no asymptote). Therefore the function can have vertical asymptotes only for those $x$ for which $f(x)$ is not defined. The function is not defined for 

\[
\begin{array}{rcll|l}
\sqrt{x^2+3}-2x&=&0\\
\sqrt{x^2+3}&=&2x&&\begin{array}{l} \text{square both sides}\\\text{may introduce extraneous solutions} \end{array}\\
x^2+3&=&4x^2\\
3x^2-3&=&0\\
3(x-1)(x+1)&=&0\\
x=1 \quad &\text{or}& \cancel{ x=-1}\\
&&x=-1 \text{ is extraneous:}\\
&& \sqrt{(-1)^2+3}-(-1)2=4\neq 0
\end{array}
\]

$x=-1$ is indeed a vertical asymptote:
\[
\lim\limits_{x\to 1^+}  \frac{x}{\sqrt{x^2+3} -2x}=\infty \quad \quad \quad 
\lim\limits_{x\to 1^-}  \frac{x}{\sqrt{x^2+3} -2x}=-\infty .
\]
\textbf{Horizontal asymptotes.} 
\[
\begin{array}{rcll|l}
\displaystyle \lim\limits_{x\to -\infty}  \frac{x}{\sqrt{x^2+3} -2x}&=&\displaystyle \lim\limits_{x\to - \infty} \frac{1}{\frac{\sqrt{x^2+3}}{x} -2} \\
&=& \displaystyle \lim\limits_{x\to - \infty} \frac{1}{-\sqrt{\frac{x^2+3}{x^2}} -2}   && \frac{1}{x} =- \sqrt{\frac{1}{x^2} } \text{ when } x<0\\
&=&\displaystyle \lim\limits_{x\to - \infty} \frac{1}{-\sqrt{1+\frac{3}{x^2}} -2}  \\
&=& \displaystyle \frac{1}{-\sqrt{1+0}-2}\\
&=&\displaystyle -\frac{1}{3}.\\
\displaystyle \lim\limits_{x\to -\infty}  \frac{x}{\sqrt{x^2+3} -2x}&=&\displaystyle \lim\limits_{x\to  \infty} \frac{1}{\frac{\sqrt{x^2+3}}{x} -2} \\
&=& \displaystyle \lim\limits_{x\to  \infty} \frac{1}{\sqrt{\frac{x^2+3}{x^2}} -2}   && \frac{1}{x} = \sqrt{\frac{1}{x^2} } \text{ when } x>0\\
&=&\displaystyle \lim\limits_{x\to  \infty} \frac{1}{\sqrt{1+\frac{3}{x^2}} -2}  \\
&=& \displaystyle \frac{1}{\sqrt{1+0}-2}\\
&=&\displaystyle -1.\\
\end{array}
\]
Therefore $y=-\frac{1}{3}$ and $y=-1$ are the two horizontal asymptotes. 


A computer generated graph confirms our computations.

\psset{xunit=0.2cm, yunit=0.2cm}
\begin{pspicture}(-16, -20)(16,17)
\tiny
\fcAxesStandard{-15}{-19.32133}{15}{16.190354}
\fcXTick{10}
\rput[t](10, -0.6){$10$}
\newcommand{\theFun}{x x x mul 3 add sqrt -2 x mul add div}
%Function formula: \frac{2 x}{\sqrt{x^{2}+x+3}-3}
\psplot[linecolor=\fcColorGraph, plotpoints=1000]{1.036}{15}{\theFun }
%Function formula: \frac{2 x}{\sqrt{x^{2}+x+3}-3}
\psplot[linecolor=\fcColorGraph, plotpoints=1000]{-15}{0.961}{\theFun }
\psline[linestyle=dotted](1,-19.3)(1,16.1)
\psline[linestyle=dashed, linecolor=blue](! -15 -1  3 div)(!15 -1 3 div)
\psline[linestyle=dashed, linecolor=blue](-15, -1)(15, -1)
\rput[b](-8, 0.2){$y=-\frac{1}{3}$}
\rput[t](8, -2){$y=-1$}
\rput[bl](5,5){$y=\frac{x}{\sqrt{x^2+3} -2x}$}
\rput[l](1.6,-8){$x=1$}
\end{pspicture}


}



\begin{problem}
Solve each equation for $x$. If available, use a calculator to give an ($\approx$) answer in decimal notation. If available, use a calculator to verify your approximate solutions.
\begin{multicols}{2}
\begin{enumerate}[ref={\fcProblemRef}]
\item $e^{7-4x}=7$.

\answer{$\frac{7-\ln 7 }{4}\approx 1.263522 $}
\item $\ln (2x-9)=2$.

\answer{$\frac{e^2+9}{2}\approx 8.194528 $}
\item $\ln (x^2-2)=3$.

\answer{$\pm \sqrt{e^3+2}\approx \pm 4.699525 $}
\item  \label{problem2^(x-3)=5} $2^{x-3}=5$.

\answer{$\log_2 5+3= \frac{\ln 5}{\ln 2}+3 \approx 5.321928 $}
\item \label{problemlnx+ln(x-1)=1} $\ln x+\ln (x-1)=1$.

\answer{$\frac{1}{2}\left(1+\sqrt{1+4e}\right)\approx 2.223$}
\item $e^{2x+1}=t$.

\answer{$\frac{\ln t-1}{2}$}
\item $\log_2(m x)=c$.

\answer{$\frac{2^c}{m}$}
\item \label{probleme-e^(-2x)=1} $e- e^{-2x}=1$.

\answer{$-\frac12\ln (e-1)\approx -0.271$}
\item $8(1+e^{-x})^{-1}=3$.

\answer{$-\ln \frac53 =\ln \frac35 \approx -0.510826 $}
\item $\ln (\ln x)=1$.

\answer{$e^e\approx 15.154$}
\item $e^{e^x}=10$.

\answer{$\ln (\ln 10)\approx 0.834$}
\item $\ln(2x+1)=3-\ln x$.

\answer{$\frac{-1+\sqrt{1+8e^3}}{4}\approx 2.928878 $}
\item $e^{2x}-4e^x+3=0$.

\answer{$x=\ln 3\approx 1.098612, ~~~, x=0$}


\item $e^{4x}+3e^{2x}-4=0$. 

\answer{$x=0$}
\item $e^{2x}-e^x-6=0$.

\answer{$x=\ln 3$}
\item 
\label{problemSolve4^(3x)-2^(3x+2)-5}

$4^{3x}-2^{3x+2}-5=0$. 

\answer{$x=\frac{\log_{2}5}{3}$}


\item \label{problemSolve32^x+2(1/2)^(x-1)-7=0}
$3\cdot 2^{x}+2 \left(\frac{1}{2}\right)^{x-1}-7=0$. 

\answer{$x=0 \text{ or } 2-\log_2  3= 2-\frac{\ln 3}{\ln 2}$}



\end{enumerate}
\end{multicols}


\end{problem}

\solution{\ref{problem2^(x-3)=5}
\[\begin{array}{rcll|l}
\displaystyle 2^{x-3} &=& 5 &&\displaystyle  \text{take } \log_2 \\
x-3&=&\displaystyle  \log_2(5) &&\text{add } 3 \text{ to both sides}\\
x&=&\displaystyle \log_2(5)+3 &&\text{answer is complete} \\
&=&\displaystyle \frac{\ln 5}{\ln 2}+3 && \text{optional step: convert to }\ln\\
&\approx &5.321928095 &&\text{calculator}
\end{array}
\]
}

\solution{ \ref{probleme-e^(-2x)=1}

\[
\begin{array}{rcll|l}
\displaystyle e-e^{-2x}&=&1\\
\displaystyle e^{-2x}&=&e-1&& \text{apply }\ln\\
\displaystyle \ln e^{-2x}&=&\displaystyle \ln(e-1)\\
-2x&=&\displaystyle\ln(e-1)\\
x&=&\displaystyle-\frac{1}{2}\ln(e-1)\\
&\approx& -0.270662427&&\text{calculator}
\end{array}
\]

}

\solution{\ref{problemlnx+ln(x-1)=1} %
\[
\begin{array}{rcl}
\displaystyle \ln x + \ln (x-1) & =& 1 \\
\displaystyle \ln \left(x^2-x\right) & =& 1 \\
\displaystyle e^{\ln (x^2-x)} & =& e^1 \\
\displaystyle x^2-x & =& e \\
\displaystyle x^2-x-e & =& 0 \\
\text{Quadratic formula:}\quad x & = &\displaystyle \frac{-(-1)\pm \sqrt{(-1)^2-4(1)(-e)}}{2(1)} \\
& =&\displaystyle  \frac{1\pm \sqrt{1+4e}}{2}.
\end{array}
\]
However $\frac{1-\sqrt{1+4e}}{2}$ is negative, so $\ln\left( \frac{1-\sqrt{1 + 4e}}{2} \right)$ is undefined.  
Hence the only solution is $x = \frac{1+\sqrt{1+4e}}{2}\approx 2.2229$.  
}%

\solution{\ref{problemSolve4^(3x)-2^(3x+2)-5}

\[
\begin{array}{rcll|l}
\displaystyle 4^{3x}-2^{3x+2}-5&=&0 \\
\displaystyle 4^{3x}-4\cdot 2^{3x}-5&=&0&&\text{Set } \begin{array}{rcl}\displaystyle 2^{3x}&=&u\\ \displaystyle 4^{3x}&=&u^2\end{array} \\
\displaystyle u^2-4u-5&=&0\\
\displaystyle (u-5)(u+1)&=&0\\
\displaystyle u=5&\text{or}& u=-1\\
\displaystyle 2^{3x}=5&&\displaystyle 2^{3x}=-1\\
\displaystyle 3x=\log_2(5)&&\text{no real solution}\\
\displaystyle x=\frac{\log_2 5}{3}\\
\text{Calculator: }x\approx 0.773976
\end{array}
\]
}

\solution{\ref{problemSolve32^x+2(1/2)^(x-1)-7=0}
\[
\begin{array}{rcll|l}
\displaystyle 3\cdot 2^{x}+2 \displaystyle \left(\frac{1}{2}\right)^{x-1}-7&=&0\\
3\cdot 2^{x}+2\displaystyle  \left(\frac{1}{2} \right)^{x}\left( \frac{1}{2}\right)^{-1} -7&=&0\\
\displaystyle 3\cdot 2^{x}+4 \left(\frac{1}{2} \right)^{x} -7&=&0&&\text{Set } 2^x=u \\
\displaystyle 3u+\frac{4}{u}-7&=&0&&\text{Multiply by }u\\
\displaystyle 3u^2-7u+4&=&0\\
\displaystyle (u-1)(3u-4)&=&0\\
\displaystyle u=1&\text{or}&\displaystyle 3u-4=0\\
\displaystyle 2^x=1&&\displaystyle u=\frac{4}{3}\\
x=0&&\displaystyle 2^x=\frac{4}{3}\\
&&\displaystyle x=\log_2\frac{4}{3}=\log_2 4- \log_2 3\\
&&\displaystyle x=2-\log_2 3\\
\text{Calculator:}&&x\approx 0.415037
\end{array}
\]
}


\begin{problem}
% begin homework inverse-functions2
Find the inverse function and its domain. 

\begin{multicols}{2}
\begin{enumerate}[ref={\fcProblemRef}]
\item \label{problemFindInversey=ln(x+3)} $\displaystyle y=\ln (x+3)$.

\answer{$f^{-1}(x)=e^x-3$}


\item \label{problemFindInversey=4ln(x-3)-4} $\displaystyle y=4 \ln{}\left({{x}}-3\right)-4$.

\answer{$f^{-1}(x)=e^{\frac{x+4}{4} x}+3$}

\item $y=2 \ln{}\left(-2 {{x}}+4\right)+1$

\answer{$ f^{-1}(x)=-\frac{1}{2} e^{\frac{1}{2} x-\frac{1}{2}}+2$}

\item  $f(x)=e^{x^3}$.

\answer{$f^{-1}(x)=\sqrt[3]{\ln x}, \quad x>0$}
\item \label{problemFindInversey=(lnx)^2} $\displaystyle y=(\ln x)^2$, $x\geq 1$.

\answer{$f^{-1}(x)=e^{\sqrt{x}}, \quad x\geq 0 $}

\pointsii{5} 
\label{problemFindInversey=e^x/(1+2e^x)}  $\displaystyle y=\frac{e^x}{1+2e^x}$.

\hiddenanswer{$f^{-1}(x)= \ln \left(\frac{x}{1-2x}\right) $, \quad $x\in \left(0, \frac12\right) $}

\item \label{problemFindInversef=2^(2x)+2^x-2} $f(x)=2^{2x}+2^{x}-2$.

\answer{$f^{-1}(x) =\log_2\frac{-1+\sqrt{9+4x}}{2}, \quad x\geq -2$}
\end{enumerate}
\end{multicols}
% end homework inverse-functions2

\end{problem}
\solution{\ref{problemFindInversey=ln(x+3)}
\begin{align*}
y & = \ln (x+3) \\
e^y & = e^{\ln (x+3)} \\
e^y & = x + 3 \\
e^y - 3 & = x \\
\text{Therefore} \quad f^{-1}(y) & = e^y - 3.
\end{align*}
The domain of $e^y$ is all real numbers, so the domain of $f^{-1}$ is all real numbers.  
}%

\solution{\ref{problemFindInversey=4ln(x-3)-4} 


\[
\renewcommand{\arraystretch}{1.6}
\begin{array}{rcll|l}
\displaystyle 4\ln (x-3)-4&=&\displaystyle y\\
\displaystyle 4\ln(x-3)&=&\displaystyle y+4\\
\displaystyle \ln(x-3)&=&\displaystyle \frac{y+4}{4}&&\text{exponentiate}\\
\displaystyle e^{\ln(x-3)}&=&\displaystyle e^{\frac{y+4}{4}}\\
\displaystyle x-3&=&\displaystyle e^{\frac{y+4}{4}}\\
\displaystyle f^{-1}(y)= x&=&\displaystyle e^{\frac{y+4}{4}}+3\\
\displaystyle f^{-1}(x)&=&\displaystyle e^{\frac{x+4}{4}}+3&& \text{relabel.}\\
\end{array}
\]
The domain of $f^{-1}$ is all real numbers (no restrictions on the domain).
}


\solution{ \ref{problemFindInversey=(lnx)^2}
\[ 
\begin{array}{rcll|l}y&=&(\ln x)^2 &&\mathrm{take~ \sqrt{~} ~on~both~sides,~ } y\geq 0 \\ \sqrt{y}&=&\ln x&&\mathrm{ ~exponentiate} \\ e^{\sqrt{y}}&=&e^{\ln x}=x \\ f^{-1}(y)&=&e^{\sqrt{y}} \\f^{-1}(x)&=&e^{\sqrt{x}} \end{array}
\]
}

\solution{\ref{problemFindInversey=e^x/(1+2e^x)}
\begin{align*}
y & = \frac{e^x}{1+2e^x} \\
y(1+2e^x) & = e^x \\
y & = e^x(1-2y) \\
\frac{y}{1-2y} & = e^x \\
\ln\frac{y}{1-2y} & = \ln e^x \\
\ln\frac{y}{1-2y} & = x \\
\text{Therefore} \quad f^{-1}(y) & = \ln\frac{y}{1-2y}.
\end{align*}
The natural logarithm function is only defined for positive input values.  
Therefore the domain is the set of all $y$ for which 
\begin{align*}
\frac{y}{1-2y} & > 0.
\end{align*}
This inequality holds if the numerator and denominator are both positive or both negative.  
This happens if either
\begin{enumerate}
\item  $y > 0$ and $y < \frac{1}{2}$, or 
\item  $y < 0$ and $y > \frac{1}{2}$.
\end{enumerate}
The latter option is impossible, so the domain is $\{ y \in \mathbb{R} \ | \ 0 < y < \frac{1}{2}\}$.  
}%

%\vskip 18cm
%\hfill \begin{tabular}{c|c|c|c|c|c|c||c}
%Problem&1 &2&3&4&5&6& $\sum$\\ \hline
%Score&&&&&&&\\ \hline
%Max&17&17&17&17&17&17&102
%\end{tabular} 


\end{document}