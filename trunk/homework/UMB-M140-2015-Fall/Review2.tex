\documentclass{article}
\ProvidesPackage{homework-problems-UMB}
\addtolength{\hoffset}{-3.5cm}
\addtolength{\textwidth}{6.8cm}
\addtolength{\voffset}{-3cm}
\addtolength{\textheight}{6cm}
\usepackage{../homework-problems} %warning folder paths are relative to the file that uses the includepackage

\renewcommand{\answer}[1]{\iftoggle{answers}{ \hfill{~} \rotatebox{180}{\tiny answer: #1}}{} }
\renewcommand{\pointsii}[1]{}
\renewcommand{\hiddenanswer}{\answer}
\renewcommand{\points}[1]{\item}
\renewcommand{\pointsii}[1]{\item}
\renewcommand{\Arctan}{\arctan}
\renewcommand{\Arcsin}{\arcsin}
\renewcommand{\Arccot}{\operatorname{arccot}}

\toggletrue{solutions}
\toggletrue{answers}
\renewcommand{\fcProblemRef}{\theproblem.\theenumi}
\renewcommand{\fcSubProblemRef}{\theproblem.\theenumi.\theenumii}


\newcommand{\hide}[1]{}
\newtheorem{problem}{Problem}
\pagestyle{empty}
\begin{document}
\begin{center}
\Large
Review sheet Test 2 \\ Math 140 Calculus I \\ \normalsize Fall 2015 \\ Instructors: Todor Milev, Ping Xu
\end{center}
%\noindent \textbf{Name:\underline{~~~~~~~~~~~~~~~~~~~~~~~} } \hfill{~}



\noindent The exam is closed textbook. \textbf{No electronic devices are allowed during the exam. } The exam will contain a number of problems and sub-problems (the number of problems may depend on your section/instructor). \textbf{Expect each problem type from this review sheet to appear on the test.}%You are allowed one single formula sheet, handwritten by you. No template problem solutions are allowed. The sheet will be collected with the test. Photocopied formula sheets are not allowed. 

\begin{problem}
Compute the derivative.
\begin{multicols}{2}
\begin{enumerate}[ref={\fcProblemRef}]
\item $\displaystyle y=x^{\frac53}-x^{\frac23}$.

\answer{$ \frac53 x^{\frac23}-2/3 x^{-\frac13}$}
\item $\displaystyle f(x)=\sqrt{x}-x$.

\answer{$-1+\frac{1}{2} x^{-\frac{1}{2}} $}
\item $\displaystyle y=\sqrt{x}(x-1)$.

\answer{$ \frac{3}{2} x^{\frac{1}{2}}- \frac{1}{2}  x^{-\frac{1}{2}}$}
\item $\displaystyle f(x)=(2x+1)^2$.

\answer{$4+8 x $}
\item $\displaystyle f(x)=4\pi x^2$.

\answer{$8 \pi x$}
\item  $\displaystyle y=\frac{ x^2+4x+3}{\sqrt{x}}$.

\answer{$ 2 x^{-\frac{1}{2}}+\frac{3}{2} x^{\frac{1}{2}}-\frac{3}{2} x^{-\frac{3}{2}}$}

\item $\displaystyle y=\frac{\sqrt{x}+x}{x^2}$.

\answer{$- x^{-2}-\frac{3}{2} x^{-\frac{5}{2}} $}

\item $\displaystyle f(x)=\left(x+x^{-1}\right)^3$.

\answer{$3x^{2}+3-3x^{-2}-3x^{-4} $}
\item $\displaystyle f(x)=\sqrt 2 x +\sqrt{5x}$.

\answer{$ \sqrt{2}+\frac{\sqrt5}{2}  x^{-\frac{1}{2}}=\sqrt{2}+\frac{\sqrt5}{2\sqrt{x}}$}
\item $\displaystyle y=\sqrt[5]x+4\sqrt{x^5}$.
\answer{$10 x^{\frac{3}{2}}+\frac{1}{5} x^{-\frac{4 }{5}} $}
\item  \label{problemd/dx((sqrt(x)+1/sqrt[3](x))^2)} $\displaystyle y=\left(\sqrt{x}+ \frac{1}{ \sqrt[3]{x}}\right)^2$.


\answer{$1+\frac{1}{3} x^{-\frac{5}{6}}-\frac{2}{3} x^{-\frac{5}{3}} $}


\item $\displaystyle f(x)=(1+2x^2)(x-x^2)$.

\answer{$1-2 x+6 x^{2}-8 x^{3}$}
\item $\displaystyle f(x)=\frac{x^4-5x^3+ \sqrt{x}}{x^2}$.

\answer{$-5+2 x-\frac{3}{2} x^{-\frac{5}{2}} $}
\item $\displaystyle f(x)=(2x^3+3)(x^4-2x)$.

\answer{$-6-4 x^{3}+14 x^{6}$}
\item $\displaystyle f(x)=(1+x+x^2)(2-x^4)$.

\answer{$ 2+4 x-4 x^{3}-5 x^{4}-6 x^{5}$}
\item $\displaystyle g(y)= \left(\frac{1}{y^2}- \frac{3}{y^4} \right)(y+5y^3)$.

\answer{$5+9 y^{-4}+14 y^{-2} $}
\item $\displaystyle f(x)=(x^3-2x)(x^{-4}+x^{-2})$.

\answer{$1+6 x^{-4}+x^{-2}$}
\item $\displaystyle f(x)=\frac{1+2x}{3-4x}$.

\answer{$ 10 (3-4 x)^{-2}$}
\end{enumerate}
\end{multicols}
\end{problem}
\solution{\ref{problemd/dx((sqrt(x)+1/sqrt[3](x))^2)}
\[
\begin{array}{rcl}
\displaystyle \left(\left(\sqrt{x}+\frac{1}{\sqrt[3]{x}}\right)^2 \right)' &=&\displaystyle \left(\left(x^{\frac{1}{2}}+x^{- \frac{1}{3}}\right)^2 \right)'\\
&=&\displaystyle \left(\left(x^{\frac{1}{2}}\right)^2 +2 x^{\frac{1}{2}}x^{-\frac{1}{3}} + \left(x^{-\frac{1}{3}}\right)^2  \right)'\\
&=&\left(x +2 x^{\frac{1}{6}} + x^{-\frac{2}{3}}\right)'\\
&=&\displaystyle 1+2\cdot \frac{1}{6} x^{\frac{1}{6}-1} + \left(-\frac{2}{3} \right)x^{-\frac{2}{3}-1}\\
&=&\displaystyle 1+\frac{1}{3} x^{-\frac{5}{6}} -\frac{2}{3}x^{-\frac{5}{3}}.
\end{array}
\]
}
\begin{problem}
Differentiate.

\begin{multicols}{2}
\begin{enumerate}
\item $\tan x$.

\answer{$\sec^2 x$}
\item $\cot x$.

\answer{$-\csc^2 x$}
\item $\sec x$.

\answer{$\sec x \tan x= \frac{\sin x}{\cos^2 x}$}
\item $\csc x$.

\answer{$-\csc x \cot x= -\frac{\cos x }{\sin^2x} $}
\item $\sec x\tan x$.

\answer{$\sec x \tan^2 x+\sec^3 x$}
\item $\sec x+\tan x$.

\answer{$\sec x(\tan x +\sec x) $}
\item $\sec^2 x$.

\answer{$2\tan x\sec^2 x$}
\item $\csc^2 x$.

\answer{$ -2\cot x\csc^2 x$}

\item \label{problemd/dx((secx)e^x)}
$\displaystyle f(x)=(\sec x )e^{x}$.

\answer{$\sec x\tan x e^x + \sec x e^x=(\tan x +1)(\sec x) e^x$}
\item \label{problemd/dx((tanx)e^x)}
$\displaystyle f(x)=(\tan x )e^{x}$.

\answer{$\sec^2x e^x + \tan x e^x$}

\item $\displaystyle \frac{\sin x}{x}$.

\answer{$\frac{x \cos{}x- \sin{}x}{x^{2}}$}

\item $\displaystyle \frac{\sin x}{e^x}$.

\answer{$\frac{\cos x -\sin x}{e^x}$}
\item $\displaystyle x(\cos x) e^x$.

\answer{$  \begin{array}{l}e^x( x \cos{}x-x  \sin{}x+ \cos{}x) \end{array}$}

\item $\displaystyle \frac{e^x}{\tan x}$.

\answer{$  e^x\left(\cot x-\csc^2 x\right)$}
\item $\displaystyle \frac{e^x}{\sec x} +\sec x$.

\answer{$e^x(\cos x-\sin x)+ \sec x \tan x $}

\end{enumerate}

\end{multicols}
\end{problem}
\begin{problem}
Differentiate. The answer key has not been proofread, use with caution.
\begin{multicols}{3}
\begin{enumerate}[ref={\fcProblemRef}]
\item $\displaystyle f(x)=\sin (-5x)$. 

\answer{$ -5 \cos(-5 x)=-5\cos (5x)$}
\item $\displaystyle f(x)=\cot (2x)$. 

\answer{$-2\csc^2 (2x)$}
\item $\displaystyle f(x)=e^{-3x}$. 

\answer{$-3 e^{-3 x} $}
\item $\displaystyle f(x)=e^{\frac{1}x}$. 

\answer{$ -\frac{e^{\frac{1}{x}}}{x^2}$}
\item $\displaystyle f(x)=e^{\sqrt{x}}$. 

\answer{$ \frac{1}{2\sqrt{x}}e^{\sqrt{x}} $}
\item $\displaystyle f(x)=\ln (1+x) $

\answer{$\frac{1}{1+x} $}
\item $\displaystyle f(x)=\ln(1+x^3) $

\answer{$\frac{3x^2}{1+x^3}$}
\item $\displaystyle f(x)=\frac{1}{2}\ln\left(\frac{1+x}{1-x}\right) $

\answer{$\frac{1}{1-x^{2}}$}
\end{enumerate}
\end{multicols}

\end{problem}


\begin{problem}
Verify that the coordinates of the given point satisfy the given equation. Use implicit differentiation to find an equation of the tangent line to the curve at the given point. 
\begin{multicols}{2}
\begin{enumerate}[ref={\fcProblemRef}]
\item 
\label{problemImplicitTangentysin(2x)=xcos(2y)point(pi/2,pi/4)} $y\sin (2x)=x\cos (2y) $, $\left(\frac{\pi}{2}, \frac{\pi}{4}\right)$. 

\psset{xunit=0.3cm, yunit=0.3cm}
\begin{pspicture}(-5.8,-5.8)(5.8,5.8)
\fcAxesStandard{-5.5}{-5.5}{5.5}{5.5}
\fcLabels{5.5}{5.5}
\fcXTickWithLabel{1}{$1$}
\fcImplicitIId[linestyle=solid, linecolor=red, linewidth-=0.05, showGridImplicitIId=false]{-5}{-5}{1000}{1000}{0.01}{0.01}{2 x mul 180 mul 3.141592654 div sin y mul 2 y mul 180 mul 3.141592654 div cos x mul sub} 

\fcFullDot[linecolor=blue]{3.141592654 2 div}{3.141592654 4 div}
\end{pspicture}

\answer{$y=\frac{1}{2}x$}

\item $ \sin (x+y)=2x-2y$, $(\pi,\pi)$ . 

\answer{$\frac{1}{3} x+\frac{2}{3} \pi $}
\item 
$x^2+x y+y^2=3 $, $(1,-2)$ (ellipse). 

\psset{xunit=0.5cm, yunit=0.5cm}
\begin{pspicture}(-3.6,-3.8)(3.6,3.6)
\tiny
\fcAxesStandard{-3.5}{-3.5}{3.5}{3.5}
\fcLabels{3.5}{3.5}
\fcXTickWithLabel{1}{$1$}
\fcImplicitIId[linestyle=solid, linecolor=red, linewidth-=0.05, showGridImplicitIId=false]{-2}{-2}{400}{400}{0.01}{0.01}{x x mul x y mul add y y mul add 3 sub}
%\psline[linecolor=blue](-3.5,-2)(3.5,-2)
\fcFullDot[linecolor=blue]{1}{-2}
\end{pspicture}


\answer{$y=-2 $}

\item  $x^2+2x y-y^2+x=2 $, $(1,2)$ (hyperbola). 

\psset{xunit=0.4cm, yunit=0.4cm}
\begin{pspicture}(-5.6,-5.6)(5.6,5.6)
\tiny
\fcAxesStandard{-5.5}{-5.5}{5.5}{5.5}
\fcLabels{5.5}{5.5}
\fcXTickWithLabel{1}{$1$}
\fcImplicitIId[linestyle=solid, linecolor=red, linewidth-=0.05, showGridImplicitIId=false]{-5}{-5}{500}{500}{0.02}{0.02}{x x mul x y mul 2 mul add y y mul sub x add 2 sub}
%\psline[linecolor=blue](-3.5,-2)(3.5,-2)
\fcFullDot[linecolor=blue]{1}{2}
\end{pspicture}

\answer{$y= \frac{7}{2} x-\frac{3}{2}$}


\item  \label{problemImplicitTangenty^3+x^3+4xy=3/4}
$\displaystyle y^{3}+x^{3}+4 x y=\frac{3}{4}$, $\left(-\frac{1}{2},-\frac{1}{2}\right)$
\psset{xunit=0.7cm, yunit=0.7cm}
\begin{pspicture}(-2.4,-2.4)(2.4,2.4)
\fcAxesStandard{-2}{-2}{2}{2}
\fcImplicitIId[linecolor=red, linestyle=dashed, dashes={[1 1] 0}, showGridImplicitIId=false, useMidpointImplicitPlots=false]{-3}{-3}{120}{120}{0.05}{0.05}{y y y mul mul x x x mul mul add 4 x y mul mul add 0.75 sub} 
\fcFullDot{-0.5}{-0.5}
\end{pspicture}


\answer{$y=-x-1$}
\item  \label{problemImplicitTangenty^3+x^3+4xy=-4at(1,-1)}
$\displaystyle y^{3}+x^{3}+4 x y=-4$, $(1,-1)$

\psset{xunit=0.7cm, yunit=0.7cm}
\begin{pspicture}(-3.1,-3.1)(3.1,3.1)
\fcAxesStandard{-2}{-2}{2}{2}
\fcImplicitIId[linecolor=red, linestyle=dashed, dashes={[1 1] 0}, showGridImplicitIId=false, useMidpointImplicitPlots=false]{-3}{-3}{300}{300}{0.02}{0.02}{y y y mul mul x x x mul mul add 4 x y mul mul add 4 add} 
\fcFullDot{1}{-1}
\end{pspicture}

\answer{$ $}

\item $x^2+y^2=(2x^2+2y^2-x)^2 $, $(0,\frac{1}{2})$. 

\answer{$y= x+\frac{1}{2}$}
\item $x^{\frac{2}{3}}+y^{\frac{2}{3}}=4$, $(-3\sqrt{3},1)$. 

\answer{$y=\frac{1}{\sqrt{3}}x+4 $}
\item $2(x^2+y^2)^2 =25(x^2-y^2)$, $(3,1)$. 

\answer{$y= -\frac{9}{13} x+\frac{40}{13}$}
\item $y^2(y^2-4)=x^2(x^2-5) $, $(0,-2)$. 

\answer{$y=-2 $}
\item $x^{\frac{4}{3}}+y^{\frac{4}{3}}=10$ at $(-3\sqrt{3}, 1)$. 

\answer{$y=\sqrt{3}x+10$ }
\item $x^2y^3+x^3-y^2=1$ at $(1,1)$. 

\answer{$y=-5 x+6$}
\end{enumerate}
\end{multicols}

\end{problem}
\solution{\ref{problemImplicitTangentysin(2x)=xcos(2y)point(pi/2,pi/4)}

\psset{xunit=0.5cm, yunit=0.5cm}
\begin{pspicture}(-2,-2)(2,2)
\fcAxesStandard{-5.5}{-5.5}{5.5}{5.5}
\fcLabels{5.5}{5.5}
\fcXTickWithLabel{1}{$1$}
\fcImplicitIId[linestyle=solid, linecolor=red, linewidth-=0.05, showGridImplicitIId=false]{-5}{-5}{1000}{1000}{0.01}{0.01}{2 x mul 180 mul 3.141592654 div sin y mul 2 y mul 180 mul 3.141592654 div cos x mul sub} 

\psline[linecolor=blue](-5,-2.5)(5,2.5)
\fcFullDot[linecolor=blue]{3.141592654 2 div}{3.141592654 4 div}
\end{pspicture}


First we verify that the point $\displaystyle (x,y)=\left(\frac{\pi}{2}, \frac{\pi}{4}\right)$ indeed satisfies the given equation:

\[
\begin{array}{rcll|l}
\displaystyle y \sin (2x)_{|x=\frac{\pi}{2}, y=\frac{\pi}{4}}= \frac{\pi}{4}\sin \pi &=& \displaystyle 0 && \text{left hand side}\\
\displaystyle x \cos (2y)_{|x=\frac{\pi}{2}, y=\frac{\pi}{4}}= \frac{\pi}{2}\cos \left(\frac{\pi}{2} \right)&=&\displaystyle 0 && \text{right hand side}\\
\end{array}
\]
so the two sides of the equation are equal (both to $0$) when $x=\frac{\pi}{2}$ and $y=\frac{\pi}{4}$.

Since we are looking an equation of the tangent line, we need to find  $\frac{\diff y}{\diff x}_{|x=\frac{\pi}{2}, y= \frac{\pi}{4}}$ - that is, the derivative of $y$ at the point $x=\frac{\pi}{2}$, $y= \frac{\pi}{4}$. To do so we use implicit differentiation.
\[
\begin{array}{rcll|l}
\displaystyle y \sin (2x)&=&\displaystyle x\cos (2y)&&\frac{\diff }{\diff x}\\
\displaystyle \frac{\diff y}{\diff x} \sin (2x) +y \frac{\diff }{\diff x}\left(\sin (2x)\right)&=&\displaystyle  \cos (2y)+x\frac{\diff }{\diff x}(\cos (2y))\\
\displaystyle \frac{\diff y}{\diff x}\sin (2x)+2y\cos (2x)&= & \displaystyle \cos (2y)-2x \sin (2y) \frac{\diff y}{\diff x}\\
\displaystyle \frac{\diff y}{\diff x}(\sin (2x)+2x\sin (2y))&=&\displaystyle \cos (2y)-2y\cos (2x)&& \text{Set }x=\frac{ \pi}{2}, y=\frac{\pi}{4}\\
\displaystyle \frac{\diff y}{\diff x}_{|x=\frac{\pi}{2}, y=\frac{\pi}{4}} \left(\sin \pi+\pi \sin \left(\frac{\pi}{2}\right)\right)&=& \displaystyle \cos \left(\frac{\pi}{2}\right)-\frac{\pi}{2}\cos \pi\\
\displaystyle \pi \frac{\diff y}{\diff x}_{|x=\frac{\pi}{2}, y=\frac{\pi}{4}} &=& -\frac{\pi}{2}\cos \pi\\
\displaystyle \frac{\diff y}{\diff x}_{|x=\frac{\pi}{2}, y=\frac{\pi}{4}}&=& \displaystyle \frac{1}{2}.
\end{array}
\]
Therefore the equation of the line through $x=\frac{\pi}{2}, y=\frac{\pi}{4}$ is 
\[
\begin{array}{rcl}
\displaystyle y-\frac{\pi}{4}&=&\displaystyle \frac{1}{2}\left( x-\frac{\pi}{2} \right)\\
y&=&\displaystyle \frac{1}{2} x .
\end{array}
\]
}

\solution{\ref{problemImplicitTangenty^3+x^3+4xy=3/4}
\[
\begin{array}{rcll|l}
\displaystyle y^{3}+x^3+4x y&=& \displaystyle \frac{3}{4} &&\text{apply }\frac{\diff }{\diff x}\\
\displaystyle 3y^2 \frac{\diff y}{\diff x}+3x^2+4\left(y+x\frac{\diff y}{\diff x}\right)&=&
\displaystyle \frac{\diff y}{\diff x}\left(3y^2+4x\right)&=&\displaystyle -3x^2-4y\\
\displaystyle \frac{\diff y}{\diff x}&=&\displaystyle  \frac{-3x^2-4y}{ 3y^2+4x}&&\text{substitute }x=-\frac{1}{2}, y=-\frac{1}{2}\\
\displaystyle \frac{\diff y}{\diff x}_{|x=-\frac{1}{2}, y=-\frac{1}{2}}&=& \frac{-3\left(-\frac{1}{2}\right)^2-4\left(-\frac{1}{2}\right)}{ 3\left(-\frac{1}{2}\right)^2+4\left(-\frac{1}{2}\right)}\\
\displaystyle \frac{\diff y}{\diff x}_{|x=-\frac{1}{2}, y=-\frac{1}{2}}&=&-1
\end{array}
\]
Therefore the equation of the tangent is $y-\left(-\frac{1}{2}\right)=-(x-\left(-\frac{1}{2}\right))$ which simplifies to $y=-x-1$. A computer-generated plot visually confirms our computations.

\psset{xunit=0.7cm, yunit=0.7cm}
\begin{pspicture}(-2.4,-2.4)(2.4,2.4)
\fcAxesStandard{-2}{-2}{2}{2}
\fcImplicitIId[linecolor=red, linestyle=dashed, dashes={[1 1] 0}, showGridImplicitIId=false, useMidpointImplicitPlots=false]{-3}{-3}{120}{120}{0.05}{0.05}{y y y mul mul x x x mul mul add 4 x y mul mul add 0.75 sub} 
\psplot[linecolor=\fcColorTangent]{-3}{2}{x -1 mul -1 add}
\fcFullDot{-0.5}{-0.5}
\end{pspicture}
}

\solution{\ref{problemImplicitTangenty^3+x^3+4xy=-4at(1,-1)}

\[
\begin{array}{rcll|l}
\displaystyle y^{3}+x^3+4x y&=& \displaystyle -4 &&\text{apply }\frac{\diff }{\diff x}\\
\displaystyle 3y^2 \frac{\diff y}{\diff x}+3x^2+4\left(y+x\frac{\diff y}{\diff x}\right)&=&0\\
\displaystyle \frac{\diff y}{\diff x}\left(3y^2+4x\right)&=&\displaystyle -3x^2-4y\\
\displaystyle \frac{\diff y}{\diff x}&=&\displaystyle  \frac{-3x^2-4y}{ 3y^2+4x}&&\text{substitute }x=1, y=-1\\
\displaystyle \frac{\diff y}{\diff x}_{|x=1, y=-1}&=&\displaystyle  \frac{-3\left(1\right)^2-4\left(-1\right)}{ 3\left(-1\right)^2+4\left(1\right)} \\
\displaystyle \frac{\diff y}{\diff x}_{|x=-\frac{1}{2}, y=-\frac{1}{2}}&=& \displaystyle  \frac{1}{7}
\end{array}
\]
Therefore the equation of the tangent is $\displaystyle y-\left(- 1\right)= \frac{1}{7}\left(x-1\right)$ which simplifies to $\displaystyle  y=\frac{x}{7}-\frac{8}{7}$. A computer-generated plot visually confirms our computations.

\psset{xunit=0.7cm, yunit=0.7cm}
\begin{pspicture}(-3.1,-3.1)(3.1,3.1)
\fcAxesStandard{-2}{-2}{2}{2}
\fcImplicitIId[linecolor=red, linestyle=dashed, dashes={[1 1] 0}, showGridImplicitIId=false, useMidpointImplicitPlots=false]{-3}{-3}{300}{300}{0.02}{0.02}{y y y mul mul x x x mul mul add 4 x y mul mul add 4 add} 
\fcFullDot{1}{-1}
\psplot[linecolor=blue]{-3}{3}{x 7 div -8 7 div add}
\end{pspicture}
}

\begin{problem}
Find the maximum and minimum values of $f$ on the given interval and the values of $x$ for which they are attained.
\begin{multicols}{2}
\begin{enumerate}[ref={\fcProblemRef}]
\item $\displaystyle f(x)=9+3x-x^2$, $x\in [0,4]$.

\answer{$f_{max}=f\left(\frac{3}{2}\right)=\frac{45}{4} $, $f_{min}=f\left(4\right)=5$}
\item $\displaystyle f(x)=5+4x-2x^3$, $x\in[-1,1] $.

\answer{$f_{max}=f\left(\frac{\sqrt{6}}{3} \right)= \frac{8}{9} \sqrt{6} +5  $, $f_{min} =f\left(-\frac{ \sqrt{6} }{3} \right)=5 - \frac{8 }{9}\sqrt{6} $}

\item $\displaystyle f(x)=2x^3-x^2-20x+1$, $x\in [-4,3]$.

\answer{$f_{max}=f\left( -\frac{5}{3}\right)=\frac{602}{27}$, $f_{min}=f\left(-4\right)=-63$}

\item $\displaystyle f(x)=3x^4-4x^3-12x^2+1$, $x\in [-2, 3]$.

\answer{$f_{max}=f\left(-2\right)=33$, $f_{min}=f\left(2\right)=-31 $}

\item \label{problemmaxminx^3-x^2-x+1over[-1,1]}
$f(x)=x^3-x^2-x+1,$  $x\in [-1,1]$.

\psset{xunit=2cm, yunit=2cm}
\begin{pspicture}(-1.5,-0.5)(1.5,1.5)
\tiny
\fcAxesStandard{-1.5}{-0.5}{1.5}{1.5}
\psplot[linecolor=red]{-1}{1}{x x x mul mul x x -1 mul mul -1 x mul 1 add add add}
\fcXTickWithLabel{1}{$1$}
\fcYTickWithLabel{1}{$1$}
\end{pspicture}
\item \label{problemmaxminx^3-x+1over[-2,1]}
$f(x)=x^3-x+1$,  $x\in[-2,1]$.

\answer{$f_{max}=f\left(-\frac{\sqrt{3}}{3}\right)= \frac{2}{9}\sqrt{3}+1, \quad f_{min}=f\left(-2\right)= -5 $}
\item $\displaystyle f(x)=(x^2-1)^3$, $x\in [-1, 2]$.

\answer{$f_{max}=f\left(2\right)=27$, $f_{min}=f\left(0\right)=-1$}
\item $\displaystyle f(x)=x+\frac{1}{x}$, $x\in [0.2,4 ]$.

\answer{$f_{max}=f\left(0.2\right)=\frac{26}{5}=5.2$, $f_{min}=f\left(1\right)=2$}
\item $\displaystyle f(x)=\frac{x}{x^2-x+1}$, $x\in [0,3 ]$.

\answer{$f_{max}=f\left(1\right)=1$, $f_{min}=f\left(-1\right)=-\frac{1}{3}$}
\item $\displaystyle f(t)=t\sqrt{4-t^2}$, $x\in [-1,2 ]$.

\answer{$f_{max}=f\left(\sqrt{2}\right)=2$, $f_{min}=f\left(-\sqrt{2}\right)=-2$}
\item $\displaystyle f(t)=\sqrt[3]{t}(8-t) $, $x\in [0,8 ]$.

\answer{$f_{max}=f\left(2\right)=6\sqrt[3]{2}$, $f_{min}=f\left(0\right)=f(8)=0$}
\item $\displaystyle f(t)=2\cos t+\sin (2t)$, $x\in [0,\frac{\pi}{2} ]$.

\answer{$f_{max}=f\left(\frac{\pi}{6}\right)=\frac{3}{2}\sqrt{3}$, $f_{min}=f\left(\frac{\pi}{2}\right)=0$}
\item $\displaystyle f(t)=t+\cot \left(\frac{t}{2}\right) $, $x\in [\frac{\pi}{4},\frac{7\pi}{4} ]$.

\answer{$f_{max}=f\left(\frac{3\pi}{2}\right)=\frac{3\pi}{2}-1$, $f_{min}=f\left(\frac{\pi}{2}\right)=\frac{\pi}{2}+1$}

\item $\displaystyle f(t)=t+\cot \left(\frac{t}{2}\right) $, $x\in [\frac{\pi}{4},\frac{7\pi}{4} ]$.
\item $\displaystyle f(x)=x e^{3 x}$, $x\in \left[-3, \frac{1}{6}\right]$.

\answer{$f_{max}=f\left( \frac{1}{6}\right)=\frac{e^{\frac{1}{2}}}{6}\approx 0.274787 $, $f_{min}=f\left( -\frac{1}{3}\right)= -\frac{1}{3e}\approx -0.122626 $}
\item $\displaystyle f(x)=\left(x-2\right) \left(x+1\right) e^{x} $, $x\in \left[-5,2\right]$.

\answer{$\begin{array}{l}
f_{max}=f\left(-\frac{\sqrt{13}}{2}-\frac{1}{2}
\right)= \left(\sqrt{13}+2\right) e^{\left(-\frac{\sqrt{13}}{2}-\frac{1}{2}\right)}\approx 0.560448\\
f_{min}=f\left(\frac{\sqrt{13}}{2}-\frac{1}{2} \right)= \left(-\sqrt{13}+2\right) e^{\left(\frac{\sqrt{13}}{2}-\frac{1}{2}\right)} \approx -5.907619
\end{array}
$}
\item $\displaystyle f(x)=$, $x\in \left[-3,3\right]$.

\answer{$\begin{array}{rcl}
f_{max}&=&f\left( \frac{\sqrt{3}}{2}-\frac{1}{2}
\right) =\left(\frac{\sqrt{3}}{2}+\frac{1}{2}\right) e^{\frac{\sqrt{3}}{2}-1}\approx 1.194743 \\
f_{min}&=&f\left(-\frac{\sqrt{3}}{2}-\frac{1}{2} \right)=\left(-\frac{\sqrt{3}}{2}+\frac{1}{2}\right) e^{-\frac{\sqrt{3}}{2}-1}\approx -0.056638 \end{array}$}
\item \label{problemmaxminxe^(2x)over[-2,1/2]}
$f(x)=x e^{2x}$, $x\in\left[ -2,\frac{1}{2}\right]$.

\answer{$ f_{max}=f\left(\frac{1}{2}\right)= \frac{e}{2}$, $f_{min}=f\left(-\frac{1}{2}\right)=-\frac{1}{2e}$}
\end{enumerate}
\end{multicols}

\end{problem}

\begin{problem}
Find the critical points of the function. Identify whether those are local maxima, minima, or neither. The answer key has not been proofread, use with caution.
\begin{multicols}{2}
\begin{enumerate}[ref={\fcProblemRef}]
\item $\displaystyle f(x)=\frac{x}{1+x^2}$.

\answer{$x=1 $, local \& global max, $x=-1$, local and global min}
\item $\displaystyle f(x)=x^3-x^2-x-1$.

\answer{$ x=-\frac{1}{3} $, local max, $x=1$, local min}
\item $\displaystyle f(x)=2x^3-x^2-20x+1$.

\answer{$x=-\frac{5}{3}$, local max $x=2 $, local min}
\item $\displaystyle f(x)=x+\frac{1}{x}$.

\answer{$x=-1$, local max, $x=1$, local min}
\item $\displaystyle f(x)=\frac{x-\frac{1}{2}}{x^{2}-2 x+\frac{7}{4}} $.

\answer{$x=-\frac{1}{2}$, local and global min, $x=\frac{3}{2}$, local and global max}
\end{enumerate}
\end{multicols}

\end{problem}


\begin{problem}
Find the maximum and minimum values of $f$ on the given interval and the values of $x$ for which they are attained.
\begin{multicols}{2}
\begin{enumerate}[ref={\fcProblemRef}]
\item $\displaystyle f(x)=9+3x-x^2$, $x\in [0,4]$.

\answer{$f_{max}=f\left(\frac{3}{2}\right)=\frac{45}{4} $, $f_{min}=f\left(4\right)=5$}
\item $\displaystyle f(x)=5+4x-2x^3$, $x\in[-1,1] $.

\answer{$f_{max}=f\left(\frac{\sqrt{6}}{3} \right)= \frac{8}{9} \sqrt{6} +5  $, $f_{min} =f\left(-\frac{ \sqrt{6} }{3} \right)=5 - \frac{8 }{9}\sqrt{6} $}

\item $\displaystyle f(x)=2x^3-x^2-20x+1$, $x\in [-4,3]$.

\answer{$f_{max}=f\left( -\frac{5}{3}\right)=\frac{602}{27}$, $f_{min}=f\left(-4\right)=-63$}

\item $\displaystyle f(x)=3x^4-4x^3-12x^2+1$, $x\in [-2, 3]$.

\answer{$f_{max}=f\left(-2\right)=33$, $f_{min}=f\left(2\right)=-31 $}

\item \label{problemmaxminx^3-x^2-x+1over[-1,1]}
$f(x)=x^3-x^2-x+1,$  $x\in [-1,1]$.

\psset{xunit=2cm, yunit=2cm}
\begin{pspicture}(-1.5,-0.5)(1.5,1.5)
\tiny
\fcAxesStandard{-1.5}{-0.5}{1.5}{1.5}
\psplot[linecolor=red]{-1}{1}{x x x mul mul x x -1 mul mul -1 x mul 1 add add add}
\fcXTickWithLabel{1}{$1$}
\fcYTickWithLabel{1}{$1$}
\end{pspicture}
\item \label{problemmaxminx^3-x+1over[-2,1]}
$f(x)=x^3-x+1$,  $x\in[-2,1]$.

\answer{$f_{max}=f\left(-\frac{\sqrt{3}}{3}\right)= \frac{2}{9}\sqrt{3}+1, \quad f_{min}=f\left(-2\right)= -5 $}
\item $\displaystyle f(x)=(x^2-1)^3$, $x\in [-1, 2]$.

\answer{$f_{max}=f\left(2\right)=27$, $f_{min}=f\left(0\right)=-1$}
\item $\displaystyle f(x)=x+\frac{1}{x}$, $x\in [0.2,4 ]$.

\answer{$f_{max}=f\left(0.2\right)=\frac{26}{5}=5.2$, $f_{min}=f\left(1\right)=2$}
\item $\displaystyle f(x)=\frac{x}{x^2-x+1}$, $x\in [0,3 ]$.

\answer{$f_{max}=f\left(1\right)=1$, $f_{min}=f\left(-1\right)=-\frac{1}{3}$}
\item $\displaystyle f(t)=t\sqrt{4-t^2}$, $x\in [-1,2 ]$.

\answer{$f_{max}=f\left(\sqrt{2}\right)=2$, $f_{min}=f\left(-\sqrt{2}\right)=-2$}
\item $\displaystyle f(t)=\sqrt[3]{t}(8-t) $, $x\in [0,8 ]$.

\answer{$f_{max}=f\left(2\right)=6\sqrt[3]{2}$, $f_{min}=f\left(0\right)=f(8)=0$}
\item $\displaystyle f(t)=2\cos t+\sin (2t)$, $x\in [0,\frac{\pi}{2} ]$.

\answer{$f_{max}=f\left(\frac{\pi}{6}\right)=\frac{3}{2}\sqrt{3}$, $f_{min}=f\left(\frac{\pi}{2}\right)=0$}
\item $\displaystyle f(t)=t+\cot \left(\frac{t}{2}\right) $, $x\in [\frac{\pi}{4},\frac{7\pi}{4} ]$.

\answer{$f_{max}=f\left(\frac{3\pi}{2}\right)=\frac{3\pi}{2}-1$, $f_{min}=f\left(\frac{\pi}{2}\right)=\frac{\pi}{2}+1$}

\item $\displaystyle f(t)=t+\cot \left(\frac{t}{2}\right) $, $x\in [\frac{\pi}{4},\frac{7\pi}{4} ]$.
\item $\displaystyle f(x)=x e^{3 x}$, $x\in \left[-3, \frac{1}{6}\right]$.

\answer{$f_{max}=f\left( \frac{1}{6}\right)=\frac{e^{\frac{1}{2}}}{6}\approx 0.274787 $, $f_{min}=f\left( -\frac{1}{3}\right)= -\frac{1}{3e}\approx -0.122626 $}
\item $\displaystyle f(x)=\left(x-2\right) \left(x+1\right) e^{x} $, $x\in \left[-5,2\right]$.

\answer{$\begin{array}{l}
f_{max}=f\left(-\frac{\sqrt{13}}{2}-\frac{1}{2}
\right)= \left(\sqrt{13}+2\right) e^{\left(-\frac{\sqrt{13}}{2}-\frac{1}{2}\right)}\approx 0.560448\\
f_{min}=f\left(\frac{\sqrt{13}}{2}-\frac{1}{2} \right)= \left(-\sqrt{13}+2\right) e^{\left(\frac{\sqrt{13}}{2}-\frac{1}{2}\right)} \approx -5.907619
\end{array}
$}
\item $\displaystyle f(x)=$, $x\in \left[-3,3\right]$.

\answer{$\begin{array}{rcl}
f_{max}&=&f\left( \frac{\sqrt{3}}{2}-\frac{1}{2}
\right) =\left(\frac{\sqrt{3}}{2}+\frac{1}{2}\right) e^{\frac{\sqrt{3}}{2}-1}\approx 1.194743 \\
f_{min}&=&f\left(-\frac{\sqrt{3}}{2}-\frac{1}{2} \right)=\left(-\frac{\sqrt{3}}{2}+\frac{1}{2}\right) e^{-\frac{\sqrt{3}}{2}-1}\approx -0.056638 \end{array}$}
\item \label{problemmaxminxe^(2x)over[-2,1/2]}
$f(x)=x e^{2x}$, $x\in\left[ -2,\frac{1}{2}\right]$.

\answer{$ f_{max}=f\left(\frac{1}{2}\right)= \frac{e}{2}$, $f_{min}=f\left(-\frac{1}{2}\right)=-\frac{1}{2e}$}
\end{enumerate}
\end{multicols}

\end{problem}


\begin{problem}
Use the Intermediate Value theorem and the Mean Value Theorem/Rolle's Theorem to prove that the function has \textbf{exactly one} real root.
\begin{enumerate}[ref={\fcProblemRef}]
\item $x^5+7x=2 $.
\item $x^7+x^5+x^3=3 $.
\item $2x-1=\sin x $.
\item $e^x+2x=3$.
\end{enumerate}

\end{problem}
\begin{problem}

Find the
\begin{multicols}{2}
\begin{itemize}
\item the implied domain of $f$,
\item $x$ and $y$ intercepts of $f$,
\item horizontal and vertical asymptotes,
\item intervals of increase and decrease,
\item local and global minima, maxima,
\item intervals of concavity,
\item points of inflection.
\end{itemize}
\end{multicols}
Label all relevant points on the graph. Show all of your computations.
\begin{enumerate}[ref={\fcProblemRef}]
\item $\displaystyle f(x)=\frac{x+\frac 1 2}{x^{2}+x+1}$

\psset{xunit=1cm, yunit=1cm}
\begin{pspicture}(-5, -1)(5,2)
\psframe*[linecolor=white](-5,-1)(5,2)
\tiny
\psaxes[ticks=none, labels=none]{<->}(0,0)(-5,-0.5)(5,1.5)
\fcLabels{5}{1.5}
%Function formula: \frac{x+1/2}{x^{2}+x+1}
\psplot[linecolor=\fcColorGraph, plotpoints=1000]{-5}{5}{0.5 x add 1 x add x 2 exp add div }
\end{pspicture}

\answer{
\begin{tabular}{l}
$y$-intercept: $\frac12$. $x$-intercept: $-\frac12$\\
Horizontal asymptote: $y=0$, vertical: none \\
local and global min at $x=\frac{ -1-\sqrt{3}}{2}$, local and global max at $x=\frac{ -1+\sqrt{3}}{2}$\\
Intervals of decrease: $ \left(-\infty, \frac{-1 -\sqrt{3} }{2}\right)\cup \left(\frac{-1 +\sqrt{3} }{2}, \infty\right) $, intervals of decrease $\left( \frac{ -1-\sqrt{3}}{2}, \frac{-1+ \sqrt{3}}{2}\right)$ \\
Concave down on $(-\infty, -2)\cup \left(-\frac12, 1\right)$, concave up on $\left(-2, -\frac12\right)\cup (1,\infty)$\\
Inflection points at: $x=-2$, $x= -\frac12$, $x=1 $ \\
\end{tabular}
}

\item \label{problemSketchCurve(2x^2-5x+9/2)/(x^2-3 x+3)} $\displaystyle f(x)=\frac{2 x^{2}-5 x+\frac{9}{2}}{x^{2}-3 x+3}$. \textbf{For this problem, indicate only the $x$-coordinates of the local maxima/minima and inflection points; you do not need to compute the $y$-coordinates of those points.} 

Computation shows that 
$\displaystyle
f'(x)=\frac{- x^{2}+3 x-\frac{3}{2} }{ \left(x^2- 3 x+3\right)^2}
$
and that 
$\displaystyle f''(x)=\frac{(2x-3)x(x-3)}{\left(x^2- 3 x+3\right)^3}$; you may use those computations without further justification. 
 
\psset{xunit=0.6cm, yunit=0.6cm}
\begin{pspicture}(-5, -1)(5,4)
\psframe*[linecolor=white](-5,-1)(5,4)
\tiny
\psaxes[ticks=none, labels=none]{<->}(0,0) (-5,-0.5) (5, 3.5)
\fcLabels{5}{3.5}
%Function formula: \frac{2 x^{2}-5 x+9/2}{x^{2}-3 x+3}
\psplot[linecolor=\fcColorGraph, plotpoints=1000]{-5}{5 } {4.5 x -5 mul add x 2 exp 2 mul add 3 x -3 mul add x 2 exp add div }
\end{pspicture}

\answer{
\begin{tabular}{l}
$y$-intercept: $\frac32$\\
horizontal asymptote: $y=2$, vertical: none\\
increasing on
$\left(\frac{3-\sqrt{3}}2, \frac{3+ \sqrt{3}}2 \right) $, decreasing on $\left(-\infty, \frac{3-\sqrt{3}}2\right)\cup \left(\frac{3+\sqrt{3}}2, \infty\right) $\\
local and global min at $x=\frac{3-\sqrt{3}}2$, local and global max at $x=\frac{3+\sqrt{3}}2$\\
concave up on $\left(0, \frac32\right)\cup \left(3, \infty \right)$, concave down $\left(-\infty, 0\right)\cup \left(\frac32, 3\right)$\\
inflection points at $x=0,x=\frac32, x=3$
\end{tabular}
}


\item $\displaystyle f(x)=\frac{2 \sqrt{- x^{2}+1}+ 1} {\sqrt{- x^{2}+1}+1}$,  $\displaystyle f(x)=\frac{1}{\sqrt{- x^{2} +1}+1}$

\psset{xunit=1cm, yunit=1cm}
\begin{pspicture}(-1, -1)(1.2,3)
\psframe*[linecolor=white](-1,-5)(1,5)
\tiny
\psaxes[ticks=none, labels=none]{<->}(0,0)(-1,-0.5)(1,2.5)
\fcLabels{1}{2.5}
%Function formula: \frac{2 (- x^{2}+1)^{1/2}+1}{(- x^{2}+1)^{1/2}+1}
\rput(1,3){}
\psplot[linecolor=brown, plotpoints=1000]{-1}{1}{1 1 x 2 exp -1 mul add 0.5 exp 2 mul add 1 1 x 2 exp -1 mul add 0.5 exp add div }
%Function formula: \frac{1}{(- x^{2}+1)^{1/2}+1}
\rput(1,3){}
\psplot[linecolor=\fcColorGraph, plotpoints=1000]{-1}{1}{1 1 1 x 2 exp -1 mul add 0.5 exp add div }
\end{pspicture}

The two functions are plotted simultaneously in the $x,y$-plane. Indicate which part of the graph is the graph of which function.

\answer{
\begin{tabular}{l}
For $f(x)=\frac{2 \sqrt{- x^{2}+1}+1}{ \sqrt{- x^{2}+1}+1}$: \\
$y$-intercept: $x=\frac{3}2$, no $x$ intercept\\
no asymptotes\\
increasing on $[-1, 0]$, decreasing on $[0, 1]$ \\
global and local max at $x=0$, global and local min at $x=\pm 1$.\\
concave down on $[-1,1]$\\
no inflection points
\end{tabular}
}
\answer{
\begin{tabular}{l}
For $f(x)=\frac{1}{\sqrt{- x^{2}+1}+1}$: \\
$y$-intercept: $x=\frac{1}2$, no $x$ intercept\\
no asymptotes\\
decreasing on $[-1, 0]$, increasing on $[0, 1]$ \\
global and local min at $x=0$, global and local max at $x=\pm 1$.\\
concave up on $[-1,1]$\\
no inflection points
\end{tabular}
}
\item $\displaystyle f(x)=\frac{e^x+e^{-x}}{e^x-e^{-x}}$

\psset{xunit=0.5cm, yunit=0.5cm}
\begin{pspicture}(-4, -5)(4,5)
\psframe*[linecolor=white](-4,-5)(4,5)
\tiny
\psaxes[ticks=none, labels=none]{<->}(0,0)(-4,-4.5)(4,4.5)
\fcLabels{4}{5}
%Function formula: \frac{e^{- x}+e^{x}}{- e^{- x}+e^{x}}
\psplot[linecolor=\fcColorGraph, plotpoints=1000]{0.2}{4}{2.718281828 x exp 2.718281828 x -1 mul exp add 2.718281828 x exp 2.718281828 x -1 mul exp -1 mul add div }
%Function formula: \frac{e^{- x}+e^{x}}{- e^{- x}+e^{x}}
\psplot[linecolor=\fcColorGraph, plotpoints=1000]{-4}{-0.2}{2.718281828 x exp 2.718281828 x -1 mul exp add 2.718281828 x exp 2.718281828 x -1 mul exp -1 mul add div }
\end{pspicture}
\item $\displaystyle f(x)=\frac{- e^{- x}+e^{x}}{e^{- x}+e^{x}}$

\psset{xunit=1cm, yunit=1cm}
\begin{pspicture}(-4, -1.5)(4,1.5)
\psframe*[linecolor=white](-4,-1.5)(4,1.5)
\tiny
\psaxes[ticks=none, labels=none]{<->}(0,0)(-4,-1.1)(4,1.1)
\fcLabels{4}{1.1}
%Function formula: \frac{- e^{- x}+e^{x}}{e^{- x}+e^{x}}
\psplot[linecolor=\fcColorGraph, plotpoints=1000]{-4}{4}{2.718281828 x exp 2.718281828 x -1 mul exp -1 mul add 2.718281828 x exp 2.718281828 x -1 mul exp add div }
\end{pspicture}
\item $\displaystyle f(x)=\ln{}\left(\frac{{{x}}+1}{- {{x}}+1}\right)$

\psset{xunit=1cm, yunit=1cm}
\begin{pspicture}(-1, -4.2)(1,4.2)
\psframe*[linecolor=white](-1,-5)(1,5)
\tiny
\psaxes[ticks=none, labels=none]{<->}(0,0)(-1.3,-4)(1.3,4)
\fcLabels{1.3}{4}
%Function formula: \log{}(\frac{x+1}{- x+1})
\psplot[linecolor=\fcColorGraph, plotpoints=1000]{-0.94}{0.94}{1 x add 1 x -1 mul add div ln }

\end{pspicture}

\item $\displaystyle f(x)=\frac{x^{2}+3 x+1}{x^{2}+2 x}$. \textbf{For this problem, indicate only the $x$-coordinates of the local maxima/minima and inflection points; you do not need to compute the $y$-coordinates of those points.} 

Computation shows that 
$\displaystyle
f'(x)= \frac{- x^{2}-2 x-2}{\left(x^{2}+2 x\right)^{2}} 
$
and that 
$\displaystyle f''(x)= \frac{2 x^{3}+6 x^{2}+12 x+8}{\left(x^{2}+2 x\right)^{3}} =\frac{\left(x+1\right) \left(2 x^{2}+4 x+8\right)}{\left(x^{2}+2 x\right)^{3}} $; you may use those computations without further justification. 

\psset{xunit=0.7cm, yunit=0.7cm}
\begin{pspicture}(-5, -4.7)(5,5)
\psframe*[linecolor=white](-5,-4.7)(5,5)
\tiny
\psaxes[ticks=none, labels=none]{<->}(0,0)(-5,-4.5)(5,4.5)
\fcLabels{5}{5}
%Function formula: \frac{x^{2}+3 x+1}{x^{2}+2 x}
\psplot[linecolor=\fcColorGraph, plotpoints=1000]{0.1}{5}{1 x 3 mul add x 2 exp add x 2 mul x 2 exp add div }
%Function formula: \frac{x^{2}+3 x+1}{x^{2}+2 x}
\psplot[linecolor=\fcColorGraph, plotpoints=1000]{-1.9}{-0.1}{1 x 3 mul add x 2 exp add x 2 mul x 2 exp add div }
%Function formula: \frac{x^{2}+3 x+1}{x^{2}+2 x}
\psplot[linecolor=\fcColorGraph, plotpoints=1000]{-5}{-2.1}{1 x 3 mul add x 2 exp add x 2 mul x 2 exp add div }
\end{pspicture}

\answer{
\begin{tabular}{l}
$y$-intercept: none, $x$-intercepts: $\frac{-3\mp\sqrt{5}}2$ \\
horizontal asymptote: $y=1$, vertical: $x=-2$ and $x=0$\\
always decreasing\\
no local/global minima/maxima\\
concave down on $\left(-\infty,-2\right)\cup \left(-1,0 \right)$, concave up on $\left(-2, -1\right)\cup \left(0, \infty\right)$\\
inflection point at $x=-1$
\end{tabular}
}


\item \label{problemSketch(x+1)/(x^2+2x+4)} $\displaystyle f(x)=\frac{x+1}{x^2+2x+4}$
\psset{xunit=1.4cm, yunit=1.4cm}
\begin{pspicture}(-5, -1.2)(3.4,1.2)
\psframe*[linecolor=white](-4.5,-1)(4.5,1)
\tiny
\fcAxesStandard{-5}{-1}{3}{1} %Function formula: \frac{x+1}{x^{2}+2 x+4}
\psplot[linecolor=\fcColorGraph, plotpoints=1000]{-5}{3}{1 x add 4 x 2 mul add x 2 exp add div }
\end{pspicture}

\answer{
\begin{tabular}{l}
$y$-intercept: $\frac14$, $x$-intercept: $-1$\\
horizontal asymptote: $y=0$, vertical: none\\
increasing on
$\left(-1-\sqrt{3}, -1+\sqrt{3}  \right) $, decreasing on $\left(-\infty, -1-\sqrt{3}\right)\cup \left(-1+\sqrt{3}, \infty\right) $\\
local and global min at $x=-1-\sqrt{3}$, local and global max at $x=-1+\sqrt{3}$\\
concave up on $\left(-4, -1\right)\cup \left(2, \infty \right)$, concave down $\left(-\infty, -4\right)\cup \left(-1, 2\right)$\\
inflection points at $x=-4,x=-1, x=2$
\end{tabular}
}
\item $\displaystyle f(x)= \frac{3 x^{3}-30 x^{2}+97 x-99}{x^{2}-6 x+8} $. \textbf{For this problem, do not find the $x$-intercepts of the function. Indicate only the $x$-coordinates of the local maxima/minima and inflection points; you do not need to compute the $y$-coordinates of those points.} 

Computation shows that 
$\displaystyle
f'(x)=\frac{3 x^{4}-36 x^{3}+155 x^{2}-282 x+182}{\left(x^{2}-6 x+8\right)^{2}}=\frac{\left(x^{2}-6 x+7\right) \left(3 x^{2}-18 x+26\right)}{\left(x^{2}-6 x+8\right)^{2}}
$
and that 
$\displaystyle f''(x)=\frac{2 x^{3}-18 x^{2}+60 x-72}{\left(x^{2}-6 x+8\right)^{3}}=\frac{\left(x-3\right) \left(2 x^{2}-12 x+24\right)}{\left(x^{2}-6 x+8\right)^{3}}$; you may use those computations without further justification. 

\psset{xunit=0.15cm, yunit=0.15cm}
\begin{pspicture}(-11,-21)(12,11)
\fcAxesStandard{-10}{-20}{10}{10}
\newcommand{\theFun}{3 x x x mul mul mul -30 x x mul mul 97 x mul -99 add add add x x mul -6 x mul 8 add add div}
\psplot[plotpoints=500, linecolor=\fcColorGraph]{-7}{1.97}{\theFun}
\psplot[plotpoints=500, linecolor=\fcColorGraph]{2.02}{3.98}{\theFun}
\psplot[plotpoints=500, linecolor=\fcColorGraph]{4.03}{10}{\theFun}
\end{pspicture}



\answer{
\begin{tabular}{l}
$y$-intercept $\frac{-99}{8}$, $x$-intercept: not requested \\
horizontal asymptote: none, vertical: $x=2$ and $x=4$\\
increasing on $\left(- \infty, -\sqrt{2}+3\right)\cup \left(-\frac{\sqrt{3}}{3}+3, \frac{\sqrt{3}}{3}+3\right)\cup \left(\sqrt{2}+3, \infty\right)
$, decreasing on the complement intervals\\
local maxima at $x=-\sqrt{2}+3$, $ x=\frac{\sqrt{3}}{3}+3$\\
local minima at $x=-\frac{\sqrt{3}}{3}+3$, $ x=\sqrt{2}+3$\\
concave up on $\left(\left(2, 3\right)\cup \left(4, \infty\right)\right)$\\ 
concave down on $\left(-\infty, 2\right)\cup \left(3, 4\right)$\\
inflection point at $x=3$
\end{tabular}
}

\end{enumerate}

\end{problem}

%\vskip 18cm
%\hfill \begin{tabular}{c|c|c|c|c|c|c||c}
%Problem&1 &2&3&4&5&6& $\sum$\\ \hline
%Score&&&&&&&\\ \hline
%Max&17&17&17&17&17&17&102
%\end{tabular} 


\end{document}