\documentclass{article}
\ProvidesPackage{homework-problems-UMB}
\addtolength{\hoffset}{-3.5cm}
\addtolength{\textwidth}{6.8cm}
\addtolength{\voffset}{-3cm}
\addtolength{\textheight}{6cm}
\usepackage{../homework-problems} %warning folder paths are relative to the file that uses the includepackage

\renewcommand{\answer}[1]{\iftoggle{answers}{ \hfill{~} \rotatebox{180}{\tiny answer: #1}}{} }
\renewcommand{\pointsii}[1]{}
\renewcommand{\hiddenanswer}{\answer}
\renewcommand{\points}[1]{\item}
\renewcommand{\pointsii}[1]{\item}
\renewcommand{\Arctan}{\arctan}
\renewcommand{\Arcsin}{\arcsin}
\renewcommand{\Arccot}{\operatorname{arccot}}

\toggletrue{solutions}
\toggletrue{answers}
%\togglefalse{solutions}
%\togglefalse{answers}
\renewcommand{\fcProblemRef}{\theproblem.\theenumi}
\renewcommand{\fcSubProblemRef}{\theproblem.\theenumi.\theenumii}


\newcommand{\hide}[1]{}
\newtheorem{problem}{Problem}
\pagestyle{empty}
\begin{document}
\begin{center}
\Large
Review sheet Final Exam \\ Math 140 Calculus I \\ \normalsize Spring 2016 \\ UMass Boston
\end{center}
%\noindent \textbf{Name:\underline{~~~~~~~~~~~~~~~~~~~~~~~} } \hfill{~}


\noindent The exam is closed textbook. {\sc No calculators or electronic devices are allowed during the exam.} The final exam will contain problems of types similar to the problem types given here. \textsc{Each problem type} given in this review sheet will be \textsc{represented by at least one exam question or sub-question}. The ratio of exercise problems to actual exam problems will be about 3 to 1 (about 3 practice questions of each problem type). \textsc{Some final problems may be formulated in a slightly different way.} 

\begin{problem}Compute the limit or show it does not exist. If the limit does not exist, indicate whether it is $\infty$, $-\infty$ or neither.
\begin{multicols}{2}
\begin{enumerate}[ref={\fcProblemRef}]
\item $\displaystyle\lim\limits_{x\to -2}\frac{x^2-4}{2x^2+5x+2} $.

\answer{$\frac{4}{3}$}

\item $\displaystyle\lim\limits_{x\to -1}\frac{2x^2+3x+1}{3x^2-2x-5} $.

\answer{$\frac{1}{8}$}

\item $\displaystyle\lim\limits_{x\to -3}\frac{x+3}{x^3+27} $.

\answer{$\frac{1}{27}$}

\item $\displaystyle\lim\limits_{h\to 0}\frac{(-3+h)^2-9}{h} $.

\answer{$-6$}

\item $\displaystyle\lim\limits_{h \to 0}\frac{(2+h)^{-1}-2^{-1}}{h} $.

\answer{$-\frac{1}{4}$}

\item \label{problemlim_hto0_(1/(x+h)^2-1/x^2)/h} $\displaystyle\lim\limits_{h\to 0}\frac{\frac{1}{(x+h)^2}-\frac{1}{x^2}}{h} $.

\answer{$-\frac{2}{x^3}$}


\end{enumerate}
\end{multicols}
\end{problem}
\solution{\ref{problemlim(xto2)(x^2-5x+6)/(x-2)}

$
\begin{array}{rcll|l}
\displaystyle 
\displaystyle \lim\limits_{x\to 2}\frac{x^2-5x+6}{x-2} &=&\displaystyle \lim\limits_{x\to 2}\frac{(x-3)\cancel{(x-2)}}{\cancel{x-2}} &&\text{factor and cancel}\\
&=&\displaystyle 2-3=-1
\end{array}
$
}
\solution{\ref{problemlimxto-2(2x^2+x-6)/(x^2-4)}

$\begin{array}{rcll|l}
\displaystyle \lim\limits_{x\to -2} \frac{2x^2+x-6}{x^2-4}&=&  \displaystyle \lim\limits_{x\to -2}\frac{ (2x -3)\cancel{( x+ 2 ) }}{ (x-2)\cancel{(x+2)}} &&\text{factor and cancel}\\ 
&=&\displaystyle  \frac{(2(-2)-3)}{-2-2} &&\text{substitute}\\
&=&\displaystyle \frac{7}{4}
\end{array}
$

}
\solution{\ref{limproblem(xto-2)(x^2-4)/(2x^2+5x+2)}

$
\begin{array}{rcll|l}
\displaystyle 
\displaystyle \lim\limits_{x\to 2}\frac{x^2-4}{2x^2+5x+2} &=&\displaystyle \lim\limits_{x\to -2} \frac{(x-2)\cancel{(x+2)}}{(2x+1) \cancel{(x+2)}} &&\text{factor and cancel}\\
&=&\displaystyle \frac{(-2)-2}{2(-2)+1}=\frac{4}{3}.
\end{array}
$
}
\solution{
\ref{problemlim(xto-1)(2x^2+3x+1)/(3x^2-2x-5)}

$
\begin{array}{rcll|l}
\displaystyle \lim\limits_{x\to-1}\frac{2x^2+3x+1}{3x^2-2x-5} &=&\displaystyle \lim\limits_{x\to -1}\frac{(2x+1)\cancel{(x+1)}}{(3x-5)\cancel{(x+1)}}&&\text{factor and cancel}\\
&=&\displaystyle \frac{2(-1)+1}{3(-1)-5} =\frac{1}{8} 
\end{array}
$
}
\solution{\ref{problemlimxto-4(x^2+7x+12)/(x^2+6x+8)}.

$\begin{array}{rcll|l}
\displaystyle \lim_{x \to -4}\frac{x^{2}+7 x+12}{x^{2}+6 x+8}&=& \displaystyle \lim_{x \to -4}\frac{(x+3)(\cancel{x+4})}{(x+2)(\cancel{x+4})} &&\text{factor}\\
&=&\displaystyle \frac{-4+3}{-4+2}=-\frac{1}{2}.
\end{array}
$

}
\solution{ \ref{problemlim_hto0_(1/(x+h)^2-1/x^2)/h}

$
\begin{array}{rcl}
\displaystyle \lim\limits_{h\to 0}\frac{\frac{1}{(x+h)^2}-\frac{1}{x^2}}{h} &=&\displaystyle \lim\limits_{h\to 0}\frac{x^2-(x+h)^2}{hx^2(x+h)^2}=\lim\limits_{h\to 0} \frac{x^2-(x^2+2xh+h^2)}{hx^2(x+h)^2}\\
&=&\displaystyle \lim\limits_{h\to 0}\frac{\cancel{h}(-2x+h)}{\cancel{h}x^2(x+h)^2}= \frac{-2x+0}{x^2(x+0)^2}=-\frac{2}{x^3}.
\end{array}
$
}

\solution{\ref{problemlimhto0(1/(2+h)^2-1/4)/h}.

\textbf{Variant I.}

$\begin{array}{rcll|l}
\displaystyle \lim_{h\to 0} \frac{\frac{1}{(2+h)^2}-\frac{1}{4}}{h}&=&\displaystyle \lim_{h\to 0}\frac{\frac{4-(2+h)^2}{4(2+h)^2}}{h}\\
&=&\displaystyle \lim_{h\to 0} \frac{4- (4+4h+h^2)}{4h(2+h)^2}\\
&=&\displaystyle \lim_{h\to 0} \frac{-4h-h^2}{4h(2+h)^2}\\
&=&\displaystyle \lim_{h\to 0} \frac{\cancel{h}(-4-h) }{4\cancel{h}(2+h)^2}&&\text{substitute }h=0\\
&=&\displaystyle \frac{-4-0}{4(2+0)^2}\\
&=&\displaystyle -\frac{1}{4}
\end{array}
$

\textbf{Variant II.}

$\begin{array}{rcll|l}
\displaystyle \lim_{h\to 0} \frac{\frac{1}{(2+h)^2}-\frac{1}{4}}{h}&=&\displaystyle \frac{\diff }{\diff x}\left(\frac{1}{x^2}\right)_{|x=2}\\
&=&\displaystyle \left(\frac{-2}{x^3}\right)_{|x=2}\\
&=&\displaystyle -\frac{1}{4}
\end{array}
$

}


\solution{\ref{problemlimhto0(1/(1+h)^2-1)/h}.

\textbf{Variant I.}

$\begin{array}{rcll|l}
\displaystyle \lim_{h\to 0} \frac{\frac{1}{(1+h)^2}-1}{h}&=&\displaystyle \lim_{h\to 0}\frac{\frac{1-(1+h)^2}{ (1+h)^2}}{h}\\
&=&\displaystyle \lim_{h\to 0} \frac{1- (1+2h+h^2)}{h(1+h)^2}\\
&=&\displaystyle \lim_{h\to 0} \frac{-2h-h^2}{h(1+h)^2}\\
&=&\displaystyle \lim_{h\to 0} \frac{\cancel{h}(-2-h) }{\cancel{h}(1+h)^2}&&\text{substitute }h=0\\
&=&\displaystyle \frac{-2-0}{(1+0)^2}\\
&=&\displaystyle -2.
\end{array}
$

\textbf{Variant II.}

$\begin{array}{rcll|l}
\displaystyle \lim_{h\to 0} \frac{\frac{1}{(1+h)^2}-1}{h}&=&\displaystyle \frac{\diff }{\diff x}\left(\frac{1}{x^2}\right)_{|x=1}&&\text{derivative definition}\\
&=&\displaystyle \left(\frac{-2}{x^3}\right)_{|x=1}\\
&=&\displaystyle -2.
\end{array}
$

}
\begin{problem}Compute the limit or show it does not exist. If the limit does not exist, indicate whether it is $\infty$, $-\infty$ or neither.
\begin{enumerate}[ref={\fcProblemRef}]
\item $\displaystyle \lim\limits_{x\to\infty }\frac{\sqrt{x}+x^2}{\sqrt{x}-x^2}$.

\answer{$-1$}

\item \label{problemLimitxtominusinftysqrt(16x^6-3x)/(x^3+2)}
$\displaystyle \lim\limits_{x\to-\infty }\frac{\sqrt{16x^6-3x}}{x^3+2}$.

\answer{$-4$}

\item \label{problemlim(xto-infty)sqrt(x^2+x)-sqrt(x^2-x)} $\lim\limits_{x\to -\infty}\sqrt{x^2+x}- \sqrt{ x^2-x}$. 

\answer{$-1 $.}

\end{enumerate}
\end{problem}
\solution{\ref{problemlim(xto-infty)sqrt(x^2+x)-sqrt(x^2-x)}.
\[ \begin{array}{rcl}
\displaystyle\lim_{x\to -\infty} \sqrt{x^2+x}-\sqrt{x^2-x} &=&\displaystyle\lim_{x\to -\infty} \left(\sqrt{x^2+x}-\sqrt{x^2-x}\right) \frac{ \left(\sqrt{x^2+x}+\sqrt{x^2-x}\right) }{\left( \sqrt{x^2+x}+\sqrt{x^2-x}\right)}
\\
&=&\displaystyle \lim_{x\to -\infty} \frac{x^2+x-(x^2-x) }{\sqrt{x^2+x}+\sqrt{x^2-x} } = \lim_{x\to -\infty} \frac{2x \frac{1}{x} }{\left(\sqrt{x^2+x}+\sqrt{x^2-x} \right)\frac{1}{x}} 
\\&=&\displaystyle \lim_{x\to -\infty} \frac{2}{\frac{\sqrt{x^2+x}}x+\frac{\sqrt{x^2-x}}x }= \lim_{x\to -\infty} \frac{2}{ {\color{red}-} \sqrt{\frac{x^2+x}{x^2}} {\color{red}-} \sqrt{\frac{x^2-x}{x^2}} }\\
&=&\displaystyle \lim_{x\to -\infty} \frac{2}{ - \sqrt{1+\frac1x} - \sqrt{1-\frac1x} }=\frac{2}{-\sqrt{1+0}-\sqrt{1-0}}=-1.
\end{array}
\]
The sign highlighted in red arises from the fact that, for negative $x$, we have that $ x={\color{red}-}\sqrt{x^2}$.
}



\begin{problem}Find the horizontal and vertical asymptotes of the graph of the function. Check your work by plotting the function.

\begin{enumerate}[ref={\fcProblemRef}]
\item $\displaystyle y=\frac{2x^2-2x-1}{x^2+x-2}$.

\answer{vertical: $x=1, x=-2$, horizontal: $y=2$}

\item $\displaystyle y=\frac{3x^2}{\sqrt{x^2+2x+10}-5}$. 

\answer{Vertical: $x=3, x=-5$, horizontal: none.}
\item 
\label{problemAsymptotesy=(2x/(sqrt(x^2+x+3)-3))} $\displaystyle y=\frac{2x}{\sqrt{x^2+x+3}-3}$. 

\answer{Vertical: $x=2, x=-3$, horizontal: $y=2, y=-2$}

\end{enumerate}
\end{problem}
\solution{\ref{problemAsymptotesy=(2x/(sqrt(x^2+x+3)-3))}
\textbf{Vertical asymptotes.} A function $f(x)$ has a vertical asymptote at $x=a$ if $\lim\limits_{x\to a} f(x)=\pm \infty$. 

The function is algebraic, and therefore has a finite limit at every point it is defined (i.e., no asymptote). Therefore the function can have vertical asymptotes only for those $x$ for which $f(x)$ is not defined. The function is not defined for $\sqrt{x^2+x+3}-3=0$, which has two solutions, $x=2$ and $x=-3$. These are precisely the vertical asymptotes: indeed, 
\[
\lim\limits_{x\to 2^+} \frac{2x}{\sqrt{x^2+x+3}-3}=\infty \quad \quad \quad 
\lim\limits_{x\to 2^-} \frac{2x}{\sqrt{x^2+x+3}-3}=-\infty 
\]
and
\[
\lim\limits_{x\to -3^+} \frac{2x}{\sqrt{x^2+x+3}-3}=\infty \quad \quad \quad 
\lim\limits_{x\to -3^-} \frac{2x}{\sqrt{x^2+x+3}-3}=-\infty 
\]

\textbf{Horizontal asymptotes.} A function $f(x)$ has a horizontal asymptote if $\lim\limits_{x\to \pm\infty} f(x)$ exists. If that limit exists, and is some number, say, $N$, then $y=N$ is the equation of the corresponding asymptote.

Consider the limit $x\to -\infty$. We have that 
\[
\begin{array}{rcll|l}
\displaystyle \lim\limits_{x\to -\infty} \frac{2x}{ \sqrt{x^2+3x+3} - 3}&=&\displaystyle \lim\limits_{x\to - \infty} \frac{ 2}{ \frac{ \sqrt{ x^2 + x+3}}x-\frac3x}\\
&=&\displaystyle  \lim\limits_{x\to - \infty} \frac{2}{-\sqrt{\frac{ x^2 +3x+3}{x^2}}-\frac3x}  && \frac{1}{x} =- \sqrt{\frac{1}{x^2} } \text{ when } x<0\\
&=&\displaystyle \lim\limits_{x\to - \infty} \frac{2}{-\sqrt{1+ \frac{3}{x} + \frac{3}{x^2}}-\frac3x}\\
&=& \displaystyle \frac{\lim\limits_{x\to - \infty} 2}{-\sqrt{ \lim \limits_{x \to - \infty} 1+\lim\limits_{x\to - \infty} \frac{3}{x} + \lim\limits_{x \to - \infty} \frac{3}{x^2}}-\lim\limits_{x\to - \infty} \frac3x}\\
&=&\displaystyle \frac{2}{-\sqrt{1+0+0}-0}\\
&=&\displaystyle -2\quad . 
\end{array}
\]
Therefore $y=-2$ is a horizontal asymptote. 

The case $x\to \infty$, is handled similarly and yields that $y=2$ is a horizontal asymptote.

A computer generated graph confirms our computations.

\psset{xunit=0.2cm, yunit=0.2cm}
\begin{pspicture}(-16, -20)(16,17)
\tiny
\fcAxesStandard{-15}{-19.32133}{15}{16.190354}
\fcXTick{10}
\rput[t](10, -0.6){$10$}
%Function formula: \frac{2 x}{\sqrt{x^{2}+x+3}-3}
\psplot[linecolor=\fcColorGraph, plotpoints=1000]{2.344}{15}{x  2 mul    -3  3 x add   x  2 exp   add    0.5 exp   add   div  }
%Function formula: \frac{2 x}{\sqrt{x^{2}+x+3}-3}
\psplot[linecolor=\fcColorGraph, plotpoints=1000]{-2.6}{1.78}{x  2 mul    -3  3 x add   x  2 exp   add    0.5 exp   add   div  }
%Function formula: \frac{2 x}{\sqrt{x^{2}+x+3}-3}
\psplot[linecolor=\fcColorGraph, plotpoints=1000]{-15}{-3.4}{x  2 mul    -3  3 x add   x  2 exp   add    0.5 exp   add   div  }
\psline[linestyle=dotted](-3,-19.3)(-3,16.1)
\psline[linestyle=dotted](2,-19.3)(2,16.1)
\psline[linestyle=dashed, linecolor=blue](-15, 2)(15, 2)
\psline[linestyle=dashed, linecolor=blue](-15, -2)(15, -2)
\rput[b](-8, 2.6){$y=2$}
\rput[t](8, -2.6){$y=-2$}
\rput[bl](5,5){$y=\frac{2x}{\sqrt{x^2+x+3}-3}$}
\rput[l](2.6,-8){$x=2$}
\rput[r](-3.6,8){$x=-3$}
\end{pspicture}

}


\solution{\ref{problemAsymptotesy=(x^2-1)/(2x^2-3x-2)}



\textbf{Vertical asymptotes.} A function $f(x)$ has a vertical asymptote at $x=a$ if $\lim\limits_{x\to a} f(x)=\pm \infty$. 

The function is algebraic, and therefore has a finite limit at every point it is defined (i.e., no asymptote). Therefore the function can have vertical asymptotes only for those $x$ for which $f(x)$ is not defined. The function is not defined for $2x^2-3x-2=0$, which has two solutions, $x=2$ and $x=-\frac{1}{2}$. These are precisely the vertical asymptotes: indeed, 
\[
\begin{array}{rcll|l}
\displaystyle 
\lim\limits_{x\to 2^+} \frac{x^2-1}{2x^2-3x-2}&=&\displaystyle  \lim\limits_{x\to 2^+} \frac{x^2-1}{2(x-2) \left(x+\frac{1}{2}\right)} = \infty &&\text{Limit of form }\frac{(+)}{(+)(+)}\\
\displaystyle 
\lim\limits_{x\to 2^-} \frac{x^2-1}{2x^2-3x-2}&=&\displaystyle  \lim\limits_{x\to 2^-} \frac{x^2-1}{2(x-2) \left(x+\frac{1}{2}\right)} = -\infty &&\text{Limit of form }\frac{(+)}{(-)(+)}\\
\end{array}
\]
and
\[
\begin{array}{rcll|l}
\displaystyle 
\lim\limits_{x\to -\frac{1}{2}^+} \frac{x^2-1}{2x^2-3x-2}&=&\displaystyle  \lim\limits_{x\to -\frac{1}{2}^+} \frac{x^2-1}{2(x-2) \left(x+\frac{1}{2}\right)} = \infty &&\text{Limit of form }\frac{(-)}{(+)(-)}\\
\displaystyle 
\lim\limits_{x\to -\frac{1}{2}^-} \frac{x^2-1}{2x^2-3x-2}&=&\displaystyle  \lim\limits_{x\to -\frac{1}{2}^-} \frac{x^2-1}{2(x-2) \left(x+\frac{1}{2}\right)} = -\infty &&\text{Limit of form }\frac{(-)}{(-)(-)}\\
\end{array}
\]

\textbf{Horizontal asymptotes.} A function $f(x)$ has a horizontal asymptote if $\lim\limits_{x\to \pm\infty} f(x)$ exists. If that limit exists, and is some number, say, $N$, then $y=N$ is the equation of the corresponding asymptote.

We have that 
\[
\begin{array}{rcll|l}\renewcommand{\arraystretch}{1.6}
\displaystyle \lim\limits_{x\to \infty} \frac{x^2-1}{2x^2-3x-2} &=&\displaystyle \lim\limits_{x\to \infty} \frac{\left(x^2-1\right)\frac{1}{x^2}}{\left(2x^2-3x-2\right)\frac{1}{x^2}}&&\text{Divide by highest term in den.}\\
&=&\displaystyle  \displaystyle \lim\limits_{x\to \infty} \frac{1-\frac{1}{x^2}}{2-\frac{3}{x}-\frac{2}{x^2}} \\
&=&\displaystyle  \displaystyle  \frac{\lim\limits_{x\to \infty}1-\lim\limits_{x\to \infty}\frac{1}{x^2}}{\lim\limits_{x\to \infty}2-\lim\limits_{x\to \infty}\frac{3}{x}-\lim\limits_{x\to \infty}\frac{2}{x^2}}&&\text{Step may be skipped}\\
&=& \displaystyle \frac{1-0}{2-0-0}\\
&=&\displaystyle \frac{1}{2}\\
\end{array}
\]
A similar computation shows that 
\[
\begin{array}{rcll|l}\renewcommand{\arraystretch}{1.6}
\displaystyle \lim\limits_{x\to -\infty} \frac{x^2-1}{2x^2-3x-2} 
&=&\displaystyle \frac{1}{2}\\
\end{array}
\]

Therefore $y=\frac{1}{2}$ is the only horizontal asymptote, valid in both directions ($x\to \pm \infty$). 


A computer generated graph confirms our computations.

\psset{xunit=0.2cm, yunit=0.2cm}
\begin{pspicture}(-16, -20)(16,17)
\tiny
\fcAxesStandard{-15}{-19.32133}{15}{16.190354}
\fcXTick{10}
\newcommand{\theFun}{x x mul 1 sub 2 x x mul mul -3 x mul -2 add add div }
\psplot[linecolor=\fcColorGraph, plotpoints=1000]{-15}{-0.508}{\theFun}
\psplot[linecolor=\fcColorGraph, plotpoints=1000]{-0.49}{1.969}{\theFun}
\psplot[linecolor=\fcColorGraph, plotpoints=1000]{2.04}{15}{\theFun}
\psline[linestyle=dotted](-0.5,-19.3)(-0.5,16.1)
\psline[linestyle=dotted](2,-19.3)(2,16.1)
\psline[linestyle=dashed, linecolor=blue](-15, 0.5)(15, 0.5)
\rput[b](-8, 0.6){$y=\frac{1}{2}$}
\rput[bl](5,2){$y= \frac{x^2-1}{2x^2-3x-2}$}
\rput[l](2.6,-8){$x=2$}
\rput[r](-0.6,8){$x=-\frac{1}{2}$}
\end{pspicture}
}

\solution{\ref{problemAsymptotesy=(-5x^2-3x+5)/(x^2-2x-3)}

\textbf{Vertical asymptotes.} The function is rational, and therefore has a finite limit (and therefore no vertical asymptote) at every point it its domain. The function is not defined for $x^2-2x-3=0$, which has two solutions, $x=-1$ and $x=3$. These are precisely the vertical asymptotes: indeed, 
\[
\begin{array}{rcll|l}
\displaystyle 
\lim\limits_{x\to -1^+} \frac{-5x^2-3x+5}{x^2-2x-3}&=&\displaystyle  \lim\limits_{x\to -1^+} \frac{-5x^2-3x+5}{(x+1) \left(x-3\right)} = -\infty &&\text{Limit of form }\frac{(+)}{(+)(-)}\\
\displaystyle 
\lim\limits_{x\to -1^-} \frac{-5x^2-3x+5}{x^2-2x-3}&=&\displaystyle  \lim\limits_{x\to -1^-} \frac{-5x^2-3x+5}{(x+1) \left(x-3\right)} = \infty &&\text{Limit of form }\frac{(+)}{(-)(-)}\\
\end{array}
\]
and
\[
\begin{array}{rcll|l}
\displaystyle 
\lim\limits_{x\to 3^+} \frac{-5x^2-3x+5}{x^2-2x-3}&=&\displaystyle  \lim\limits_{x\to 3^+} \frac{-5x^2-3x+5}{(x+1) \left(x-3\right)} = -\infty &&\text{Limit of form }\frac{(-)}{(+)(+)}\\
\displaystyle 
\lim\limits_{x\to 3^-} \frac{-5x^2-3x+5}{x^2-2x-3}&=&\displaystyle  \lim\limits_{x\to 3^-} \frac{-5x^2-3x+5}{(x+1) \left(x-3\right)} = \infty &&\text{Limit of form }\frac{(-)}{(+)(-)}\\
\end{array}
\]

\textbf{Horizontal asymptotes.} 
\[
\begin{array}{rcll|l}\renewcommand{\arraystretch}{1.6}
\displaystyle \lim\limits_{x\to \pm\infty} \frac{-5x^2-3x+5}{x^2-2x-3} &=&\displaystyle \lim\limits_{x\to \pm\infty} \frac{\left(-5x^2-3x+5\right)\frac{1}{x^2}}{\left(x^2-2x-3\right)\frac{1}{x^2}}&&\text{Divide by highest term in den.}\\
&=&\displaystyle  \displaystyle \lim\limits_{x\to \pm\infty} \frac{-5-\frac{3}{x}+\frac{5}{x^2}}{1-\frac{2}{x}-\frac{3}{x^2}} \\
&=&\displaystyle  \displaystyle  \frac{-\lim\limits_{x\to \pm\infty}5-\lim\limits_{x\to \pm\infty}\frac{3}{x}+\lim\limits_{x\to \pm\infty}\frac{5}{x^2}}{\lim\limits_{x\to \pm\infty}1-\lim\limits_{x\to \pm\infty}\frac{2}{x}-\lim\limits_{x\to \pm\infty}\frac{3}{x^2}}&&\text{Step may be skipped}\\
&=& \displaystyle \frac{-5-0+0}{1-0-0}\\
&=&\displaystyle -5.\\
\end{array}
\]

Therefore $y=-5$ is the only horizontal asymptote, valid in both directions ($x\to \pm \infty$). 


A computer generated graph confirms our computations.

\psset{xunit=0.2cm, yunit=0.2cm}
\begin{pspicture}(-16, -20.9)(16,17.2)
\tiny
\fcAxesStandard{-15}{-20.8}{15}{17.5}
\fcXTick{10}
\newcommand{\theFun}{x x -5 mul mul -3 x mul 5 add add x x mul -2 x mul -3 add add div\space}
\psplot[linecolor=\fcColorGraph, plotpoints=1000]{-15}{-1.04}{\theFun}
\psplot[linecolor=\fcColorGraph, plotpoints=1000]{-0.96}{2.95}{\theFun}
\psplot[linecolor=\fcColorGraph, plotpoints=1000]{3.05}{15}{\theFun}
\psline[linestyle=dotted](-1,-19.3)(-1,16.1)
\psline[linestyle=dotted](3,-19.3)(3,16.1)
\psline[linestyle=dashed, linecolor=blue](-15, -5)(15, -5)
\rput[t](-8, -5.2){$y=-5$}
\rput[bl](5,2){$y= \frac{-5x^2-3x+5}{x^2-2x-3}$}
\rput[l](3.2,-8){$x=3$}
\rput[r](-0.6,8){$x=-1$}
\end{pspicture}
}

\solution{\ref{problemAsymptotesy=x/(sqrt(x^2+3) -2x)}

\textbf{Vertical asymptotes.} A function $f(x)$ has a vertical asymptote at $x=a$ if $\lim\limits_{x\to a} f(x)=\pm \infty$. 

The function is algebraic, and therefore has a finite limit at every point it is defined (i.e., no asymptote). Therefore the function can have vertical asymptotes only for those $x$ for which $f(x)$ is not defined. The function is not defined for 

\[
\begin{array}{rcll|l}
\sqrt{x^2+3}-2x&=&0\\
\sqrt{x^2+3}&=&2x&&\begin{array}{l} \text{square both sides}\\\text{may introduce extraneous solutions} \end{array}\\
x^2+3&=&4x^2\\
3x^2-3&=&0\\
3(x-1)(x+1)&=&0\\
x=1 \quad &\text{or}& \cancel{ x=-1}\\
&&x=-1 \text{ is extraneous:}\\
&& \sqrt{(-1)^2+3}-(-1)2=4\neq 0
\end{array}
\]

$x=-1$ is indeed a vertical asymptote:
\[
\lim\limits_{x\to 1^+}  \frac{x}{\sqrt{x^2+3} -2x}=\infty \quad \quad \quad 
\lim\limits_{x\to 1^-}  \frac{x}{\sqrt{x^2+3} -2x}=-\infty .
\]
\textbf{Horizontal asymptotes.} 
\[
\begin{array}{rcll|l}
\displaystyle \lim\limits_{x\to -\infty}  \frac{x}{\sqrt{x^2+3} -2x}&=&\displaystyle \lim\limits_{x\to - \infty} \frac{1}{\frac{\sqrt{x^2+3}}{x} -2} \\
&=& \displaystyle \lim\limits_{x\to - \infty} \frac{1}{-\sqrt{\frac{x^2+3}{x^2}} -2}   && \frac{1}{x} =- \sqrt{\frac{1}{x^2} } \text{ when } x<0\\
&=&\displaystyle \lim\limits_{x\to - \infty} \frac{1}{-\sqrt{1+\frac{3}{x^2}} -2}  \\
&=& \displaystyle \frac{1}{-\sqrt{1+0}-2}\\
&=&\displaystyle -\frac{1}{3}.\\
\displaystyle \lim\limits_{x\to -\infty}  \frac{x}{\sqrt{x^2+3} -2x}&=&\displaystyle \lim\limits_{x\to  \infty} \frac{1}{\frac{\sqrt{x^2+3}}{x} -2} \\
&=& \displaystyle \lim\limits_{x\to  \infty} \frac{1}{\sqrt{\frac{x^2+3}{x^2}} -2}   && \frac{1}{x} = \sqrt{\frac{1}{x^2} } \text{ when } x>0\\
&=&\displaystyle \lim\limits_{x\to  \infty} \frac{1}{\sqrt{1+\frac{3}{x^2}} -2}  \\
&=& \displaystyle \frac{1}{\sqrt{1+0}-2}\\
&=&\displaystyle -1.\\
\end{array}
\]
Therefore $y=-\frac{1}{3}$ and $y=-1$ are the two horizontal asymptotes. 


A computer generated graph confirms our computations.

\psset{xunit=0.2cm, yunit=0.2cm}
\begin{pspicture}(-16, -20)(16,17)
\tiny
\fcAxesStandard{-15}{-19.32133}{15}{16.190354}
\fcXTick{10}
\rput[t](10, -0.6){$10$}
\newcommand{\theFun}{x x x mul 3 add sqrt -2 x mul add div}
%Function formula: \frac{2 x}{\sqrt{x^{2}+x+3}-3}
\psplot[linecolor=\fcColorGraph, plotpoints=1000]{1.036}{15}{\theFun }
%Function formula: \frac{2 x}{\sqrt{x^{2}+x+3}-3}
\psplot[linecolor=\fcColorGraph, plotpoints=1000]{-15}{0.961}{\theFun }
\psline[linestyle=dotted](1,-19.3)(1,16.1)
\psline[linestyle=dashed, linecolor=blue](! -15 -1  3 div)(!15 -1 3 div)
\psline[linestyle=dashed, linecolor=blue](-15, -1)(15, -1)
\rput[b](-8, 0.2){$y=-\frac{1}{3}$}
\rput[t](8, -2){$y=-1$}
\rput[bl](5,5){$y=\frac{x}{\sqrt{x^2+3} -2x}$}
\rput[l](1.6,-8){$x=1$}
\end{pspicture}


}



\begin{problem}Differentiate.
\begin{multicols}{2}
\begin{enumerate}[ref={\fcProblemRef}]
\item $\displaystyle \frac{\sin x}{x}$.

\answer{$\frac{x \cos{}x- \sin{}x}{x^{2}}$}

\item $\displaystyle x(\cos x) e^x$.

\answer{$  \begin{array}{l}e^x( x \cos{}x-x  \sin{}x+ \cos{}x) \end{array}$}

\item $\displaystyle \frac{e^x}{\sec x} +\sec x$.

\answer{$e^x(\cos x-\sin x)+ \sec x \tan x $}

\item $\displaystyle f(x)=\frac{x-3}{x+3}$.

\answer{$6 (3+x)^{-2} $}

\item $\displaystyle y=\frac{x-1}{x^3+x-2}$.

\answer{$  \frac{-2 x-1}{\left(x^{2}+x+2\right)^{2}} $}
\item \label{problemd/dt(t/(t-1)^2)} $\displaystyle y=\frac{t}{(t-1)^2}$.

\answer{$-\frac{t +1}{(t-1)^3} $}
\item $e^{2x}$.

\answer{$2e^{2x}$}

\item $e^{-x^2}$

\answer{$-2xe^{-x^2}$}

\item $e^{\sqrt{x}}$

\answer{$\frac{1}{2\sqrt{x}}e^{\sqrt{x}}$}

\item $\displaystyle f(x)=\sqrt{1+x^2}$

\answer{$x (x^{2}+1)^{-\frac{1}{2}}  $}
\item \label{problemDifferentialtexDivsqrt(1+2divx^2)}  $\displaystyle  f(x)=\frac{x }{\sqrt{1+\frac{2}{x^2}}}$.

\answer{$\frac{\pm x^2}{\sqrt{x^2+2}} $}

\item $\displaystyle f(x)= \sqrt{x+\sqrt{x}}$.

\answer{$ \left(\frac{1}{2} +\frac{1}{4} x^{-\frac{1}{2}}\right) \left(x^{\frac{1}{2}}+x\right)^{-\frac{1}{2}}$}

\item $\displaystyle \ln{}\left(-6 {{x}}+2\right)$

\answer{$\frac{3}{3 x-1}$}

\item $\displaystyle \ln{}\left(\cot x\right) $

\answer{$ -\csc x \sec x$}

\item $\displaystyle \ln{}\left(\sec (2x)\right) $

\answer{$ 2\tan (2x) $}


\end{enumerate}
\end{multicols}
\end{problem}
\solution{\ref{problemd/dt(t/(t-1)^2)}
This can be differentiated more efficiently using the chain rule, however let us show how the problem can be solved directly using the quotient rule.
\[
\begin{array}{rcl}
\displaystyle  \left(\frac{t}{(t-1)^2}\right)'&=&\displaystyle \frac{(t)' (t-1)^2-t\left((t-1)^2\right)' }{(t-1)^4}\\
&=&\displaystyle \frac{(t-1)^2 - t \left(t^2-2t+1\right)' }{(t-1)^4}\\
&=&\displaystyle \frac{(t-1)^2 - t \left(2t-2\right) }{(t-1)^4}\\
&=&\displaystyle \frac{\cancel{(t-1)} \left((t-1) - 2t \right)}{(t-1)^{\cancel{4} ~3}}\\
&=&\displaystyle \frac{-t -1}{(t-1)^3}\\
&=&\displaystyle-\frac{t+1}{(t-1)^3}
\end{array}
\]


}


\solution{\ref{problemd/dx((x+1)/(x^3+1))}

\[\begin{array}{rcl}
\displaystyle \frac{\diff }{\diff x}\left(\frac{x+1}{x^3+1}\right)&=&\displaystyle \frac{\diff }{\diff x}\left(\frac{\cancel{x+1}}{\cancel{(x+1)}(x^2-x+1)}\right)\\
&=&\displaystyle \frac{\diff }{\diff x}\left(\frac{1}{x^2-x+1}\right)\\
\multicolumn{3}{l}{\textbf{Variant I: use quotient rule.}}\\
&=&\displaystyle \frac{ \frac{\diff }{\diff x}(1)\cdot (x^2-x+1)-1\cdot\frac{\diff }{\diff x}\left(x^2-x+1\right)}{\left(x^2-x+1\right)^2}\\
&=&\displaystyle \frac{-2x+1}{\left(x^2-x+1\right)^2}\\
\multicolumn{3}{l}{\textbf{Variant I: use chain rule.}}\\
&=&\displaystyle \frac{\diff }{\diff x}\left(\left(x^2-x+1\right)^{-1}\right)\\
&=&\displaystyle -(x^2-x+1)^{-2}\frac{\diff}{\diff x}(x^2-x+1)\\
&=&\displaystyle -(x^2-x+1)^{-2}(2x-1)\\
&=&\displaystyle \frac{-2x+1}{\left(x^2-x+1\right)^2}.
\end{array}
\]
}

\solution{\ref{problemDifferentialtexDivsqrt(1+2divx^2)}
\[
\begin{array}{rclr|r}
\displaystyle\left(\frac{x }{\sqrt{1+\frac{2}{x^2}}}\right)'&=&\displaystyle\frac{\sqrt{1+\frac{2}{x^2}}- x\left(\sqrt{1+\frac{2}{x^2}}\right)'}{1+\frac{2}{x^2}} =\frac{\sqrt{1+\frac{2}{x^2}}-  x\frac{\frac12}{\sqrt{1 +\frac{ 2}{ x^2 }}}  \left(\frac{2}{x^2}\right)'}{1+\frac{2}{x^2}}\\
&=& \displaystyle\frac{\sqrt{1+\frac{2}{x^2}}+  \frac{2}{x^2\sqrt{1 +\frac{ 2}{ x^2 }}} }{1+\frac{2}{x^2}} = \frac{x^2\left(1+\frac{2}{x^2}\right)+  2 }{x^2\left(1+\frac{2}{x^2}\right)^{\frac32}}= \frac{x^2+4}{x^2\left(1+\frac{2}{x^2}\right)^{\frac32}}
\end{array}
\]
Please note that this problem can be solved also by applying the transformation 
\[
\displaystyle  \frac{x}{\sqrt{1+\frac{2}{x^2}}}= \frac{x}{\sqrt{\frac{x^2+2}{x^2}}} =\frac{x}{\frac{1}{\pm x}\sqrt{x^2+2}} = \frac{\pm x^2}{\sqrt{x^2+2}}
\]
before differentiating, however one must not forget the $\pm $ sign arising from $\sqrt{x^2}=\pm x$. Our original approach resulted in more algebra, but did not have the disadvantage of dealing with the $\pm$ sign.
}


\begin{problem}Verify that the coordinates of the given point satisfy the given equation. Use implicit differentiation to find an equation of the tangent line to the curve at the given point.

\begin{enumerate}[ref={\fcProblemRef}]
\item \label{problemImplicitTangentysin(2x)=xcos(2y)point(pi/2,pi/4)} $y\sin (2x)=x\cos (2y) $, $\left(\frac{\pi}{2}, \frac{\pi}{4}\right)$. 

\psset{xunit=0.3cm, yunit=0.3cm}
\begin{pspicture}(-5.8,-5.8)(5.8,5.8)
\fcAxesStandard{-5.5}{-5.5}{5.5}{5.5}
\fcLabels{5.5}{5.5}
\fcXTickWithLabel{1}{$1$}
\fcImplicitIId[linestyle=solid, linecolor=red, linewidth-=0.05, showGridImplicitIId=false]{-5}{-5}{1000}{1000}{0.01}{0.01}{2 x mul 180 mul 3.141592654 div sin y mul 2 y mul 180 mul 3.141592654 div cos x mul sub} 

\fcFullDot[linecolor=blue]{3.141592654 2 div}{3.141592654 4 div}
\end{pspicture}

\answer{$y=\frac{1}{2}x$}

\item $x^2+x y+y^2=3 $, $(1,-2)$ (ellipse). 

\psset{xunit=0.5cm, yunit=0.5cm}
\begin{pspicture}(-3.6,-3.8)(3.6,3.6)
\tiny
\fcAxesStandard{-3.5}{-3.5}{3.5}{3.5}
\fcLabels{3.5}{3.5}
\fcXTickWithLabel{1}{$1$}
\fcImplicitIId[linestyle=solid, linecolor=red, linewidth-=0.05, showGridImplicitIId=false]{-2}{-2}{400}{400}{0.01}{0.01}{x x mul x y mul add y y mul add 3 sub}
%\psline[linecolor=blue](-3.5,-2)(3.5,-2)
\fcFullDot[linecolor=blue]{1}{-2}
\end{pspicture}


\answer{$y=-2 $}

\item $x^2+2x y-y^2+x=2 $, $(1,2)$ (hyperbola). 

\psset{xunit=0.4cm, yunit=0.4cm}
\begin{pspicture}(-5.6,-5.6)(5.6,5.6)
\tiny
\fcAxesStandard{-5.5}{-5.5}{5.5}{5.5}
\fcLabels{5.5}{5.5}
\fcXTickWithLabel{1}{$1$}
\fcImplicitIId[linestyle=solid, linecolor=red, linewidth-=0.05, showGridImplicitIId=false]{-5}{-5}{500}{500}{0.02}{0.02}{x x mul x y mul 2 mul add y y mul sub x add 2 sub}
%\psline[linecolor=blue](-3.5,-2)(3.5,-2)
\fcFullDot[linecolor=blue]{1}{2}
\end{pspicture}

\answer{$y= \frac{7}{2} x-\frac{3}{2}$}


\end{enumerate}
\end{problem}
\solution{\ref{problemImplicitTangentysin(2x)=xcos(2y)point(pi/2,pi/4)}

\psset{xunit=0.5cm, yunit=0.5cm}
\begin{pspicture}(-2,-2)(2,2)
\fcAxesStandard{-5.5}{-5.5}{5.5}{5.5}
\fcLabels{5.5}{5.5}
\fcXTickWithLabel{1}{$1$}
\fcImplicitIId[linestyle=solid, linecolor=red, linewidth-=0.05, showGridImplicitIId=false]{-5}{-5}{1000}{1000}{0.01}{0.01}{2 x mul 180 mul 3.141592654 div sin y mul 2 y mul 180 mul 3.141592654 div cos x mul sub} 

\psline[linecolor=blue](-5,-2.5)(5,2.5)
\fcFullDot[linecolor=blue]{3.141592654 2 div}{3.141592654 4 div}
\end{pspicture}


First we verify that the point $\displaystyle (x,y)=\left(\frac{\pi}{2}, \frac{\pi}{4}\right)$ indeed satisfies the given equation:

\[
\begin{array}{rcll|l}
\displaystyle y \sin (2x)_{|x=\frac{\pi}{2}, y=\frac{\pi}{4}}= \frac{\pi}{4}\sin \pi &=& \displaystyle 0 && \text{left hand side}\\
\displaystyle x \cos (2y)_{|x=\frac{\pi}{2}, y=\frac{\pi}{4}}= \frac{\pi}{2}\cos \left(\frac{\pi}{2} \right)&=&\displaystyle 0 && \text{right hand side}\\
\end{array}
\]
so the two sides of the equation are equal (both to $0$) when $x=\frac{\pi}{2}$ and $y=\frac{\pi}{4}$.

Since we are looking an equation of the tangent line, we need to find  $\frac{\diff y}{\diff x}_{|x=\frac{\pi}{2}, y= \frac{\pi}{4}}$ - that is, the derivative of $y$ at the point $x=\frac{\pi}{2}$, $y= \frac{\pi}{4}$. To do so we use implicit differentiation.
\[
\begin{array}{rcll|l}
\displaystyle y \sin (2x)&=&\displaystyle x\cos (2y)&&\frac{\diff }{\diff x}\\
\displaystyle \frac{\diff y}{\diff x} \sin (2x) +y \frac{\diff }{\diff x}\left(\sin (2x)\right)&=&\displaystyle  \cos (2y)+x\frac{\diff }{\diff x}(\cos (2y))\\
\displaystyle \frac{\diff y}{\diff x}\sin (2x)+2y\cos (2x)&= & \displaystyle \cos (2y)-2x \sin (2y) \frac{\diff y}{\diff x}\\
\displaystyle \frac{\diff y}{\diff x}(\sin (2x)+2x\sin (2y))&=&\displaystyle \cos (2y)-2y\cos (2x)&& \text{Set }x=\frac{ \pi}{2}, y=\frac{\pi}{4}\\
\displaystyle \frac{\diff y}{\diff x}_{|x=\frac{\pi}{2}, y=\frac{\pi}{4}} \left(\sin \pi+\pi \sin \left(\frac{\pi}{2}\right)\right)&=& \displaystyle \cos \left(\frac{\pi}{2}\right)-\frac{\pi}{2}\cos \pi\\
\displaystyle \pi \frac{\diff y}{\diff x}_{|x=\frac{\pi}{2}, y=\frac{\pi}{4}} &=& -\frac{\pi}{2}\cos \pi\\
\displaystyle \frac{\diff y}{\diff x}_{|x=\frac{\pi}{2}, y=\frac{\pi}{4}}&=& \displaystyle \frac{1}{2}.
\end{array}
\]
Therefore the equation of the line through $x=\frac{\pi}{2}, y=\frac{\pi}{4}$ is 
\[
\begin{array}{rcl}
\displaystyle y-\frac{\pi}{4}&=&\displaystyle \frac{1}{2}\left( x-\frac{\pi}{2} \right)\\
y&=&\displaystyle \frac{1}{2} x .
\end{array}
\]
}
\begin{problem}
Find the maximum and minimum values of $f$ on the given interval. Indicate the values of $x$ for which those are attained.
\begin{enumerate}[ref={\fcProblemRef}]
\item $\displaystyle f(x)=5+4x-2x^3$, $x\in[-1,1] $.

\answer{$f_{max}=f\left(\frac{\sqrt{6}}{3} \right)= \frac{8}{9} \sqrt{6} +5  $, $f_{min} =f\left(-\frac{ \sqrt{6} }{3} \right)=5 - \frac{8 }{9}\sqrt{6} $}

\item $\displaystyle f(x)=2x^3-x^2-20x+1$, $x\in [-4,3]$.

\answer{$f_{max}=f\left( -\frac{5}{3}\right)=\frac{602}{27}$, $f_{min}=f\left(-4\right)=-63$}

\item $\displaystyle f(x)=3x^4-4x^3-12x^2+1$, $x\in [-2, 3]$.

\answer{$f_{max}=f\left(-2\right)=33$, $f_{min}=f\left(2\right)=-31 $}

\item $\displaystyle f(x)=x e^{3 x}$, $x\in \left[-3, \frac{1}{6}\right]$.

\answer{$f_{max}=f\left( \frac{1}{6}\right)=\frac{e^{\frac{1}{2}}}{6}\approx 0.274787 $, $f_{min}=f\left( -\frac{1}{3}\right)= -\frac{1}{3e}\approx -0.122626 $}
\item $\displaystyle f(x)=\left(x-2\right) \left(x+1\right) e^{x} $, $x\in \left[-5,2\right]$.

\answer{$\begin{array}{l}
f_{max}=f\left(-\frac{\sqrt{13}}{2}-\frac{1}{2}
\right)= \left(\sqrt{13}+2\right) e^{\left(-\frac{\sqrt{13}}{2}-\frac{1}{2}\right)}\approx 0.560448\\
f_{min}=f\left(\frac{\sqrt{13}}{2}-\frac{1}{2} \right)= \left(-\sqrt{13}+2\right) e^{\left(\frac{\sqrt{13}}{2}-\frac{1}{2}\right)} \approx -5.907619
\end{array}
$}
\item $\displaystyle f(x)=$, $x\in \left[-3,3\right]$.

\answer{$\begin{array}{rcl}
f_{max}&=&f\left( \frac{\sqrt{3}}{2}-\frac{1}{2}
\right) =\left(\frac{\sqrt{3}}{2}+\frac{1}{2}\right) e^{\frac{\sqrt{3}}{2}-1}\approx 1.194743 \\
f_{min}&=&f\left(-\frac{\sqrt{3}}{2}-\frac{1}{2} \right)=\left(-\frac{\sqrt{3}}{2}+\frac{1}{2}\right) e^{-\frac{\sqrt{3}}{2}-1}\approx -0.056638 \end{array}$}
\end{enumerate}
\end{problem}

\begin{problem}~
\begin{enumerate}[ref={\fcProblemRef}]
\item Find the linearization of $f(x) = \sqrt{x}$ at $a = 100$ and use it to approximate
$\sqrt{99.8}$.

\answer{$L(x) = 10 + 0.05(x-100)$. Therefore $\sqrt{99.8} \approx L(99.8) = 9.99$.}

\item Find the linearization of $f(x)=\sqrt[3]{8+x}$ at $a=0$ and use it to approximate $\sqrt[3] {7.97}$.

\answer{ $\sqrt[3]{8+x}\approx \frac{1}{12}x+2$. Therefore $\sqrt[3]{7.97}\simeq \frac{799}{400} =1.9975$}

\item \label{problem-linearization-estimate0.9999power2014} Use a linear approximation to estimate $(0.9999)^{2014}$. 

\answer{$(0.9999)^{2014} \approx 0.7986$.}

\end{enumerate}
\end{problem}
\solution{\ref{problem-linearization-estimate0.9999power2014}
Let $f(x)=x^{2014}$. We are looking to approximate $(0.9999)^{2014}= f(0.9999)$. As $f(1)=1^{2014}=1$ is easy to compute, is makes sense to use linear approximation at $a=1$ to approximate $(0.9999)^{2014}$. We have that 
\[
f'(x)=2014x^{2013} \quad .
\]
Therefore the linear approximation of $f(x)=x^{2014}$ at $a=1$ is:
\[
f(x)\approx f(1) +f'(1)(x-1)= 1^{2014}+2014 \cdot 1^{2013}(x-1)=1+ 2014(x-1)=2014x-2013 \quad .
\]
Therefore 
\[
f(0.9999)\approx 2014\cdot 0.9999 -2013=1\cdot 0.9999 +2013(0.9999-1)=0.9999-2013\cdot 0.0001= 0.9999-0.2013=0.7986
\]


A computation with computer shows that $0.999^{2014}=0.817577\dots $. While our approximation of $0.7986$ is less than perfect, it is within the same order of magnitude. We study techniques for estimating errors in linear approximations later.
}


\begin{problem}
Find the
\begin{multicols}{2}
\begin{itemize}
\item the implied domain of $f$,
\item $x$ and $y$ intercepts of $f$,
\item horizontal and vertical asymptotes,
\item intervals of increase and decrease,
\item local and global minima, maxima,
\item intervals of concavity,
\item points of inflection.
\end{itemize}
\end{multicols}
Label all relevant points on the graph. Show all of your computations.
\begin{enumerate}[ref={\fcProblemRef}]
\item \label{problemSketchCurve(2x^2-5x+9/2)/(x^2-3 x+3)} $\displaystyle f(x)=\frac{2 x^{2}-5 x+\frac{9}{2}}{x^{2}-3 x+3}$. \textbf{For this problem, indicate only the $x$-coordinates of the local maxima/minima and inflection points; you do not need to compute the $y$-coordinates of those points.} 

Computation shows that 
$\displaystyle
f'(x)=\frac{- x^{2}+3 x-\frac{3}{2} }{ \left(x^2- 3 x+3\right)^2}
$
and that 
$\displaystyle f''(x)=\frac{(2x-3)x(x-3)}{\left(x^2- 3 x+3\right)^3}$; you may use those computations without further justification. 
 
\psset{xunit=0.6cm, yunit=0.6cm}
\begin{pspicture}(-5, -1)(5,4)
\psframe*[linecolor=white](-5,-1)(5,4)
\tiny
\psaxes[ticks=none, labels=none]{<->}(0,0) (-5,-0.5) (5, 3.5)
\fcLabels{5}{3.5}
%Function formula: \frac{2 x^{2}-5 x+9/2}{x^{2}-3 x+3}
\psplot[linecolor=\fcColorGraph, plotpoints=1000]{-5}{5 } {4.5 x -5 mul add x 2 exp 2 mul add 3 x -3 mul add x 2 exp add div }
\end{pspicture}

\answer{
\begin{tabular}{l}
$y$-intercept: $\frac32$\\
horizontal asymptote: $y=2$, vertical: none\\
increasing on
$\left(\frac{3-\sqrt{3}}2, \frac{3+ \sqrt{3}}2 \right) $, decreasing on $\left(-\infty, \frac{3-\sqrt{3}}2\right)\cup \left(\frac{3+\sqrt{3}}2, \infty\right) $\\
local and global min at $x=\frac{3-\sqrt{3}}2$, local and global max at $x=\frac{3+\sqrt{3}}2$\\
concave up on $\left(0, \frac32\right)\cup \left(3, \infty \right)$, concave down $\left(-\infty, 0\right)\cup \left(\frac32, 3\right)$\\
inflection points at $x=0,x=\frac32, x=3$
\end{tabular}
}

\item $\displaystyle f(x)=\frac{x^{2}+3 x+1}{x^{2}+2 x}$. \textbf{For this problem, indicate only the $x$-coordinates of the local maxima/minima and inflection points; you do not need to compute the $y$-coordinates of those points.} 

Computation shows that 
$\displaystyle
f'(x)= \frac{- x^{2}-2 x-2}{\left(x^{2}+2 x\right)^{2}} 
$
and that 
$\displaystyle f''(x)= \frac{2 x^{3}+6 x^{2}+12 x+8}{\left(x^{2}+2 x\right)^{3}} =\frac{\left(x+1\right) \left(2 x^{2}+4 x+8\right)}{\left(x^{2}+2 x\right)^{3}} $; you may use those computations without further justification. 

\psset{xunit=0.7cm, yunit=0.7cm}
\begin{pspicture}(-5, -4.7)(5,5)
\psframe*[linecolor=white](-5,-4.7)(5,5)
\tiny
\psaxes[ticks=none, labels=none]{<->}(0,0)(-5,-4.5)(5,4.5)
\fcLabels{5}{5}
%Function formula: \frac{x^{2}+3 x+1}{x^{2}+2 x}
\psplot[linecolor=\fcColorGraph, plotpoints=1000]{0.1}{5}{1 x 3 mul add x 2 exp add x 2 mul x 2 exp add div }
%Function formula: \frac{x^{2}+3 x+1}{x^{2}+2 x}
\psplot[linecolor=\fcColorGraph, plotpoints=1000]{-1.9}{-0.1}{1 x 3 mul add x 2 exp add x 2 mul x 2 exp add div }
%Function formula: \frac{x^{2}+3 x+1}{x^{2}+2 x}
\psplot[linecolor=\fcColorGraph, plotpoints=1000]{-5}{-2.1}{1 x 3 mul add x 2 exp add x 2 mul x 2 exp add div }
\end{pspicture}

\answer{
\begin{tabular}{l}
$y$-intercept: none, $x$-intercepts: $\frac{-3\mp\sqrt{5}}2$ \\
horizontal asymptote: $y=1$, vertical: $x=-2$ and $x=0$\\
always decreasing\\
no local/global minima/maxima\\
concave down on $\left(-\infty,-2\right)\cup \left(-1,0 \right)$, concave up on $\left(-2, -1\right)\cup \left(0, \infty\right)$\\
inflection point at $x=-1$
\end{tabular}
}

\item $\displaystyle f(x)= \frac{3 x^{3}-30 x^{2}+97 x-99}{x^{2}-6 x+8} $. \textbf{For this problem, do not find the $x$-intercepts of the function. Indicate only the $x$-coordinates of the local maxima/minima and inflection points; you do not need to compute the $y$-coordinates of those points.} 

Computation shows that 
$\displaystyle
f'(x)=\frac{3 x^{4}-36 x^{3}+155 x^{2}-282 x+182}{\left(x^{2}-6 x+8\right)^{2}}=\frac{\left(x^{2}-6 x+7\right) \left(3 x^{2}-18 x+26\right)}{\left(x^{2}-6 x+8\right)^{2}}
$
and that 
$\displaystyle f''(x)=\frac{2 x^{3}-18 x^{2}+60 x-72}{\left(x^{2}-6 x+8\right)^{3}}=\frac{\left(x-3\right) \left(2 x^{2}-12 x+24\right)}{\left(x^{2}-6 x+8\right)^{3}}$; you may use those computations without further justification. 

\psset{xunit=0.15cm, yunit=0.15cm}
\begin{pspicture}(-11,-21)(12,11)
\fcAxesStandard{-10}{-20}{10}{10}
\newcommand{\theFun}{3 x x x mul mul mul -30 x x mul mul 97 x mul -99 add add add x x mul -6 x mul 8 add add div}
\psplot[plotpoints=500, linecolor=\fcColorGraph]{-7}{1.97}{\theFun}
\psplot[plotpoints=500, linecolor=\fcColorGraph]{2.02}{3.98}{\theFun}
\psplot[plotpoints=500, linecolor=\fcColorGraph]{4.03}{10}{\theFun}
\end{pspicture}



\answer{
\begin{tabular}{l}
$y$-intercept $\frac{-99}{8}$, $x$-intercept: not requested \\
horizontal asymptote: none, vertical: $x=2$ and $x=4$\\
increasing on $\left(- \infty, -\sqrt{2}+3\right)\cup \left(-\frac{\sqrt{3}}{3}+3, \frac{\sqrt{3}}{3}+3\right)\cup \left(\sqrt{2}+3, \infty\right)
$, decreasing on the complement intervals\\
local maxima at $x=-\sqrt{2}+3$, $ x=\frac{\sqrt{3}}{3}+3$\\
local minima at $x=-\frac{\sqrt{3}}{3}+3$, $ x=\sqrt{2}+3$\\
concave up on $\left(\left(2, 3\right)\cup \left(4, \infty\right)\right)$\\ 
concave down on $\left(-\infty, 2\right)\cup \left(3, 4\right)$\\
inflection point at $x=3$
\end{tabular}
}

\end{enumerate}
\end{problem}
\solution{\ref{problemSketchCurve(2x^2-5x+9/2)/(x^2-3 x+3)}

\textbf{Domain.} We have that $f$ is not defined only when we have division by zero, i.e.,  if $x^2-3x+3$ equals zero. However, the roots of $x^{2}-3x+3$ are not real numbers: they are $\frac{3\pm \sqrt{3^2-4\cdot 3 }}{2}= \frac{3\pm \sqrt{-3}}{2}$, and therefore $x^2-3x+3$ can never equal zero. Alternatively, completing the square shows that the denominator is always positive:
\[
x^2-3x+3=x^2-2\cdot \frac{3}{2} x+\frac{9}{4}-\frac{9}{4}+3=\left(x-\frac{3}{2}\right)^2+\frac{3}{4} >0 
\]
Therefore the domain of $f$ is all real numbers.

\textbf{$x$, $y$-intercepts.}  The $y$-intercept of $f$ equals by definition $\displaystyle f(0)= \frac{ 2\cdot 0^2-5\cdot 0+ \frac{9}{2}}{0^2-3\cdot 0 + 3}=\frac{\frac{9}{2}}{3}= \frac{3}{2}$. The $x$ intercept of $f$ is those values of $x$ for which $f(x)=0$. The graph of $f$ shows no such $x$, and that is confirmed by solving the equation $f(x)=0$:

\[
\begin{array}{rcll|l}
f(x)&=&0\\
\displaystyle \frac{2x^2-5x+\frac{9}{2}}{x^2-3 x+3}&=&0&&\text{Mult. by }x^2-3 x+3\\
\displaystyle 2x^2-5x+\frac{9}{2}&=&0\\
\displaystyle x_1, x_2&=&\displaystyle \frac{5 \pm \sqrt{25- 4\cdot 2\cdot \frac{9}{2}}}{4}=\frac{5\pm \sqrt{-9}}{4}\quad ,
\end{array}
\]
so there are no real solutions (the number $\sqrt{-9}$ is not real).

\textbf{Asymptotes.} Since $f$ is defined for all real numbers, its graph has no vertical asymptotes. To find the horizontal asymptote(s), we need to compute the limits $\lim\limits_{x\to \infty } f(x)$ and $\lim\limits_{x\to -\infty} f (x)$. The two limits are equal, as the direct computation below shows:
\[
\begin{array}{rcll|l}
\displaystyle \lim_{x\to \pm\infty} \frac{2x^2-5x+\frac{9}{2}}{x^2-3 x+3}&=& \displaystyle  \lim_{x\to \pm\infty}\frac{\left(2x^2-5x+ \frac{9}{2}\right)\frac{1}{x^2}}{\left(x^2-3 x+3\right)\frac{1}{x^2}} &&\begin{array}{l}\text{Divide by leading}\\ \text{monomial in denominator}\end{array}\\
&=&\displaystyle\lim_{x\to \pm \infty}\frac{2-\frac{5}{x} +\frac{9}{2x^2}}{1-\frac{3}{x}+\frac{3}{x^2}}\\
&=&\displaystyle \frac{2-0+0}{1-0+0}\\
&=& 2
\end{array}
\]
Therefore the graph of $f(x)$ has a single horizontal asymptote at $y=2$.

\textbf{Intervals of increase and decrease.}
The intervals of increase and decrease of $f$ are governed by the sign of $f'$. We compute:

\[
\begin{array}{rcl}
f'(x)&=&\displaystyle \left(\frac{2x^2-5x+\frac{9}{2} }{x^2- 3 x+3} \right)' \\
&=&\displaystyle \frac{\left(2x^2-5x+\frac{9}{2}\right)'\left(x^2- 3 x+3\right)-\left(2x^2-5x+\frac{9}{2}\right)\left(x^2- 3 x+3\right)' }{ \left(x^2- 3 x+3\right)^2}\\
&=&\displaystyle \frac{- x^{2}+3 x-\frac{3}{2} }{ \left(x^2- 3 x+3\right)^2}
\end{array}
\]
As the denominator is a square, the sign of $f'$ is governed by the sign of $- x^{2}+3 x-\frac{3}{2}$. To find where $- x^{2}+3 x-\frac{3}{2}$ changes sign, we compute the zeroes of this expression:

\[
\begin{array}{rcll|l}
\displaystyle - x^{2}+3 x-\frac{3}{2}&=&0&& \text{Mult. by }-2\\
\displaystyle  2x^{2}-6 x+3&=&0\\
x_1, x_2&=&\displaystyle \frac{ 6\pm \sqrt{36-24 }}{4}=\frac{6\pm \sqrt{12}}{4}\\
x_1, x_2&=&\displaystyle \frac{3\pm \sqrt{3}}{2} 
\end{array}
\]
Therefore the quadratic $- x^{2}+3 x-\frac{3}{2}$ factors as 
\begin{equation}
\label{eq1problemSketch(2x^2-5x+9/2)/(x^2-3 x+3)}
-(x-x_1)(x-x_2)=-\left(x-\left(\frac{3- \sqrt{3}}{2} \right)\right)\left(x-\left(\frac{3+ \sqrt{3}}{2}\right)\right)
\end{equation} 

The points $x_1, x_2$ split the real line into three intervals: $\left(-\infty, \frac{3- \sqrt{3}}{2}\right)$, $\left(\frac{3- \sqrt{3}}{2}, \frac{3+ \sqrt{3}}{2} \right)$ and $\left(\frac{3+ \sqrt{3}}{2}, \infty \right)$, and each of the factors of \eqref{eq1problemSketch(2x^2-5x+9/2)/(x^2-3 x+3)} has constant sign inside each of the intervals. If we choose $x$ to be a very negative number, it follows that $-(x-x_1)(x-x_2)$ is a negative, and therefore $ f'(x)$ is negative for $x\in(-\infty, \frac{3- \sqrt{3}}{2})$. For $x\in (\frac{3- \sqrt{3}}{2}, \frac{3+ \sqrt{3}}{2})$, exactly one factor of $f'$ changes sign and therefore $f'(x)$ is positive in that interval; finally only one factor of $f'(x)$ changes sign in the last interval so $f'(x)$ is negative on $(\frac{3+ \sqrt{3}}{2}, \infty )$.

Our computations can be summarized in the following table. 

\begin{tabular}{|lll|}\hline
Interval & $f'(x)$ & $f(x)$   \\\hline
$\left(-\infty, \frac{3- \sqrt{3}}{2}\right)$ & $-$& $\searrow $ \\\hline
$\left(\frac{3- \sqrt{3}}{2}, \frac{3+ \sqrt{3}}{2} \right)$ &$+$&$\nearrow$\\\hline
$\left( \frac{3+ \sqrt{3}}{2}, \infty\right)$&$-$&$\searrow$ \\\hline
\end{tabular}

\textbf{Local and global minima and maxima. } The table above shows that $f(x)$ changes from decreasing to increasing at $x=x_1=\frac{3- \sqrt{3}}{2}$ and therefore $f$ has a local minimum at that point. The table also shows that $f(x)$ changes from increasing to decreasing at $ x=x_2=\frac{3+ \sqrt{3}}{2}$ and therefore $f$ has a local maximum at that point. The so found local maximum and local minimum turn out to be global: there are two things to consider here. First, no other finite point is critical and thus cannot be maximum or minimum - however this leaves out the possibility of a maximum/minimum ``at infinity''. This possibility can be quickly ruled out by looking at the graph of $f$. To do so via algebra, compute first $f(x_1)$ and $f(x_2)$:

\[
\begin{array}{rcl}
\displaystyle f(x_1)= f\left(\frac{3- \sqrt{3}}{2} \right)&=& \displaystyle \frac{2\left(\frac{3- \sqrt{3}}{2} \right)^2-5\left(\frac{3- \sqrt{3}}{2} \right)+\frac{9}{2} }{\left(\frac{3- \sqrt{3}}{2} \right)^2- 3 \left(\frac{3- \sqrt{3}}{2} \right)+3}=2-\frac{\sqrt{3}}{3} \\

\displaystyle f(x_2)= f\left(\frac{3+ \sqrt{3}}{2} \right)&=& \displaystyle \frac{2\left(\frac{3+ \sqrt{3}}{2} \right)^2-5\left(\frac{3+ \sqrt{3}}{2} \right)+\frac{9}{2} }{\left(\frac{3+ \sqrt{3}}{2} \right)^2- 3 \left(\frac{3+ \sqrt{3}}{2} \right)+3}=2+\frac{\sqrt{3}}{3}\quad . 
\end{array}
\]
On the other hand, while computing the horizontal asymptotes, we established that $\lim\limits_{x\to\pm \infty}f(x)=2$. This implies that all $x$ sufficiently far away from $x=0$, we have that $f(x)$ is close to $2 $. Therefore $f(x)$ is larger than $f(x_1)$ and smaller than $f(x_2)$ for all sufficiently far away from $x=0$. This rules out the possibility for a maximum or a minimum ``at infinity'', as claimed above.

\textbf{Intervals of concavity. } 
The intervals of concavity of $f$ are governed by the sign of $f''$. The second derivative of $f$ is:
\[
\begin{array}{rcll|@{}l}
f''(x)&=&\displaystyle (f'(x))'= \left( \frac{- x^{2}+3 x-\frac{3}{2} }{ \left(x^2- 3 x+3\right)^2 } \right)'\\
&=&\displaystyle \left(- x^{2}+3 x-\frac{3}{2} \right)' \left(\frac{1 }{\left(x^2- 3 x+3\right)^2}\right)+ \left(- x^{2}+3 x-\frac{3}{2} \right)\left(\frac{1}{\left(x^2- 3 x+3\right)^2}\right)' &&\begin{array}{@{}l}\text{second differentiation:}\\\text{chain rule }\end{array}\\
&=&\displaystyle (-2x+3)\left(\frac{1}{\left(x^2- 3 x+3\right)^2} \right)+\left(- x^{2}+3 x-\frac{3}{2} \right)(-2)\frac{\left(x^2- 3 x+3\right)'}{\left(x^2- 3 x+3\right)^{3}}\\
&=&\displaystyle (-2x+3)\left(\frac{1}{\left(x^2- 3 x+3\right)^2}\right) +\left(2x^{2}-6 x+3 \right) \frac{(2x-3)}{\left(x^2- 3 x+3\right)^{3}}&&\text{factor out }\frac{(2x-3)}{\left(x^2- 3 x+3\right)^2}\\
&=&\displaystyle \frac{(2x-3)}{\left(x^2- 3 x+3\right)^2}\left(-1+\frac{(2x^{2}-6 x+3)}{\left(x^2- 3 x+3\right)}\right)\\
&=&\displaystyle \frac{(2x-3)}{\left(x^2- 3 x+3\right)^2}\left(\frac{-\left(x^2- 3 x+3\right)+(2x^{2}-6 x+3)}{\left(x^2- 3 x+3\right)} \right)\\
&=&\displaystyle \frac{(2x-3)(x^{2}-3 x )}{\left(x^2- 3 x+3\right)^3}\\
&=&\displaystyle \frac{(2x-3)x(x-3)}{\left(x^2- 3 x+3\right)^3}
\end{array}
\]
When computing the domain of $f$, we established that the denominator of the above expression is always positive. Therefore $f''(x)$ changes sign when the terms in the numerator change sign, namely, at $x=0$, $x=\frac{3}{2}$ and $x=3$. 

Our computations can be summarized in the following table. In the table, we use the $\cup$ symbol to denote that the function is concave up in the indicated interval, and $\cap$ to denote that the function is concave down.

\begin{tabular}{|lll|}\hline
Interval & $f''(x)$ & $f(x)$   \\\hline
$(-\infty, 0)$ & $-$& $\cap$ \\\hline
$(0, \frac{3}{2})$ &$+$&$\cup$\\\hline
$(\frac{3}{2}, 3)$&$-$&$\cap$ \\\hline
$(3, \infty)$&$+$&$\cup$ \\\hline
\end{tabular}

\textbf{Points of inflection.} The preceding table shows that $f''(x)$ changes sign at $0, \frac{3}{2}, 3$ and therefore the points of inflection are located at $x=0, x=\frac{3}{2}$ and $x=3$, i.e., the points of inflection are $\left(0, f(0)\right)= \left(0, \frac{3}{2} \right) $, $\left(\frac {3}{2}, f\left(\frac{3}{2}\right)\right) =\left(\frac{3}{2}, 2\right)$, $\left(3, f(3)\right)=\left(3, \frac{5}{2}\right)$.

We can command our graphing device to use the so computed information to label the graph of the function. Finally, we can confirm visually that our function does indeed behave in accordance with our computations.

\psset{xunit=0.6cm, yunit=0.6cm}
\begin{pspicture}(-5, -1)(5.2,4)
\psframe*[linecolor=white](-5,-1)(5,4)
\tiny
\psaxes[ticks=none, labels=none]{<->}(0,0) (-5,-0.5) (5, 3.5)
\fcLabels{5}{3.5}
%Function formula: \frac{2 x^{2}-5 x+9/2}{x^{2}-3 x+3}
\psplot[linecolor=\fcColorGraph, plotpoints=1000]{-5}{5 } {4.5 x -5 mul add x 2 exp 2 mul add 3 x -3 mul add x 2 exp add div }
\fcFullDot[linecolor=green]{0}{3 2 div}
\rput[r](-2, 0.2){infl.: $\left(0, \frac{3}{2}\right)$}
\psline[linestyle=dotted, arrows=->](-2, 0.2)(-0.05, 1.45)
\fcFullDot[linecolor=green]{3 2 div }{2}
\rput[r](1.2, 0.2){infl.: $\left(\frac{3}{2},2 \right)$}
\psline[linestyle=dotted, arrows=->](1.2, 0.2)(1.45, 1.95)
\fcFullDot[linecolor=green]{3 }{5 2 div}
\rput[l](2, 0.2){infl.: $\left(3, \frac{5}{2}\right)$}
\psline[linestyle=dotted, arrows=->](2, 0.2)(2.95, 2.45)
\fcFullDot[linecolor=blue]{3 3 sqrt add 2 div}{2 3 sqrt 3 div add}
\rput(2, 3){$\left(\frac{3+\sqrt{3}}{2}, 2+\frac{\sqrt{3}}{3} \right)$}
\fcFullDot[linecolor=blue]{3 3 sqrt sub 2 div}{2 3 sqrt 3 div sub}
\rput[r](-2, 3){$\left(\frac{3-\sqrt{3}}{2}, 2-\frac{\sqrt{3}}{3} \right)$}
\psline[linestyle=dotted, arrows=->](-2, 3)(! 3 3 sqrt sub 2 div 0.05 add 2 3 sqrt 3 div sub 0.05 add)
\end{pspicture}

}


\begin{problem}
Estimate the integral using a Riemann sum using the indicated sample points and interval length.
\begin{enumerate}[ref={\fcProblemRef}]
\item \label{problemRiemannSum-sqrt(8x+1)} $\displaystyle \int_0^4 \left(\sqrt{8x+1}\right)\diff x$. Use four intervals of equal width, choose the sample point to be the left endpoint of each interval. 

\answer{ $\Delta x = 1$ and $f(x) = \sqrt{8x+1}$. Thus ${\displaystyle \int_0^4 f(x) \diff x \approx 9 + \sqrt{17}}$.}

\item \label{problemRiemannSum-1div1plusxsquared} $\displaystyle \int\limits_{-3.5}^{-0.5} \frac{\diff x}{x^2+1} $. Use three intervals of equal width, choose the sample point to be the midpoint of each interval. 

\answer{ $\Delta x = 1$ and $f(x) = \frac{1}{x^2+1}$. Thus $\displaystyle \int \limits_{-3.5}^{-0.5} f(x) \diff x  \approx \Delta x\left(f{} \left(-3 \right)+ f{}\left( -2\right)+f{}\left(-1\right)\right)=\frac{4}{5}=0.8$.}
\item $\displaystyle\int_{0}^2 \frac{\diff x}{1+x+x^3}$. Use $\Delta x=\frac{1}2 $ and right endpoint sampling points.

\answer{$ \frac{1}{2}\left(\frac{8}{13}+\frac{1}{3}+\frac{8}{47}+\frac{1}{11}\right)=\frac{12197}{20163}\approx 0.604920$}

\end{enumerate}
\end{problem}
\solution{\ref{problemRiemannSum-sqrt(8x+1)}. The interval $[0,4]$ is subdivided into $n=4$ intervals, therefore the length of each is $\Delta x=1$. The intervals are therefore
\[
[0,1], [1,2], [2,3], [3,4]\quad .
\]
The problem asks us to use the left endpoints of each interval as sampling points. Therefore our sampling points are $0,1,2,3$. Therefore the Riemann sum we are looking for is
\[
\Delta x\left(f(0)+f(1)+f(2)+f(3) \right)=1\cdot \left(\sqrt{8\cdot 0+1}+\sqrt{8\cdot 1+1}+\sqrt{8\cdot 2+1}+\sqrt{8\cdot 3+1}\right)= 9+\sqrt{17}\approx 13.1231
\]

\hfil \hfil \psset{xunit=1cm, yunit=1cm}
\begin{pspicture}(-0.9, -0.9)(4.4,6.233433)
\tiny
\psline*[linecolor=\fcColorAreaUnderGraph, linewidth=0.1pt](0.000000, 0.000000)(0.000000, 1.000000)(1.000000, 1.000000)(1.000000, 0.000000)(0.000000, 0.000000)
\psline*[linecolor=\fcColorAreaUnderGraph, linewidth=0.1pt](1.000000, 0.000000)(1.000000, 3.000000)(2.000000, 3.000000)(2.000000, 0.000000)(1.000000, 0.000000)
\psline*[linecolor=\fcColorAreaUnderGraph, linewidth=0.1pt](2.000000, 0.000000)(2.000000, 4.123106)(3.000000, 4.123106)(3.000000, 0.000000)(2.000000, 0.000000)
\psline*[linecolor=\fcColorAreaUnderGraph, linewidth=0.1pt](3.000000, 0.000000)(3.000000, 5.000000)(4.000000, 5.000000)(4.000000, 0.000000)(3.000000, 0.000000)
\psline[linecolor=blue, linewidth=0.1pt](0.000000, 0.000000)(0.000000, 1.000000)(1.000000, 1.000000)(1.000000, 0.000000)(0.000000, 0.000000)
\psline[linecolor=blue, linewidth=0.1pt](1.000000, 0.000000)(1.000000, 3.000000)(2.000000, 3.000000)(2.000000, 0.000000)(1.000000, 0.000000)
\psline[linecolor=blue, linewidth=0.1pt](2.000000, 0.000000)(2.000000, 4.123106)(3.000000, 4.123106)(3.000000, 0.000000)(2.000000, 0.000000)
\psline[linecolor=blue, linewidth=0.1pt](3.000000, 0.000000)(3.000000, 5.000000)(4.000000, 5.000000)(4.000000, 0.000000)(3.000000, 0.000000)
%Function formula: (8 x+1)^{1/2}
\psplot[linecolor=\fcColorGraph, plotpoints=1000]{0}{4}{ 1 x 8 mul add 0.5 exp }
\psaxes(0,0)(-0.65,-0.65)(4.15,5.883433)
\end{pspicture}
}

\solution{
\ref{problemRiemannSum-1div1plusxsquared}. The interval $[-3.5,-0.5]$ is subdivided into $n=3$ intervals, therefore the length of each is $\Delta x=\frac{-0.5-(-3.5)}{3}=\frac{3}{3}= 1$. The intervals are therefore
\[
[-3.5,-2.5], [-2.5,-1.5], [-1.5,-0.5]\quad .
\]
The problem asks us to use the midpoint of each interval as a sampling point. Therefore our sampling points are $-3,-2,-1$. Therefore the Riemann sum we are looking for is
\[
\Delta x\left(f(-3)+f(-2)+f(-1) \right)=1\cdot \left( \frac{1}{10}+\frac{1}{5}+\frac{1}{2}\right)= 0.8\quad .
\]

\hfil \hfil 
\psset{xunit=1cm, yunit=1cm}
\begin{pspicture}(-3.9, -0.9)(1.4,1.499857)
\tiny

\psline*[linecolor=\fcColorAreaUnderGraph, linewidth=0.1pt](-3.500000, 0.000000)(-3.500000, 0.100000)(-2.500000, 0.100000)(-2.500000, 0.000000)(-3.500000, 0.000000)
\psline*[linecolor=\fcColorAreaUnderGraph, linewidth=0.1pt](-2.500000, 0.000000)(-2.500000, 0.200000)(-1.500000, 0.200000)(-1.500000, 0.000000)(-2.500000, 0.000000)
\psline*[linecolor=\fcColorAreaUnderGraph, linewidth=0.1pt](-1.500000, 0.000000)(-1.500000, 0.500000)(-0.500000, 0.500000)(-0.500000, 0.000000)(-1.500000, 0.000000)
\psline[linecolor=blue, linewidth=0.1pt](-3.500000, 0.000000)(-3.500000, 0.100000)(-2.500000, 0.100000)(-2.500000, 0.000000)(-3.500000, 0.000000)
\psline[linecolor=blue, linewidth=0.1pt](-2.500000, 0.000000)(-2.500000, 0.200000)(-1.500000, 0.200000)(-1.500000, 0.000000)(-2.500000, 0.000000)
\psline[linecolor=blue, linewidth=0.1pt](-1.500000, 0.000000)(-1.500000, 0.500000)(-0.500000, 0.500000)(-0.500000, 0.000000)(-1.500000, 0.000000)
\rput[t](-3.500000,-0.03){$-\frac{7}{2}$}\rput[t](-2.500000,-0.03){$-\frac{5}{2}$}\rput[t](-1.500000,-0.03){$-\frac{3}{2}$}\rput[t](-0.500000,-0.03){$-\frac{1}{2}$}
%Function formula: (x^{2}+1)^{-1}
\psplot[linecolor=\fcColorGraph, plotpoints=1000]{-3.5}{1}{ 1 x 2 exp add -1 exp }
\psaxes[ticks=none, labels=none, arrows=<-> ](0,0)(-3.65,-0.65)(1.15,1.149857)
\fcLabels{1.15}{1.149857}
\end{pspicture}
}
\solution{\ref{problemRiemannSum1/(3x^2+1)from-1to0with3intervalsLeftEndpt}

$\Delta x = \frac{1}{3}$ and $f(x) =\frac{1}{3 {{x}}^{2}+1}$. Thus $\displaystyle \int\limits_{-1}^0 f(x) \diff x$  is approximated by $\Delta x \left(f{}\left(-1\right)+f{}\left(-\frac{2}{3}\right)+f{}\left(-\frac{1}{3}\right)\right)=\frac{10}{21}$.

}



\begin{problem}
Integrate.
\begin{multicols}{2}
\begin{enumerate}[ref={\fcProblemRef}]
\item $\displaystyle \int\limits_{1}^{4}\sqrt{x}(1+x) \diff x$.

\answer{$\left[\frac{2}{5} x^{\frac{5}{2}}+\frac{2}{3} x^{\frac{3}{2}} \right]_{1}^{4}=\frac{256}{15}$}

\item $\displaystyle \int \limits_{1}^{4} \frac{ \frac{ 1}{ \sqrt{x}}+1+x}{ \sqrt{x}}  \diff x$.

\answer{$\left[ \frac{2}{3} x^{\frac{3}{2}}+2 \sqrt{x}+\ln{}\left|x\right|\right]_{1}^{4}=\ln{}\left(4\right)+\frac{20}{3}$}

\item $\displaystyle \int\limits_0^{1} \left(\sqrt[5]{x^6} + \sqrt[6]{x^5}\right) \diff x $.

\answer{$\left[\frac{5}{11} x^{\frac{11}{5}}+\frac{6}{11} x^{\frac{11}{6}} \right]_{0}^{1}=1$}

\vfill
\item $\displaystyle\int \frac{x}{1+x^2} \diff x$. 

\answer{$\frac{1}{2} \ln\left(x^2+1\right)+C $}

\item \label{problemIntx/(2x^2+1)} $\displaystyle\int\limits_{1}^{2} \frac{x}{2x^2+1 }  \diff x$.

\answer{$\frac14 \ln 3$}

\item $\displaystyle\int \frac{x}{\sqrt{1-x^2}} \diff x$. 

\answer{$-\sqrt{1-x^2}+C $}
\item $\displaystyle\int \frac{(\ln x)^3}{x} \diff x$.

\answer{$ \frac{(\ln x)^4}{4} +C$}

\vfill
\item $\displaystyle\int\limits_{e}^{e^3}\frac{\diff x}{x \sqrt[3]{\ln x}} $.

\answer{$ \frac{3}{2}\left( \sqrt[3]{9} -1\right)$}

\item $\displaystyle\int \frac{\cos \left(\ln x\right)}{x} \diff x$.

\answer{$ \sin (\ln x)+C$}

\end{enumerate}
\end{multicols}
\end{problem}
\input{\freecalcBaseFolder/modules/substitution-rule/homework/substitution-rule-definite-1-problem-4-solution}
\solution{\ref{problemIntegratefrom-3to2_x/(1-x^2)dx}

\[\begin{array}{rcll|l}
\displaystyle \int_{-3}^{-2} \frac{x}{1-x^2}\diff x&=&\displaystyle \int\limits_{\tiny \begin{array}{rcl}x&=&-3\\ u&=&-8\end{array}}^{\tiny  \begin{array}{rcl}x&=&-2\\ u&=&-3\end{array}} \frac{1}{u}\left(-\frac{1}{2}\diff u\right)&& \begin{array}{rcl}
u&=&1-x^2\\
\diff u&=&-2x\diff x\\
x\diff x&=&-\frac{1}{2}\diff u
\end{array}\\
&=&\displaystyle -\frac{1}{2}\left[\ln |u|\right]_{-8}^{-3}\\~\\
&=&\displaystyle -\frac{1}{2} \left(\ln|3|-\ln|8| \right)\\~\\
&=&\displaystyle \frac{\ln\left|\frac{8}{3}\right|}{2}
\end{array}
\]

}
\input{\freecalcBaseFolder/modules/substitution-rule/homework/substitution-rule-definite-1-problem-6-solution}




\begin{problem}Differentiate $f(x)$ using the Fundamental Theorem of Calculus part 1.
\begin{enumerate}[ref={\fcProblemRef}]
\item \label{problemDifferentiateFTC1int_x^1(2+t^4)^5dt}  ${\displaystyle f(x) = \int\limits_x^1 (2+t^4)^5 \; \diff t}$

\answer{${\displaystyle -\left(2+x^4\right)^5}$} 

\item $\displaystyle f(x)=\int\limits_{0}^{x^2} t^2\diff t $.

\answer{$f'(x)=2x^5$ }

\item \label{problemd/dx(int_(ln x)^(e^x)t^3dt)} $\displaystyle f(x)=\int\limits_{\ln x}^{e^x} t^3\diff t $.

\answer{$f'(x)=e^{4x}-\frac{(\ln x)^3}{x}$ }

\end{enumerate}
\end{problem}
\solution{\ref{problemd/dxint_1^x(t-sqrt(t))dt}
\[
\begin{array}{rcll|l}
\displaystyle \frac{\diff }{\diff x}\left(\int_{1}^x\left(t-\sqrt{t}\right)\diff t\right)&=& x-\sqrt{x}. &&\text{FTC, part 1}
\end{array}
\]
}

\solution{\ref{problemDifferentiateFTC1int_x^1(2+t^4)^5dt} %(Contributed by student Anamaria Ronayne)
We recall that the Fundamental Theorem of Calculus part 1 states that $\frac{\diff}{\diff x}\left(\int_{a}^{x}h(t)dt\right)=h(x)$
where $a$ is a constant. We can rewrite the integral so it has $x$ as the upper limit:
\[
f(x)=\int_{x}^{1}(2+1^4)^5\diff t =-\int_{1}^{x}(2+1^4)^5\diff t\quad.
\]
Therefore
\[
\frac{\diff}{\diff x}\left( -\int_{1}^{x}(2+t^4)^5 \diff t\right)=- \frac{\diff }{\diff x}\left(\int_{1}^{x}(2+t^4)^5\diff t \right)\stackrel{\text{FTC part 1}}{=}
-(2+x^4)^5\quad .
\]

}

\solution{\ref{problemd/dx(int_(ln x)^(e^x)t^3dt)}

\[
\begin{array}{rcl}
f'(x)&=&\displaystyle \frac{\diff }{\diff x}\left( \int\limits_{\ln x}^{e^x} t^3\diff t  \right)\\
&=&\displaystyle \frac{\diff }{\diff x}\left(\int \limits_{\ln x}^{0} t^3\diff t+ \int\limits_{0}^{e^x} t^3\diff t  \right)\\
&=&\displaystyle \frac{\diff }{\diff x}\left(- \int \limits_{0}^{\ln x} t^3\diff t+ \int\limits_{0}^{e^x} t^3\diff t  \right).
\end{array}
\]
The Fundamental Theorem of Calculus part I states that for an arbitrary constant $a$,  $\displaystyle \frac{\diff }{\diff u} \left(\int_{a}^{u}g(t)\diff t \right)=g(u) $ (for a continuous $g$). We use this two compute the two derivatives:
\[
\begin{array}{rcll|l}
\displaystyle \frac{\diff }{\diff x}\left( \int \limits_{0}^{\ln x} t^3\diff t\right)&=&\displaystyle \frac{\diff }{\diff x}\left(\int \limits_{0}^{u} t^3\diff t\right)&&\text{Set } u=\ln x\\
&=&\displaystyle  u^3 \cdot \frac{\diff u}{\diff x}\\
&=&\displaystyle \frac{(\ln x)^3}{x}\\
\displaystyle \frac{\diff }{\diff x}\left( \int \limits_{0}^{e^x} t^3\diff t\right)&= & \displaystyle \frac{\diff }{\diff x}\left( \int \limits_{0}^{w} t^3\diff t\right)&&\text{Set }w=e^{x}\\
&=&\displaystyle w^3\cdot \frac{\diff w}{\diff x}\\
&=&\displaystyle e^{3x}e^{x}=e^{4x}
\end{array}
\]
Finally, we combine the above computations to a single answer. 
\[
f'(x)=e^{4x}-\frac{(\ln x)^3}{x}.
\]
}


\solution{\ref{problemd/dxint_1^x(sqrt(t)-t^(1/3))dt}

\[\begin{array}{rcll|l}
\displaystyle \frac{\diff }{\diff x}\int_1^x\left(\sqrt{t}-\sqrt[3]{t }\right)\diff t&=&\sqrt{x}-\sqrt[3]{x}&&\text{FTC part I} \\
\end{array}
\]
}
\solution{\ref{problemd/dxint_1^(1/(x+1))sin(t^2)dt}
\[
\begin{array}{rcll|l}
\displaystyle \frac{\diff }{\diff x}\int_{1}^{\frac{1}{x+1}} \sin \left(t^2\right)\diff t&=&\displaystyle \frac{\diff}{\diff x}\int_1^{u}\sin (t^2)\diff t   &&u=\frac{1}{x+1}, \text{ use FTC part I, chain rule} \\
&=&\displaystyle \sin\left(u^2\right)\frac{\diff u}{\diff x}\\
&=&\displaystyle \sin\left(\frac{1}{(x+1)^2} \right)\frac{\diff }{\diff x}\left(\frac{1}{x+1} \right)\\
&=&\displaystyle \sin\left(\frac{1}{(x+1)^2} \right)\left(-\frac{1}{(x+1)^2} \right)\\
&=&\displaystyle -\frac{1}{(x+1)^2}\sin\left(\frac{1}{(x+1)^2} \right)\\
\end{array}
\]
}
\solution{\ref{problemd/dxint_1^(1/(1+x))cos(t^2)dt}
\[
\begin{array}{rcll|l}
\displaystyle \frac{\diff }{\diff x}\left(\int_{1}^{\frac{1}{x+1}}\cos\left(t^2\right)\diff t\right)&=&\displaystyle   \frac{\diff }{\diff x}\left(\int_{1}^{u}\cos\left(t^2\right)\diff t\right)&&\text{Set }\frac{1}{x+1}=u\\
&=&\displaystyle   \cos\left(u^2\right)\frac{\diff }{\diff x}(u) &&\text{FTC part I and Chain Rule}\\
&=&\displaystyle   -\frac{1}{(x+1)^2}\cos\left(\frac{1}{(x+1)^2}\right) \\
\end{array}
\]
}


\begin{problem}
\begin{enumerate}[ref={\fcProblemRef}]
% Area problems
\item 
\label{problemAreaBetweeny=2x^2,y=4+x^2} Find the area of the region bounded by the curves $y = 2x^2$ and $y = 4 + x^2$.

\answer{$\frac{32}{3}$}
\item \label{problemAreaBetweeny=2-x,x=4-y^2} Find the area of the region bounded by the curves $x = 4 - y^2$ and $y = 2 - x$.

\answer{$\frac92$}
\item \label{problemAreaBetweeny=x^2} Find the area of the region bounded by the curves $y=x^2$ and $y=2x^2+x-2$.

\answer{$\frac{9}{2}$}


\item \label{problemareabetweeny=x^2andy=2x^2+x-2}
\begin{itemize}
\item Sketch the region bounded by the curves $y=x^2$ and $y=2x^2+x-2$.

\psset{xunit=0.5cm, yunit=0.5cm}
\begin{pspicture}(-3.4,-3.4)(3,5.7)
\fcAxesStandardNoFrame{-3.5}{-3.5}{2.5}{5.5}
\fcGrid[linestyle=dashed, linewidth=0.5, linecolor=gray]{-3}{-3}{5}{8}{1}{1}{}
\rput[t](0.9,-0.2){$1$}
\fcLabels{3.5}{5.5}
%\psplot{-3}{2}{x x mul}
%\psplot{-3}{2}{x x mul 2 mul x -2 add add}
\end{pspicture}

\vskip 2cm


\item Find the area of the region.

\answer{$\frac{9}{2}$}
\end{itemize}
\item \label{problemAreaBetween-x^2+2x-1and-2x^2+3x+1}
~
\begin{itemize}
\item Sketch the region bounded by the curves $y=- x^{2}+2 x-1$ and $y=-2 x^{2}+3 x+1$. Make sure to indicate the points where the curves intersect.

\psset{xunit=0.5cm, yunit=0.5cm}
\begin{pspicture}(-3.5,-8.8)(3.7,5.7)
\fcAxesStandard{-3.5}{-8.4}{3.5}{5.5}
\fcGrid[linestyle=dashed, linewidth=0.5, linecolor=gray]{-2}{-8}{5}{13}{1}{1}{}
\rput[t](0.9,-0.2){$1$}
\fcLabels{3.5}{5.5}
%\psplot[linecolor=\fcColorGraph]{-1.3}{2.7}{x x -1 mul mul 2 x mul -1 add add}
%\psplot[linecolor=\fcColorGraph]{-1.3}{2.7}{x x -2 mul mul 3 x mul 1 add add}
\end{pspicture}
\item Find the area of the region.
\end{itemize}
\end{enumerate}

\end{problem}
\solution{\noindent \ref{problemAreaBetweeny=2-x,x=4-y^2}.
$x=4-y^2$ is a parabola (here we consider $x$ as a function of $y$). $y=-x+2$ implies that $x=2-y$ and so the two curves intersect when
\[
\begin{array}{rcl}
4-y^2&=&2-y\\
-y^2+y+2&=&0\\
-(y+1)(y-2)&=&0\\
y&=& -1\text{~or~}2\quad \quad .
\end{array}
\]
As $x=2-y$, this implies that $x=0$ when $y=2$ and $x=3$ when $y=-1$, or in other words the points of intersection are $(0,2)$ and $(3, -1)$. Therefore we the region is the one plotted below. Integrating with respect to $y$, we get that the area is
\[
\begin{array}{rcl}
A&=&\displaystyle \int\limits_{-1}^{2} \left|4-x^2-(-x+2) \right| \diff y = \int\limits_{-1}^2 \left(-y^2+y+2\right)\diff y \\
&=& \displaystyle \left[- \frac{y^3}3 +\frac{y^2}{2}+ 2y\right]_{-1}^2
=-\frac{8}{3}+2+4 -\left(-\frac{(-1)^3}{3} +\frac{ (-1)^2}{2}-2 \right)\\
&=&\displaystyle \frac{9}{2}\quad .
\end{array}
\]
\psset{xunit=0.5cm, yunit=0.5cm}
\begin{pspicture}(-3.500000, -5)(4.500000,5.5)
\psframe*[linecolor=white](-3.500000,-5)(4.500000,5)
\tiny
\pscustom*[linecolor=cyan]{
\psplot[linecolor=\fcColorGraph, plotpoints=1000]{0}{4}{4 x -1 mul add 0.5 exp }
\psplot[linecolor=\fcColorGraph, plotpoints=1000]{4}{3}{4 x -1 mul add 0.5 exp -1 mul }
}
\rput(-1.5,5){$y=- x+2$}
\psplot[linecolor=\fcColorGraph, plotpoints=1000]{-3.000000}{4.000000}{2 x -1 mul add }
%Function formula: - (- x+4)^{1/2}
\psplot[linecolor=\fcColorGraph, plotpoints=1000]{-3.000000}{4.000000}{4 x -1 mul add 0.5 exp -1 mul }
%Function formula: (- x+4)^{1/2}
\rput(2,2){$x=4-y^2$}
\psplot[linecolor=\fcColorGraph, plotpoints=1000]{-3.000000}{4.000000}{4 x -1 mul add 0.5 exp }
\psaxes[arrows=<->, ticks=none, labels=none](0,0) (-3.000000,-4.5)(4.5,4.5) %Function formula: - x+2
\end{pspicture}
}
\solution{\ref{problemareabetweeny=x^2andy=2x^2+x-2}

\textbf{Region plot.}
\psset{xunit=0.5cm, yunit=0.5cm}
\begin{pspicture}(-3,-3)(3,5.7)
\fcAxesStandard{-3.5}{-3.5}{2.5}{5.5}
\pscustom*[linecolor=\fcColorAreaUnderGraph]{%
\psplot{-2}{1}{x x mul}%
\psplot{1}{-2}{x x mul 2 mul x -2 add add}%
}%
\psplot{-3}{2}{x x mul}
\psplot{-3}{2}{x x mul 2 mul x -2 add add}
\fcGrid[linestyle=dashed, linewidth=0.5, linecolor=gray]{-3}{-3}{5}{8}{1}{1}{}
\rput[t](0.9,-0.2){$1$}
\fcLabels{3.5}{5.5}
\end{pspicture}

The intersection between the two parabolas are found via
\[
\begin{array}{rcl}
x^2&=&2x^2+x-2\\
x^2+x-2&=&0\\
(x-1)(x+2)&=&0\\
x=1&& x=-2\\
y=1&&y=4.
\end{array}
\]

\textbf{Area of the region.} 
\[
\begin{array}{rcll|l}
A&=&\displaystyle\int_{1}^{-2}\left|x^2-(2x^2+x-2) \right|\diff x&&x^2>(2x^2+x-2) \text{ for }x\in [-2,1] \text{ (from plot)}\\
&=&\displaystyle\int_{1}^{-2}\left(x^2-(2x^2+x-2) \right)\diff x\\
&=&\displaystyle \left[-\frac{1}{3} x^{3}-\frac{1}{2} x^{2}+2 x \right]_{-2}^1\\
&=&\displaystyle \frac{9}{2}.
\end{array}
\]
}
\solution{\ref{problemAreaBetween-x^2+2x-1and-2x^2+3x+1}

\textbf{Region plot.}

\psset{xunit=0.5cm, yunit=0.5cm}
\begin{pspicture}(-3.5,-8.8)(3.7,5.7)
\fcAxesStandard{-3.5}{-8.4}{3.5}{5.5}
\pscustom*[linecolor=cyan]{
\psplot{-1}{2}{x x -1 mul mul 2 x mul -1 add add}
\psplot{2}{-1}{x x -2 mul mul 3 x mul 1 add add}
}
\fcGrid[linestyle=dashed, linewidth=0.5, linecolor=gray]{-2}{-8}{5}{13}{1}{1}{}
\rput[t](0.9,-0.2){$1$}
\fcLabels{3.5}{7.5}
\psplot[linecolor=\fcColorGraph]{-1.3}{2.7}{x x -1 mul mul 2 x mul -1 add add}
\psplot[linecolor=\fcColorGraph]{-1.3}{2.7}{x x -2 mul mul 3 x mul 1 add add}
\end{pspicture}

The intersections between the two parabolas are found via
\[
\begin{array}{rcl}
-2x^2+3x+1&=&-x^2+2x-1\\
-x^2+x+2&=&0\\
-(x+1)(x-2)&=&0\\
x=-1&\text{or}& x=2\\
y=-4&&y=-1.
\end{array}
\]

\textbf{Area of the region.} 
\[
\begin{array}{rcll|l}
A&=&\displaystyle\int_{-1}^{2}\left|-2x^2+3x+1-(-x^2+2x-1) \right|\diff x&& \begin{array}{l} -2x^2+3x+1>-x^2+2x-1 \\ \text{ for }x\in [-1,2] \text{ (from plot)}\end{array}\\
&=&\displaystyle\int_{-1}^{2}\left(-2x^2+3x+1-(-x^2+2x-1) \right)\diff x\\
&=&\displaystyle\int_{-1}^{2}\left(-x^2+x+2 \right)\diff x\\
&=&\displaystyle \left[-\frac{1}{3} x^{3}+\frac{1}{2} x^{2}+2 x \right]_{-1}^2\\
&=&\displaystyle \left(-\frac{1}{3} 2^{3}+\frac{1}{2} 2^{2}+2 \cdot 2 \right)-\left( -\frac{1}{3} (-1)^{3}+\frac{1}{2} (-1)^{2}+2 (-1) \right)\\
&=&\displaystyle \frac{9}{2}.
\end{array}
\]
}




\begin{problem}
\begin{enumerate}[ref={\fcProblemRef}]
% Volume problems
\item 
\label{problemVolumeRegionBoundedByy=2x^2-x+1,y=x^2+1rotatedAroundx=0} Consider the region bounded by the curves $y = 2x^2-x+1$ and $y =x^2+1$. What is the volume of the solid obtained by rotating this region about the line $x = 0$?

\answer{$\frac{2}{5}\pi$.} 
\item Consider the region bounded by the curves $y = 1-x^2$ and $y =0$. What is the volume of the solid obtained by
rotating this region about the line $y = 0$?

\answer{$\frac{16 \pi}{15}$}
 
\item Consider the region bounded by the curves $y = x^2$ and $x = y^2$. What is the volume of the solid obtained by
rotating this region about the line $x = 2$?

\answer{ $\frac{31 \pi}{30}$}
\item \label{problemVolumeAreay=-x^2+2andy=0rotatedAroundy=0andy=-3}
Set up \textsc{but do not evaluate} an integral to calculate the volume of the solid obtained by rotating the region bounded by $y=-x^2+2$ and $y=0$ about the given line. 

\begin{itemize}
\item The $x$ axis.
\item The line $y=-3$.
\end{itemize}

\item \label{problemVolumeRevolution-x^2+1aroundy=0andy=-4}
Set up \textsc{but do not evaluate} an integral to calculate the volume of the solid obtained by rotating the region bounded by $y=-x^2+1$ and $y=0$ about the given line. 
\begin{itemize}
\item The $x$ axis.
\item The line $y=-4$.
\end{itemize}


\end{enumerate}

\end{problem}
\solution{\ref{problemVolumeRegionBoundedByy=2x^2-x+1,y=x^2+1rotatedAroundx=0}
First, plot $y=2x^2-x+1$ and $y=x^2+1$. 

\psset{xunit=2cm, yunit=2cm}
\begin{pspicture}(-0.5,-0.5)(1.3,3)
\tiny
\fcAxesStandard{-1}{-0.4}{1.3}{3}
\pscustom*[linecolor=\fcColorAreaUnderGraph]{
\psplot[linecolor=\fcColorGraph]{0}{1}{x x 2 mul mul x sub 1 add}
\psplot[linecolor=\fcColorGraph]{1}{0}{x x mul 1 add}
}
\psplot[linecolor=\fcColorGraph]{-0.5}{1.2}{x x 2 mul mul x sub 1 add}
\psplot[linecolor=\fcColorGraph]{-0.5}{1.2}{x x mul 1 add}
\psline[](! 0.7 11 8 div)(! 0.8 11 8 div)
\psline[](! 0.7 25 16 div)(! 0.8 25 16 div)
\psline[](0.75, 0)(! 0.75 25 16 div)
\rput[br](! 0.75 25 16 div){$r_{outer}~$}
\rput[tl](! 0.8 11 8 div 0.05 sub){$~r_{inner}$}
\psline[linewidth=2pt, linecolor=blue](0, 0)(1,0)
\psline[linestyle=dashed](1,2)(1,0)
\fcFullDot[linecolor=blue]{0}{0}
\fcFullDot[linecolor=blue]{1}{0}
\end{pspicture}
\psset{xunit=2cm, yunit=2cm}
\begin{pspicture}(-1,-2.3)(3,3)
\renewcommand{\fcScreenStyle}{x}
\renewcommand{\fcScreen}{[-0.2 -0.2 -1] 0}
\fcStartIIIdScene
\fcAxesIIIdInScene{2.2}{2.2}{2.2}
\fcSurfaceInScene[iterationsV=7, iterationsU=4, linewidth=0.3, arrows=(none)]{0}{0}{1}{360}{[u 2 u u mul mul u sub 1 add v cos mul 2 u u mul mul u sub 1 add v sin mul]}{}
\fcSurfaceInScene[iterationsV=7, iterationsU=4, arrows=(none), linewidth=0.3, colorUV={1 0.5 0.5}, colorVU={1 0.5 0.5}]{0}{0}{1}{360}{[u u u mul 1 add v cos mul u u mul 1 add v sin mul]}{}
\fcSurfaceInScene[iterationsV=1, iterationsU=3, colorUV=cyan, colorVU=cyan, forceForeground=true]{ 0 }{0}{1}{1}{[u u u 2 mul mul u sub 1 add v mul u u mul 1 add 1 v sub mul add 0]}{}
\fcLineIIIdInScene[linewidth=2, linecolor=blue]{[0 0 0]}{[1 0 0]}
\fcFinishIIIdScene[true]
\fcDotIIId[linecolor=blue]{[0 0 0]}
\fcDotIIId[linecolor=blue]{[1 0 0]}
\end{pspicture}
\psset{xunit=2cm, yunit=2cm}
\begin{pspicture}(-1,-2)(3,3)
\renewcommand{\fcScreenStyle}{x}
\renewcommand{\fcScreen}{[-0.2 -0.2 -1] 0}
\fcStartIIIdScene
\fcAxesIIIdInScene{2.2}{2.2}{2.2}
\fcSurfaceInScene[arrows=(none), iterationsV=1, iterationsU=4, colorUV=cyan, colorVU=cyan, linewidth=0.3, forceForeground=true]{ 0 }{0}{1}{1}{[u u u 2 mul mul u sub 1 add v mul u u mul 1 add 1 v sub mul add 0]}{}
\fcSurfaceInScene[arrows=(none), iterationsV=20, iterationsU=1, colorUV=blue, colorVU=blue, linewidth=0.3]{0.75 0.75 mul 2 mul 0.75 sub 1 add}{0}{0.75 0.75 mul 1 add}{360}{[0.75 v cos u mul v sin u mul]}{}
\fcFinishIIIdScene[true]
\fcLineIIId[]{[0.75  0 0]}{[0.75 11 8 div 0]}
\fcPutIIId[br]{[0.75 25 16 div 0]}{$r_{outer}~$}
\fcPutIIId[tl]{[0.75 11 8 div 0.1 sub 0]}{$~r_{inner}$}
\fcLineIIId{[0 0 0]}{[1 0 0]}
\fcDotIIId[linecolor=blue]{[0 0 0]}
\fcDotIIId[linecolor=blue]{[1 0 0]}

\end{pspicture}

\noindent The two curves intersect when 
\[ 
\begin{array}{rcl}
2x^2-x+1&=&x^2+1\\
x^2-x&=&0\\
x(x-1)&=&0\\
x=0 &\text{or}& x= 1.
\end{array}
\]
Therefore the two points of intersection have $x$-coordinates between $x=0$ and $x=1$. Therefore we need we need to integrate the volumes of washers with inner radii $r_{inner}=2x^2-x+1 $, outer radii $r_{outer}=x^2+1$ and infinitesimal heights $\diff x$. The volume of an individual infinitesimal washer is then $ \pi(r^2_{outer}- r^2_{inner})\diff x$
\[
\begin{array}{rcl}
V&=&\displaystyle \int_{0}^1\pi\left(\left(x^2+1\right)^2- \left(2x^2-x+1\right)^2\right)\diff x\\
&=&\displaystyle \pi\int_{0}^1\left(-3 x^{4}+4 x^{3}-3 x^{2}+2 x \right)\diff x\\
&=&\displaystyle \pi\left[-\frac{3}{5} x^{5}+x^{4}- x^{3}+x^{2} \right]_0^1 \\
&=& \displaystyle \frac{2}{5} \pi.
\end{array}
\]
}
\solution{\ref{problemVolumeAreay=-x^2+2andy=0rotatedAroundy=0andy=-3}
First, we plot the 2d region. The two curves intersect when $-x^2+2=0$, i.e., when $x=\pm \sqrt{2}$


\hfil \hfil \begin{pspicture}(-6.2,-3.2)(6.2,3.2)\tiny
\pscustom*[linecolor=\fcColorAreaUnderGraph]{
\psplot[linecolor=\fcColorGraph]{2 sqrt -1 mul}{2 sqrt}{x x mul -1 mul 2 add}
}
\newcommand{\theFuN}{x x mul -1 mul 2 add\space}%
\psplot[linecolor=\fcColorGraph]{-2}{2}{x x mul -1 mul 2 add}
\psline[linecolor=\fcColorGraph](-2,0)(2,0)
\fcAxesStandardNoFrame{-2}{-3.1}{2}{3}
\rput[t](! 2 sqrt -0.1){$\sqrt{2}$}
\rput[t](! 2 sqrt -1 mul -0.1){$-\sqrt{2}$}
\psline[arrows=<->](1, 0)(! 1 1 dict begin /x 1 def \theFuN end)
\rput[l](1.1,0.3){cross-section rad., $y=0$}
\psline[arrows=<->](-1, -3)(! -1 1 dict begin /x -1 def \theFuN end)
\rput[r](-1.1,-1){cross-section rad., $y=-3$}
\psline[linecolor=green](-2,-3)(2,-3)
\end{pspicture}

\textbf{Rotation about $y=0$. }

Unless explicitly stated in the problem, a 3d plot of the solid is not required in the solution. Nevertheless generating such a plot helps to understand the problem. 

To generate a 3d plot of the solid, we draw the circular cross-sections of the solid of revolution. By hand, this can be done roughly by drawing ovals (circles look like ovals when observed at an angle) centered at the axis about which we revolve the 2d-region. We include a computer-generated plot below; the plot's precision is above what is expected on an exam.

\hfil \hfil \begin{pspicture}(-3,-3)(4.2,3.2)%
\newcommand{\theFun}{u u mul -1 mul 2 add\space}%
\renewcommand{\fcScreenStyle}{x}
\renewcommand{\fcScreen}{[-1 -0.2 -0.75] -1}
\fcStartIIIdScene%
\fcAxesIIIdFullInScene{-3}{-3}{-3}{3}{3}{3}%
\fcSurfaceInScene[arrows=(none), iterationsV=15, iterationsU=8, colorVU={1 0.5 0.5}]{2 sqrt -1 mul 0.001 add}{0}{2 sqrt -0.001 add}{360}{[u v cos \theFun mul v sin \theFun mul]}{}%
\fcSurfaceInScene[arrows=(none), iterationsV=4, iterationsU=3, colorUV={0.3 0.7 1}, forceForeground=true]{2 sqrt -1 mul }{0}{2 sqrt}{1}{[u \theFun v mul  0]}{}%
\fcFinishIIIdScene[true]%
\fcPutIIId{[3 0 0]}{$x$}
\fcPutIIId{[0 3 0]}{$y$}
\fcPutIIId{[0 0 3]}{$z$}
\end{pspicture}

The volume of a solid (and in particular, of a solid of revolution) is computed by integrating the area $A(x)=\pi(\text{radius cross-section})= \pi (-x^2+2)^2 $ of the cross-section of the solid. Therefore the volume $V$ equals
\[
\begin{array}{rcll|l}
V&=&\displaystyle \int_{a}^bA(x)\diff x\\
&=&\displaystyle\int_{-\sqrt{2}}^{\sqrt{2}}\pi (-x^2+2)^2  \diff x\\
&=&\displaystyle\pi\left[\frac{1}{5} x^{5}-\frac{4}{3} x^{3}+4 x\right]_{-\sqrt{2}}^{\sqrt 2}&&\text{step not required by problem}\\
&=&\displaystyle \pi \frac{64}{15}\sqrt{2}&&\text{step not required by problem.}
\end{array}
\]


\textbf{Rotation about $y=-3$. } The cross-section of this solid of revolution is a washer with inner radius $ 3$ and outer radius $-x^2+2-(-3)=5-x^2$. Therefore the area of the cross-section is $\pi (5-x^2)^2-\pi 3^2$ and the volume is computed via

\[
\begin{array}{rcll|l}
V&=&\displaystyle \int_{a}^bA(x)\diff x\\
&=&\displaystyle\int_{-\sqrt{2}}^{\sqrt{2}} \pi \left( (5-x^2)^2- 3^2\right)  \diff x\\
&=&\displaystyle\pi\left[\frac{1}{5} x^{5}-\frac{10}{3} x^{3}+16 x  \right]_{-\sqrt{2}}^{\sqrt 2}&&\text{step not required by problem}\\
&=&\displaystyle \pi  \frac{304}{15}\sqrt{2}&&\text{step not required by problem.}
\end{array}
\]
\hfil \hfil 
\begin{pspicture}(-4,-9)(5,4.2)%
\newcommand{\theFuN}{u u mul -1 mul 2 add\space}%
\renewcommand{\fcScreenStyle}{x}
\fcStartIIIdScene%
\fcAxesIIIdFullInScene{-3}{-9}{-4}{3}{3}{4}%
\renewcommand{\fcScreen}{[-1 -0.2 -0.75] -1}
\fcSurfaceInScene[arrows=(none), iterationsV=15, iterationsU=8, colorVU={0.7 0.2 0.2}, colorUV={0.7 0.2 0.2}]{2 sqrt -1 mul }{0}{2 sqrt -0.001 add}{360}{[u v cos 3 mul -3 add v sin 3 mul]}{}%
\fcSurfaceInScene[arrows=(none), iterationsV=15, iterationsU=8, linecolor=black,colorUV={1 0.5 0.5}, colorVU={1 0.5 0.5}]{2 sqrt -1 mul 0.01 add}{0}{2 sqrt -0.01 add}{360}{[u v cos \theFuN 3 add mul -3 add v sin \theFuN 3 add mul]}{}%
\fcSurfaceInScene[arrows=(none), iterationsV=1, iterationsU=8, colorUV={0.3 0.7 1}, forceForeground=true]{2 sqrt -1 mul }{0}{2 sqrt}{1}{[u \theFuN v mul  0]}{}%
\fcLineIIIdInScene[linecolor=green, linewidth=2]{[-6 -3 0]}{[6 -3 0]}
\fcCurveIIIdInScene[linecolor=red, arrows=(none), linewidth=2]{2 sqrt -1 mul}{2 sqrt}{[1 dict begin /u t def u \theFuN 0 end]}
\fcFinishIIIdScene[true]%
\fcPutIIId{[3 0 0]}{$x$}
\fcPutIIId{[0 3 0]}{$y$}
\fcPutIIId{[0 0 3]}{$z$}
\end{pspicture}
}
\solution{\ref{problemVolumeRevolution-x^2+1aroundy=0andy=-4}

First, we plot the 2d region. The two curves intersect when $-x^2+1=0$, i.e., when $x=\pm 1$.


\hfil \hfil \begin{pspicture}(-6.2,-4.2)(6.2,3.2)\tiny
\pscustom*[linecolor=\fcColorAreaUnderGraph]{
\psplot[linecolor=\fcColorGraph]{-1}{1}{x x mul -1 mul 1 add}
}
\newcommand{\theFuN}{x x mul -1 mul 1 add\space}%
\psplot[linecolor=\fcColorGraph]{-2}{2}{x x mul -1 mul 1 add}
\psline[linecolor=\fcColorGraph](-2,0)(2,0)
\fcAxesStandardNoFrame{-2}{-3.1}{2}{3}
\rput[t](! 1  -0.1){$1$}
\rput[t](! -1 -0.1){$-1$}
\psline[arrows=<->](0.5, 0)(! 0.5 1 dict begin /x 0.5 def \theFuN end)
\rput[l](0.6,0.3){cross-section rad., $y=0$}
\psline[arrows=<->](-0.5, -4)(! -0.5 1 dict begin /x -0.5 def \theFuN end)
\rput[r](-0.6,-2){cross-section rad., $y=-4$}
\rput[b](-2,-4){axis $y=-4$}
\psline[linecolor=green](-2,-4)(2,-4)
\end{pspicture}

\textbf{Rotation about $y=0$. }

\hfil \hfil \begin{pspicture}(-3,-3)(4.2,3.2)%
\newcommand{\theFun}{u u mul -1 mul 1 add\space}%
\renewcommand{\fcScreenStyle}{x}
\renewcommand{\fcScreen}{[-1 -1 -4] -1}
\fcStartIIIdScene%
\fcAxesIIIdFullInScene{-2}{-2}{-6}{2}{2}{6}%
\fcSurfaceInScene[arrows=(none), iterationsV=15, iterationsU=8, colorVU={1 0.5 0.5}]{-0.99}{0}{0.99}{360}{[u v cos \theFun mul v sin \theFun mul]}{}%
\fcSurfaceInScene[arrows=(none), iterationsV=4, iterationsU=3, colorUV={0.3 0.7 1}, forceForeground=true]{-1}{0}{1}{1}{[u \theFun v mul  0]}{}%
\fcFinishIIIdScene[true]%
\fcPutIIId{[3 0 0]}{$x$}
\fcPutIIId{[0 3 0]}{$y$}
\fcPutIIId{[0 0 3]}{$z$}
\end{pspicture}

The volume of a solid (and in particular, of a solid of revolution) is computed by integrating the area $A(x)=\pi(\text{radius cross-section})^2= \pi (-x^2+1)^2 $ of the cross-section of the solid. Therefore the volume $V$ equals
\[
\begin{array}{rcll|l}
V&=&\displaystyle \int_{a}^bA(x)\diff x\\
&=&\displaystyle\int_{-1}^{1}\pi (-x^2+1)^2  \diff x\\
&=&\displaystyle\pi\left[\frac{1}{5} x^{5}-\frac{2}{3} x^{3}+x\right]_{-1}^{1}&&\text{step not required by problem}\\
&=&\displaystyle \pi \frac{16}{15}&&\text{step not required by problem.}
\end{array}
\]


\textbf{Rotation about $y=-4$. } The cross-section of this solid of revolution is a washer with inner radius $ 4$ and outer radius $-x^2+1-(-4)=5-x^2$. Therefore the area of the cross-section is $\pi (5-x^2)^2-\pi 4^2$ and the volume is computed via

\[
\begin{array}{rcll|l}
V&=&\displaystyle \int_{a}^bA(x)\diff x\\
&=&\displaystyle\int_{-1}^{1} \pi \left( (5-x^2)^2- 4^2\right)  \diff x\\
&=&\displaystyle\pi\left[\frac{1}{5} x^{5}-\frac{10}{3} x^{3}+9 x   \right]_{-1}^{1}&&\text{step not required by problem}\\
&=&\displaystyle \frac{176}{15} \pi &&\text{step not required by problem.}
\end{array}
\]
\hfil \hfil 
\begin{pspicture}(-4,-9)(5,4.2)%
\newcommand{\theFuN}{u u mul -1 mul 1 add\space}%
\renewcommand{\fcScreenStyle}{x}
\fcStartIIIdScene%
\fcAxesIIIdFullInScene{-3}{-9}{-4}{3}{3}{4}%
\renewcommand{\fcScreen}{[-1 -1 -4] -1}%
\fcSurfaceInScene[arrows=(none), iterationsV=15, iterationsU=8, colorVU={0.7 0.2 0.2}, colorUV={0.7 0.2 0.2}]{1 -1 mul }{0}{1 -0.001 add}{360}{[u v cos 4 mul -4 add v sin 4 mul]}{}%
\fcSurfaceInScene[arrows=(none), iterationsV=15, iterationsU=8, linecolor=black,colorUV={1 0.5 0.5}, colorVU={1 0.5 0.5}]{-1 0.01 add}{0}{1 -0.01 add}{360}{[u v cos \theFuN 4 add mul -4 add v sin \theFuN 4 add mul]}{}%
\fcSurfaceInScene[arrows=(none), iterationsV=1, iterationsU=8, colorUV={0.3 0.7 1}, forceForeground=true]{-1}{0}{1}{1}{[u \theFuN v mul  0]}{}%
\fcLineIIIdInScene[linecolor=green, linewidth=2]{[-6 -4 0]}{[6 -4 0]}
\fcCurveIIIdInScene[linecolor=red, arrows=(none), linewidth=2]{-1}{1}{[1 dict begin /u t def u \theFuN 0 end]}
\fcFinishIIIdScene[true]%
\fcPutIIId{[3 0 0]}{$x$}
\fcPutIIId{[0 3 0]}{$y$}
\fcPutIIId{[0 0 3]}{$z$}
\end{pspicture}
}



%\vskip 18cm
%\hfill \begin{tabular}{c|c|c|c|c|c|c||c}
%Problem&1 &2&3&4&5&6& $\sum$\\ \hline
%Score&&&&&&&\\ \hline
%Max&17&17&17&17&17&17&102
%\end{tabular} 



\end{document}