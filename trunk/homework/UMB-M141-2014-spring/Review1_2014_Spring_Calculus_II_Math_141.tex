\documentclass{article}
\usepackage{../homework-problems-UMB}

\toggletrue{solutions}
\togglefalse{solutions}
\toggletrue{answers}

\begin{document}
\begin{center}
\Large
Review problems Test 2\\ Math 141 Calculus II \\ \normalsize Instructor: Todor Milev
\end{center}

%\noindent \textbf{Name:} \hfill{~}
%\begin{tabular}{c|c|c|c|c|c|c|c|c||c}
%Problem&1 &2&3&4&5&6&7&8& $\sum$\\ \hline
%Score&&&&&&&&&\\ \hline
%Max&20&20&20&20&20&10&20&20&150
%\end{tabular} 

\noindent The exam is closed books, no calculators will be allowed. The time for work will be 80 minutes. The problems on the exam will be similar to the problems in the review sheet. You will get one problem from each problem type. You will be asked to derive at least one Euler substitution.

\begin{enumerate}
%\item Differentiate
\begin{enumerate}
\item $x^{\sin x}$.
\solution{
$\displaystyle \left(x^{\sin x}\right)'=\left(e^{(\ln x)\sin x}\right)'= e^{(\ln x)\sin x} (\ln x\sin x)'=x^{\sin x}\left( \frac{\sin x}{x}+\ln x\cos x\right) $
}
\answer{$x^{\sin x}\left(\frac{\sin x}{x} +\cos x\ln x\right) $}
\item $x^{\tan x}$.
\answer{$x^{\tan x}\left(\frac{\tan x}{x}- (\ln x)\sec^2 x \right) $}
\end{enumerate}
%\item Compute the limit. 
\begin{enumerate}
\item $\displaystyle\lim\limits_{x\to\infty} \left(\frac{x-2 }{x} \right)^{2 x}$ 

\answer{$e^{-4}$.}
\item $\displaystyle \lim\limits_{x\to\infty} \left(\frac{x}{ x + 3} \right)^{2x}$ 

\answer{$e^{-6}$}
\end{enumerate}
%\item Rewrite as a rational function of $t$.

\begin{enumerate}
\item $\cot (2\arctan t)$.
\answer{$ \frac{1}{2}\left(\frac{1}{t}-t\right)$.}
\item $\cos (2\arctan t)$.
\answer{$ \frac{1-t^2}{1+t^2}$.}
\item $\sec (2\arctan t)$.
\answer{$\frac{1+t^2}{1-t^2}$}
\end{enumerate}
%\item Evaluate the indefinite integral. Illustrate the steps of your solution.
\begin{enumerate}
\item $\displaystyle \int x\sin (-2x)\diff x$.
\item $\displaystyle \int x^2\cos (3x)\diff x$.
\item $\displaystyle \int x e^{-x}\diff x$.
\end{enumerate}

%\item Integrate. Illustrate the steps of your solution.
\begin{multicols}{2}
\begin{enumerate}[ref={\fcProblemRef}]
\item $\displaystyle \int \frac{1}{x+1}\diff {}x$

\answer{$\ln |x+1|+C$}
\item $\displaystyle \int \frac{x-1}{x+1}\diff {}x$

\answer{$x-2\ln |x+1|+C$}
\item $\displaystyle \int \frac{ 1}{(x+1)^2}\diff {}x$

\answer{$-\frac{1}{x+1}+C$}
\item $\displaystyle \int \frac{x}{(x+1)^2}\diff {}x$

\answer{$\ln |x+1|+\frac{1 }{x+ 1 }+C$}
\item $\displaystyle \int \frac{ 1}{ (2x+3)^2}\diff {}x$

\answer{$-\frac{1}{2(2x+3)}+C$ }

\item $\displaystyle
\int \frac{x}{ 2x^2+3}\diff x
$

\answer{$\frac{1}{4}\ln \left(2 x^2+ 3\right)+C$ }
\item 
$\displaystyle
\int \frac{1}{ 2x^2+3}\diff x
$

\answer{$\frac{\sqrt{6}}{6} \Arctan \left(\sqrt{ \frac{ 2}{ 3} } x \right) $}

\item \label{problemIntegrate x/(2x^2+x+1) dx}
$\displaystyle\int \frac{x }{2x^2+x+1}\diff{}x\quad .
$

\answer{$\frac{1}{4}\ln \left(x^2+ \frac{1}{2}x+ \frac{ 1 }{2} \right) - \frac{\sqrt{7}}{14} \Arctan\left( \frac{ 4x+1}{\sqrt{7}} \right) +C$}

\item $\displaystyle
\int \frac{x}{ 2x^2+x+3}\diff x
$

\answer{ $\frac{1}{4}\ln \left( 2 x^{2}+x+3 \right)- \frac{1}{ 2 \sqrt{23}} \arctan \left( \frac{ 4x +1}{ \sqrt{23} } \right) +C$}

\item $\displaystyle
\int \frac{x}{ x^2-x+3}\diff x
$

\answer{$ \frac{1}{2} \ln{} \left| x^{2}- x+3\right| + \frac{ \sqrt{11}}{11} \Arctan{} \left( \frac{ x - \frac{ 1}{2} }{ \frac{\sqrt{11}}{2}} \right) + C $}

\item \label{problemint1/(x^2+1)^2dx}  $\displaystyle
\int \frac{1}{ \left(x^2+1\right)^2}\diff x
$

\answer{$ \frac{1}{2} x \left(x^{2}+1\right)^{ -1} + \frac{ 1}{2} \Arctan{}\left( x\right) +C$}
\item \label{problemint1/(x^2+x+1)^2dx} $\displaystyle
\int \frac{1}{ \left(x^2+x +1 \right)^2 } \diff x
$

\answer{$ \frac{2}{3} x \left(x^{2}+x+1\right)^{ -1} + \frac{ 1}{3} \left(x^{2}+x+ 1\right)^{ -1} +\frac{4 }{ 9} \sqrt{3} \Arctan{} \left( \frac{2}{3} \sqrt{3} x +\frac{\sqrt{3}}{3}\right) $}

\item $\displaystyle \int \frac{1}{ \left(x^2+ 1 \right)^3 }\diff x
$

\answer{$\frac{3}{8} x \left(x^{2}+ 1\right)^{ -1} + \frac{1}{4} x \left(x^{2}+1\right)^{-2}+\frac{3}{8} \Arctan{}\left( x\right)+C$}
\end{enumerate}
\end{multicols}

%\item Integrate.
\begin{enumerate}
\item $\displaystyle\int \frac{x^3}{x^2+2x-3}\diff x$.
\answer{$\frac{1}{4} \ln{}|x-1|+\frac{27}{4} \ln{}|x+3|+\frac{1}{2} x^{2}-2 x$}
\item $\displaystyle\int \frac{x^3}{x^2+3x-4}\diff x$.
\answer{$\frac{1}{2}x^2-3x+\frac{64}{5}\ln|x+4| + \frac{1}{5}\ln|x-1| + C$}

\end{enumerate}
%\item \begin{enumerate}
\item Express $x, dx $ and $\sqrt{x^2+1}$ via $t$ and $\diff t$ for the Euler substitution $x=\cot(2\arctan t)$.
\item Express $x, dx $ and $\sqrt{1-x^2}$ via $t$ and $\diff t$ for the Euler substitution $x=\cos(2\arctan t)$.
\item Express $x, dx $ and $\sqrt{x^2-1}$ via $t$ and $\diff t$ for the Euler substitution $x=\sec(2\arctan t)$.
\end{enumerate}
%\item \begin{enumerate} 
%\item Integrate
\begin{enumerate}
\item 
\[
\int \sqrt{x^2+x+1}\diff x
\]
\item 
\[
\int \sqrt{x^2+x+2}\diff x
\]
\end{enumerate}

%\item Integrate
\begin{enumerate}
\item 
\[
\int \sqrt{1-x-x^2}\diff x
\]
\item 
\[
\int \sqrt{2-x-x^2}\diff x
\]
\end{enumerate}
%\item Integrate
\begin{enumerate}
\item 
\[
\int \sqrt{x^2+x-1}\diff x
\]
\item 
\[
\int \sqrt{x^2+x-2}\diff x
\]
\end{enumerate}

%\end{enumerate}
\item Integrate 
\begin{enumerate}
\item $\int\limits_{1}^3 \frac{x^4 }{(x+1)^2(x+2) } \diff x$
\item $\int\limits_{0}^1 \frac{x^4 }{(x^2+2)(x+2) } \diff x$
\end{enumerate}

\item Find the limit.
\begin{multicols}{2}
\begin{enumerate}
\item $\displaystyle \lim\limits_{x\to 0} \frac{\sin x-x }{\arcsin x-x } $.
\answer{$ -1$}
\item \label{problemLHospital (sin (pi x) ln x )/ (cos pi x +1)}  $\displaystyle \lim\limits_{x\to 1} \frac{\sin \left(\pi x\right)\ln x }{\cos(\pi x)+1 } $.
\answer{$-\frac{2}{\pi}$}
\item \label{problemlim x to 0 (sin x - x)/(arctan x - x)} $\lim\limits_{x\to 0}\frac{\sin x- x}{\Arctan x -x}$.
\answer{$\frac{1}{2}$}

\end{enumerate}
\end{multicols}
\item Determine if the integral is convergent or divergent. If it is convergent, compute the value of the integral.
\begin{multicols}{2}
\begin{enumerate}
\item $\displaystyle \int\limits_{1}^{2} \frac{x}{\sqrt{x^2-1}}\diff x$ 
\answer{$\sqrt{3}$}

\item $\displaystyle \int\limits_{0}^{1} x^2\ln x\diff x$ \label{problemImproperIntegral x^2ln x dx}
\answer{$- \frac{1}{9} $}

\item \label{problemImproperIntegral e^(-sqrt x)/sqrt(x)dx}
$\displaystyle \int\limits_{0}^{\infty} \frac{e^{-\sqrt{x}}}{ \sqrt{x}} \diff x $
\answer{ $2 $} 

\item \label{problemImproperIntegral1/(x ln x)dx} $\displaystyle \int\limits_{100}^{\infty} \frac{1}{x\ln x}\diff x $
\answer{$\infty$- the integral is divergent} 

\item $\displaystyle \int\limits_{0}^{1} \frac{1}{x\ln x}\diff x $
\answer{$-\infty$- the integral is divergent} 

\item \label{problemImproperIntegral(1+e^(-x))/(xlnx)dx}
$\displaystyle \int\limits_{100}^{\infty} \frac{1+e^{-x}}{x\ln x }\diff x $ 
\answer{$\infty$-the integral is divergent}
\end{enumerate}
\end{multicols}
\item ~\begin{enumerate}[ref={\fcProblemRef}]
\item Sketch the curve given in polar coordinates by $r=2\sin \theta $. What kind of a figure is this curve? Find an equation satisfied by the curve in the $(x,y)$-coordinates.
\item Sketch the curve given in polar coordinates by $r=4\cos \theta $. What kind of a figure is this curve? Find an equation satisfied by the curve in the $(x,y)$-coordinates.
\item \label{problemPolarSketchr=2sec(theta)}  Sketch the curve given in polar coordinates by $r=2\sec \theta $. What kind of a figure is this curve? Find an equation satisfied by the curve in the $(x,y)$-coordinates.
\answer{the curve is the line $x=2$}
\item Sketch the curve given in polar coordinates by $r=2\csc \theta $. What kind of a figure is this curve? Find an equation satisfied by the curve in the $(x,y)$-coordinates.
\item \label{problemPolarSketchr=2sec(theta+pi/4)} Sketch the curve given in polar coordinates by $r=2\sec \left(\theta + \frac{\pi}{4} \right) $. What kind of a figure is this curve? Find an equation satisfied by the curve in the $(x,y)$-coordinates.
\answer{the curve is the line $y=x-2\sqrt{2}$}

\item Sketch the curve given in polar coordinates by $r=2\csc\left(\theta +\frac{\pi}{6}\right)$. What kind of a figure is this curve? Find an equation satisfied by the curve in the $(x,y)$-coordinates.

\end{enumerate}


\item 
Find the values of the parameter $t$ for which the curve has horizontal and vertical tangents.
\begin{multicols}{2}
\begin{enumerate}
\item $y=t^2-t+1$, $x=t^2+t-1$

\psset{xunit=0.25cm, yunit=0.25cm}
\begin{pspicture}(-0.9, -1.65)(13.4,11.416228)
\tiny
\fcAxesStandard{-0.65}{-1.4}{13.15}{11.066228}

%Calculator input: plotCurve{}(t^{2}- t+1, t^{2}+t-1, -3, 3)
\parametricplot[linecolor=\fcColorGraph, plotpoints=1000]{-3}{3}{ 1 t -1 mul add t 2 exp add -1 t add t 2 exp add }
\end{pspicture}
\item $x=t^3-t^2-t+1$, $y=t^2-t-1 $.

\psset{xunit=1cm, yunit=1cm}
\begin{pspicture}(-0.9, -1.649998)(3.358221,1.5)
\tiny
\fcAxesStandard{-0.65}{-1.399998}{3.108221}{1.15}

%Calculator input: plotCurve{}(t^{3}- t^{2}- t+1, t^{2}- t-1, -1, 2)
\parametricplot[linecolor=\fcColorGraph, plotpoints=1000]{-1}{2}{ 1 t -1 mul add t 2 exp -1 mul add t 3 exp add -1 t -1 mul add t 2 exp add }
\end{pspicture}
\item $x=\cos (t)$, $y=\sin (3t)$

\psset{xunit=1cm, yunit=1cm}
\begin{pspicture}(-1.4, -1.399999)(1.4,1.499999)
\tiny
\fcAxesStandard{-1.15}{-1.149999}{1.15}{1.149999}

%Calculator input: plotCurve{}(\cos{}t, \sin{}(3 t), - \pi, \pi)
\parametricplot[linecolor=\fcColorGraph, plotpoints=1000]{-3.14159}{3.14159}{t 57.29578 mul cos t 3 mul 57.29578 mul sin }
\end{pspicture}
\item $x=\cos (t)+\sin (t)$ , $y=\sin (t)$.

\psset{xunit=1cm, yunit=1cm}
\begin{pspicture}(-1.814213, -1.399999)(1.81421,1.499999)
\tiny
\fcAxesStandard{-1.564213}{-1.149999}{1.56421}{1.149999}

%Calculator input: plotCurve{}(\sin{}t+\cos{}t, \sin{}t, - \pi, \pi)
\parametricplot[linecolor=\fcColorGraph, plotpoints=1000]{-3.14159}{3.14159}{t 57.29578 mul cos t 57.29578 mul sin add t 57.29578 mul sin }
\end{pspicture}
\end{enumerate}
\end{multicols}


\item Give a geometric definition of the cycloid curve using a circle of radius 1. Using that definition, derive equations for the cycloid curve. Find area locked between one ``arch'' of the cycloid curve and the $x$ axis.





\item \begin{enumerate}
\item \label{problem-Area-swept-by-r=1+sin2theta} The curve given in polar coordinates by $r=1+\sin 2\theta$ is plotted below by computer. Find the area lying outside of this curve and inside of the circle $x^2+y^2=1$.
\psset{xunit=1cm, yunit=1cm}
\begin{pspicture}(-2.016386, -2.016424)(2.016407,2.116335) 
\tiny 
\psaxesStandard{-1.766386}{-1.766424}{1.766407}{1.766335}
%Calculator command: drawPolar{}(\sin{}(2 t)+1, 0, 2 \pi) 
\parametricplot[linecolor=\psColorGraph, plotpoints=1000, algebraic=false]{0}{6.28319}{ 1 t 2 mul 57.29578 mul sin add t 57.29578 mul cos mul 1 t 2 mul 57.29578 mul sin add t 57.29578 mul sin mul }
\end{pspicture} 
\answer{$a=2-\frac{\pi}{4}$}
\item \label{problem-Area-swept-by-r=cos2theta} The curve given in polar coordinates by $r=\cos (2\theta)$ is plotted below by computer. Find the area lying inside the curve and outside of the circle $x^2+y^2=\frac14$.
\psset{xunit=1cm, yunit=1cm}
\begin{pspicture}(-1.399902, -1.399975)(1.4,1.499975) 
\tiny 
\psaxesStandard{-1.149902}{-1.149975}{1.15}{1.149975}
%Calculator command: drawPolar{}(\cos{}(2 t), 0, 2 \pi) 
\parametricplot[linecolor=\psColorGraph, plotpoints=1000, algebraic=false]{0}{6.28319}{t 2 mul 57.29578 mul cos t 57.29578 mul cos mul t 2 mul 57.29578 mul cos t 57.29578 mul sin mul }
\end{pspicture}

%\item The curve given in polar coordinates by $r=1+\cos (3t)$ is plotted below. Find the area \textbf{outside} of the curve and \textbf{inside} the circle $x^2+y^2=\frac14$.

%\begin{pspicture}(-1.611991, -2.182513)(2.4,2.282513) \tiny \psaxesStandard{-1.361991}{-1.932513}{2.15}{1.932513}
%Calculator command: drawPolar{}(\cos{}(3 t)+1, 0, 2 \pi) 
%\parametricplot[linecolor=\psColorGraph, plotpoints=1000, algebraic=false]{0}{6.28319}{ 1 t 3 mul 57.29578 mul cos add t 57.29578 mul cos mul 1 t 3 mul 57.29578 mul cos add t 57.29578 mul sin mul }
%\end{pspicture} 
 
\end{enumerate}


\end{enumerate}
\end{document}





\end{document}