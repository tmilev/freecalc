\documentclass{article}
\usepackage{../homework-problems-UMB}

\toggletrue{solutions}
\toggletrue{answers}
%\togglefalse{solutions}

\renewcommand{\course}{Math 141}
\renewcommand{\Arccos}{\arccos}
\renewcommand{\Arcsin}{\arcsin}
\renewcommand{\Arctan}{\arctan}
\renewcommand{\Arccot}{{\text{arccot }}}


\begin{comment}
\homeworkStart{on Lecture 1 \\Quiz date to be announced in class}{}
\item Differentiate.
\begin{multicols}{2}
\begin{enumerate}
\item $10^{x^3}$. \answer{$3(\ln 10) x^{2} (10)^{x^{3}}$}
\item $2^{\tan x}$. \answer{ $(\ln 2) 2^{\tan x}  \sec^2 x $  }
\item $x^x $. \answer{$x^x(\log{}(x) +1)$}
\item $x^{x^x}$. \answer{$(\ln(x))^{2}  x^{x^{x}+x}+x^{x^{x}+x-1}+(\ln x) x^{x^{x}+x}$}
\item $(\sin x)^{\cos x}$. \answer{$\frac{- \ln(\sin{}x)  (\sin{}x)^{\cos{}x+2} +(\sin{}x)^{\cos{}x} \cos^{2}{}x}{\sin{}x}$}
\item $(\ln x)^{\ln x}$. \answer{$\ln{}(\ln{}(x)) x^{-1} (\ln{}(x))^{\ln{}(x)}+x^{-1} (\ln{}(x))^{\ln{}(x)}$}
\end{enumerate}
\end{multicols}
\item Find the limit.
\begin{multicols}{2}
\begin{enumerate}
\item $\displaystyle \lim\limits_{x\to \infty} \left(1-\frac{2}{x} \right)^x$. \answer{$e^{-2}$}
\item $\displaystyle \lim\limits_{x\to 0} \left(1-x\right)^{\frac{1}{x}}$.
\answer{ $e^{-1}$}
\item $\displaystyle \lim\limits_{x\to \infty} \left(\frac{x}{x-5}\right)^{x}$.
\answer{$e^5$}
\item $\displaystyle \lim\limits_{x\to \infty} \left(\frac{x}{x-2}\right)^{3x+2}$.
\answer{$e^6$}
\end{enumerate}
\end{multicols}
\item \begin{multicols}{2}
\begin{enumerate}
\item A sum is held under a yearly compound interest of 1\%. Make an approximation by hand (no calculators allowed) by what factor will have the money increased after 200 years. Can you do the computation in your head?
\item Decide, without using a calculator, which is more profitable: earning a yearly compound interest of 2\% for 150 years or earning yearly simple interest of 11\% for 150 years?
\end{enumerate}
\end{multicols}
\item $1,000,000$ servers are handling Internet users. Suppose we distribute the computation load as follows. The computation load distributing program directs every new user to a server chosen at random (one server is allowed to process more than one user at a time). Suppose one server has defective hardware and crashes. We are testing the system by simulating $X$ Internet users.
\begin{itemize}
\item What is the chance we catch the defective server using $1$ simulated user?
\item Without using a calculator, estimate the chance we fail to catch the defective server using $1,000,000$ simulated users.
\item Using a calculator, estimate the chance we fail to catch the defective server using $100,000$ simulated users. Write an expression using $e$ which approximates this chance. Evaluate the latter with a calculator. Are the two numbers close? 
\end{itemize}

\textbf{Remark.} While such a simple system architecture would not be practical, it is not to be immediately dismissed as terrible. For example, if we need to handle 2 million users per second, our load distributing mechanism might not be fast enough to keep track of each server's load. On the other hand, an inexpensive modern pc will easily generate 2 million random numbers per second.


\homeworkEnd
\end{comment}
\begin{comment}
\homeworkStart{on Lecture 2 \\Quiz time to be announced in class}{}
\item Let $x\in (0,1)$. Express the following using $x$ and $\sqrt{1-x^2}$.  
\begin{multicols}{2}
\begin{enumerate} [ref={\fcProblemRef}]
\item $\sin(\Arcsin (x))$. 

\answer{$x$}

\pointsii{3} \label{problemsin(2arcsin x)}  $\sin(2\Arcsin (x))$. 

\answer{$2x\sqrt{1-x^2} $}

\item \label{problemsin(3arcsin x)} $\sin(3\Arcsin (x))$. 

\answer{$ -4x^3+3x $}
\item $\sin(\Arccos (x))$. 

\answer{$\sqrt{1-x^2} $}
\item $\sin(2\Arccos (x))$. 

\answer{$2x \sqrt{1-x^2 }$}
\item \label{problemsin(3arccos(x))}  $\sin(3\Arccos (x))$. 

\answer{
\begin{tabular}{l}
$\left(4x^2-1\right)\sqrt{1-x^2}$ \\= $-4\left(\sqrt{1-x^2}\right)^3+3\sqrt{1-x^2} $
\end{tabular}
}

\item $\cos(2\Arcsin (x))$. 

\answer{$ 1-2x^2$}

\item $\cos(3\Arccos (x))$. 

\answer{$4x^3-3x $}

\end{enumerate}
\end{multicols}
\item Rewrite as algebraic expression of $x$ which uses $\sqrt{~}$ radicals (i.e., get rid of the trigonometric and inverse trigonometric expressions). The answer key has not been fully proofread, use with caution.

\begin{multicols}{2}
\begin{enumerate}
\item $\cos^2(\Arctan x)$. \answer{$\frac{1}{1+x^2} $}
\item $-\sin^2(\Arccot x)$. \answer{ $-\frac{1}{1+x^2}$}
\item $\frac{1}{\cos(\Arcsin x)}$. \answer{$\frac{1}{\sqrt{1-x^2}}$}
\item $-\frac{1}{\sin(\Arccos x)}$.\answer{$-\frac{1}{\sqrt{1-x^2}}$}
\end{enumerate}
\end{multicols}

\item Rewrite as a rational function of $t$.

\begin{enumerate}
\item $\cot (2\arctan t)$.
\answer{$ \frac{1}{2}\left(\frac{1}{t}-t\right)$.}
\item $\cos (2\arctan t)$.
\answer{$ \frac{1-t^2}{1+t^2}$.}
\item $\sec (2\arctan t)$.
\answer{$\frac{1+t^2}{1-t^2}$}
\end{enumerate}



\item Compute the derivative (derive the formula).

\begin{multicols}{2}
\begin{enumerate}
\item $(\Arctan x) '$.
\item $(\Arccot x)'$.
\item $(\Arcsin x)'$.
\item $(\Arccos x)'$.
\item Let $\text{arcsec}$ denote the inverse of the secant function. Compute $(\text{arcsec} x)'$.
\end{enumerate}
\end{multicols}

\item \begin{enumerate}[ref={\fcProblemRef}]
\item \label{problemTangentAngleSumLaw}  Let $a+b\neq k\pi $, $a\neq k\pi+\frac{\pi}{2}$ and $b\neq k\pi +\frac{\pi}{2}$ for any $k\in \mathbb Z$ (integers). Prove that
\[
\frac{\tan a + \tan b}{1-\tan a \tan b }= \tan (a+b)\quad .
\]
\item \label{problemArctangentAngleSumLaw} Let $x$ and $y$ be real. Prove that, for $xy\neq 1$, we have
\[\Arctan x+\Arctan y= \Arctan\left( \frac{x+y}{1-xy}\right)
\]
if the left hand side lies between $\left(-\frac{\pi}{2}, \frac{\pi}{2}\right)$.
\end{enumerate}


\homeworkEnd
\end{comment}
\begin{comment}
\homeworkStart{on Lectures 4 \\Will be quizzed: date to be announced}{}
\item Evaluate the indefinite integral. Illustrate the steps of your solutions.
\begin{multicols}{2}
\begin{enumerate}[ref={\fcProblemRef}]
\item \label{problemIntegratex*sin(x)dx} $\displaystyle \int x \sin x \diff x$.

\answer{$ -x \cos x +\sin x +C$}
\item $\displaystyle \int x e^{-x}\diff x$.

\answer{$-(1+x)e^{-x} +C$}
\item \label{problemIntegratex^2e^xdx} $\displaystyle \int x^2 e^x \diff x$.

\answer{$ x^2e^x-2xe^x+2e^x+C$}

\item $\displaystyle \int x\sin (-2x)\diff x$.

\answer{$ \frac{x}{2}\cos (-2x) +\frac{1}{4}\sin (-2x)+C$}

\item $\displaystyle \int x^2\cos (3x)\diff x$.

\answer{$ \frac{x^2}{3}\sin (3x)+\frac{2x}{9}\cos (3x)-\frac{2}{27}\sin (3x)  +C$}

\item \label{problemintx^2e^(-2x)dx} $\displaystyle \int x^2 e^{-2x}\diff x$.

\answer{$-\frac{x^2e^{-2x}}{2}-\frac{xe^{-2x}}{2}- \frac{ e^{-2x}}{4}+C$}

\item $\displaystyle \int x \sin (2x)\diff x$.

\answer{$-\frac{x}{2}\cos (2x)+\frac{1}{4}\sin (2x) +C$}
\item $\displaystyle \int x \cos (3x)\diff x$.

\answer{$\frac{x}{3}\sin (3x)+\frac{1}{9}\cos (3x) +C$}
\item $\displaystyle \int x^2 e^{2x}\diff x$.

\answer{$\frac{x^2}{2}e^{2x}-\frac{x}{2}e^{2x} +\frac{e^{2x}}{4}+C$}
\item $\displaystyle \int x^3 e^x \diff x$.

\answer{$x^3 e^x-3x^2e^x+6x e^x-6e^x +C$}
\end{enumerate}
\end{multicols}
\item Evaluate the indefinite integral. Illustrate the steps of your solutions.
\begin{multicols}{2}
\begin{enumerate}[ref={\fcProblemRef}]
\item $\displaystyle\int x^2\cos (2x) \diff x$.

\answer{$ \frac{1}{2}x^2\sin(2x)+\frac{1}{2}x\cos(2x) - \frac{1}{4} \sin(2x) +C$}
\item 
$\displaystyle\int x^2e^{a x} \diff x$, where $a$ is a constant.

\answer{$ \frac{1}{a} x^2 e^{a x} -\frac{2}{a^2}x e^{a x}+\frac{2}{a^3} e^{a x}+C$}
\item 
$\displaystyle\int x^2e^{-ax}\diff x$, where $a$ is a constant.

\answer{$ -\frac{1}{a} x^2 e^{-a x} -\frac{2}{a^2}x e^{- a x}-\frac{2}{a^3} e^{-a x}+C$}
\item \label{problemintx^2(e^(ax)+e^(-ax))^2/4dx}
$\displaystyle\int x^2\frac{(e^{ax}+e^{-ax})^2}4\diff x$, where $a$ is a constant. 

\answer{$\begin{array}{l} \frac{1}{8}\left(a^{-1} x^{2} e^{2 a x}- a^{-1} x^{2} e^{-2 a x} \right. \\
\left. - a^{-2} x e^{2 a x}- a^{-2} x e^{-2 a x}+\frac{1}{2} a^{-3} e^{2 a x} \right.\\
\left. -\frac{1}{2} a^{-3} e^{-2 a x}+\frac{2}{3} x^{3} \right) +C
\end{array}
$ }
\item \label{problemint(e^(-sqrtx)dx)}
$ \displaystyle 
\int e^{-\sqrt{x}}\diff x\quad .
$

\answer{$-2\sqrt{x}e^{-\sqrt{x}} -2e^{-\sqrt{x}}+C$}
\item 
$\displaystyle\int \cos^2x ~ \diff x$ 

\answer{$\frac{1}{4}\sin(2x) +\frac{x}{2} +C$}
\item 
$\displaystyle \int x\ln |x|  ~ \diff x $

\answer{$ \frac{1}{2}x^2\ln |x| -\frac{x^2}{4} +C$}
\item $\displaystyle\int \frac{x}{1+x^2} \diff x$  (Hint: use substitution rule, don't use integration by parts)

\answer{$\frac{\ln\left(1+x^2\right)}{2}+C$}
\item 
$\displaystyle \int (\Arctan x) \diff x$

\answer{$x\Arctan x -  \frac{\ln\left(1+ x^2\right) }{2}+C$}
\item 
$\displaystyle \int (\Arcsin x) \diff x
$

\answer{$x\Arcsin x+ \sqrt{1-x^2}+C$}
\item $\displaystyle\int \frac{\ln x}{\sqrt{x}}\diff x $

\answer{$ 2\sqrt{x}(\ln x-2)+C$ }
\item \label{problemIntegrateArcsinSquared} $\displaystyle\int (\Arcsin x)^2 \diff x $ \quad \quad (Hint: Try substituting $x=\sin y$)

\answer{$x(\arcsin x)^2+ 2\sqrt{1-x^2}\arcsin x - 2x+C$}

\item $\displaystyle\int \frac{1}{\cos^2 x}\diff x$\quad \quad (Hint: What is the derivative of $\tan x$?)

\answer{$\tan x+C$}
\item $\displaystyle\int (\tan^2 x) \diff x $ \quad \quad (Hint: $\tan^2 x = \frac{1}{\cos^2x }-1$ )


\item \label{problemintxtan^2xdx} $\displaystyle\int x \tan^2 x \diff x $ \quad \quad (Hint: $\tan^2 x \diff x= \diff (F(x))$, where $F(x)$ is the answer from the preceding problem).

\answer{$-\frac{x^2}{2}+x\tan x + \ln |\cos x|+C$}
\item 
$\displaystyle
\int\Arctan \left(\frac{1}x\right)\diff x
$
\item \label{problemintsin(ln x)dx}

$\displaystyle\int \sin (\ln (x)) \diff x $

\answer{$ \frac{x}{2}\left(\sin (\ln x)-\cos (\ln x) \right) +C$}
\item 
$\displaystyle\int \cos (\ln (x)) \diff x $

\answer{$ \frac{x}{2}\left(\cos (\ln x)+\sin (\ln x) \right) +C$}

\item $\displaystyle\int (\ln x)^2 \diff x$.
\item $\displaystyle\int (\ln x)^3 \diff x$.
\item $\displaystyle\int x^2\cos^2x \diff x$ (This problem can be solved directly with integration by parts. An alternative and quicker solution is to use the fact that $\cos x= \frac{ e^{ix} + e^{-ix}}{2}$ and problem \ref{problemintx^2(e^(ax)+e^(-ax))^2/4dx}).
\end{enumerate}
\end{multicols}


\homeworkEnd
\end{comment}

\begin{comment}
\homeworkStart{on Lecture 5 \\Will be quizzed: date to be announced}{}
\item 
Integrate. Illustrate the steps of your solution.
\begin{multicols}{2}
\begin{enumerate}[ref={\fcProblemRef}]
\item $\displaystyle \int \frac{1}{x+1}\diff {}x$

\answer{$\ln |x+1|+C$}
\item $\displaystyle \int \frac{x-1}{x+1}\diff {}x$

\answer{$x-2\ln |x+1|+C$}
\item $\displaystyle \int \frac{ 1}{(x+1)^2}\diff {}x$

\answer{$-\frac{1}{x+1}+C$}
\item $\displaystyle \int \frac{x}{(x+1)^2}\diff {}x$

\answer{$\ln |x+1|+\frac{1 }{x+ 1 }+C$}
\item $\displaystyle \int \frac{ 1}{ (2x+3)^2}\diff {}x$

\answer{$-\frac{1}{2(2x+3)}+C$ }

\item $\displaystyle
\int \frac{x}{ 2x^2+3}\diff x
$

\answer{$\frac{1}{4}\ln \left(2 x^2+ 3\right)+C$ }
\item 
$\displaystyle
\int \frac{1}{ 2x^2+3}\diff x
$

\answer{$\frac{\sqrt{6}}{6} \Arctan \left(\sqrt{ \frac{ 2}{ 3} } x \right) $}

\item \label{problemIntegrate x/(2x^2+x+1) dx}
$\displaystyle\int \frac{x }{2x^2+x+1}\diff{}x\quad .
$

\answer{$\frac{1}{4}\ln \left(x^2+ \frac{1}{2}x+ \frac{ 1 }{2} \right) - \frac{\sqrt{7}}{14} \Arctan\left( \frac{ 4x+1}{\sqrt{7}} \right) +C$}

\item $\displaystyle
\int \frac{x}{ 2x^2+x+3}\diff x
$

\answer{ $\frac{1}{4}\ln \left( 2 x^{2}+x+3 \right)- \frac{1}{ 2 \sqrt{23}} \arctan \left( \frac{ 4x +1}{ \sqrt{23} } \right) +C$}

\item $\displaystyle
\int \frac{x}{ x^2-x+3}\diff x
$

\answer{$ \frac{1}{2} \ln{} \left| x^{2}- x+3\right| + \frac{ \sqrt{11}}{11} \Arctan{} \left( \frac{ x - \frac{ 1}{2} }{ \frac{\sqrt{11}}{2}} \right) + C $}

\item \label{problemint1/(x^2+1)^2dx}  $\displaystyle
\int \frac{1}{ \left(x^2+1\right)^2}\diff x
$

\answer{$ \frac{1}{2} x \left(x^{2}+1\right)^{ -1} + \frac{ 1}{2} \Arctan{}\left( x\right) +C$}
\item \label{problemint1/(x^2+x+1)^2dx} $\displaystyle
\int \frac{1}{ \left(x^2+x +1 \right)^2 } \diff x
$

\answer{$ \frac{2}{3} x \left(x^{2}+x+1\right)^{ -1} + \frac{ 1}{3} \left(x^{2}+x+ 1\right)^{ -1} +\frac{4 }{ 9} \sqrt{3} \Arctan{} \left( \frac{2}{3} \sqrt{3} x +\frac{\sqrt{3}}{3}\right) $}

\item $\displaystyle \int \frac{1}{ \left(x^2+ 1 \right)^3 }\diff x
$

\answer{$\frac{3}{8} x \left(x^{2}+ 1\right)^{ -1} + \frac{1}{4} x \left(x^{2}+1\right)^{-2}+\frac{3}{8} \Arctan{}\left( x\right)+C$}
\end{enumerate}
\end{multicols}

\item 
\label{problemIntegrateBuildingBlockIIaandIIIa} Let $a,b,c,A, B$ be real numbers. Suppose in addition $a\neq 0$ and  $b^2-4ac<0$. Integrate
\[
\int \frac{Ax +B}{ax^2+bx+c}\diff x\quad .
\]

The purpose of this exercise is to produce a formula in form ready for implementation in a computer algebra system.
\solution{\ref{problemIntegrateBuildingBlockIIaandIIIa}. 

\[
\begin{array}{rcll|l}
\displaystyle\int \frac{Ax +B}{ax^2+bx+c}\diff x&=&\displaystyle \int \frac{Ax+B}{a\left( x^2 + 2x \frac{b}{2a} +\frac{c}{a} \right) }\diff x =  \int \frac{Ax+B}{a\left( x^2 + 2x \frac{b}{2a}+\frac{b^2}{4a^2}- \frac{b^2}{4a^2} +\frac{c}{a} \right) }\diff x &&\begin{array}{l} \text{complete square}\\\text{in denominator}\end{array}\\ 
&=&\displaystyle \frac{1}{a}\int \frac{Ax+B}{ \left(x+ \frac{b}{2a} \right)^2 + \frac{4ac-b^2}{4a^2} } \diff x &&\text{Set }D=  \frac{4ac-b^2}{4a^2} \\
&=&\displaystyle \frac{1}{a}\int \frac{A \left( x+ \frac{b}{2a} - \frac{b}{2a}\right) +B}{ \left(x+ \frac{b}{2a} \right)^2 + D } \diff \left(x+\frac{b}{2a}\right) &&\text{Set } u=x+\frac{b}{2a}\\
&=&\displaystyle \frac{1}{a}\int \frac{Au+ B- \frac{Ab}{2a}}{u^2+D} \diff u &&\text{Set }C=B-\frac{Ab}{2a} \\
&=&\displaystyle \frac{1}{a}\left(A\int \frac{u}{u^2+D}\diff u + C\int \frac{1}{u^2+D}\diff u \right) \\
&=&\displaystyle \frac{1}{a} \left(\frac{A}{2}\ln (u^2+D) + \frac{C}{\sqrt{D}}\Arctan\left(\frac{u}{\sqrt{D}}\right)\right) +K\\
&=& \displaystyle \frac{1}{a} \left(\frac{A}{2}\ln \left(x^2+ \frac{b}{a}x+\frac{c}{a} \right) + \frac{C}{\sqrt{D}}\Arctan\left(\frac{x+\frac{b}{2a}}{\sqrt{D}} \right)\right) +K \quad .
\end{array}
\] 
The solution is complete. Question to the student: where do we use $b^2-4ac<0$?
}
\item 

Let $a, b, c, A, B$ be real numbers and let $n>1$ be an integer. Suppose in addition $a\neq 0$ and $b^2-4ac<0$. Let 
\[
J(n)=\int \frac{1}{ \left( x^2+\frac{b}{a}x + \frac{ c}{a}\right)^n}\diff x\quad .
\]
\begin{enumerate}
\item \label{problemIntegrateBuildingBlockIIandIIIbPart1} Express the integral 
\[
\int \frac{Ax +B}{ \left( ax^2 +bx +c\right)^n}\diff x
\]
via $J(n)$.
\item \label{problemIntegrateBuildingBlockIIandIIIbPart2} Express $J(n)$ recursively via $J(n-1)$
\end{enumerate}
The purpose of this exercise is to produce a formula in form ready for implementation in a computer algebra system.

\solution{\ref{problemIntegrateBuildingBlockIIandIIIbPart1}.\[
\begin{array}{rcll|l}
\displaystyle\int \frac{Ax +B}{(ax^2+bx+c)^n}\diff x&=&\displaystyle \int \frac{Ax+B}{a^n\left( x^2 + 2x \frac{b}{2a} +\frac{c}{a} \right)^n }\diff x =  \int \frac{Ax+B}{a^n\left( x^2 + 2x \frac{b}{2a}+\frac{b^2}{4a^2}- \frac{b^2}{4a^2} +\frac{c}{a} \right)^n }\diff x &&\begin{array}{l} \text{complete square}\\\text{in denominator}\end{array}\\ 
&=& \displaystyle \frac{1}{a^n} \int \frac{Ax+B}{ \left( \left( x + \frac{b}{2a} \right)^2 + \frac{ 4ac-b^2}{4a^2} \right)^n } \diff x &&\text{Set }D=  \frac{ 4ac-b^2}{4a^2} \\
&=&\displaystyle \frac{1}{ a^n}\int \frac{A \left( x+ \frac{b}{2a} - \frac{b}{ 2a} \right) +B}{ \left(\left(x+ \frac{b}{2a} \right)^2 + D \right)^n } \diff \left(x + \frac{ b}{2a}\right) && \text{Set } u=x+\frac{b}{2a}\\
&=& \displaystyle \frac{1}{a^n} \int \frac{Au+ B- \frac{Ab}{2a} }{ \left(u^2+D\right)^n} \diff u &&\text{Set }C = B - \frac{A b}{2 a} \\
&=&\displaystyle \frac{1 }{a^n} \left( A\int \frac{u}{ \left( u^2 +D\right)^n}\diff u + C\int \frac{1}{\left(u^2+D\right)^n}\diff u \right) \\
&=& \displaystyle \frac{1}{a^n} \left(\frac{A}{2(1-n)} \left( u^2 +D\right)^{1-n} + C J(n ) \right) \\
&=&\displaystyle \frac{1}{a^n} \left(\frac{A}{2(1-n)} \left( x^2 + \frac{ b}{a}x + \frac{c }{a } \right)^{1-n} + C J(n ) \right) \\
\end{array}
\]

}
\solution{\ref{problemIntegrateBuildingBlockIIandIIIbPart2}.
We use all notation and computations from the previous part of the problem. According to theory, in order to solve that integral, we are supposed to integrate by parts the simpler integral 

\[
\begin{array}{rcll|l}
\displaystyle J\left(n-1\right) &=&\displaystyle \int \frac{1}{\left( x^2 +  \frac{b}{a}x +\frac{c}{a} \right)^{n-1}}\diff x = \int \frac{1}{\left(u^2+D\right)^{n-1}}\diff u &&\text{Integrate by parts}\\
&=&\displaystyle  \frac{u}{\left(u^2+D\right)^{n-1}}- \int u ~\diff \left(\frac{1}{\left(u^2+D\right)^{n-1}}\right)\\
&=&\displaystyle \frac{u}{\left(u^2+D\right)^{n-1}}+  2(n-1) \int \frac{u^2}{\left(u^2+D\right)^{n}}\diff u \\
&=&\displaystyle \frac{u}{\left(u^2+D\right)^{n-1}}+  2(n-1) \int \frac{u^2+D-D}{\left(u^2+D\right)^{n}}\diff u \\
&=&\displaystyle \frac{u}{\left(u^2+D\right)^{n-1}}+  2(n-1)J\left(n-1\right)-2D(n-1)  \int \frac{1}{\left(u^2+D\right)^{n}}\diff u \\
&=&\displaystyle \frac{u}{\left(u^2+D\right)^{n-1}}+  2(n-1)J\left(n-1\right)-2D(n-1)  J\left(n \right)
\end{array}
\]
In the above equality, we rearrange terms to get that
\[
\begin{array}{rcl}
\displaystyle 2D(n-1)  J\left(n \right)& = & \displaystyle \frac{u}{ \left(u^2 + D \right)^{n-1}} + (2n- 3) J\left( n-1\right) \\
J(n)&=&\displaystyle \frac{1}{D} \left( \frac{ u}{2( n-1) \left( u^2 +D\right)^{n-1}} +\frac{2n-3}{2n-2}J(n-1)\right) \\
&=& \displaystyle \frac{1}{D} \left( \frac{x +\frac{b }{2a}}{ (2n-2)\left(x^2+\frac{b}{a}x +\frac{c}{ a} \right)^{n-1}} +\frac{2n-3}{2n-2}J(n-1)\right)\quad .
\end{array}
\]
}
\homeworkEnd
\end{comment}

\begin{comment}
\homeworkStart{on all problems in freecalc involving integration of rational functions}{}
\item 
Integrate. Illustrate the steps of your solution.
\begin{multicols}{2}
\begin{enumerate}[ref={\fcProblemRef}]
\item $\displaystyle \int \frac{1}{x+1}\diff {}x$

\answer{$\ln |x+1|+C$}
\item $\displaystyle \int \frac{x-1}{x+1}\diff {}x$

\answer{$x-2\ln |x+1|+C$}
\item $\displaystyle \int \frac{ 1}{(x+1)^2}\diff {}x$

\answer{$-\frac{1}{x+1}+C$}
\item $\displaystyle \int \frac{x}{(x+1)^2}\diff {}x$

\answer{$\ln |x+1|+\frac{1 }{x+ 1 }+C$}
\item $\displaystyle \int \frac{ 1}{ (2x+3)^2}\diff {}x$

\answer{$-\frac{1}{2(2x+3)}+C$ }

\item $\displaystyle
\int \frac{x}{ 2x^2+3}\diff x
$

\answer{$\frac{1}{4}\ln \left(2 x^2+ 3\right)+C$ }
\item 
$\displaystyle
\int \frac{1}{ 2x^2+3}\diff x
$

\answer{$\frac{\sqrt{6}}{6} \Arctan \left(\sqrt{ \frac{ 2}{ 3} } x \right) $}

\item \label{problemIntegrate x/(2x^2+x+1) dx}
$\displaystyle\int \frac{x }{2x^2+x+1}\diff{}x\quad .
$

\answer{$\frac{1}{4}\ln \left(x^2+ \frac{1}{2}x+ \frac{ 1 }{2} \right) - \frac{\sqrt{7}}{14} \Arctan\left( \frac{ 4x+1}{\sqrt{7}} \right) +C$}

\item $\displaystyle
\int \frac{x}{ 2x^2+x+3}\diff x
$

\answer{ $\frac{1}{4}\ln \left( 2 x^{2}+x+3 \right)- \frac{1}{ 2 \sqrt{23}} \arctan \left( \frac{ 4x +1}{ \sqrt{23} } \right) +C$}

\item $\displaystyle
\int \frac{x}{ x^2-x+3}\diff x
$

\answer{$ \frac{1}{2} \ln{} \left| x^{2}- x+3\right| + \frac{ \sqrt{11}}{11} \Arctan{} \left( \frac{ x - \frac{ 1}{2} }{ \frac{\sqrt{11}}{2}} \right) + C $}

\item \label{problemint1/(x^2+1)^2dx}  $\displaystyle
\int \frac{1}{ \left(x^2+1\right)^2}\diff x
$

\answer{$ \frac{1}{2} x \left(x^{2}+1\right)^{ -1} + \frac{ 1}{2} \Arctan{}\left( x\right) +C$}
\item \label{problemint1/(x^2+x+1)^2dx} $\displaystyle
\int \frac{1}{ \left(x^2+x +1 \right)^2 } \diff x
$

\answer{$ \frac{2}{3} x \left(x^{2}+x+1\right)^{ -1} + \frac{ 1}{3} \left(x^{2}+x+ 1\right)^{ -1} +\frac{4 }{ 9} \sqrt{3} \Arctan{} \left( \frac{2}{3} \sqrt{3} x +\frac{\sqrt{3}}{3}\right) $}

\item $\displaystyle \int \frac{1}{ \left(x^2+ 1 \right)^3 }\diff x
$

\answer{$\frac{3}{8} x \left(x^{2}+ 1\right)^{ -1} + \frac{1}{4} x \left(x^{2}+1\right)^{-2}+\frac{3}{8} \Arctan{}\left( x\right)+C$}
\end{enumerate}
\end{multicols}

\item 
\label{problemIntegrateBuildingBlockIIaandIIIa} Let $a,b,c,A, B$ be real numbers. Suppose in addition $a\neq 0$ and  $b^2-4ac<0$. Integrate
\[
\int \frac{Ax +B}{ax^2+bx+c}\diff x\quad .
\]

The purpose of this exercise is to produce a formula in form ready for implementation in a computer algebra system.
\solution{\ref{problemIntegrateBuildingBlockIIaandIIIa}. 

\[
\begin{array}{rcll|l}
\displaystyle\int \frac{Ax +B}{ax^2+bx+c}\diff x&=&\displaystyle \int \frac{Ax+B}{a\left( x^2 + 2x \frac{b}{2a} +\frac{c}{a} \right) }\diff x =  \int \frac{Ax+B}{a\left( x^2 + 2x \frac{b}{2a}+\frac{b^2}{4a^2}- \frac{b^2}{4a^2} +\frac{c}{a} \right) }\diff x &&\begin{array}{l} \text{complete square}\\\text{in denominator}\end{array}\\ 
&=&\displaystyle \frac{1}{a}\int \frac{Ax+B}{ \left(x+ \frac{b}{2a} \right)^2 + \frac{4ac-b^2}{4a^2} } \diff x &&\text{Set }D=  \frac{4ac-b^2}{4a^2} \\
&=&\displaystyle \frac{1}{a}\int \frac{A \left( x+ \frac{b}{2a} - \frac{b}{2a}\right) +B}{ \left(x+ \frac{b}{2a} \right)^2 + D } \diff \left(x+\frac{b}{2a}\right) &&\text{Set } u=x+\frac{b}{2a}\\
&=&\displaystyle \frac{1}{a}\int \frac{Au+ B- \frac{Ab}{2a}}{u^2+D} \diff u &&\text{Set }C=B-\frac{Ab}{2a} \\
&=&\displaystyle \frac{1}{a}\left(A\int \frac{u}{u^2+D}\diff u + C\int \frac{1}{u^2+D}\diff u \right) \\
&=&\displaystyle \frac{1}{a} \left(\frac{A}{2}\ln (u^2+D) + \frac{C}{\sqrt{D}}\Arctan\left(\frac{u}{\sqrt{D}}\right)\right) +K\\
&=& \displaystyle \frac{1}{a} \left(\frac{A}{2}\ln \left(x^2+ \frac{b}{a}x+\frac{c}{a} \right) + \frac{C}{\sqrt{D}}\Arctan\left(\frac{x+\frac{b}{2a}}{\sqrt{D}} \right)\right) +K \quad .
\end{array}
\] 
The solution is complete. Question to the student: where do we use $b^2-4ac<0$?
}
\item 

Let $a, b, c, A, B$ be real numbers and let $n>1$ be an integer. Suppose in addition $a\neq 0$ and $b^2-4ac<0$. Let 
\[
J(n)=\int \frac{1}{ \left( x^2+\frac{b}{a}x + \frac{ c}{a}\right)^n}\diff x\quad .
\]
\begin{enumerate}
\item \label{problemIntegrateBuildingBlockIIandIIIbPart1} Express the integral 
\[
\int \frac{Ax +B}{ \left( ax^2 +bx +c\right)^n}\diff x
\]
via $J(n)$.
\item \label{problemIntegrateBuildingBlockIIandIIIbPart2} Express $J(n)$ recursively via $J(n-1)$
\end{enumerate}
The purpose of this exercise is to produce a formula in form ready for implementation in a computer algebra system.

\solution{\ref{problemIntegrateBuildingBlockIIandIIIbPart1}.\[
\begin{array}{rcll|l}
\displaystyle\int \frac{Ax +B}{(ax^2+bx+c)^n}\diff x&=&\displaystyle \int \frac{Ax+B}{a^n\left( x^2 + 2x \frac{b}{2a} +\frac{c}{a} \right)^n }\diff x =  \int \frac{Ax+B}{a^n\left( x^2 + 2x \frac{b}{2a}+\frac{b^2}{4a^2}- \frac{b^2}{4a^2} +\frac{c}{a} \right)^n }\diff x &&\begin{array}{l} \text{complete square}\\\text{in denominator}\end{array}\\ 
&=& \displaystyle \frac{1}{a^n} \int \frac{Ax+B}{ \left( \left( x + \frac{b}{2a} \right)^2 + \frac{ 4ac-b^2}{4a^2} \right)^n } \diff x &&\text{Set }D=  \frac{ 4ac-b^2}{4a^2} \\
&=&\displaystyle \frac{1}{ a^n}\int \frac{A \left( x+ \frac{b}{2a} - \frac{b}{ 2a} \right) +B}{ \left(\left(x+ \frac{b}{2a} \right)^2 + D \right)^n } \diff \left(x + \frac{ b}{2a}\right) && \text{Set } u=x+\frac{b}{2a}\\
&=& \displaystyle \frac{1}{a^n} \int \frac{Au+ B- \frac{Ab}{2a} }{ \left(u^2+D\right)^n} \diff u &&\text{Set }C = B - \frac{A b}{2 a} \\
&=&\displaystyle \frac{1 }{a^n} \left( A\int \frac{u}{ \left( u^2 +D\right)^n}\diff u + C\int \frac{1}{\left(u^2+D\right)^n}\diff u \right) \\
&=& \displaystyle \frac{1}{a^n} \left(\frac{A}{2(1-n)} \left( u^2 +D\right)^{1-n} + C J(n ) \right) \\
&=&\displaystyle \frac{1}{a^n} \left(\frac{A}{2(1-n)} \left( x^2 + \frac{ b}{a}x + \frac{c }{a } \right)^{1-n} + C J(n ) \right) \\
\end{array}
\]

}
\solution{\ref{problemIntegrateBuildingBlockIIandIIIbPart2}.
We use all notation and computations from the previous part of the problem. According to theory, in order to solve that integral, we are supposed to integrate by parts the simpler integral 

\[
\begin{array}{rcll|l}
\displaystyle J\left(n-1\right) &=&\displaystyle \int \frac{1}{\left( x^2 +  \frac{b}{a}x +\frac{c}{a} \right)^{n-1}}\diff x = \int \frac{1}{\left(u^2+D\right)^{n-1}}\diff u &&\text{Integrate by parts}\\
&=&\displaystyle  \frac{u}{\left(u^2+D\right)^{n-1}}- \int u ~\diff \left(\frac{1}{\left(u^2+D\right)^{n-1}}\right)\\
&=&\displaystyle \frac{u}{\left(u^2+D\right)^{n-1}}+  2(n-1) \int \frac{u^2}{\left(u^2+D\right)^{n}}\diff u \\
&=&\displaystyle \frac{u}{\left(u^2+D\right)^{n-1}}+  2(n-1) \int \frac{u^2+D-D}{\left(u^2+D\right)^{n}}\diff u \\
&=&\displaystyle \frac{u}{\left(u^2+D\right)^{n-1}}+  2(n-1)J\left(n-1\right)-2D(n-1)  \int \frac{1}{\left(u^2+D\right)^{n}}\diff u \\
&=&\displaystyle \frac{u}{\left(u^2+D\right)^{n-1}}+  2(n-1)J\left(n-1\right)-2D(n-1)  J\left(n \right)
\end{array}
\]
In the above equality, we rearrange terms to get that
\[
\begin{array}{rcl}
\displaystyle 2D(n-1)  J\left(n \right)& = & \displaystyle \frac{u}{ \left(u^2 + D \right)^{n-1}} + (2n- 3) J\left( n-1\right) \\
J(n)&=&\displaystyle \frac{1}{D} \left( \frac{ u}{2( n-1) \left( u^2 +D\right)^{n-1}} +\frac{2n-3}{2n-2}J(n-1)\right) \\
&=& \displaystyle \frac{1}{D} \left( \frac{x +\frac{b }{2a}}{ (2n-2)\left(x^2+\frac{b}{a}x +\frac{c}{ a} \right)^{n-1}} +\frac{2n-3}{2n-2}J(n-1)\right)\quad .
\end{array}
\]
}
\item 
Evaluate the indefinite integral. Illustrate all steps of your solution. 
\begin{enumerate}
\item (3 points) $\int \frac1{x^2+x+1}dx$
\item (3 points) $\int \frac1{4x^2+4x+1}dx$
\item (3 points) $\int \frac1{2x^2+5x+1}dx$
\item (5 points) $\int \frac{15x^2-4x-81}{(x-3)(x+4)(x-1)}dx$
\item (5 points) $\int \frac {x^{4}+10x^{3}+18x^{2}+2x-13}{x^{4}+4x^{3}+3x^{2}-4x-4}dx$ 

Check first that $(x-1)(x+2)^2(x+1)= x^{4}+4x^{3}+3x^{2}-4x-4$. \end{enumerate}
\item 
Integrate.
\begin{enumerate}
\item $\displaystyle\int \frac{x^3}{x^2+2x-3}\diff x$.
\answer{$\frac{1}{4} \ln{}|x-1|+\frac{27}{4} \ln{}|x+3|+\frac{1}{2} x^{2}-2 x$}
\item $\displaystyle\int \frac{x^3}{x^2+3x-4}\diff x$.
\answer{$\frac{1}{2}x^2-3x+\frac{64}{5}\ln|x+4| + \frac{1}{5}\ln|x-1| + C$}

\end{enumerate}
\item 
Integrate 
\begin{enumerate}
\item $\int\limits_{1}^3 \frac{x^4 }{(x+1)^2(x+2) } \diff x$
\item $\int\limits_{0}^1 \frac{x^4 }{(x^2+2)(x+2) } \diff x$
\end{enumerate}

\item 
Integrate 
\begin{multicols}{2}
\begin{enumerate}
\item $\displaystyle \int \frac{x^2+1}{(x-3)(x-2)^2}\diff x$.

\answer{$\displaystyle  10\ln |x-3|-9\ln |x-2|+\frac{5}{x-2}+C $}
\item $\displaystyle  \int \frac{x^3+4}{x^2+4}\diff x$.

\answer{$\displaystyle \frac{x^2}{2} +2\arctan \left(\frac{x}{2}\right)-2\ln \left(x^2+4\right)+C$}
\end{enumerate}
\end{multicols}
\item 
Integrate. Some of the examples require partial fraction decomposition and some do not. Illustrate the steps of your solution. 
\begin{multicols}{2}
\begin{enumerate}
\item $\displaystyle\int \frac{1}{4x^2+4x+1}\diff x$

\answer{$-\frac{1}{2} (2 x+1)^{-1}+C$}
\item $\displaystyle \int \frac{1}{1-x^2}\diff x$

\answer{$-\frac{1}{2} \ln{}\left|x-1\right|+\frac{1}{2} \ln{} \left| x+1\right|+C $}

\item $\displaystyle \int \frac{1}{5-x^2}\diff x$

\answer{$-\frac{\sqrt{5}}{10} \ln{}\left|x- \sqrt{5} \right|+ \frac{\sqrt{5}}{10} \ln{}\left|x+ \sqrt{5} \right|+C $}

\item $\displaystyle \int \frac{x}{4x^2+x+5}\diff {}x$

\answer{$\frac{1}{8} \ln{}\left(x^{2}+\frac{1}{4} x+\frac{5}{4}\right)-\frac{1}{316}\sqrt{79} \arctan{}\left(\frac{x+\frac{1}{8}}{\frac{1}{8}\sqrt{79}}\right) +C $}
\item $\displaystyle \int \frac{x}{4x^2+x-5}\diff {}x$
\answer{$ \frac{5}{36} \ln{}\left|x+\frac{5}{4}\right|+\frac{1}{9} \ln{}\left|x-1\right|+C$}
\item $\displaystyle \int \frac{x}{4x^2 +x+ \frac{ 1}{16}}\diff {}x$
\answer{$\frac{1}{4} (8 x+1)^{-1}+\frac{1}{4} \ln{}\left|8 x+1\right|+C $}

\item $\displaystyle \int \frac{x+1}{2x^2+x}\diff {}x$

\answer{$-\frac{1}{2} \ln{}\left|2 x+1\right|+\ln{}\left|x\right| +C$}
\item $\displaystyle \int \frac{x}{2x^2+x+1}\diff {}x$

\answer{$\frac{1}{4} \ln\left(x^{2}+\frac{1}{2} x+\frac{1}{2}\right)-\frac{1}{14}\sqrt{7} \Arctan{}\left(\frac{x+\frac{1}{4}}{\frac{1}{4}\sqrt{7}}\right) +C $}
\item \label{problemIntegrate x/(2x^2+x-1)dx}
$\displaystyle\int \frac{x }{2x^2+x-1}\diff{}x
$

\answer{$\displaystyle \frac{1}{3} \ln |x+1|+\frac{1}{6} \ln \left|x-\frac{1}{2}\right| +C$}

\item $\displaystyle\int \frac{1}{x^2+x+1}\diff x$

\answer{$\frac{2}{3}\sqrt{3} \arctan{} \left( \frac{ x+ \frac{1}{2}}{\frac{1}{2}\sqrt{3}}\right) +C $}
\item $\displaystyle\int \frac{1}{2x^2+5x+1}\diff x$

\answer{$ \frac{\sqrt{17}}{17} \ln{}\left|x- \frac{ \sqrt{17} }{4} + \frac{5}{4}\right|- \frac{ \sqrt{17} }{17} \ln{}\left|x+ \frac{\sqrt{17}}{4}  + \frac{ 5}{4}\right| $}
\end{enumerate}
\end{multicols}

\homeworkEnd

\end{comment}


\begin{comment}
\homeworkStart{on Lecture 6 \\Will be quizzed: time to be announced in class}{}
\item 
Evaluate the indefinite integral. Illustrate all steps of your solution. 
\begin{enumerate}
\item (3 points) $\int \frac1{x^2+x+1}dx$
\item (3 points) $\int \frac1{4x^2+4x+1}dx$
\item (3 points) $\int \frac1{2x^2+5x+1}dx$
\item (5 points) $\int \frac{15x^2-4x-81}{(x-3)(x+4)(x-1)}dx$
\item (5 points) $\int \frac {x^{4}+10x^{3}+18x^{2}+2x-13}{x^{4}+4x^{3}+3x^{2}-4x-4}dx$ 

Check first that $(x-1)(x+2)^2(x+1)= x^{4}+4x^{3}+3x^{2}-4x-4$. \end{enumerate}
\homeworkEnd
\end{comment}

\begin{comment}
\homeworkStart{on Lecture 7 \\Will be quizzed: time to be announced in class}{}
\item 
Integrate.
\begin{multicols}{2}
\begin{enumerate}[ref={\fcProblemRef}]
\item \label{problemInt1/(3+cos x)dx} $\displaystyle \int \frac{1}{3+\cos x}\diff x$.

\answer{$\displaystyle\frac{1}{\sqrt{2}} \Arctan \left( \frac{ 1}{ \sqrt{2}} \tan \left( \frac{x }{2} \right)\right)+C$}
\item $\displaystyle \int \frac{1}{4+\cos x}\diff x$.

\answer{$\displaystyle  \frac{2}{15}\sqrt{15} \Arctan{}\left(\frac{\sqrt{15}}{5}\tan\left(\frac{x}{2}\right)\right)+C$}
\item $\displaystyle \int \frac{1}{3+\sin x}\diff x$.

\answer{$\displaystyle \frac{1}{\sqrt{2}} \Arctan \left(\frac{3 \tan \left(\frac{x}{2} \right)+1 }{2\sqrt{2}}\right) +C$}
\item \label{problemInt1/(2+tan x)dx}$\displaystyle \int \frac{1}{2+\tan x}\diff x$.  (Hint: this integral can be done simply with the substitution $x=\Arctan t$.)

\answer{$\displaystyle \frac{1}{5} \ln \left(\sin x+2\cos x\right)+\frac{2}{5}x+C$}

\item \label{problemint1/(2sinx-cosx+5)dx} $\displaystyle \int \frac{ \diff x }{ 2\sin x - \cos x +5}$.


\answer{$\displaystyle \frac{ \sqrt{5}}{5}\Arctan \left( \frac{3}{ \sqrt{5}} \left({ \tan \left(\frac{\theta}{2} \right)}+\frac{1}{3} \right) \right)+C$}
\end{enumerate}
\end{multicols}
Integrate.
\begin{enumerate}
\item $\displaystyle \int \sin (3 x) \cos (2x)\diff x$.
\item $\displaystyle \int \sin x \cos (5x)\diff x$.
\item $\displaystyle \int \cos (3x) \sin (2x)\diff x$.
\end{enumerate}

Integrate.
\begin{multicols}{2}
\begin{enumerate}[ref={\fcProblemRef}]
\item $\displaystyle \int \sin^2 x \cos x\diff x$.

\answer{$ \frac{1}{3}\sin^3 x+C$}

\item $\displaystyle \int \sin^2 x\diff x$.

\answer{$ \frac{x}{2} -\frac{1}{4}\sin (2x) +C$}
\item $\displaystyle \int \cos^3 x\diff x$.

\answer{$ \sin x -\frac{1}{3}\sin^3 x+C$}


\item 
$\displaystyle \int \sin^3 x \cos^4 x\diff x$.

\answer{$ \frac{1}{7} \cos^{7}{}x-\frac{1}{5} \cos^{5}{}x +C$}

\end{enumerate}
\end{multicols}
Integrate 
\begin{enumerate}
\item $\displaystyle \int \sec^3x  \diff x$.
\item $\displaystyle \int \tan^3x \diff x$.
\item $\displaystyle \int \sec^2x\tan^2x \diff x$.

\end{enumerate}

\homeworkEnd
\end{comment}

\begin{comment}
\homeworkStart{on Lecture 8 \\Will be quizzed: date to be announced}{}
\item Compute the integral.

\begin{enumerate}[ref={\fcProblemRef}]
\item \label{problemIntegratesqrt(x^2+1)dx}
$\displaystyle
\int \sqrt{x^2+1}\diff x
$

\answer{$\frac{1}{2}x\sqrt{x^2+1}+\frac{1}{2}\ln\left( \sqrt{x^2+1}+x\right)+C $}

\item $\displaystyle
\int \sqrt{x^2+2}\diff x
$

\answer{$ \ln{}\left(\sqrt{\frac{1}{2} x^{2}+1}+\frac{\sqrt{2}}{2} x\right)+\frac{\sqrt{2}}{2} x \sqrt{\frac{1}{2} x^{2}+1}+C$}

\item 
$\displaystyle
\int \sqrt{x^2+x+1}\diff x
$

\answer{$\frac{3}{4} \left(\frac{1}{2} \ln{}\left(\sqrt{\frac{4}{3} (x+\frac{1}{2})^{2}+1}+\frac{2}{3}\sqrt{3} \left(x+\frac{1}{2}\right)\right)+\frac{\sqrt{3}}{3} \left(x+\frac{1}{2}\right) \sqrt{\frac{4}{3} (x+\frac{1}{2})^{2}+1}\right)  +C$}
\item \label{problemIntegrate sqrt(2x^2+2x+1)dx}
$\displaystyle
\int \sqrt{\left(2x^2+2x+1\right)}\diff x
$

\answer{$\frac{\sqrt{2}}{4} \left( \frac{1}{2} (2x+1)\sqrt{(2x+1)^2+1 }+ \frac{1}{2}\ln \left( \sqrt{(2x+1)^2+1 }+2x+1\right) \right)+C$}

\item \label{problemIntegrate sqrt(3x^2+2x+1)dx}
$\displaystyle
\int \sqrt{\left(3x^2+2x+1\right)}\diff x
$

\answer{$
\begin{array}{l}
\frac{2}{9}\sqrt{3} \left(\frac{1}{2} \ln{}\left(\sqrt{\frac{9}{2} (x+\frac{1}{3})^{2}+1}+\frac{3}{2}\sqrt{2} \left(x+ \frac{1}{3} \right)\right)\right. \\\left. +\frac{3}{4}\sqrt{2} \left(x+\frac{1}{3}\right) \sqrt{\frac{9}{2} (x+\frac{1}{3})^{2}+1}\right) +C \end{array}$}

\item \label{problemintsqrt(x^2+1)/(x+1)dx}
$\displaystyle\int \frac{\sqrt{x^2+1}}{x+1}\diff x $

\answer{$ \begin{array}{l} -\sqrt{2} \ln{}\left(\sqrt{x^{2}+1}- x+\sqrt{2}-1\right) \\
+\sqrt{2} \ln{}\left(\sqrt{x^{2}+1}- x-\sqrt{2}-1\right)\\
+ \ln{} \left( \sqrt{x^{2}+1}- x\right)\\
+ \sqrt{x^{2}+1}
\end{array}
$}
\end{enumerate}

\item Integrate
\begin{enumerate}[ref={\fcProblemRef}]
\item  \label{problemintsqrt(1-x^2)dx}
$\displaystyle
\int \sqrt{1-x^2}\diff x$

\answer{$ $}
\item 
$\displaystyle
\int \sqrt{2-x^2}\diff x
$
\item 
$\displaystyle
\int \sqrt{-x^2+x+1}\diff x
$
\item 
$\displaystyle
\int \sqrt{2-x-x^2}\diff x
$

\item \label{problemintsqrt(1-x^2)/(1+x)dx}
$\displaystyle
\int \frac{\sqrt{1-x^2} }{1+x}\diff x
$
\item 
$\displaystyle
\int \frac{\sqrt{1-x^2} }{2+x}\diff x
$
\end{enumerate}
\item Integrate
\begin{enumerate}[ref={\fcProblemRef}]
\item 
\[
\int \sqrt{x^2-1}\diff x
\]
\item 
\[
\int \sqrt{x^2-2}\diff x
\]
\item 
\[
\int \sqrt{2x^2+x-1}\diff x
\]
\item 
\[
\int \sqrt{x^2+x-1}\diff x
\]
\end{enumerate}

\homeworkEnd
\end{comment}

\begin{comment}
\homeworkStart{on Lecture 9 \\Quiz time to be announced in class}{}
\item Compute the limits. The answer key has not been fully proofread, use with caution.
\begin{multicols}{2}
\begin{enumerate}
\item $\displaystyle \lim\limits_{x\to 0} \frac{\sin x  }{x}$. 

\answer{$1$}
\item $\displaystyle \lim\limits_{x\to 0} \frac{x}{\ln (1+x)}$. 

\answer{$ 1$}
\item $\displaystyle \lim\limits_{x\to 0} \frac{x^2}{x-\ln (1+x)}$. 

\answer{$2$}
\item $\displaystyle \lim\limits_{x\to 0} \frac{x^2}{\sin x\ln (1+x)}$. 

\answer{$ 1$}
\item $\displaystyle \lim\limits_{x\to 0} \frac{\sin^2 x  }{\left(\ln (1+x)\right)^2}$.

\answer{$ 1$}
\item $\displaystyle \lim\limits_{x\to 0} \frac{\cos x- 1}{\sin x\ln (1+x)}$.

\answer{$- \frac{1}{2} $}
\item $\displaystyle \lim\limits_{x\to 0} \frac{\arctan x -x}{x^3} $.

\answer{$ -\frac{1}{3} $}
\item $\displaystyle \lim\limits_{x\to 0} \frac{\arcsin x -x}{x^3} $.

\answer{$ \frac{1}{6}$}
\item $\displaystyle \lim\limits_{x\to 1} \frac{x}{x-1}-\frac{1}{\ln x}$.

\answer{$ \frac{1}{2}$}
\item $\displaystyle \lim\limits_{x\to 0} \frac{\cos (nx) -\cos (mx)}{x^2 }$.

\answer{$\frac{m^2-n^2}2 $}
\item \label{eqProblemLimlixto0(arcsinx-x-x^3/6)/(sin^5 x)}  $\displaystyle \lim \limits_{x\to 0} \frac{\arcsin x-x-\frac{1}{6}x^3}{\sin^5 x} $. 

\answer{$\frac{3}{40}$}
\item \label{problemLHospital (sin (pi x) ln x )/ (cos pi x +1)}  $\displaystyle \lim\limits_{x\to 1} \frac{\sin \left(\pi x \right)\ln x }{\cos(\pi x)+1 } $.

\answer{$-\frac{2}{\pi}$}

\item $\displaystyle \lim\limits_{x\to 0} \frac{\sin x-x }{\arcsin x-x } $.

\answer{$ -1$}
\item \label{problemlim x to 0 (sin x - x)/(arctan x - x)} $\displaystyle \lim\limits_{x\to 0}\frac{\sin x- x}{\Arctan x -x}$.

\answer{$\frac{1}{2}$}

\item 
\label{problemlimxtoinftysin(2/x)}
$ {\displaystyle \lim_{x \to \infty} x \sin\left(\frac{2}{x}\right)}.$

\answer{$2$}

\end{enumerate}
\end{multicols}
The very last problem can be done easily using Maclaurin series, but we challenge the student to try it using L'Hospital's rule. This problem is solved in Chapter 9 of the Calculus notes ``Calculus for beginners''.
\homeworkEnd
\end{comment}
\begin{comment}
\homeworkStart{on Lecture  10\\ will be quizzed: date to be announced in class}{}
\item Determine whether the integral is convergent or divergent. Motivate your answer.
\begin{multicols}{2}
\begin{enumerate}[ref={\fcProblemRef}]
\item $\displaystyle \int\limits_{2}^{\infty}\frac{1}{(x-1)^{\frac32}} \diff x$.

\answer{convergent}
\item $\displaystyle \int\limits_{-1}^{1}\frac{1}{\sqrt[5]{1+x}} \diff x$.

\answer{convergent}
\item $\displaystyle \int\limits_{1}^{\infty }\frac{1}{\sqrt[5]{1+x}} \diff x$.

\answer{divergent}
\item $\displaystyle \int\limits_{-1}^{\infty }\frac{1}{\sqrt[5]{1+x}} \diff x$.

\answer{divergent}
\item $\displaystyle \int\limits_{-\infty }^{0}\frac{1}{2-3x} \diff x$.

\answer{divergent}
\item $\displaystyle \int\limits_{-\infty }^{0}\frac{1}{(2-3x)^{2}} \diff x$.

\answer{convergent}
\item $\displaystyle \int\limits_{-\infty }^{0}\frac{1}{(2-3x)^{1.00000001}} \diff x$.

\answer{convergent}
\item $\displaystyle \int\limits_{-2}^{ \frac{1}{ 2}} \frac{1}{2x-1} \diff x$.

\answer{divergent}

\item $\displaystyle \int\limits_{-1}^{\infty} e^{-3x} \diff x$.

\answer{convergent, equals $\frac{e^{3}}{3}$}

\item $\displaystyle \int\limits_{-\infty }^{5}  2^x \diff x$.

\answer{convergent}
\item $\displaystyle \int\limits_{-\infty }^{\infty}x^3 \diff x$.

\answer{divergent}
\item $\displaystyle \int\limits_{-\infty}^{\infty} x e^{-x^2} \diff x$.

\answer{convergent, equals $0$}

\item \label{problemConvergencesqrt(x)e^-sqrt(x)zerotoinfty} $\displaystyle \int\limits_{0}^{\infty} \sqrt{x} e^{-\sqrt{x}} \diff x$.

\answer{convergent, equals $4 $}

\item $\displaystyle \int\limits_{0}^{\infty}\sin^2 x \diff x$.

\answer{divergent}
\item $\displaystyle \int\limits_{0}^{5}\frac{1}{x^2+x-2} \diff x$.

\answer{divergent}
\item $\displaystyle \int\limits_{0}^{\infty}\frac{1}{x^2+x+1} \diff x$.

\answer{convergent}
\item $\displaystyle \int\limits_{2}^{\infty}\frac{1}{x^2-x-1} \diff x$.

\answer{convergent}
\item $\displaystyle \int\limits_{0}^{\infty}\frac{1}{x^2-x-1} \diff x$.

\answer{divergent}
\item \label{problemConvergencex^2/(x^4+2)from-inftyto+infty}

$\displaystyle \int\limits_{-\infty}^{\infty} \frac{x^2}{x^4+2} \diff x$.
\answer{convergent}

\item 
$\displaystyle \int\limits_{100}^{\infty} \frac{1}{x\ln x} \diff x$.

\answer{divergent}

\item $\displaystyle \int\limits_{100}^{\infty} \frac{1}{x(\ln x)^2} \diff x$.

\answer{convergent}
\item $\displaystyle \int\limits_{0}^{1}\ln x \diff x$.

\answer{convergent}
\item $\displaystyle \int\limits_{0}^{1}\frac{\ln x}{\sqrt{x}} \diff x$.

\answer{convergent}
\item $\displaystyle \int\limits_{0}^{2}x^3\ln x \diff x$.

\answer{convergent, equals $-1+4 \ln 2 $}

\item $\displaystyle \int\limits_{0}^{1} \frac{e^{\frac{1}{x}}}{x^2} \diff x$.

\answer{divergent}
\item $\displaystyle \int\limits_{-1}^{0} \frac{e^{\frac{1}{x}}}{x^2} \diff x$.

\answer{convergent}

\end{enumerate}
\end{multicols}

\item Determine whether the integral is convergent or divergent. Motivate your answer. The answer key has not been proofread, use with caution.

\begin{multicols}{2}
\begin{enumerate}
\item $\displaystyle \int\limits_{100}^{\infty} \frac{1}{x\ln x} \diff x$.
\answer{divergent}
\item $\displaystyle \int\limits_{100}^{\infty} \frac{1}{x(\ln x)^2} \diff x$.
\answer{convergent}
\item $\displaystyle \int\limits_{0}^{1}\ln x \diff x$.
\answer{convergent}
\item $\displaystyle \int\limits_{0}^{1}\frac{\ln x}{\sqrt{x}} \diff x$.
\answer{convergent}
\item $\displaystyle \int\limits_{0}^{2}x^3\ln x \diff x$.
\answer{convergent}
\item $\displaystyle \int\limits_{0}^{1} \frac{e^{\frac{1}{x}}}{x^2} \diff x$.
\answer{divergent}
\item $\displaystyle \int\limits_{-1}^{0} \frac{e^{\frac{1}{x}}}{x^2} \diff x$.
\answer{convergent}
\item $\displaystyle \int \limits_{0}^{\infty}\sin x^2\diff x$ (This problem is more difficult and may require knowledge of sequences to solve).
\answer{convergent}
\end{enumerate}
\end{multicols}

\homeworkEnd
\end{comment}

%\begin{comment}
\homeworkStart{on Lecture 11 \\Will be quizzed on Wednesday}{}
\item List the first 4 elements of the sequence. 
\begin{multicols}{2}
\begin{enumerate}
\item $\displaystyle a_n= \frac{(-1)^n}{n}$.
\item $\displaystyle a_n=\frac{1}{n!}$.
\item $\displaystyle a_n=\cos (\pi n)$.
\item $\displaystyle a_n=\frac{(-1)^n}{2n+1}$.
\item $\displaystyle a_n=\frac{\sqrt{5}}{5}\left( \left(\frac{1+\sqrt{5 }}{2} \right)^n- \left(\frac{1-\sqrt{5}}{2}\right)^n\right) $
\end{enumerate}
\end{multicols}
\item List the first 5 elements of the sequence. 
\begin{multicols}{2}
\begin{enumerate}
\item $\displaystyle a_{n+1}=\frac{1}{2}\left(a_n+ \frac{3}{a_n}\right)$.
\item $\displaystyle a_n=a_{n-1}+a_{n-2}$, $a_1=1$, $a_2=1$.
\item $\displaystyle a_n= \frac{\left(\frac{1}{2}-n\right)}{n} a_{n-1} $, $a_0=1$.
\item $\displaystyle a_n= a_{n-1}+2n+1$, $a_0=1$.
\item $\displaystyle a_n:=\frac{1}{n} a_n$.
\end{enumerate}
\end{multicols}
\item Give a simple sequence formula that matches the pattern below. 

\begin{multicols}{2}
\begin{enumerate}
\item $\displaystyle \left(1, \frac{1}{3}, \frac{1}{5}, \frac{1}{7},\frac{1}{9},\dots \right)$.

\answer{$a_n=\frac{1}{2n-1}$}
\item $\displaystyle \left(-1, \frac{1}{5}, -\frac{1}{25}, \frac{1}{125},-\frac{1}{625}, \frac{1}{3125}\dots \right)$

\answer{$a_n=\left(-\frac{1}{5}\right)^{n-1}$}
\item $\displaystyle \left(-5, 2, -\frac{4}{5}, \frac{8}{25}, -\frac{16}{125}, \frac{32}{625},\dots \right)$

\answer{$a_n=-5\left(-\frac{2}{5}\right)^{n-1}$}

\item $\displaystyle \left(4, 7, 10, 13, 16, 19,\dots\right)$

\answer{$a_n=3 {{n}}+1 $}

\item $(-2, \frac{3}{4}, -\frac{4}{9}, \frac{5}{16}, -\frac{6}{25}, \frac{7}{36})$

\answer{$a_n=(-1)^{n}\left(\frac{n +1}{n^{2}}\right)$}

\item $\left(0,-1, 0, 1,0,-1, 0, 1,0,-1, 0, 1,\dots \right)$

\answer{$a_n=\cos\left(n\frac{\pi}{2}\right)$}
\end{enumerate}
\end{multicols}
\item 
Determine if the sequence is convergent or divergent. If convergent, find the limit of the sequence.
\begin{multicols}{2}
\begin{enumerate}
\item $\displaystyle a_n=n$.
\answer{divergent}
\item $\displaystyle a_n=2^n$.
\answer{divergent}
\item $\displaystyle a_n=1.0001^n$.
\answer{divergent}
\item $\displaystyle a_n=0.999999^n$.
\answer{convergent, $\lim_{n\to \infty} a_n=0$}
\item $\displaystyle a_n=n-\sqrt{n+1}\sqrt{n+2}$
\answer{convergent, $\lim_{n\to \infty} a_n=-\frac{3}{2}$}
\item $\displaystyle a_n=\frac{\ln n}{n}$.
\answer{convergent, $\lim_{n\to \infty} a_n=0$}
\item $\displaystyle a_n=\frac{\ln n}{\sqrt[10]{n}}$.
\answer{convergent, $\lim_{n\to \infty} a_n=0$}
\item $\displaystyle a_n=\frac{1}{n}$.
\answer{convergent, $\lim_{n\to \infty} a_n=0$}
\item $\displaystyle a_n=\frac{1}{n!}$.
\answer{convergent, $\lim_{n\to \infty} a_n=0$}
\item $\displaystyle a_n=\frac{n^n}{n!}$.
\answer{divergent}
\item $\displaystyle a_n=\cos n$.
\answer{divergent}
\item $\displaystyle a_n=\cos\left(\frac{1}{n}\right)$
\answer{convergent, $\lim_{n\to \infty} a_n=1$}

\item $\displaystyle a_n= \left(\frac{n+1}{n}\right)^{n}$.
\answer{convergent, $\lim_{n\to \infty} a_n=e$}
\item $\displaystyle a_n= \left(\frac{2n+1}{n}\right)^{n}$.
\answer{divergent}

\item $\displaystyle a_n= \left(\frac{n+1}{n}\right)^{2n}$.
\answer{convergent, $\lim_{n\to \infty} a_n=e^2$}

\item $\displaystyle a_n= \left(\frac{n+1}{2n}\right)^{n}$.
\answer{convergent, $\lim_{n\to \infty} a_n=0$}
\end{enumerate}
\end{multicols}
\homeworkEnd
%\end{comment}
\begin{comment}
\homeworkStart{on Lecture 12 \\Will not be quizzed, problem types will be on test}{}
\item Express the infinite decimal number as a rational number.
\begin{multicols}{2}
\begin{enumerate}
\item $1.\overline{6}=1.6666\dots$

\answer{ $\frac{5}{3} $}
\item $1.\overline{3}=1.3333\dots$

\answer{ $\frac{4}{3} $}
\item $2.\overline{16}=2.16161616\dots$

\answer{$\frac{214}{99}$ }
\item $2014.\overline{2014}=2014.2014201420142014\dots$

\answer{$\frac{20140000}{9999}$}
\end{enumerate}
\end{multicols}
\item Use partial fractions to sum the telescoping series (a sum is ``telescoping'' if it can be broken into summands so that consecutive terms cancel).
\begin{multicols}{2}
\begin{enumerate}
\item $\displaystyle \sum\limits_{x=1}^\infty \frac{1}{x^{2}+x}$

\answer{$1$}
\item $\displaystyle \sum\limits_{x=2}^\infty\frac{2 x+1}{x^{4}+2 x^{3}- x^{2}-2 x}$

\answer{$\frac{1}{3}$}

\item $\displaystyle \sum\limits_{x=1}^\infty \frac{2 x}{x^{4}-3 x^{2}+1}$

\answer{$-1$}

\item $\displaystyle \sum\limits_{x=1}^\infty \frac{x^{2}+x+2}{ x^{4}- 5  x^{2}+4}$
\answer{$-\frac{1}{2}$}

\end{enumerate}
\end{multicols}
\item (Problem (e) will NOT appear on the quiz) Use the integral test, the comparison test or the limit comparison test to determine whether the series is convergent or divergent. Justify your answer.
\begin{multicols}{2}
\begin{enumerate}[ref={\fcProblemRef}]
\item $\displaystyle \sum\limits_{n=1}^{\infty} \frac{1}{2n+1}$.

\answer{divergent}

\item $\displaystyle \sum\limits_{n=1}^{\infty} \frac{1}{2n^2+n^3}$.

\answer{convergent, compare to $\sum\limits_{n=1}^{\infty} \frac{1}{2n^2}$}

\item $\displaystyle \sum\limits_{n=1}^{\infty}\frac{n^2+3}{3n^5+n}$

\answer{convergent, can use limit comparison test}
\item $\displaystyle \sum\limits_{n=0}^{\infty} \frac{1}{3^n+5}$.

\answer{convergent, compare to $\sum_{n=0}^{\infty} \frac{1}{3^n}$}

\item \label{problemConvergencesum_2^infty1/(xlnx)dx}
$\displaystyle \sum_{n=2}^\infty \frac{1}{n \ln n}$

\answer{divergent, integral test}
\item  \label{problemConvergencesum_2^infty1/((2n+1)ln(n)}
$\displaystyle \sum\limits_{n=2}^{\infty} \frac{1}{(2n+1)\ln (n)}$.

\answer{divergent}
\item $\displaystyle \sum\limits_{n=2}^{\infty}\frac{1}{n(\ln n)^2}$

\answer{convergent, can use integral test}
\item 
$\displaystyle \sum\limits_{n=2}^{\infty} \frac{1}{(2n+1)(\ln (n))^2}$.

\answer{convergent}
\item 
Determine all values of $p$, $q$ $r$ for which the series 
\[
\displaystyle \sum_{n=30}^{\infty} \frac{1}{n^p(\ln n)^q(\ln (\ln n))^r}
\]
is convergent.


\end{enumerate}
\end{multicols}


\homeworkEnd
\end{comment}
\begin{comment}
\homeworkStart{on Lecture 13 \\Quiz date to be announced}{}
\item Determine the interval of convergence for the series. You may use either the ratio test or the root test, or any other method that works.

\begin{multicols}{2}
\begin{enumerate}
\item 
$\displaystyle \sum\limits_{n=0}^{\infty} \frac{x^n}{n!}$

\answer{converges for all $x$}
\item 
$\displaystyle \sum\limits_{n=0}^{\infty} (n+1)x^n $

\answer{converges for $|x|<1$}
\item 
$\displaystyle \sum\limits_{n=1}^{\infty} \frac{x^n}{n}$
\answer{converges for $|x|\in[-1,1)$.}
\item 
$\displaystyle\sum\limits_{n=1}^{\infty} (-1)^n\frac{x^{2n+1}}{2n+1}$

\answer{converges for $|x|\in (-1, 1]$}.
\item 
$\displaystyle \sum\limits_{n=1}^{\infty} \binom{\frac{1}{2}}{n}x^{n}$, where we recall that the binomial coefficient $\displaystyle \binom{q}{n}$ stands for $\displaystyle\frac{q (q-1)\dots (q-n+1)}{n!}$.

\answer{converges for $x\in (-1,1]$. } 
\end{enumerate}
\end{multicols}
\item Except for $x=\pm e$, use the ratio test to determine all real values of $x$ for which 
\[
\sum_{n=0}^{\infty}x^n\frac{n!}{n^n}
\]

You are expected to use in your solution the already studied fact that 
\[
\lim_{x\to 0}\left(1+\frac{x}{n}\right)^n=e^x\quad .
\]

\item Compute the Maclaurin series of the following. If the power series were derived in your study materials, derive them again with ``closed textbook''. 
\begin{enumerate}
\item $\displaystyle \Arctan x $.
\item $\sqrt{1+x^2}$.
\item $\arccos (2x)$.
\item $\displaystyle \ln (1+x)$.
\item $\displaystyle \ln (1-x)$.
\item $\ln (1+x^2)$.
\item $e^x$ (for this example just recall the power series, do not derive them).
\item $e^{-x^2}$.
\item $xe^{-x^2}$.
\item $\sin x$ (for this example just recall the power series, do not derive them).
\item $\cos x$ (for this example just recall the power series, do not derive them).
\end{enumerate}


\item \textbf{(This problem is of higher difficulty, it will not appear on the quiz.) }
Let $f(x)$ be defined as 
\[
f(x):=\doublebrace{e^{-\frac{1}{x^2}}}{\mathrm{if~} x>0}{0}{\mathrm{otherwise.}}
\]
\begin{enumerate}[ref={\fcProblemRef}]
\item Prove that if $R(x)$ is an arbitrary rational function, 
\[
\lim\limits_{\substack{x\to 0\\ x>0}} R(x)e^{-\frac{1}{x^2}}=0
\]
\item Prove that $f(x)$ is differentiable at $0$ and $f'(0)=0$.
\item Prove that the Maclaurin series of $f(x)$ are 0 (but $f(x)$ is clearly a non-zero function).
\end{enumerate}
\homeworkEnd
\end{comment}

\begin{comment}
\homeworkStart{on Lectures 13,14, will not be quizzed}{}
\item Match the graphs of the parametric equations $x=f(t)$, $y=g(t)$ with the graph of the parametric curve $ \gamma \left| \begin{array}{rcl}x&=&f(t)\\y&=&g(t) \end{array}\right.$
\psset{xunit=0.5cm, yunit=0.5cm, algebraic=true}
\begin{multicols}{2}
\begin{enumerate}
\item 
\begin{pspicture}(-0.2, -1.2)(2.2,1.2)
\tiny
\psaxesStandard{-0.5}{-1.2}{2.2}{1.2}
\psXTickWithLabel{1}{$1$}
\parametricplot[linecolor=\psColorGraph, plotpoints=500]{-3.14159}{3.14159}{sin(t)+1|sin(2*t)}
\end{pspicture}
\answer{matches to \ref{itemMatchx=1+sin(t),y=sin(2t)}}

\item 
\begin{pspicture}(-1.5, -1.5)(1.5,1.5)
\tiny
\psaxesStandard{-1.4}{-1.4}{1.4}{1.4}
\psXTickWithLabel{1}{$1$}
\parametricplot[linecolor=\psColorGraph, plotpoints=500]{-3.14159}{3.14159}{sin(2*t)|sin(3*t)}
\end{pspicture}
\answer{matches to \ref{itemMatchx=sin2t,y=sin3t}}

\item 
\begin{pspicture}(-2.7, -2.7)(2.7,2.7)
\tiny
\psaxesStandard{-2.5}{-0.5}{2.5}{2.5}
\psXTickWithLabel{1}{$1$}
\parametricplot[linecolor=\psColorGraph,  plotpoints=500]{-2}{2}{t*(t-1.75)*(t+1.75)|sqrt(4-t*t) }
\end{pspicture}
\answer{matches to \ref{itemMatchx=cubic,y=sqrt}}
\end{enumerate}

\columnbreak
\begin{enumerate}
\item \label{itemMatchx=sin2t,y=sin3t}
\begin{pspicture}(-3.3, -1.2)(3.3,1.2)
\tiny
\psaxes[ticks=none, labels=none, arrows=<->](0,0)(-3.2, -1.1)(3.2, 1.1)
\psLabels[$t$][$x$]{3.2}{1.1}
\psXTickWithLabel{1}{$1$}
\psplot[linecolor=blue, plotpoints=500]{-3.14159}{3.14159}{sin(2*x)}
\end{pspicture}
\begin{pspicture}(-3.3, -1.2)(3.3,1.2)
\tiny 
\psaxes[ticks=none, labels=none, arrows=<->](0,0)(-3.2, -1.1)(3.2, 1.1)
\psLabels[$t$][$y$]{3.2}{1.1}
\psXTickWithLabel{1}{$1$}
\psplot[linecolor=blue, plotpoints=500]{-3.14159}{3.14159}{sin(3*x)}
\end{pspicture}
\item \label{itemMatchx=cubic,y=sqrt}
\begin{pspicture}(-2.7, -2.7)(2.7,2.7)
\tiny
\psaxes[ticks=none, labels=none, arrows=<->](0,0)(-2.5, -2.5)(2.5, 2.5)
\psLabels[$t$][$x$]{2.5}{2.5}
\psXTickWithLabel{1}{$1$}
\psplot[linecolor=blue, plotpoints=500]{-2}{2}{x*(x-1.75)*(x+1.75)}
\end{pspicture}
\begin{pspicture}(-2.3, -0.5)(2.3,2,3)
\psaxes[ticks=none, labels=none, arrows=<->](0,0)(-2.2, -0.5)(2.2, 2.2)
\psLabels[$t$][$y$]{2.2}{2.2}
\psXTickWithLabel{1}{$1$}
\psplot[linecolor=blue, plotpoints=500]{-2}{2}{sqrt(4-x*x)}
\end{pspicture}

\item \label{itemMatchx=1+sin(t),y=sin(2t)}
\begin{pspicture}(-3.3, -0.5)(3.3,2.2)
\psaxes[ticks=none, labels=none, arrows=<->](0,0)(-3.2, -0.5)(3.2, 2.5)
\psLabels[$t$][$x$]{3.2}{2.5}
\psXTickWithLabel{1}{$1$}
\psplot[linecolor=blue, plotpoints=500]{-3.14159}{3.14159}{sin(x)+1}
\end{pspicture}
\begin{pspicture}(-3.3, -1.2)(3.3,1.2)
\psaxes[ticks=none, labels=none, arrows=<->](0,0)(-3.2, -1.1)(3.2, 1.1)
\psLabels[$t$][$y$]{3.2}{1.1}
\psXTickWithLabel{1}{$1$}
\psplot[linecolor=blue, plotpoints=500]{-3.14159}{3.14159}{sin(2*x)}
\end{pspicture}


\end{enumerate}
\end{multicols}
\item Match the graph of the curve to its graph in polar coordinates and to its polar parametric equations.
\psset{xunit=0.5cm, yunit=0.5cm}
\begin{multicols}{3}
\begin{enumerate}
\item \begin{pspicture}(-2.323256, -3.101443)(3.400000,3.201443) 
\tiny 
\psaxesStandard{-2.073256}{-2.851443}{3.150000}{2.851443}
%Calculator command: drawPolarExtended{}(\cos{}(3 t)+2, 0, 2 \pi) 
\parametricplot[linecolor=\psColorGraph, plotpoints=1000, algebraic=false]{0}{6.28319}{2.0000000 t 3.0000000 mul 57.29578 mul cos add t 57.29578 mul cos mul 2.0000000 t 3.0000000 mul 57.29578 mul cos add t 57.29578 mul sin mul }
\end{pspicture} 
\answer{matches \ref{itemMatchPolarGraph,r=2+cos(3*t)}, \ref{itemMatchPolarFormula,r=2+cos(3*t)}}
\item 
\begin{pspicture}(-3.269100, -2.893230)(3.268991,3.499862) 
\tiny 
\psaxesStandard{-3.019100}{-2.643230}{3.018991}{3.149862}
%Calculator command: drawPolarExtended{}(\sin{}(5 t)+2, 0, 2 \pi) 
\parametricplot[linecolor=\psColorGraph, plotpoints=1000, algebraic=false]{0}{6.28319}{2.0000000 t 5.0000000 mul 57.29578 mul sin add t 57.29578 mul cos mul 2.0000000 t 5.0000000 mul 57.29578 mul sin add t 57.29578 mul sin mul }
\end{pspicture} 
\answer{matches \ref{itemMatchPolarGraph,r=2+sin(5t)}, \ref{itemMatchPolarFormula,r=2+sin(5t)}}
\item 
\begin{pspicture}(-3.515530, -0.900000)(3.541593,2.319691) 
\tiny 
\psaxesStandard{-3.265530}{-0.650000}{3.291593}{1.969691}
\parametricplot[linecolor=\psColorGraph, plotpoints=1000, algebraic=false]{-3.14159}{3.14159}{t t 57.29578 mul cos mul t t 57.29578 mul sin mul }
\end{pspicture} 
\answer{matches \ref{itemMatchPolarGraph,r=t}, \ref{itemMatchPolarFormula,r=t} }
\item 
\begin{pspicture}(-0.962477, -1.280083)(1.400000,1.380083) 

\tiny 
\psaxesStandard{-0.712477}{-1.030083}{1.150000}{1.030083}
%Calculator command: drawPolarExtended{}(\cos{}(3 t), 0, 2 \pi) 
\parametricplot[linecolor=\psColorGraph, plotpoints=1000, algebraic=false]{0}{6.28319}{t 3.0000000 mul 57.29578 mul cos t 57.29578 mul cos mul t 3.0000000 mul 57.29578 mul cos t 57.29578 mul sin mul }

\end{pspicture}
\answer{matches \ref{itemMatchPolarGraph,r=cos(3t)}, \ref{itemMatchPolarFormula,r=cos(3t)}}

\item 
\begin{pspicture}(-1.122338, -1.122308)(2.550479,2.650486) 
\tiny 
\psaxesStandard{-0.872338}{-0.872308}{2.300479}{2.300486}
%Calculator command: drawPolarExtended{}(\sin{}t+\cos{}t+1, 0, 2 \pi) 
\parametricplot[linecolor=\psColorGraph, plotpoints=1000, algebraic=false]{0}{6.28319}{1.0000000 t 57.29578 mul cos add t 57.29578 mul sin add t 57.29578 mul cos mul 1.0000000 t 57.29578 mul cos add t 57.29578 mul sin add t 57.29578 mul sin mul }
\end{pspicture} 
\answer{matches \ref{itemMatchPolarGraph,r=1+sin(t)+cos(t)}, \ref{itemMatchPolarFormula,r=1+sin(t)+cos(t)}}

\item 
\begin{pspicture}(-1.729462, -1.765832)(1.787465,1.792057) 
\tiny 
\psaxesStandard{-1.479462}{-1.515832}{1.537465}{1.442057}
%Calculator command: drawPolarExtended{}(1/4 \sqrt{t}, 0, 10 \pi) 
\parametricplot[linecolor=\psColorGraph, plotpoints=1000, algebraic=false]{0}{31.4159}{t sqrt 0.2500000 mul t 57.29578 mul cos mul t sqrt 0.2500000 mul t 57.29578 mul sin mul }
\end{pspicture}
\answer{matches \ref{itemMatchPolarGraph,r=sqrt(t)}, \ref{itemMatchPolarFormula,r=1/4sqrt(t)}}

\end{enumerate}

\columnbreak
\begin{enumerate}

\item \label{itemMatchPolarGraph,r=sqrt(t)}
\begin{pspicture}(-0.900000, -0.900000)(5.9,1.898448) 
\tiny 
\psaxesStandard{-0.650000}{-0.650000}{5.565927}{1.548448}
\parametricplot[linecolor=\psColorTangent, plotpoints=1000, algebraic=false]{0}{5.9}{t t sqrt 0.2500000 mul }
\end{pspicture} 

\item \label{itemMatchPolarGraph,r=2+cos(3*t)}
\begin{pspicture}(-0.900000, -0.900000)(6.683185,3.500000) 
\tiny 
\psaxesStandard{-0.650000}{-0.650000}{6.433185}{3.150000}
\parametricplot[linecolor=\psColorTangent, plotpoints=1000, algebraic=false]{0}{6.28319}{t 2.0000000 t 3.0000000 mul 57.29578 mul cos add }
\end{pspicture} 

\item \label{itemMatchPolarGraph,r=2+sin(5t)}
\begin{pspicture}(-0.900000, -0.900000)(6.683185,3.499995) 
\tiny 
\psaxesStandard{-0.650000}{-0.650000}{6.433185}{3.149995}
\parametricplot[linecolor=\psColorTangent, plotpoints=1000, algebraic=false]{0}{6.28319}{t 2.0000000 t 5.0000000 mul 57.29578 mul sin add }
\end{pspicture} 

\item \label{itemMatchPolarGraph,r=1+sin(t)+cos(t)}
\begin{pspicture}(-0.900000, -0.900000)(6.683185,2.914198) 
\tiny 
\psaxesStandard{-0.650000}{-0.650000}{6.433185}{2.564198}
\parametricplot[linecolor=\psColorTangent, plotpoints=1000, algebraic=false]{0}{6.28319}{t 1.0000000 t 57.29578 mul cos add t 57.29578 mul sin add }
\end{pspicture} 
\item \label{itemMatchPolarGraph,r=t}
\begin{pspicture}(-3.541593, -3.541593)(3.541593,3.616510) 
\tiny 
\psaxesStandard{-3.291593}{-3.291593}{3.291593}{3.266510}\parametricplot[linecolor=\psColorTangent, plotpoints=1000, algebraic=false]{-3.14159}{3.14159}{t t}
\end{pspicture} 

\item \label{itemMatchPolarGraph,r=cos(3t)}
\begin{pspicture}(-0.900000, -1.399823)(6.683185,1.500000) 
\tiny 
\psaxesStandard{-0.650000}{-1.149823}{6.433185}{1.150000}
\parametricplot[linecolor=\psColorTangent, plotpoints=1000, algebraic=false] {0}{6.28319}{t t 3.0000000 mul 57.29578 mul cos }

\end{pspicture} 


\end{enumerate}
\columnbreak
\renewcommand\theenumii{\roman{enumii}}
\begin{enumerate}
\item \label{itemMatchPolarFormula,r=1+sin(t)+cos(t)} $r=1+\sin(\theta)+cos(\theta)$
\item \label{itemMatchPolarFormula,r=t} $r= \theta, \theta\in [-\pi, \pi]$.
\item \label{itemMatchPolarFormula,r=cos(3t)} $r= \cos(3\theta), \theta\in [0, 2\pi]$.
\item \label{itemMatchPolarFormula,r=1/4sqrt(t)}
$r=\frac{1}4\sqrt{\theta}, \theta\in [0, 10\pi]$.
\item \label{itemMatchPolarFormula,r=2+sin(5t)} $r=2+\sin (5\theta) $.
\item \label{itemMatchPolarFormula,r=2+cos(3*t)} $r=2+cos(3\theta)$.
\end{enumerate}
\end{multicols}
\homeworkEnd
\end{comment}
\begin{comment}
\homeworkStart{on Lecture 15, will be quizzed Monday April 7}{}
\item Find the values of the parameter $t$ for which the curve has horizontal and vertical tangents.
\begin{multicols}{2}
\begin{enumerate}
\item $y=t^2-t+1$, $x=t^2+t-1$

\psset{xunit=0.25cm, yunit=0.25cm}
\begin{pspicture}(-0.9, -1.65)(13.4,11.416228)
\tiny
\fcAxesStandard{-0.65}{-1.4}{13.15}{11.066228}

%Calculator input: plotCurve{}(t^{2}- t+1, t^{2}+t-1, -3, 3)
\parametricplot[linecolor=\fcColorGraph, plotpoints=1000]{-3}{3}{ 1 t -1 mul add t 2 exp add -1 t add t 2 exp add }
\end{pspicture}
\item $x=t^3-t^2-t+1$, $y=t^2-t-1 $.

\psset{xunit=1cm, yunit=1cm}
\begin{pspicture}(-0.9, -1.649998)(3.358221,1.5)
\tiny
\fcAxesStandard{-0.65}{-1.399998}{3.108221}{1.15}

%Calculator input: plotCurve{}(t^{3}- t^{2}- t+1, t^{2}- t-1, -1, 2)
\parametricplot[linecolor=\fcColorGraph, plotpoints=1000]{-1}{2}{ 1 t -1 mul add t 2 exp -1 mul add t 3 exp add -1 t -1 mul add t 2 exp add }
\end{pspicture}
\item $x=\cos (t)$, $y=\sin (3t)$

\psset{xunit=1cm, yunit=1cm}
\begin{pspicture}(-1.4, -1.399999)(1.4,1.499999)
\tiny
\fcAxesStandard{-1.15}{-1.149999}{1.15}{1.149999}

%Calculator input: plotCurve{}(\cos{}t, \sin{}(3 t), - \pi, \pi)
\parametricplot[linecolor=\fcColorGraph, plotpoints=1000]{-3.14159}{3.14159}{t 57.29578 mul cos t 3 mul 57.29578 mul sin }
\end{pspicture}
\item $x=\cos (t)+\sin (t)$ , $y=\sin (t)$.

\psset{xunit=1cm, yunit=1cm}
\begin{pspicture}(-1.814213, -1.399999)(1.81421,1.499999)
\tiny
\fcAxesStandard{-1.564213}{-1.149999}{1.56421}{1.149999}

%Calculator input: plotCurve{}(\sin{}t+\cos{}t, \sin{}t, - \pi, \pi)
\parametricplot[linecolor=\fcColorGraph, plotpoints=1000]{-3.14159}{3.14159}{t 57.29578 mul cos t 57.29578 mul sin add t 57.29578 mul sin }
\end{pspicture}
\end{enumerate}
\end{multicols}

\item Show that the parametric curve has multiple tangents at the point and find their slopes.
\begin{multicols}{2}
\begin{enumerate}
\item $x=\cos t$, $y=2\sin (2t)$, two tangents at $(x,y)=(0,0)$.

\psset{xunit=1cm, yunit=1cm}
\begin{pspicture}(-1.4, -2.399998)(1.4,2.499998)
\tiny
\fcAxesStandard{-1.15}{-2.149998}{1.15}{2.149998}
%Calculator input: plotCurve{}(\cos{}t, 2 \sin{}(2 t), - \pi, \pi)
\parametricplot[linecolor=\fcColorGraph, plotpoints=1000]{-3.14159}{3.14159}{t 57.29578 mul cos t 2 mul 57.29578 mul sin 2 mul }
\end{pspicture}
\item $x=\cos t \sin (3t)$, $y=\sin(t)\sin (3t)$, six tangents at $(x,y)=(0,0)$.
\psset{xunit=1cm, yunit=1cm}
\begin{pspicture}(-1.280086, -1.399988)(1.4,1.0625)
\tiny
\fcAxesStandard{-1.030086}{-1.149988}{1.15}{0.7125}
%Calculator input: plotCurve{}(\cos{}t \sin{}(3 t), \sin{}t \sin{}(3 t), -2 \pi, 2 \pi)
\parametricplot[linecolor=\fcColorGraph, plotpoints=1000]{-6.28319}{6.28319}{t 3 mul 57.29578 mul sin t 57.29578 mul cos mul t 3 mul 57.29578 mul sin t 57.29578 mul sin mul }
\end{pspicture}
\item $x=\cos t, y=\sin (3t)$, find the two points at which the curve has double tangent and find the slopes of both pairs of tangents.
\psset{xunit=1cm, yunit=1cm}
\begin{pspicture}(-1.399995, -1.399999)(1.4,1.499999)
\tiny
\fcAxesStandard{-1.149995}{-1.149999}{1.15}{1.149999}

%Calculator input: plotCurve{}(\cos{}t, \sin{}(3 t), -2 \pi, 2 \pi)
\parametricplot[linecolor=\fcColorGraph, plotpoints=1000]{-6.28319}{6.28319}{t 57.29578 mul cos t 3 mul 57.29578 mul sin }
\end{pspicture}
\item $x=t^3-t^2-t+1$, $y=t^2-t-1 $, find a point where the curve has double tangent and find the slopes of the tangents.

\psset{xunit=1cm, yunit=1cm}
\begin{pspicture}(-0.9, -1.649998)(3.358221,1.5)
\tiny
\fcAxesStandard{-0.65}{-1.399998}{3.108221}{1.15}
%Calculator input: plotCurve{}(t^{3}- t^{2}- t+1, t^{2}- t-1, -1, 2)
\parametricplot[linecolor=\fcColorGraph, plotpoints=1000]{-1}{2}{ 1 t -1 mul add t 2 exp -1 mul add t 3 exp add -1 t -1 mul add t 2 exp add }
\end{pspicture}

\end{enumerate}
\end{multicols}

\homeworkEnd

\end{comment}


\begin{comment}
\homeworkStart{on Lecture  \\}{}
\item Find the length of the curve. 
\begin{enumerate}[ref={\fcProblemRef}]

\item \label{problemlengthy=x^2from1to2}
$y=x^2$, $x\in [1, 2]$.

\answer{$L=\sqrt{17}+\frac{1}{4} \log{}\left(\sqrt{17}+4\right)-\frac{1}{4} \log{}\left(\sqrt{5}+2\right)-\frac{\sqrt{5}}{2} \approx 3.167841$}

\item \label{problemlengthy=sqrt(x)from1to2}

$y=\sqrt{x}$, $x\in [1, 2]$.

\answer{$L= \frac{1}{8}\left( 12\sqrt{2} +\ln (17+12\sqrt{2})-4\sqrt{5}-\ln (9+4\sqrt{5})\right) \approx 1.083$}
\item \label{problemlengthx=sqrt(t)-2t,y=8/3t^(3/4)} $\displaystyle x = \sqrt{t} - 2t$ and $\displaystyle y = \frac{8}{3}t^{\frac{3}{4}}$ from $t = 1$ to $t = 4$.

\answer{$L=7$}



\item $\gamma:\left| 
\begin{array}{rcl}
x(t)&=&\frac{1}{t}+\frac{t^3}{3}\\
y(t)&=&2t\\
\end{array}\right., t\in [1,2]\quad . $

\answer{$L=\frac{17}{6}$}
\item  $\gamma:\left| 
\begin{array}{rcl}
x(t)&=&\frac{1}{t}+t\\
y(t)&=&2\ln t\\
\end{array}\right., t\in [1,2]\quad . $

\answer{$L=\frac{3}{2}$}
\item One arch of the cycloid 
\[
\gamma: \left|\begin{array}{rcl}
x(t)&=& t-\sin t  \\
y(t)&=&1-\cos t \\
\end{array} \right., t\in[0,2\pi]
\]

\begin{pspicture}(-0.5,-0.5)(8.1,2.1)
\pstVerb{5 dict begin /pi 3.141592654 def}
\fcAxesStandard{-0.5}{-0.5}{7}{2}
\fcLabels{7}{2}
\parametricplot[linecolor=red]{0}{360}{t 180 div pi mul t sin sub 1 t cos sub}
\pstVerb{end}
\end{pspicture}

\answer{$L=8$}
\item The cardioid

\[
\gamma: \left|\begin{array}{rcl}
x(t)&=&(1+\sin t)\cos t  \\
y(t)&=&(1+\sin t)\sin t \\
\end{array} \right., t\in[0,2\pi]
\]

\begin{pspicture}(-2,-1.1)(2,2.4)
\pstVerb{5 dict begin /pi 3.141592654 def}
\fcAxesStandard{-1.8}{-1}{1.8}{2.3}
\fcLabels{1.8}{2.3}
\parametricplot[linecolor=red]{0}{360}{t sin 1 add t cos mul t sin 1 add t sin mul}
\pstVerb{end}
\end{pspicture}

\answer{$L=8$}

\end{enumerate}

\homeworkEnd
\end{comment}
\begin{comment}
\homeworkStart{on Lecture  18\\will be quizzed: Wednesday April 23}{}
\item 
\begin{enumerate}
\item \label{problemMixingProblem1}
A tank contains 30 kg of salt dissolved in water to form $10000$ liters of solution. Brine that contains $0.05$ kg of salt per liter enters the tank at a rate of 10 liters per minute. The solution is kept thoroughly mixed and drains from the tank at the same rate (10 liters per minute). Determine how much salt remains in the tank after half an hour.
\item \label{problemMixingProblem2} A tank contains $1000$ kg of salt dissolved in water to form $10000$ liters of solution. Brine that contains $0.01$ kg of salt per liter of water enters the tank at a rate of $30$ liters per minute. The solution is kept thoroughly mixed and drains from the tank at the same rate ($30$ liters per minute). 
\begin{enumerate} 
\item Determine how much salt remains in the tank after an hour. 
\item How long should the procedure continue  so that the solution in the tank gets to a salt concentration of $0.101$ kg/L? 
\end{enumerate}
\end{enumerate}


\solution{\ref{problemMixingProblem1}


}
\solution{\ref{problemDFQseparable-mixing-problem-1}. Let 
\[
y(t)=\text{salt in the tank after } t \text{ minutes (in kg)}\quad .
\] 
We are given $y(0)= 30$kg, the initial amount of salt. We are looking to find $y(45)$, the amount of salt after $45$ minutes. We have that 
\[
\frac{\diff y}{\diff t}= \text{(rate in)} - \text{(rate~out)} \quad .
\]
The rate of salt entering the tank is constant: 
\[
\text{(rate in)}=0.05 kg/L \cdot 10 L/min= 0.5 kg/min\quad .
\] 
As the solution is thoroughly mixed, at any time the concentration of salt in the tank is 
\[
\displaystyle \frac{y}{ 10000} kg/L.
\] 
Therefore the rate of salt going out of the tank is 
\[
\text{(rate out)}=\frac{y}{10000} kg/L * 10 L/min = \frac{y}{1000} kg/min\quad .
\] 
Therefore the differential equation for the amount of salt in the tank is
\[
\frac{\diff y}{\diff t}= \underbrace{ 0.5}_{\text{(rate  in)}}- \underbrace{ \frac{y}{1000} }_{\text{(rate out)}}\quad .
\]
To find $y(45)$, we integrate from $t=0$ to $t=45$:
\[
{\renewcommand{\arraystretch}{1.5}
\begin{array}{rcll|l}
\displaystyle \int\limits_{t=0}^{45} \frac{1000}{500- y} \underbrace{ \frac{\diff y}{\diff t} \diff t}_{\diff (y(t))} &=& \displaystyle \int\limits_{t=0}^{45} \diff t \\
\displaystyle \int\limits_{t=0}^{t=45} \frac{1000}{500-y(t)}\diff (y(t))&=& 45&&\text{set }z=y(t) \\
\displaystyle -1000 \int \limits_{z=y(0)=30}^{z=y(45)} \frac{1}{500-z}d(500-z)&=& 45 \\
\displaystyle \left. -1000 \ln |500-y| \right]_{y(0)=30}^{y(45)}&=& 45 \\
\displaystyle -1000 \left( \ln |500-y(45)|\right. \\
\left. -\ln |500- 30|  \right) &=& 45 \\
\displaystyle \ln \left| \frac{470}{500-y(45)} \right|  &=& \displaystyle \frac{45}{1000}
\\
\displaystyle \ln \left( \frac{470}{500-y(45)} \right)  &=& \displaystyle \frac{45}{1000} &&\text{see below}
\\
\displaystyle \frac{470}{500-y(30)}&=&\displaystyle e^{\frac{45}{1000}}\\
\displaystyle 500-y(30)&=&\displaystyle  470e^{-\frac{9}{200}}\\
\displaystyle y(30)&=&\displaystyle 500-470e^{-\frac{9}{200}}\\
&\approx& 500-470\cdot 0.955997 \\
&\approx& 50.681184 \quad ,
\end{array}
}
\]
where we have used that $\displaystyle \frac{470}{500-y(t)}>0 $. The fact that $\displaystyle \frac{470}{500-y(t)}>0 $ can be seen as follows. As $500-y(0)=470>0$ and $y(t)$ is continuous, in order to have $500-y(t)<0$ there must exist some $x_1$ for which $y(x_1)=500$. However this is impossible since $\displaystyle x=\ln \left|\frac{470}{500-y(x)}\right|  $. 

As the unit of measurement is $kg$, the final answer to the problem is $\approx 50.68 kg$ salt.
}
\item
Mixing problem. A tank contains $1000$ kg of salt dissolved in 10000 liters of water. Brine that contains $0.05$ kg of salt per liter of water enters the tank at a rate of $30$ liters per minute. The solution is kept thoroughly mixed and drains from the tank at the same rate ($30$ liters per minute). 


\begin{enumerate}
\item Determine how much salt remains in the tank after an hour. The answer key has not been proofread, use with caution.

\answer{$\displaystyle 500+ 500 e^{-0.18}\approx 917.64kg$}
\item Determine how much time will be needed in order to have the concentration of salt in the tank reach $0.0501$kg/liter. The answer key has not been proofread, use with caution.

\answer{$\frac{1000}{3}\ln 500\approx 2071.54min\approx 34.53 hours$}
\end{enumerate}

\item 
\begin{enumerate}

\item \label{problemDFQseparable-yprime=ysquared-1}
\begin{equation}\label{eqDFQseparable-yprime=ysquared-1}
\frac{\diff y}{\diff x}= y^2-1\quad .
\end{equation}
\begin{enumerate}
\item \label{problemDFQseparable-yprime=ysquared-1-part1} Find all solutions of the differential equation above.
\item \label{problemDFQseparable-yprime=ysquared-1-part2} Find a solution for which $y(0)=-\frac{3}{5}$.
\end{enumerate}

\item 
\begin{enumerate}
\item Find the general solution to the differential equation 
\[
\frac{\diff y}{\diff x}= y^2-4\quad .
\]
The drawing below is a computer-generated plot of the direction field  $\displaystyle \frac{dy}{dx}=y^2-4$, you may use it to get a feeling for what your answer should look like.

\begin{pspicture}(-6,-6)(6,6)
\newcommand{\Dconst}{4}
\psplot[linecolor=green]{-4}{4}{1 \Dconst\space 2.718281828 4 x mul exp mul sub 1 \Dconst\space 2.718281828 4 x mul exp mul add div 2 mul} 

\psaxes{<->}(0,0)(-6,-2)(6,6)
\rput (5,5){The direction field  $\frac{dy}{dx}=y^2-4$}
  \psset{arrows=->}
  \multido{\ra=-4+0.5}{17}{%
    \multido{\rb=-4+0.5}{17}{%
      \pstVerb{/xC \ra\space def
               /yC \rb\space def
               /F  yC yC mul 4 sub \space def
}
%\psline[linecolor=blue](! xC  yC )(! xC yC)
\psdot[linecolor=red!60](! xC yC)
\psline[linecolor=blue](! xC F ATAN 57.295 mul cos 0.2 mul sub yC F ATAN 57.295 mul sin 0.2 mul sub)(! xC F ATAN 57.295 mul cos 0.2 mul add yC F ATAN 57.295 mul sin 0.2 mul add )
}}
\end{pspicture}

\item  Find a solution of the above equation for which $ y(0)= -\frac{6}{5}$. 
\end{enumerate}
\end{enumerate}

\solution{\ref{problemDFQseparable-yprime=ysquared-1}
(\ref{problemDFQseparable-yprime=ysquared-1-part1}). We proceed as explained in the theory.

\noindent Case 1. Suppose there exists a number $x_0$ such that $(y(x_0) )^2 - 1\neq 0$. Since $y$ is a differentiable function of $x$, it is also continuous. Therefore for some $t$ sufficiently close to $x_0$, all numbers $x$ in the interval between $t$ and $x_0$ satisfy $ y(x)^2-1\neq 0$.
\[
\begin{array}{rcll|l}
\displaystyle \frac{\frac{ \diff y}{ \diff x}}{y^2-1}&=&1 \\
\displaystyle\int\limits_{x=x_0}^{x=t} \frac{1}{ y^2- 1} \underbrace{ \frac{\diff y}{\diff x}\diff x}_{=\diff (y(x)) }&=&\displaystyle\int\limits_{x=x_0}^{x=t}\diff x &&\text{can integrate as }  y(x)^2-1\neq 0\\
\displaystyle\int\limits_{t=x_0 }^{x=t} \frac{\diff (y(x))}{ (y(x))^2-1}& =& \displaystyle\left.x \right|_{ x=x_0}^{x=t} &&\text{set } z=y(x)\\
\displaystyle\int\limits_{z=y(x_0) }^{z=y(t)} \frac{\diff z}{ z^2-1}& =& \displaystyle t-x_0 \\
\displaystyle\int\limits_{z=y(x_0)}^{z=y(t)} \left(\frac{\frac12 }{z-1}- \frac{\frac12}{z+1}\right)\diff z&=& t-x_0\\
\displaystyle\left .\frac{1}2 \ln \left|\frac{z-1}{z+1}\right|\right]_{z=y(x_0)}^{z=y(t)}&=& t-x_0\\
\displaystyle \ln \left|\frac{y(t)-1}{y(t)+1}\right|&=& 2t - C&&\text{relabel dummy variable } t \text { to } x \\
\displaystyle
\ln \left|\frac{y(x)-1}{y(x)+1}\right|&=& 2x - C
\end{array}
\]
where we have set 
\[\displaystyle C=2x_0-  \ln \left|\frac{y(x_0)-1}{y(x_0)+1}\right| \quad .
\] 
Set  
\[
D:=e^{-C}\quad .
\] 
By the assumption of our case, $ (y(x_0))^2-1\neq 0$, so there are two remaining cases: $ (y(x_0))^2-1>0$ and $ (y(x_0))^2-1<0$.

\noindent Case 1.1. Suppose $\displaystyle \frac{y(x_0)-1}{ y(x_0)+1}>0$. As the function $y(x)$ is differentiable, it is also continuous. Therefore $\displaystyle \frac{y(x)-1}{y(x)+1}>0$ for all $x$ near $x_0$. Then we can remove the absolute values around from the equality above to get that for all $x$ close to $x_0$ we have that
\[
\begin{array}{rcl}
\displaystyle \ln \left(\frac{y(x)-1}{y(x)+1}\right)&=& 2x - C\\
\displaystyle \frac{y(x)-1}{y(x)+1}&=& D e^{2x}\\
\displaystyle y(x)-1&=&\displaystyle  De^{2x}(y(x)+1)\\
\displaystyle y(x)\left(1- De^{2x}\right)&=&\displaystyle  De^{2x}+1\\
\displaystyle y(x)&=&\displaystyle  \frac{ 1+De^{2x}}{1- De^{2x}}\quad .\\
\end{array}
\]
The solution $y(x)$ given above satisfies $\displaystyle \frac{y(x)-1}{y(x)+1}= De^{2x}$ for all $x$. As $D>0$, this implies that $\displaystyle \frac{y(x)-1}{ y(x)+1}>0$. Therefore the considerations above are valid for all $x$, rather than only for those $x$ near $x_0$. Therefore our first case yields the solution
\[
y(x)=\frac{ 1+De^{2x}}{1- De^{2x}}\quad .
\]

\noindent Case 1.2. Suppose  $\displaystyle \frac{y(x_0) -1}{y(x_0) +1} <0$. Then for all $x$ near $x_0$ we get $\displaystyle \ln \left| \frac{y(x) -1}{y(x) +1}\right|= \ln \left( \frac{ 1- y(x) }{ y( x) +1}\right)$ and, similarly to Case 1, we get 
\[
\begin{array}{rcl}
\displaystyle \frac{1-y(x)}{y(x)+1}&=& D e^{2x}\\
1-y(x)&=& De^{2x}(y(x)+1)\\
y(x)\left(1+ De^{2x}\right)&=& 1-De^{2x}\\
y(x)&=&\displaystyle \frac{1- De^{2x}}{1+ De^{2x}}\quad .
\end{array}
\]
Since $D$ is a positive constant, we conclude in a fashion analogous to Case 1 that $y(x)<0$ for all $ x$.

Case 2.  Suppose $\displaystyle  (y(x_0))^2-1=0 $.  Then $y(x_0)=\pm 1$. Clearly the constant functions $y(x)= \pm 1$ are two solutions: if we can plug back $y=\pm 1$ in the original equation we get that $\frac{\diff y}{\diff x}= 0$ and $y$ is a constant function of $x$. From the preceding two cases we know that if $\frac{y(x) -1}{y(x) +1}$ is defined and not equal to zero for some value of $x$, then $\frac{y(x)-1}{y(x)+1}$ is defined and not equal to zero for all values of $x$. Therefore the present case yields only two solutions, the constant functions $y(x)=\pm 1$. 

Our final answer is 
\[
y(x)= \frac{1+De^{2x}}{1-De^{2x}} \quad \text{ or }\quad y(x)=0,
\]
where $D$ is an arbitrary rea(l number. Notice that in the above answer, we have combined Cases 1.1, 1.2 and the case $y(x)=1 $: by allowing $D$ to be negative we included Case 1.2 and by allowing $D $ to be zero we included the case $y(x)=1$. Finally, we note that if we let $D\to \infty$, the solution $y(x) = \frac{1+De^{2x }}{ 1- De^{2x}}  $ tends to the solution $y(x)=-1$ (for all values of $x$).

We may plot solutions for a few values of $D$ as follows. We overlay the solutions on top of the direction field of the differential equation. The picture tells us a lot about the properties of the solutions of the differential equations. 
%\optionalDisplay{
\begin{pspicture}(-6,-6)(6,6)
\directionField{}

\newcommand{\Dconst}{1}
\psplot[linecolor=green]{-4}{4}{1 \Dconst\space 2.718281828 2 x mul exp mul sub 1 \Dconst\space 2.718281828 2 x mul exp mul add div} 
\renewcommand{\Dconst}{0.25}
\psplot[linecolor=green]{-4}{4}{1 \Dconst\space 2.718281828 2 x mul exp mul sub 1 \Dconst\space 2.718281828 2 x mul exp mul add div} 
\renewcommand{\Dconst}{4}
\psplot[linecolor=green]{-4}{4}{1 \Dconst\space 2.718281828 2 x mul exp mul sub 1 \Dconst\space 2.718281828 2 x mul exp mul add div} 
\rput[l](5,2 ){$\frac{1- \frac{1}4 e^{2x}}{1+\frac 14 e^{2x}}$ }
\rput[l](5,0.5 ){$\frac{1- e^{2x}}{1+e^{2x}}$ }
\rput[l](5,-2 ){$\frac{1- 4e^{2x}}{1+4e^{2x}}$ }
\psline[arrows=->, linestyle=dotted](5,2)(0,0.6)
\psline[arrows=->, linestyle=dotted](5,0.5)(0,0)
\psline[arrows=->, linestyle=dotted](5,-2)(0,-0.6)

\renewcommand{\Dconst}{1}
\psplot[linecolor=green]{-4}{-0.17}{1 \Dconst\space 2.718281828 2 x mul exp mul add 1 \Dconst\space 2.718281828 2 x mul exp mul sub  div} 
\psplot[linecolor=green]{0.17}{4}{1 \Dconst\space 2.718281828 2 x mul exp mul add 1 \Dconst\space 2.718281828 2 x mul exp mul sub  div} 
\renewcommand{\Dconst}{4}

\psplot[linecolor=green]{-4}{-0.863147181}{1 \Dconst\space 2.718281828 2 x mul exp mul add 1 \Dconst\space 2.718281828 2 x mul exp mul sub  div} 
\psplot[linecolor=green]{-0.523147181}{4}{1 \Dconst\space 2.718281828 2 x mul exp mul add 1 \Dconst\space 2.718281828 2 x mul exp mul sub  div} 

\renewcommand{\Dconst}{0.25}
\psplot[linecolor=green]{-4}{0.523147181}{1 \Dconst\space 2.718281828 2 x mul exp mul add 1 \Dconst\space 2.718281828 2 x mul exp mul sub  div} 
\psplot[linecolor=green]{0.863147181}{4}{1 \Dconst\space 2.718281828 2 x mul exp mul add 1 \Dconst\space 2.718281828 2 x mul exp mul sub  div} 
\rput[r](-5,0.5 ){$\frac{1+\frac 14 e^{2x}}{1- \frac{1}4 e^{2x}}$ }
\rput[r](-5,2 ){$\frac{1+e^{2x}}{1- e^{2x}}$ }
\rput[r](-5,-2 ){$\frac{1+4e^{2x}}{1- 4e^{2x}}$ }
\psline[arrows=->, linestyle=dotted](-5,0.5)(0,1.6667)
\psline[arrows=->, linestyle=dotted](-5,0.5)(1,-3.360539267)
\psline[arrows=->, linestyle=dotted](-5,2)(-0.2,5.066489563)
\psline[arrows=->, linestyle=dotted](-5,2)(0.2,-5.066489563)
\psline[arrows=->, linestyle=dotted](-5,-2)(0,-1.6667)
\psline[arrows=->, linestyle=dotted](-5,-2)(-1,3.360539267)
\psaxes{<->}(0,0)(-4.5,-4.5)(4.5,4.5)
\rput (5,5){The direction field  $\frac{\diff y}{\diff x}=y^2-1$}
  \psset{arrows=->}
  \multido{\ra=-4+0.5}{17}{%
    \multido{\rb=-4+0.5}{17}{%
      \pstVerb{/xC \ra\space def
               /yC \rb\space def
               /F  yC yC mul 1 sub \space def
}
%\psline[linecolor=blue](! xC  yC )(! xC yC)
\psdot[linecolor=red!60](! xC yC)
\psline[linecolor=blue](! xC F ATAN 57.295 mul cos 0.2 mul sub yC F ATAN 57.295 mul sin 0.2 mul sub)(! xC F ATAN 57.295 mul cos 0.2 mul add yC F ATAN 57.295 mul sin 0.2 mul add )
}}
\end{pspicture}
%} %optionalDisplay

\noindent \ref{problemDFQseparable-yprime=ysquared-1} (\ref{problemDFQseparable-yprime=ysquared-1-part1}).
From the computer generated picture above, we may visually estimate that $y(x)=\frac{1-4 e^{2x} }{1+4 e^{2x} }$ intersects the $x$-axis at $(0, -\frac 35)$. Furthermore, we may and check directly that for 
\[
y(x)=\frac{1-4 e^{2x} }{1+4 e^{2x} }
\] 
we have $y(0)= \frac{1-4}{1+5}= \frac{-3}{5}= -\frac 35$ and that is a solution to our problem (this however does not prove the solution is unique). 

Alternatively, let us give an algebraic solution. As we are given that $y(0)=-\frac35$ and so 
\[
\begin{array}{rcl}
\displaystyle -\frac35&=&\displaystyle y(0)= \frac{1-De^{2\cdot 0}}{1+ De^{2\cdot 0}}= \frac{1-D}{1+D}\\
\displaystyle -\frac{3}{5} (1+D)&=&1-D\\
\displaystyle \frac{2}{5} D&=&\displaystyle \frac{8}{5}\\
D&=&4\quad ,
\end{array}
\] 
and $D=4$ is our final answer.
}

\solution{\noindent \ref{problemDFQseparable-yprime=ysquared-1-part1}.
There are two variants for solving this problem. The first variant uses indefinite integration and is slightly informal, but easier to apply and remember. The second variant is more rigorous but more difficult to write up. Both solutions are acceptable for full credit in a Calculus exam. Variant I is recommended when taking exams and Variant II is recommended when writing scientific texts.

\textbf{Variant I}

\renewcommand{\arraystretch}{2}
\[
\begin{array}{rcll|l}
\displaystyle \frac{ \diff y}{ \diff x}&=&y^2-1 && \text{Suppose } y^2-1\neq 0 \\
\displaystyle \frac{\frac{ \diff y}{ \diff x}}{y^2-1}&=&1 \\
\displaystyle\int\frac{1}{ y^2- 1} \underbrace{ \frac{\diff y}{\diff x}\diff x}_{=\diff y } &=& \displaystyle\int \limits \diff x \\
\displaystyle\int \frac{\diff y}{ y^2-1}& =& \displaystyle x +C \\
\displaystyle\int \left(\frac{\frac{1}{2} }{y-1}- \frac{\frac{1 }{2}}{ y+1}\right)\diff y&=& x +C \\
\displaystyle \frac{1}2 \ln \left|\frac{y-1}{y+1}\right| &=& x+C\\
\displaystyle \ln \left|\frac{y-1}{y+1}\right|&=& 2 x + 2C\\
\displaystyle\left|\frac{y-1}{y+1}\right|&=& e^{ 2x +2C} \\
\displaystyle\frac{y-1}{y+1}&=& \pm e^{ 2x +2C} \\
\displaystyle y-1&=&\displaystyle \pm e^{2x+2C} (y+1)\\
\displaystyle y(1\mp e^{2x+2C})&=&\displaystyle 1\pm e^{2x+2C} \\
\displaystyle y&=&\displaystyle \frac{1\pm e^{2x+2C}}{1\mp e^{2x+2C}}  \\
\displaystyle y&=&\displaystyle \frac{1\pm e^{2C} e^{2x}}{1\mp e^{2C}e^{2x}}&&\text{Set }D=\pm e^{2C}\\
\displaystyle y&=&\displaystyle  \frac{1+D e^{2x}}{1- De^{2x}}\quad .
\end{array}
\]
The above solution works on condition that $y^2-1\neq 0$. So the only case not covered is that of $y^2-1=0$, which yields the two solutions $y=\pm 1$.

Our final answer is
\[
y(x)= \frac{1+De^{2x}}{1-De^{2x}} \quad \text{ or }\quad y(x)=-1,
\]
where $D$ is an arbitrary real number. Notice that in the above answer, by allowing $D=0$, we have covered the case $y(x)=1 $. Finally, we note that if we let $D\to \infty$, the solution $y(x) = \frac{1+De^{2x }}{ 1- De^{2x}}  $ tends to the solution $y(x)=-1$ (here we fix a value of $x$ before we let $D\to \infty$).


\textbf{Variant II}

\noindent Case 1. Suppose there exists a number $x_0$ such that $(y(x_0) )^2 - 1\neq 0$. Since $y$ is a differentiable function of $x$, it is also continuous. Therefore for some $t$ sufficiently close to $x_0$, all numbers $x$ in the interval between $t$ and $x_0$ satisfy $ y(x)^2-1\neq 0$.
\[
\begin{array}{rcll|l}
\displaystyle \frac{\frac{ \diff y}{ \diff x}}{y^2-1}&=&1 \\
\displaystyle\int\limits_{x=x_0}^{x=t} \frac{1}{ y^2- 1} \underbrace{ \frac{\diff y}{\diff x}\diff x}_{=\diff (y(x)) }&=&\displaystyle\int\limits_{x=x_0}^{x=t}\diff x &&\text{can integrate as }  y(x)^2-1\neq 0\\
\displaystyle\int\limits_{t=x_0 }^{x=t} \frac{\diff (y(x))}{ (y(x))^2-1}& =& \displaystyle\left.x \right|_{ x=x_0}^{x=t} &&\text{set } z=y(x)\\
\displaystyle\int\limits_{z=y(x_0) }^{z=y(t)} \frac{\diff z}{ z^2-1}& =& \displaystyle t-x_0 \\
\displaystyle\int\limits_{z=y(x_0)}^{z=y(t)} \left(\frac{\frac12 }{z-1}- \frac{\frac12}{z+1}\right)\diff z&=& t-x_0
\\
\displaystyle\left .\frac{1}2 \ln \left|\frac{z-1}{z+1}\right|\right]_{z=y(x_0)}^{z=y(t)}&=& t-x_0 && \text{Set } C=2x_0-  \ln \left|\frac{y(x_0)-1}{ y(x_0)+ 1} \right|\\
\displaystyle \ln \left|\frac{y(t)-1}{y(t)+1}\right|&=& 2t - C&&\text{relabel dummy variable } t \text { to } x \\
\displaystyle
\ln \left|\frac{y(x)-1}{y(x)+1}\right|&=& 2x - C
\end{array}
\]
Set
\[
D=e^{-C}\quad .
\]
By the assumption of our case, $ (y(x_0))^2-1\neq 0$, so there are two remaining cases: $ (y(x_0))^2-1>0$ and $ (y(x_0))^2-1<0$.

\noindent Case 1.1. Suppose $\displaystyle \frac{y(x_0)-1}{ y(x_0)+1}>0$. As the function $y(x)$ is differentiable, it is also continuous. Therefore $\displaystyle \frac{y(x)-1}{y(x)+1}>0$ for all $x$ near $x_0$. Then we can remove the absolute values in the equality above to get that for all $x$ close to $x_0$ we have that
\[
\begin{array}{rcll|l}
\displaystyle \ln \left(\frac{y(x)-1}{y(x)+1}\right)&=& 2x - C&&\text{exponentiate, recall }D=e^{-C}\\
\displaystyle \frac{y(x)-1}{y(x)+1}&=& D e^{2x}\\
\displaystyle y(x)-1&=&\displaystyle  De^{2x}(y(x)+1)\\
\displaystyle y(x)\left(1- De^{2x}\right)&=&\displaystyle  De^{2x}+1\\
\displaystyle y(x)&=&\displaystyle  \frac{ 1+De^{2x}}{1- De^{2x}}\quad .\\
\end{array}
\]
The solution $y(x)$ given above satisfies $\displaystyle \frac{y(x)-1}{y(x)+1}= De^{2x}$ for all $x$. As $D>0$, this implies that $\displaystyle \frac{y(x)-1}{ y(x)+1}>0$. Therefore the considerations above are valid for all $x$, rather than only for those $x$ near $x_0$. Therefore our first case yields the solution
\[
y(x)=\frac{ 1+De^{2x}}{1- De^{2x}}\quad .
\]

\noindent Case 1.2. Suppose  $\displaystyle \frac{y(x_0) -1}{y(x_0) +1} <0$. Then for all $x$ near $x_0$ we get $\displaystyle \ln \left| \frac{y(x) -1}{y(x) +1}\right|= \ln \left( \frac{ 1- y(x) }{ y( x) +1}\right)$ and, similarly to Case 1, we get
\[
\begin{array}{rcl}
\displaystyle \frac{1-y(x)}{y(x)+1}&=& D e^{2x}\\
1-y(x)&=& De^{2x}(y(x)+1)\\
y(x)\left(1+ De^{2x}\right)&=& 1-De^{2x}\\
y(x)&=&\displaystyle \frac{1- De^{2x}}{1+ De^{2x}}\quad .
\end{array}
\]
Since $D$ is a positive constant, we conclude in a fashion analogous to Case 1 that $y(x)<0$ for all $ x$.

Case 2.  Suppose $\displaystyle  (y(x_0))^2-1=0 $.  Then $y(x_0)=\pm 1$. Clearly the constant functions $y(x)= \pm 1$ are two solutions: if we can plug back $y=\pm 1$ in the original equation we get that $\frac{\diff y}{\diff x}= 0$ and $y$ is a constant function of $x$. From the preceding two cases we know that if $\frac{y(x) -1}{y(x) +1}$ is defined and not equal to zero for some value of $x$, then $\frac{y(x)-1}{y(x)+1}$ is defined and not equal to zero for all values of $x$. Therefore the present case yields only two solutions, the constant functions $y(x)=\pm 1$.

Our final answer is
\[
y(x)= \frac{1+De^{2x}}{1-De^{2x}} \quad \text{ or }\quad y(x)=-1,
\]
where $D$ is an arbitrary real number. Notice that in the above answer, we have combined Cases 1.1, 1.2 and the case $y(x)=1 $: by allowing $D$ to be negative we included Case 1.2 and by allowing $D $ to be zero we included the case $y(x)=1$. Finally, we note that if we let $D\to \infty$, the solution $y(x) = \frac{1+De^{2x }}{ 1- De^{2x}}  $ tends to the solution $y(x)=-1$ (for all values of $x$).


\textbf{Solution plots.}

We may plot solutions for a few values of $D$ as follows. We overlay the solutions on top of the direction field of the differential equation. The picture tells us a lot about the properties of the solutions of the differential equations.

\optionalDisplay{
\begin{pspicture}(-6,-6)(6,6)
\newcommand{\Dconst}{1}
\psplot[linecolor=green]{-4}{4}{1 \Dconst\space 2.718281828 2 x mul exp mul sub 1 \Dconst\space 2.718281828 2 x mul exp mul add div}
\renewcommand{\Dconst}{0.25}
\psplot[linecolor=green]{-4}{4}{1 \Dconst\space 2.718281828 2 x mul exp mul sub 1 \Dconst\space 2.718281828 2 x mul exp mul add div}
\renewcommand{\Dconst}{4}
\psplot[linecolor=green]{-4}{4}{1 \Dconst\space 2.718281828 2 x mul exp mul sub 1 \Dconst\space 2.718281828 2 x mul exp mul add div}
\rput[l](5,2 ){$\frac{1- \frac{1}4 e^{2x}}{1+\frac 14 e^{2x}}$ }
\rput[l](5,0.5 ){$\frac{1- e^{2x}}{1+e^{2x}}$ }
\rput[l](5,-2 ){$\frac{1- 4e^{2x}}{1+4e^{2x}}$ }
\psline[arrows=->, linestyle=dotted](5,2)(0,0.6)
\psline[arrows=->, linestyle=dotted](5,0.5)(0,0)
\psline[arrows=->, linestyle=dotted](5,-2)(0,-0.6)

\renewcommand{\Dconst}{1}
\psplot[linecolor=green]{-4}{-0.17}{1 \Dconst\space 2.718281828 2 x mul exp mul add 1 \Dconst\space 2.718281828 2 x mul exp mul sub  div}
\psplot[linecolor=green]{0.17}{4}{1 \Dconst\space 2.718281828 2 x mul exp mul add 1 \Dconst\space 2.718281828 2 x mul exp mul sub  div}
\renewcommand{\Dconst}{4}

\psplot[linecolor=green]{-4}{-0.863147181}{1 \Dconst\space 2.718281828 2 x mul exp mul add 1 \Dconst\space 2.718281828 2 x mul exp mul sub  div}
\psplot[linecolor=green]{-0.523147181}{4}{1 \Dconst\space 2.718281828 2 x mul exp mul add 1 \Dconst\space 2.718281828 2 x mul exp mul sub  div}

\renewcommand{\Dconst}{0.25}
\psplot[linecolor=green]{-4}{0.523147181}{1 \Dconst\space 2.718281828 2 x mul exp mul add 1 \Dconst\space 2.718281828 2 x mul exp mul sub  div}
\psplot[linecolor=green]{0.863147181}{4}{1 \Dconst\space 2.718281828 2 x mul exp mul add 1 \Dconst\space 2.718281828 2 x mul exp mul sub  div}
\rput[r](-5,0.5 ){$\frac{1+\frac 14 e^{2x}}{1- \frac{1}4 e^{2x}}$ }
\rput[r](-5,2 ){$\frac{1+e^{2x}}{1- e^{2x}}$ }
\rput[r](-5,-2 ){$\frac{1+4e^{2x}}{1- 4e^{2x}}$ }
\psline[arrows=->, linestyle=dotted](-5,0.5)(0,1.6667)
\psline[arrows=->, linestyle=dotted](-5,0.5)(1,-3.360539267)
\psline[arrows=->, linestyle=dotted](-5,2)(-0.2,5.066489563)
\psline[arrows=->, linestyle=dotted](-5,2)(0.2,-5.066489563)
\psline[arrows=->, linestyle=dotted](-5,-2)(0,-1.6667)
\psline[arrows=->, linestyle=dotted](-5,-2)(-1,3.360539267)
\psaxes[arrows=<->](0,0)(-4.5,-4.5)(4.5,4.5)

\rput (5,5){The direction field  $\frac{\diff y}{\diff x}=y^2-1$}

\fcDirectionFieldDefault{y y mul 1 sub}{-4}{-4}{0.5}{17}
\end{pspicture}
} %optionalDisplay

\noindent \ref{problemDFQseparable-yprime=ysquared-1-part2}.
From the computer generated picture above, we may visually estimate that $y(x)=\frac{1-4 e^{2x} }{1+4 e^{2x} }$ intersects the $x$-axis at $\left(0, -\frac {3}{ 5}\right)$. Furthermore, we may check directly that for
\[
y(x)=\frac{1-4 e^{2x} }{1+4 e^{2x} }
\]
we have $y(0)= \frac{1-4}{1+5}=  -\frac{3}{5}$ and that is a solution to our problem (this however does not prove the solution is unique).

Alternatively, let us give an algebraic solution. As we are given that $y(0)=-\frac35$ and so
\[
\begin{array}{rcl}
\displaystyle -\frac{3}{5}&=&\displaystyle y(0)= \frac{1-De^{2\cdot 0}}{1+ De^{2\cdot 0}}= \frac{1-D}{1+D}\\
\displaystyle -\frac{3}{5} (1+D)&=&1-D\\
\displaystyle \frac{2}{5} D&=&\displaystyle \frac{8}{5}\\
D&=&4\quad ,
\end{array}
\]
which is our final answer.
}

\solution{\ref{problemy'=y^2(1+x),y(0)=3}. 

This is a concise solution written up in a form suitable for exam taking.
\[\begin{array}{rcl}
\displaystyle \frac{\diff y}{\diff x}&=&\displaystyle y^2(1+x)\\
\displaystyle \frac{\diff y}{y^2} &=&\displaystyle  (1+x) \diff x\\
\displaystyle \int \frac{\diff y}{y^2} &=&\displaystyle \int (1+x) \diff x\\
\displaystyle -\frac{1}{y} &=&\displaystyle  x + \frac{x^2}{2} + C\\
\displaystyle -\frac{1}{3}& =&\displaystyle  0 + 0 + C\\
\displaystyle y &=&\displaystyle  -\frac{1}{\frac{x^2}{2}+x  - \frac{1}{3}} = -\frac{3}{3x^2+6x-2}\quad .
\end{array}
\]
}

\solution{\ref{problemDFQseparabley'=xtany_initial_condition1} and 
\ref{problemDFQseparabley'=xtany_initial_condition2}
\[\begin{array}{rcll|l}
\displaystyle y'&=&\displaystyle x\tan y\\
\displaystyle\frac{y'}{\tan y}&=&\displaystyle x\\
\displaystyle\frac{(\cos y) y'}{\sin y}&=&\displaystyle x &&\text{Integrate from }0\\
\displaystyle \int\limits_{t=0}^{t=x} \frac{\cos(y(t))}{\sin (y(t))} (y' \diff t)&=&\displaystyle  \int\limits_{t=0}^xt \diff t \\
\displaystyle \int\limits_{t=0}^{t=x} \frac{\cos(y(t))}{\sin (y(t))}\diff (y(t))&=&\displaystyle  \frac{x^2}{2} &&\text{Set }z=y(t)\\
\displaystyle \int \limits_{z=y(0)}^{z=y(x)} \frac{\cos z}{\sin z} \diff z&=&\displaystyle  \frac{x^2}{2}\\
\displaystyle \int \limits_{z=y(0)}^{z=y(x)} \frac{\diff (\sin z)}{\sin z} &=&\displaystyle  \frac{x^2}{2}\\
\displaystyle \left[\ln | \sin z|\right]_{y(0)}^{y} & = & \displaystyle  \frac{x^2}{2}\\
\displaystyle \ln |\sin y|- \ln |\sin (y(0))|&=& \displaystyle  \frac{x^2}{2} \\
\displaystyle \ln |\sin y|&=& \displaystyle  \frac{x^2}{2}+\ln |\sin (y(0))| \\
|\sin y|&=&\displaystyle  e^{\frac{x^2}{2}+\ln |\sin (y(0))|}\\ |\sin y|&=&\displaystyle \doublebrace{e^{\frac{x^2}{2}+\ln \left|\sin \left(\Arcsin \left(\frac{1}{e}\right) \right)\right|}}{\text{for problem \ref{problemDFQseparabley'=xtany_initial_condition1}}}{e^{\frac{x^2}{2}+\ln \left|\sin \left(\pi+ \Arcsin \left(\frac{1}{e}\right) \right)\right|} }{\text{for problem \ref{problemDFQseparabley'=xtany_initial_condition2}}} \\
|\sin y|&=&\displaystyle  e^{\frac{x^2}{2}+ \ln \left(\frac{1}{ e}\right)}\\
\displaystyle |\sin y|&=&\displaystyle e^{\frac{x^2}{2}-1} &&\begin{array}{l} y(0)>0 \text{ for both problems}\\
\text{therefore }  \sin y(0) > 0\end{array}\\
\displaystyle \sin y&=&e^{\frac{x^2}{2}-1} \quad.
\end{array}
\]
From the elementary properties of the trigonometric functions, we know that  $\sin y=\sin \alpha$ implies that either
\begin{itemize}
\item $y=\alpha +2k\pi$, where $k$ is an arbitrary integer or
\item $y=(2k+1)\pi-\alpha$, where $k$ is an arbitrary integer.
\end{itemize}
In other words, if we are given $\sin y$, we know $y$ up to a choice of sign and a choice of an integer $k$. For our problem, this means that 

\[
y=\left\{\begin{array}{ll} 2 k \pi+\Arcsin\left( e^{\frac{x^2}{2}-1} \right) &{k -\text{integer}} \\{\text{or}}\\{(2k+1)\pi- \Arcsin\left( e^{\frac{x^2}{2}-1} \right)}&{k-\text{integer}}\end{array}\right.
\]

For problem \ref{problemDFQseparabley'=xtany_initial_condition1}, 
the only choice for $k$ and sign which fits the initial condition $y(0)= \Arcsin\left(\frac{1}{e}\right)$ is
\[
y=\Arcsin\left(e^{\frac{x^2}{2}-1} \right)\quad ,
\]
which is our final answer. 

For problem \ref{problemDFQseparabley'=xtany_initial_condition2}, 
the only choice for $k$ and sign which fits the initial condition $y(0)=\pi+ \Arcsin\left(-\frac{1}{e}\right)=\pi- \Arcsin \left( \frac{1}{e}\right) $ is
\[
y=\pi- \Arcsin\left(e^{\frac{x^2}{2}-1} \right)\quad, 
\]
which is our final answer.
}

\item \begin{enumerate}
\item Find the general solution to the differential equation
\[
\frac{\diff y}{\diff x}= y^2-4\quad .
\]
Below is a computer-generated plot of the direction field  $\displaystyle \frac{\diff y}{\diff x}=y^2-4$, you may use it to get a feeling for what your answer should look like.

\optionalDisplay{
\begin{pspicture}(-6,-6)(6,6)
\newcommand{\Dconst}{4}
\psplot[linecolor=green]{-4}{4}{1 \Dconst\space 2.718281828 4 x mul exp mul sub 1 \Dconst\space 2.718281828 4 x mul exp mul add div 2 mul}
\psaxes{<->}(0,0)(-5,-5)(5,5)
\rput[l](0.4,5){The direction field  $\frac{\diff y}{\diff  x}=y^2-4$}
\fcDirectionFieldDefault{y y mul 4 sub}{-4}{-4}{0.5}{17}
\end{pspicture}
}
\item {Find a solution of the above equation for which $ y(0)= -\frac{6}{5}$.}
\end{enumerate}


\homeworkEnd
\end{comment}
