\documentclass{article}
\ProvidesPackage{homework-problems-UMB}
\addtolength{\hoffset}{-3.5cm}
\addtolength{\textwidth}{6.8cm}
\addtolength{\voffset}{-3cm}
\addtolength{\textheight}{6cm}
\usepackage{../homework-problems} %warning folder paths are relative to the file that uses the includepackage

\renewcommand{\answer}[1]{\iftoggle{answers}{ \hfill{~} \rotatebox{180}{\tiny answer: #1}}{} }
\renewcommand{\pointsii}[1]{}
\renewcommand{\hiddenanswer}{\answer}
\renewcommand{\points}[1]{\item}
\renewcommand{\pointsii}[1]{\item}
\renewcommand{\Arctan}{\arctan}
\renewcommand{\Arcsin}{\arcsin}
\renewcommand{\Arccot}{\operatorname{arccot}}

\toggletrue{solutions}
\toggletrue{answers}
\renewcommand{\fcProblemRef}{\theproblem.\theenumi}
\renewcommand{\fcSubProblemRef}{\theproblem.\theenumi.\theenumii}


\newcommand{\hide}[1]{}
\newtheorem{problem}{Problem}
\pagestyle{empty}
\begin{document}
\begin{center}
\Large
Review sheet Final exam \\ Math 141 Calculus II \\ \normalsize Summer 2016 \\ Instructor(s): Todor Milev
\end{center}
%\noindent \textbf{Name:\underline{~~~~~~~~~~~~~~~~~~~~~~~} } \hfill{~}



\noindent The exam is closed textbook. \textbf{No electronic devices are allowed during the exam. } Expect one or two problems from each problem type from this review sheet. The review sheet has roughly 3 problems for every problem type that will appear on the final.

%You are allowed one single formula sheet, handwritten by you. No template problem solutions are allowed. The sheet will be collected with the test. Photocopied formula sheets are not allowed. 

\begin{problem}Let $x\in (0,1)$. Express the following using $x$ and $\sqrt{1-x^2}$.  
\begin{enumerate}[ref={\fcProblemRef}]
\item $\displaystyle \int\frac{x^4 }{(x^2+2)(x+2) } \diff x$

\answer{$\displaystyle \frac{x^2}{2} -2x + \frac{8}{3}\ln|x +2|- \frac{1}{3}\ln \left(x^2+2\right)+\frac{2\sqrt{2}}{3} \Arctan \left(\frac{\sqrt{2}}{2}x\right)+C$ }

\item \label{problemintx^5/(x^3-1)dx} $\displaystyle \int \frac{x^5 }{x^3-1}\diff x$

\answer{$
\begin{array}{l}
\frac{1}{3} \ln{} \left| x^{2} + x + 1\right|+\frac{1}{3} \ln{} \left|x-1\right|+\frac{1}{3} x^{3}+C\\
=\frac{1}{3} \ln{} \left| x^3-1\right|+\frac{1}{3} x^{3}+C
\end{array}
$}
\item \label{problemint(3x^2+2x-1)/((x-1)(x^2+1))dx} $\displaystyle\int \frac{3x^2 + 2x - 1}{(x-1)(x^2+1)} \diff x$

\answer{$2 \ln|x-1| + \frac{1}{2} \ln\left(x^2+1\right) + 3 \Arctan x+C$}

\end{enumerate}
\end{problem}
\solution{\ref{problemintx^5/(x^3-1)dx}
This problem can be solved directly with a substitution shortcut, or by the standard method. 

\textbf{Variant I (standard method).}

\noindent$\begin{array}{@{}r@{}c@{}l@{}l@{}|l}
\displaystyle \int \frac{x^5}{x^3-1}\diff x&=&\displaystyle \int\left( x^2+\frac{x^2}{x^3-1}\right)\diff x  &&\text{Polyn. long div. }\\
&=&\displaystyle\frac{x^3}{3}+\int \frac{x^2}{(x-1)(x^2+x+1)}\diff x&&\text{part. frac.}\\
&=&\displaystyle\frac{x^3}{3}+\int \left(\frac{\frac{1}{3}}{x -1}+\frac{\frac{2}{3}x +\frac{1}{3}}{x^{2}+x +1}\right)\diff x &&\text{complete square}\\
&=&\displaystyle \frac{x^3}{3}+\frac{1}{3}\ln |x-1|+\frac{2}{3}\int \frac{x+\frac{1}{2}}{\left(x+\frac{1}{2}\right)^2+ \frac{3}{4}}\diff x &&\text{Set } \begin{array}{rcl} u&=&\left(x+\frac{1}{2}\right)^2+ \frac{3}{4}\\\frac{1}{2}\diff u&= &\left(x+\frac{1}{2}\right) \diff x\end{array}\\
&=&\displaystyle \frac{x^3}{3}+\frac{1}{3}\ln |x-1|+\frac{1}{3} \int \frac{\diff u}{u}\\
&=&\displaystyle \frac{x^3}{3}+\frac{1}{3}\ln |x-1|+\frac{1}{3}\ln |u|+C\\
&=&\displaystyle \frac{x^3}{3}+\frac{1}{3}\ln |x-1|+\frac{1}{3}\ln |x^2+x+1|+C\\
\end{array}
$

\textbf{Variant II (shortcut method).}
\[
\begin{array}{rcll|l}
\displaystyle \int \frac{x^5}{x^3-1}\diff x&=&\displaystyle \int \frac{x^5-x^2+x^2}{x^3-1}\diff x\\
&=&\displaystyle \int \frac{x^2(x^3-1)+x^2}{x^3-1}\diff x\\
&=&\displaystyle\int x^2\diff x+ \int \frac{x^2}{x^3-1}\diff x\\
&=&\displaystyle \frac{x^3}{3}+\int \frac{\diff \left(\frac{x^3}{3}\right)}{x^3-1}\\
&=& \displaystyle \frac{x^3}{3}+\frac{1}{3} \int \frac{\diff \left(x^3-1\right)}{x^3-1}&&\text{Set }u=x^3-1\\
&=&\displaystyle \frac{x^3}{3}+\frac{1}{3}\int \frac{\diff u}{u}\\
&=&\displaystyle\frac{x^3}{3}+\frac{1}{3}\ln |u|+C\\
&=&\displaystyle \frac{x^3}{3}+\frac{1}{3} \ln \left|x^3-1\right|+C\quad .
\end{array}
\]
The answers obtained in the two solution variants are of course equal since 
\[
\ln |x-1|+\ln |x^2+x+1|= \ln \left|\left(x-1\right)\left(x^2+x+1\right)\right|=\ln \left|x^3-1\right|\quad .
\]
}

\solution{\ref{problemint(3x^2+2x-1)/((x-1)(x^2+1))dx}. 
This is a concise solution written in a form suitable for exam taking. To make this solution as short as possible we have omitted many details. On an exam, the student would be expected to carry out those omitted computations on the side. We set up the partial fraction decomposition as follows.
\[
\displaystyle \frac{3x^2 + 2x - 1}{(x-1)(x^2+1)} = \frac{A}{x-1} + \frac{Bx+C}{x^2+1}\quad .
\]
Therefore $3x^2 + 2x - 1 = A(x^2+1) + (Bx+C)(x-1)$. 
\begin{itemize}
\item We set $x = 1$ to get $4 = 2A$, so $A = 2$.
\item We set $x = 0$ to get $-1=A-C$, so $C=3$.
\item Finally, set $x = 2$ to get $15=5A+2B+C$, so $B=1$.
\end{itemize}
We can now compute the integral as follows.
\[
\displaystyle \int\left( \frac{2}{x-1} + \frac{x+3}{x^2+1} \right) \diff x = 2 \ln(|x-1|) + \frac{1}{2} \ln(x^2+1) + 3 \Arctan x+K\quad .
\]
}



\begin{problem}Integrate.
\begin{enumerate}
\item $\displaystyle \int \sin^2 x \cos x\diff x$.

\answer{$ \frac{1}{3}\sin^3 x+C$}

\item $\displaystyle \int \cos^3 x\diff x$.

\answer{$ \sin x -\frac{1}{3}\sin^3 x+C$}

\item $\displaystyle \int \sin^3 x \cos^4 x\diff x$.

\answer{$ \frac{1}{7} \cos^{7}{}x-\frac{1}{5} \cos^{5}{}x +C$}

\end{enumerate}
\end{problem}


\begin{problem}Compute the limit.
\begin{enumerate}[ref={\fcProblemRef}]
\item \label{problemLHospital (sin (pi x) ln x )/ (cos pi x +1)}  $\displaystyle \lim\limits_{x\to 1} \frac{\sin \left(\pi x \right)\ln x }{\cos(\pi x)+1 } $.

\answer{$-\frac{2}{\pi}$}

\item \label{problemlim x to 0 (sin x - x)/(arctan x - x)} $\displaystyle \lim\limits_{x\to 0}\frac{\sin x- x}{\Arctan x -x}$.

\answer{$\frac{1}{2}$}

\item \label{problemlimxtoinftysin(2/x)}
$ {\displaystyle \lim_{x \to \infty} x \sin\left(\frac{2}{x}\right)}.$

\answer{$2$}

\end{enumerate}
\end{problem}
\solution{\ref{problemLHospital (sin (pi x) ln x )/ (cos pi x +1)}
The limit is of the form ``$\frac{0}{0}$'' so we are allowed to use L'Hospital's rule.
\[
\begin{array}{rcll|l}
\displaystyle
\lim\limits_{x\to 1} \frac{\sin \left(\pi x\right)\ln x }{\cos(\pi x)+1 } &=&\displaystyle \lim\limits_{x\to 1}  \frac{\left(\sin \left(\pi x\right)\ln x\right)' }{\left(\cos(\pi x)+1\right)' }\\
&=&\displaystyle \lim\limits_{x\to 1}  \frac{\left(\pi \cos \left(\pi x\right)\ln x+ \sin\left(\pi x\right)\frac{1}{x}\right) }{\left(-\pi\sin(\pi x)\right) } &&\text{type ``$\frac{0}{0}$'',  L'Hospital's rule}\\
&=&\displaystyle \lim\limits_{x\to 1}  \frac{\left(\pi \cos \left(\pi x\right)\ln x+ \sin\left(\pi x\right)\frac{1}{x}\right)' }{\left(-\pi\sin(\pi x)\right)' } \\
&=&\displaystyle \lim\limits_{x\to 1}  \frac{- \pi^{2} \sin{}\left(\pi x\right) \ln{}\left(x\right)+2 \pi \cos{}\left(\pi x\right) x^{-1}- \sin{}\left(\pi x\right) x^{-2} }{\left(-\pi^2\cos(\pi x)\right) } \\
&=&\displaystyle  \frac{- \pi^{2} \sin{}\left(\pi \right) \ln(1)+2 \pi \cos{}\left(\pi \right) - \sin{}\left(\pi \right)  }{\left(-\pi^2\cos(\pi )\right) } \\
&=&\displaystyle -\frac{2}{\pi}\quad .
\end{array}
\]
}


\solution{
\ref{problemlim x to 0 (sin x - x)/(arctan x - x)}
\noindent \textbf{Solution I.} 
\[
\begin{array}{rcll|l}
\displaystyle \lim\limits_{x\to 0}\frac{\sin x- x}{\Arctan x -x}&=&\displaystyle \lim\limits_{x\to 0} \frac{\cos x -1 }{ \frac{1}{1+x^2}-1 } &&\text{L'Hospital rule}\\
&=&\displaystyle \lim\limits_{x\to 0}\frac{-\sin x }{\frac{ -2x}{(1+x^2)^2} }&& \text{L'Hospital rule again}\\
&=&\displaystyle \lim\limits_{x\to 0} \frac{(1+x^2)^2}{2}\frac{\sin x}{x}  \\
&=&\displaystyle \lim\limits_{x\to 0} \frac{(1+x^2)^2}{2}\lim\limits_{x\to 0}\frac{\sin x}{x} \\
&=&\displaystyle \frac{1}{2}\quad .
\end{array}
\] 

\noindent \textbf{Solution II.}
\[
\begin{array}{rcll|l}
\displaystyle \lim\limits_{x\to 0}\frac{\sin x- x}{\Arctan x -x}&=&\displaystyle \lim\limits_{x\to 0} \frac{\left( x-\frac{x^3}{3!}+\frac{x^5}{5!}-\dots\right) -x}{\left( x-\frac{x^3}{3}+\frac{x^5}{5}-\dots\right)-x} &&\text{use the Maclaurin series of }\sin, \Arctan \\
&=&\displaystyle \lim\limits_{x\to 0}\frac{- \frac{x^3 }{6} + x^5\left(\frac{1}{5!}-\dots\right) }{- \frac{ x^3}{3} + x^5 \left(\frac{1}{5}-\dots \right)  }&& \begin{array}{rcl}
\text{The expressions in parenthesis }\\
\text{are continous functions in x }
\end{array}\\
&=&\displaystyle \lim\limits_{x\to 0} \frac{-\frac{1}{6}+ x^2 \left(\frac{1}{5!}-\dots\right) }{- \frac{1}{3} +x^2 \left( \frac{1}{5}-\dots \right)  }\\
&=&\displaystyle \frac{-\frac{1}{6}+0}{\frac{1}{3}+0}\\
&=&\displaystyle \frac{1}{2}\quad .
\end{array}
\] 
}

\solution{\ref{problemlimxtoinftysin(2/x)}.
\[
\begin{array}{rcll|l}
\displaystyle \lim_{x\to\infty }x\sin\left( \frac{2}{x} \right)& =&\displaystyle \lim_{x \to \infty} \frac{\sin \left(\frac{2}{x}\right) }{ \frac{ 1}{x}}
 &&\begin{array}{l}
\text{indeterminate form }\\
\text{Use L'Hospital's rule}
\end{array}\\
&=&\displaystyle \lim_{x \to \infty} \frac{\cos\left(\frac{2}{x}\right) \left (-\frac{2}{x^2}\right)}{-\frac{1}{x^2}} \\ 
&=&\displaystyle \lim_{x \to \infty} 2 \cos\left(\frac{x}{2}\right) \\
&=&\displaystyle 2\quad .
\end{array}
\]
}




\begin{problem}Determine whether the integral is convergent or divergent. If convergent, evaluate it.

\begin{multicols}{2}
\begin{enumerate}[ref={\fcProblemRef}]
\item $\displaystyle \int\limits_{-2}^{ \frac{1}{ 2}} \frac{1}{2x-1} \diff x$.

\answer{divergent}

\item $\displaystyle \int\limits_{0}^{2}x^3\ln x \diff x$.

\answer{convergent, equals $-1+4 \ln 2 $}


\item $\displaystyle \int\limits_{-1}^{\infty} e^{-3x} \diff x$.

\answer{convergent, equals $\frac{e^{3}}{3}$}

\item $\displaystyle \int\limits_{-\infty}^{\infty} x e^{-x^2} \diff x$.

\answer{convergent, equals $0$}

\item \label{problemConvergencesqrt(x)e^-sqrt(x)zerotoinfty} $\displaystyle \int\limits_{0}^{\infty} \sqrt{x} e^{-\sqrt{x}} \diff x$.

\answer{convergent, equals $4 $}

\item $\displaystyle \int\limits_{100}^{\infty} \frac{1}{x\ln x} \diff x$.

\answer{divergent}


\end{enumerate}
\end{multicols}
\end{problem}


\solution{\ref{problemConvergencesqrt(x)e^-sqrt(x)zerotoinfty}
It is possible to show that this integral is convergent by using the comparison theorem. However, we shall use direct integration instead. First, we solve the indefinite integral:

\[
\begin{array}{rcll|l}
\displaystyle \int \sqrt{x} e^{-\sqrt{x}} \diff x&=& \displaystyle \int \sqrt{x} e^{-\sqrt{x}} \frac{2\sqrt{x} \diff x}{2\sqrt{x}} &&\text{use }  \diff \sqrt{x} = \frac{\diff x}{2\sqrt{x}} \\
&=& \displaystyle \int \sqrt{x} e^{-\sqrt{x}} \left(2 \sqrt{x} \diff \sqrt{x}\right) &&\text{Set } \sqrt{x}=u\\
&=& \displaystyle 2\int u^2 e^{-u}\diff u\\
&=& \displaystyle 2\left( -\int u^2 \diff \left( e^{-u} \right) \right) &&\text{integrate by parts} \\
&=& \displaystyle 2\left(- u^2e^{-u}+\int e^{-u}\diff \left(u^2\right) \right) \\
&=&\displaystyle 2\left(- u^2e^{-u}+\int 2 u e^{-u}\diff u \right)\\
&=&\displaystyle 2\left(- u^2e^{-u}-\int 2 u \diff e^{ -u} \right) &&\text{integrate by parts again}\\
&=&\displaystyle 2\left(- u^2e^{-u}- 2 u  e^{-u}+ \int 2e^{-u}\diff u \right) \\
&=&\displaystyle 2\left( - u^2e^{-u} -2ue^{-u} -2e^{ -u} \right)+C\\
&=&\displaystyle 2\left( - xe^{-\sqrt{x}} -2\sqrt{x } e^{ -\sqrt{x}} -2e^{-\sqrt{x}}\right)+C
\end{array}
\]
Therefore 
\[
\begin{array}{rcll|l}
\displaystyle \int\limits_{0}^\infty \sqrt{x} e^{-\sqrt{x}} \diff x&=& \displaystyle \lim\limits_{t\to \infty} 2\left[ - xe^{- \sqrt{ x} } -2\sqrt{x}e^{-\sqrt{x}} -2e^{- \sqrt{x}} \right]_{ 0}^{\infty}\\
&=&\displaystyle 4+ \lim\limits_{t\to \infty} 4\left( -te^{ -\sqrt{t}} -\sqrt{t}e^{-\sqrt{t}}- e^{-\sqrt{t}} \right) && \text{Set }u=\sqrt{t}\\
&=&\displaystyle 4- 4\lim\limits_{u\to \infty} \left( u^2 e^{-u} + ue^{-u}+e^{-u}\right)\\
&=& \displaystyle 4- 4\lim\limits_{u\to \infty} \frac{ u^2 + u+1}{e^u} &&\text{use L'Hospital's rule for limit, see below}\\
&=& 4\quad ,
\end{array}
\]
and the integral converges to $4$. In the above computation we used the following limit computation
\[
\begin{array}{rcll|l}
\displaystyle \lim\limits_{u\to \infty}\frac{u^2+u+1}{e^u}&=&\displaystyle  \lim\limits_{u\to \infty} \frac{2u+1}{e^u}&&\text{Apply L'Hospital's rule}\\
&=&\displaystyle \lim\limits_{u\to \infty} \frac{2}{e^u}\\
&=&\displaystyle 0\quad .
\end{array}
\]

}

\begin{problem}The series below are convergent; compute them. 
\begin{enumerate}[ref={\fcProblemRef}]
\item \label{problemsumn=0^infty(2^n+5^n)/10^n}
$\displaystyle\sum_{n=0}^{\infty} \frac{2^n+5^n}{10^n}$

\answer{$\frac{13}{4}$}
 
\item \label{sum_n=0^infty(2^(n+1)+(-3)^(n-1))/5^n}
$\displaystyle
\sum_{n=0}^\infty \frac{2^{n+1}+(-3)^{n-1}}{5^n}
$

\answer{$ \frac{25}{8} $}
 
\item \label{sum_n=0^infty(2^(n+1)+(-3)^(n-1))/5^n}
$\displaystyle
\sum_{n=0}^\infty \frac{2^{n+1}+(-3)^{n-1}}{5^n}
$

\answer{$ \frac{25}{8} $}
 

\end{enumerate}
\end{problem}
\solution{\ref{problemsumn=0^infty(2^n+5^n)/10^n}.
\[
\begin{array}{rcll|l}
\displaystyle \sum\limits_{n=0}^{\infty}\frac{2^n+5^n}{10^n}&=&\displaystyle  \sum \limits_{ n=0}^{\infty}\left(\frac{1}{5^n}+\frac{1}{2^n}\right)&&\text{use } \sum\limits_{ n= 0}^{\infty} r^n=\frac{1}{1-r}, \text{ for } |r|<1\\
&=&\displaystyle \frac{1}{1-\frac{1}{2}} +\frac{1}{1-\frac{1}{5}}\\
&=&\displaystyle \frac{13}{4}\quad .
\end{array}
\]
} 
\solution{\ref{sum_n=1^infty(3^(n+1)+7^(n-1))/21^n}.
\[
\begin{array}{rcll|l}
\displaystyle \sum\limits_{n=1}^{\infty} \frac{3^{n+1}+ 7^{ n-1}}{21^n} &=& \displaystyle \sum \limits_{n=1}^{ \infty}\left(3 \cdot \frac{3^{n}}{21^n}+ \frac{ 1 }{7}\cdot \frac{ 7^n }{21^n}\right)\\
&=&\displaystyle 3\sum_{n =1}^{\infty} \left(\frac{1}{7} \right)^n + \frac{1 }{7} \sum_{n=1}^{\infty}  \left(\frac{1}{3} \right)^n\\
&=&\displaystyle \frac{3}{7} \sum_{n=0}^{\infty} \left( \frac{1}{7 } \right)^n +\frac{1 }{21} \sum_{ n=0}^{ \infty}\left(\frac{1}{3 } \right) ^n &&\text{use }\sum_{n= 0 }^\infty r^n=\frac{1}{1-r}, |r|<1\\
&=&\displaystyle \frac{3}{7}\cdot  \frac{1}{ \left(1 -\frac{ 1}{7} \right)}+ \frac{ 1}{21}\cdot  \frac{1 }{ \left(1-\frac{1 }{ 3} \right)}\\
&=&\displaystyle \frac{4}{7}\quad .
\end{array}
\]
} 
\solution{\ref{sum_n=0^infty(2^(n+1)+(-3)^(n-1))/5^n}.
\[
\begin{array}{rcll|l}
\displaystyle \sum\limits_{n=0}^{\infty} \frac{2^{n+1}+ (-3)^{ n-1}}{5^n} &=& \displaystyle \sum \limits_{n=0}^{ \infty} \left(2\cdot \frac{2^{n}}{5^n}-\frac{ 1 }{3}\cdot  \frac{ (-3)^n }{5^n}\right)\\
&=&\displaystyle 2 \sum_{n=0}^{ \infty}\left(\frac{2}{5 } \right)^n -\frac{1}{3} \sum_{ n=0}^{\infty}\left(-\frac{3}{5 } \right) ^n &&\text{use }\sum_{n= 0 }^\infty r^n=\frac{1}{1-r}, |r|<1\\
&=&\displaystyle 2\cdot  \frac{1}{ \left(1 -\frac{ 2}{5} \right)}- \frac{ 1}{3}\cdot  \frac{1 }{ \left(1-\left(-\frac{3}{ 5}\right) \right)} \\
&=&\displaystyle \frac{25}{8} \quad .
\end{array}
\]

} 


%\vskip 18cm
%\hfill \begin{tabular}{c|c|c|c|c|c|c||c}
%Problem&1 &2&3&4&5&6& $\sum$\\ \hline
%Score&&&&&&&\\ \hline
%Max&17&17&17&17&17&17&102
%\end{tabular} 


\end{document}