\documentclass{article}
\usepackage{amsmath, amsfonts, amssymb, verbatim, hyperref}
\usepackage{auto-pst-pdf}
\usepackage{pst-plot}
\pagestyle{empty}
\usepackage{pstricks}
\usepackage{../homework-problems}
\usepackage{../pstricks-commands}

\usepackage{multicol}
\renewcommand{\Re}{\mathrm{Re~}}
\renewcommand{\Im}{\mathrm{Im~}}
\begin{document}
\begin{center}
\Large
Exam III review problems%\\ Math 140 Calculus I \\  \normalsize \today \\ Instructor: Todor Milev
\end{center}
%\noindent \textbf{Name:} \hfill{~}
%\begin{tabular}{c|c|c|c|c|c|c|c|c|c|c||c}
%Problem&1 &2&3&4&5&6&7&8& $\sum$\\\hline
%Score &&&&&&&&&\\\hline
%Max & 15& 10 & 26 & 10 & 10 & 10 & 14 & 15 & 110
%\end{tabular} 

\noindent The exam is closed book, no calculators allowed.


\begin{problem}
Find the 
\begin{itemize}
\item $x$ and $y$ intercepts of $f$.
\item horizontal and vertical asymptotes.
\item intervals of increase and decrease
\item local and global minima, maxima,
\item intervals of concavity 
\item points of inflection
\end{itemize}
Label all relevant points on the graph. 
\begin{enumerate}
\item $\displaystyle f(x)=\frac{x+1}{x^2+2x+4}$
\psset{xunit=2cm, yunit=2cm}
\begin{pspicture}(-4.500000, -5)(4.500000,5) 
\psframe*[linecolor=white](-4.500000,-1)(4.500000,1) 
\tiny 
\psaxesStandard{-5.000000}{-1}{3.000000}{1} %Function formula: \frac{x+1}{x^{2}+2 x+4} 
\psplot[linecolor=\psColorGraph, plotpoints=1000]{-5.000000}{3.000000}{1 x add 4 x 2 mul add x 2 exp add div }
\end{pspicture} 

\answer{
\begin{tabular}{l}
$y$-intercept: $\frac14$, $x$-intercept: $-1$\\
horizontal asymptote: $y=0$, vertical: none\\
increasing on 
$\left(-1-\sqrt{3}, -1+\sqrt{3}  \right) $, decreasing on $\left(-\infty, -1-\sqrt{3}\right)\cup \left(-1+\sqrt{3}, \infty\right) $\\
local and global min at $x=-1-\sqrt{3}$, local and global max at $x=-1+\sqrt{3}$\\
concave up on $\left(-4, -1\right)cup \left(2, \infty \right)$, concave down $\left(-\infty, -4\right)\cup \left(-1, 2\right)$\\
inflection points at $x=-4,x=-1, x=2$
\end{tabular}
}
\item $f(x)=\frac{x^{2}+3 x+1}{x^{2}+2 x}$
\psset{xunit=0.7cm, yunit=0.7cm}
\begin{pspicture}(-4.500000, -5)(4.500000,5) 
\psframe*[linecolor=white](-4.500000,-5)(4.500000,5) 
\tiny 
\psaxesStandard{-4.000000}{-4.5}{4.000000}{4.5} %Function formula: \frac{x^{2}- x-1}{x^{2}-2 x} 
\psplot[linecolor=\psColorGraph, plotpoints=1000]{2.100000}{4.000000}{-1 x -1 mul add x 2 exp add x -2 mul x 2 exp add div }
%Function formula: \frac{x^{2}- x-1}{x^{2}-2 x} 
\psplot[linecolor=\psColorGraph, plotpoints=1000]{0.100000}{1.900000}{-1 x -1 mul add x 2 exp add x -2 mul x 2 exp add div }
%Function formula: \frac{x^{2}- x-1}{x^{2}-2 x} 
\psplot[linecolor=\psColorGraph, plotpoints=1000]{-4.000000}{-0.100000}{-1 x -1 mul add x 2 exp add x -2 mul x 2 exp add div }
\end{pspicture} 

\answer{
\begin{tabular}{l}
$y$-intercept: none, $x$-intercepts: $\frac{3\pm \sqrt{5}}2$ ($\frac{3+ \sqrt{5}}2$ is the golden ratio)\\
horizontal asymptote: $y=1$, vertical: $x=0$ and $x=2$\\
always decreasing\\
no local/global minima/maxima\\
concave down on $\left(-\infty,0\right)cup \left(1,2 \right)$, concave up on $\left(0, 1\right)\cup \left(2, \infty\right)$\\
inflection point at $x=1$
\end{tabular}
}
\end{enumerate}
\end{problem}

\begin{problem}
Use the Mean Value Theorem/Rolle's Theorem to prove that the function has \textbf{exactly one} real root. See Lecture 19 for a similar problem solution.
\begin{enumerate}
\item $f(x)= x^3 +x^2+x+1$.
\item $f(x)=\cos^3 \left({\frac{x}{3}}\right) +\sin x-  3x$.
\end{enumerate}

\end{problem}

\begin{problem}
Find the linearization of
\begin{enumerate}
\item $f(x)=\sqrt{8+x}$ at $a=1$ and use it to approximate $\sqrt{9.02}$.

\answer{ $f(x)\approx 3+ \frac16 (x-1)=1/6 x+17/6$. Therefore $\sqrt{9.02}\approx 901/300\approx 3.003333$}
\item $f(x)=\sqrt[3]{8+x}$ at $a=0$ and use it to approximate $\sqrt[3] {7.97}$.
\answer{ $\sqrt[3]{8+x}\approx \frac{1}{12}x+2$. Therefore $\sqrt[3]{7.97}\simeq 799/400=1.9975$}

\end{enumerate}
\end{problem}
\begin{problem}
Write down and compute a Riemann sum estimate for the integral 
\begin{enumerate}
\item $\displaystyle\int_{0}^2 \frac{dx}{1+x+x^3}$ using $\Delta x=\frac{1}2 $ and right endpoint sampling points.
\answer{$ \frac{1}{2}\left(\frac{8}{13}+\frac{1}{3}+\frac{8}{47}+\frac{1}{11}\right)=\frac{12197}{20163}\approx 0.604920$}
\item $\displaystyle\int_{-2}^{0} \frac{dx}{1+x+x^2}$ using $\Delta x=\frac23 $ and left endpoint sampling points.

\answer{$\frac23\left(\frac{1}{3}+\frac{9}{13}+\frac{9}{7}\right)=\frac{1262}{819}\approx 1.540904$}
\end{enumerate}
\end{problem}

\begin{problem}
Evaluate the definite integral 
\begin{enumerate}
\item $\displaystyle\int\limits_{1}^{8} \frac{t-t^{1/3}+2}{t^{4/3}} dt\quad .$
\answer{$- \ln8+\frac{15}{2}$}
\item $\displaystyle\int\limits_{1}^{4} (x+\sqrt{x})^2 dx\quad .$
\answer{$\frac{119}{2}$}
\end{enumerate}
\end{problem}

\begin{problem}
Evaluate the indefinite integral 
\begin{enumerate}
\item $\displaystyle\int \tan (2x) ~dx$.
\answer{$-\frac{1}{2}\ln(\cos (2x) )) +C$}
\item $\displaystyle \int \cot \left(\frac{x}{2}\right)~dx$
\answer{$2\ln\left(\sin \left(\frac{x}{2}\right) \right)$}
\end{enumerate}
\end{problem}

\begin{problem}
Evaluate the definite integral 
\begin{itemize}
\item $\displaystyle\int\limits_{1}^{2} \frac{x}{2x^2+1 }  ~dx$.
\answer{$\frac14 \ln 3$}
\item $\displaystyle\int\limits_{0}^{\frac{1}4}\frac{x }{\sqrt{1-3x^2}}dx$.

\answer{$\frac{1}3\left(1-\sqrt{\frac{13}{16}} \right)$}
\end{itemize}
\end{problem}
\begin{problem}
State the Fundamental Theorem of Calculus (both parts). (Lecture 22)
\end{problem}
\begin{problem}~
\begin{enumerate}
% Area problems
\item Find the area of the region bounded by the curves $y = 2x^2$ and $y = 4 + x^2$.

\answer{$\frac{32}{3}$}
\item Find the area of the region bounded by the curves $x = 4 - y^2$ and $y = 2 - x$

\answer{$\frac92$}
\end{enumerate}
\end{problem}
\begin{problem}~
\begin{enumerate}
% Volume problems
\item Consider the region bounded by the curves $y = 1-x^2$ and $y =0$. What is the volume of the solid obtained by
rotating this region about the line $y = 0$?

\answer{$\frac{16 \pi}{15}$}
\item Consider the region bounded by the curves $y = x^2$ and $x = y^2$. What is the volume of the solid obtained by
rotating this region about the line $x = 2$?

\answer{ $\frac{31 \pi}{30}$}
\end{enumerate}
\end{problem}

\end{document}