\documentclass{article}
\usepackage{amsmath, amsfonts, amssymb, verbatim, hyperref}
\usepackage{auto-pst-pdf}
\usepackage{pst-plot}
%\usepackage{pst-solides3d}
%\usepackage{pst-3dplot}
\usepackage{pstricks}

\usepackage{multicol}
\addtolength{\hoffset}{-3.5cm}
\addtolength{\textwidth}{6.8cm}
\addtolength{\voffset}{-3.3cm}
\addtolength{\textheight}{6.3cm}
\renewcommand{\Re}{\mathrm{Re~}}
\renewcommand{\Im}{\mathrm{Im~}}
\newcommand{\doublebrace}[4]{\left\{\begin{array}{ll} #1 & #2 \\#3 & #4  \end{array} \right.}
\newcommand{\triplebrace}[6]{\left\{\begin{array}{ll} #1 & #2 \\#3 & #4  \\#5 & #6\end{array} \right.}
\date{}
\newtheorem{problem}{Problem}
\newcommand{\psHollowDot}[2]{
\pscircle*[fillcolor=white, linecolor=red](#1, #2){0.07}
\pscircle*[fillcolor=white, linecolor=white](#1, #2){0.04}
}
\newcommand{\psHollowDotBlue}[2]{
\pscircle*[fillcolor=white, linecolor=blue](#1, #2){0.07}
\pscircle*[fillcolor=white, linecolor=white](#1, #2){0.04}
}
\newcommand{\psFullDot}[2]{
\pscircle*[fillcolor=white, linecolor=red](#1, #2){0.07}
}
\newcommand{\psFullDotBlack}[2]{
\pscircle*[fillcolor=white, linecolor=black](#1, #2){0.07}
}
\newcommand{\psLabelXOne}{\psline(1, -0.1)(1,0.1) \rput[t](1, -0.2 ) {\footnotesize $1$} }
\newcommand{\psLabelYOne}{\psline(-0.1, 1)(0.1, 1) \rput[r](-0.2, 1 ) {\footnotesize $1$} }

\title{
Homework Math 140 \\
Lecture 8, 9,10 \\
Will be Tested on March 7
}
\begin{document}
\maketitle
\textbf{This homework contains copyrighted material from  James Stewart, Calculus, 7th edition, 2012. 
You are not permitted to copy this file for any purpose other than completing your homework. You are not allowed to give a copy of this file to anyone outside of our course. 
}
\begin{problem} (Textbook, page 235, problems 9-30).
Find the limit or show that it does not exist. I the limit does not exist, indicate whether it is $\pm\infty$, or neither. 
\begin{multicols}{3}
\begin{enumerate}
\item $\lim\limits_{x\to\infty }\frac{3x-2}{2x+1}$.
\item $\lim\limits_{x\to\infty }\frac{1-x^2}{x^3-x+1}$.
\item $\lim\limits_{x\to-\infty }\frac{x-2}{x^2+1}$.
\item $\lim\limits_{x\to-\infty }\frac{4x^3+6x^2-2}{2x^3-4x+5}$.
\item $\lim\limits_{x\to\infty }\frac{\sqrt{t}+t^2}{2t-t^2}$.
\item $\lim\limits_{x\to\infty }\frac{t-t\sqrt{t}}{2t^{3/2}+3t-5}$.
\item $\lim\limits_{x\to\infty }\frac{(2x^2+1)^2}{(x-1)^2(x^2+x)}$.
\item $\lim\limits_{x\to\infty }\frac{x^2}{\sqrt{x^4+1}}$.
\item $\lim\limits_{x\to\infty }\frac{\sqrt{9x^6-x}}{x^3+1}$.
\item $\lim\limits_{x\to-\infty }\frac{\sqrt{9x^6-x}}{x^3+1}$.
\item $\lim\limits_{x\to\infty}\sqrt{9x^2+x}-3x$.
\item $\lim\limits_{x\to-\infty}x+\sqrt{x^2+2x} $.
\item $\lim\limits_{x\to\infty}\sqrt{x^2+ax}-\sqrt{x^2+bx}$.
\item $\lim\limits_{x\to\infty}\cos x$.
\item $\lim\limits_{x\to\infty}\frac{x^4-3x^2+x}{x^3-x+2}$.
\item $\lim\limits_{x\to\infty}\sqrt{x^2+1}$.
\item $\lim\limits_{x\to-\infty}(x^4+x^5)$.
\item $\lim\limits_{x\to-\infty}\frac{1+x^6}{1+x^4}$.
\item $\lim\limits_{x\to\infty}(x-\sqrt{x})$.
\item $\lim\limits_{x\to\infty}(x^2-x^4)$.
\item $\lim\limits_{x\to\infty}x\sin \frac{1}{x}$.
\item $\lim\limits_{x\to\infty}\sqrt{x}\sin \frac{1}{x}$.
\end{enumerate}
\end{multicols}
\end{problem}
\begin{problem}(Textbook, page 235, problems 33-38).
Find the horizontal and vertical asymptotes of each curve. If you have a graphing device, check your work by graphing the curve and estimating the asymptotes.
\begin{multicols}{3}
\begin{enumerate}
\item $y=\frac{2x+1}{x-2}$.
\item $y=\frac{x^2+1}{2x^2-3x-2}$.
\item $y=\frac{2x^2+x-1}{x^2+x-2}$.
\item $y=\frac{1+x^4}{x^2-x^4}$.
\item $y=\frac{x^3-x}{x^2-6x+5}$.
\item $y=\frac{x-9}{\sqrt{4x^2+3x+2}}$.
\end{enumerate}
\end{multicols}
\end{problem}
\begin{problem}(Textbook, page 122, problem 3)
Match the graph of each the following functions  
\begin{multicols}{2}
\begin{enumerate}
\item 
\psset{xunit=1cm, yunit=1cm}
\begin{pspicture}(-5, -5)(5,5) 
\psframe*[linecolor=white](-5,-5)(5,5) 
\psaxes[ticks=none, labels=none]{<->}(0,0)(-2.5,-2.5)(2.5,2.5)
%Function formula: -8 ((x) ((x) (x)))+2 (x) 
\psplot[linecolor=red, plotpoints=1000]{-0.8}{0.8}{x 2 mul x x mul x mul -8 mul add }
\end{pspicture} 
\item 
\psset{xunit=1cm, yunit=1cm}
\begin{pspicture}(-5, -5)(5,5) 
\psframe*[linecolor=white](-5,-5)(5,5) 
\psaxes[ticks=none, labels=none]{<->}(0,0)(-2.5,-2.5)(2.5,2.5)
%Function formula: -2+x 
\psplot[linecolor=red, plotpoints=1000]{1}{2.5}{x -2 add } %Function formula: - (x) 
\psplot[linecolor=red, plotpoints=1000]{-1}{1}{x -1 mul } %Function formula: 2+x 
\psplot[linecolor=red, plotpoints=1000]{-2.5}{-1}{x 2 add }
\end{pspicture} 
\item 
\psset{xunit=1cm, yunit=1cm}
\begin{pspicture}(-5, -5)(5,5) 
\psframe*[linecolor=white](-5,-5)(5,5) 
\psaxes[ticks=none, labels=none]{<->}(0,0)(-2.5,-2.5)(2.5,2.5)
%Function formula: - ((1)/((x)^{2}+1)) 
\psplot[linecolor=red, plotpoints=1000]{-2.5}{2.5}{1 1 x 2 exp add div -1 mul }
\end{pspicture} 
\item 
\psset{xunit=1cm, yunit=1cm}
\begin{pspicture}(-5, -5)(5,5) 
\psframe*[linecolor=white](-5,-5)(5,5) 
\psaxes[ticks=none, labels=none]{<->}(0,0)(-2.5,-2.5)(2.5,2.5)
%Function formula: - (((x)^{2}) ((x) (x)))+(x)^{2} 
\psplot[linecolor=red, plotpoints=1000]{-1.46}{1.46}{x 2 exp x x mul x 2 exp mul -1 mul add }
\end{pspicture} 
\end{enumerate}
\end{multicols}
to the graph of its derivative among the graphs below
\begin{multicols}{2}
\begin{enumerate}
\item 
\psset{xunit=1cm, yunit=1cm}
\begin{pspicture}(-5, -5)(5,5) 
\psframe*[linecolor=white](-5,-5)(5,5) 
\psaxes[ticks=none, labels=none]{<->}(0,0)(-2.5,-2.5)(2.5,2.5)
%Function formula: (x)/(((x)^{2}+1)^{2}) 
\psplot[linecolor=blue, plotpoints=1000]{-2.5}{2.5}{x 1 x 2 exp add 2 exp div }
\end{pspicture} 

\item \psset{xunit=1cm, yunit=1cm}
\begin{pspicture}(-5, -5)(5,5) 
\psframe*[linecolor=white](-5,-5)(5,5) 
\psaxes[ticks=none, labels=none]{<->}(0,0)(-2.5,-2.5)(2.5,2.5)
%Function formula: -24 ((x) (x))+2 
\psplot[linecolor=blue, plotpoints=1000]{-0.427}{0.427}{2 x x mul -24 mul add }
\end{pspicture} 
\item 
\psset{xunit=1cm, yunit=1cm}
\begin{pspicture}(-5, -5)(5,5) 
\psframe*[linecolor=white](-5,-5)(5,5) 
\psaxes[ticks=none, labels=none]{<->}(0,0)(-2.5,-2.5)(2.5,2.5)
%Function formula: -4 ((x)^{3})+2 (x) 
\psplot[linecolor=blue, plotpoints=1000]{-1.045}{1.045}{x 2 mul x 3 exp -4 mul add }
\end{pspicture} 

\item 
\psset{xunit=1cm, yunit=1cm}
\begin{pspicture}(-5, -5)(5,5) 
\psframe*[linecolor=white](-5,-5)(5,5) 
\psaxes[ticks=none, labels=none]{<->}(0,0)(-2.5,-2.5)(2.5,2.5)
%Function formula: 1 
\psplot[linecolor=blue, plotpoints=1000]{1}{2.5}{1} 
\psHollowDotBlue{-1}{1}
%Function formula: -1 
\psHollowDotBlue{-1}{-1}
\psplot[linecolor=blue, plotpoints=1000]{-1}{1}{-1} 
\psHollowDotBlue{1}{-1}
%Function formula: 1 
\psHollowDotBlue{1}{1}
\psplot[linecolor=blue, plotpoints=1000]{-2.5}{-1}{1}
\end{pspicture} 

\end{enumerate}
\end{multicols}
Give reasons for your choices. Can you guess a formulas that would give a similar (or precisely the same) graph, and confirm visually your guess using a graphing device?
\end{problem}
\end{document}