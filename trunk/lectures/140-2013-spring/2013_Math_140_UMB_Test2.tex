\documentclass{article}
\usepackage{amsmath, amsfonts, amssymb, verbatim, hyperref}
\usepackage{auto-pst-pdf}
\usepackage{pst-plot}
%\usepackage{pst-solides3d}
%\usepackage{pst-3dplot}
\usepackage{pstricks}
\usepackage{rotating}

\usepackage{multicol}
\addtolength{\hoffset}{-3.5cm}
\addtolength{\textwidth}{6.8cm}
\addtolength{\voffset}{-3.3cm}
\addtolength{\textheight}{6.3cm}
\renewcommand{\Re}{\mathrm{Re~}}
\renewcommand{\Im}{\mathrm{Im~}}
\newcommand{\doublebrace}[4]{\left\{\begin{array}{ll} #1 & #2 \\#3 & #4  \end{array} \right.}
\newcommand{\triplebrace}[6]{\left\{\begin{array}{ll} #1 & #2 \\#3 & #4  \\#5 & #6\end{array} \right.}
\newtheorem{problem}{Problem}
\newcommand{\fcHollowDot}[2]{
\pscircle*[fillcolor=white, linecolor=red](#1, #2){0.07}
\pscircle*[fillcolor=white, linecolor=white](#1, #2){0.04}
}
\newcommand{\fcHollowDotBlue}[2]{
\pscircle*[fillcolor=white, linecolor=blue](#1, #2){0.07}
\pscircle*[fillcolor=white, linecolor=white](#1, #2){0.04}
}
\newcommand{\fcFullDot}[2]{
\pscircle*[fillcolor=white, linecolor=red](#1, #2){0.07}
}
\newcommand{\fcFullDotBlack}[2]{
\pscircle*[fillcolor=white, linecolor=black](#1, #2){0.07}
}
\newcommand{\fcLabelXOne}{\psline(1, -0.1)(1,0.1) \rput[t](1, -0.2 ) {\footnotesize $1$} }
\newcommand{\fcLabelYOne}{\psline(-0.1, 1)(0.1, 1) \rput[r](-0.2, 1 ) {\footnotesize $1$} }

%\title{Exam II\\ Math 140 Calculus I \\Instructor: Todor Milev}
%\date{April 9 2012}
\begin{document}
\begin{center}
\Large
Exam II\\ Math 140 Calculus I \\ \normalsize Instructor: Todor Milev
\end{center}

\noindent \textbf{Name:} \hfill{~}
\begin{tabular}{c|c|c|c|c|c|c|c|c|c|c||c}
Problem&1 &2&3&4&5&6&7&8&9&10& $\sum$\\\hline
Score&&&&&&&&&&&
\end{tabular}

\noindent The exam is closed books, no calculators allowed. 100 points are worth 100\% of the grade.
\begin{problem}(10 pts)  Define what it means for a function to be differentiable at a point. Define derivative of a function.
\end{problem}

\vskip 8cm

\begin{problem}(10 pts)
Compute the derivative of the function. Simplify your answer to a single fraction.
\[f(x)=\frac{1-2x }{1+\frac{3}x}
\]
\end{problem}
\newpage
\begin{problem}(10 pts)Compute the derivative of the function.
\[
f(x)=2^{3^{(\ln x)}}\quad \quad.
\]
\end{problem}
\vskip 9cm

\begin{problem}(10 pts)Compute the derivative of the function.
\[f(x)= \sec^3 \left(\frac{2}{x}\right).
\]
\end{problem}

\newpage

\begin{problem}(15 pts)Compute the limit.

\[\lim\limits_{x\to 0}
\frac{x^2}{\cos(8x) -1} \quad .
\]

\end{problem}
\vskip 9cm


\begin{problem}(10 pts)
Find the equation of the tangent line to the function
\[
y=x(\ln x)
\]
at the point $(1, 0)$.
\end{problem}

\newpage

\begin{problem}(10 pts)
Use implicit differentiation to express $\frac{dy}{dx}$ via $y $ and $x$, where $x$ and $y$ satisfy the following relation.
$x^4(x-y)=y^2(3x+y)$.
\end{problem}
\vskip 9cm

\begin{problem}(10 pts)
Use implicit differentiation to find an equation of the tangent line to the curve
\[
x^{5}+x^{3}y -y^6=-1
\]
at the point $(-1, -1)$.
\end{problem}
\newpage
\begin{problem}(15 pts)
Compute $\lim\limits_{x\to \infty} \left( 1- \frac{3}{x} \right)^x$.
\end{problem}
\vskip 9cm
The problems before this one sum up to 100\%.
\begin{problem}(10 pts)
For a positive integer $n$, prove the power rule $\frac{d}{dx}(x^n)=n x^{n-1} $ using the definition of derivative.
\end{problem}
\end{document}
