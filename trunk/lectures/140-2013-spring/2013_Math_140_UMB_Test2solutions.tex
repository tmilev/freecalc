\documentclass{article}
\usepackage{amsmath, amsfonts, amssymb, verbatim, hyperref}
\usepackage{auto-pst-pdf}
\usepackage{pst-plot}
%\usepackage{pst-solides3d}
%\usepackage{pst-3dplot}
\usepackage{pstricks}
\usepackage{rotating}
\usepackage{cancel}

\usepackage{multicol}
\addtolength{\hoffset}{-3.5cm}
\addtolength{\textwidth}{6.8cm}
\addtolength{\voffset}{-3.3cm}
\addtolength{\textheight}{6.3cm}
\renewcommand{\Re}{\mathrm{Re~}}
\renewcommand{\Im}{\mathrm{Im~}}
\newcommand{\doublebrace}[4]{\left\{\begin{array}{ll} #1 & #2 \\#3 & #4  \end{array} \right.}
\newcommand{\triplebrace}[6]{\left\{\begin{array}{ll} #1 & #2 \\#3 & #4  \\#5 & #6\end{array} \right.}
\newtheorem{problem}{Problem}
\newcommand{\psHollowDot}[2]{
\pscircle*[fillcolor=white, linecolor=red](#1, #2){0.07}
\pscircle*[fillcolor=white, linecolor=white](#1, #2){0.04}
}
\newcommand{\psHollowDotBlue}[2]{
\pscircle*[fillcolor=white, linecolor=blue](#1, #2){0.07}
\pscircle*[fillcolor=white, linecolor=white](#1, #2){0.04}
}
\newcommand{\psFullDot}[2]{
\pscircle*[fillcolor=white, linecolor=red](#1, #2){0.07}
}
\newcommand{\psFullDotBlack}[2]{
\pscircle*[fillcolor=white, linecolor=black](#1, #2){0.07}
}
\newcommand{\psLabelXOne}{\psline(1, -0.1)(1,0.1) \rput[t](1, -0.2 ) {\footnotesize $1$} }
\newcommand{\psLabelYOne}{\psline(-0.1, 1)(0.1, 1) \rput[r](-0.2, 1 ) {\footnotesize $1$} }

%\title{Exam II\\ Math 140 Calculus I \\Instructor: Todor Milev}
%\date{April 9 2012}
\begin{document}
\begin{center}
\Large
Exam II\\ Math 140 Calculus I \\ \normalsize Instructor: Todor Milev
\end{center}

\noindent \textbf{Name:} \hfill{~}
\begin{tabular}{c|c|c|c|c|c|c|c|c|c|c||c}
Problem&1 &2&3&4&5&6&7&8&9&10& $\sum$\\\hline
Score&&&&&&&&&&&
\end{tabular} 

\noindent The exam is closed books, no calculators allowed. 100 points are worth 100\% of the grade. 
\begin{problem}(10 pts)  Define what it means for a function to be differentiable at a point. Define derivative of a function.
\end{problem}
\textbf{Solution.} By definition, a function $f$ is differentiable at a point $x$ if the limit $\lim_{h\to 0} \frac{f(x+h)-f(x)}{h}=0$ exists. If this the case, the latter limit is called the derivative  of $f$ at $x$. 

\vskip 8cm

\begin{problem}(10 pts)
Compute the derivative of the function. Simplify your answer to a single fraction.
\[f(x)=\frac{1-2x }{1+\frac{3}x}
\]
\end{problem}
\textbf{Solution.} 
\[
\begin{array}{rcl}
\left(\frac{1-2x }{1+\frac{3}x}\right)'&=&\frac{(1-2x)' (1+\frac{3}{x})-(1-2x)(1+3x^{-1})' }{\left(1+\frac{3}x\right)^2}\\
&=& \frac{-2(1+3x^{-1})- (1-2x)(-3)x^{-2} }{x^{-2}(x+3)^2 }\\
&=&\frac{-2-6x^{-1}+3x^{-2}-6x^{-1}}{x^{-2} (x+3)^2}= \frac{\cancel{x^{-2}}(-2x^2-12x+3)}{\cancel{x^{-2}}(x+3)^2}\\
&=&\frac{-2x^2-12x+3}{(x+3)^2}
\end{array}
\]
\newpage
\begin{problem}(10 pts)Compute the derivative of the function.
\[
f(x)=2^{3^{(\ln x)}}\quad \quad.
\]
\end{problem}
\textbf{Solution.}  We studied in class that $(p^{x})'=p^x \ln p$. Therefore we can apply consecutively the chain rule to get
\[
\left(2^{3^{(\ln x)}}\right)'= 2^{3^{(\ln x)}} (\ln 2) \left(3^{(\ln x)}\right)'=2^{3^{(\ln x)}} (\ln 2) 3^{\ln x} (\ln 3) \underbrace{(\ln x)'}_{=\frac{1}{x}}=\frac{\ln 2\ln 3}{x} 2^{3^{(\ln x)}} 3^{\ln x}\quad .
\]

\vskip 9cm

\begin{problem}(10 pts)Compute the derivative of the function.
\[f(x)= \sec^3 \left(\frac{2}{x}\right).
\]
\end{problem}
\textbf{Solution. } As studied in class, $(\sec x)'=\left(\frac{1}{\cos x}\right)'=\frac{-(\cos x)'}{\cos^2 x}=\frac{\sin x}{\cos^2 x}= \sec x \tan x $. Therefore
\[
\sec^3 \left(\frac{2}{x}\right)'= 3\sec^2\left( \frac{2}{x} \right) \left(\sec\left( \frac{2}{x} \right)\right)'= 3\sec^2\left( \frac{2}{x} \right) \sec \left( \frac{2}{x} \right) \tan \left( \frac{2}{x} \right) \left( \frac{2}{x} \right)'=-\frac{6}{x^2}\sec^3\left( \frac{2}{x} \right) \tan \left( \frac{2}{x} \right) \quad .
\]
\newpage

\begin{problem}(15 pts)Compute the limit.

\[\lim\limits_{x\to 0} 
\frac{x^2}{\cos(8x) -1} \quad .
\] 

\end{problem}
\textbf{Solution. } We have that $\cos (8x) = 1-2\sin^2(4x)$. As studied in class, $\lim\limits_{y\to 0} \frac{\sin y}{y}=1$ and therefore 

\[\lim\limits_{x\to 0} 
\frac{x^2}{\cos(8x) -1} =\lim\limits_{x\to 0} 
\frac{x^2}{\cancel{1}-2\sin^2(4x) -\cancel{1}}=\lim\limits_{x\to 0} 
\frac{\frac{(4x)^2}{16}}{-2\sin^2(4x) }=-\frac{1}{32}\lim\limits_{x\to 0} 
\frac{(4x)^2}{\sin^2(4x) }=-\frac{1}{32}\lim\limits_{x\to 0} 
\frac{1}{\frac{\sin^2(4x)}{(4x)^2} }=-\frac{1}{32}\quad .
\] 
\vskip 9cm


\begin{problem}(10 pts)
Find the equation of the tangent line to the function 
\[
y=x(\ln x) 
\] 
at the point $(1, 0)$. 
\end{problem}
\textbf{Solution.}
\[
\frac{dy}{dx}= (x\ln x)'=(x)'\ln x+ x(\ln x)'=\ln x +x \frac{1}{x}= \ln x +1\quad .
\]
Therefore $\frac{dy}{dx}_{| x=1}= \ln 1 +1 = 1$. Therefore the equation of the tangent line to the graph of $y=x(\ln x) $ at $(1,0)$ is $(y-0)= 1\times (x-1)$, or in other words, $y=x-1$.
\newpage

\begin{problem}(10 pts)
Use implicit differentiation to express $\frac{dy}{dx}$ via $y $ and $x$, where $x$ and $y$ satisfy the following relation.
$x^4(x-y)=y^2(3x+y)$.
\end{problem}
\textbf{Solution.}
\[
\begin{array}{rcl}
x^4(x-y)&=&y^2(3x+y)\\
x^5-x ^4y&=&3y^2x+y^3 ~~~~~~~~~~~~~~~ ~~~~~~~~~~~~~~~\frac{d}{dx}\\
5x^4-4x^3y-x^4\frac{dy}{dx}&=&3y^2+6xy\frac{dy}{dx} +3y^2 \frac{dy}{dx}\\
\frac{dy}{dx}\left(x^4+6xy+3y^2 \right)&=&5x^4-4x^3y-3y^2\\
\frac{dy}{dx}&=&\frac{ 5x^4-4x^3y-3y^2}{x^4+6xy+3y^2}\quad .
\end{array}
\]
\vskip 9cm

\begin{problem}(10 pts)
Use implicit differentiation to find an equation of the tangent line to the curve 
\[
x^{5}+x^{3}y -y^6=-1
\] 
at the point $(-1, -1)$.
\end{problem}
\textbf{Solution.}
Direct substitution shows that $(-1, -1)$ satisfies the above equation.
\[
\begin{array}{rcl}
x^{5}+x^{3}y -y^6&=&-1 ~~~~~~~~~~~~~~~ ~~~~~~~~~~~~~~~\frac{d}{dx}\\
5x^4+3x^2y+x^3\frac{dy}{dx} - 6y^5\frac{dy}{dx}&=&0
\end{array}
\]
We can substitute $x=-1, y=-1$ in the above expression to get
\[
\begin{array}{rcl}
5(-1)^4+3(-1)^2(-1)+(-1)^3\frac{dy}{dx}_{|x=-1}-6(-1)^5\frac{dy}{dx}_{|x=-1}&=&0\\
5-3-\frac{dy}{dx}_{|x=-1}+6\frac{dy}{dx}_{|x=-1}&=&0\\
\frac{dy}{dx}_{|x=-1}&=&-\frac{2}{5}\quad .
\end{array}
\]
Therefore the equation of the tangent line at the point $(-1, -1)$ is 
$y-(-1) =-\frac{2}{5}(x-(-1))$ or $y= -\frac{2}{5}x-\frac{7}{5}$.

\newpage 
\begin{problem}(15 pts)
Compute $\lim\limits_{x\to \infty} \left( 1- \frac{3}{x} \right)^x$. 
\end{problem}
\textbf{Solution.} As studied in class, $\lim\limits_{x\to 0 }\left(1+x\right)^{\frac{1}{x}}=e$. Set $t=-\frac{3}{x}$. Therefore as $x\to \infty$, $t\to 0^-$, and we have
\[
\lim_{\substack{ x\to \infty \\ t=-\frac{3}{x} \\ t\to 0^-}} \left( 1- \frac{3}{x} \right)^x= \lim_{t\to 0^-} \left( 1 +t \right)^{-\frac{3}{t}}=\lim\limits_{t\to 0^-} \left(\left(1+t \right)^{\frac{1}{t}}\right)^{-3}= \left(\lim\limits_{t\to 0^-}\left(1+t \right)^{\frac{1}{t}}\right)^{-3}=e^{-3}=\frac{1}{e^3}\quad .
\]

\vskip 9cm
The problems before this one sum up to 100\%.
\begin{problem}(10 pts)
For a positive integer $n$, prove the power rule $\frac{d}{dx}(x^n)=n x^{n-1} $ using the definition of derivative.
\end{problem}
\textbf{Solution.}
By the definition of derivative we have
\[\begin{array}{rcl}
\frac{d}{dx}(x^n)&=& \lim\limits_{h\to 0}\frac{(x+h)^n-x^n}{h}=\lim\limits_{h\to 0} \frac{(\cancel{x}+h-\cancel{x}) ((x+h)^{n-1}+ (x+h)^{n-2}x +\dots +(x+h)x^{n-2}+ x^{n-1}) }{h}\\
&=&\lim\limits_{h\to 0} \left(\underbrace{ (x+h)^{n-1}+ (x+h)^{n-2}x +\dots +(x+h)x^{n-2}+ x^{n-1}}_{n~\textrm{summands~total}}\right) = nx^{n-1}\quad .
\end{array}
\]
\end{document}