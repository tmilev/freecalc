\documentclass%
%[handout]
{beamer}
% % % % % % % %
% % % % % % % %
% % % % % % % %
%IMPORTANT
%compiles with 
%pdflatex -shell-escape 
%IMPORTANT
% % % % % % % %
% % % % % % % %
% % % % % % % %
\mode<presentation>
{
\useinnertheme{rounded}
\useoutertheme{infolines}
\usecolortheme{orchid}
\usecolortheme{whale}
}

\usepackage[english]{babel}
\usepackage[latin1]{inputenc}
\usepackage[all,cmtip]{xy}
\usepackage{times}
\usepackage[T1]{fontenc}
\usepackage{../example-templates}
\usepackage{auto-pst-pdf}
\usepackage{pst-plot}

% Or whatever. Note that the encoding and the font should match. If T1
% does not look nice, try deleting the line with the fontenc.

\graphicspath{{../../modules/}}

\newtheoremstyle{partialproof}{3pt}{3pt}{}{}{}{.}{.5em}{}
\theoremstyle{partialproof} \newtheorem{partialproof}[theorem]{Proof.}
%\DeclareMathOperator{\diff}{d}
\newcommand{\diff}{\text{d}}
\setbeamertemplate{navigation symbols}{}

\includeonlylecture{1}

\newcommand{\lect}[3]{
  \date{#1}
  \lecture[#1]{#2}{#3}
}

\setbeamertemplate{footline}
{
  \leavevmode%
  \hbox{%
  \begin{beamercolorbox}[wd=.333333\paperwidth,ht=2.25ex,dp=1ex,center]{author in head/foot}%
    \usebeamerfont{author in head/foot}\insertshortauthor
  \end{beamercolorbox}%
  \begin{beamercolorbox}[wd=.333333\paperwidth,ht=2.25ex,dp=1ex,center]{title in head/foot}%
    \usebeamerfont{title in head/foot}\insertshorttitle
  \end{beamercolorbox}%
  \begin{beamercolorbox}[wd=.333333\paperwidth,ht=2.25ex,dp=1ex,center]{date in head/foot}%
    \usebeamerfont{date in head/foot}\insertshortdate{}
  \end{beamercolorbox}}%
  \vskip0pt%
}

% If you have a file called "university-logo-filename.xxx", where xxx
% is a graphic format that can be processed by latex or pdflatex,
% resp., then you can add a logo as follows:

%\pgfdeclareimage[height=0.8cm]{logo}{bluelogo}
%\logo{\pgfuseimage{logo}}

\begin{document}
\newcommand{\psHollowDot}[2]{
\pscircle*[fillcolor=white, linecolor=red](#1, #2){0.07}
\pscircle*[fillcolor=white, linecolor=white](#1, #2){0.04}
}
\newcommand{\psHollowDotBlue}[2]{
\pscircle*[fillcolor=white, linecolor=blue](#1, #2){0.07}
\pscircle*[fillcolor=white, linecolor=white](#1, #2){0.04}
}
\newcommand{\psFullDot}[2]{
\pscircle*[fillcolor=white, linecolor=red](#1, #2){0.07}
}
\newcommand{\psFullDotBlack}[2]{
\pscircle*[fillcolor=white, linecolor=black](#1, #2){0.07}
}
\newcommand{\psLabelXOne}{\psline(1, -0.1)(1,0.1) \rput[t](1, -0.2 ) { $1$} }
\newcommand{\psLabelYOne}{\psline(-0.1, 1)(0.1, 1) \rput[r](-0.2, 1 ) { $1$} }

\AtBeginLecture{%

\title[\insertlecture]{FreeCalc}
\subtitle{\insertlecture}
\author[FreeCalc]{}
\institute[UMass Boston]{University of Massachusetts Boston}
\date{\insertshortlecture}
\begin{frame}
  \titlepage
\end{frame}
}%

% begin lecture
\lect{\today}{Sample}{1}
% begin module higher-derivatives-ex6
\begin{frame}
\begin{example} %[Example 7, p. 153]
If $f(x) = x^3-x$, find $f''(x)$.
\begin{columns}[c]
\column{.4\textwidth}

\psset{xunit=1.2cm, yunit=1.2cm}
\begin{pspicture}(-5, -5)(5,5) 
\psframe*[linecolor=white](-5,-5)(5,5) 
\tiny
\psaxes[ticks=none, labels=none]{<->}(0,0)(-2,-2)(2,2)
\psLabelXOne
\psLabelYOne
\uncover<8->{
%Function formula: 6 (x) 
\psplot[linecolor=green, plotpoints=1000]{-0.333333333}{0.333333333}{x 6 mul } 
\rput[br](-0.4,-2){$f''(x)$}
}
\uncover<2->{
%Function formula: 3 ((x)^{2})-1 
\psplot[linecolor=blue, plotpoints=1000]{-1}{1}{-1 x 2 exp 3 mul add } 
\rput[r](-1, 1){$f'(x)$}
}
%Function formula: - (x)+(x)^{3} 
\psplot[linecolor=red, plotpoints=1000]{-1.521379707}{1.521379707}{x 3 exp x -1 mul add }
\rput[bl](1.2, 0.1){$f(x)$}
\end{pspicture} 
\column{.6\textwidth}
\uncover<2->{%
In a previous exercise we found that the first derivative is $f'(x) = 3x^2 - 1$.
}%
\abovedisplayskip=0pt
\belowdisplayskip=0pt
\begin{align*}
&  \uncover<3->{f''(x)}\\%
 & \uncover<3->{ = }  %
\uncover<3->{\lim_{h\rightarrow 0}\frac{f'(x+h)-f'(x)}{h}}\\%
 & \uncover<4->{ = }  %
\uncover<4->{\lim_{h\rightarrow 0}\frac{[3(x+h)^2-1]-[3x^2-1]}{h}}\\%
 & \uncover<5->{ = }  %
\uncover<5->{\lim_{h\rightarrow 0}\frac{3x^2 + 6xh + 3h^2 -1 - 3x^2 +1}{h}}\\%
 & \uncover<6->{ = }  %
\uncover<6->{\lim_{h\rightarrow 0}\frac{6xh + 3h^2}{h}}\\%
 & \uncover<7->{ = }  %
\uncover<7->{\lim_{h\rightarrow 0}(6x + 3h)}\uncover<8->{ = 6x}%
\end{align*}
\end{columns}
\end{example}
\end{frame}
% end module higher-derivatives-ex6

% end lecture

\end{document}
