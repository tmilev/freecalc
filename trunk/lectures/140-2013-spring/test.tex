\documentclass%
%[handout]
{beamer}
% % % % % % % %
% % % % % % % %
% % % % % % % %
%IMPORTANT
%compiles with 
%pdflatex -shell-escape 
%IMPORTANT
% % % % % % % %
% % % % % % % %
% % % % % % % %
\mode<presentation>
{
\useinnertheme{rounded}
\useoutertheme{infolines}
\usecolortheme{orchid}
\usecolortheme{whale}
}

\usepackage[english]{babel}
\usepackage[latin1]{inputenc}
\usepackage[all,cmtip]{xy}
\usepackage{times}
\usepackage[T1]{fontenc}
\usepackage{../example-templates}
\usepackage{auto-pst-pdf}
\usepackage{pst-plot}
\usepackage{cancel}


% Or whatever. Note that the encoding and the font should match. If T1
% does not look nice, try deleting the line with the fontenc.

\graphicspath{{../../modules/}}

\newtheoremstyle{partialproof}{3pt}{3pt}{}{}{}{.}{.5em}{}
\theoremstyle{partialproof} \newtheorem{partialproof}[theorem]{Proof.}
%\DeclareMathOperator{\diff}{d}
\newcommand{\diff}{\text{d}}
\setbeamertemplate{navigation symbols}{}

\includeonlylecture{1}

\newcommand{\lect}[3]{
  \date{#1}
  \lecture[#1]{#2}{#3}
}

\setbeamertemplate{footline}
{
  \leavevmode%
  \hbox{%
  \begin{beamercolorbox}[wd=.333333\paperwidth,ht=2.25ex,dp=1ex,center]{author in head/foot}%
    \usebeamerfont{author in head/foot}\insertshortauthor
  \end{beamercolorbox}%
  \begin{beamercolorbox}[wd=.333333\paperwidth,ht=2.25ex,dp=1ex,center]{title in head/foot}%
    \usebeamerfont{title in head/foot}\insertshorttitle
  \end{beamercolorbox}%
  \begin{beamercolorbox}[wd=.333333\paperwidth,ht=2.25ex,dp=1ex,center]{date in head/foot}%
    \usebeamerfont{date in head/foot}\insertshortdate{}
  \end{beamercolorbox}}%
  \vskip0pt%
}

% If you have a file called "university-logo-filename.xxx", where xxx
% is a graphic format that can be processed by latex or pdflatex,
% resp., then you can add a logo as follows:

%\pgfdeclareimage[height=0.8cm]{logo}{bluelogo}
%\logo{\pgfuseimage{logo}}

\begin{document}
\newcommand{\psHollowDot}[2]{
\pscircle*[fillcolor=white, linecolor=red](#1, #2){0.07}
\pscircle*[fillcolor=white, linecolor=white](#1, #2){0.04}
}
\newcommand{\psHollowDotBlue}[2]{
\pscircle*[fillcolor=white, linecolor=blue](#1, #2){0.07}
\pscircle*[fillcolor=white, linecolor=white](#1, #2){0.04}
}
\newcommand{\psFullDot}[2]{
\pscircle*[fillcolor=white, linecolor=red](#1, #2){0.07}
}
\newcommand{\psFullDotBlack}[2]{
\pscircle*[fillcolor=white, linecolor=black](#1, #2){0.07}
}
\newcommand{\psFullDotBlue}[2]{
\pscircle*[fillcolor=white, linecolor=blue](#1, #2){0.07}
}
\newcommand{\psLabelXOne}{\psline(1, -0.1)(1,0.1) \rput[t](1, -0.2 ) { $1$} }
\newcommand{\psLabelYOne}{\psline(-0.1, 1)(0.1, 1) \rput[r](-0.2, 1 ) { $1$} }

\AtBeginLecture{%

\title[\insertlecture]{FreeCalc}
\subtitle{\insertlecture}
\author[FreeCalc]{}
\institute[UMass Boston]{University of Massachusetts Boston}
\date{\insertshortlecture}
\begin{frame}
  \titlepage
\end{frame}
}%

% begin lecture
\lect{\today}{Sample}{1}
% begin module linearization-def
\begin{frame}
\begin{definition}[Linearization of $f$ at $a$]
The linear function whose graph is the tangent line at $(a,f(a))$ is called the linearization of $f$ at $a$.  Its equation is
\[
L(x) = f(a) + f'(a)(x-a).
\]
\end{definition}
\begin{definition}[Linear Approximation of $f(x)$ near $a$]
The approximation
\[
f(x) \approx f(a) + f'(a)(x-a)
\]
is called the linear approximation of $f$ at $a$.
\end{definition}
Let $y=f(x)$, $\Delta y:= f(x)-f(a)$, and $\Delta x:= x-a$.
\begin{definition}[Linear approx. $y=f(x)$ near $a$, alternative notation]
\[
\Delta y\approx \frac{d y}{d x}\Delta x\quad .
\]
\end{definition}
\end{frame}
\begin{frame}
\frametitle{Linear approximations}
\begin{center}%

\psset{xunit=1.5cm, yunit=0.8cm}
\begin{pspicture}(-5, -5)(5,5)
\tiny
\psframe*[linecolor=white](-5,-5)(5,5)
\psaxes[ticks=none, labels=none]{<->}(0,0)(-0.5,-0.5)(5,4.5)
%Function formula: 1/2+x
\psplot[linecolor=blue, plotpoints=1000]{-0.2}{4.3}{x 0.5 add }
\rput(3,4.2){$y=L(x)$}
 %Function formula: 7/2- ((-2+1/2 (x))^{2})
\rput(4.3,3.1){$y=f(x)$}
\psplot[linecolor=black, plotpoints=1000]{-0.2}{5}{x 0.5 mul -2 add 2 exp -1 mul 3.5 add }
\rput[br](2,2.6){$(x,f(x))$}
\fcFullDotBlack{2}{2.5}
\psline[linestyle=dashed](2, 2.5)(2,0)
\psline[linestyle=dashed](3, 3.25)(3,0)

\only<1>{\psline(2, 2.5)(3, 2.5) (3, 3.25)}
\only<2>{\psline[linecolor=red](2, 2.5)(3, 2.5) (3, 3.25)}
\only<3>{\psline[linecolor=red](2, 2.5)(3, 2.5) (3, 3.5)}

\rput[t](2.5, 2.4 ){\alertNoH{2-3}{$\Delta x$}}

\only<1-2>{ \rput[l](3.1,2.875){\alertNoH{2}{$\Delta f$}}
}
\only<3>{ \rput[l](3.1,3){\alertNoH{3}{$\Delta L$}}
}

\rput[t](2,-0.1){$x$}
\rput[t](3,-0.1){$x+\Delta x$}
\end{pspicture}

\begin{tabular}{|l|c|c|}
\hline
Function & \alert<handout:1| 2>{$f$} & \alert<handout:2| 3>{$L$}\\
\hline
Run & \alert<handout:1| 2>{$\Delta x$} & \alert<handout:2| 3>{$\Delta x$}\\
Rise & \alert<handout:1| 2>{$\Delta f$} & \alert<handout:2| 3>{$\Delta L$}\\
Formula & \alert<handout:1| 2>{$\Delta f = f(x+\Delta x) - f(x)$} & \alert<handout:2| 3>{$\Delta L =(\Delta x)  f'(x) $}\\
\hline
\end{tabular}
\end{center}%

\end{frame}
% end module linearization-def

% begin module differential-def
\begin{frame}


\frametitle{Differentials}
\begin{itemize}
\item<1-> From previous slides:
\[ \only<1-2, 7->{\Delta y} \only<3-6>{\alert<3,6> {dy}} \only<1-3, 7->{\approx}\only<4-6>{\alert<4> =} \only<1-5,7->{\frac{dy}{dx}}
\only<6>{{\frac{dy}{\cancel{dx}}}}
\only<1-4, 7->{ \Delta x} 
\only<5>{\alert<5>{dx}} 
\only<6>{\alert<5>{\cancel{dx}}} 
\only<6>{=\alert<6>{dy} }
\]
\item<2-> If in the above we \alert<3-6>{formally substitute} \alert<3>{$\Delta y $ by $dy$}, \alert<4>{$=$ by $\approx$} and \alert<5>{$\Delta x$ by $dx$}, we get a \alert<6>{formal identity}.
\item<7-> Define formally the \alert<8>{\emph{differential operator $d$}} and the \alert<9>{\emph{differential form $dx$}} by requesting that $d$ and $dx$ satisfy the transformation law 
\[
\alert<8>{d}(f(x))=f'(x) \alert<9>{ \alert<8>{d}x}
\] for any differentiable function $f(x)$. In abbreviated notation:
\[ \alert<8>{d}f = f' \alert<9>{ \alert<8>{d}x}
\]
\item<10-> Differential forms express the idea of approximating differentiable functions by linear.
\item<11-> The strict definition of differential forms  is outside of the scope of Calc I and II.
\end{itemize}
\end{frame}





%\begin{frame}
%\begin{itemize}
%\item
% The rest of this slide is for your information only; you will not be tested on it.
%\item To define differentials strictly:
%\begin{itemize}
%\item Let $\xi$ be a function mapping functions that have first derivative to functions that have first derivative. 
%\item Define  $\xi$ to be a \emph{differential operator} if it satisfies the Leibniz rule: $\xi (f g)= \xi (f) g+ f\xi (g) $ for any two functions $f,g$ with first derivative evaluated at an arbitrary point $x$.
%\item Define $w $ to be a differential operator 
%\end{itemize} 

%\item  $\diff y$ depends on $x$ and $\diff x$.
%\item  It measures the ``rise'' for the linear approximation $L(x)$:
%\end{itemize}
%\begin{align*}
%\uncover<1->{%
%\diff y%
%}%
%& \uncover<1->{ = } %
%\uncover<1->{%
%L(x+\diff x) - L(x)%
%}%
%\\%
%& \uncover<2->{ = } &%
%\uncover<2->{%
%[f(x) + f'(x)(x+\diff x - x)] - [f(x) + f'(x)(x-x)]%
%}\\%
%& \uncover<2->{ = } &%
%\uncover<2->{%
%f'(x)\diff x%
%}\\%
%\end{align*}
%\end{frame}

% end module differential-def

% end lecture

\end{document}
