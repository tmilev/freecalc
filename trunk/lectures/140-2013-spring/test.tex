\documentclass%
%[handout]
{beamer}
% % % % % % % %
% % % % % % % %
% % % % % % % %
%IMPORTANT
%compiles with 
%pdflatex -shell-escape 
%IMPORTANT
% % % % % % % %
% % % % % % % %
% % % % % % % %
\mode<presentation>
{
\useinnertheme{rounded}
\useoutertheme{infolines}
\usecolortheme{orchid}
\usecolortheme{whale}
}

\usepackage[english]{babel}
\usepackage[latin1]{inputenc}
\usepackage[all,cmtip]{xy}
\usepackage{times}
\usepackage[T1]{fontenc}
\usepackage{../example-templates}
\usepackage{cancel}

\usepackage{auto-pst-pdf}
\usepackage{pst-plot}
%\usepackage{pstricks-add} 

% Or whatever. Note that the encoding and the font should match. If T1
% does not look nice, try deleting the line with the fontenc.

\graphicspath{{../../modules/}}

\newtheoremstyle{partialproof}{3pt}{3pt}{}{}{}{.}{.5em}{}
\theoremstyle{partialproof} \newtheorem{partialproof}[theorem]{Proof.}
%\DeclareMathOperator{\diff}{d}
\newcommand{\diff}{\text{d}}
\setbeamertemplate{navigation symbols}{}

\includeonlylecture{1}

\newcommand{\lect}[3]{
  \date{#1}
  \lecture[#1]{#2}{#3}
}

\setbeamertemplate{footline}
{
  \leavevmode%
  \hbox{%
  \begin{beamercolorbox}[wd=.333333\paperwidth,ht=2.25ex,dp=1ex,center]{author in head/foot}%
    \usebeamerfont{author in head/foot}\insertshortauthor
  \end{beamercolorbox}%
  \begin{beamercolorbox}[wd=.333333\paperwidth,ht=2.25ex,dp=1ex,center]{title in head/foot}%
    \usebeamerfont{title in head/foot}\insertshorttitle
  \end{beamercolorbox}%
  \begin{beamercolorbox}[wd=.333333\paperwidth,ht=2.25ex,dp=1ex,center]{date in head/foot}%
    \usebeamerfont{date in head/foot}\insertshortdate{}
  \end{beamercolorbox}}%
  \vskip0pt%
}

% If you have a file called "university-logo-filename.xxx", where xxx
% is a graphic format that can be processed by latex or pdflatex,
% resp., then you can add a logo as follows:

%\pgfdeclareimage[height=0.8cm]{logo}{bluelogo}
%\logo{\pgfuseimage{logo}}

\begin{document}
\newcommand{\psHollowDot}[2]{
\pscircle*[fillcolor=white, linecolor=red](#1, #2){0.07}
\pscircle*[fillcolor=white, linecolor=white](#1, #2){0.04}
}
\newcommand{\psHollowDotBlue}[2]{
\pscircle*[fillcolor=white, linecolor=blue](#1, #2){0.07}
\pscircle*[fillcolor=white, linecolor=white](#1, #2){0.04}
}
\newcommand{\psFullDot}[2]{
\pscircle*[fillcolor=white, linecolor=red](#1, #2){0.07}
}
\newcommand{\psFullDotBlack}[2]{
\pscircle*[fillcolor=white, linecolor=black](#1, #2){0.07}
}
\newcommand{\psFullDotBlue}[2]{
\pscircle*[fillcolor=white, linecolor=blue](#1, #2){0.07}
}
\newcommand{\psLabelXOne}{\psline(1, -0.1)(1,0.1) \rput[t](1, -0.2 ) { $1$} }
\newcommand{\psLabelYOne}{\psline(-0.1, 1)(0.1, 1) \rput[r](-0.2, 1 ) { $1$} }

\AtBeginLecture{%

\title[\insertlecture]{FreeCalc}
\subtitle{\insertlecture}
\author[FreeCalc]{}
\institute[UMass Boston]{University of Massachusetts Boston}
\date{\insertshortlecture}
\begin{frame}
  \titlepage
\end{frame}
}%

% begin lecture
\lect{\today}{Sample}{1}
%% begin module optimization-ex5
\begin{frame}
\begin{example}
Find the largest possible area of a rectangle inscribed in a semicircle of radius $r$.
\begin{columns}[c]
\column{.5\textwidth}
\psset{xunit=1.8cm, yunit=1.8cm}
\begin{pspicture}(0,0)(1,1)
\psframe*[linecolor=white](-5,-5)(5,5) 
\psaxes[ticks=none, labels=none]{<->}(0,0)(-1.5,-0.5)(1.5,1.5)
\tiny
%Function formula: sqrt{}(1- ((x)^{2})) 
\psplot[linecolor=black, plotpoints=1000]{-1}{1}{x 2 exp -1 mul 1 add sqrt }
\rput[t](1,-0.1){$r$}
\rput[t](-1,-0.1){$-r$}

\uncover<10->{
\psline[linecolor=red](-0.866025404, 0.5,)(0.866025404, 0.5)
\pcline[offset=-5pt]{<->}(-0.866025404, 0.5)(0.866025404, 0.5)
\lput*{:U}(0.7){$2x$}
\psline[linecolor=red](-0.866025404, 0)(-0.866025404, 0.5)(0.866025404, 0.5)(0.866025404, 0)
\psFullDot{0.866025404}{0.5}
\rput[bl](0.866025404,0.6){$(x,y)$}
}
\uncover<2>{
\psline[linecolor=red](-0.9, 0) (-0.9, 0.43589) (0.9, 0.43589) (0.9, 0)
}
\uncover<3>{
\psline[linecolor=red](-0.8, 0) (-0.8, 0.6) (0.8, 0.6) (0.8, 0)
}
\uncover<4>{
\psline[linecolor=red](-0.7, 0) (-0.7, 0.714143) (0.7, 0.714143) (0.7, 0)
}
\uncover<5>{
\psline[linecolor=red](-0.6, 0) (-0.6, 0.8) (0.6, 0.8) (0.6, 0)
}
\uncover<6>{
\psline[linecolor=red](-0.5, 0) (-0.5, 0.866025) (0.5, 0.866025) (0.5, 0)
}
\uncover<7>{
\psline[linecolor=red](-0.4, 0) (-0.4, 0.916515) (0.4, 0.916515) (0.4, 0)
}
\uncover<8>{
\psline[linecolor=red](-0.3, 0) (-0.3, 0.953939) (0.3, 0.953939) (0.3, 0)
}
\uncover<9>{
\psline[linecolor=red](-0.2, 0) (-0.2, 0.979796) (0.2, 0.979796) (0.2, 0)
}
\end{pspicture}

\uncover<12->{To eliminate $y$, use the fact that $(x,y)$ lies on the semicircle.}%
\abovedisplayskip=0pt
\belowdisplayskip=0pt
\abovedisplayshortskip=0pt
\belowdisplayshortskip=0pt
\begin{align*}
\uncover<12->{y^2} & \uncover<12->{=}  \uncover<12->{r^2-x^2}\\
\uncover<13->{\alert<handout:0| 15>{y}} & \uncover<13->{\alert<handout:0| 15>{=}}  \uncover<13->{\alert<handout:0| 15>{\sqrt{r^2-x^2}}}%
\end{align*}
%\uncover<20->{%
%\abovedisplayskip=0pt
%\belowdisplayskip=0pt
%\[
%\begin{array}{r|r}
%x & A(x)\\
%\hline
%\alert<handout:0| 21-22>{0} & \alert<handout:0| 22>{\uncover<22->{0}}\\
%\alert<handout:0| 23-24,27>{600} & \alert<handout:0| 24,27>{\uncover<24->{720,000}}\\
%\alert<handout:0| 25-26>{2400} & \uncover<26->{\alert<handout:0| 26>{0}}
%\end{array}
%\]
%}%
\column{.5\textwidth}
\uncover<10->{%
Let the semicircle have center at the origin.  Let $(x,y)$ be the coordinates of the top right corner of the rectangle.  Let $A$ be its area.
}%

\uncover<14->{Notice that $0\leq x \leq r$.}
\abovedisplayskip=0pt
\belowdisplayskip=0pt
\abovedisplayshortskip=0pt
\belowdisplayshortskip=0pt
\begin{align*}
\uncover<11->{A} & \uncover<11->{=}  \uncover<11->{2x\alert<handout:0| 15>{y}}  \uncover<15->{=}  \uncover<15->{2x\alert<handout:0| 15>{\sqrt{r^2-x^2}}}\\
\uncover<16->{\alert<handout:0| 16-17>{A'}} & \uncover<16->{\alert<handout:0| 16-17>{=}}  \uncover<17->{\alert<handout:0| 17>{2\sqrt{r^2-x^2} - \frac{2x^2}{\sqrt{r^2-x^2}}}}\\
& \uncover<18->{=}  \uncover<18->{\frac{2(r^2-2x^2)}{\sqrt{r^2-x^2}}}
\end{align*}

\uncover<19->{%
Critical number: \alert<handout:0| 19-20>{$x = $ \uncover<20->{$\frac{r}{\sqrt{2}}$.}} 
}%
\end{columns}
\uncover<21->{%
There is a local max. here because $A(0) = 0 = A(r)$.  Therefore the maximum area is $A(\frac{r}{\sqrt{2}}) = $ $ 2\frac{r}{\sqrt{2}}\sqrt{r^2 - \frac{r^2}{2}} = r^2$, achieved for $x=y=\frac{r}{\sqrt{2}}$
}%
\end{example}
\end{frame}
% end module optimization-ex5

% begin module area-def
\begin{frame}
Estimate the area under the curve $y = f(x)$ between $a$ and $b$.
\begin{columns}
\column{.55\textwidth}
\psset{xunit=3cm, yunit=3cm}
\begin{pspicture}(-5, -5)(5,5) 
\psframe*[linecolor=white](-5,-5)(5,5) 
\psaxes[ticks=none, labels=none]{<->}(0,0)(-0.1,-0.1)(1.7,0.9)
\tiny

\uncover<1>{
\psline*[linecolor=cyan, linewidth=0.1pt](0.425, 0)(0.425, 0.339931)(0.3, 0.339931)(0.3, 0)(0.55, 0)(0.55, 0.316508)(0.425, 0.316508)(0.425, 0)(0.675, 0)(0.675, 0.348456)(0.55, 0.348456)(0.55, 0)(0.8, 0)(0.8, 0.416)(0.675, 0.416)(0.675, 0)(0.925, 0)(0.925, 0.499364)(0.8, 0.499364)(0.8, 0)(1.05, 0)(1.05, 0.578773)(0.925, 0.578773)(0.925, 0)(1.175, 0)(1.175, 0.634452)(1.05, 0.634452)(1.05, 0)(1.3, 0)(1.3, 0.646625)(1.175, 0.646625)(1.175, 0)(1.425, 0)(1.425, 0.595517)(1.3, 0.595517)(1.3, 0)(1.55, 0)(1.55, 0.461352)(1.425, 0.461352)(1.425, 0)
\psline[linecolor=blue, linewidth=0.1pt](0.425, 0)(0.425, 0.339931)(0.3, 0.339931)(0.3, 0)(0.55, 0)(0.55, 0.316508)(0.425, 0.316508)(0.425, 0)(0.675, 0)(0.675, 0.348456)(0.55, 0.348456)(0.55, 0)(0.8, 0)(0.8, 0.416)(0.675, 0.416)(0.675, 0)(0.925, 0)(0.925, 0.499364)(0.8, 0.499364)(0.8, 0)(1.05, 0)(1.05, 0.578773)(0.925, 0.578773)(0.925, 0)(1.175, 0)(1.175, 0.634452)(1.05, 0.634452)(1.05, 0)(1.3, 0)(1.3, 0.646625)(1.175, 0.646625)(1.175, 0)(1.425, 0)(1.425, 0.595517)(1.3, 0.595517)(1.3, 0)(1.55, 0)(1.55, 0.461352)(1.425, 0.461352)(1.425, 0)
}
\uncover<2->{
\psline*[linecolor=cyan, linewidth=0.1pt](0.3, 0)(0.3, 0.4385)(0.425, 0.4385)(0.425, 0)(0.425, 0)(0.425, 0.339931)(0.55, 0.339931)(0.55, 0)(0.55, 0)(0.55, 0.316508)(0.675, 0.316508)(0.675, 0)(0.675, 0)(0.675, 0.348456)(0.8, 0.348456)(0.8, 0)(0.8, 0)(0.8, 0.416)(0.925, 0.416)(0.925, 0)(0.925, 0)(0.925, 0.499364)(1.05, 0.499364)(1.05, 0)(1.05, 0)(1.05, 0.578773)(1.175, 0.578773)(1.175, 0)(1.175, 0)(1.175, 0.634452)(1.3, 0.634452)(1.3, 0)(1.3, 0)(1.3, 0.646625)(1.425, 0.646625)(1.425, 0)(1.425, 0)(1.425, 0.595517)(1.55, 0.595517)(1.55, 0)
\psline[linecolor=blue, linewidth=0.1pt](0.3, 0)(0.3, 0.4385)(0.425, 0.4385)(0.425, 0)(0.425, 0)(0.425, 0.339931)(0.55, 0.339931)(0.55, 0)(0.55, 0)(0.55, 0.316508)(0.675, 0.316508)(0.675, 0)(0.675, 0)(0.675, 0.348456)(0.8, 0.348456)(0.8, 0)(0.8, 0)(0.8, 0.416)(0.925, 0.416)(0.925, 0)(0.925, 0)(0.925, 0.499364)(1.05, 0.499364)(1.05, 0)(1.05, 0)(1.05, 0.578773)(1.175, 0.578773)(1.175, 0)(1.175, 0)(1.175, 0.634452)(1.3, 0.634452)(1.3, 0)(1.3, 0)(1.3, 0.646625)(1.425, 0.646625)(1.425, 0)(1.425, 0)(1.425, 0.595517)(1.55, 0.595517)(1.55, 0)
}
\rput[t](0.3,-0.03){$a$}
\rput[t](0.425,-0.03){$x_1$}
\rput[t](0.55,-0.03){$x_2$}
%\rput[t](0.675,-0.03){}
%\rput[t](0.8,-0.03){$\frac{4}{5}$}
\rput[t](0.7375,-0.06){$\dots$}
\rput[t](0.925,-0.03){$x_{i-1}$}
\rput[t](1.05,-0.03){$x_{i}$}
%\rput[t](1.175,-0.03){$x_i$}
%\rput[t](1.3,-0.03){$\frac{13}{10}$}
\rput[t](1.3,-0.06){$\dots$}
%\rput[t](1.425,-0.03){$\frac{57}{40}$}
\rput[t](1.55,-0.03){$b$}
%Function formula: -171/50 (x)-27/16 ((x)^{3})+11/10+729/160 ((x)^{2}) 
\psplot[linecolor=red, plotpoints=1000]{0.3}{1.55}{x 2 exp 4.55625 mul 1.1 add x 3 exp -1.6875 mul add x -3.42 mul add }
\psline{<->}(1.6,0)(1.6, 0.578773)
\psline[linestyle=dotted](1.6, 0.578773)(1.05, 0.578773)
\rput[l](1.6, 0.3343865){$f(x_i)$}
\rput[b](0.9875,0.70773){$\Delta x$}
\psline(0.925,0.668773)(1.05, 0.668773)
\psline(0.925,0.648773)(0.925,0.688773) 
\psline(1.05, 0.648773)(1.05, 0.688773)

%\psbrace[linecolor=red,ref=lC](2)(3){Text I}
%\uput{:U}{$\overbrace{}^\text{\normalsize A line}$}
\end{pspicture} 


\begin{itemize}
\item<1->  The width of the interval is $b-a$.
\item<1->  The width of each of the $n$ strips is $\Delta x = \frac{b-a}{n}$.
\item<1->  The strips divide $[a,b]$ into $n$ subintervals:\\ $[x_0, x_1]$, $[x_1,x_2]$, $\ldots ,$ $[x_{n-1},x_n]$,\\ where $x_0 = a$ and $x_n = b$.
\end{itemize}
\column{.45\textwidth}
\begin{itemize}
\item<1->  The \only<handout:1| -1>{right}\only<handout:2| 2->{\alert<handout:2| 2>{left}} endpoints of the subintervals are
\end{itemize}
\abovedisplayskip=0pt
\belowdisplayskip=0pt
\begin{eqnarray*}
x_{\only<handout:1| -1>{1}\only<handout:2| 2->{\alert<handout:2| 2>{0}}} & = & a \only<handout:1| -1>{+ \Delta x}\\
x_{\only<handout:1| -1>{2}\only<handout:2| 2->{\alert<handout:2| 2>{1}}} & = & a + \only<handout:1| -1>{2}\Delta x\\
x_{\only<handout:1| -1>{3}\only<handout:2| 2->{\alert<handout:2| 2>{2}}} & = & a + \only<handout:1| -1>{3}\only<handout:2| 2->{2}\Delta x\\
& \vdots & 
\end{eqnarray*}
\begin{itemize}
\item<1->  The height of the $i$th rectangle is $f(x_{i\only<handout:2| 2->{\alert<handout:2| 2>{-1}}})$.
\item<1->  The area of the $i$th rectangle is $f(x_{i\only<handout:2| 2->{\alert<handout:2| 2>{-1}}})\Delta x$.
\end{itemize}
\end{columns}
\abovedisplayskip=0pt
\belowdisplayskip=0pt
\[
\only<handout:1| -1>{R_n}\only<handout:2| 2->{\alert<handout:2| 2>{L_n}} = f(x_{\only<handout:1| -1>{1}\only<handout:2| 2->{\alert<handout:2| 2>{0}}})\Delta x + f(x_{\only<handout:1| -1>{2}\only<handout:2| 2->{\alert<handout:2| 2>{1}}})\Delta x + f(x_{\only<handout:1| -1>{3}\only<handout:2| 2->{\alert<handout:2| 2>{2}}})\Delta x + \cdots + f(x_{n\only<handout:2| 2->{\alert<handout:2| 2>{-1}}})\Delta x
\]
\end{frame}

\begin{frame}
\begin{definition}[Area Under a Curve]
The area of the region $S$ that lies under the curve $y = f(x)$ is the limit of the sum of the areas of the approximating rectangles:
\abovedisplayskip=0pt
\belowdisplayskip=0pt
\[
A = \lim_{n\to\infty} R_n = \lim_{n\to\infty} [ f(x_1)\Delta x + f(x_2) \Delta x + \cdots + f(x_n) \Delta x]
\]
\end{definition}
\begin{itemize}
\item<2->  This limit always exists if $f$ is continuous.
\item<3->  We get the same answer if we use left endpoints:
\abovedisplayskip=0pt
\belowdisplayskip=0pt
\[
A = \lim_{n\to\infty} L_n = \lim_{n\to\infty} [ f(x_0)\Delta x + f(x_1) \Delta x + \cdots + f(x_{n-1}) \Delta x]
\]
\item<4->  We get the same answer if we use any number $x_i^*$ in the interval $[x_{i-1},x_i]$.  $x_i^*$ is called a sample point.
\abovedisplayskip=0pt
\belowdisplayskip=0pt
\[
A = \lim_{n\to\infty} [ f(x_1^*)\Delta x + f(x_2^*) \Delta x + \cdots + f(x_{n}^*) \Delta x]
\]
\end{itemize}
\uncover<5->{%
\begin{definition}[Riemann Sum]
A Riemann sum is any sum of the form
\abovedisplayskip=0pt
\belowdisplayskip=0pt
\[
f(x_1^*)\Delta x + f(x_2^*) \Delta x + \cdots + f(x_{n}^*) \Delta x.
\]
\end{definition}
}%
\end{frame}
% end module area-def

% end lecture

\end{document}
