\documentclass%
%[handout]
{beamer}
% % % % % % % %
% % % % % % % %
% % % % % % % %
%IMPORTANT
%compiles with 
%pdflatex -shell-escape 
%IMPORTANT
% % % % % % % %
% % % % % % % %
% % % % % % % %
\mode<presentation>
{
\useinnertheme{rounded}
\useoutertheme{infolines}
\usecolortheme{orchid}
\usecolortheme{whale}
}

\usepackage[english]{babel}
\usepackage[latin1]{inputenc}
\usepackage{times}
\usepackage[T1]{fontenc}
\usepackage{../example-templates}
\usepackage{auto-pst-pdf}
\usepackage{pst-plot}

% Or whatever. Note that the encoding and the font should match. If T1
% does not look nice, try deleting the line with the fontenc.

\graphicspath{{../../modules/}}

\newtheoremstyle{partialproof}{3pt}{3pt}{}{}{}{.}{.5em}{}
\theoremstyle{partialproof} \newtheorem{partialproof}[theorem]{Proof.}
%\DeclareMathOperator{\diff}{d}
\newcommand{\diff}{\text{d}}
\setbeamertemplate{navigation symbols}{}

\includeonlylecture{1}

\newcommand{\lect}[3]{
  \date{#1}
  \lecture[#1]{#2}{#3}
}

\setbeamertemplate{footline}
{
  \leavevmode%
  \hbox{%
  \begin{beamercolorbox}[wd=.333333\paperwidth,ht=2.25ex,dp=1ex,center]{author in head/foot}%
    \usebeamerfont{author in head/foot}\insertshortauthor
  \end{beamercolorbox}%
  \begin{beamercolorbox}[wd=.333333\paperwidth,ht=2.25ex,dp=1ex,center]{title in head/foot}%
    \usebeamerfont{title in head/foot}\insertshorttitle
  \end{beamercolorbox}%
  \begin{beamercolorbox}[wd=.333333\paperwidth,ht=2.25ex,dp=1ex,center]{date in head/foot}%
    \usebeamerfont{date in head/foot}\insertshortdate{}
  \end{beamercolorbox}}%
  \vskip0pt%
}

% If you have a file called "university-logo-filename.xxx", where xxx
% is a graphic format that can be processed by latex or pdflatex,
% resp., then you can add a logo as follows:

%\pgfdeclareimage[height=0.8cm]{logo}{bluelogo}
%\logo{\pgfuseimage{logo}}

\begin{document}


\AtBeginLecture{%

\title[\insertlecture]{FreeCalc}
\subtitle{\insertlecture}
\author[FreeCalc]{}
\institute[UMass Boston]{University of Massachusetts Boston}
\date{\insertshortlecture}
\begin{frame}
  \titlepage
\end{frame}
}%

% begin lecture
\lect{\today}{Sample}{1}
% begin module trig-functions-graphs
\begin{frame}
\frametitle{Graphs of the Trigonometric Functions}
\begin{tabular}{cc}
\begin{tabular}{c}
\psset{xunit=0.6cm,yunit=0.6cm}
\begin{pspicture}(-5,-1.4)(10,1.4)
\tiny
\psaxes[labels=none, Dx=1.570796327, Dy=1] {<->}(0,0)(-4,-1.4)(10,1.4)
\psplot[linecolor=red, plotpoints=1000]{-4}{10}{x 57.295779513 mul sin}

\rput[t](-3.14, -0.3){$-\pi$}
\rput[t](-1.57, -0.3){$-\frac{\pi}{2}$}
\rput[t](1.57, -0.3){$\frac{\pi}{2}$}
\rput[t](3.14, -0.3){$\pi$}
\rput[t](4.71, -0.3){$\frac{3\pi}{2}$}
\rput[t](6.28, -0.3){$2\pi$}
\rput[t](7.85, -0.3){$\frac{5\pi}{2}$}
\rput[t](9.42, -0.3){$3\pi$}
\rput[bl](0.2,1){\tiny $1$}
\end{pspicture}

\end{tabular}
& $y = \sin x$\\
\begin{tabular}{c}
\psset{xunit=0.6cm,yunit=0.6cm}
\begin{pspicture}(-5,-1.4)(10,1.4)
\tiny
\psaxes[labels=none, Dx=1.570796327, Dy=1] {<->}(0,0)(-4,-1.4)(10,1.4)
\psplot[linecolor=red, plotpoints=1000]{-4}{10}{x 57.295779513 mul cos}

\rput[t](-3.14, -0.3){$-\pi$}
\rput[t](-1.57, -0.3){$-\frac{\pi}{2}$}
\rput[t](1.57, -0.3){$\frac{\pi}{2}$}
\rput[t](3.14, -0.3){$\pi$}
\rput[t](4.71, -0.3){$\frac{3\pi}{2}$}
\rput[t](6.28, -0.3){$2\pi$}
\rput[t](7.85, -0.3){$\frac{5\pi}{2}$}
\rput[t](9.42, -0.3){$3\pi$}
\rput[bl](0.2,1){\tiny $1$}
\end{pspicture}
\end{tabular}
& $y = \cos x$
\end{tabular}
\begin{itemize}
\item<2->  $\sin x$ has zeroes at $n\pi$ for all integers $n$.
\item<3->  $\cos x$ has zeroes at $\pi /2 + n\pi$ for all integers $n$.
\item<4->  $-1 \leq \sin x \leq 1$. 
\item<5->  $-1 \leq \cos x \leq 1$. 
\end{itemize}
\end{frame}


\begin{frame}
\begin{tabular}{cc}
\psset{xunit=0.35cm,yunit=0.35cm}
\begin{pspicture*}(-7,-10)(7,10)
\psaxes[labels=none, ticks=x, Dx=1.570796327] {<->}(0,0)(-5.5,-10)(5.5,10)
\rput[lt](5.5,0){$x$}
\rput[lb](0.2,9){$y$}
%\rput[t](1,-0.1){1}
\psline[linecolor=gray](1,-0.1)(1,0.1) % x unit mark
\rput[lb](1.570796327,0.1){$\frac{\pi}2$}
\psline[linecolor=gray](1.570796327,-0.1)(1.570796327,0.1) % pi/2 unit mark
%\rput[br](0,1){1}
\psline[linecolor=gray](-0.1,1)(0.1,1) % y unit mark

\psplot[linecolor=red]{-1.57}{1.57}{ 180 x mul  3.1415 div tan} 
\psplot[linecolor=red]{-4.71}{-1.58}{ 180 x mul  3.1415 div tan} 
\psplot[linecolor=red]{1.58}{4.71}{ 180 x mul  3.1415 div tan} 

\psline[linestyle=dotted](-4.71238898,-10)(-4.71238898,10)
\psline[linestyle=dotted](-1.570796327,-10)(-1.570796327,10)
\psline[linestyle=dotted](1.570796327,-10)(1.570796327,10)
\psline[linestyle=dotted](4.71238898,-10)(4.71238898,10)
\end{pspicture*}

&%
\psset{xunit=0.35cm,yunit=0.35cm}
\begin{pspicture*}(-7,-10)(8,10)
\psaxes[labels=none, ticks=x, Dx=1.570796327] {<->}(0,0)(-7,-10)(7,10)
\rput[lt](7,0){$x$}
\rput[lb](0.2,9){$y$}
%\rput[t](1,-0.1){1}
\psline[linecolor=gray](1,-0.1)(1,0.1) % x unit mark
\rput[rb](3.13,0.1){$\pi$}
\psline[linecolor=gray](1.570796327,-0.1)(1.570796327,0.1) % pi/2 unit mark
%\rput[br](0,1){1}
\psline[linecolor=gray](-0.1,1)(0.1,1) % y unit mark

\psplot[linecolor=red]{0.01}{3.14}{1 180 x mul  3.1415 div tan div} 
\psplot[linecolor=red]{3.15}{6.28}{1 180 x mul  3.1415 div tan div} 
\psplot[linecolor=red]{-3.14}{-0.01}{1 180 x mul  3.1415 div tan div} 
\psplot[linecolor=red]{-6.28}{-3.15}{1 180 x mul  3.1415 div tan div} 
%\psplot[linecolor=red]{-4.71}{-1.58}{ 180 x mul  3.1415 div cot} 
%\psplot[linecolor=red]{1.58}{4.71}{ 180 x mul  3.1415 div cot} 

\psline[linestyle=dotted](-6.283185307,-10)(-6.283185307,10)
\psline[linestyle=dotted](-3.141592654,-10)(-3.141592654,10)
\psline[linestyle=dotted](3.141592654,-10)(3.141592654,10)
\psline[linestyle=dotted](6.283185307,-10)(6.283185307,10)
\end{pspicture*}

\\%
$y = \tan x$ & $y = \cot x$\\
\end{tabular}
\end{frame}


\begin{frame}
\begin{tabular}{cc}
\psset{xunit=0.5cm,yunit=0.5cm}
\begin{pspicture}(-4.8,-7.1)(6.2,7.1)
\psaxes[labels=none, ticks=x, Dx=1.570796327] {<->}(0,0)(-3.2,-7)(6.2,7)
\psline(-0.15, 1)(0.15,1)
\psplot[linecolor=blue, linestyle=dashed, plotpoints=1000]{-3.2}{6}{x 57.295779513 mul sin}
\uncover<2->{
\psplot[linecolor=red, plotpoints=1000]{0.15}{2.991592654}{1 x 57.295779513 mul sin div}
\psplot[linecolor=red, plotpoints=1000]{-2.991592654}{-0.15}{1 x 57.295779513 mul sin div}
\psplot[linecolor=red, plotpoints=1000]{3.291592654}{6.133185307}{1 x 57.295779513 mul sin div}
}

\psline[linestyle=dotted](3.14159,-7)(3.14159,7)
\rput[t](-3.14, -0.3){\tiny$-\pi$}
\rput[t](-1.57, -0.3){\tiny$-\frac{\pi}{2}$}
\rput[t](1.57, -0.3){\tiny$\frac{\pi}{2}$}
\rput[t](3, -0.3){\tiny$\pi$}
\rput[t](4.71238898, -0.3){\tiny$\frac{3\pi}{2}$}

\rput[bl](0.2,1){$1$}
\end{pspicture}

&%
\psset{xunit=0.5cm,yunit=0.5cm}
\begin{pspicture}(-4.7,-7.1)(6.1,4.8)
\psaxes[labels=none, ticks=x, Dx=1.570796327] {<->}(0,0)(-4.7,-7.1)(4.8,7.1)
\psline(-0.15, 1)(0.15,1)
\psplot[linecolor=blue, linestyle=dashed, plotpoints=1000]{-4.7}{4.7}{x 57.295779513 mul cos}
\uncover<3->{
\psplot[linecolor=red, plotpoints=1000]{-1.420796327}{1.420796327}{1 x 57.295779513 mul cos div}
\psplot[linecolor=red, plotpoints=1000]{1.720796327}{4.56238898}{1 x 57.295779513 mul cos div}
\psplot[linecolor=red, plotpoints=1000]{-4.56238898}{-1.720796327}{1 x 57.295779513 mul cos div}
}

\psline[linestyle=dotted](1.570796327,-7.1)(1.570796327,7.1)
\psline[linestyle=dotted](-1.570796327,-7.1)(-1.570796327,7.1)
\rput[t](-3.14, -0.3){\tiny$-\pi$}
\rput[t](-1.57, -0.3){\tiny$-\frac{\pi}{2}$}
\rput[t](1.57, -0.3){\tiny$\frac{\pi}{2}$}
\rput[t](3, -0.3){\tiny$\pi$}
\rput[t](4.71238898, -0.3){\tiny$\frac{3\pi}{2}$}

\rput[bl](0.2,1){$1$}
\end{pspicture}
\\%
$y = \csc x$  & $y = \sec x$\pause\pause\\
\end{tabular}
\end{frame}
% end module trig-functions-graphs

% end lecture

\end{document}
