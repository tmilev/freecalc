\documentclass%
%[handout]
{beamer}
% % % % % % % %
% % % % % % % %
% % % % % % % %
%IMPORTANT
%compiles with 
%pdflatex -shell-escape 
%IMPORTANT
% % % % % % % %
% % % % % % % %
% % % % % % % %
\mode<presentation>
{
\useinnertheme{rounded}
\useoutertheme{infolines}
\usecolortheme{orchid}
\usecolortheme{whale}
}

\usepackage[english]{babel}
\usepackage[latin1]{inputenc}
\usepackage{times}
\usepackage[T1]{fontenc}
\usepackage{../example-templates}
\usepackage{auto-pst-pdf}
\usepackage{pst-plot}

% Or whatever. Note that the encoding and the font should match. If T1
% does not look nice, try deleting the line with the fontenc.

\graphicspath{{../../modules/}}

\newtheoremstyle{partialproof}{3pt}{3pt}{}{}{}{.}{.5em}{}
\theoremstyle{partialproof} \newtheorem{partialproof}[theorem]{Proof.}
%\DeclareMathOperator{\diff}{d}
\newcommand{\diff}{\text{d}}
\setbeamertemplate{navigation symbols}{}

\includeonlylecture{1}

\newcommand{\lect}[3]{
  \date{#1}
  \lecture[#1]{#2}{#3}
}

\setbeamertemplate{footline}
{
  \leavevmode%
  \hbox{%
  \begin{beamercolorbox}[wd=.333333\paperwidth,ht=2.25ex,dp=1ex,center]{author in head/foot}%
    \usebeamerfont{author in head/foot}\insertshortauthor
  \end{beamercolorbox}%
  \begin{beamercolorbox}[wd=.333333\paperwidth,ht=2.25ex,dp=1ex,center]{title in head/foot}%
    \usebeamerfont{title in head/foot}\insertshorttitle
  \end{beamercolorbox}%
  \begin{beamercolorbox}[wd=.333333\paperwidth,ht=2.25ex,dp=1ex,center]{date in head/foot}%
    \usebeamerfont{date in head/foot}\insertshortdate{}
  \end{beamercolorbox}}%
  \vskip0pt%
}

% If you have a file called "university-logo-filename.xxx", where xxx
% is a graphic format that can be processed by latex or pdflatex,
% resp., then you can add a logo as follows:

%\pgfdeclareimage[height=0.8cm]{logo}{bluelogo}
%\logo{\pgfuseimage{logo}}

\begin{document}


\AtBeginLecture{%

\title[\insertlecture]{FreeCalc}
\subtitle{\insertlecture}
\author[FreeCalc]{}
\institute[UMass Boston]{University of Massachusetts Boston}
\date{\insertshortlecture}
\begin{frame}
  \titlepage
\end{frame}
}%

% begin lecture
\lect{\today}{Sample}{1}
% begin module trig-example
\begin{frame}
\begin{example}
\begin{columns}[c]
\column{.5\textwidth}

\psset{xunit=1.8cm,yunit=1.8cm}
\begin{pspicture}(-2.3,-0.5)(0.5,2.2)
\psframe*[linecolor=white, fillcolor=white](-2.3,-0.5)(0.8,2.5)
\psaxes[fillcolor=white, fillstyle=solid, labels=none, ticks=none]{<->}(0,0)(-2.3,-0.5)(0.5,2.2)
\rput[l](0.55, 0){$x$}
\rput[b](0, 2.25){$y$}

\psline[linecolor=blue](0,0)(-1,1.732)
\psline[linecolor=blue](0,0)(0.5,0)
\uncover<3->{
\psline[linestyle=dotted](-1,1.732)(-1, 0)
\psline[linestyle=dotted](-1,1.732)(0, 1.732)
}
\pscircle*(-1,1.732){0.07}

\rput[l](0.15, 0.35){$\frac{2\pi}{3}$}
\psarc[linecolor=red](0,0){0.5}{0}{120}
\uncover<2->{
\rput(-0.25, 0.15){$\frac{\pi}{3}$}
\psarc[linecolor=red](0,0){0.3}{120}{180}
}
\uncover<3->{
\rput[br](-1,1.732){$(1,\sqrt{3})$}
\alert<5,7,11,13>{\rput[lb](-0.45, 0.85){$2$}}
\alert<5,9,11,15>{\rput[r](-1.1, 0.85){$\sqrt{3}$}}
\alert<7,9,13,15>{\rput[t](-0.6, -0.05){$1$}}
}
\end{pspicture}
\column{.5\textwidth}
Find the exact trigonometric ratios for $\theta = 2\pi /3=120^\circ$.
\end{columns}
\begin{align*}
\alert<handout:0| 4-5>{\sin \frac{2\pi}{3}} & \alert<handout:0| 4-5>{= \uncover<5->{\frac{\sqrt{3}}{2}}} &
\alert<handout:0| 6-7>{\cos \frac{2\pi}{3}} & \alert<handout:0| 6-7>{= \uncover<7->{-\frac{1}{2}}} &
\alert<handout:0| 8-9>{\tan \frac{2\pi}{3}} & \alert<handout:0| 8-9>{= \uncover<9->{\frac{\sqrt{3}}{-1}= -\sqrt{3}}} \\
\alert<handout:0| 10-11>{\csc \frac{2\pi}{3}} & \alert<handout:0| 10-11>{= \uncover<11->{\frac{2}{\sqrt{3}}}} &
\alert<handout:0| 12-13>{\sec \frac{2\pi}{3}} & \alert<handout:0| 12-13>{= \uncover<13->{-\frac{2}{1}=-2}} &
\alert<handout:0| 14-15>{\cot \frac{2\pi}{3}} & \alert<handout:0| 14-15>{= \uncover<15->{-\frac{1}{\sqrt{3}}}}
\end{align*}
\end{example}
\end{frame}
% end module trig-example

% end lecture

\end{document}
