\documentclass%
%[handout]
{beamer}

% % % % % % % %
% % % % % % % %
% % % % % % % %
%IMPORTANT
%compiles with 
%pdflatex -shell-escape 
%IMPORTANT
% % % % % % % %
% % % % % % % %
% % % % % % % %


\usepackage{etex} %avoiding error: too many packages. This is a LaTeX bug (``feature'')


\mode<presentation>
{
\useinnertheme{rounded}
\useoutertheme{infolines}
\usecolortheme{orchid}
\usecolortheme{whale}
}

\usepackage[english]{babel}
\usepackage[latin1]{inputenc}
\usepackage[all,cmtip]{xy}
\usepackage{times}
\usepackage{auto-pst-pdf}
\usepackage{pstricks-add}
\usepackage{pst-plot}
\usepackage{pst-math}
\usepackage{pst-node}
\usepackage{cancel}

\usepackage[T1]{fontenc}
% Or whatever. Note that the encoding and the font should match. If T1
% does not look nice, try deleting the line with the fontenc.

\usepackage{../pstricks-commands}
\usepackage{../example-templates}
\graphicspath{{../../modules/}}

\setbeamertemplate{navigation symbols}{}

\includeonlylecture{4}

\newcommand{\diff}{\text{d}}
\newcommand{\lect}[3]{
  \date{#1}
  \lecture[#1]{#2}{#3}
}

\setbeamertemplate{footline}
{
  \leavevmode%
  \hbox{%
  \begin{beamercolorbox}[wd=.333333\paperwidth,ht=2.25ex,dp=1ex,center]{author in head/foot}%
    \usebeamerfont{author in head/foot}\insertshortauthor
  \end{beamercolorbox}%
  \begin{beamercolorbox}[wd=.333333\paperwidth,ht=2.25ex,dp=1ex,center]{title in head/foot}%
    \usebeamerfont{title in head/foot}\insertshorttitle
  \end{beamercolorbox}%
  \begin{beamercolorbox}[wd=.333333\paperwidth,ht=2.25ex,dp=1ex,center]{date in head/foot}%
    \usebeamerfont{date in head/foot}\insertshortdate{}
  \end{beamercolorbox}}%
  \vskip0pt%
}

% If you have a file called "university-logo-filename.xxx", where xxx
% is a graphic format that can be processed by latex or pdflatex,
% resp., then you can add a logo as follows:

%\pgfdeclareimage[height=0.8cm]{logo}{bluelogo}
%\logo{\pgfuseimage{logo}}

\begin{document}

\AtBeginLecture{%

\title[\insertlecture]{Algebraic Topology}
\subtitle{\insertlecture}
\author[Algebraic Topology]{Greg Maloney}
\institute{Newcastle University}
\date{\insertshortlecture}
\begin{frame}
  \titlepage
\end{frame}

\begin{frame}{Outline}
  \tableofcontents[pausesections]
\end{frame}
}%

% begin lecture
\lect{January 31, 2014}{Lecture  2}{2}
\section{The Chain Rule}
% begin module chain-rule-intro-alternative
\begin{frame}
\frametitle{The Chain Rule}
\begin{itemize}
\item  What is the derivative of $F(x) = \cos x^3$?
\item<2->  The Power Rule doesn't tell us how to solve this.
\item<3->  $F$ is a composite function $f\circ g$:
\item<3-| alert@4-5,9,11-12>  $y = f(u) = \uncover<5->{\cos u}$.
\item<3-| alert@6-8,13-14>  $u = g(x) = \uncover<7->{x^3}$.
\item<3->  Then $y = F(x) = f(\alertNoH{ 8}{g(x)}) = \uncover<8->{\alertNoH{ 9}{f(\alertNoH{ 8}{x^3}) =}}  \uncover<9->{\alertNoH{ 9}{\cos x^3}.}$
\item<10->  We know the derivatives of $f$ and $g$:
\item<10-| alert@11-12>  $f'(u) = \uncover<12->{-\sin u}$.
\item<10-| alert@13-14>  $g'(x) = \uncover<14->{3x^2}$.
\item<15->  It would be nice if we could find the derivative of $F$ in terms of the derivatives of $y$ and $u$.
\item<16->  It turns out that the derivative of the composition $f\circ g$ is the product of the derivative of $f$ and the derivative of $g$.
\item<17->  This important fact is called the Chain Rule.
\end{itemize}
\end{frame}
% end module chain-rule-intro-alternative

% begin module chain-rule-statement
\begin{frame}
The Chain Rule

If $g$ is differentiable at $x$ and $f$ is differentiable at $g(x)$, then the composite function $F = f\circ g$ defined by $F(x) = f(g(x))$ is differentiable at $x$ and $F'$ is given by the product
\[
F'(x) = f'(g(x))\cdot g'(x)
\]
In Leibniz notation, if $y = f(u)$ and $u = g(x)$ are both differentiable functions, then
\[
\frac{\diff y}{\diff x} = \frac{\diff y}{\diff u} \frac{\diff u}{\diff x}
\]

%I want to cover the proof of the chain rule in my classes. We need to come up with a scheme for including optional material, so we don't need to modify
%the modules themselves (maybe \newcommand{\includeOptionalMaterial}[1]{#1} or something of the sort?).
%\uncover<2->{%
%We will not prove this in class, but a proof can be found in the textbook.
%}%
\end{frame}
% end module chain-rule-statement

% begin module chain-rule-cos-poly.tex
\begin{frame}
\chainruley{\cos x^3}{x^3}{\cos u}{-\sin UU}{3x^2}{-3x^2 \sin UU}{0}
\end{frame}
% end module chain-rule-cos-poly.tex
% begin module chain-rule-poly-cos.tex
\begin{frame}
\chainruley{\cos^3 x}{\cos x}{u^3}{3 UU^2}{-\sin x}{-3\sin x (UU)^2}{0}
\end{frame}
% end module chain-rule-poly-cos.tex
% begin module chain-rule-power-rule
\begin{frame}
\begin{itemize}
\item  In the example $y = \cos^3 x$, the outer function was a power function: $y = u^3$.
\item<2->  The derivative was $\frac{\diff y}{\diff x} = 3u^2 \frac{\diff u}{\diff x} = (3\cos^2 x) (-\sin x)$.  
\item<3->  We can generalize this:
\end{itemize}

\uncover<3->{%
The Power Rule Combined with the Chain Rule

If $n$ is any real number and $u = g(x)$ is differentiable, then
\[
\frac{\diff}{\diff x} (u^n) = nu^{n-1} \frac{\diff u}{\diff x}
\]
Alternatively,
\[
\frac{\diff}{\diff x}[g(x)]^n = n [g(x)]^{n-1} \cdot g'(x)
\]
}%
\end{frame}
% end module chain-rule-power-rule

% begin module chain-rule-ex3
\begin{frame}
\[
\frac{\diff}{\diff x}(u^n) = nu^{n-1}\frac{\diff u}{\diff x}
\]
\chainruley{(x^3-1)^{100}}{x^3-1}{u^{100}}{100UU^{99}}{3x^2}{300x^2(UU)^{99}}{Power Rule}
\end{frame}
% end module chain-rule-ex3

% begin module chain-rule-ex4
\begin{frame}
\[
\frac{\diff}{\diff x}[g(x)]^n = n[g(x)]^{n-1}\cdot g'(x)
\]
\chainrulefofx{\frac{1}{\sqrt[3]{x^2+x+1}}}{x^2+x+1}{x^{-1/3}}{-\frac{1}{3}(UU)^{-4/3}}{2x+1}{-\frac{2x+1}{3}(UU)^{-4/3}}{Power Rule}
\end{frame}
% end module chain-rule-ex4

% begin module chain-rule-ex5
\begin{frame}
\begin{example}[Chain Rule and Quotient Rule]
Find the derivative of
\abovedisplayskip=0pt
\belowdisplayskip=0pt
\abovedisplayshortskip=0pt
\belowdisplayshortskip=0pt
\begin{align*}
g(t) & = \left( \frac{t-2}{2t+1}\right)^9. \\%
&  \uncover<2->{\text{Power Rule and Chain Rule:}}\\%
\uncover<2->{%
g'(t)%
}%
& \uncover<2->{ = } %
\uncover<2->{%
9\left( \frac{t-2}{2t+1}\right)^8 \alertNoH{ 3}{\frac{\diff}{\diff t}\left( \frac{t-2}{2t+1}\right)}%
}\\%
&  \uncover<3->{\text{Quotient Rule:}}\\%
& \uncover<3->{ = } %
\uncover<3->{%
9\left( \frac{t-2}{2t+1}\right)^8 \alertNoH{ 3}{\frac{(2t+1)\alertNoH{ 4-5}{\frac{\diff}{\diff t}(t-2)}-(t-2)\alertNoH{ 6-7}{\frac{\diff}{\diff t}(2t+1)}}{(2t+1)^2}}%
}\\%
& \uncover<4->{ = } %
\uncover<4->{%
9\left( \frac{t-2}{2t+1}\right)^8 \frac{(2t+1)\cdot \alertNoH{ 4-5}{\fcAnswerUncover{4}{5}{1}}-(t-2)\cdot \alertNoH{ 6-7}{\fcAnswerUncover{4}{7}{2}}}{(2t+1)^2}%
}\\%
& \uncover<8->{ = } %
\uncover<8->{%
9\left( \frac{t-2}{2t+1}\right)^8 \frac{\fcCancel{9}{2t}+1-\fcCancel{9}{2t}+4}{(2t+1)^2}%
}  \uncover<9->{ = } \uncover<9->{%
\frac{45(t-2)^8}{(2t+1)^{10}}.%
}%
\end{align*}
\end{example}
\end{frame}
% end module chain-rule-ex5

% begin module chain-rule-extra-links
\begin{frame}
\begin{itemize}
\item  We can add more ``links'' when we use the Chain Rule.
\item<2-| alert@3>  $y = f(u)$
\item<2-| alert@4>  $u = g(x)$
\item<2-| alert@4>  $x = h(t)$
\item<3->  Use the Chain Rule twice:
\end{itemize}
\[
\uncover<3->{%
\alertNoH{ 3}{\frac{\diff y}{\diff t} = \frac{\diff y}{\diff u}\alertNoH{ 4}{\frac{\diff u}{\diff t}}}%
}%
\uncover<4->{%
 = \frac{\diff y}{\diff u}\alertNoH{ 4}{\frac{\diff u}{\diff x}\frac{\diff x}{\diff t}}%
}%
\]
\end{frame}
% end module chain-rule-extra-links

% begin module chain-rule-twice-ex3
\begin{frame}
\chainruletwice%
{\tan \Big( \frac{1}{x^2+1}\Big)}%
{\sec^2 \Big( \frac{1}{x^2+1}\Big)}%
{\frac{1}{x^2+1}}%
{\frac{-1}{(x^2+1)^2}}%
{x^2+1}%
{2x}%
{}%
{-\frac{2x\sec^2 \Big( \frac{1}{x^2+1}\Big)}{(x^2+1)^2}}%
{}%
\end{frame}
% end module chain-rule-twice-ex3

% end lecture

% begin lecture
\lect{February 7, 2014}{Lecture  4}{4}
\section{Exponential Functions}
\subsection{Definition and Properties}
% begin module exponential-function-def
\begin{frame}
\frametitle{Exponential Functions}
The function $f(x) = 2^x$ is called an exponential function because the variable $x$ is the exponent.
\begin{columns}[c]
\column{.5\textwidth}
\psset{xunit=0.8cm, yunit=0.8cm}
\begin{pspicture}(-2.5,-0.5)(2.5,6.25)
\psframe*[linecolor=white](-2.5,-0.5)(2.5,6.25)
\psaxes[labels=none]{<->}(0,0)(-2.5,-0.5)(2.5,6.25)
\rput[t](1, -0.2){$1$}
\uncover<3->{
\fcFullDot{2}{4}
}
\uncover<5->{
\fcFullDot{1}{2}
}
\uncover<7->{
\fcFullDot{0}{1}
}
\uncover<9->{
\fcFullDot{-1}{0.5}
}
\uncover<11->{
\fcFullDot{-2}{0.25}
}
\uncover<12->{
\rput[r](-0.5, 1.5){$y=2^x$}
%Function formula: 2^{x}
\psplot[linecolor=red, plotpoints=1000]{-2.5}{2.5}{2 x exp }
}
\end{pspicture}
\column{.5\textwidth}
\[
\begin{array}{r|l}
x & y\\
\hline
\alertNoH{ 2-3}{2} & \fcAnswer{ 3}{{4}} \\
\alertNoH{ 4-5}{1} & \fcAnswer{ 5}{{2}} \\
\alertNoH{ 6-7}{0} & \fcAnswer{ 7}{{1}} \\
\alertNoH{ 8-9}{-1} & \fcAnswer{ 9}{{\frac{1}{2}}} \\
\alertNoH{ 10-11}{-2} & \fcAnswer{ 11}{{ \frac{1}{4}}}
\end{array}
\]
\uncover<13->{
\begin{emptyTheorem}[Exponential Function Terminology]
An exponential function is a function of the form $f(x) = a^x$, where $a$ is a positive constant.
\end{emptyTheorem}
}
\end{columns}
\end{frame}
% end module exponential-function-def

% begin module exponential-function-graphs-plus
\begin{frame}
\begin{columns}
\column{0.7\textwidth}
Graphs of various exponential functions.

\psset{xunit=2cm, yunit=2cm}
\begin{pspicture}(-2.1,-0.2)(2.2,3.6) 
\psframe*[linecolor=white](-2.1,-0.2)(2.1,3.5)
\psaxes[labels=none]{<->}(0,0)(-2.1,-0.2)(2.1,3.5)
\uncover<1->{
\rput[r](1.8, 2.3){$y=2^x$}
%Function formula: 2^{x} 
\psplot[linecolor=red, plotpoints=1000]{-2}{1.584962501}{2 x exp }
}
\uncover<2->{
\rput[l](0.7, 3.1){$y=4^x$}
%Function formula: 4^{x} 
\psplot[linecolor=black, plotpoints=1000]{-2}{0.79248125}{4 x exp }
}
\uncover<3->{
\rput[b](0.4, 3.05){$y=10^x$}
%Function formula: 4^{x} 
\psplot[linecolor=blue, plotpoints=1000]{-2}{0.477121255}{10 x exp }
}
\uncover<4->{
\rput[l](1.15, 1.5){$y=1.5^x$}
%Function formula: 4^{x} 
\psplot[linecolor=green, plotpoints=1000]{-2}{2}{1.5 x exp }
}
\uncover<5->{
\rput[l](-1.9, 2){$y=0.5^x$}
%Function formula: 4^{x} 
\psplot[linecolor=purple, plotpoints=1000]{-1.584962501}{2}{0.5 x exp }
}
\uncover<6->{
\rput[l](-1.2, 3.1){$y=0.25^x$}
%Function formula: 4^{x} 
\psplot[linecolor=brown, plotpoints=1000]{-0.79248125}{2}{0.25 x exp }
}
\end{pspicture}

\column{0.3\textwidth}
\uncover<7->{Observations}
 
\begin{itemize}
\item<8-| alert@8-9> $a^x$ is always \uncover<9->{positive.}  
\item<10-| alert@10-11> $a^0 = \uncover<11->{1}$ for all $a$.  
\end{itemize}
\uncover<12->{$a > 1$:}
\begin{itemize}
\item<12-| alert@12-13> $\displaystyle \lim_{x\to\infty}a^x = \uncover<13->{\infty.}$
\item<12-| alert@14-15> $\displaystyle \lim_{x\to-\infty}a^x = \uncover<15->{0.}$
\end{itemize}
\uncover<12->{$a < 1$:}
\begin{itemize}
\item<12-| alert@16-17> $\displaystyle \lim_{x\to\infty}a^x = \uncover<17->{0.}$
\item<12-| alert@18-19> $\displaystyle \lim_{x\to-\infty}a^x = \uncover<19->{\infty.}$
\end{itemize}
\end{columns}

\end{frame}
% end module exponential-function-graphs-plus

% begin module exponential-function-properties
\begin{frame}
\begin{columns}[t]
\column{.3\textwidth}
\includegraphics[height=3cm]{exponential-functions/pictures/07-02-exptypesa.pdf}%

$y = a^x, a > 1$
\begin{itemize}
\item<2->  Increasing.
\item<3->  Passes through $(0,1)$.
\item<4->  $\lim_{x \rightarrow \infty} a^x = \infty$.
\item<5->  $\lim_{x \rightarrow -\infty} a^x = 0$.
\end{itemize}
\column{.35\textwidth}
\includegraphics[height=3cm]{exponential-functions/pictures/07-02-exptypesb.pdf}%

$y = a^x, 0 < a < 1$
\begin{itemize}
\item<6->  Decreasing.
\item<7->  Passes through $(0,1)$.
\item<8->  $\left(\frac{1}{a}\right)^x = a^{-x}$.
\item<9->  Reflection of $y = \left(\frac{1}{a}\right)^x$ in the $y$-axis.
\item<10->  $\lim_{x \rightarrow \infty} a^x = 0$.
\item<11->  $\lim_{x \rightarrow -\infty} a^x = \infty$.
\end{itemize}
\column{.3\textwidth}
\includegraphics[height=3cm]{exponential-functions/pictures/07-02-exptypesc.pdf}%

$y = a^x, a = 1$
\begin{itemize}
\item<12->  Horizontal.
\end{itemize}
\end{columns}
\end{frame}


\begin{frame}
\begin{theorem}[Properties of Exponential Functions]
If $a > 0$ and $a \neq 1$, then $f(x) = a^x$ is a continuous function with domain $\mathbb{R}$ and range $(0,\infty )$.  In particular, $a^x > 0$ for all $x$.  If $0 < a < 1$, then $f(x) = a^x$ is a decreasing function; if $a > 1$, then $f(x) = a^x$ is an increasing function.  If $a, b > 0$ and $x, y \in\mathbb{R}$, then
\begin{enumerate}
\item  $a^{x+y} = a^xa^y$
\item  $a^{x-y} = \frac{a^x}{a^y}$
\item  $(a^x)^y = a^{xy}$
\item  $(ab)^x = a^xb^x$
\item  If $a > 1$, then $\lim_{x\rightarrow \infty} a^x = \infty$ and $\lim_{x\rightarrow -\infty} a^x = 0$.
\item  If $0 < a < 1$, then $\lim_{x\rightarrow \infty} a^x = 0$ and $\lim_{x\rightarrow -\infty} a^x = \infty$.
\end{enumerate}
In particular, if $a \neq 1$, then the $x$-axis is a horizontal asymptote for $y = a^x$. 
\end{theorem}
\end{frame}
% end module exponential-function-properties

% begin module exponential-equation1
\begin{frame}
\begin{example}[Solving an exponential equation]
Solve for $t$.  
\begin{align*}
16^{4t} & = 8^{t-2} \\
\uncover<2->{\text{Find a common base:}\quad \alert<2-3>{\big( \uncover<3->{2^4}\big)}^{4t}} & \uncover<2->{ = \alert<2-3>{\big( \uncover<3->{2^3}\big)}^{t-2} } \\
\uncover<4->{2^{16t}} & \uncover<4->{ = 2^{3t-6}} \\
\uncover<5->{16t} & \uncover<5->{ = 3t - 6} \\
\uncover<6->{13t} & \uncover<6->{ =  -6} \\
\uncover<7->{t} & \uncover<7->{ =  -6/13.} 
\end{align*}
\end{example}
\end{frame}
% end module exponential-equation1

% begin module exponential-equation2
\begin{frame}
\begin{example}[Solving a quadratic exponential equation]
Solve for $x$.  
\abovedisplayskip=0pt
\belowdisplayskip=0pt
\begin{align*}
9^x & = 2\cdot 3^x + 63 \\
\uncover<2-| handout:0>{\alert<handout:0| 4-5>{9^x} -2\cdot \alert<handout:0| 3>{3^x} - 63} & \uncover<2->{ = 0} \\
\intertext{\uncover<3->{Substitute $\alert<handout:0| 9>{u = \uncover<3-| handout:0>{3^x}}$:}}
\uncover<3-| handout:0>{\alert<handout:0| 4-5>{\uncover<5->{u^2}} - 2\alert<handout:0| 3>{u} - 63} & \uncover<3->{ = 0} \\
\uncover<6->{\alert<handout:0| 6-7>{(\uncover<7-| handout:0>{u-9})(\uncover<7-| handout:0>{u+7})}} & \uncover<6->{ = 0 } 
\end{align*}
\begin{align*}
\uncover<8->{ \alert<handout:0| 9>{u} } & \uncover<8->{ = \uncover<8-| handout:0>{9} } & \uncover<8->{\text{or}} & & \uncover<8->{\alert<handout:0| 9>{u}} & \uncover<8->{ = \uncover<8-| handout:0>{-7}} \\
\uncover<9-| handout:0>{ \alert<handout:0| 9>{3^x} } & \uncover<9-| handout:0>{ = 9 } & \uncover<9-| handout:0>{\text{or}} & & \uncover<9-| handout:0>{\alert<handout:0| 9>{3^x}} & \uncover<9-| handout:0>{ = -7} \invisible{99999999} \\
\uncover<10-| handout:0>{ \alert<handout:0| 10-11>{x} } & \uncover<10-| handout:0>{ \alert<handout:0| 10-11>{ = \uncover<11->{2} }} & & & \uncover<10-| handout:0>{\alert<handout:0| 12-13>{\uncover<-12>{x}}} & \uncover<10-| handout:0>{ \alert<handout:0| 12-13>{ \only<-12>{ =} \only<13->{\text{no solution}} }} 
\end{align*}
\uncover<14-| handout:0>{Therefore $x = 2$ is the solution.}
\end{example}
\end{frame}
% end module exponential-equation2

%\input{../../modules/exponential-functions/exponential-versus-polynomial}
%% begin module exponential-function-ex-sketch
\begin{frame}
\begin{example}
Draw the graph of the function $y = 2^{-x}-1= 0.5^x-1= \left(\frac{1}{2}\right)^x-1 $.
\begin{columns}[c]
\column{.7\textwidth}
\psset{xunit=1cm, yunit=1cm}
\begin{pspicture}(-5, -5)(5,5) 
\psframe*[linecolor=white](-5,-5)(5,5) 
\psaxes[ticks=none, labels=none]{<->}(0,0)(-4, -2.5)(4.1,4)
\psline(-0.1,1)(0.1, 1)
\rput[l](0.2, 1){$1$} 
\psline(1,0.1)(1, -0.1)
\rput[t](1, -0.2){$1$} 
\psline(-0.1,-1)(0.1, -1)
\rput[tl](0.2, -1){$-1$} 
%Function formula: 2^{x} 
\uncover<2>{
\psplot[linecolor=red, plotpoints=1000]{-4}{2}{2 x exp}
\rput[l](1.7, 3){$y=2^x$}
}
\uncover<3->{
\psplot[linecolor=gray, plotpoints=1000]{-4}{2}{2 x exp}
\rput[l](1.7, 3){\color{gray}$y=2^x$}
}
\uncover<3>{
\psplot[linecolor=red, plotpoints=1000]{-2}{4}{0.5 x exp}
\rput[l](-1.5, 3.2){$y=0.5^x$}
}
\uncover<4->{
\psplot[linecolor=gray, plotpoints=1000]{-2}{4}{0.5 x exp}
\rput[l](-1.5, 3.2){\color{gray}$y=0.5^x$}
}
\uncover<4->{
\psplot[linecolor=red, plotpoints=1000]{-2.321928095}{4}{0.5 x exp -1 add}
\rput[r](-1.7, 2){$y=0.5^x-1$}
\psline[linecolor=blue, linestyle=dashed](-4, -1)(4, -1)
}
\end{pspicture}
%\only<handout:0| -1>{%
%\includegraphics[height=7cm]{exponential-functions/pictures/07-02-ex1a.pdf}%
%}%
%\only<handout:0| 2>{%
%\includegraphics[height=7cm]{exponential-functions/pictures/07-02-ex1b.pdf}%
%}%
%\only<handout:0| 3>{%
%\includegraphics[height=7cm]{exponential-functions/pictures/07-02-ex1c.pdf}%
%}%
%\only<4->{%
%\includegraphics[height=7cm]{exponential-functions/pictures/07-02-ex1d.pdf}%
%}%
\column{.3\textwidth}
Recall from previous lectures.
\begin{itemize}
\item<2-> Plot of $2^x$ assumed given.
\item<3-> Plot $f(-x)$ = reflect $f(x)$ across $y$ axis.
\item<4-> Plot $g(x)-1$ = shift graph $g(x)$ 1 unit down.
\end{itemize}
\end{columns}
\end{example}
\end{frame}
% end module exponential-function-ex-sketch
\subsection{The Natural Exponential Function}
% begin module exponential-function-derivative
\begin{frame}
\frametitle{Derivatives of Exponential Functions}
Compute the derivative of $f(x) = a^x$ using the definition:
\begin{align*}
\uncover<2->{f'(x) = \lim_{h\to 0} \frac{f(x+h)-f(x)}{h}} & \uncover<3->{=}  \uncover<3->{\lim_{h\to 0} \frac{\alert<handout:0| 4>{a^{x+h}}-a^x}{h}}\\
 & \uncover<4->{=}  \uncover<4->{\lim_{h\to 0} \frac{\alert<handout:0| 5>{\alert<handout:0| 4>{a^x a^h}-a^x}}{h}}\\
 & \uncover<5->{=}  \uncover<5->{\lim_{h\to 0} \frac{\alert<handout:0| 5>{\alert<handout:0| 6>{a^x} (a^h- 1)}}{h}}\\
 & \uncover<6->{=}  \uncover<6->{\alert<handout:0| 6>{a^x} \alert<handout:0| 7>{\lim_{h\to 0} \frac{a^h- 1}{h}}}\\
 & \uncover<7->{=}  \uncover<7->{a^x \alert<handout:0| 7>{f'(0)}.}
\end{align*}
\end{frame}


\begin{frame}
We have shown that, if $f(x) = a^x$ is differentiable at 0, then it is differentiable everywhere, and
\[
f'(x) = f'(0)a^x .
\]
\uncover<2->{
We leave the following theorem without proof. 
\begin{theorem}
Let $a$ be a positive number and let $f(x)=a^x$. Then the limit \[f'(0)=\lim_{h\rightarrow 0}\frac{a^h - 1}{h}\] exists. 
\end{theorem}

In fact, it can be shown, as was/will be done in Calculus I that the limit above equals $\ln a$, i.e., $f'(0)=\ln(a)$. Here, $\ln$ is the natural logarithm function (was/will be defined in Calculus I).
}
\end{frame}
% end module exponential-function-derivative

% begin module e-def
\begin{frame}
\[
\text{If}\quad  f(x) = a^x, \quad \text{then}\quad f'(x) = f'(0)a^x .
\]
The simplest differential formula occurs when $f'(0) = 1$.  Since $\lim_{h\rightarrow 0}\frac{2^h-1}{h}\approx 0.69$ and $\lim_{h\rightarrow 0}\frac{3^h-1}{h}\approx 1.10$, we expect there is a number $a$ between 2 and 3 such that $\lim_{h\rightarrow 0}\frac{a^h-1}{h} = 1$.  
\uncover<2->{
\begin{definition}[$e$]
$e$ is the number such that $\lim_{h\rightarrow 0}\frac{e^h-1}{h} = 1$.
\end{definition}
}

\begin{columns}
\column{.3\textwidth}
\includegraphics[height=4cm]{exponential-functions/pictures/exp-tangent-two.pdf}%
\column{.3\textwidth}
\uncover<handout: 1|3->{%
\includegraphics[height=4cm]{exponential-functions/pictures/exp-tangent-e.pdf}%
}%
\column{.3\textwidth}
\includegraphics[height=4cm]{exponential-functions/pictures/exp-tangent-three.pdf}%
\end{columns}
\end{frame}
% end module e-def

% begin module natural-exponential-def
\begin{frame}
\begin{definition}[Natural Exponential Function]
$e^x$ is called the natural exponential function.  Its derivative is
\[
\frac{\diff}{\diff x} \left(e^x\right) = e^x .
\]
\end{definition}
\end{frame}
% end module natural-exponential-def

% begin module derivative-e-plus-polynomial
\begin{frame}
\begin{example}[Derivative of a Polynomial and the Natural Exponential Function]
\abovedisplayskip=0pt
\belowdisplayskip=-15pt
\abovedisplayshortskip=0pt
\belowdisplayshortskip=0pt
\begin{align*}
\text{Differentiate}\quad y & = e^x+x^7.\\
\uncover<2->{\frac{\diff y}{\diff x} & = \alertNoH{3-4}{\frac{\diff}{\diff x}(e^x)} + \alertNoH{5-6}{\frac{\diff}{\diff x}(x^7)}}\\
& \uncover<3->{= \fcAnswer{4}{\alertNoH{4}{e^x}}  + \fcAnswerUncover{3}{6}{\alertNoH{6}{7x^6}.}}
\end{align*}
\end{example}
\end{frame}
% end module derivative-e-plus-polynomial

% begin module chain-rule-twice-ex2
\begin{frame}
\chainruletwice%
{ e^{\tan \left(\pi x\right)}}%
{e^{\tan \left(\pi x\right)}}%
{\tan\left(\pi x\right)}%
{\sec^2 \left(\pi x\right)}%
{\pi x}%
{\pi}%
{}%
{\pi e^{\tan \left(\pi x\right)}\sec^2 \left(\pi x\right)}%
{}%
\end{frame}
% end module chain-rule-twice-ex2

% end lecture

% begin lecture
\lect{February 14, 2014}{Lecture  6}{6}
\section{Inverse Functions}
\subsection{One-to-one Functions}
% begin module one-to-one-def
\begin{frame}
\frametitle{One-to-one Functions}
\begin{definition}[One-to-one Function]
A function $f$ is a one-to-one function if it never takes on the same value twice; that is,
\[
f(x_1) \neq f(x_2) \ \text{whenever }  \ x_1 \neq x_2 .
\]
\end{definition}
\begin{columns}[c]
\column{.5\textwidth}
\includegraphics[height=5cm]{inverse-functions/pictures/07-01-1-1def.pdf}%
\column{.5\textwidth}
$\leftarrow$ This function is not one-to-one.
\end{columns}
\end{frame}
% end module one-to-one-def

% begin module horizontal-line-test
\begin{frame}
Question: How can we tell from the graph of a function whether it is one-to-one or not?

Answer: Use the horizontal line test.

\begin{proof}[The Horizontal Line Test]
A function is one-to-one if and only if no horizontal line intersects it more than once.
\end{proof}

\begin{tabular}{cc}
\psset{xunit=0.7cm, yunit=0.7cm}
\begin{pspicture}(-5, -5)(5,5) 
\psframe*[linecolor=white](-5,-5)(5,5) 
\psaxes[ticks=none, labels=none]{<->}(0,0)(-3,-3)(3,3)
%Function formula: 1/2 (x)+1/2 
\psplot[linecolor=red, plotpoints=1000]{1}{2}{0.5 x 0.5 mul add } %Function formula: (x)^{3} 
\psplot[linecolor=red, plotpoints=1000]{-1}{1}{x 3 exp } %Function formula: 3/2 (x)+1/2 
\psplot[linecolor=red, plotpoints=1000]{-2}{-1}{0.5 x 1.5 mul add }
\end{pspicture} 
%\includegraphics[height=4cm]{inverse-functions/pictures/07-01-onetoone.pdf} 
 %\includegraphics[height=4cm]{inverse-functions/pictures/07-01-notonetoonea.pdf}%
&%
\uncover<handout:0| 2->{%
\psset{xunit=0.7cm, yunit=0.7cm}
\begin{pspicture}(-5, -5)(5,5) 
\psframe*[linecolor=white](-5,-5)(5,5) 
\psaxes[ticks=none, labels=none]{<->}(0,0)(-3,-3)(3,3)
 %Function formula: -2/5+((6/5+x)^{2}) ((x) (x))-6/25 ((6/5+x)^{2})- (((6/5+x)^{2}) (x)) 
 \psplot[linecolor=red, plotpoints=1000]{-2}{1.5}{x x 1.2 add 2 exp mul -1 mul x 1.2 add 2 exp -0.24 mul add x x mul x 1.2 add 2 exp mul add -0.4 add }
 \uncover<3->{
 \psline[linestyle=dashed](-3, 1)(3, 1)
 }
 \end{pspicture} 
}
\\
\uncover<2->{\alert<handout:0| 2>{One-to-one}} &
\uncover<3->{\alert<handout:0| 3>{Not one-to-one}}
\end{tabular}
\end{frame}
% end module horizontal-line-test


% begin module horizontal-line-test
\begin{frame}
Question: How can we tell from the graph of a function whether it is one-to-one or not?

Answer: Use the horizontal line test.

\begin{proof}[The Horizontal Line Test]
A function is one-to-one if and only if no horizontal line intersects it more than once.
\end{proof}

\begin{tabular}{cc}
&%
\only<handout:0| -2>{%
}%
\only<handout:1| 3->{%
\includegraphics[height=4cm]{inverse-functions/pictures/07-01-notonetooneb.pdf}%
}\\
\uncover<2->{\alert<handout:0| 2>{One-to-one}} &
\uncover<3->{\alert<handout:0| 3>{Not one-to-one}}
\end{tabular}
\end{frame}
% end module horizontal-line-test

\subsection{The Definition of the Inverse of $f$}
\input{../../modules/inverse-functions/inverse-function-def}
% begin module inverse-notation-warning
\begin{frame}
\alert<1->{WARNING:}

Do not mistake the $-1$ in $f^{-1}(x)$ for an exponent.
\[
f^{-1}(x) \ \text{does not mean } \ \frac{1}{f(x)} .
\]

If you want to write $\frac{1}{f(x)}$ using exponents, you can write $(f(x))^{-1}$.
\begin{itemize}
\item<2->  $f^{-1}(x)$ is the compositional inverse of $f$.
\item<3->  $\frac{1}{f(x)}$ is the multiplicative inverse of $f$.
\end{itemize}
\end{frame}
% end module inverse-notation-warning

% begin module inverse-function-equations
\begin{frame}
\[
\alert<handout:0| 4>{f^{-1}(y) = x} \qquad \Leftrightarrow \qquad \alert<handout:0| 3>{f(x) = y} .
\]
\uncover<2->{Therefore
\[
(f^{-1}\circ f)(x) = f^{-1}(\alert<handout:0| 3>{f(x)}) = \uncover<3->{\alert<handout:0| 4>{f^{-1}(\alert<handout:0| 3-4>{y})}} \uncover<4->{\alert<handout:0| 4>{ = x}.}
\]
}
\uncover<5->{
\psset{xunit=0.75cm, yunit=0.75cm}
\begin{pspicture}(-4, -2)(13,2)
\footnotesize
\rput[r] (-3.1, 0){$x$}
\psline[linewidth=3pt]{->}(-3,0)(-2.25,0)

\rput(0.25,0){
\fcMachine{$f$}{red}
}
\rput (4, 0){$f(x)$}
\psline[linewidth=3pt]{->}(2.75,0)(3.5,0)
\psline[linewidth=3pt]{->}(4.5,0)(5.25,0)
\rput(7.75,0){
\fcMachine{$f^{-1}$}{blue}
}
\rput[l](11.25, 0){$x$}
\psline[linewidth=3pt]{->}(10.25,0)(11,0)
\end{pspicture}
%\includegraphics[height=2cm]{inverse-functions/pictures/07-01-machinesb.pdf}%
}
\uncover<6->{
Switch the roles of $x$ and $y$:
\[
\alert<handout:0| 8>{f^{-1}(x) = y} \qquad \Leftrightarrow \qquad \alert<handout:0| 9>{f(y) = x} .
\]
}
\uncover<7->{Therefore
\[
(f\circ f^{-1})(x) = f(\alert<handout:0| 8>{f^{-1}(x)}) = \uncover<8->{\alert<handout:0| 9>{f(\alert<handout:0| 8-9>{y})}} \uncover<9->{\alert<handout:0| 9>{ = x}.}
\]
}
\uncover<10->{
\psset{xunit=0.75cm, yunit=0.75cm}
\begin{pspicture}(-4, -2.5)(13,2.5)
\footnotesize
\rput[r] (-3.1, 0){$x$}
\psline[linewidth=3pt]{->}(-3,0)(-2.25,0)

\rput(0.25,0){
\fcMachine{$f^{-1}$}{blue}
}
\rput (4, 0){ $f^{-1}(x)$}
\psline[linewidth=3pt]{->}(2.75,0)(3.4,0)
\psline[linewidth=3pt]{->}(4.7,0)(5.25,0)
\rput(7.75,0){
\fcMachine{$f$}{red}
}
\rput (11.75, 0){$x$}
\psline[linewidth=3pt]{->}(10.25,0)(11,0)
\end{pspicture}
%\includegraphics[height=2cm]{inverse-functions/pictures/07-01-machinesa.pdf}%
}
\end{frame}
% end module inverse-function-equations

% begin module inverse-function-solve-for
\begin{frame}
\frametitle{How to Find the Inverse of a One-to-one Function}
\begin{enumerate}
\item<1-| alert@3>  Write $y = f(x)$.
\item<1-| alert@4-5>  Solve this equation for $x$ in terms of $y$ (if possible).
\item<1-| alert@6>  Interchange $x$ and $y$.  The resulting equation is $y = f^{-1}(x)$. 
\end{enumerate}
\uncover<2->{
\begin{example}%[Example 4, p. 388]
If $f(x) = x^3 + 2$, find a formula for $f^{-1}(x)$.
\begin{align*}
\uncover<3->{y} & \uncover<3->{=}  \uncover<3->{x^3 + 2}\\
\uncover<4->{x^3} & \uncover<4->{=}  \uncover<4->{y - 2}\\
\uncover<5->{\alert<handout:0| 6>{x}} & \uncover<5->{=}  \uncover<5->{\sqrt[3]{\alert<handout:0| 6>{y} - 2}}\\% 
\uncover<6->{\alert<handout:0| 6>{y}} & \uncover<6->{=}  \uncover<6->{\sqrt[3]{\alert<handout:0| 6>{x} - 2}} \qquad \uncover<6->{\alert<handout:0| 6>{\text{(New equation.)}}}
\end{align*}
\uncover<7->{
Therefore $f^{-1}(x) = \sqrt[3]{x - 2}$.
}
\end{example}
}
\end{frame}
% end module inverse-function-solve-for

% begin module guess-and-check
\begin{frame}
\begin{example}[Guess and Check]
If $f(x) = 2x + \sin 2x + e^{\frac{x}{2}}$, find $f^{-1}(1)$. You do not need to show that $f$ has an inverse.
\begin{align*}
\uncover<2->{f\alert<handout:0| 2-3>{(\uncover<3-| handout:0>{0})}} & \uncover<2->{=}  \uncover<2->{2\alert<handout:0| 2-3>{(\uncover<3-| handout:0>{0})}  + \sin 2\alert<handout:0| 2-3>{(\uncover<3-| handout:0>{0})}  + e^{\alert<handout:0| 2-3>{\frac{( \uncover<3-| handout:0>{0})}{2}} }}\\ 
 & \uncover<2->{=}  \uncover<3-| handout:0>{0 + 0 + 1} \\
 & \uncover<2->{=}  \uncover<2->{1.} \\
\uncover<4->{\text{Therefore}\quad f^{-1}(1)} & \uncover<4->{=}  \uncover<4-| handout:0>{0.}
\end{align*}
\end{example}
\end{frame}
% end module guess-and-check

% begin module inverse-function-graph
\begin{frame}
\begin{tabular}{cc}
\psset{xunit=1cm, yunit=1cm}
\begin{pspicture}(-1.5,-1.5)(3.5,3.2)
\psaxes[ticks=none, labels=none]{<->}(0,0)(-1.5,-1.5)(3.4,3.1)
\psLabels{3.4}{3.1}
\uncover<6->{
\psline(2.5, 1)(1, 2.5)
\psline(1.85, 1.65)(1.95, 1.75)(1.85, 1.85)
\psline(1.85, 1.85)(1.75, 1.95)(1.65, 1.85)

\psline(1.9875, 1.3125)(2.1875, 1.5125)
\psline(2.0625, 1.2375)(2.2625, 1.4375)
\psline(1.3125, 1.9875)(1.5125, 2.1875)
\psline(1.2375, 2.0625)(1.4375, 2.2625)

\psline[linecolor=blue](-1.35, -1.35)(2.8,2.8)
\rput[l](-1, -1.2){\footnotesize $y=x$}
}
\uncover<2->{
\psFullDot{2.5}{1}
\rput[lt](2.6, 1.1){\footnotesize $(a,b)$}
}
\uncover<5->{
\psFullDot{1}{2.5}
\rput[rb](0.9, 2.6){\footnotesize $(b,a)$}
}
\end{pspicture}
%\ \only<handout:0| -4>{%
%\includegraphics[height=4cm]{inverse-functions/pictures/07-01-reflecta.pdf}%
%}%
%\only<handout:0| 5>{%
%\includegraphics[height=4cm]{inverse-functions/pictures/07-01-reflectb.pdf}%
%}%
%\only<handout:1| 6->{%
%\includegraphics[height=4cm]{inverse-functions/pictures/07-01-reflectc.pdf}%
%}%
&%
\psset{xunit=1cm, yunit=1cm}
\begin{pspicture}(-1.55,-1.5)(3.5,3.2)
\psaxes[ticks=none, labels=none]{<->}(0,0)(-1.55,-1.5)(3.4,3.1)
\psLabels{3.4}{3.1}
\uncover<7->{
\psline(0.75, 2.36359)(2.36359, 0.75)
\psline(1.65679, 1.65679)(1.55679, 1.75679)(1.45679, 1.65)
\psline(1.65679, 1.65679)(1.75679, 1.55679)(1.65679, 1.45679)
\psFullDot{0.75}{2.36359}
\psFullDot{2.36359}{0.75}

\psline[linecolor=blue](-1.35, -1.35)(2.8,2.8)
\rput[l](-1, -1.2){\footnotesize $y=x$}
\psplot[linecolor=red, plotpoints=1000]{-0.292893219}{3}{x 1 add ln  0.693147181 div 1 sub}
\rput[lb](0.9, 2.4){\footnotesize $y=f^{-1}(x)$}
%Function formula: 2^{1+x}-1 
\psplot[linecolor=red, plotpoints=1000]{-1.5}{0.95}{-1 2 x 1 add exp add }
\psline[linecolor=blue](-1.4, -1.4)(2.9,2.9)
\rput[l](-1, -1.2){\footnotesize $y=x$}
\rput[tr](2.8, 0.4){\footnotesize $y=f(x)$}
}
\end{pspicture} 
%\only<handout:0| -6>{%
%\includegraphics[height=4cm]{inverse-functions/pictures/07-01-reflect-functionb.pdf}%
%}%
%\only<handout:1| 7->{%
%\includegraphics[height=4cm]{inverse-functions/pictures/07-01-reflect-f unctiona.pdf}%
%}%
\end{tabular}

\footnotesize 
Interchanging $x$ and $y$ suggests a relation between the graphs of $f^{-1}$ and $f$:
\begin{itemize}
\item<2->  Suppose $(a,b)$ is on the graph of $f$.
\item<3->  Then $f(a) = b$.
\item<4->  Then $f^{-1}(b) = a$.
\item<5->  Then $(b,a)$ is on the graph of $f^{-1}$.
\item<6->  $(b,a)$ is the reflection of $(a,b)$ in the line $y = x$.
\item<7->  Thus the graph of $f^{-1}$ is obtained by reflecting the graph of $f$ in the line $y = x$.
\end{itemize}
\end{frame}
% end module inverse-function-graph

% begin module inverse-function-ex5
\begin{frame}
\begin{example}%[Example 5, p. 388]
\begin{columns}[c]
\column{.5\textwidth}
\psset{xunit=0.5cm, yunit=0.5cm}
\begin{pspicture}(-5, -5)(5,5) 
\psframe*[linecolor=white](-5,-5)(5,5) 
\psaxes[ticks=none, labels=none]{<->}(0,0)(-4.5,-5)(4.5,5)
\psline(1, -0.1)(1, 0.1)
\rput[t](1, -0.2){\tiny$1$}
\psline(-1, -0.1)(-1, 0.1)
\psline(-0.1, 1)(0.1, 1)
\rput[br](-0.2, 1){\tiny$1$}
\psline(-0.1, -1)(0.1, -1)
\uncover<3>{
%Function formula: sqrt{}(- (x)) 
\psplot[linecolor=red, plotpoints=1000]{-4.5}{0}{x -1 mul sqrt }
\rput(-2, 2.2){\tiny$y=\sqrt{-x}$}
}
\uncover<4->{
%Function formula: sqrt{}(- (x)) 
\psplot[linecolor=gray, plotpoints=1000]{-4.5}{0}{x -1 mul sqrt }
\rput(-2, 2.2){\color{gray}\tiny$y=\sqrt{-x}$}
}
\uncover<2>{
 %Function formula: sqrt{}(x) 
\psplot[linecolor=red, plotpoints=1000]{0}{4.5}{x sqrt } 
\rput(2.4, 1){\tiny $y=\sqrt{x}$}
}
\uncover<3->{
 %Function formula: sqrt{}(x) 
\psplot[linecolor=gray, plotpoints=1000]{0}{4.5}{x sqrt } 
\rput(2.4, 1){\color{gray}\tiny $y=\sqrt{x}$}
}
\uncover<4>{
%Function formula: sqrt{}(- (x)-1) 
\psplot[linecolor=red, plotpoints=1000]{-4.5}{-1}{-1 x -1 mul add sqrt }
\rput[r](-2.3, 0.55){\tiny$y=f(x)$}
}
\uncover<5->{
%Function formula: sqrt{}(- (x)-1) 
\psplot[linecolor=gray, plotpoints=1000]{-4.5}{-1}{-1 x -1 mul add sqrt }
\rput[r](-2.3, 0.55){\color{gray}\tiny$y=f(x)$}
}

\uncover<5>{
%Function formula: - ((x)^{2})-1 
\psplot[linecolor=red, plotpoints=1000]{0}{2}{-1 x 2 exp -1 mul add } 
\rput[l](1.3, -2){\tiny$y=f^{-1}(x)$}
\psline[linecolor=blue, linestyle=dashed] (-4.5, -4.5)(4.5, 4.5)
\rput[tl](-3, -3.2){\tiny $y=x$}
}
\end{pspicture} 
%\ \only<handout:0| -1>{%
%\includegraphics[height=4.5cm]{inverse-functions/pictures/07-01-ex5a.pdf}%
%}%
%\only<handout:0| 2>{%
%\includegraphics[height=4.5cm]{inverse-functions/pictures/07-01-ex5b.pdf}%
%}%
%\only<handout:0| 3>{%
%\includegraphics[height=4.5cm]{inverse-functions/pictures/07-01-ex5c.pdf}%
%}%
%\only<handout:0| 4>{%
%\includegraphics[height=4.5cm]{inverse-functions/pictures/07-01-ex5d.pdf}%
%}%
%\only<handout:1| 5->{%
%\includegraphics[height=4.5cm]{inverse-functions/pictures/07-01-ex5e.pdf}%
%}%

\column{.5\textwidth}
Sketch the graph of $f(x) = \sqrt{-x - 1}$ and its inverse function.
\end{columns}
\begin{itemize}
\item<2->  First draw the graph of $y = \sqrt{x}$.
\item<3->  $y = \sqrt{-x}$ is the reflection of $y = \sqrt{x}$ in the $y$-axis.
\item<4->  $y = f(x) = \sqrt{-x - 1}$ is the shift of $y = \sqrt{-x}$ one unit to the left.
\item<5->  $y = f^{-1}(x)$ is the reflection of $y = f(x)$ across the line $y = x$.
\end{itemize}
\end{example}
\end{frame}
% end module inverse-function-ex5
% begin module inverse-function-solve-for
\begin{frame}
%\frametitle{Inverse of a ne-to-one Function}
\begin{example}[\uncover<16->{\alert<handout:0| 16,17>{What if we change the problem to $x\leq -\frac{2}3$?}}]
Given: $\alert<handout:0| 2>{f(x) =  3x^2+4x-7}$ \alert<handout:0| 3,16,17>{with domain $x\only<1-16| handout:0>{\geq}\only<17->{\leq} -\frac{2}{3}$}.  Find $f^{-1}(x)$.
\begin{columns}
\column{0.4\textwidth}
\psset{xunit=0.35cm, yunit=0.35cm}
\begin{pspicture}(-9,-9)(5,5)
\psframe*[linecolor=white](-9,-9)(5,5) 
\tiny
\psaxes[ticks=none, labels=none]{<->}(0,0)(-9,-9)(4.7,4.7)
\uncover<2-16| handout:0>{ %
\psplot[linecolor=\psColorGraph, plotpoints=1000] {-0.66}{1.401612274}{x 2 exp 3 mul x 4 mul add -7 add }
}
\uncover<17->{ %
\psplot[linestyle=dashed, linecolor=gray!50, plotpoints=1000]{-0.66}{1.401612274}{x 2 exp 3 mul x 4 mul add -7 add }
}
\uncover<2,17>{ %
\psplot[linecolor=\psColorGraph, plotpoints=1000]{-2.7349}{-0.67}{x 2 exp 3 mul x 4 mul add -7 add }
}
\uncover<3-16| handout:0>{ %
\psplot[linestyle=dashed, linecolor=gray!50, plotpoints=1000] {-2.7349}{-0.67}{x 2 exp 3 mul x 4 mul add -7 add }
}
\uncover<14->{
\psline[linecolor=blue, linestyle=dashed](-6.5, -6.5)(4.5,4.5)
}
\uncover<15-16| handout:0>{
\psplot[linecolor=red, plotpoints=1000]{-8.33333}{4.5}{-0.666667 25 x 3 mul add sqrt 0.333333 mul add }
}
\uncover<17->{
\psplot[linestyle=dashed, linecolor=gray!50, plotpoints=1000] {-8.33333}{4.5}{-0.666667 25 x 3 mul add sqrt 0.333333 mul add }
}
\uncover<15-16>{
\psplot[linecolor=gray!50, linestyle=dashed, plotpoints=1000] {-8.33333}{4.5}{-0.666667 25 x 3 mul add sqrt -0.333333 mul add }
}
\uncover<17->{
\psplot[linecolor=\psColorGraph, plotpoints=1000] {-8.33333}{4.5}{-0.666667 25 x 3 mul add sqrt -0.333333 mul add }
}
\uncover<15-16| handout:0>{\rput[lb](-6, 1){$y=f^{-1}(x)$}}
\uncover<11->{\rput (-3.5, -8 ){$(-\frac{2}{3}, -\frac{25}{3})$}}
\uncover<2-16| handout:0>{\rput[tl](1.8, 4.45){$y=f(x)$}}
\uncover<14->{\rput[l] (-7.8, -2.2 ){$(-\frac{25}{3}, -\frac{2}{3})$}}
\uncover<17->{\rput[rt](4.5, -3){$y=f^{-1}(x)$}}
\uncover<17>{ \rput[tr](-3, 4.5){$y=f(x)$}}
\uncover<14->{\psFullDot{-8.33333}{-0.666667}}
\psFullDot{-0.666667}{-8.33333}
\end{pspicture} 
\uncover<13->{Final }\uncover<12->{answer}\uncover<13->{, \alert<handout:0| 13>{relabelled}:}
\[
\uncover<12->{
f^{-1}(\only<12| handout:0>{y}\only<13->{\alert<handout:0| 13>{x}} )=-\frac{2}{3} \only<1-16| handout:0>{+}\only<17->{\alert<handout:0| 17>{-}} \frac{\sqrt{25 +3\only<12| handout:0>{y} \only<13->{\alert<handout:0| 13>{x}}\phantom{y} }}{3}\quad.
}
\]

\column{0.6\textwidth}

\[\begin{array}{rcl}
\uncover<4->{3x^2+4x-7&=&y } \\
\uncover<4->{\alert<handout:0| 7>{3}x^2+\alert<handout:0| 6>{4}x+\alert<handout:0| 8>{(-7-y)}&=&0 }
\end{array}
\]
\uncover<5->{That's \alert<handout:0| 6,7,8>{a quadratic equation in $x$}. Solve:}
\[\begin{array}{l}
\uncover<5->{
\phantom{=}\displaystyle \frac{-\alert<handout:0| 6>{4} \pm \sqrt{\alert<handout:0| 6>{4}^2-4\cdot\alert<handout:0| 7>{3}\cdot\alert<handout:0| 8>{(-y-7)} }}{2\cdot\alert<handout:0| 7>{3}} \\
%~&=& \frac{-2 \pm \sqrt{25+3y}}{3}\\
}
\\
\uncover<9->{=\displaystyle-\frac{2 \pm \sqrt{25+3y}}{3}=} \uncover<10->{\displaystyle-\frac{2}3 \pm \frac{\sqrt{25+3y}}{3}\quad .}
\end{array}
\]
\uncover<11->{
We are given $x\only<11-16| handout:0>{\geq}\only<17->{\alert<handout:0| 17>{\leq}}-\frac{2}3 $, therefore $x=-\frac{2}{3}\only<11-16| handout:0>{+}\only<17->{\alert<handout:0| 17>{-}}\frac{\sqrt{25+3y}}{3}=f^{-1}(y)$.
}
\end{columns}
\vspace{-10pt}
\end{example}
\end{frame}
% end module inverse-function-solve-for

% end lecture

% begin lecture
\lect{February 21, 2014}{Lecture  8}{8}
\section{Logarithmic Functions}
\subsection{Definition and Properties}
\input{../../modules/logarithms/logarithm-def}
% begin module logarithm-def-ex1
\begin{frame}
If $x > 0$, then $\log_a x$ is the exponent to which the base $a$ must be raised to give $x$.
\begin{example}
Evaluate:
\begin{enumerate}
\item<1-| alert@2-3> $\log_3 81 =$ \fcAnswerUncover{2}{3}{$4$ because $3^4 = 81$.}
\item<1-| alert@4-5> $\log_{25} 5 =$ \fcAnswerUncover{2}{5}{$\frac{1}{2}$ because $25^{\frac{1}{2}} =\sqrt{25}= 5$.}
\item<1-| alert@6-7> $\log_{10} 0.001 =$ \fcAnswerUncover{2}{7}{$-3$ because $10^{-3} = 0.001$.}
\end{enumerate}
\end{example}
\end{frame}
% end module logarithm-def-ex1

% begin module log-and-exp
\begin{frame}
\begin{columns}[c]
\column{.6\textwidth}
\psset{xunit=1cm, yunit=1cm}
\begin{pspicture}(-2,-2.1)(4.2, 4.2)
\psaxes[ticks=none, labels=none]{<->}(0,0)(-2,-2.1)(4.2, 4.2)
\psline(-0.1, 1)(0.1,1)
\rput[r](-0.2, 1){\footnotesize$1$}
\rput(0.9, 3){\footnotesize$y=a^x$}
%Function formula: 2^{x} 
\psplot[linecolor=red, plotpoints=1000]{-2}{2}{2 x exp }
\psplot[linecolor=red, plotpoints=1000]{0.25}{4}{x ln 0.693147181 div }
\psline[linestyle=dashed, linecolor=blue](-1.9, -1.9)(4,4) 
\rput[tl](2, 0.7){\footnotesize$y=\log_ax$}
\rput[tl](-1, -1.1){\footnotesize$y=x$}
\end{pspicture} 
%\includegraphics[height=7cm]{logarithms/pictures/07-03-logandexpb.pdf}%
\column{.4\textwidth}
\begin{itemize}
\item  Suppose $a > 1$.
\item<2-| alert@3-4>  Domain of $a^x$: \fcAnswerUncover{2}{4}{$\mathbb{R}$.}
\item<2-| alert@5-6>  Range of $a^x$: \fcAnswerUncover{2}{6}{$(0, \infty )$.}
\item<2-| alert@7-8>  Domain of $\log_a x$: \fcAnswerUncover{2}{8}{$(0, \infty )$.}
\item<2-| alert@9-10>  Range of $\log_a x$: \fcAnswerUncover{2}{10}{$\mathbb{R}$.}
\item<11->  $\log_a (a^x) = x$ for $x\in \mathbb{R}$.
\item<11->  $a^{\log_a x} = x$ for $x > 0$.
%\item<12-| alert@13-14>  $\lim_{x\rightarrow \infty}\log_a x = \uncover<14->{\infty .}$
%\item<12-| alert@15-16>  $\lim_{x\rightarrow 0^+}\log_a x = \uncover<16->{-\infty .}$
\end{itemize}
\end{columns}
\end{frame}
% end module log-and-exp

% begin module logarithm-graphs
\begin{frame}
\begin{center}
Graphs of various logarithmic functions with $a > 1$
\psset{xunit=1cm, yunit=1cm}
\begin{pspicture}(-0.5,-4.5)(7.5,3.2)
\psaxes[ticks=none, labels=none]{<->}(0,0)(-0.5,-4.5)(7.5,3.2)
\psline(1,-0.1)(1,0.1)
%Function formula: ln(x)/ln(2)
\psplot[linecolor=red, plotpoints=1000]{0.044194174}{7.5}{x ln 0.693147181 div}
\rput[r](3, 1.8){\footnotesize $y=\log_2 x$}
\uncover<2->{
%Function formula: ln{x}/ln(3) 
\psplot[linecolor=black, plotpoints=1000]{0.007127781}{7.5}{x ln 1.098612289 div}
\rput[l](3.6, 1.6 ){\footnotesize $y=\log_3 x$}
}
\uncover<3->{
%Function formula: ln{x}/ln(5) 
\psplot[linecolor=blue, plotpoints=1000]{0.000715542}{7.5}{x ln 1.609437912 div}
\rput[l](3.7, 1.1){\footnotesize $y=\log_5 x$}
}
\uncover<4->{
%Function formula: ln{x}/ln(5) 
\psplot[linecolor=green, plotpoints=1000]{0.000031623}{7.5}{x ln 2.302585093 div}
\rput[tl](3.6, 0.6){\footnotesize $y=\log_{10} x$}
}
%\rput(6, 1){\color{red!1} .}
\end{pspicture} 
%\ \only<handout:0| -1>{%
%\includegraphics[height=6cm]{logarithms/pictures/07-03-manylogsa.pdf}%
%}%
%\only<handout:0| 2>{%
%\includegraphics[height=6cm]{logarithms/pictures/07-03-manylogsb.pdf}%
%}%
%\only<handout:0| 3>{%
%\includegraphics[height=6cm]{logarithms/pictures/07-03-manylogsc.pdf}%
%}%
%\only<4->{%
%\includegraphics[height=6cm]{logarithms/pictures/07-03-manylogsd.pdf}%
%}%
\pause\pause\pause
\end{center}
\end{frame}
% end module logarithm-graphs
% begin module logarithm-properties
\begin{frame}
\begin{theorem}[Properties of Logarithmic Functions]
If $a > 1$, the function $f(x) = \log_a x$ is a one-to-one, continuous, increasing function with domain $(0, \infty )$ and range $\mathbb{R}$.  If $x, y, a, b > 0$ and $r$ is any real number, then
\begin{enumerate}
\item  $\log_a (xy) = \log_a x + \log_a y$.
\item  $\log_a \left( \frac{x}{y}\right) = \log_a x - \log_a y$.
\item  $\log_a (x^r) = r\log_a x$.
\item  $\log_{\frac{1}{a}}x=-\log_a x$
\item  $\log_{a}b=\frac{1}{\log_b a}$
\item  $\log_{a}(x)=\log_b x \log_{a} b=\frac{\log_b x}{\log_{b} a}=  \frac{\ln x}{\ln a}$
\end{enumerate}
\end{theorem}
\end{frame}
% end module logarithm-properties

% begin module logarithm-properties-ex2
\begin{frame}
\begin{example}
Use the properties of logarithms to evaluate the following:
\begin{columns}[t]
\column{.5\textwidth}
\begin{align*}
& \invisible{=} \log_{\alertNoH{2}{4}} \alertNoH{3}{2} + \log_{\alertNoH{2}{4}} \alertNoH{3}{32} \\
&\uncover<2->{=}  \uncover<2-| handout:0>{ \log_{ \alertNoH{2}{4}} (\alertNoH{4}{ \alertNoH{3}{2}\cdot \alertNoH{3}{32}} )} \\
&\uncover<4->{=}  \uncover<4-| handout:0>{ \alertNoH{5,6}{ \log_{ \alertNoH{7}{4}} (\alertNoH{4,9}{64})}} \\
&\uncover<5->{\alertNoH{5,6}{=}}  \fcAnswerNoH{6}{\alertNoH{8}{3}} \\
& \uncover<6-| handout:0>{\text{(because ${\alertNoH{7}{4}}^{\alertNoH{8}{3}} = \alertNoH{9}{64}$.)}}
\end{align*}
\column{.5\textwidth}
\begin{align*}
& \invisible{=} \log_{\alertNoH{10}{2}} 80 - \log_{ \alertNoH{10}{2} } 5 \\
&\uncover<10->{=}  \uncover<10-| handout:0>{ \log_{\alertNoH{10}{2}} \left(\alertNoH{11}{ \frac{80}{5}} \right) } \\
&\uncover<11->{=}  \uncover<11-| handout:0>{\alertNoH{12,13 }{ \log_{\alertNoH{14}{2}} (\alertNoH{11,16}{16})}} \\
&\uncover<12->{\alertNoH{12,13}{=}}  \fcAnswerNoH{13}{\alertNoH{15}{4}} \\
& \uncover<13-| handout:0>{\text{(because $\alertNoH{14}{2}^{\alertNoH{15}{4}} = \alertNoH{16}{16}$.)}}
\end{align*}
\end{columns}
\end{example}
\end{frame}
% end module logarithm-properties-ex2

\subsection{Natural Logarithms}
\input{../../modules/logarithms/natural-logarithm-def}
% begin module natural-logarithm-def-ex5
\begin{frame}
\begin{example}
\begin{align*}
\text{Solve the equation} \quad e^{5-3x} & =  10 \\
\uncover<2-| handout:0>{\alertNoH{3}{\ln} ({\alertNoH{3}{e}}^{5-3x})} & \uncover<2->{=}  \uncover<2-| handout:0>{\ln 10} \\
\uncover<3-| handout:0>{\alertNoH{4}{5-}3x} & \uncover<3->{=}  \uncover<3-| handout:0>{\ln 10} \\
\uncover<4-| handout:0>{\alertNoH{5}{3}x} & \uncover<4->{=}  \uncover<4-| handout:0>{\alertNoH{4}{5-}\ln 10} \\
\uncover<5->{x} & \uncover<5->{=}  \uncover<5-| handout:0>{\frac{5-\ln 10}{\alertNoH{5}{3}}} \\
\uncover<6->{\text{Calculator:}\quad x} & \uncover<6->{\approx 0.8991.}
\end{align*}
\end{example}
\end{frame}
% end module natural-logarithm-def-ex5

% begin module exponential-equation3
\begin{frame}
\begin{columns}
\column{.55\textwidth}
\begin{example}[Finding the intersection of two exponential graphs]
Find the point(s) of intersection of $y = e^{2x}+4$ and $y = 5e^x$.  
\abovedisplayskip=0pt
\belowdisplayskip=0pt
\begin{align*}
\uncover<2->{ e^{2x} + 4 & = 5e^x }\\
\uncover<3-| handout:0>{\alert<handout:0| 5-6>{e^{2x}} - 5\alert<handout:0| 4>{e^x} + 4} & \uncover<3->{ = 0} \\
\intertext{\uncover<4->{Substitute $\alert<handout:0| 10>{u = \uncover<4-| handout:0>{e^x}}$:}}
\uncover<4-| handout:0>{\alert<handout:0| 5-6>{\uncover<6->{u^2}} - 5\alert<handout:0| 4>{u} + 4} & \uncover<4->{ = 0} \\
\uncover<7->{\alert<handout:0| 7-8>{(\uncover<8-| handout:0>{u-4})(\uncover<8-| handout:0>{u-1})}} & \uncover<7->{ = 0 } 
\end{align*}
\begin{align*}
\uncover<9->{ \alert<handout:0| 10>{u} } & \uncover<9->{=} \uncover<9-| handout:0>{ 4 } & \uncover<9->{\text{or}} & & \uncover<9->{\alert<handout:0| 10>{u}} & \uncover<9->{=} \uncover<9-| handout:0>{ 1} \\
\uncover<10-| handout:0>{ \alert<handout:0| 10>{e^x} } & \uncover<10-| handout:0>{ = 4 } & \uncover<10-| handout:0>{\text{or}} & & \uncover<10-| handout:0>{\alert<handout:0| 10>{e^x}} & \uncover<10-| handout:0>{ = 1} \invisible{99999999} \\
\uncover<11-| handout:0>{ \alert<handout:0| 11-12>{x} } & \uncover<11-| handout:0>{ \alert<handout:0| 11-12>{ = \uncover<12->{\ln 4} }} & \uncover<11-| handout:0>{\text{or}} & & \uncover<11-| handout:0>{\alert<handout:0| 13-14>{x}} & \uncover<11-| handout:0>{ \alert<handout:0| 13-14>{ = \uncover<14->{0} }} 
\end{align*}
\uncover<15-19| handout:0>{Therefore the points of intersection are \alert<handout:0| 15-16>{$(\ln 4,\uncover<16->{20})$} and \alert<handout:0| 17-18>{$(0,\uncover<18->{5})$}.}  
\end{example}

\column{.4\textwidth}

\uncover<19-| handout:0>{%
\psset{xunit=0.3cm,yunit=0.13cm,algebraic=true,dotstyle=o,dotsize=3pt 0,linewidth=0.8pt,arrowsize=3pt 2,arrowinset=0.25}
\begin{pspicture*}(-6.92,-4.61)(9.16,53.92)
\psaxes[labelFontSize=\scriptstyle,xAxis=true,yAxis=true,Dx=5,Dy=5,ticksize=-2pt 0,subticks=0]{<->}(0,0)(-6.92,-4.61)(9.16,53.92)
\psplot[linecolor=blue,plotpoints=200]{-6.915925925925918}{9.164074074074078}{5*EXP(x)}
\psplot[linecolor=red,plotpoints=200]{-6.915925925925918}{9.164074074074078}{EXP(2*x)+4}
\begin{scriptsize}
\rput[tl](2.6,40.0){$y=5e^x$}
\rput[tl](-7.0,6.5){$y=e^x+4$}
\psdots[dotstyle=*](0,5)
\rput[bl](0.1,3.99){$(0, 5)$}
\psdots[dotstyle=*](1.39,20)
\rput[bl](1.46,18.52){$(\ln4,20)$}
\end{scriptsize}
\end{pspicture*}
}%

\end{columns}
\end{frame}
% end module exponential-equation3

% begin module exponential-equation4
\begin{frame}
\begin{example}[Solving a quadratic exponential equation]
Solve for $x$.
\abovedisplayskip=0pt
\belowdisplayskip=0pt
\begin{align*}
4^{x+1} & = 2^{x+2} + 3 \\
\uncover<2-| handout:0>{\alertNoH{ 6}{4^{x+1}} - \alertNoH{ 4}{2^{x+2}} - 3} & \uncover<2->{ = 0} \\
\intertext{\uncover<3->{Substitute $\alertNoH{ 10}{u = \uncover<3-| handout:0>{2^x}}$.  Then $\alertNoH{ 3-4}{2^{x+2} = \uncover<4-| handout:0>{4u}}$ and $\alertNoH{ 5-6}{4^{x+1} = \uncover<6-| handout:0>{4u^2.}}$ }}
\uncover<3-| handout:0>{\alertNoH{ 6}{\uncover<6->{4u^2}} - \alertNoH{ 4}{\uncover<4->{4u}} - 3} & \uncover<3->{ = 0} \\
\uncover<7->{\alertNoH{ 7-8}{(\uncover<8-| handout:0>{2u-3})(\uncover<8-| handout:0>{2u+1})}} & \uncover<7->{ = 0 }
\end{align*}
\begin{align*}
\uncover<9->{ \alertNoH{ 10}{u} } & \uncover<9->{=} \uncover<9-| handout:0>{ \frac{3}{2} } & \uncover<9->{\text{or}} & & \uncover<9->{\alertNoH{ 10}{u}} & \uncover<9->{=} \uncover<9-| handout:0>{ -\frac{1}{2}} \\
\uncover<10-| handout:0>{ \alertNoH{ 10}{2^x} } & \uncover<10-| handout:0>{ = \frac{3}{2} } & \uncover<10-| handout:0>{\text{or}} & & \uncover<10-| handout:0>{\alertNoH{ 10}{2^x}} & \uncover<10-| handout:0>{ = -\frac{1}{2}} \invisible{99999999} \\
\uncover<11-| handout:0>{ \alertNoH{ 11-12}{x} } & \uncover<11-| handout:0>{ \alertNoH{ 11-12}{ = \uncover<12->{\log_2 \Big( \frac{3}{2}\Big)} }} & & & \uncover<11-| handout:0>{\alertNoH{ 13-14}{\uncover<-13>{x}}} & \uncover<11-| handout:0>{ \alertNoH{ 13-14}{ \only<-13>{ =} \only<14->{\text{no solution}} }} \\
\uncover<15-| handout:0>{x & = \frac{\ln (3/2)}{\ln2} & & & & }
\end{align*}
\uncover<16-| handout:0>{Therefore the solution is $x = \frac{\ln (3/2)}{\ln 2} \approx  0.584962501$.}
\end{example}
\end{frame}
% end module exponential-equation4

% begin module natural-logarithm-def-ex8
\begin{frame}
\begin{example}%[Example 8, p. 408]
Draw the graph of $y = \ln (x - 2) -1$.
\ \only<handout:0| -1>{%
\includegraphics[height=6cm]{logarithms/pictures/07-03-ex8a.pdf}%
}%
\only<handout:0| 2>{%
\includegraphics[height=6cm]{logarithms/pictures/07-03-ex8b.pdf}%
}%
\only<3->{%
\includegraphics[height=6cm]{logarithms/pictures/07-03-ex8c.pdf}%
}%
\end{example}
\end{frame}
% end module natural-logarithm-def-ex8

% begin module inverse-function-solve-for
\begin{frame}
%\frametitle{Inverse of a ne-to-one Function}
\vskip -0.2cm
\begin{example}
\begin{columns}
\column{0.35\textwidth}
Find $f^{-1}(x)$ for $\displaystyle f(x) = \frac{e^x-e^{-x}}{e^x+e^{-x}}  $. 
\uncover<18->{\alertNoH{18}{$f=\tanh$ = hyperbolic tangent function.}}

\psset{xunit=0.6cm, yunit=0.6cm}
\begin{pspicture}(-3.05,-3.1)(3.15,3.2)
\psframe*[linecolor=white](-3.05,-3.1)(3.15,3.2)
\tiny
\psaxes[ticks=none, labels=none]{<->}(0,0)(-3.05,-3.1)(3.05,3.1)
\fcLabels{3.05}{3.1}
%Function formula: (679570457/250000000^{x}- (679570457/250000000^{- (x)}))/(679570457/250000000^{- (x)}+679570457/250000000^{x})
\uncover<1-16>{
\rput(-2,-0.5){$y=f(x)$}
\psplot[linecolor=\fcColorGraph, plotpoints=1000]{-3}{3}{2.71828 x -1 mul exp -1 mul 2.71828 x exp add 2.71828 x exp 2.71828 x -1 mul exp add div }
}
\uncover<17->{
\rput(-2,-0.5){$y=f(x)$}
\psplot[linecolor=gray!50, plotpoints=1000]{-3}{3}{2.71828 x -1 mul exp -1 mul 2.71828 x exp add 2.71828 x exp 2.71828 x -1 mul exp add div }
}
%Function formula: 1/2 (\ln{}((1+x)/(1- (x))))
\uncover<17->{
\psline[linestyle=dashed, linecolor=blue](-3, -3)(3,3)
\rput[tl](1.1,2.5){$y=f^{-1}(x)$}
\psplot[linecolor=\fcColorGraph, plotpoints= 1000]{ -0.994}{ 0.994 }{x 1 add x -1 mul 1 add div ln 0.5 mul }
}
\end{pspicture}
\uncover<17->{Final }\uncover<16->{answer}\uncover<17->{, \alertNoH{17}{relabelled}:}
$\displaystyle 
\uncover<16->{
f^{-1}( \only<handout:0|16>{y}\only<17->{\alertNoH{17}{x}} )=\frac12\ln\left( \frac{1+\only<handout:0|16>{y} \only<17->{ \alertNoH{17}{x}} }{1-\only<handout:0|16>{y} \only< 17->{ \alertNoH{17 }{x}}}\right)
}
$
\column{0.65\textwidth}
$ \begin{array}{@{}r@{~}c@{~}l@{}l|l}
\displaystyle \uncover<2->{ \frac{ \alertNoH{3}{e^x}- \alertNoH{4, 5}{e^{-x} } }{ \alertNoH{3}{e^x}+\alertNoH{4,5}{e^{-x}}} &=&y\uncover<3->{ &&\begin{array}{l} \text{Set } \alertNoH{3, 12}{u=e^x} \\\uncover<4->{ \alertNoH{4, 5}{e^{-x}=} \fcAnswer{5}{\frac{1}{e^x}= \frac{1}{u}}} \end{array}}}  \\
\displaystyle\uncover<3->{ \frac{ \left(\alertNoH{3}{ u} -\fcAnswerUncover{3}{5}{\frac{1}{u}}\right)\uncover<6->{\alertNoH{6}{u}} }{\left( \alertNoH{3}{u}+ \fcAnswerUncover{3}{5}{\frac{1}{u}}\right)\uncover<6->{\alertNoH{6}{u}}}&=&y}\\
\displaystyle\uncover<7->{\frac{u^2-1}{\alertNoH{8}{u^2+1}}&=&y}\\
\displaystyle\uncover<8->{ \alertNoH{9}{u^2}- \alertNoH{10}{1} &=&\alertNoH{9,10}{ y}( \alertNoH{8}{ \alertNoH{9}{u^2} +\alertNoH{10 }{1}})}\\
\displaystyle\uncover<9->{\alertNoH{9}{ u^2(\alertNoH{11}{1-y})}&=&\alertNoH{10} {1+y}}\\
\displaystyle\uncover<11->{ {\alertNoH{12}{u}}^2&=&\displaystyle \frac{1+y}{ \alertNoH{11}{1-y} }}\\
\displaystyle \uncover<12->{\left(\alertNoH{12}{e^{\alertNoH{13}{x}}} \right)^{\alertNoH{13}{2}}&=&\displaystyle \frac{1+y}{1-y}}\\
\uncover<13->{\displaystyle e^{\alertNoH{13}{2x}}&=&\displaystyle \frac{1+y}{1-y}}\uncover<14->{ &&\text{Take }\alertNoH{14}{\ln}} \\
\uncover<14->{\displaystyle \uncover<handout:0|14,15>{ \alertNoH{15}{2}} x&=&\displaystyle \uncover<16->{ \alertNoH{16}{\frac{1}{2}}} \alertNoH{14}{\ln} \left(\frac{1+y}{1-y} \right)}\\
\end{array}
$
\end{columns}
\vskip -0.3cm
\end{example}
\end{frame}
% end module inverse-function-solve-for

% end lecture

% begin lecture
\lect{February 28, 2014}{Lecture 10}{10}
\section{Implicit Differentiation}
% begin module implicit-differentiation-intro
\begin{frame}
\frametitle{Implicit Differentiation}
\begin{itemize}
\item  So far, we have seen functions with formulas that express one varable explicitly in terms of the other.
\item<2->  $y = \sqrt{x^3+1}$, $y = x\sin x$, etc.
\item<3->  Some functions are given implicitly by a relation between $x$ and $y$.
\item<4->  $x^2 + y^2 = 25$ isn't the equation of any one function.
\item<5->  Implicitly it gives two functions $y = \sqrt{25-x^2}$ and $y = -\sqrt{25-x^2}$.
\item<6->  How do we differentiate this?
\item<7->  Differentiate both sides with respect to $x$, and then solve for $y'$.
\end{itemize}
\begin{center}
\ \uncover<5->{%
\includegraphics[height=2.5cm]{implicit-differentiation/pictures/03-06-circletop.pdf}%
}%
\ \uncover<4->{%
\includegraphics[height=2.5cm]{implicit-differentiation/pictures/03-06-circle.pdf}%
}%
\ \uncover<5->{%
\includegraphics[height=2.5cm]{implicit-differentiation/pictures/03-06-circlebottom.pdf}%
}%
\end{center}
\end{frame}
% end module implicit-differentiation-intro

% begin module implicit-tangent-line
\abovedisplayskip=0pt
\belowdisplayskip=0pt
\abovedisplayshortskip=0pt
\belowdisplayshortskip=0pt
\begin{frame}
\begin{example}
Find an equation of the tangent line to $(x-1)^2 + (y+2)^2 = 25$ at $(-2,2)$.
\begin{columns}
\column{1.2in}
\begin{center}
\psset{xunit=0.25cm, yunit=0.25cm}
\begin{pspicture}(-4.8, -7.5)(6.9,4) 
\psframe*[linecolor=white](-4.8, -7.5)(6.9,4) 
\tiny 
\psaxes[ticks=none, labels=none]{<->}(0,0)(-4.7,-7.5)(6.5,3.8)
\psLabels{6.5}{3.8}
%\psXTickWithLabel{-2.5}{$-2.5$}
\psXTickWithLabel{5}{$5$}
\psYTickWithLabel{-5}{$-5$}
%Function formula: -2- (\sqrt{25- ((-1+x)^{2})}) 
\psplot[linecolor=\psColorGraph, plotpoints=1000]{-4}{6}{x -1 add 2 exp -1 mul 25 add sqrt -1 mul -2 add }
%Function formula: -2+\sqrt{25- ((-1+x)^{2})} 
\psplot[linecolor=\psColorGraph, plotpoints=1000]{-4}{6}{x -1 add 2 exp -1 mul 25 add sqrt -2 add }
\psFullDot{-2}{2}
\rput[br](-1.9, 2.3){$(-2, 2)$}
\uncover<16->{
\psline[linecolor=\psColorTangent](-4.666666667,0)(0.2,3.65)
}
\end{pspicture} %\only<handout:0|-15>{\includegraphics[height=3cm]{implicit-differentiation/pictures/implicit-tangent-linea.jpg}}
%\only<16->{\includegraphics[height=3cm]{implicit-differentiation/pictures/implicit-tangent-lineb.jpg}}
\end{center}
\begin{align*}
\uncover<15->{&\text{Plug in} \ (-2,2):}\\
\uncover<15->{&\frac{\diff y}{\diff x}  = \frac{1+2}{2+2} = \frac{3}{4}}\\
\uncover<16->{&\text{Point-slope form:}\\
&y-2 = \frac{3}{4} (x+2)}
\end{align*}
\column{3in}
\abovedisplayskip=0pt
\belowdisplayskip=0pt
\abovedisplayshortskip=0pt
\belowdisplayshortskip=0pt
\begin{align*}
\uncover<2->{\text{Find} \ \frac{\diff y}{\diff x}, \ \text{given} \ (x-1)^2 \ + (y+2)^2 & = 25:} \\
\uncover<3->{\alert<handout:0|3-4>{\frac{\diff}{\diff x}\left((x-1)^2\right)} + \alert<handout:0|5-6>{\frac{\diff}{\diff x}\left((y+2)^2\right)}   &= \alert<handout:0|7-8>{\frac{\diff}{\diff x}(25)}}\\
\uncover<3->{\uncover<4->{\alert<handout:0|4>{2(x-1)\alert<handout:0|9>{\frac{\diff}{\diff x}(x-1)}}} +\uncover<6->{\alert<handout:0|6>{ 2(y+2)\alert<handout:0|11-12>{\frac{\diff}{\diff x}(y+2) }}}  &= \uncover<8->{\alert<handout:0|8>{0}}}\\
\uncover<9->{2(x-1)\alert<handout:0|9-10> {(\uncover<10->{1})}+ 2(y+2)\alert<handout:0|11-12>{\left( \uncover<12->{\frac{\diff y}{\diff x}} \right)}  & = 0}\\
\uncover<13->{2(y+2)\left( \frac{\diff y}{\diff x} \right) & = 2(1-x)}\\
\uncover<14->{\frac{\diff y}{\diff x} &=  \frac{1-x}{y+2}}
\end{align*}
\end{columns}
\end{example}
\end{frame}
% end module implicit-tangent-line
% begin module implicit-differentiation-ex3
\begin{frame}[t]
\vskip -0.15cm
\begin{example}
\vskip 0.2cm
\begin{columns}[T]
\column{0.25\textwidth}
\psset{xunit=0.23cm, yunit=0.23cm}
\begin{pspicture}(-4.4,-4.4)(4.4,4.4)
\tiny
\fcBoundingBox{-4.4}{-4.4}{4.4}{4.4}
\fcAxesStandardNoFrame{-4.2}{-4.2}{4.2}{4.2}
\fcLabels{4.2}{4.2}
\rput[t](1,-0.4){$1$}
\psline(1,-0.3)(1,0.3)
\fcImplicitIId[linewidth=0.1, linestyle=solid, linecolor=red, showGridImplicitIId=false]{-4}{-4}{500}{500}{0.016}{0.016}{ x y add 114.591559026 mul sin y y mul x 114.591559026 mul cos mul sub}
\end{pspicture}

\column{0.75\textwidth}


Find $y'$ as an expression of $x$ and $y$. 
\end{columns}

\vskip -1.4cm
$
\renewcommand{\arraystretch}{1.5}
\begin{array}{@{\!\!\!}r@{\!\!}c@{\!}l@{}}
\displaystyle \sin (2(x+y)) & =&\displaystyle  y^2\cos (2x).\\
\uncover<2->{\displaystyle \alertNoH{ 3-4}{\frac{\diff}{\diff x} (\sin (2(x+y)))} }%
\uncover<2->{ &  = & \displaystyle 
\alertNoH{ 5-6}{\frac{\diff}{\diff x} (y^2\cos (2x))}%
}\\%
\fcAnswerUncover{3}{4}{%
\displaystyle \alertNoH{ 3-4}{\cos(2(x+y))\alertNoH{ 7-8}{\frac{\diff}{\diff x}(2(x+y))}}%
}%
\uncover<3->{&  = & %
\fcAnswerUncover{3}{6}{%
\displaystyle \alertNoH{ 5-6}{ \alertNoH{ 9-10}{\frac{\diff}{\diff x}(y^2)}\cos (2x) +(y^2)\alertNoH{ 11-12}{\frac{\diff}{\diff x}(\cos (2x)) }  }%
}}\\%
\uncover<7->{\displaystyle \cos(2(x+y))\alertNoH{ 7-8}{\left(\fcAnswerUncover{7}{8}{\alertNoH{13}{2}+\alertNoH{13}{2}y'}\right)}& = &
\displaystyle \alertNoH{ 9-10}{\fcAnswerUncover{7}{10}{2yy'}}\cos (2x) +y^2\alertNoH{ 11-12}{\fcAnswerUncover{7}{12}{(\alertNoH{14}{-}\sin (2x)) \alertNoH{15,16}{\frac{\diff}{\diff x}(2x)}} }%
}\\%
\uncover<13->{\displaystyle \alertNoH{13,17}{2} \alertNoH{18,19}{ \cos(2( x+y ))} \left(\alertNoH{18}{1}+\alertNoH{19}{y'}\right)& = &
\displaystyle \alertNoH{17}{2}yy'\cos (2x) \alertNoH{14}{-}y^2\sin (2x)\fcAnswerUncover{13}{16}{\alertNoH{17}{2}} }%
\\%
\uncover<18->{\displaystyle \alertNoH{18,21}{\cos(2(x+y))} +\alertNoH{19,20}{ y'\cos(2(x+y))}%
&  = & \displaystyle \alertNoH{20}{yy'\cos (2x)}\alertNoH{21}{-y^2\sin (2x)} }\\%
\uncover<20->{%
\displaystyle \alertNoH{20}{\alertNoH{ 22}{y'}\cos(2(x+y)) - y\alertNoH{ 22}{y'}\cos (2x)}%
&  = &%
\displaystyle \alertNoH{21}{ -\cos(2(x+y))-y^2\sin (2x)}%
}\\%
\uncover<22->{ \displaystyle \alertNoH{ 22}{y'}(\alertNoH{23}{\cos(2(x+y)) - y\cos (2x)})%
&  = &\displaystyle  -\cos(2(x+y))-y^2\sin (2x)%
}\\%
\uncover<23->{y'}& \uncover<23->{ = }& \uncover<23->{\displaystyle \frac{ - \cos(2(x+y))-y^2\sin(2 x)}{\alertNoH{23}{\cos (2(x+y)) - y\cos (2x)}}.%
}%
\end{array}
$
\end{example}
\end{frame}
% end module implicit-differentiation-ex3

% begin module implicit-differentiation-ex3
\begin{frame}
\begin{example}[Example 3, p. 213]
\abovedisplayskip=0pt
\belowdisplayskip=-15pt
\abovedisplayshortskip=0pt
\belowdisplayshortskip=0pt
\begin{align*}
\text{Find $y'$ if } \tan xy &= x^2-y^2.\\
\uncover<2->{\alert<handout:0|3-4>{\frac{\diff}{\diff x}(\tan xy)}&= \alert<handout:0|5-6>{ \frac{\diff}{\diff x}(x^2-y^2)}}\\
\uncover<3->{\alert<handout:0|4>{\uncover<4->{(\sec^2xy)\alert<handout:0|7-8>{\frac{\diff}{\diff x}(xy)}}} & = \uncover<6->{\alert<handout:0|6>{ 2x-2yy'}}}\\
\uncover<7->{\alert<handout:0|7-8>{\left(\uncover<8->{y\alert<handout:0|9-10>{\frac{\diff}{\diff x}(x)}+x\alert<handout:0|11-12>{\frac{\diff}{\diff x}(y)}}\right)}\sec^2 xy &= 2x -2yy'}\\
\uncover<9->{\left(y\alert<handout:0|9-10>{(\uncover<10->{1})}+x\alert<handout:0|11-12>{(\uncover<12->{y'})}\right)\sec^2 xy &= 2x -2yy'}\\
\uncover<13->{y\sec^2xy+xy' \sec^2xy &= 2x -2yy'}\\
\uncover<14->{xy' \sec^2xy+2yy'&=2x-y\sec^2xy}\\
\uncover<15->{y'(x\sec^2xy+2y) &= 2x-y\sec^2xy}\\
\uncover<16->{y'&=\frac{2x-y\sec^2xy}{x\sec^2xy+2y}.}
\end{align*}
\end{example}
\end{frame}
% end module implicit-differentiation-ex3

% begin module implicit-differentiation-ex4
\begin{frame}
\begin{example}
%\uncover<2->{Differentiate implicitly:}%
\abovedisplayskip=0pt
\belowdisplayskip=0pt
\abovedisplayshortskip=0pt
\belowdisplayshortskip=0pt
\begin{align*}
\text{Let ~~} \alertNoH{ 15}{x^4 + y^4} & \alertNoH{ 15}{ = 16}. \text{ Find }y''. \\
\uncover<2->{%
\alertNoH{3}{4x^3} + \alertNoH{3}{4y^3}y'%
}%
& \uncover<2->{ = } %
\uncover<2->{%
0%
}\\%
\uncover<3->{%
\alertNoH{ 11}{y'}%
}%
& \uncover<3->{ \alertNoH{ 11}{=} } %
\uncover<3->{%
\alertNoH{3,4, 11}{-\frac{x^3}{y^3}}.%
}\\%
\uncover<4->{%
y''%
}%
& \uncover<4->{ = } %
\uncover<4->{%
\frac{\diff}{\diff x}\left(\alertNoH{4}{ -\frac{\alertNoH{5}{x^3} }{\alertNoH{6}{y^3}} } \right)%
}  \uncover<5->{ = } %
\uncover<5->{%
- \frac{\alertNoH{6}{y^3} \alertNoH{ 7-8}{\frac{\diff}{\diff x}\left(\alertNoH{5}{x^3}\right)} - \alertNoH{5}{x^3} \alertNoH{ 9-10}{\frac{\diff}{\diff x}\left(\alertNoH{6}{y^3} \right)}}{\left( \alertNoH{6}{ y^3 }\right)^2}%
}\\%
& \uncover<7->{ = } %
\uncover<7->{%
- \frac{y^3 (\alertNoH{ 7-8}{\fcAnswer{8}{3x^2}}) - x^3( \alertNoH{ 9-10}{\fcAnswerUncover{7}{10}{3y^2\alertNoH{ 11}{y'}}})}{y^6}%
} \uncover<11->{ = }%
\uncover<11->{%
- \frac{3x^2y^3  - 3x^3y^2\alertNoH{ 11}{\left( -\frac{x^3}{y^3}\right)}}{y^6}%
}\\%
& \uncover<12->{ = } %
\uncover<12->{%
- \frac{3x^2(y^3+\frac{x^4}{y})}{y^6}%
} \uncover<13->{ = }%
\uncover<13->{%
- \frac{3x^2\left( \frac{y^4+x^4}{\alertNoH{ 14}{y}}\right)}{\alertNoH{ 14}{y^6}}%
}\\%
& \uncover<14->{ = } %
\uncover<14->{%
- \frac{3x^2(\alertNoH{ 15}{y^4+x^4})}{\alertNoH{ 14}{y^7}}%
}  \uncover<15->{ = }%
\uncover<15->{%
- \frac{3x^2(\alertNoH{ 15}{16})}{y^7}%
} \uncover<16->{ = }%
\uncover<16->{%
-48 \frac{x^2}{y^7}.%
}%
\end{align*}
\end{example}
\end{frame}
% end module implicit-differentiation-ex4

%\input{../../modules/implicit-differentiation/implicit-second-derivative}
% end lecture

% begin lecture
\lect{March 7, 2014}{Lecture 12}{12}
\section{Derivatives of Logarithmic and Exponential Functions}
\subsection{Derivatives of Logarithmic Functions}
% begin module general-log-derivative-implicit
\begin{frame}
\frametitle{Derivatives of Logarithmic Functions}
\begin{theorem}[The Derivative of $\log_a x$]
\[
\frac{\diff}{\diff x} (\log_a x) = \frac{1}{x\ln a} .
\]
\end{theorem}
\begin{proof}
\abovedisplayskip=0pt
\belowdisplayskip=0pt
\abovedisplayshortskip=0pt
\belowdisplayshortskip=0pt
\begin{align*}
\uncover<2->{%
\text{Let } y %
}%
& \uncover<2->{%
 = \log_a x. %
}\\%
\uncover<3->{%
\text{Then } \alertNoH{ 4-5,9}{a^y} %
}%
& \uncover<3->{%
 \alertNoH{ 9}{=} \alertNoH{ 6-7,9}{x}. %
}\\%
\uncover<4->{%
\text{Differentiate implicitly:}\quad \alertNoH{ 4-5}{\fcAnswer{5}{a^y (\ln a) y'}} %
}%
& \uncover<4->{%
 = \alertNoH{ 6-7}{\fcAnswerUncover{4}{7}{1}} %
}\\%
\uncover<8->{%
y' %
}%
& \uncover<8->{%
 = \frac{1}{\alertNoH{ 9}{a^y}\ln a} %
}\\%
& \uncover<9->{%
 = \frac{1}{\alertNoH{ 9}{x}\ln a}. %
}%
\end{align*}
\end{proof}
\end{frame}
% end module general-log-derivative-implicit

% begin module general-log-derivative-ex.tex
\begin{frame}
\chainrulefofx{\log_3(5^{x}+1)}{5^x+1}{\log_3 x}{\frac{1}{ UU \ln 3}}{5^x \ln 5 }{\frac{5^x \ln 5}{(UU) \ln 3}}{0}
\end{frame}
% end module general-log-derivative-ex.tex
% begin module natural-log-derivative-from-general
\begin{frame}
\begin{theorem}[The Derivative of $\log_a x$]
\[
\frac{\diff}{\diff x} (\log_a x) = \frac{1}{x\ln a} .
\]
\end{theorem}

$\ln x = \log_e x$, so plug in $a = e$ to find the derivative of $\ln x$.
\begin{align*}
\frac{\diff}{\diff x}(\ln x) & = \frac{1}{x\alert<handout:0| 2-3>{\ln e}} \\
 & \uncover<2->{= \frac{1}{x\alert<handout:0| 2-3>{(\uncover<3->{1})}}}\\
 & \uncover<4->{= \frac{1}{x}.}
\end{align*}
\uncover<5->{%
\begin{theorem}[The Derivative of $\ln x$]
\[
\frac{\diff}{\diff x} (\ln x) = \frac{1}{x}.
\]
\end{theorem}
}%
\end{frame}
% end module natural-log-derivative-from-general

% begin module natural-log-derivative-ex-simplify.tex
\begin{frame}
\begin{example}[Chain Rule, Natural Logarithm]
\abovedisplayskip=0pt
\belowdisplayskip=0pt
\abovedisplayshortskip=0pt
\belowdisplayshortskip=0pt
\begin{align*}
\text{Differentiate}\quad y &= \ln (e^x \sec x)\\
\uncover<2->{ &= \alert<handout:0|3-4>{\ln e^x} + \ln \sec x}\\
\uncover<3->{ &=\uncover<4->{\alert<handout:0|4>{\alert<handout:0|5-6>{ x}}} + \alert<handout:0|7-8>{\ln \sec x.}}\\
\uncover<5->{\frac{\diff y}{\diff x} & =\alert<handout:0|6>{\uncover<6->{1} }+ \uncover<8->{ \alert<handout:0|8>{\alert<handout:0|12>{\frac{\diff}{\diff x} (\ln \sec x)}}}}\\
\uncover<9->{\text{Let} \ \ \alert<handout:0|15-18>{u} &  \alert<handout:0|15-18>{= \uncover<10->{ \sec x}}}\\
\uncover<11->{\text{Then} \ \ \ln \sec x & = { \ln u}.}\\
\uncover<12->{\text{Chain Rule:} \quad \frac{\diff y}{\diff x} & = 1+  \alert<handout:0|12>{\alert<handout:0|13-14>{\frac{\diff}{\diff u}(\ln u)} \alert<handout:0|15-16>{\frac{\diff u}{\diff x}}}}\\
\uncover<13->{& = 1+ \uncover<14->{\alert<handout:0|14>{\frac{1}{\alert<handout:0|17-18>{u}}}}\uncover<16->{\alert<handout:0|16>{\sec x \tan x}}}\\
\uncover<17->{& = 1+ \frac{1}{\alert<handout:0|17-18>{(\uncover<18->{\sec x})}}\sec x \tan x} \\
\uncover<19->{& = 1+ \tan x}
\end{align*}
\end{example}
\end{frame}
% begin module natural-log-derivative-ex-simplify.tex
% begin module natural-logarithm-derivative-ex7
\begin{frame}
\begin{example}
\abovedisplayskip=0pt
\belowdisplayskip=0pt
\abovedisplayshortskip=0pt
\belowdisplayshortskip=0pt
\begin{align*}
\text{Find $f'(x)$ if } f(x) & = \ln |x|.\\
\uncover<2->{%
f(x) %
}% 
& \uncover<2->{
 = \left\{ \begin{array}{lr}
\alert<handout:0| 3-4>{\ln x} & \text{ if } x > 0 \\
\alert<handout:0| 5-6>{\ln (-x)} & \text{ if } x < 0 \\
\end{array}\right. .
}\\%
\uncover<3->{%
f'(x) %
}%
& \uncover<3->{
 = \left\{ \begin{array}{lr}
\alert<handout:0| 3-4>{\uncover<4->{\frac{1}{x}}} & \text{ if } x > 0 \\
\alert<handout:0| 5-6>{\uncover<6->{\frac{1}{-x}(-1)}} & \text{ if } x < 0 \\
\end{array}\right.
}\\%
& \uncover<7->{
 = \left\{ \begin{array}{lr}
\frac{1}{x} & \text{ if } x > 0 \\
\frac{1}{x} & \text{ if } x < 0 \\
\end{array}\right.
}\\%
& \uncover<8->{%
 = \frac{1}{x} \text{ if } x \neq 0.
}%
\end{align*}
\end{example}
\end{frame}
% end module natural-logarithm-derivative-ex7

\subsection{Derivatives of General Exponential Functions}
% begin module chain-rule-general-exponential.tex
\begin{frame}
\begin{example}[Chain Rule, general exponential function]
\begin{align*}
\text{Differentiate}\quad y & = \alert<handout:0|2-3>{2}^x.\\
\uncover<2->{y} & \uncover<2->{= \alert<handout:0|2-3>{( e^{\uncover<3->{\ln 2}} )}^x}\\
\uncover<4->{y} & \uncover<4->{= e^{x\ln 2}.}\\
\uncover<5->{\text{Let} \quad \alert<handout:0|5-6>{\alert<handout:0|13-14>{\alert<handout:0|11-12>{u}}} &\alert<handout:0|5-6>{\alert<handout:0|11-14>{=}} \uncover<6->{\alert<handout:0|6>{\alert<handout:0|11-12>{\alert<handout:0|13-14>{ x\ln 2.}}}}} \\
\uncover<7->{\text{Then} \quad \alert<handout:0|9-10>{y} &\alert<handout:0|9-10>{= e^u.}} \\
\uncover<8->{\text{Chain Rule:}\quad \frac{\diff y}{\diff x} &= \alert<handout:0|9-10>{ \frac{\diff y}{\diff u}}\alert<handout:0|11-12>{ \frac{\diff u}{\diff x}}}\\
\uncover<9->{&= \alert<handout:0|9-10>{(\uncover<10->{e^{\alert<handout:0|13-14>{u}}})}\alert<handout:0|11-12>{(\uncover<12->{\ln 2})} }\\
\uncover<13->{ &= (e^{\alert<handout:0|13-14>{(\uncover<14->{x\ln2})}})(\ln 2)}\\
\uncover<15->{& = (e^{\ln 2})^x(\ln 2)}\\
\uncover<16->{& = (2^x)(\ln 2).} 
\end{align*}
\end{example}
\end{frame}
% begin module chain-rule-general-exponential.tex
% begin module general-exponential-derivative
\begin{frame}
\begin{theorem}[The Derivative of $a^x$]
\[
\frac{\diff}{\diff x} (a^x) = a^x \ln a .
\]
\end{theorem}
\end{frame}
% end module general-exponential-derivative

% begin module chain-rule-twice-ex1
\begin{frame}
\chainruletwice%
{\sin\sqrt{10^x+1}}%
{\cos\sqrt{10^x+1}}%
{\sqrt{10^x+1}}%
{\frac{1}{2\sqrt{10^x+1}}}%
{10^x+1}%
{10^x\ln 10}%
{}%
{\frac{(\ln 10)10^x\cos \sqrt{10^x+1}}{2\sqrt{10^x+1}}}%
{}%
\end{frame}
% end module chain-rule-twice-ex1

\subsection{Logarithmic Differentiation}
% begin module logarithmic-differentiation-ex.tex
\begin{frame}
\begin{example}[Logarithmic Differentiation]
\abovedisplayskip=0pt
\belowdisplayskip=0pt
\abovedisplayshortskip=0pt
\belowdisplayshortskip=0pt

\begin{align*}
\text{Differentiate} \quad \alertNoH{14}{y} &\alertNoH{14}{= \frac{(x-1)^{5/3} \sin^3 x}{\sqrt{e^x + 1}}}.\\
\intertext{\uncover<2->{Take the natural logarithm of both sides:}}
\uncover<2->{\ln y &= \ln  \frac{(x-1)^{5/3} \sin^3 x}{\sqrt{e^x + 1}} }\\
\uncover<3->{ \ln y &=(5/3) \ln(x-1) +3\ln \left( \sin x\right)-(1/2)\ln\left(e^x + 1\right).}\\
\uncover<4->{\alertNoH{4-5}{\frac {\diff }{\diff x} (\ln y)} &= \alertNoH{6-11}{ \frac{\diff}{\diff x}} \left( \alertNoH{6-7}{ (5/3) \ln(x-1)} +\alertNoH{8-9}{3\ln \left( \sin x\right)}-\alertNoH{10-11}{(1/2)\ln\left(e^x + 1\right)}\right)}\\
\uncover<4->{ \alertNoH{4-5}{ \fcAnswerUncover{4}{5}{ \frac{1}{ \alertNoH{12}{y} } \left(\frac {\diff y}{\diff x}\right)}} &= \alertNoH{6-7}{ \left(\fcAnswerUncover{4}{7}{ \frac{5}{3}\left(\frac{1}{x-1} \right)}  \right) }+ \alertNoH{8-9}{ \left( \fcAnswerUncover{4}{9}{ \frac{3\alertNoH{13}{\cos x}}{\alertNoH{13}{\sin x}}} \right) }- \alertNoH{10-11}{\left(\fcAnswerUncover{4}{11}{ \frac{1}{2} \left(\frac{e^x}{e^x+1}  \right)} \right)}}\\
 \uncover<12->{\frac {\diff y}{\diff x} &=  \left( \frac{5}{3(x-1)}  + 3\alertNoH{13}{\cot x} -\frac{e^x}{2(e^x+1)} \right)\alertNoH{12,14}{y}}\\
 \uncover<14->{& = \left( \frac{5}{3(x-1)}  + 3\cot x -\frac{e^x}{2(e^x+1)} \right) \alertNoH{14}{\frac{(x-1)^{5/3} \sin^3 x}{\sqrt{e^x + 1}}}.}
\end{align*}

\end{example}
\end{frame}
% begin module logarithmic-differentiation-ex.tex

% begin module logarithmic-differentiation
\begin{frame}
Steps in Logarithmic Differentiation
\begin{enumerate}
\item  Take natural logarithms of both sides of an equation $y = f(x)$.
\item  Use the properties of logarithms to simplify.
\item  Differentiate implicitly with respect to $x$.
\item  Solve the resulting equation for $y'$.
\end{enumerate}
Note: If $f(x) < 0$, then we use $\ln |f(x)|$ instead as $\ln (f(x))$ is not defined. We computed the derivative of $\ln |f(x)|$ in the previous lecture.
\end{frame}
% end module logarithmic-differentiation

% begin module logarithmic-differentiation-ex-base-and-power1
\begin{frame}
\logdiffbaseandexp{(3x+1)}{\ln x}{\frac{1}{3x+1}\cdot 3}{\frac{1}{x}}{\frac{3\ln x}{3x+1} + \frac{\ln (3x+1)}{x}}
\end{frame}
% end module logarithmic-differentiation-ex-base-and-power1

% begin module logarithmic-differentiation-ex-base-and-power2
\begin{frame}
\logdiffbaseandexp{x}{\tan x}{\frac{1}{x}}{\sec^2 x}{\frac{\tan x}{x} + (\ln x)\sec^2 x}
\end{frame}
% end module logarithmic-differentiation-ex-base-and-power2

\subsection{The Number $e$ as a Limit}
% begin module e-limit
\begin{frame}
\begin{theorem}[The Number $e$ as a Limit]
\[
e = \lim_{x\rightarrow 0} (1 + x)^{\frac{1}{x}} = \lim_{y\to \infty} \left(1+\frac{1}{y}\right)^y.
\]
\end{theorem}
\begin{proof}
\uncover<2->{Let $f(x) = \ln x$.  }%
\uncover<3->{Then $f'(x) = \frac{1}{x}$, so $f'(1) = 1$.}%
\abovedisplayskip=0pt
\belowdisplayskip=0pt
\abovedisplayshortskip=0pt
\belowdisplayshortskip=0pt
\begin{align*}
\uncover<4->{\alertNoH{ 10}{1} = f'(1)} & \uncover<4->{=} %
\uncover<4->{\lim_{h\rightarrow 0}\frac{f(1+h)-f(1)}{h}}%
\uncover<5->{ = \lim_{x\rightarrow 0}\frac{f(1+x)-f(1)}{x}}\\%
& \uncover<6->{=}  %
\uncover<6->{\lim_{x\rightarrow 0}\frac{\ln (1+x)-\ln (1)}{x}}%
\uncover<7->{ = \lim_{x\rightarrow 0}\frac{1}{x}\ln (1 + x)}\\%
& \uncover<8->{\alertNoH{ 10}{=}}  %
\uncover<8->{\alertNoH{ 10}{\lim_{x\rightarrow 0}\ln (1+x)^{\frac{1}{x}}}.}
\end{align*}
\uncover<9->{Then use the fact that \alertNoH{ 11}{the exponential function is continuous}:}
\[
\uncover<9->{e = e^{\alertNoH{ 10}{1}} =}%
\uncover<10->{\alertNoH{ 11}{e^{\alertNoH{ 10}{\lim\limits_{x\rightarrow 0}\ln (1+x)^{\frac{1}{x}}}} =}}%
\uncover<11->{\alertNoH{ 11}{\lim\limits_{x\rightarrow 0}e^{\ln (1+x)^{\frac{1}{x}}}} =}%
\uncover<12->{\lim\limits_{x\rightarrow 0} (1+x)^{\frac{1}{x}}.}\qedhere
\]
\end{proof}
\end{frame}
% end module e-limit

% end lecture

% begin lecture
\lect{March 14, 2014}{Lecture 14}{14}
\section{Inverse Trigonometric Functions}
% begin module arcsin-def
\begin{frame}
\frametitle{Inverse Trigonometric Functions}
%\ \only<handout:0| -1>{%
%\includegraphics[width=12cm]{inverse-trig/pictures/07-06-arcsina.pdf}%
%}%
%\only<handout:0| 2>{%
%\includegraphics[width=12cm]{inverse-trig/pictures/07-06-arcsinb.pdf}%
%}%
%\only<3>{%
%\includegraphics[width=12cm]{inverse-trig/pictures/07-06-arcsinc.pdf}%
%}%
%\only<handout:0| 4->{%
%\includegraphics[width=12cm]{inverse-trig/pictures/07-06-arcsind.pdf}%
%}%
\psset{xunit=0.6cm,yunit=0.6cm}
\begin{pspicture}(-5,-1.4)(10,1.4)
\tiny
\psaxes[labels=none, Dx=1.570796327, Dy=1] {<->}(0,0)(-4,-1.8)(10,1.8)

\uncover<1-2>{\psplot[linecolor=red, plotpoints=1000]{-4}{10}{x 57.295779513 mul sin}}
\uncover<2>{\psline(-4,0.6)(10,0.6 )}

\uncover<3>{\psplot[linecolor=red, plotpoints=1000]{-1.570796327}{1.570796327}{x 57.295779513 mul sin}
\rput[bl](3, 1){\alert<3>{$y=\sin x, -\frac{\pi}{2}\leq x\leq \frac{\pi}{2}$} }
}
\uncover<4->{\psplot[linecolor=gray, plotpoints=1000]{-1.570796327}{1.570796327}{x 57.295779513 mul sin}
\rput[bl](3, 1){\color{gray}{$y=\sin x, -\frac{\pi}{2}\leq x\leq \frac{\pi}{2}$} }
}

\uncover<4->{\psplot[linecolor=red, plotpoints=1000]{-1}{1}{x ASIN}
\rput[r](-1.5, -1){\alert<4>{$y=\Arcsin x$} }

}

\rput[t](-3.14, -0.3){$-\pi$}
\rput[t](-1.57, -0.3){$-\frac{\pi}{2}$}
\rput[t](1.57, -0.3){$\frac{\pi}{2}$}
\rput[t](3.14, -0.3){$\pi$}
\rput[t](4.71, -0.3){$\frac{3\pi}{2}$}
\rput[t](6.28, -0.3){$2\pi$}
\rput[t](7.85, -0.3){$\frac{5\pi}{2}$}
\rput[t](9.42, -0.3){$3\pi$}
\rput[bl](0.2,1){\tiny $1$}
\end{pspicture}
\begin{columns}[c]
\column{.65\textwidth}
\begin{itemize}
\item<2->  $\sin x$ isn't one-to-one.
\item<3->  It is if we restrict the domain to $[-\pi /2, \pi /2]$.
\item<4->  Then it has an inverse function.
\item<4->  We call it $\arcsin$ or $\sin^{-1}$.
\item<6->  $\Arcsin x = y \Leftrightarrow \sin y = x$ and $-\pi /2 \leq y \leq \pi /2$.
\end{itemize}
\column{.35\textwidth}
\psset{xunit=1cm,yunit=1cm}
\uncover<5->{
\begin{pspicture}(-5,-1.4)(10,1.4)
\tiny
\psaxes[ticks=none, labels=none]{<->}(0,0)(-1.5,-2)(1.5,2)
\psLabels{1.5}{2}
\psLabelXOne
\psline(-1, -0.1)(-1,0.1)
\rput[t](-1,  -0.1){$-1$}

\psline(-0.1, 1.570796327)(0.1,1.570796327)
\rput[r](-0.1,  1.570796327){$\frac{\pi}{2}$}
\psline(-0.1, -1.570796327)(0.1,-1.570796327)
\rput[r](-0.1,  -1.570796327){$-\frac{\pi}{2}$}

\psplot[linecolor=red, plotpoints=1000]{-1}{1}{x ASIN}
\rput[rb](-0.05, 0.2){\alert<4>{$y=\Arcsin x$} }
\psFullDot{1}{1.570796327}
\psFullDot{-1}{-1.570796327}
\end{pspicture}
}
%\uncover<5->{%
%\includegraphics[height=4cm]{inverse-trig/pictures/07-06-arcsine.pdf}%
%}%

\end{columns}
\end{frame}
% end module arcsin-def

% begin module arcsin-ex1
\begin{frame}
\vskip -0.1cm
\begin{example}
\begin{columns}[t]
\column{.3\textwidth}
\hfil \hfil Find $\displaystyle \Arcsin \left( \frac{1}{2}\right)$.
\begin{itemize}
\item<2->  $\displaystyle \sin \left(\uncover<2-| handout:0>{\frac{\pi}{ 6}}\right) = \frac{1}{2}$.
\item<3->  $\displaystyle -\frac{\pi}{2} \leq \uncover<3-| handout:0>{\frac{\pi}{6}} \leq \frac{\pi}{2}$.
\item<4->  Therefore $ \Arcsin \left( \frac{1}{2}\right) = \uncover<3-| handout:0>{\frac{\pi}{6}}$.
\end{itemize}
\column{.7\textwidth}
\hfil \hfil Find $\displaystyle
\tan \left( \Arcsin \left( \frac{1}{3}\right) \right)
$.
\begin{itemize}
\item<5->  Let $\theta = \Arcsin \left(\frac{1}{3}\right)$, so $\sin \theta = \frac{ \alertNoH{6}{1}}{\alertNoH{7}{3}}$.
\item<6->  Draw a right triangle with  \alertNoH{6}{opposite side $1$} and \alertNoH{7}{hypotenuse $3$}.
\item<8-> Let the angle $\theta$ be as labeled. \uncover<9->{ Then $ \alertNoH{9,10}{\sin\theta= \frac{1 }{3}}$ \uncover<10->{and so $\alertNoH{10}{ \theta= \arcsin\left(\frac{1}{3}\right)}$}.}
\item<11->  \alertNoH{11-12}{Length of adjacent side $ =\worksheet{ \fcAnswer{12}{\sqrt{3^2-1^2}} \uncover<13->{ = \sqrt{8} = 2\sqrt{2}.}}$}
\item<14->  Then \alertNoH{ 14-15}{$\tan \left(\Arcsin \left(\frac{1}{3}\right)\right) = \fcAnswer{15}{ \frac{1}{ 2\sqrt{2}}.}$}
\end{itemize}

\vskip -0.1cm
\hfil\hfil \psset{xunit=1.5cm, yunit=1.5cm}
\begin{pspicture}(-0.2, -0.35)(3.5,1.05)
\small%
\fcBoundingBox{-0.2}{-0.35}{ 10 sqrt 0.3 add}{1.15}
\psline[linecolor=red!1](2.828427125, 1)(2.828427125, 1.01)
\psline[linecolor=red!1](0, -0.35)(0.001, -0.35)
\psdot[linecolor=white](3.1, 1.0)
\psdot[linecolor=white](-0.1, -0.3)
\uncover<6->{%
\psline(0,0)(! 10 sqrt 0)(! 10 sqrt  1)(0,0)
\psline(! 10 sqrt 0.1)(! 10 sqrt 0.1)(! 10 sqrt 0)
\rput[b](1.41, 0.55){$3$}
\rput[l](! 10 sqrt 0.05 add 0.5){$1$}
\rput(0.8, 0.13){$ \alertNoH{8}{\theta}$}
\fcAngle{0}{0.339837}{0.4}{}
}
\uncover<handout:0|6,9,15>{\psline[linewidth=2pt, linecolor=blue](! 10 sqrt 0)(! 10 sqrt  1)}%
\uncover<handout:0|15>{\psline[linewidth=2pt, linecolor=green](0,0)(! 10 sqrt 0)}%
\uncover<handout:0|7,9>{\psline[linewidth=2pt, linecolor=orange](! 0 0)(! 10 sqrt  1)}%
\uncover<11-| handout:0>{\rput[t](! 10 sqrt 2 div -0.1){$ \fcAnswerUncover{11}{13}{ 2 \sqrt{2}}$}}
\end{pspicture}
\end{columns}

\end{example}
\end{frame}
% end module arcsin-ex1

% begin module arcsin-derivative
\begin{frame}
\begin{theorem}[The Derivative of $\Arcsin x$]
\[
\frac{\diff}{\diff x} \left( \Arcsin x\right) = \frac{1}{\sqrt{1-x^2}}, \qquad -1 < x < 1.
\]
\end{theorem}
\begin{proof}
\abovedisplayskip=0pt
\belowdisplayskip=-15pt
\abovedisplayshortskip=0pt
\belowdisplayshortskip=0pt
\begin{align*}
\uncover<2->{%
\text{Let}\quad y %
}%
& \uncover<2->{%
 = \Arcsin x.
}\\%
\uncover<2->{%
\text{Then}\quad \alert<handout:0| 3-4,10>{\sin y} %
}%
& \uncover<2->{%
 \alert<handout:0| 10>{=}  \alert<handout:0| 5-6,10>{x}\quad \text{and} \quad \alert<handout:0| 9>{-\pi/2 \leq y \leq \pi/2}.
}\\%
\uncover<3->{%
\text{Differentiate implicitly:}\quad \alert<handout:0| 3-4>{\uncover<4->{\cos y \cdot y'}} %
}%
& \uncover<3->{%
 = \uncover<6->{\alert<handout:0| 6>{1}} 
}\\%
\uncover<7->{%
y' %
}%
& \uncover<7->{%
 = \frac{1}{\alert<handout:0| 8>{\cos y}} %
}\\%
& \uncover<8->{%
 = \frac{1}{\alert<handout:0| 8>{\alert<handout:0| 9>{\pm}\sqrt{1-\sin^2y}}} %
}\\%
\uncover<9->{%
\text{But \alert<handout:0| 9>{$\cos y > 0$}:}\quad %
}%
& \uncover<9->{%
 = \frac{1}{\sqrt{1-\alert<handout:0| 10>{\sin^2y}}} %
}%
\uncover<10->{%
 = \frac{1}{\sqrt{1-\alert<handout:0| 10>{x^2}}}. \qedhere%
}%
\end{align*}
\end{proof}
\end{frame}
% end module arcsin-derivative

% begin module arcsin-properties
\begin{frame}
Important facts about $\Arcsin$:
\begin{columns}[c]
\column{.5\textwidth}
\psset{xunit=2cm,yunit=2cm}
\begin{pspicture}(-1.5,-2)(1.6,2.1)
\tiny
\psaxes[ticks=none, labels=none]{<->}(0,0)(-1.5,-2)(1.5,2)
\psLabels{1.5}{2}
\psLabelXOne
\psline(-1, -0.1)(-1,0.1)
\rput[t](-1,  -0.1){$-1$}

\psline(-0.1, 1.570796327)(0.1,1.570796327)
\rput[r](-0.1,  1.570796327){$\frac{\pi}{2}$}
\psline(-0.1, -1.570796327)(0.1,-1.570796327)
\rput[r](-0.1,  -1.570796327){$-\frac{\pi}{2}$}

\psplot[linecolor=red, plotpoints=1000]{-1}{1}{x ASIN}
\rput[rb](-0.05, 0.2){$y=\Arcsin x$} 
\psFullDot{1}{1.570796327}
\psFullDot{-1}{-1.570796327}
\uncover<3| handout:0>{\psline[linecolor=red, linewidth=2pt]{<->}(-1,0)(1,0) }
\uncover<5| handout:0>{\psline[linecolor=red, linewidth=2pt]{<->}(0,-1.570796327)(0,1.570796327) }

\end{pspicture}
\column{.5\textwidth}
\begin{enumerate}
\item  \alert<handout:0| 2-3>{Domain: \uncover<3-| handout:0>{$[-1, 1]$.}}
\item  \alert<handout:0| 4-5>{Range: \uncover<5-| handout:0>{$[-\pi /2, \pi /2]$.}}
\item  $\Arcsin x = y \Leftrightarrow \sin y = x$ and $-\pi /2 \leq y \leq \pi /2$.
\item  $\Arcsin (\sin x) = x$ for $-\pi /2 \leq x \leq \pi /2$.
\item  $\sin (\Arcsin x) = x$ for $-1 \leq x \leq 1$.
\item  $\frac{\diff}{\diff x} (\Arcsin x) = \frac{1}{\sqrt{1-x^2}}$.
\end{enumerate}
\end{columns}
\end{frame}
% end module arcsin-properties


% begin module arccos-def
\begin{frame}
\ \only<handout:0| -1>{%
\includegraphics[width=12cm]{inverse-trig/pictures/07-06-arccosa.pdf}%
}%
\only<2>{%
\includegraphics[width=12cm]{inverse-trig/pictures/07-06-arccosb.pdf}%
}%
\only<handout:0| 3->{%
\includegraphics[width=12cm]{inverse-trig/pictures/07-06-arccosc.pdf}%
}%
\begin{columns}[c]
\column{.65\textwidth}
\begin{itemize}
\item<1->  Same for $\cos x$.
\item<2->  Restrict the domain to $[0, \pi ]$.
\item<3->  The inverse is called $\cos^{-1}$ or $\arccos$.
\item<5->  $\cos^{-1} (x) = y \Leftrightarrow \cos y = x$ and $0 \leq y \leq \pi$.
\end{itemize}
\column{.35\textwidth}
\uncover<4->{%
\includegraphics[height=4cm]{inverse-trig/pictures/07-06-arccosd.pdf}%
}%
\end{columns}
\end{frame}
% end module arccos-def

% begin module arccos-properties
\begin{frame}
Important facts about $\Arccos$:
\begin{columns}[c]
\column{.5\textwidth}
\psset{xunit=1.2cm,yunit=1.2cm}
\begin{pspicture}(-1.9,-0.5)(1.9,3.8)
\tiny
\psaxes[labels=none, ticks=none] {<->}(0,0)(-1.8,-0.5)(1.8,3.7)
\psLabels{1.8}{3.7}
\psline(1,-0.1)(1,0.1)
\psplot[linecolor=red, plotpoints=1000]{-1}{1}{x ACOS}
\psline(-0.1,3.141592654)(0.1,3.141592654)
\rput[l](0.15,3.141592654){$\pi$}
\psline(-1,-0.1)(-1,0.1)
\rput[t](-1,-0.1){$-1$}
\rput[t](1,-0.1){$1$}
\rput[r](-0.6, 2){$y=\Arccos x$}
\uncover<handout:0| 3>{
\psline[arrows=<->, linecolor=red, linewidth=3pt](-1,0)(1,0)
} 
\uncover<handout:0| 5>{
\psline[arrows=<->, linecolor=red, linewidth=3pt](0,0)(0,3.141592654)
} 
\end{pspicture}

%\ \includegraphics[height=6cm]{inverse-trig/pictures/07-06-arccosd.pdf}%
\column{.5\textwidth}
\begin{enumerate}
\item  \alert<handout:0| 2-3>{Domain: \uncover<3-| handout:0>{$[-1, 1]$.}}
\item  \alert<handout:0| 4-5>{Range: \uncover<5-| handout:0>{$[0, \pi ]$.}}
\item  $\Arccos x = y \Leftrightarrow \cos y = x$ and $0 \leq y \leq \pi$.
\item  $\Arccos (\cos x) = x$ for $0 \leq x \leq \pi$.
\item  $\cos (\Arccos x) = x$ for $-1 \leq x \leq 1$.
\item  $\frac{\diff}{\diff x} (\Arccos x) = -\frac{1}{\sqrt{1-x^2}}$.  \uncover<6->{(The proof is similar to the proof of the formula for the derivative of $\Arcsin x$.)}
\end{enumerate}
\end{columns}
\end{frame}
% end module arccos-properties

% begin module arctan-def
\begin{frame}
\begin{columns}[c]
\column{.5\textwidth}
\ \only<handout:0| -1>{%
\includegraphics[width=5cm]{inverse-trig/pictures/07-06-arctana.pdf}%
}%
\only<handout:1| 2>{%
\includegraphics[width=5cm]{inverse-trig/pictures/07-06-arctanb.pdf}%
}%
\only<handout:2| 3->{%
\includegraphics[width=5cm]{inverse-trig/pictures/07-06-arctanc.pdf}%
}%
\column{.5\textwidth}
\begin{itemize}
\item<1->  $\tan x$ isn't one-to-one.
\item<2->  Restrict the domain to $(-\pi /2, \pi /2)$.
\item<3->  The inverse is called $\Arctan$ or $\arctan$.
\item<4->  $\Arctan x = y \Leftrightarrow \tan y = x$ and $-\pi /2 < y < \pi /2$.
\item<5->  \alert<handout:0| 5-6>{Domain of $\Arctan$: \uncover<6-| handout:0>{$(-\infty,\infty)$.}}
\item<5->  \alert<handout:0| 7-8>{Range of $\Arctan$: \uncover<8-| handout:0>{$(-\pi / 2, \pi / 2)$.}}
\item<9->  \alert<handout:0| 9-10>{$\displaystyle \lim_{x\rightarrow \infty} \Arctan x = \uncover<10-| handout:0>{\pi / 2.}$}
\item<9->  \alert<handout:0| 11-12>{$\displaystyle \lim_{x\rightarrow - \infty} \Arctan x = \uncover<12-| handout:0>{- \pi / 2.}$}
\end{itemize}
\end{columns}
\end{frame}
% end module arctan-def

% begin module arctan-ex3
\begin{frame}
\begin{example} %[Example 3, p. 218]
Simplify the expression $\cos (\Arctan x)$.
\begin{itemize}
\item<2->  Let $y = \Arctan x$, so $\tan y = x$.
\item<3->  Draw a right triangle with opposite $x$ and adjacent $1$.
\item<4->  \alert<handout:0| 4-5>{Length of hypotenuse $ = \uncover<5->{\sqrt{1^2+x^2}.}$}
\item<6->  Then \alert<handout:0| 6-7>{$\cos (\Arctan x) = \uncover<7-| handout:0>{\frac{1}{\sqrt{1+x^2}}.}$}
\end{itemize}
\begin{pspicture}(-0.2,-0.5)(4.5,3.2)
\psframe*[linecolor=white](-0.2,-0.5)(4.5,3.2)
\psline[linecolor=red!1](4.5,-0.5)(4.5,-0.49)
\psline(0,0)(4,0)(4,3)(0,0)
\psline(3.8,0)(3.8,0.2)(4,0.2)
\fcAngle{0}{0.643501}{0.5}{$y$}
\uncover<3->{%
\rput[l](4.1, 1.5){$x$}
\rput[t](2,-0.1 ){\alertNoH{6,7}{$1$}}
}%
\uncover<5->{%
\rput[rb](1.9,1.6){\alertNoH{6,7}{$\sqrt{x^2+1}$}}
}%
\end{pspicture}
%\ \only<handout:0| -2>{%
%\includegraphics[width=5cm]{inverse-trig/pictures/07-06-ex3a.pdf}%
%}%
%\only<handout:0| 3-4>{%
%\includegraphics[width=5cm]{inverse-trig/pictures/07-06-ex3b.pdf}%
%}%
%\only<5->{%
%\includegraphics[width=5cm]{inverse-trig/pictures/07-06-ex3c.pdf}%
%}%
\end{example}
\end{frame}
% end module arctan-ex3

% begin module arctan-ex4
\begin{frame}
\begin{example}
Evaluate 
\[
\lim_{x\rightarrow 2^+} \arctan \left( \frac{1}{x-2}\right) .
\]
\uncover<2->{
\[
\frac{1}{x-2} \rightarrow \infty \qquad \text{ as } \qquad x\rightarrow 2^+.
\]
}
\uncover<3->{
Therefore 
\[
\alert<handout:0| 3-4>{\lim_{x\rightarrow 2^+} \arctan \left( \frac{1}{x-2}\right) = \uncover<4-| handout:0>{\frac{\pi}{2}.}}
\]
}
\end{example}
\end{frame}
% end module arctan-ex4

% begin module arctan-derivative
\begin{frame}
\begin{theorem}[The Derivative of $\tan^{-1} x$]
\[
\frac{\diff}{\diff x} (\tan^{-1} x) = \frac{1}{1 + x^2}.
\]
\end{theorem}
\begin{proof}
\begin{itemize}
\item  $\tan$ is differentiable, so $\tan^{-1}$ is too.
\item<2->  Let $y = \tan^{-1} x$.  Then $\alert<handout:0| 3,7>{\tan y = x}$.
\item<3->  Differentiate implicitly with respect to $x$:
\item<4->  $\sec^2 y \frac{\diff y}{\diff x} = 1$.  
\item<5->  $\frac{\diff y}{\diff x} = \frac{1}{\sec^2 y} \uncover<6->{= \frac{1}{1+\alert<handout:0| 7>{\tan^2y}}} \uncover<7->{= \frac{1}{1 + \alert<handout:0| 7>{x^2}}.}$  
\end{itemize}
\end{proof}
\end{frame}
% end module arctan-derivative

% begin module inverse-trig-summary
\begin{frame}
The remaining inverse trigonometric functions aren't used often, and are summarized here.
\[
\begin{array}{llcrcl}
y = \csc^{-1} x &%
(|x| \geq 1) &%
\Leftrightarrow &%
\csc y = x &%
\text{ and } &%
y\in (0,\pi /2] \cup (\pi , 3\pi /2] \\%
y = \sec^{-1} x &%
(|x| \geq 1) &%
\Leftrightarrow &%
\sec y = x &%
\text{ and } &%
\alert<2>{y\in [0,\pi /2) \cup [\pi , 3\pi /2)} \\%
y = \cot^{-1} x &%
(|x| \in \mathbb{R}) &%
\Leftrightarrow &%
\cot y = x &%
\text{ and } &%
y\in (0,\pi )
\end{array}
\]

\ \only<handout:0| -1>{%
\includegraphics[width=5cm]{inverse-trig/pictures/07-06-seca.pdf}%
}%
\only<2->{%
\includegraphics[width=5cm]{inverse-trig/pictures/07-06-secb.pdf}%
}%
\end{frame}

\begin{frame}
Table of derivatives of inverse trigonometric functions: 
\begin{align*}
\frac{\diff}{\diff x} (\Arcsin x) & = %
\frac{1}{\sqrt{1-x^2}} &%
\frac{\diff}{\diff x} (\csc^{-1} x) & = %
-\frac{1}{x\sqrt{x^2-1}} \\%
\frac{\diff}{\diff x} (\Arccos x) & = %
-\frac{1}{\sqrt{1-x^2}} &%
\frac{\diff}{\diff x} (\sec^{-1} x) & = %
\frac{1}{x\sqrt{x^2-1}} \\%
\frac{\diff}{\diff x} (\Arctan x) & = %
\frac{1}{1+x^2} &%
\frac{\diff}{\diff x} (\Arccot x) & = %
-\frac{1}{1+x^2} %
\end{align*}
\end{frame}
% end module inverse-trig-summary

% begin module arcsin-ex5
\begin{frame}
\chainruley{\frac{1}{\Arcsin x}}{\Arcsin x}{u^{-1}}{-UU^{-2}}{\frac{1}{\sqrt{1-x^2}}}{-\frac{1}{(UU)^2\sqrt{1-x^2}}}{0}
\end{frame}
% end module arcsin-ex5

% end lecture

% begin lecture
\lect{March 21, 2014}{Lecture 16}{16}
% end lecture

% begin lecture
\lect{March 28, 2014}{Lecture 18}{18}
% end lecture

\end{document}
