\documentclass{article}
\usepackage{amsmath, amsfonts, amssymb, verbatim, hyperref}
\usepackage{enumitem}
\usepackage{pst-plot}
\usepackage{pstricks}

%\addtolength{\hoffset}{-3.5cm}
%\addtolength{\textwidth}{6.8cm}
%\addtolength{\voffset}{-3.3cm}
%\addtolength{\textheight}{6.3cm}

\newcommand{\websitebase}{https://piazza.com/umb/summer2015/m130}

\usepackage{pdfpages}

\title{Math 130 Precalculus \\ Summer 2015}
\date{}
\begin{document}

%\color{green}
\maketitle
%\noindent\textbf{Time and place.}
%Monday, Wednesday, Friday 10-10:50, McCormack, Room 417, first floor. Monday 11:00-11:50,

\noindent \textbf{Instructor.} Todor Milev, \href{mailto:todor.milev@umb.edu}{\nolinkurl{todor.milev@umb.edu}} \quad \quad \quad .

\medskip
\noindent \textbf{Office hours. } Office hours by appointment after class. Walk in office hours, Monday, Tuesday, Wednesday, Thursday - after class (12:00-13:00), only if I am in the office. Room: S-03-65.

\medskip
\noindent \textbf{Online resources. }  \url{\websitebase/home}  \quad \quad \quad .


\medskip\noindent \textbf{Textbook. } Ron Larson, Precalculus, a concise course. 

\medskip \noindent \textbf{Lecture slides. }  \url{\websitebase/resources} \quad \quad \quad .

%\medskip \noindent \textbf{Master problem sheet (latest version). }  \url{\websitebase/resources} \quad \quad \quad .


\medskip\noindent Lecture slides will become available as the course progresses.

%\medskip
%\noindent \textbf{Prerequisite. } A standard pre-calculus course or equivalent.


\medskip
\noindent \textbf{Grades.} Your grade will consist of two tests, a comprehensive final exam, and a number of quizzes. 
\begin{itemize}
\item The quizzes will account for 20\% of your total grade.
\item The tests will account for 50\% (25\% each) of your total grade.
\item The final will account for 30\% of your total grade.
\end{itemize}
Please note that missed tests can not be made up, unless there is a valid medical reason accompanied with an official signed document from a medical doctor. Letter grades will be assigned as follows. 

\begin{center}
\begin{tabular}{lc|lc}
A & 85-100 & C & 65-69 \\
A-& 82-84 & C- & 62-64 \\
B+& 80-81 & D+ & 60-61 \\
B & 75-79& D & 55-59\\
B-& 72-74& D- & 50-54\\
C+& 70-71& F & below 50\\
\end{tabular}

\end{center}

No books, notes, calculators or any other electronic device (such as mobile phones) are allowed during any exam unless otherwise stated.

\medskip
\noindent \textbf{Homework.} You will be assigned homework, which will be posted on

\url{https://piazza.com/umb/spring2015/m141/resources} \quad \quad \quad .

\noindent I will expect you to complete the homework in written form in a convenient for you format (notebook, folder, etc.). However, \textbf{I will not check/collect/proofread homework.} 
 
\medskip
\noindent \textbf{Quizzes.} You will be given quizzes in class. \textbf{The time of the quiz will be announced in class}. Quizzes may be announced from one lecture day to the next. Your quiz problem will be one of your homework problems, verbatim. 

\medskip
\noindent \textbf{Student conduct.} Students  are required to adhere the University Policy on Academic Standards and Cheating, to the University Statement of Plagiarism and the Documentation of Written Work, and to the Code of Student Conduct as described in the catalog of Undergraduate programs, pages 44-45 and 48-52. The code is available at the following web-page.

\noindent\url{http://www.umb.edu/life_on_campus/policies/code/}

%\medskip
%\noindent \textbf{Topics to be covered.} 

%\medskip
%\noindent \textbf{List of topics from previous years.} The list of topics is a preliminary guideline, and will be subject to change.
%\begin{enumerate}[label*=\arabic*.]
%\item Functions.
%\begin{enumerate}[label*=\arabic*.]
%\item Ways to represent a function.
%\item Some essential functions.
%\item New functions from old.
%\end{enumerate}
%\end{enumerate}


\end{document}