\documentclass%
%[handout]
{beamer}
% % % % % % % %
% % % % % % % %
% % % % % % % %
%IMPORTANT
%compiles with 
%pdflatex -shell-escape 
%IMPORTANT
% % % % % % % %
% % % % % % % %
% % % % % % % %
\mode<presentation>
{
\useinnertheme{rounded}
\useoutertheme{infolines}
\usecolortheme{orchid}
\usecolortheme{whale}
}

\usepackage[english]{babel}
\usepackage[latin1]{inputenc}
\usepackage[all,cmtip]{xy}
\usepackage{times}
\usepackage[T1]{fontenc}
\usepackage{../example-templates}
\usepackage{../pstricks-commands}

\usepackage{auto-pst-pdf}
\usepackage{pst-plot}
%\usepackage{pstricks-add} 

% Or whatever. Note that the encoding and the font should match. If T1
% does not look nice, try deleting the line with the fontenc.

\graphicspath{{../../modules/}}

\newtheoremstyle{partialproof}{3pt}{3pt}{}{}{}{.}{.5em}{}
\theoremstyle{partialproof} \newtheorem{partialproof}[theorem]{Proof.}
%\DeclareMathOperator{\diff}{d}
\setbeamertemplate{navigation symbols}{}

\includeonlylecture{1}

\newcommand{\lect}[3]{
  \date{#1}
  \lecture[#1]{#2}{#3}
}

\setbeamertemplate{footline}
{
  \leavevmode%
  \hbox{%
  \begin{beamercolorbox}[wd=.333333\paperwidth,ht=2.25ex,dp=1ex,center]{author in head/foot}%
    \usebeamerfont{author in head/foot}\insertshortauthor
  \end{beamercolorbox}%
  \begin{beamercolorbox}[wd=.333333\paperwidth,ht=2.25ex,dp=1ex,center]{title in head/foot}%
    \usebeamerfont{title in head/foot}\insertshorttitle
  \end{beamercolorbox}%
  \begin{beamercolorbox}[wd=.333333\paperwidth,ht=2.25ex,dp=1ex,center]{date in head/foot}%
    \usebeamerfont{date in head/foot}\insertshortdate{}
  \end{beamercolorbox}}%
  \vskip0pt%
}

% If you have a file called "university-logo-filename.xxx", where xxx
% is a graphic format that can be processed by latex or pdflatex,
% resp., then you can add a logo as follows:

%\pgfdeclareimage[height=0.8cm]{logo}{bluelogo}
%\logo{\pgfuseimage{logo}}
\renewcommand{\Arcsin}{\arcsin}
\renewcommand{\Arccos}{\arccos}
\renewcommand{\Arccot}{\arccot}
\renewcommand{\Arctan}{\arctan}


\begin{document}

\AtBeginLecture{%

\title[\insertlecture]{FreeCalc}
\subtitle{\insertlecture}
\author[FreeCalc]{}
\institute[UMass Boston]{University of Massachusetts Boston}
\date{\insertshortlecture}
\begin{frame}
  \titlepage
\end{frame}
}%

% begin lecture
\lect{\today}{Sample}{1}
%%begin module partial-fractions-building-blocks-intro
\begin{frame}
\frametitle{Integrating arbitrary rational functions}
Let $\frac{P(x)}{Q(x)}$ be an arbitrary rational function, i.e., a quotient of polynomials.
\begin{question}
Can we integrate $\displaystyle\int \frac{P(x)}{Q(x)}dx$?
\end{question}
\begin{itemize}
\item The answer is Yes. We will now proceed to learn how.
\item The algorithm for integration is roughly:
\begin{itemize}
\item We use algebra to split $\frac{P(x)}{Q(x)}$ into partial fractions. 
\item We use linear substitutions to bring each partial fraction integral to one of 4 basic building block integrals.
\item We solve each building block integral and collect the terms.
\end{itemize}
\item We study the algorithm ``from the ground up'': we start with the building blocks.
\end{itemize}
\end{frame}

%end module partial-fractions-building-blocks-intro
%%begin module partial-fractions-building-blocks-3-and-4-intro
\begin{frame}
\frametitle{The building blocks}
Let $n$ be a positive integer.
\begin{itemize}
\item (Building block I) The first building block integral is:  

$\displaystyle \int \frac{1}{x^n }\diff x\quad .$
\item<2-> (Building block II) The second building block integral is: 

$\displaystyle \int \frac{\alert<4>{x}}{(1+\alert<3>{x^2})^n }\alert<4>{\diff x}\quad .$ \quad \uncover<3->{ (Note: $\alert<3>{u=x^2}, \alert<4>{xdx=\frac{1}{2}\diff u}$ transforms II to I).}
\item<5-> (Building block III) The third building block integral is: 

$\displaystyle \int \frac{1}{(1+x^2)^n }\diff x\quad .$
\item<6-> The case $n=1$ is special for each of the building blocks: 

$\displaystyle \int \frac{1}{x}\diff x$, $\displaystyle \int \frac{x}{1+x^2 }\diff x$ and $\displaystyle \int \frac{1}{1+x^2 }\diff x$.
\item<7-> The case $n=1$ we call respectively building block Ia, IIa and IIIa. 
\uncover<8-> {The case $n>1$ we call respectively building block Ib, IIb and IIIb.} \uncover<9->{ This ``building block'' terminology serves our convenience, and is not a part of standard mathematical terminology. }
\end{itemize}

\end{frame}
%end module partial-fractions-building-blocks-3-and-4-intro
%%begin module partial-fractions-building-block-1a
\begin{frame}
\frametitle{Building block Ia}
Building block Ia: $\displaystyle \int \frac{1}{x }\diff x$.
\begin{example} Integrate building block Ia
\[
\int \frac{1}{x }\diff x \uncover<2->{\alertNoH{2,3}{ =}}\fcAnswer{3}{\ln | x | +C} 
\]
\end{example}
\end{frame}
\begin{frame}
\frametitle{Linear substitutions leading to building block Ia}
Building block Ia: $\displaystyle \int \frac{1}{x }\diff x=\ln |x|+ C$.
\begin{example} Integrate
\[
\begin{array}{r@{~}c@{~}ll|l}
\displaystyle \alertNoH{9}{\int \frac{1}{-4x+5 }\diff x} \uncover<2->{&=&\displaystyle \int \frac{1}{(-4x+5) }\frac{\alertNoH{3}{ \diff (\alertNoH{2}{-4} x)}}{ (\alertNoH{2}{-4})}} \\
\uncover<3->{&=&\displaystyle \int \frac{1}{\alertNoH{4}{(-4x+5)} } \frac{\alertNoH{3}{\diff (\alertNoH{4}{-4x+5})}}{ (-4)} \uncover<4->{&&\text{Set } \alertNoH{4,8}{u=-4x+5}}}\\
\uncover<4->{&=&\displaystyle \int \alertNoH{6}{\frac{1}{\alertNoH{4}{u}}}\frac{\diff \alertNoH{4}{u}}{(\alertNoH{5}{-4})}}\\
\uncover<5->{&=&\displaystyle \alertNoH{5}{-\frac{1}{4}} \alertNoH{7}{\int \alertNoH{6}{u^{-1}} \diff u} }\uncover<7->{=-\frac{1}{4}\alertNoH{7}{\ln |\alertNoH{8}{u}|}+C}\\
\uncover<8->{&=&\displaystyle \alertNoH{9}{-\frac{1}{4}\ln |\alertNoH{8}{-4x+5}|  +C}\quad .}
\end{array}
\]

\end{example}
\end{frame}
\begin{frame}
\frametitle{Lin. subst. leading to building block Ia: general case}
Building block Ia: $\displaystyle \int \frac{1}{x }\diff x=\ln |x|+ C$.
\begin{example} Integrate
\[
\begin{array}{r@{~}c@{~}ll|l}
\displaystyle \alertNoH{1}{ \int \frac{1}{\phantom{-}ax+b }\diff x}&=&\displaystyle \int \frac{1}{(\phantom{-}ax + b) }\frac{\diff (\phantom{-}a x)}{a} \\
&=&\displaystyle \int \frac{1}{(\phantom{-}ax+b) }\frac{\diff (\phantom{-}ax+b)}{a} &&\text{Set }u=ax+b{~~~~~~~~~~~~~~~~~}\\
&=&\displaystyle \int\frac{1}{u}\frac{\diff u}{\phantom{(-}a\phantom{)}}\\
&=&\displaystyle \phantom{-}\frac{1}{a}\int u^{-1} \diff u =\phantom{-}\frac{1}{a}\ln |u|+C\\
&=&\displaystyle\alertNoH{1}{ \phantom{-}\frac{1}{a}\ln |ax+b|  +C}\quad .
\end{array}
\]

\end{example}
\end{frame}

%end module partial-fractions-building-block-1a



%%begin module partial-fractions-building-block-1b
\begin{frame}
\frametitle{Building block Ib}
Building block Ib: $\displaystyle \int \frac{1}{x^n }\diff x=\int x^{-n}\diff x$, $n\neq 1$. 
\begin{example} Integrate the building block integral Ib
\[
\int \frac{1}{x^n }\diff x\quad , n\neq 1.
\]

\[
\int \frac{1}{x^n}\diff x \uncover<2->{\alert<2,3>{= \int x^{-n}\diff x}}\uncover<3->{\alert<3>{=}} \uncover<4->{\alert<4>{\frac{x^{-n+1}}{-n+1} +C}}
\]
\end{example}
\end{frame}
\begin{frame}
\frametitle{Linear substitutions leading to building block Ib}
Building block Ib: $\displaystyle \int \frac{1}{x^n }\diff x=\int x^{-n}\diff x= \frac{x^{-n+1}}{-n+1} +C$, $n\neq 1$. 
\begin{example} Integrate 
\[
\begin{array}{rcll|l}
\alert<8>{\displaystyle \int \frac{1}{(3x+5)^3 }\diff x} \uncover<2->{&=&\displaystyle \int \frac{1}{(3x+5)^3 }\frac{\alert<3>{ \diff (\alert<2>{3} x)}}{\alert<2>{3}}} \\
\uncover<3->{&=&\displaystyle \int \frac{1}{(\alert<4>{3x+5})^3 }\frac{\alert<3>{\diff ( \alert<4>{3 x+5})}}{3}} \uncover<4->{&&\text{Set }\alert<4,7>{ u=3x+5}}\\
\uncover<4->{&=&\displaystyle \int\alert<5>{ \frac{1}{{\alert<4>{u}}^3} } \frac{\diff \alert<4>{u}}{3}}\\
\uncover<5->{ &=&\displaystyle \frac{1}{3} \alert<6>{\int \alert<5>{ u^{-3}} \diff u}} \uncover<6->{ =\frac{1}{3} \alert<6>{ \frac{{\alert<7>{u}}^{-2}}{(-2)}}+C}\\
\uncover<7->{&=&\displaystyle \alert<8>{-\frac{1}{6(\alert<7>{3x+5})^2}+C}\quad .}
\end{array}
\]

\end{example}
\end{frame}
\begin{frame}
\frametitle{Lin. subst. leading to building block Ib: general case}
Building block Ib: $\displaystyle \int \frac{1}{x^n }\diff x=\int x^{-n}\diff x= \frac{x^{-n+1}}{-n+1} +C$, $n\neq 1$. 
\begin{example} Let $n\neq 1$. Integrate 
\[
\begin{array}{rcll|l}
\displaystyle \alert<1>{\int \frac{1}{(ax+b)^n }\diff x} &=&\displaystyle \int \frac{1}{(ax+b)^n }\frac{\diff (a x)}{a} \\
&=&\displaystyle \int \frac{1}{(ax+b)^n }\frac{\diff (a x+b)}{a} &&\text{Set }u=ax+b\\
&=&\displaystyle \int\frac{1}{u^3}\frac{\diff u}{a}\\
&=&\displaystyle \frac{1}{a}\int u^{-n} \diff u =-\frac{1}{a} \frac{u^{-n+1}}{(n-1)}+C\\
&=&\displaystyle \alert<1>{-\frac{1}{ a(n-1)(ax+b)^{n-1}}+C}\quad .
\end{array}
\]

\end{example}
\end{frame}

%end module partial-fractions-building-block-1b
%%begin module partial-fractions-building-blocks-2a-and-3a
%This module may be too long, perhaps a split is needed.


%\begin{comment}






%\end{comment}
%end module partial-fractions-building-blocks-2a-and-3a
%%begin module Building block IIb

\begin{frame}
\frametitle{Building block IIb}
\begin{example}
\[
\begin{array}{rcl}
\displaystyle \int \frac{\alert<2>{ x} }{(x^2+1)^n} \alert<2>{\diff x} \uncover<2->{&=&\displaystyle  \int \frac{1}{(\alert<3>{x^2+1})^n} \alert<2>{\frac{\diff \left(\alert<3>{x^2+1} \right )}{2}}} \\
\uncover<3->{&=& \displaystyle \frac{1}{2}\int {\alert<3>{u}}^{-n}\diff \alert<3>{ u}} \\
\uncover<4->{&=&\left\{\begin{array}{ll}\displaystyle
\uncover<5->{\alert<5>{ \frac{1}{2}\ln (x^2+1) +C}} & \alert<4,5>{\text{if }n=1} \\
\uncover<7->{\alert<7>{\displaystyle \frac{1}{2} \frac{(x^2+1)^{-n+1}}{(-n+1)}+C}} &\alert<6>{ \text{if }n\neq -1}
\end{array} 
\right. ,}
\end{array}
\]
\uncover<3->{where we used the substitution $\alert<3>{u=x^2+1}$.}

\end{example}

\end{frame}

%end module Building block IIb
%begin module Building block IIIb

\begin{frame}
\frametitle{Building block IIIb: example illustrating main idea}
\begin{example}
Integrate $\int \frac{\diff x}{(x^2+1)^2}$. We start with an already known integral:
\[
\begin{array}{rcl}
\uncover<2->{\alert<12,13>{\alert<15>{\Arctan x}+C}} 
\only<1-12>{
\uncover<2->{
&=&\displaystyle \int \alert<3>{\frac{1}{x^2+1}}\diff \alert<4>{ x}}\\
\uncover<3->{&=&\displaystyle \alert<3>{\frac{1}{x^2+1}} \alert<4>{ x}-\int \alert<4>{x} \alert<5,6>{ \diff \left(\alert<3>{\frac{1}{x^2+1}}\right)} }\\
\uncover<5->{&=&\displaystyle \frac{x}{x^2+1} \alert<7>{-} \int \alert<7>{ x} \only<5>{\alert<5>{\textbf{?}}} \uncover<6->{\alert<6>{ \left( \alert<7>{-} \frac{\alert<7>{ 2x} }{ (x^2+1)^2}\right)\diff x}}}\\
\uncover<7->{&=&\displaystyle \frac{x}{x^2+1}\alert<7>{ +2} \int \frac{ \uncover<8->{\alert<8>{-\alert<10>{1}}+}\alert<9>{ \alert<7>{ x^2} \uncover<8->{\alert<8>{+1}}}}{\alert<9,10>{(x^2+1)^2}}\diff x }\\
\uncover<9->{&=&\displaystyle  \frac{x}{x^2+1}+2\alert<11>{\int \alert<9>{\frac{1}{x^2+1}} \diff x}-2\int \alert<10>{\frac{1}{(x^2+1)^2}}\diff x}\\
} %only<1-12>
\uncover<11->{ &\alert<12,13>{=}&\displaystyle \alert<12,13>{ \frac{x}{x^2+1}+ \alert<15>{2\alert<11>{\Arctan x} }\alert<14>{ -2 \int \frac{\diff x}{(x^2+1)^2}}} {~~~~~~~~~~~~~~~}} 
\end{array}
\]
\only<13->{
\uncover<14->{Rearrange terms \uncover<16->{and divide by $2$ to get the desired integral:}
\[
\alert<14>{\uncover<14,15>{2} \int \frac{\diff x}{(1+x^2)^2}}=\uncover<16->{\frac{1}{2}} \left(\frac{x}{x^2+1}+ \alert<15>{\Arctan x}  \right)+\uncover<14->{C'}\uncover<16->{'}\quad .
\]
}%uncover14
}%uncover13
\end{example}
\vspace{8cm}
\end{frame}

\begin{frame}
\frametitle{Building block IIIb}
\begin{itemize}
\item<1-> Building block IIIa: 
\[
\uncover<6->{\alert<6>{J(1)=}} \int \frac{1}{(x^2+1)}\diff x=\alert<6>{\arctan x+C}\quad .
\] 
\item<2-> Block IIIb:
\[
\uncover<4->{\alert<4>{J(n)=}} \alert<4>{\int \frac{1}{(x^2+1)^n}\diff x}
\] 
\item<3-> Unlike other cases, IIIb is much harder than IIIa.
\item<4-> Set $\alert<4>{J(n)=\int \frac{1}{(x^2+1)^n}\diff x}$. \uncover<5->{We are looking for a formula for $J(n)$.} \uncover<6->{We know $\alert<6>{J(1)=\arctan x+C}$ (this is block IIIa).}
\item<7-> We start by $J(n-1) =\int \frac{1}{(x^2+1)^{n-1}} \diff x$ and integrate by parts.
\item<8-> In this way we end up expressing $J(n)$ via $J(n-1)$.
\item<9-> We work our way from $J(n)$ to $J(n-1)$, from $J(n-1)$ to $J(n-2)$, and so on, until we get to $J(1)$.
\end{itemize} 
\end{frame}

\begin{frame}
\begin{example}
Recall that $\alert<11,12>{J(n)=\int \frac{1}{(x^2+1)^{n}}\diff x}$. %\uncover<3->{Set $\alert<3>{u=\frac{1}{(1+x^2)^{n-1}}}$.} 
\uncover<2->{We have that:}
\[
\begin{array}{rcl}
\uncover<2->{\alert<13,14,16>{J(n-1)}}
\only<1-13>{\uncover<2->{&\alert<13>{=} & 
\displaystyle \int \alert<3>{\frac{1}{(x^2+1)^{n-1 }}} \diff \alert<4>{x} } \\
\uncover<3->{&=&\displaystyle  \alert<3>{\frac{1}{(x^2+1)^{n-1}} } \alert<4>{x}-\int  \alert<4>{x} \alert<5,5>{ \diff \left(\alert<3>{ \frac{1}{ (1+x^2)^{ n-1}}}\right)}}\\
\uncover<5->{&=&\displaystyle  \frac{x}{(x^2+1)^{n-1}} \alert<7>{-} \int \alert<7>{ x}  \only<5>{\alert<5>{\textbf{?}}} \uncover<6->{\alert<6>{\frac{ \alert<7>{(-n+1) 2 x}}{ (1+x^2 )^{n}} \diff x}}} \\
\uncover<7->{ &=&\displaystyle  \frac{x}{(x^2+1)^{n-1}} \alert<7>{+ 2(n-1)} \int \frac{\uncover<8->{ \alert<8,9>{1+}} \alert<7,9>{x^2} \uncover<8->{\alert<8,10>{-1}}}{ \alert<9,10>{ (1+x^2)^n} }\diff x}\\
\uncover<9->{ &=&\displaystyle  \frac{x}{(x^2+1)^{n-1}}+ 2(n-1)\alert<11>{ \int \alert<9>{\frac{1}{(1+x^2)^{n-1}}} \diff x}} \\
\uncover<9->{&&\displaystyle \alert<10>{-} 2(n-1)\alert<12>{ \int \alert<10>{\frac{1}{(1+x^2)^n}}\diff x} }\\
} %only<1-13>
\uncover<11->{&\alert<13,14>{=}&\displaystyle \alert<13,14>{ \frac{x}{(x^2+1)^{n-1}}+\alert<16>{ 2(n-1)\alert<11>{ J(n-1)}}  \alert<15>{ -2(n-1)\alert<12>{J(n)}}}\quad .}
\end{array}
\]

\only<14->{
\uncover<15->{
Rearrange to get:
\[
\begin{array}{rcl}
\alert<15>{ \alert<17>{2(n-1)}J(n) } &=& \displaystyle \frac{x}{(x^2+1)^{n-1}}+\alert<16>{(2n-3) J(n-1)} \\
\uncover<17->{ \alert<19,20,21>{J(n)}&\alert<19,20,21>{=}&\displaystyle \alert<19,20,21>{ \frac{x}{ \alert<17>{(2n-2)} (x^2+ 1)^{ n-1}}+ \frac{2n-3}{\alert<17>{ 2n-2}}J(n-1)} \quad .}
\end{array}
\]

\uncover<18->{In this way we expressed $J(n)$ using $J(n-1)$.} \uncover<19->{We apply the above formula consecutively:

$
\alert<19>{ J(n)=  \frac{x}{ (2n-2) (x^2+ 1)^{ n-1}}+   \frac{2n-3}{ 2n-2}\only<19,20>{\alert<20>{J(n-1)}  \phantom{\left(\frac{x}{(2n-4)(x^2+1)^{n-2}}\right)} 
}} \only<21->{\alert<21>{\left(\frac{x}{(2n-4)(x^2+1)^{n-2}}+\frac{2n-5 }{2n-4} \alert<22,23>{ J(n-2)} \right)}}  \uncover<22->{\alert<22>{=\dots}}
$
}

\noindent \uncover<22->{\alert<22>{and so on.}} \uncover<23->{A formula for the final result can be written using the above (found in Calculus for beginners, Chapter ``Techniques of integration'').}

} %uncover15
} %uncover14
\end{example}


\vspace{8cm}
\end{frame}
%end module Building block IIIb


%% begin module partial-fractions-intro
\begin{frame}
\frametitle{Integration of Rational Functions by Partial Fractions}

A rational function is a function which can be written as ratio of polynomials, $\frac{P(x)}{Q(x)}$. To integrate any rational function, we will express $\frac{P(x)}{Q(x)}$ as a sum of simpler expressions called partial fractions. We start with an example. 

Put $2/(x-1)$ and $1/(x+2)$ over a common denominator:
\[
\alert<5>{\frac{2}{x-1} - \frac{1}{x+2} } = %
\uncover<2->{%
\frac{2(x+2) - (x-1)}{(x-1)(x+2)} = %
}%
\uncover<3->{%
\alert<5>{ \frac{x + 5}{x^2+x-2} }%
}%
\]

\uncover<4->{%
We can now solve the following integral:
\[
\int \alert<5>{ \frac{x+5}{x^2+x-2}}\diff x = %
\uncover<5->{%
\int \left(\alert<5>{\frac{2}{x-1} - \frac{1}{x+2}} \right) \diff x = %
}%
\uncover<6->{%
2\ln | x - 1| - \ln | x + 2| + C
}%
\]
}%

\end{frame}
% end module partial-fractions-intro

%% begin module partial-fractions-long-division
\begin{frame}
\frametitle{Review of polynomial notation}
Consider a rational function
\[
f(x) = \frac{P(x)}{Q(x)}
\]
where $P$ and $Q$ are polynomials.  Recall that the degree of $P$ is the highest power of $x$ in $P$ that has a non-zero coefficient.  That is, if
\[
P(x) = a_nx^n + a_{n-1}x^{n-1} + \cdots + a_1x + a_0
\]
where $a_n \neq 0$, then the degree of $P$ is $n$, and we write deg$(P) = n$.
\end{frame}
\begin{frame}\frametitle{Ensure denominator degree > numerator degree}
\begin{itemize}
\item To compute a partial fraction decomposition we need that the degree of the fraction numerator be less than the degree of the denominator.
\item<2-> Therefore our first step is to transform $\frac{P(x)}{Q(x)}$ to

$\displaystyle \frac{\alertNoH{6}{P(x)}}{\alertNoH{7}{Q(x)} }= \alertNoH{8}{S(x)}+\frac{\alertNoH{9}{R(x)}}{\alertNoH{7}{Q(x)}} $
where $S(x), R(x), Q(x)$ are polynomials and $\deg R<\deg Q$.
\item<3-> This is done using polynomial long division.
\item<4-> We recall that to divide the \alertNoH{6}{dividend $P(x)$} by the \alertNoH{7}{divisor $Q(x)$} to get \alertNoH{8}{quotient $S(x)$} with \alertNoH{9}{remainder $R(x)$} means \uncover<5->{ to find polynomials  $S(x), R(x)$ such that \alertNoH{9}{$\deg R<\deg Q$} and
\[
\alertNoH{6} {P(x)} = \alertNoH{8}{S(x)} \alertNoH{7}{Q(x)} + \alertNoH{9}{ R(x)}
\]
}
\item<10-> We review polynomial long division on examples.
\end{itemize}
\end{frame}
% end module partial-fractions-long-division

%% begin module partial-fractions-long-division-ex1
\begin{frame}
\begin{example} %[Example 1, p. 510]
Find $\int \frac{x^3 + x}{x - 1}\diff x$.
\begin{columns}[t]
\column{.45\textwidth}
\uncover<2->{%
\[
\begin{array}{r@{}r@{}c@{}r@{}c@{}c@{}c@{}r@{}r@{}}
& & & %
\uncover<4->{\alert<handout:0| 4-6,23>{x^2}} & %
\uncover<11->{\alert<handout:0| 11-13,23>{+}} & %
\uncover<11->{\alert<handout:0| 11-13,23>{ x}} & %
\uncover<17->{\alert<handout:0| 17-19,23>{+}} & %
\uncover<17->{\alert<handout:0| 17-19,23>{ 2}} & \\%
\cline{3-8}
\alert<handout:0| 3-6,10-13,16-19>{x} & %
\alert<handout:0| 5-6,12-13,18-19>{-1} & %
\Big) & %
\alert<handout:0| 3-4,7-8>{x^3} & %
 & %
 & %
\alert<handout:0| 9>{+} & %
\alert<handout:0| 9>{ x} & \\%
& & & %
\uncover<6->{\alert<handout:0| 6-8>{x^3}} & %
\uncover<6->{\alert<handout:0| 6-8>{-}} & %
\uncover<6->{\alert<handout:0| 6-8>{ x^2}} & %
&  & \\%
\cline{4-6}%
& & & %
&  & %
\uncover<8->{\alert<handout:0| 8,10-11,14-15>{x^2}} & %
\uncover<9->{\alert<handout:0| 9,14-15>{+}} & %
\uncover<9->{\alert<handout:0| 9,14-15>{ x}} & \\%
& & & %
&  & %
\uncover<13->{\alert<handout:0| 13-15>{x^2}} & %
\uncover<13->{\alert<handout:0| 13-15>{-}} & %
\uncover<13->{\alert<handout:0| 13-15>{ x}} & \\%
\cline{6-8}
& & & %
 & %
 & %
 & %
 & %
\uncover<15->{\alert<handout:0| 15-17,20-21>{2x}} & \\%
& & & %
 & %
 & %
 & %
 & %
\uncover<19->{\alert<handout:0| 19-21>{2x}} & %
\uncover<19->{\alert<handout:0| 19-21>{- 2}} \\%
\cline{8-9}%
& & & %
 & %
 & %
 & %
 & %
 & %
\uncover<21->{\alert<handout:0| 21,24>{2}} \\%
\end{array}
\]
}%

\only<handout:0| -4,10-11,16-17>{\uncover<3->{%
Divide %
}}%
\only<handout:0| 5-6,12-13,18-19>{%
Multiply %
}%
\only<handout:0| 7-8,14-15,20-21>{%
Subtract %
}%
\only<handout:0| 3-4>{%
$x^3$ %
}%
\only<handout:0| 10-11>{%
$x^2$ %
}%
\only<handout:0| 16-17>{%
$2x$ %
}%
\only<handout:0| 5-6>{%
$x^2$ %
}%
\only<handout:0| 12-13>{%
$x$ %
}%
\only<handout:0| 18-19>{%
$2$ %
}%
\only<handout:0| 7-8>{%
$x^3-x^2$ %
}%
\only<handout:0| 14-15>{%
$x^2-x$ %
}%
\only<handout:0| 20-21>{%
$2x-2$ %
}%
\only<handout:0| -4,10-11,16-17>{\uncover<3->{%
by %
}}%
\only<handout:0| 5-6,12-13,18-19>{%
by %
}%
\only<handout:0| 7-8,14-15,20-21>{%
from %
}%
\only<handout:0| 3-4,10-11,16-17>{%
$x$ %
}%
\only<handout:0| 5-6,12-13,18-19>{%
$x-1$ %
}%
\only<handout:0| 7-8>{%
$x^3$ %
}%
\only<handout:0| 14-15>{%
$x^2+x$ %
}%
\only<handout:0| 20-21>{%
$2x$ %
}%
\only<handout:0| 9>{%
Bring down the $x$%
}%
\invisible<1->{%
y%
}%
\column{.55\textwidth}
\begin{eqnarray*}
& & %
\uncover<22->{%
\int \frac{x^3 + x}{x - 1}\diff x %
}\\%
& \uncover<22->{ = } & %
\uncover<22->{%
\int \left( \alert<handout:0| 23>{x^2 + x + 2} + \frac{\alert<handout:0| 24>{2}}{x - 1}\right) \diff x
}\\%
& \uncover<25->{ = } & %
\uncover<25->{%
\frac{x^3}{3} + \frac{x^2}{2} + 2x %
}\\%
& & \uncover<25->{%
\qquad + 2\ln | x - 1 | + C %
}%
\end{eqnarray*}
\end{columns}
\end{example}
\end{frame}
% end module partial-fractions-long-division-ex1


\end{document}
