\documentclass%
%[handout]
{beamer}
% % % % % % % %
% % % % % % % %
% % % % % % % %
%IMPORTANT
%compiles with 
%pdflatex -shell-escape 
%IMPORTANT
% % % % % % % %
% % % % % % % %
% % % % % % % %
\mode<presentation>
{
\useinnertheme{rounded}
\useoutertheme{infolines}
\usecolortheme{orchid}
\usecolortheme{whale}
}

\usepackage[english]{babel}
\usepackage[latin1]{inputenc}
\usepackage[all,cmtip]{xy}
\usepackage{times}
\usepackage[T1]{fontenc}
\usepackage{../example-templates}
\usepackage{../pstricks-commands}

\usepackage{auto-pst-pdf}
\usepackage{pst-plot}
%\usepackage{pstricks-add} 

% Or whatever. Note that the encoding and the font should match. If T1
% does not look nice, try deleting the line with the fontenc.

\graphicspath{{../../modules/}}

\newtheoremstyle{partialproof}{3pt}{3pt}{}{}{}{.}{.5em}{}
\theoremstyle{partialproof} \newtheorem{partialproof}[theorem]{Proof.}
%\DeclareMathOperator{\diff}{d}
\newcommand{\diff}{\text{d}}
\setbeamertemplate{navigation symbols}{}

\includeonlylecture{1}

\newcommand{\lect}[3]{
  \date{#1}
  \lecture[#1]{#2}{#3}
}

\setbeamertemplate{footline}
{
  \leavevmode%
  \hbox{%
  \begin{beamercolorbox}[wd=.333333\paperwidth,ht=2.25ex,dp=1ex,center]{author in head/foot}%
    \usebeamerfont{author in head/foot}\insertshortauthor
  \end{beamercolorbox}%
  \begin{beamercolorbox}[wd=.333333\paperwidth,ht=2.25ex,dp=1ex,center]{title in head/foot}%
    \usebeamerfont{title in head/foot}\insertshorttitle
  \end{beamercolorbox}%
  \begin{beamercolorbox}[wd=.333333\paperwidth,ht=2.25ex,dp=1ex,center]{date in head/foot}%
    \usebeamerfont{date in head/foot}\insertshortdate{}
  \end{beamercolorbox}}%
  \vskip0pt%
}

% If you have a file called "university-logo-filename.xxx", where xxx
% is a graphic format that can be processed by latex or pdflatex,
% resp., then you can add a logo as follows:

%\pgfdeclareimage[height=0.8cm]{logo}{bluelogo}
%\logo{\pgfuseimage{logo}}
\renewcommand{\Arcsin}{\arcsin}
\renewcommand{\Arccos}{\arccos}
\renewcommand{\Arccot}{\arccot}
\renewcommand{\Arctan}{\arctan}


\begin{document}

\AtBeginLecture{%

\title[\insertlecture]{FreeCalc}
\subtitle{\insertlecture}
\author[FreeCalc]{}
\institute[UMass Boston]{University of Massachusetts Boston}
\date{\insertshortlecture}
\begin{frame}
  \titlepage
\end{frame}
}%

% begin lecture
\lect{\today}{Sample}{1}
%% begin module e-limit
\begin{frame}
\frametitle{The Number $e$ as a Limit}
\begin{theorem}[The Number $e$ as a Limit]
\[
e = \lim_{x\rightarrow 0} (1 + x)^{1/x}.
\]
\end{theorem}
\begin{proof}
\uncover<2->{Let $f(x) = \ln x$.  }%
\uncover<3->{Then $f'(x) = 1/x$, so $f'(1) = 1$.}%
\abovedisplayskip=0pt
\belowdisplayskip=0pt
\abovedisplayshortskip=0pt
\belowdisplayshortskip=0pt
\begin{align*}
\uncover<4->{\alert<handout:0| 10>{1} = f'(1)} & \uncover<4->{=} %
\uncover<4->{\lim_{h\rightarrow 0}\frac{f(1+h)-f(1)}{h}}%
\uncover<5->{ = \lim_{x\rightarrow 0}\frac{f(1+x)-f(1)}{x}}\\%
& \uncover<6->{=}  %
\uncover<6->{\lim_{x\rightarrow 0}\frac{\ln (1+x)-\ln (1)}{x}}%
\uncover<7->{ = \lim_{x\rightarrow 0}\frac{1}{x}\ln (1 + x)}\\%
& \uncover<8->{\alert<handout:0| 10>{=}}  %
\uncover<8->{\alert<handout:0| 10>{\lim_{x\rightarrow 0}\ln (1+x)^{1/x}}.}
\end{align*}
\uncover<9->{Then use the fact that \alert<handout:0| 11>{the exponential function is continuous}:}
\[
\uncover<9->{e = e^{\alert<handout:0| 10>{1}} =}%
\uncover<10->{\alert<handout:0| 11>{e^{\alert<handout:0| 10>{\lim_{x\rightarrow 0}\ln (1+x)^{1/x}}} =}}%
\uncover<11->{\alert<handout:0| 11>{\lim_{x\rightarrow 0}e^{\ln (1+x)^{1/x}}} =}%
\uncover<12->{\lim_{x\rightarrow 0} (1+x)^{1/x}.}\qedhere
\]
\end{proof}
\end{frame}
% end module e-limit

%%begin module e-limit-problems-ex1
\begin{frame}
\begin{example}
Compute \[\lim_{x\to \infty} \left(\frac{x+3}{x} \right)^{x}\].

\[
\begin{array}{rcll|l}
\displaystyle\lim\limits_{x\to \infty}\left(\frac{x+3}{x}\right)^x &=&\displaystyle  \lim\limits_{x\to \infty}\left(1+\frac{3}{x}\right)^x \\
&=& \displaystyle \lim\limits_{x\to \infty}\left(1+\frac{1}{\frac{x}{3}}\right)^{3\frac{x}{3} } && \text{Set } \frac{x}{3}=y\\
&=&\displaystyle \lim\limits_{\substack{x\to \infty \\ \frac{x}{3}=y\to \infty }}\left(1+\frac{1}{y}\right)^{3y} \\
&=&\displaystyle \displaystyle\lim\limits_{y\to \infty}\left(\left(1+\frac{1}{y}\right)^y\right)^3= e^3 \quad . 
\end{array}
\]

\end{example}
\end{frame}
\begin{frame}
\begin{example}
Compute 
\[
\begin{array}{rll|l}
&\displaystyle \lim_{x\to \infty} \left(\frac{x}{x-2} \right)^{2x+2}\\
=&\displaystyle
\lim\limits_{x\to \infty}\left(\frac{x-2 +2}{x-2} \right)^{2x+2}
= 
\lim\limits_{x\to \infty}\left(1+\frac{2}{x-2} \right)^{2x+2} \\
=&\displaystyle \lim\limits_{x\to \infty}\left(1+\frac{1}{\frac{x-2}{2}} \right)^{2(x-2+2)+2}\\
=&\displaystyle \lim\limits_{x\to \infty}\left(1+\frac{1}{\frac{x-2}{2}} \right)^{4\frac{x-2}{2}+6} = \displaystyle \lim\limits_{\frac{x-2}{2}
= y\to\infty} \left(1+\frac{1}{y}\right)^{4y+6}  && \text{Set } y=\frac{x-2}{2}\\
=&\displaystyle \lim\limits_{y\to \infty} \left(\left(1+\frac{1}{y} \right)^{y}\right)^4 \lim\limits_{y\to\infty} \left(1+\frac{1}{y}\right)^6 =e^4\cdot (1+0)^6\\
=& e^4
\quad . 
\end{array}
\]

\end{example}
\end{frame}
%end module e-limit-problems-ex1

%% begin module general-exponential-derivative
\begin{frame}
\begin{theorem}[The Derivative of $a^x$]
\[
\frac{\diff}{\diff x} (a^x) = a^x \ln a .
\]
\end{theorem}
\end{frame}
% end module general-exponential-derivative

%begin module differentials-rules
\begin{frame}
\begin{itemize}
\item All rules for computing with derivatives have analogues for computing with differential forms.
\item<2-> The rules for computing differential forms are a direct consequence of the corresponding derivative rules and the transformation law $\diff (f(x))=f'(x)\diff x$.
\end{itemize}
\end{frame}
\begin{frame}
\uncover<3>{Let $c$ be a constant.} Rule name: \phantom{p}
\only<1,2>{\alert<1,2>{product rule. }}
\only<3,4>{\alert<3,4>{constant derivative rule. }}
\only<7,8>{\alert<7,8>{sum rule. }}
\only<9,10>{\alert<9,10>{chain rule. }}
\only<11-12>{\alert<11-12>{power rule. }}
\only<13-14>{\alert<13-14>{exponent derivative rule. }}
%\only<5,6>{\alert<5,6>{ Constant derivative rule. }}

\begin{tabular}{lll}
Differential rule & Derivative rule\\
\uncover<2->{$\diff (fg)=g \diff f +f \diff g$} & 
\uncover<1->{$(fg)'=f'g +f g'$} \\
\uncover<4->{$\diff c=0$} &
\uncover<3->{$(c)'=0$} & $c$-const.\\
\uncover<6->{$\diff (cf)=c\diff f $} & 
\uncover<5->{$(cf)'=cf'$} &$c$-const. \\
\uncover<8->{$\diff (f+g)=\diff f +\diff g$} & 
\uncover<7->{$(f+g)'=f'+g'$}\\
\uncover<10->{$\diff f(g(x))=$ $ f'(g(x))\diff g(x) $}\\
\uncover<10->{$\phantom{\diff f(g(x))}=$ $ f'(g(x))g'(x)\diff x$} & 
\uncover<9->{$(f(g(x)))'= f'(g(x))g'(x)$} \\ 
\uncover<10->{$\diff f(g)\phantom{(x)}=f'(g)\diff g$} \\\hline
\uncover<12->{$\diff \left( x^n\right)= nx^{n-1}\diff x$} & 
\uncover<11->{$(x^n)'=nx^{n-1}$}\\
\uncover<14->{$\diff  e^x= e^x \diff x$} & 
\uncover<13->{$\left(e^x\right)'=e^x$}\\
\uncover<16->{$\diff \sin x = \cos x \diff x$} & 
\uncover<15->{$(\sin x)'= \cos x$}\\
\uncover<18->{$\diff \cos x = -\sin x \diff x$} & 
\uncover<17->{$(\cos x)'= -\sin x$}\\
\uncover<20->{$\displaystyle \diff \ln x=\frac{1}{x}\diff x$} & 
\uncover<19->{$\left(\ln x\right)' =\displaystyle \frac1x$}\\
\end{tabular}


\end{frame}
%end module differentials-rules
%begin module differentials-rules
\begin{frame}
\begin{itemize}
\item All rules for computing with derivatives have analogues for computing with differential forms.
\item<2-> The rules for computing differential forms are a direct consequence of the corresponding derivative rules and the transformation law $\diff (f(x))=f'(x)\diff x$.
\end{itemize}
\end{frame}
\begin{frame}
\uncover<3->{Let $c$ be a constant.} \uncover<handout:0| 1->{Rule name: \phantom{p}}
\only<handout:0|1,2>{\alertNoH{1,2}{product rule. }}
\only<handout:0|3,4>{\alertNoH{3,4}{constant derivative rule. }}
\only<handout:0|7,8>{\alertNoH{7,8}{sum rule. }}
\only<handout:0|9,10>{\alertNoH{9,10}{chain rule. }}
\only<handout:0|11-12>{\alertNoH{11-12}{power rule. }}
\only<handout:0|13-14>{\alertNoH{13-14}{exponent derivative rule. }}
%\only<5,6>{\alertNoH{5,6}{ Constant derivative rule. }}
\only<handout:0|22>{\alertNoH{22}{Integration by parts.}}
\only<handout:0|23>{\alertNoH{23}{Integration is linear.}}
\only<handout:0|24>{\alertNoH{24}{Substitution rule.}}

\uncover<handout:2|21->{Corresponding \alertNoH{21}{integration rules.} \uncover<handout:2|25->{\alertNoH{25}{ Integration rules justified via the Fundamental Theorem of Calculus}}}

\begin{tabular}{ll}
\only<handout:1|1-20>{Differential}\only<handout:2|21->{\alertNoH{21}{Integration}} rule & Derivative rule \\
\uncover<2->{\alertNoH{22}{$\uncover<handout:2|21->{\alertNoH{21}{\int}} \diff (fg)=\uncover<handout:2| 21->{ \alertNoH{21}{ \int }} g \diff f +\uncover<handout:2| 21->{ \alertNoH{21}{ \int }} f \diff g$} }&
\uncover<1->{$ (fg)'=f'g +f g'$} \\
\uncover<4->{$\uncover<handout:2|21->{\alertNoH{21}{\int}}\diff c= 0$} &
\uncover<3->{$(c)'=0$}\\
\uncover<6->{\alertNoH{23}{$\uncover<handout:2|21->{\alertNoH{21}{\int}}\diff (cf)=c \uncover<handout:2|21-> {\alertNoH{21}{\int}}\diff f $}} &
\uncover<5->{$(cf)'=cf'$} \\
\uncover<8->{\alertNoH{23}{$\uncover<handout:2|21->{\alertNoH{21}{\int}}\diff (f+g) = \uncover<handout:2|21->{\alertNoH{21}{\int}}\diff f +\uncover<handout:2|21->{\alertNoH{21}{\int}}\diff g$}} &
\uncover<7->{$(f+g)'=f'+g'$}\\
\uncover<10->{\alertNoH{24}{$\uncover<handout:2|21->{\alertNoH{21}{\int}}\diff f(g(x))=$ $ \uncover<handout:2|21->{\alertNoH{21}{\int}}f'(g(x))\diff g(x) $} }\\
\uncover<10->{\alertNoH{24}{$\phantom{\int \diff f(g(x))}= $  $\uncover<handout:2|21->{\alertNoH{21}{\int}} f'(g(x))g'(x)\diff x$} } &
\uncover<9->{$(f(g(x)))'= f'(g(x))g'(x)$} \\
\uncover<10->{\alertNoH{24}{$\uncover<handout:2|21->{\alertNoH{21}{\int}} \diff f(g)\phantom{(x)}= \uncover<handout:2|21->{\alertNoH{21}{\int}} f'(g) \diff g$} } \\\hline
\uncover<12->{$\diff  x^n= nx^{n-1}\diff x$} &
\uncover<11->{$(x^n)'=nx^{n-1}$}\\
\uncover<14->{$\diff  e^x= e^x \diff x$} &
\uncover<13->{$\left(e^x\right)'=e^x$}\\
\uncover<16->{$\diff \sin x = \cos x \diff x$} &
\uncover<15->{$(\sin x)'= \cos x$}\\
\uncover<18->{$\diff \cos x = -\sin x \diff x$} &
\uncover<17->{$(\cos x)'= -\sin x$}\\
\uncover<20->{$\displaystyle \diff \ln x=\frac{1}{x}\diff x$} &
\uncover<19->{$(\ln x)'=\displaystyle \frac1x$}\\
\end{tabular}


\end{frame}
%end module differentials-rules

\end{document}
