\documentclass%
%[handout]
{beamer}
% % % % % % % %
% % % % % % % %
% % % % % % % %
%IMPORTANT
%compiles with 
%pdflatex -shell-escape 
%IMPORTANT
% % % % % % % %
% % % % % % % %
% % % % % % % %
\mode<presentation>
{
\useinnertheme{rounded}
\useoutertheme{infolines}
\usecolortheme{orchid}
\usecolortheme{whale}
}

\usepackage[english]{babel}
\usepackage[latin1]{inputenc}
\usepackage[all,cmtip]{xy}
\usepackage{times}
\usepackage[T1]{fontenc}
\usepackage{../example-templates}
\usepackage{../pstricks-commands}

\usepackage{auto-pst-pdf}
\usepackage{pst-plot}
%\usepackage{pstricks-add} 

% Or whatever. Note that the encoding and the font should match. If T1
% does not look nice, try deleting the line with the fontenc.

\graphicspath{{../../modules/}}

\newtheoremstyle{partialproof}{3pt}{3pt}{}{}{}{.}{.5em}{}
\theoremstyle{partialproof} \newtheorem{partialproof}[theorem]{Proof.}
%\DeclareMathOperator{\diff}{d}
\newcommand{\diff}{\text{d}}
\setbeamertemplate{navigation symbols}{}

\includeonlylecture{1}

\newcommand{\lect}[3]{
  \date{#1}
  \lecture[#1]{#2}{#3}
}

\setbeamertemplate{footline}
{
  \leavevmode%
  \hbox{%
  \begin{beamercolorbox}[wd=.333333\paperwidth,ht=2.25ex,dp=1ex,center]{author in head/foot}%
    \usebeamerfont{author in head/foot}\insertshortauthor
  \end{beamercolorbox}%
  \begin{beamercolorbox}[wd=.333333\paperwidth,ht=2.25ex,dp=1ex,center]{title in head/foot}%
    \usebeamerfont{title in head/foot}\insertshorttitle
  \end{beamercolorbox}%
  \begin{beamercolorbox}[wd=.333333\paperwidth,ht=2.25ex,dp=1ex,center]{date in head/foot}%
    \usebeamerfont{date in head/foot}\insertshortdate{}
  \end{beamercolorbox}}%
  \vskip0pt%
}

% If you have a file called "university-logo-filename.xxx", where xxx
% is a graphic format that can be processed by latex or pdflatex,
% resp., then you can add a logo as follows:

%\pgfdeclareimage[height=0.8cm]{logo}{bluelogo}
%\logo{\pgfuseimage{logo}}
\renewcommand{\Arcsin}{\arcsin}
\renewcommand{\Arccos}{\arccos}
\renewcommand{\Arccot}{\arccot}
\renewcommand{\Arctan}{\arctan}


\begin{document}

\AtBeginLecture{%

\title[\insertlecture]{FreeCalc}
\subtitle{\insertlecture}
\author[FreeCalc]{}
\institute[UMass Boston]{University of Massachusetts Boston}
\date{\insertshortlecture}
\begin{frame}
  \titlepage
\end{frame}
}%

% begin lecture
\lect{\today}{Sample}{1}
% begin module e-limit
\begin{frame}
\frametitle{The Number $e$ as a Limit}
\begin{theorem}[The Number $e$ as a Limit]
\[
e = \lim_{x\rightarrow 0} (1 + x)^{1/x}.
\]
\end{theorem}
\begin{proof}
\uncover<2->{Let $f(x) = \ln x$.  }%
\uncover<3->{Then $f'(x) = 1/x$, so $f'(1) = 1$.}%
\abovedisplayskip=0pt
\belowdisplayskip=0pt
\abovedisplayshortskip=0pt
\belowdisplayshortskip=0pt
\begin{align*}
\uncover<4->{\alert<handout:0| 10>{1} = f'(1)} & \uncover<4->{=} %
\uncover<4->{\lim_{h\rightarrow 0}\frac{f(1+h)-f(1)}{h}}%
\uncover<5->{ = \lim_{x\rightarrow 0}\frac{f(1+x)-f(1)}{x}}\\%
& \uncover<6->{=}  %
\uncover<6->{\lim_{x\rightarrow 0}\frac{\ln (1+x)-\ln (1)}{x}}%
\uncover<7->{ = \lim_{x\rightarrow 0}\frac{1}{x}\ln (1 + x)}\\%
& \uncover<8->{\alert<handout:0| 10>{=}}  %
\uncover<8->{\alert<handout:0| 10>{\lim_{x\rightarrow 0}\ln (1+x)^{1/x}}.}
\end{align*}
\uncover<9->{Then use the fact that \alert<handout:0| 11>{the exponential function is continuous}:}
\[
\uncover<9->{e = e^{\alert<handout:0| 10>{1}} =}%
\uncover<10->{\alert<handout:0| 11>{e^{\alert<handout:0| 10>{\lim_{x\rightarrow 0}\ln (1+x)^{1/x}}} =}}%
\uncover<11->{\alert<handout:0| 11>{\lim_{x\rightarrow 0}e^{\ln (1+x)^{1/x}}} =}%
\uncover<12->{\lim_{x\rightarrow 0} (1+x)^{1/x}.}\qedhere
\]
\end{proof}
\end{frame}
% end module e-limit

%begin module e-limit-problems-ex1
\begin{frame}
\begin{example}
Compute \[\lim_{x\to \infty} \left(\frac{x+3}{x} \right)^{x}\].

\[
\begin{array}{rcll|l}
\displaystyle\lim\limits_{x\to \infty}\left(\frac{x+3}{x}\right)^x &=&\displaystyle  \lim\limits_{x\to \infty}\left(1+\frac{3}{x}\right)^x \\
&=& \displaystyle \lim\limits_{x\to \infty}\left(1+\frac{1}{\frac{x}{3}}\right)^{3\frac{x}{3} } && \text{Set } \frac{x}{3}=y\\
&=&\displaystyle \lim\limits_{\substack{x\to \infty \\ \frac{x}{3}=y\to \infty }}\left(1+\frac{1}{y}\right)^{3y} \\
&=&\displaystyle \displaystyle\lim\limits_{y\to \infty}\left(\left(1+\frac{1}{y}\right)^y\right)^3= e^3 \quad . 
\end{array}
\]

\end{example}
\end{frame}
\begin{frame}
\begin{example}
Compute 
\[
\begin{array}{rll|l}
&\displaystyle \lim_{x\to \infty} \left(\frac{x}{x-2} \right)^{2x+2}\\
=&\displaystyle
\lim\limits_{x\to \infty}\left(\frac{x-2 +2}{x-2} \right)^{2x+2}
= 
\lim\limits_{x\to \infty}\left(1+\frac{2}{x-2} \right)^{2x+2} \\
=&\displaystyle \lim\limits_{x\to \infty}\left(1+\frac{1}{\frac{x-2}{2}} \right)^{2(x-2+2)+2}\\
=&\displaystyle \lim\limits_{x\to \infty}\left(1+\frac{1}{\frac{x-2}{2}} \right)^{4\frac{x-2}{2}+6} = \displaystyle \lim\limits_{\frac{x-2}{2}
= y\to\infty} \left(1+\frac{1}{y}\right)^{4y+6}  && \text{Set } y=\frac{x-2}{2}\\
=&\displaystyle \lim\limits_{y\to \infty} \left(\left(1+\frac{1}{y} \right)^{y}\right)^4 \lim\limits_{y\to\infty} \left(1+\frac{1}{y}\right)^6 =e^4\cdot (1+0)^6\\
=& e^4
\quad . 
\end{array}
\]

\end{example}
\end{frame}
%end module e-limit-problems-ex1


\end{document}
