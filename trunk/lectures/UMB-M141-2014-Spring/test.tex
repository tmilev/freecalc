\documentclass%
%[handout]
{beamer}
% % % % % % % %
% % % % % % % %
% % % % % % % %
%IMPORTANT
%compiles with 
%pdflatex -shell-escape 
%IMPORTANT
% % % % % % % %
% % % % % % % %
% % % % % % % %
\mode<presentation>
{
\useinnertheme{rounded}
\useoutertheme{infolines}
\usecolortheme{orchid}
\usecolortheme{whale}
}

\usepackage[english]{babel}
\usepackage[latin1]{inputenc}
\usepackage[all,cmtip]{xy}
\usepackage{times}
\usepackage[T1]{fontenc}
\usepackage{../example-templates}
\usepackage{../pstricks-commands}

\usepackage{auto-pst-pdf}
\usepackage{pst-plot}
%\usepackage{pstricks-add} 

% Or whatever. Note that the encoding and the font should match. If T1
% does not look nice, try deleting the line with the fontenc.

\graphicspath{{../../modules/}}

\newtheoremstyle{partialproof}{3pt}{3pt}{}{}{}{.}{.5em}{}
\theoremstyle{partialproof} \newtheorem{partialproof}[theorem]{Proof.}
%\DeclareMathOperator{\diff}{d}
\setbeamertemplate{navigation symbols}{}

\includeonlylecture{1}

\newcommand{\lect}[3]{
  \date{#1}
  \lecture[#1]{#2}{#3}
}

\setbeamertemplate{footline}
{
  \leavevmode%
  \hbox{%
  \begin{beamercolorbox}[wd=.333333\paperwidth,ht=2.25ex,dp=1ex,center]{author in head/foot}%
    \usebeamerfont{author in head/foot}\insertshortauthor
  \end{beamercolorbox}%
  \begin{beamercolorbox}[wd=.333333\paperwidth,ht=2.25ex,dp=1ex,center]{title in head/foot}%
    \usebeamerfont{title in head/foot}\insertshorttitle
  \end{beamercolorbox}%
  \begin{beamercolorbox}[wd=.333333\paperwidth,ht=2.25ex,dp=1ex,center]{date in head/foot}%
    \usebeamerfont{date in head/foot}\insertshortdate{}
  \end{beamercolorbox}}%
  \vskip0pt%
}

% If you have a file called "university-logo-filename.xxx", where xxx
% is a graphic format that can be processed by latex or pdflatex,
% resp., then you can add a logo as follows:

%\pgfdeclareimage[height=0.8cm]{logo}{bluelogo}
%\logo{\pgfuseimage{logo}}
\renewcommand{\Arcsin}{\arcsin}
\renewcommand{\Arccos}{\arccos}
\renewcommand{\Arccot}{\arccot}
\renewcommand{\Arctan}{\arctan}


\begin{document}

\AtBeginLecture{%

\title[\insertlecture]{FreeCalc}
\subtitle{\insertlecture}
\author[FreeCalc]{}
\institute[UMass Boston]{University of Massachusetts Boston}
\date{\insertshortlecture}
\begin{frame}
  \titlepage
\end{frame}
}%

% begin lecture
\lect{\today}{Sample}{1}
%\section{Integrals of form $\int R(x,\sqrt{ax^2+bx+c}) \diff x$, $R$ - rational function}
%%begin module Euler-substitution-intro
\begin{frame}
\frametitle{Integrals of form $\int R(x,\sqrt{ax^2+bx+c}) \diff x$, $R$ - rational function}
Let $R(x,y)$ be an arbitrary rational expression in two variables (quotient of polynomials in two variables).
\begin{question}
Can we integrate $\alert<10>{\displaystyle\int R(x,\sqrt{ax^2+bx+c})\diff x}$?
\end{question}
\begin{itemize}
\item<2-> Yes. We will learn how in what follows.
\item<3-> The algorithm for integration is roughly:
\begin{itemize}
\item<4-> Use linear substitution to transform to one of three integrals: 
\uncover<5->{$\int R(x, \sqrt{-x^2+1})\diff x$, } \uncover<6->{ $\int R(x, \sqrt{x^2+1})\diff x$, } \uncover<7->{$\int R(x, \sqrt{x^2-1}) \diff x$.}
\item<8-> Use Euler substitution to transform to rational function integral (no radicals).
\item<9-> Solve as previously studied.
\end{itemize}
\item<10,11-> We motivate why we need \alert<10>{such integrals later}; we promise they allow to compute ellipse area and the volume of a ball.
\end{itemize}

\end{frame}
%end module Euler-substitution-intro
%\subsection{Euler substitution}
%%begin module Euler substitution
\begin{frame}
\frametitle{Euler substitution}
\begin{itemize}
\item Using linear substitutions, radicals of form  $\sqrt{ay^2+by+c})$, $a\neq 0$, $b^2-4ac\neq 0$ can be transformed to (multiple of):
\begin{itemize}
\item $\sqrt{x^2+1}$ 
\item $\sqrt{-x^2+1}$
\item $\sqrt{x^2-1}$.
\end{itemize}
\item We already studied how to do that using completing the square. 
\end{itemize}
\end{frame}
\begin{frame}
Recall that a (real) linear substitution is a substitution of the form $u=px+q$, $p,q$- (real) constants.
\begin{example}
Use a linear substitution to transform $\sqrt{x^2+x+1}$ to a multiple of an expression of the form $\sqrt{u^2+1}$. 

\[
\begin{array}{rcl}
\sqrt{x^2+x+1}&=&\sqrt{ x^2+2\frac{1}{2}x +\frac{1}{4}\textbf{?}-\frac{1}{4}\textbf{?} +1} \\
&=& \sqrt{ \left(x+\frac{1}{2}\textbf{?} \right)^2-\textbf{?} }\\
&=&\sqrt{\frac{3}{4}\left( \frac{4}{3} \left(x+\frac{1}{2}\right)^2+1 \right)}\\
&=&\frac{\sqrt{3}}{2}\sqrt{\left(\frac{2}{\sqrt{3}}\left( x+\frac{1}{2}\right)\right)^2+1}\\
&=& \frac{\sqrt{3}}{2} \sqrt{u^2+1},
\end{array}
\]
where $u=\frac{2}{\sqrt{3}}\left( x+\frac{1}{2}\right) \textbf{?}=\frac{2\sqrt{3}}{3}x+\frac{\sqrt{3}}{3}$.
\end{example}
\vspace{5cm}
\end{frame}
\begin{frame}
Recall that a (real) linear substitution is a substitution of the form $u=px+q$, $p,q$- (real) constants.
\begin{example}
Use a linear substitution to transform $\sqrt{-2x^2+x+1}$ to a multiple of an expression of the form $\sqrt{-u^2+1}$. 

\[
\begin{array}{rcl}
\sqrt{-2x^2+x+1}&=&\sqrt{ -2\left(x^2-\frac{1}{2}x -\frac{1}{2}\right) } \\
&=& \sqrt{ -2\left(x^2-2\frac{1}{4}x +\frac{1}{16}-\frac{1}{16}-\frac{1}{2}\right) }\\
&=&\sqrt{-2\left(\left(x-\frac{1}{16}\right)^2-\frac{9}{16} \right)}\\
&=&\sqrt{\frac{9}{8}\left(-\frac{16}{9}\left(x-\frac{1}{16}\right)^2+1 \right)}\\
&=&\frac{3}{\sqrt{8}}\sqrt{-\left(\frac{4}{3}\left(x-\frac{1}{16}\right)\right)^2+1 }\\
&=&\frac{ 3}{\sqrt{8}} \sqrt{-u^2+1}
\end{array}
\]
where $u=\frac{4}{3}\left(x-\frac{1}{16}\right)  \textbf{?}=\frac{4}{3}x-\frac{1}{12}$.
\end{example}
\end{frame}
%end module Euler substitution.
%%begin module Euler-substitution-case-1
\begin{frame}
\frametitle{Euler substitution to handle $\sqrt{x^2+1} $}
\begin{itemize}
\item<1-> Suppose we want to integrate 
\[
\int R(x, \sqrt{x^2+1})\diff x\quad .
\]
\only<2-11>{
\item<2-> Substitute $\alert<11>{\sqrt{x^2+1}}= \alert<7>{x} +t \uncover<7->{=\alert<7>{ \frac12\left(\frac{1}{t}- t\right)}+t =\alert<11>{\frac12\left(\frac{1}{t}+ t\right)}} $\uncover<7->{.} 
\item<3-> How is this a substitution of $x$ via t?
\uncover<4->{ Square both sides: 
\[
\begin{array}{rcl}
x^2+1&=&x^2+2xt+t^2\\
\uncover<5->{2xt&=&1-t^2}\\
\uncover<6->{\alert<7,8,11>{x}&\alert<7,8,11>{=}& \alert<7,8,11>{ \frac12\left(\frac{1}{t}- t\right)}}
\end{array}
\]
}
\item<8-> Take differentials on both sides:
\[
\alert<8>{\alert<11>{ \diff x}=  \alert<9,10>{\frac12 \diff \left(\frac{1}{t}- t\right)}}  \uncover<9->{\alert<9,10,11>{=}} \only<9>{\alert<9>{\textbf{?}}} \uncover<10->{\alert<10,11>{ -\frac12\left(\frac{1}{t^2} +1\right) dt} } {~~~~~~~~~~~~~~~~~~~~~~~~~~~~~~~~~~~~~~~~~~~~~~~~~~~~~~~~~~~~~~~~~~}
\] 
}
\end{itemize}
\uncover<12->{
\begin{definition}
The Euler substitution for $\sqrt{x^2+1} $ is the substitution given by
\[
\begin{array}{rcl}
\displaystyle\alert<12>{\sqrt{x^2+1}}&\alert<12>{=}&\displaystyle \alert<12>{\frac12 \left(\frac1t +t\right)} \\
\displaystyle\alert<12>{x}&\alert<12>{=}&\displaystyle \alert<12>{ \frac12 \left(\frac{1}{t}- t\right)} \\
\displaystyle \alert<12>{\diff x}&\alert<12>{=}&\displaystyle \alert<12>{ -\frac12\left(\frac{1}{t^2}+1\right) dt}
\quad .
\end{array}
\]
\end{definition}
}
\vspace{10cm}
\end{frame}
%end module Euler-substitution-case-1
%%begin module Euler-substitution-case-2
\begin{frame}
\frametitle{Euler substitution to handle $\sqrt{-x^2+1} $}
\begin{itemize}
\item<1-> Suppose we want to integrate 
\[
\int R(x, \sqrt{-x^2+1})\diff x\quad .
\]
\only<2-15>{
\item<2-> Substitute $\alert<9,15>{\sqrt{-x^2+1}}=(1-\alert<11>{x})t  \uncover<11->{ =\left( 1 -\alert<11>{ \left(1-\frac{2}{ t^2+1} \right)} \right)t =\alert<15>{\frac{2t}{t^2+1}}}$. 
\item<3-> How is this a substitution of $x$ via t?
\uncover<4->{
\[
\begin{array}{rcll|l}
\sqrt{-x^2+1}&=&(1-x)t&&\text{square}\\
\uncover<5->{(1-x)(1+x)&=&(1-x)^2t^2&&\text{divide by }1-x}\\
\uncover<6->{1+x&=&(1-x)t^2}\\
\uncover<7->{xt^2+x &=&t^2-1}\\
\uncover<8->{x(t^2+1) &=&t^2-1 &&\text{divide by }t^2+1}\\
\uncover<9->{\alert<11,12,15>{x}&\alert<11,15>{=}& \frac{  t^2\uncover<10->{+1-1}-1}{t^2+1} \uncover<10->{ = \alert<11,12,15>{1 -\frac{2}{ t^2+1} }\quad .}}
\end{array}
\] 
\item<12-> 
\[
\alert<12,15>{\diff x =} \alert<12,13,14>{ \diff \left(1-\frac{2}{t^2+1}\right)} \uncover<13->{\alert<13,14,15>{=}}\only<13>{\alert<13>{\textbf{?}}} \uncover<14->{\alert<14,15>{ \frac{4t}{(1+t^2)^2}dt} \quad .}
 {~~~~~~~~~~~~~~~~~~~~~~~~~~~~~~~~~~~~~~~~~~~~~~~~~~~~}
\] 
}
}
\end{itemize}
\uncover<16->{
\begin{definition}
The Euler substitution for $\sqrt{-x^2+1}$ is the substitution given by:
\[
\begin{array}{rcl}
\displaystyle \alert<16>{\sqrt{-x^2+1}}&\alert<16>{=}& \displaystyle  \alert<16>{\frac{2t}{t^2+1}}\\
\displaystyle  \alert<16>{x}&\alert<16>{=}&\displaystyle  \alert<16>{1-\frac{2}{t^2+1}}\\
\displaystyle \alert<16>{\diff x }&\alert<16>{=} & \displaystyle  \alert<16>{\frac{4t}{(t^2+1)^2}dt }\quad .
\end{array}
\] 
\end{definition}
}
\vspace{5cm}
\end{frame}
%end module Euler-substitution-case-2
%%begin module Euler-substitution-case-3
\begin{frame}

\frametitle{Euler substitution to handle $\sqrt{x^2-1} $}

\begin{itemize}
\item<1-> Suppose we want to integrate 
\[
\int R(x, \sqrt{x^2-1})\diff x\quad .
\]
\only<1-14>{
\item<2-> Substitute $\alert<14>{\sqrt{x^2-1}=}(\alert<10>{x}-1)t\uncover<10->{= \left(\alert<10>{ 1+\frac{2}{t^2-1}} -1\right)t =\alert<14>{\frac{2t}{t^2-1}} .}$ 
\item<3-> How is this a substitution of $x$ via $t$? 
\[
\begin{array}{rcll|l}
\uncover<4->{\sqrt{x^2-1}&=&(x-1)t&&\text{square}}\\
\uncover<5->{(x-1)(x+1)  &=&(x-1)^2t^2&&  \text{divide by } x-1} \\
\uncover<6->{x+1 &=&(x-1)t^2}\\
\uncover<7->{x(t^2-1) &=&1+t^2}\\
\uncover<8->{\alert<10,11,14>{x}&\alert<10,11,14>{=}&\displaystyle \frac{t^2\uncover<9->{-1+1}+1}{t^2-1} \uncover<9->{=\alert<10,11, 14>{1+\frac{2}{t^2-1}}}}
\end{array}
\]
\item<11->
\[
\alert<11>{ \alert<14>{\diff x=} \alert<12,13>{\diff \left(1+\frac{2}{t^2-1} \right)}} \uncover<12->{\alert<12,13>{=}} \only<12>{\alert<12>{\textbf{?}}} \uncover<13->{\alert<13,14>{\frac{-4t}{(t^2-1)^2}\diff t} .} {~~~~~~~~~~~~~~~~~~~~~~~~~~~~~~~~~~~~~~~~~~~~~~~~}
\] 
}
\end{itemize}
\uncover<15->{
\begin{definition}
\[
\begin{array}{rcl}
\displaystyle \alert<15>{\sqrt{-x^2+1}}&\alert<15>{=}&\displaystyle \alert<15>{\frac{2t}{t^2-1} }\\
\displaystyle \alert<15>{x}&\alert<15>{=}&\displaystyle \alert<15>{ 1+\frac{2} {t^2-1} }\\
\displaystyle \alert<15>{\diff x}&\alert<15>{=}&\displaystyle \alert<15>{ \frac{-4t }{(t^2-1 )^2}dt }\quad .
\end{array}
\] 
\end{definition}
}
\vspace{5cm}
\end{frame}
%end module Euler-substitution-case-3
%%begin module Euler-substitution-theorem
\begin{frame}
\frametitle{Summary of Euler substitution}

\begin{theorem}
Let $R(z,w)$ be an arbitrary rational function in two variables. Every integral of the form $\int R(x, \sqrt{ax^2+bx+c}) \diff x$ can be integrated using square roots, rational functions, logarithms, the $\Arctan$ function, and their compositions.
\end{theorem}
\begin{proof}
\begin{itemize}
\item<1-> Using linear substitutions we transform to integral of the form
$\int R(x, \sqrt{x^2+1}) \diff x$, $\int R(x, \sqrt{-x^2+1}) \diff x$ or $\int R(x, \sqrt{x^2-1}) \diff x$\quad .
\item<2-> Using one of the substitutions $t=\sqrt{x^2+1}-x$, $t=\frac{\sqrt{-x^2+1}}{-x+1}$,  $t=\frac{\sqrt{x^2-1}}{x-1}$ the integral is transformed to the form $ \int \frac{P(t)}{Q(t)}\diff t$. 
\item<3-> That is a rational function integral and can be solved using rational functions, logarithms, and the $\Arctan $ function.
\item<4-> Substitution of $t$ via $x$ may introduce extra square root.
\end{itemize}
\end{proof}
\end{frame}
%end module Euler-substitution-theorem
\section{Integrals of form $\int R(\cos x,\sin x) \diff x$, $R$ - rational function}
%begin module trig-integrals-rationalizing-substitution
\begin{frame}
\frametitle{Integrals of the form $\int R(\cos \theta,\sin \theta) \diff \theta$, $R$}

Let $R$ be an arbitrary rational function in two variables (quotient of polynomials in two variables).
\begin{question}
Can we integrate $\int R(\cos \theta, \sin \theta)\diff \theta$?
\end{question}
\begin{itemize}
\item<2-> Yes. We will learn how in what follows.
\item<3-> The algorithm for integration is roughly:
\begin{itemize}
\item<4-> Apply the substitution $\theta=2\Arctan t$ to transform to integral of rational function.
\item<5-> Solve as previously studied.
\end{itemize}
\end{itemize}
\end{frame}

\begin{frame}
\frametitle{The rationalizing substitution $\theta= 2\Arctan t$}
\uncover<13->{
\noindent 
Let $R$- rational function in two variables. 
$\int R(\alert<14,21>{\cos \theta,\sin \theta} ) \alert<22>{\diff \theta} $
can be integrated via the  substitution $\alert<14,15,18,23, 26>{ \theta=2\arctan t} $.
\uncover<14->{ How does this transform \alert<14,21>{$\sin \theta$, $\cos\theta$}? }\uncover<22->{How does this transform $\alert<22>{\diff \theta} $?} \uncover<26->{\alert<26>{How is $t$ expressed via $\theta$?}}
\[
\begin{array}{rcl}
\uncover<14->{ \alert<14,21>{\sin\alert<15>{\theta}}} &\uncover<14->{=} &\displaystyle \uncover<15->{ \alert<16>{\sin (\alert<15>{ 2\Arctan t} )}} \uncover<16->{ \alert<16>{= \frac{2 \alert<17>{\tan\left( \Arctan t\right)} }{1 + {\alert<17>{\tan}}^2 \alert<17>{ \left(\Arctan t \right)}} }} \uncover<17->{\alert<21>{ = \frac{2\alert<17>{ t}}{1+ {\alert<17>{t }}^2}}}\\
\uncover<14->{\alert<14,21>{\cos \alert<18>{\theta}}  } &\uncover<14->{=} &\displaystyle \alert<19>{ \uncover<18-> {\alert<19>{ \cos (\alert<18>{2\Arctan t}) }} } \uncover<19->{ \alert<19>{= \frac{1-{\alert<20>{\tan} }^2 \alert<20>{ (\Arctan t)}}{1+ {\alert<20>{\tan}}^2 \alert<20>{ (\Arctan t) }}}} \uncover<20->{  \alert<21>{= \frac{1- {\alert<20>{ t}}^2 }{1 +{\alert<20>{t} }^2}}} \\
\only<22->{
\uncover<22->{
\alert<22,23>{\diff \theta}}&\uncover<23->{\alert<23>{=}}& \displaystyle \uncover<23->{ \alert<23,24,25>{2 \diff \left(\Arctan t\right)}}\uncover<24->{ \alert<24,25>{=  \uncover<25->{ \alert<25>{\frac{1}{ 1+t^2}}} \uncover<24->{ \uncover<24>{ \textbf{?}}} \diff t}}\\
\uncover<26->{\alert<26,27>{t}&\alert<26,27>{=}&}\displaystyle \uncover<27->{\alert<27>{\tan \left(\frac{\theta}{2}\right)} }
}
\end{array}
\]
}

\only<1-20>{
Recall the expression of $\sin (2z), \cos (2z)$ via $\tan z$:
\[
\begin{array}{rcl}
\uncover<1->{\alert<1,2,16>{\sin \left(2z\right)}} &\uncover<1->{\alert<1>{=}}&\displaystyle  \uncover<2->{ \alert<2>{2\sin z\cos z}} \uncover<3->{=\frac{2 \alert<5>{\sin z\cos z} \uncover<4->{\alert<4>{ \frac{1}{\alert<5>{ \cos^2z}}}}}{\alert<3,6>{( \cos^2z +\sin^2z) }\uncover<4->{\alert<4,6>{\frac{1}{\cos^2z}}}}} \uncover<5->{\alert<16>{= \frac{2\alert<5>{\tan z} }{ \alert<6>{ 1+ \tan^2z}} } \quad .}\\
\uncover<1->{\alert<7,8,19>{\cos (2z)} }& \uncover<7->{ \alert<7,8>{= }}&\displaystyle\uncover<8->{\alert<8>{ \cos^2z-\sin^2z}} \uncover<9->{= \frac{ \alert<11>{ \left(\cos^2 z-\sin^2 z\right) \uncover<10->{\alert<10>{ \frac{1}{ \cos^2z} }}}}{\alert<12>{ \alert<9>{\left(\cos^2z +\sin^2 z\right)} \uncover<10->{\alert<10>{ \frac{1}{ \cos^2z }}} }}} \uncover<11->{\alert<19>{ =\frac{\alert<11>{ 1-\tan^2 z} }{\alert<12>{1+\tan^2z}}} ~ .}
\end{array}
\]
}



\uncover<28->{ 
\begin{theorem}
The substitution given above  transforms $ \int R(\cos \theta, \sin\theta)\diff \theta$ to an integral of a rational function of $t$.
\end{theorem}
}
\vspace{10cm}
\end{frame}
%end module trig-integrals-rationalizing-substitution
%begin module trig-integrals-rationalizing-substitution-ex1
\begin{frame}
\begin{example}
\uncover<2->{Let $\alert<2,25>{\theta=2\arctan t}$, \alert<4>{$\cos \theta=\frac{1-t^2}{1+t^2}$}, \alert<3>{$\sin \theta=\frac{2t}{1+t^2}$}}\uncover<22->{, $\alert<22,24>{z= \frac{3}{\sqrt{5}} \left(t + \frac{1}{3} \right)}$.} 
\[
\begin{array}{rcl}
\displaystyle \int \frac{\alert<2>{ \diff \theta} }{ 2\alert<3>{\sin \theta} -\alert<4>{ \cos \theta} +5}
\only<1-16>{
\uncover<2->{&=& \displaystyle \int \frac{\alert<2>{ 2\diff t} }{\alert<2>{\alert<6,7,9>{(\alert<8>{1}+\alert<5>{t^2})}} \left(\alert<7>{ 2 \alert<3>{\frac{2 t}{ t^2+1}} } \alert<6,9>{-} \alert<4>{ \frac{(\alert<9>{ 1} \alert<6>{- t^2}) }{\alert<6,9>{1+t^2}}}+\alert<5,8>{5}\right)}} \\
\uncover<5->{ &=&\displaystyle \int \frac{\alert<10>{2} \diff t}{ \alert<5,6>{ \alert<10>{6}t^2} +\alert<7>{ \alert<10>{4} t} +\alert<8,9,10> {4}}}\\
\uncover<10->{&=&\displaystyle  \int \frac{\diff t}{\alert<10,11>{3}t^2+\alert<10>{2}t+\alert<10>{2}}}\\
\uncover<11->{ \uncover<12>{\alert<12>{\text{(complete square)}}} &=&\displaystyle \int \frac{\diff t}{ \alert<11>{3}\left(\alert<13>{ t^2+ 2t\frac{ 1}{\alert<11>{3}}} \uncover<12->{\alert<12>{\alert<13>{+ \frac{1}{9}} \alert<14>{-\frac{1}{9}}}} \alert<14>{+ \frac{ 2}{ \alert<11>{3}}} \right)}} \\
\uncover<13->{ &=& \displaystyle \frac{1}{3}\int\frac{\diff t}{\alert<13>{ \left(t+\frac{1}{3}\right)^2} + \alert<14,15>{ \frac{ 5}{9}}}} \\
}
\uncover<15->{&\alert<16,17>{=}&\alert<16,17>{\displaystyle \alert<18>{\frac{1}{3}} \int \frac{\diff t }{ \alert<15,18>{\frac{5}{9}} \left( \alert<15,19>{\frac{9}{5}} \left(t+ \frac{1}{3} \right)^2 +\alert<15>{1} \right)}}} {~~~~~~~~~~~~~~~~~~~~~~~~~~~~~~~} {~~~~~~~~~~~~~~~}\\
\only<17->{
\uncover<18->{&=&\displaystyle \alert<18>{\frac{3}{5}}\int \frac{\uncover<20->{\alert<20>{ \frac{\sqrt{5}}{3}}} \diff \left( \alert<22>{ \uncover<20->{\alert<20>{\frac{3}{ \sqrt{5}}}}\left( t \uncover<21->{\alert<21>{+\frac{1}{3}}}\right)} \right) }{\left(\left(\alert<22>{ \alert<19>{ \frac{3}{ \sqrt{5}}} \left( t+\frac{1}{3}\right)}\right)^{\alert<19>{2} }+1 \right)} }\\
\uncover<22->{ &=&\displaystyle \frac{\sqrt{5}}{5} \alert<23>{\int \frac{\diff \alert<22>{ z}}{{\alert<22>{ z}}^2+1}}}\\
\uncover<23->{&=& \displaystyle  \frac{\sqrt{5}}{5} \alert<23>{\Arctan \alert<24>{z}} +C} \\
\uncover<24->{ &=&\displaystyle \frac{ \sqrt{5}}{5}\Arctan \left(\alert<24>{ \frac{3}{ \sqrt{5}} \left(\alert<25>{ t}+\frac{1}{3} \right)} \right)+C}\\
\uncover<25->{&=&\displaystyle \frac{ \sqrt{5}}{5}\Arctan \left( \frac{3}{ \sqrt{5}} \left(\alert<25>{ \tan \left(\frac{\theta}{2} \right)}+\frac{1}{3} \right) \right)+C
}
}
\end{array}
\]

\end{example}
\vspace{5cm}

\end{frame}
%end module trig-integrals-rationalizing-substitution-ex1

 


\end{document}
