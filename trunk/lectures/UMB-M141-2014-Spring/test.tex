\documentclass%
%[handout]
{beamer}
% % % % % % % %
% % % % % % % %
% % % % % % % %
%IMPORTANT
%compiles with 
%pdflatex -shell-escape 
%IMPORTANT
% % % % % % % %
% % % % % % % %
% % % % % % % %
\mode<presentation>
{
\useinnertheme{rounded}
\useoutertheme{infolines}
\usecolortheme{orchid}
\usecolortheme{whale}
}

\usepackage[english]{babel}
\usepackage[latin1]{inputenc}
\usepackage[all,cmtip]{xy}
\usepackage{times}
\usepackage[T1]{fontenc}
\usepackage{../example-templates}
\usepackage{../pstricks-commands}

\usepackage{auto-pst-pdf}
\usepackage{pst-plot}
%\usepackage{pstricks-add} 

% Or whatever. Note that the encoding and the font should match. If T1
% does not look nice, try deleting the line with the fontenc.

\graphicspath{{../../modules/}}

\newtheoremstyle{partialproof}{3pt}{3pt}{}{}{}{.}{.5em}{}
\theoremstyle{partialproof} \newtheorem{partialproof}[theorem]{Proof.}
%\DeclareMathOperator{\diff}{d}
\newcommand{\diff}{\text{d}}
\setbeamertemplate{navigation symbols}{}

\includeonlylecture{1}

\newcommand{\lect}[3]{
  \date{#1}
  \lecture[#1]{#2}{#3}
}

\setbeamertemplate{footline}
{
  \leavevmode%
  \hbox{%
  \begin{beamercolorbox}[wd=.333333\paperwidth,ht=2.25ex,dp=1ex,center]{author in head/foot}%
    \usebeamerfont{author in head/foot}\insertshortauthor
  \end{beamercolorbox}%
  \begin{beamercolorbox}[wd=.333333\paperwidth,ht=2.25ex,dp=1ex,center]{title in head/foot}%
    \usebeamerfont{title in head/foot}\insertshorttitle
  \end{beamercolorbox}%
  \begin{beamercolorbox}[wd=.333333\paperwidth,ht=2.25ex,dp=1ex,center]{date in head/foot}%
    \usebeamerfont{date in head/foot}\insertshortdate{}
  \end{beamercolorbox}}%
  \vskip0pt%
}

% If you have a file called "university-logo-filename.xxx", where xxx
% is a graphic format that can be processed by latex or pdflatex,
% resp., then you can add a logo as follows:

%\pgfdeclareimage[height=0.8cm]{logo}{bluelogo}
%\logo{\pgfuseimage{logo}}
\renewcommand{\Arcsin}{\arcsin}
\renewcommand{\Arccos}{\arccos}
\renewcommand{\Arccot}{\arccot}
\renewcommand{\Arctan}{\arctan}


\begin{document}

\AtBeginLecture{%

\title[\insertlecture]{FreeCalc}
\subtitle{\insertlecture}
\author[FreeCalc]{}
\institute[UMass Boston]{University of Massachusetts Boston}
\date{\insertshortlecture}
\begin{frame}
  \titlepage
\end{frame}
}%

% begin lecture
\lect{\today}{Sample}{1}
% begin module partial-fractions-intro
\begin{frame}
\frametitle{Integration of Rational Functions by Partial Fractions}

A rational function is a function which can be written as ratio of polynomials, $\frac{P(x)}{Q(x)}$. To integrate any rational function, we will express $\frac{P(x)}{Q(x)}$ as a sum of simpler expressions called partial fractions. We start with an example. 

Put $2/(x-1)$ and $1/(x+2)$ over a common denominator:
\[
\alert<5>{\frac{2}{x-1} - \frac{1}{x+2} } = %
\uncover<2->{%
\frac{2(x+2) - (x-1)}{(x-1)(x+2)} = %
}%
\uncover<3->{%
\alert<5>{ \frac{x + 5}{x^2+x-2} }%
}%
\]

\uncover<4->{%
We can now solve the following integral:
\[
\int \alert<5>{ \frac{x+5}{x^2+x-2}}\diff x = %
\uncover<5->{%
\int \left(\alert<5>{\frac{2}{x-1} - \frac{1}{x+2}} \right) \diff x = %
}%
\uncover<6->{%
2\ln | x - 1| - \ln | x + 2| + C
}%
\]
}%

\end{frame}
% end module partial-fractions-intro

% begin module partial-fractions-long-division
\begin{frame}
\frametitle{Review of polynomial notation}
Consider a rational function
\[
f(x) = \frac{P(x)}{Q(x)}
\]
where $P$ and $Q$ are polynomials.  Recall that the degree of $P$ is the highest power of $x$ in $P$ that has a non-zero coefficient.  That is, if
\[
P(x) = a_nx^n + a_{n-1}x^{n-1} + \cdots + a_1x + a_0
\]
where $a_n \neq 0$, then the degree of $P$ is $n$, and we write deg$(P) = n$.
\end{frame}
\begin{frame}\frametitle{Ensure denominator degree > numerator degree}
\begin{itemize}
\item To compute a partial fraction decomposition we need that the degree of the fraction numerator be less than the degree of the denominator.
\item<2-> Therefore our first step is to transform $\frac{P(x)}{Q(x)}$ to

$\displaystyle \frac{\alertNoH{6}{P(x)}}{\alertNoH{7}{Q(x)} }= \alertNoH{8}{S(x)}+\frac{\alertNoH{9}{R(x)}}{\alertNoH{7}{Q(x)}} $
where $S(x), R(x), Q(x)$ are polynomials and $\deg R<\deg Q$.
\item<3-> This is done using polynomial long division.
\item<4-> We recall that to divide the \alertNoH{6}{dividend $P(x)$} by the \alertNoH{7}{divisor $Q(x)$} to get \alertNoH{8}{quotient $S(x)$} with \alertNoH{9}{remainder $R(x)$} means \uncover<5->{ to find polynomials  $S(x), R(x)$ such that \alertNoH{9}{$\deg R<\deg Q$} and
\[
\alertNoH{6} {P(x)} = \alertNoH{8}{S(x)} \alertNoH{7}{Q(x)} + \alertNoH{9}{ R(x)}
\]
}
\item<10-> We review polynomial long division on examples.
\end{itemize}
\end{frame}
% end module partial-fractions-long-division

% begin module partial-fractions-long-division-ex1
\begin{frame}
\begin{example} %[Example 1, p. 510]
Find $\int \frac{x^3 + x}{x - 1}\diff x$.
\begin{columns}[t]
\column{.45\textwidth}
\uncover<2->{%
\[
\begin{array}{r@{}r@{}c@{}r@{}c@{}c@{}c@{}r@{}r@{}}
& & & %
\uncover<4->{\alert<handout:0| 4-6,23>{x^2}} & %
\uncover<11->{\alert<handout:0| 11-13,23>{+}} & %
\uncover<11->{\alert<handout:0| 11-13,23>{ x}} & %
\uncover<17->{\alert<handout:0| 17-19,23>{+}} & %
\uncover<17->{\alert<handout:0| 17-19,23>{ 2}} & \\%
\cline{3-8}
\alert<handout:0| 3-6,10-13,16-19>{x} & %
\alert<handout:0| 5-6,12-13,18-19>{-1} & %
\Big) & %
\alert<handout:0| 3-4,7-8>{x^3} & %
 & %
 & %
\alert<handout:0| 9>{+} & %
\alert<handout:0| 9>{ x} & \\%
& & & %
\uncover<6->{\alert<handout:0| 6-8>{x^3}} & %
\uncover<6->{\alert<handout:0| 6-8>{-}} & %
\uncover<6->{\alert<handout:0| 6-8>{ x^2}} & %
&  & \\%
\cline{4-6}%
& & & %
&  & %
\uncover<8->{\alert<handout:0| 8,10-11,14-15>{x^2}} & %
\uncover<9->{\alert<handout:0| 9,14-15>{+}} & %
\uncover<9->{\alert<handout:0| 9,14-15>{ x}} & \\%
& & & %
&  & %
\uncover<13->{\alert<handout:0| 13-15>{x^2}} & %
\uncover<13->{\alert<handout:0| 13-15>{-}} & %
\uncover<13->{\alert<handout:0| 13-15>{ x}} & \\%
\cline{6-8}
& & & %
 & %
 & %
 & %
 & %
\uncover<15->{\alert<handout:0| 15-17,20-21>{2x}} & \\%
& & & %
 & %
 & %
 & %
 & %
\uncover<19->{\alert<handout:0| 19-21>{2x}} & %
\uncover<19->{\alert<handout:0| 19-21>{- 2}} \\%
\cline{8-9}%
& & & %
 & %
 & %
 & %
 & %
 & %
\uncover<21->{\alert<handout:0| 21,24>{2}} \\%
\end{array}
\]
}%

\only<handout:0| -4,10-11,16-17>{\uncover<3->{%
Divide %
}}%
\only<handout:0| 5-6,12-13,18-19>{%
Multiply %
}%
\only<handout:0| 7-8,14-15,20-21>{%
Subtract %
}%
\only<handout:0| 3-4>{%
$x^3$ %
}%
\only<handout:0| 10-11>{%
$x^2$ %
}%
\only<handout:0| 16-17>{%
$2x$ %
}%
\only<handout:0| 5-6>{%
$x^2$ %
}%
\only<handout:0| 12-13>{%
$x$ %
}%
\only<handout:0| 18-19>{%
$2$ %
}%
\only<handout:0| 7-8>{%
$x^3-x^2$ %
}%
\only<handout:0| 14-15>{%
$x^2-x$ %
}%
\only<handout:0| 20-21>{%
$2x-2$ %
}%
\only<handout:0| -4,10-11,16-17>{\uncover<3->{%
by %
}}%
\only<handout:0| 5-6,12-13,18-19>{%
by %
}%
\only<handout:0| 7-8,14-15,20-21>{%
from %
}%
\only<handout:0| 3-4,10-11,16-17>{%
$x$ %
}%
\only<handout:0| 5-6,12-13,18-19>{%
$x-1$ %
}%
\only<handout:0| 7-8>{%
$x^3$ %
}%
\only<handout:0| 14-15>{%
$x^2+x$ %
}%
\only<handout:0| 20-21>{%
$2x$ %
}%
\only<handout:0| 9>{%
Bring down the $x$%
}%
\invisible<1->{%
y%
}%
\column{.55\textwidth}
\begin{eqnarray*}
& & %
\uncover<22->{%
\int \frac{x^3 + x}{x - 1}\diff x %
}\\%
& \uncover<22->{ = } & %
\uncover<22->{%
\int \left( \alert<handout:0| 23>{x^2 + x + 2} + \frac{\alert<handout:0| 24>{2}}{x - 1}\right) \diff x
}\\%
& \uncover<25->{ = } & %
\uncover<25->{%
\frac{x^3}{3} + \frac{x^2}{2} + 2x %
}\\%
& & \uncover<25->{%
\qquad + 2\ln | x - 1 | + C %
}%
\end{eqnarray*}
\end{columns}
\end{example}
\end{frame}
% end module partial-fractions-long-division-ex1


\end{document}
