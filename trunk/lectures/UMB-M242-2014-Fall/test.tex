\documentclass%
%[handout]
{beamer}
% % % % % % % %
% % % % % % % %
% % % % % % % %
%IMPORTANT
%compiles with
%pdflatex -shell-escape
%IMPORTANT
% % % % % % % %
% % % % % % % %
% % % % % % % %
\mode<presentation>
{
\useinnertheme{rounded}
\useoutertheme{infolines}
\usecolortheme{orchid}
\usecolortheme{whale}
}

\usepackage[english]{babel}
\usepackage[latin1]{inputenc}
\usepackage{times}
\usepackage[T1]{fontenc}
\usepackage{ifthen}
\usepackage{amsmath}
\usepackage{amssymb}
\usepackage{cancel}
\usepackage{comment}
\usepackage{multirow}
\usepackage{psfrag}
\usepackage{rotating}
\usepackage{fp}
\usepackage{calc}
\usepackage{bm}
\usepackage[all,cmtip]{xy}
\RequirePackage{xstring}

%%%%%%%%%%%%%%%%%%%%%%%%%%%%%%%%%%%%%%%%%%
%
% List of commands in this document
%
%
% \logdiffbaseandexp
% \logdifftwouponedown
% \productrulefofx
% \quotientruley
% \limitradical  (broken)
% \limitsub
% \chainruley
% \chainrulefofx
% \chainruleStyleOne
% \chainruleStyleTwo
% \chainruleStyleThree
% \infinitelimit
% \limitfactor
% \newtonsmethod
% \constantmultiple
% \chainruletwice
% \youWillNotBeTested
% \optionalDisplay  %Dummy command needed for compatibility with Calculus notes.
% \Arcsin
% \Arccos
% \Arctan
% \Arccot
% \diff
%%%%%%%%%%%%%%%%%%%%%%%%%%%%%%%%%%%%%%%%%%

\newcommand{\diff}{{\normalfont \text{d}}}
\newtheorem{question}{Question}
\newtheorem{observation}{Observation}
\newtheorem{proposition}{Proposition}
\newtheorem{remark}{Remark}
\newcommand{\youWillNotBeTested}{\begin{frame}You will not be tested on the material in the following slide.\end{frame}}
\DeclareMathOperator{\Vol}{Vol}

\DeclareMathOperator{\Arcsin}{\sin^{-1}}
\DeclareMathOperator{\Arccos}{\cos^{-1}}
\DeclareMathOperator{\Arctan}{\tan^{-1}}
\DeclareMathOperator{\Arccot}{{\cot^{-1}}}
\DeclareMathOperator{\Arcsec}{{\sec^{-1}}}
\DeclareMathOperator{\Arccsc}{{\csc^{-1}}}
\DeclareMathOperator{\maclaurin}{{\normalfont{Mc}}}
\newcommand{\taylor}{{\normalfont{T}}}

\newcommand{\optionalDisplay}[1]{#1}
\renewcommand{\Im}{\mathrm{Im}}
\renewcommand{\Re}{\mathrm{Re}}

%\DeclareMathOperator{\Re}{Re}
%\DeclareMathOperator{\Im}{Im}
\newcommand{\fcv}[1]{{\bf #1}} %this command stands for freecalc Vector
\DeclareMathOperator{\curl}{\fcv{curl}}
\DeclareMathOperator{\divg}{div}
\DeclareMathOperator{\proj}{\fcv{proj}}
\DeclareMathOperator{\orth}{\fcv{orth}}
\DeclareMathOperator{\grad}{\fcv{grad}}
\newcommand{\RR}{{\mathbb{R}}}
\newcommand{\cR}{{\mathcal{R}}}
\newcommand{\cD}{{\mathcal{D}}}
\newcommand{\cP}{{\mathcal{P}}}
\newcommand{\fcUncoverAlert}[2]{\uncover<#1->{\alert<#1>{#2}}}
\newcommand{\alertNoH}[2]{\alert<handout:0|#1>{#2}}
\newcommand{\fcAnswerNoH}[2]{
\FPeval{\fcResult}{clip(#1-1)}
\uncover<handout:0|\fcResult>{\alert<handout:0|\fcResult>{\textbf{?} }} \uncover<handout:0| #1->{\alert<handout:0|#1>{\!\!\!#2}}
}
\newcommand{\fcAnswer}[2]{
\FPeval{\fcResult}{clip(#1-1)}
\uncover<handout:0|\fcResult>{\alertNoH{\fcResult}{\textbf{?} }} \uncover<#1->{\alertNoH{#1}{\!\!\!#2}}
}
\newcommand{\fcAnswerUncover}[3]{
\FPeval{\fcResult}{clip(#2-1)}
\uncover<handout:0|#1-\fcResult>{\alertNoH{\fcResult}{\textbf{?}}} \uncover<#2->{\alertNoH{#2}{\!\!\!#3}}
}
\newcommand{\fcAnswerUncoverNoH}[3]{
\FPeval{\fcResult}{clip(#2-1)}
\uncover<handout:0|#1-\fcResult>{\alertNoH{\fcResult}{\textbf{?}}} \uncover<handout:0|#2->{\alertNoH{#2}{\!\!\!#3}}
}

\newcommand{\fcQuestion}[2]{%
\FPeval{\fcResult}{clip(#1+1)}%
\uncover<#1->{\alertNoH{ #1,\fcResult}{#2}}%
}
\newcommand{\fcEvalToInt}[1]{\FPeval{\fcResult}{clip(#1)}\fcResult}
\newcommand{\refBad}[3]{%
\ifthenelse{\equal{#1}{??}}%
{#2}%
{#3}%
}%example usage: \refBad{\ref{eqMacLaurinDef}}{their definition}{their definition (Definition \ref{eqMacLaurinDef})}
\newcommand{\fcCancel}[2]{
\FPeval{\fcResult}{clip(#1-1)}
\only<handout:0|-\fcResult>{#2} \only<#1->{\alertNoH{#1}{\cancel{\alertNoH{0}{#2}}}}
\vphantom{\cancel{#2}}
}
%<-WARNING: the superflous-looking \alertNoH{0} is needed:
% for some unknown to me reason it causes LaTeX to add the correct amount of spacing.

%code blocks regular expression that replaces all strings of the form \alert<handout:0| a> by \alertNoH{a}:
%Find:
%\\alert<[^|^0]*0|\([^>]*\)>
%Replace:
%\\alertNoH{\1}
%code blocks regular expression that replaces all strings of the form \alert<a> but not containing | by \alertNoH{a}:
%Find:
%\\alert<\([^|^>]*\)>
%Replace:
%\\alertNoH{\1}


\newcommand{\fcLicense}{
\begin{frame}
\frametitle{License to use and redistribute}
These lecture slides and their $\LaTeX${} source code are licensed to you under the Creative Commons license CC BY 3.0. You are free
\begin{itemize}
\item to Share - to copy, distribute and transmit the work,
\item to Remix - to adapt, change, etc., the work,
\item to make commercial use of the work,
\end{itemize}
as long as you reasonably acknowledge the original project (a notice of use freecalc is sufficient).
\begin{itemize}
\item Latest version of the .tex sources of the slides: \url{https://sourceforge.net/p/freecalculus/code/HEAD/tree/}
\item Should the link be outdated/moved, search for  ``freecalc project''.
\item Creative Commons license CC BY 3.0:
\url{https://creativecommons.org/licenses/by/3.0/us/}
and the links therein.
\end{itemize}
\end{frame}
}


\newcommand{\onlyNoH}[2]{\only<handout:0|#1>{#2}}
%
%  An example of logarithmic differentiation of a function with a
%  variable base and exponent.
%  #1 is the base.
%  #2 is the exponent.
%  #3 is the derivative of the natural logarithm of the base.
%  #4 is the derivative of the exponent.
%  #5 is (base)(exponent)' + (exponent)(base)' after simplification.
%
\newcommand{\logdiffbaseandexp}[5]{
\begin{example}[Variable base and exponent]
\abovedisplayskip=0pt
\belowdisplayskip=0pt
\abovedisplayshortskip=0pt
\belowdisplayshortskip=0pt
\begin{align*}
\text{Differentiate}\quad \alertNoH{ 13}{y} %
 & \alertNoH{ 13}{=} %
\alertNoH{ 13}{%
#1^{#2}%
}.%
\uncover<2->{%
\intertext{
Take logarithms of both sides:%
}
}%
\uncover<2->{%
\ln y
}%
 & \uncover<2->{ = } %
\uncover<2->{%
\ln #1^{\alertNoH{ 3}{#2}}%
}\\%
\uncover<3->{%
\alertNoH{ 4-5}{\ln y}%
}%
 & \uncover<3->{ = } %
\uncover<3->{%
\alertNoH{ 6-7}{%
\alertNoH{ 3}{#2} \ln #1%
}.}%
\uncover<4->{%
\intertext{
Differentiate implicitly with respect to $x$:%
}%
}%
\fcAnswer{5}{\frac{1}{y} y'}%
 & \uncover<4->{ = } %
\fcAnswerUncover{4}{7}{%
\left( #2 \right) \alertNoH{ 8-9}{\frac{\diff}{\diff x} \left( \ln #1 \right)} + \left( \ln #1 \right)\alertNoH{ 10-11}{\frac{\diff}{\diff x}\left( #2 \right)} %
}\\%
\uncover<8->{%
\frac{1}{\alertNoH{12}{y}} y'%
}%
 & \uncover<8->{ = } %
\uncover<8->{%
( #2 ) \alertNoH{8-9}{\left( \fcAnswerUncover{8}{9}{ #3 }\right)} + \left( \ln #1 \right) \alertNoH{ 10-11}{ \left( \fcAnswerUncover{8}{11}{ #4 } \right) }
}\\%
\uncover<12->{%
y'%
}%
 & \uncover<12->{ = } %
\uncover<12->{%
\alertNoH{ 12-13}{y} \left( #5 \right)%
}\\%
 & \uncover<13->{ = } %
\uncover<13->{%
\alertNoH{ 13}{#1^{#2}} \left( #5 \right).%
}%
\end{align*}
\end{example}
}


%
%  An example of logarithmic differentiation of a function.
%  It looks as follows:
%
%  Differentiate y = (#1 #2)/#3.
%  Take logarithms of both sides:
%  ln y = ln((#1 #2)/#3)
%  ln y = ln#1 + ln#2 - ln#3
%  ln y = #4 + #5 - #6
%  Differentiate implicitly with respect to x:
%  (1/y)y' = #7 + #8 - #9
%  y' = y(#7 + #8 - #9)
%  y' = ((#1 #2)/#3)(#7 + #8 - #9)
%
\newcommand{\logdifftwouponedown}[9]{
\begin{example}[Logarithmic Differentiation%
]
\abovedisplayskip=0pt
\belowdisplayskip=0pt
\abovedisplayshortskip=0pt
\belowdisplayshortskip=0pt
\begin{align*}
\text{Differentiate}\quad \alertNoH{ 18}{y} %
 & \alertNoH{ 18}{=} %
\alertNoH{ 18}{%
\frac{#1 #2}{#3}%
}.%
\uncover<2->{%
\intertext{
Take logarithms of both sides:%
}
}%
\uncover<2->{%
\ln y
}%
 & \uncover<2->{ = } %
\uncover<2->{%
\ln \frac{\alertNoH{ 3-4}{#1}\alertNoH{ 5-6}{#2}}{\alertNoH{ 7-8}{#3}}%
}\\%
\uncover<2->{%
\ln y
}%
 & \uncover<2->{ = } %
\uncover<2->{%
\ln \alertNoH{ 3-4}{#1} + \ln \alertNoH{ 5-6}{#2} -  \ln \alertNoH{ 7-8}{#3}%
}\\%
\uncover<3->{%
\alertNoH{ 9-10}{\ln y}%
}%
 & \uncover<3->{ = } %
\uncover<3->{%
\alertNoH{ 3-4,11-12}{%
\left( \uncover<4->{#4}\right) %
}%
\alertNoH{ 5-6}{%
\uncover<6->{+} \alertNoH{ 13-14}{\left( \uncover<6->{#5}\right)} %
}%
\alertNoH{ 7-8}{%
\uncover<8->{-} \alertNoH{ 15-16}{\left( \uncover<8->{#6}\right)} %
}%
}%
\uncover<9->{%
\intertext{
Differentiate implicitly with respect to $x$:%
}%
}%
\uncover<10->{%
\alertNoH{ 10}{\frac{1}{\alertNoH{ 17}{y}} y'}%
}%
 & \uncover<9->{ = } %
\uncover<9->{%
\alertNoH{ 11-12}{\left( \uncover<12->{#7} \right)} + %
\alertNoH{ 13-14}{\left( \uncover<14->{#8} \right)} - %
\alertNoH{ 15-16}{\left( \uncover<16->{#9} \right)} %
}\\%
\uncover<17->{%
y'%
}%
 & \uncover<17->{ = } %
\uncover<17->{%
\alertNoH{ 17-18}{y} \left( #7 + #8 - #9 \right)%
}\\%
 & \uncover<18->{ = } %
\uncover<18->{%
\alertNoH{ 18}{\frac{#1 #2}{#3}} \left( #7 + #8 - #9 \right)%
}%
\end{align*}
\end{example}
}


%
%  An example of a derivative with the Product Rule, using the symbol f(x).
%  It looks as follows:
%
%  Differentiate f(x) = #1 #2.
%  Product Rule: f'(x) = (#1)(d/dx)(#2) + (#2)(d/dx)(#1)
%   = (#1)(#4) + (#2)(#3)
%   = #5.
%
%  #6 appears in the subtitle of the example.
%
\newcommand{\productrulefofx}[6]{%
\begin{example}[Product Rule%
\ifthenelse{\equal{#6}{0}}%
{}%
{, #6}%
]%
\abovedisplayskip=0pt
\belowdisplayskip=0pt
\abovedisplayshortskip=0pt
\belowdisplayshortskip=0pt
\begin{align*}
\text{Differentiate}\quad f(x) & = \alertNoH{2}{ #1}\alertNoH{3}{ #2.}\\%
\uncover<2->{%
\text{Product Rule:}\quad f'(x)%
}%
& \uncover<2->{%
 =  \alertNoH{ 6-7}{\frac{\diff}{\diff x}\left( \alertNoH{2}{#1} \right)}\left( \alertNoH{3}{#2} \right)+\left( \alertNoH{2}{#1} \right) \alertNoH{ 4-5}{\frac{\diff}{\diff x}\left( \alertNoH{3}{#2} \right)} %
}\\%
& \uncover<4->{%
 = \alertNoH{ 6-7}{\left( \fcAnswerUncover{4}{7}{#3} \right)}\left( #2 \right)+ \left( #1 \right) \alertNoH{ 4-5}{\left(\fcAnswer{5}{ #4 }\right)}  %
}\\%
& \uncover<8->{%
 = #5.%
}%
\end{align*}
\end{example}
}


%
%  An example of a derivative with the Constant Multiple Rule.
%  It looks as follows:
%
%  Find the derivative of #1 = #2.
%   #1 = (#3)(#4).
%   d#1/dx = (d/dx)((#3)(#4))
% Constant Multiple Rule: = (#3)(d/dx)(#4)
%   = (#3)(#5)
%   = #6.
%
%  #7 appears in the subtitle of the example.
%
\newcommand{\constantmultiple}[7]{%
\begin{example}[Constant Multiple Rule%
\ifthenelse{\equal{#7}{0}}%
{}%
{, #7}%
]%
\abovedisplayskip=0pt
\belowdisplayskip=0pt
\abovedisplayshortskip=0pt
\belowdisplayshortskip=0pt
\begin{align*}
\text{Find the derivative of}\quad #1 & = #2.\\%
\uncover<2->{%
#1 %
}%
& \uncover<2->{%
 = \left( #3\right)\left( #4\right).
}\\%
\uncover<3->{%
\frac{\diff #1}{\diff x} %
}%
& \uncover<3->{%
 = \frac{\diff}{\diff x}\left[ \alertNoH{ 4}{\left( #3\right)}\left( #4\right)\right]
}\\%
\uncover<4->{%
\text{Constant Multiple Rule:}\quad %
}%
& \uncover<4->{%
 =  \alertNoH{ 4}{\left( #3\right)}\alertNoH{ 5-6}{\frac{\diff}{\diff x}\left( #4\right)}
}\\%
& \uncover<5->{%
 =  \left( #3\right)\alertNoH{ 5-6}{\left( \fcAnswer{6}{#5}\right)}
}\\%
& \uncover<7->{%
 =  #6.
}%
\end{align*}
\end{example}
}


%
%  An example of a derivative with the Quotient Rule, using the symbol y.
%  It looks as follows:
%
%  Differentiate y = #1 / #2.
%  Quotient Rule: dy/dx = ((#2)(d/dx)(#1)-(#1)(d/dx)(#2))/(#2)^2
%   = ((#2)(#3)-(#1)(#4))/(#2)^2
%   = #5
%   = #6.
%
%  #7 appears in the subtitle of the example.
%
\newcommand{\quotientruley}[7]{%
\begin{example}[Quotient Rule%
\ifthenelse{\equal{#7}{0}}%
{}%
{, #7}%
]%
\abovedisplayskip=0pt
\belowdisplayskip=0pt
\abovedisplayshortskip=0pt
\belowdisplayshortskip=0pt
\begin{align*}
\text{Differentiate}\quad y & = \frac{\alertNoH{2}{ #1}}{\alertNoH{3}{#2}}.%
\uncover<2->{%
\intertext{Quotient Rule:}%
}%
%&\\%
\uncover<2->{%
\frac{\diff y}{\diff x}%
}%
& \uncover<2->{%
 = \frac%
{ \alertNoH{ 4-5}{\frac{\diff}{\diff x}\left( \alertNoH{2}{ #1} \right)}\left( \alertNoH{3}{#2} \right) - \left( \alertNoH{2}{#1} \right) \alertNoH{ 6-7}{\frac{\diff}{\diff x}\left( \alertNoH{3}{#2} \right)}}%
{\left( \alertNoH{3}{#2}\right)^2}%
}\\%
& \uncover<4->{%
 = \frac%
{\alertNoH{ 4-5}{\left(\fcAnswer{5}{ #3 }\right)}\left( #2 \right)  - \left( #1 \right) \alertNoH{ 6-7}{\left( \fcAnswerUncover{4}{7}{#4} \right)}}%
{\left( #2\right)^2}%
}\\%
& \uncover<8->{%
 = #5%
}\\%
& \uncover<9->{%
 = #6.%
}%
\end{align*}
\end{example}
}

%
%  An example of an indefinite integral with the Substitution Rule.
%  It looks as follows:
%
%  Find \int (#1, with nothing substituted for UU and VV).
%  Let u = #2
%  Then du = #3.
%  Therefore #4 = #5.
%  Substitute: \int (#1, with the alert command for u and du
%          substituted for UU and VV respectively)
%  = \int (#6, with the alert command for u and du substituted for UU and VV)
%  = (#7, with u substituted for UU) + C
%  = (#8, with #2 substituted for UU) + C
%
%  #9 appears in the subtitle of the example.
%
\newcommand{\subrule}[9]{%
\begin{example}[Substitution Rule%
\ifthenelse{\equal{#9}{0}}%
{}%
{, #9}%
]%
\abovedisplayskip=0pt
\belowdisplayskip=0pt
\abovedisplayshortskip=0pt
\belowdisplayshortskip=0pt
\begin{align*}
\text{Find}\quad \int %
 \noexpandarg\exploregroups\StrSubstitute{\StrSubstitute{#1}{UU}{3}}{VV}{6-7}\noexploregroups\expandarg. & \\%
\uncover<2->{%
\text{Let}\quad\alertNoH{ 2-3,8,13}{u}%
}%
& \uncover<2->{%
\alertNoH{ 2-3,8,13}{ = \uncover<3->{#2.}}%
}\\%
\uncover<4->{%
\text{Then}\quad \alertNoH{ 4-5}{\diff u}%
}%
& \uncover<4->{%
\alertNoH{ 4-5}{ = \uncover<5->{#3}}%
}\\%
\uncover<6->{%
\alertNoH{ 6-7,9}{#4}%
}%
& \uncover<6->{%
\alertNoH{ 6-7,9}{ = \uncover<7->{#5.}}%
}\\%
\uncover<8->{%
\text{Substitute:}\quad \int%
 \noexpandarg\exploregroups\StrSubstitute{\StrSubstitute{#1}{UU}{8}}{VV}{9}\noexploregroups\expandarg}%
& \uncover<8->{= \alertNoH{ 10-11}{\int\noexpandarg\exploregroups\StrSubstitute{\StrSubstitute{#6}{UU}{8}}{VV}{9}\noexploregroups\expandarg %
}}\\%
& \uncover<10->{\alertNoH{ 10-11}{%
 = \uncover<11->{\noexpandarg\exploregroups \StrSubstitute{#7}{UU}{\alertNoH{ 13}{u}}\noexploregroups\expandarg} \uncover<12->{\alertNoH{ 12}{+C}}%
}}\\%
& \uncover<13->{%
 = \noexpandarg\exploregroups \StrSubstitute{#8}{UU}{\alertNoH{ 13}{#2}}\noexploregroups\expandarg +C.%
}%
\end{align*}
\end{example}
}

%
%  An example of a definite integral with the Substitution Rule.
%  There are nine arguments to the function.  The ninth is a string of four
%  groups of the form {AA}{BB}{CC}{DD} where AA is the lower limit of
%  integration, BB is the upper limit of integration, CC is the lower limit
%  of integration with respect to u, and DD is the upper limit of integration
%  with respect to u.
%  It looks as follows:
%
%  Find \int_{AA}^{BB} (#1, with nothing substituted for UU and VV).
%  Let u = #2
%  Then du = #3.
%  #4 = #5.
%  When x = AA, u = CC.
%  When x = BB, u = DD.
%  Substitute: \int_{AA}^{BB} (#1, with the alert command for u and du
%          substituted for UU and VV respectively)
%  = \int_{CC}^{DD} (#6, with the alert command for u and du substituted for UU and VV)
%  = [#7, with u substituted for UU]_{CC}^{DD}
%  = #8.
%
%
\newcommand{\subruledefbounds}[9]{%
\begin{example}[Substitution Rule, Definite Integral%
]%
\abovedisplayskip=0pt
\belowdisplayskip=0pt
\abovedisplayshortskip=0pt
\belowdisplayshortskip=0pt
\begin{align*}
\text{Find}\quad \int%
_{\StrMid{#9}{1}{1}}%
^{\StrMid{#9}{2}{2}} %
 \noexpandarg\exploregroups\StrSubstitute{\StrSubstitute{#1}{UU}{3}}{VV}{6-7}\noexploregroups\expandarg. & \\%
\uncover<2->{%
\text{Let}\quad\alertNoH{ 2-3,8-12}{u}%
}%
& \uncover<2->{%
\alertNoH{ 2-3,8-12}{ = \uncover<3->{#2.}}%
}\\%
\uncover<4->{%
\text{Then}\quad \alertNoH{ 4-5}{\diff u}%
}%
& \uncover<4->{%
\alertNoH{ 4-5}{ = \uncover<5->{#3}}%
}\\%
\uncover<6->{%
\alertNoH{ 6-7,13}{#4}%
}%
& \uncover<6->{%
\alertNoH{ 6-7,13}{ = \uncover<7->{#5.}}%
}\\%
\uncover<8->{%
\alertNoH{ 8-9,14}{\text{When } x = \StrMid{#9}{1}{1}, \quad u }%
}%
& \uncover<8->{%
\alertNoH{ 8-9,14}{ = \uncover<9->{\StrMid{#9}{3}{3}.}}%
}\\%
\uncover<10->{%
\alertNoH{ 10-11,15}{\text{When } x = \StrMid{#9}{2}{2}, \quad u }%
}%
& \uncover<10->{%
\alertNoH{ 10-11,15}{ = \uncover<11->{\StrMid{#9}{4}{4}.}}%
}\\%
\uncover<12->{%
\text{Substitute:}\quad \int%
_{\alertNoH{ 14}{\StrMid{#9}{1}{1}}}%
^{\alertNoH{ 15}{\StrMid{#9}{2}{2}}} %
 \noexpandarg\exploregroups\StrSubstitute{\StrSubstitute{#1}{UU}{12}}{VV}{13}\noexploregroups\expandarg}%
& \uncover<12->{= \alertNoH{ 16-17}{{\int}%
_{\uncover<14->{\alertNoH{ 14}{
\StrMid{#9}{3}{3}}}}%
^{\uncover<15->{
\alertNoH{ 15}{
\StrMid{#9}{4}{4}}}} %
\noexpandarg\exploregroups\StrSubstitute{\StrSubstitute{#6}{UU}{12}}{VV}{13}\noexploregroups\expandarg %
}}\\%
& \uncover<16->{\alertNoH{ 16-17}{%
 = {\left[ \uncover<17->{%
\noexpandarg\exploregroups\StrSubstitute{#7}{UU}{u}\noexploregroups\expandarg %
}\right]}_{\StrMid{#9}{3}{3}}^{\StrMid{#9}{4}{4}}%
}}\\%
& \uncover<18->{%
 = #8.
}%
\end{align*}
\end{example}
}


%
%  An example of a definite integral with the Substitution Rule.
%  There are nine arguments to the function.  The ninth is a string of two
%  groups of the form {AA}{BB} where AA is the lower limit of
%  integration and BB is the upper limit of integration.
%  It looks as follows:
%
%  Find \int_{AA}^{BB} (#1, with nothing substituted for UU and VV).
%  Let u = #2
%  Then du = #3.
%  #4 = #5.
%  Substitute: \int (#1, with the alert command for u and du
%          substituted for UU and VV respectively)
%  = \int (#6, with the alert command for u and du substituted for UU and VV)
%  = #7, with u substituted for UU
%  = #8.
%  Therefore int_{AA}^{BB} (#1, with nothing substituted for UU and VV)
%      = [#8]_{AA}^{BB}
%  = #9.
%
%
\newcommand{\subruledefvar}[9]{%
\begin{example}[Substitution Rule, Definite Integral%
]%
\abovedisplayskip=0pt
\belowdisplayskip=0pt
\abovedisplayshortskip=0pt
\belowdisplayshortskip=0pt
\begin{align*}
\text{Find}\quad \int%
_{\StrMid{#9}{1}{1}}%
^{\StrMid{#9}{2}{2}} %
 \noexpandarg\exploregroups\StrSubstitute{\StrSubstitute{#1}{UU}{3}}{VV}{6-7}\noexploregroups\expandarg. & \\%
\uncover<2->{%
\text{Let}\quad\alertNoH{ 2-3,8,12}{u}%
}%
& \uncover<2->{%
\alertNoH{ 2-3,8,12}{ = \uncover<3->{#2.}}%
}\\%
\uncover<4->{%
\text{Then}\quad \alertNoH{ 4-5}{\diff u}%
}%
& \uncover<4->{%
\alertNoH{ 4-5}{ = \uncover<5->{#3}}%
}\\%
\uncover<6->{%
\alertNoH{ 6-7,9}{#4}%
}%
& \uncover<6->{%
\alertNoH{ 6-7,9}{ = \uncover<7->{#5.}}%
}\\%
\uncover<8->{%
\text{Substitute:}\quad \int%
 \noexpandarg\exploregroups\StrSubstitute{\StrSubstitute{#1}{UU}{8}}{VV}{9}\noexploregroups\expandarg}%
& \uncover<8->{= \alertNoH{ 10-11}{{\int}%
\noexpandarg\exploregroups\StrSubstitute{\StrSubstitute{#6}{UU}{8}}{VV}{9}\noexploregroups\expandarg %
}}\\%
& \uncover<10->{%
 \alertNoH{ 10-11}{ = \uncover<11->{%
\noexpandarg\exploregroups{\StrSubstitute{#7}{UU}{\alertNoH{ 12}{u}}}\noexploregroups\expandarg%
}}%
  \uncover<12->{%
 = \noexpandarg\exploregroups{\StrSubstitute{#7}{UU}{\alertNoH{ 12}{#2}}}\noexploregroups\expandarg.%
}%
}\\%
\uncover<13->{%
\text{Therefore}\quad \int%
_{\StrMid{#9}{1}{1}}%
^{\StrMid{#9}{2}{2}} %
 \noexpandarg\exploregroups\StrSubstitute{\StrSubstitute{#1}{UU}{0}}{VV}{0}\noexploregroups\expandarg}%
& \uncover<13->{%
 = \left[%
 \noexpandarg\exploregroups{\StrSubstitute{#7}{UU}{#2}}\noexploregroups\expandarg%
\right]%
_{\StrMid{#9}{1}{1}}%
^{\StrMid{#9}{2}{2}} %
}\\%
& \uncover<14->{%
 = #8.
}%
\end{align*}
\end{example}
}

%
%  An example of a derivative with the Chain Rule, using the symbol y.
%  It looks as follows:
%
%  Differentiate y = #1.
%  Let u = #2
%  Then y = #3
%  Chain Rule: dy/dx = (dy/du)(du/dx)
%  = (#4, with u substituted for UU)(#5)
%  = #6, with #2 substituted for UU
%
%  #7 appears in the subtitle of the example.
%
\newcommand{\chainruley}[7]{%
\begin{example}[Chain Rule%
\ifthenelse{\equal{#7}{0}}%
{}%
{, #7}%
]%
\abovedisplayskip=0pt
\belowdisplayskip=0pt
\abovedisplayshortskip=0pt
\belowdisplayshortskip=0pt
\begin{align*}
\text{Differentiate}\quad y & = #1.\\%
\uncover<2->{%
\text{Let}\quad\alertNoH{ 2-3,8-10}{u}%
}%
& \uncover<2->{%
\alertNoH{ 2-3,8-10}{ = \uncover<3-| handout:0>{#2.}}%
}\\%
\uncover<4->{%
\text{Then}\quad \alertNoH{ 6-7}{y}%
}%
& \uncover<4->{%
\alertNoH{ 6-7}{ = \uncover<4-| handout:0>{#3.}}%
}\\%
\uncover<5->{%
\text{Chain Rule:}\quad%
\frac{\diff y}{\diff x}%
}%
& \uncover<5->{%
 = \alertNoH{ 6-7}{\frac{\diff y}{\diff u}}%
\alertNoH{ 8-9}{\frac{\diff u}{\diff x}}%
}\\%
& \uncover<6->{%
 = \alertNoH{ 6-7}{\left( \uncover<7-| handout:0>{\noexpandarg\exploregroups\StrSubstitute{#4}{UU}{\alertNoH{ 10}{u}}\noexploregroups\expandarg}\right)}%
\alertNoH{ 8-9}{\left( \uncover<9-| handout:0>{#5}\right)}%
}\\%
& \uncover<10->{ = } \uncover<10-| handout:0>{%
 \noexpandarg\exploregroups \StrSubstitute{#6}{UU}{\alertNoH{ 10}{#2}}.\noexploregroups\expandarg%
}%
\end{align*}
\end{example}
}





%
%  An example of a derivative with the Chain Rule, using the symbol f(x).
%  It looks as follows:
%
%  Differentiate f(x) = #1.
%  Let h(x) = #2
%  Let g(x) = #3
%  Then f(x) = g(h(x))
%  f'(x) = g'(h(x))h'(x)
%  = (#4, with h(x) substituted for UU)(#5)
%  = #6, with #2 substituted for UU
%
%  #7 appears in the subtitle of the example.
%
\newcommand{\chainrulefofx}[7]{%
\begin{example}[Chain Rule%
\ifthenelse{\equal{#7}{0}}%
{}%
{, #7}%
]%
\abovedisplayskip=0pt
\belowdisplayskip=0pt
\abovedisplayshortskip=0pt
\belowdisplayshortskip=0pt
\begin{align*}
\text{Differentiate}\quad f(x) & = #1.\\%
\uncover<2->{%
\text{Let}\quad\alertNoH{ 2-3,9-11}{h(x)}%
}%
& \uncover<2->{%
\alertNoH{ 2-3,9-11}{ = \fcAnswerNoH{3}{#2.}}%
}\\%
\uncover<2->{%
\text{Let}\quad\alertNoH{ 4-5,7-8}{g(x)}%
}%
& \uncover<2->{%
\alertNoH{ 4-5,7-8}{ = \fcAnswerUncover{2}{5}{#3.}}%
}\\%
\uncover<2-| handout:0>{%
\text{Then}\quad f(x)%
}%
& \uncover<2-| handout:0>{%
 = g(h(x)).%
}\\%
\uncover<6-| handout:0>{%
\text{Chain Rule:}\quad%
f'(x)%
}%
& \uncover<6-| handout:0>{%
 = \alertNoH{ 7-8}{g'(h(x))}%
\alertNoH{ 9-10}{h'(x)}%
}\\%
& \uncover<7-| handout:0>{%
=}\uncover<7-| handout:0>{\alertNoH{ 7-8}{\left( \fcAnswerNoH{8}{\noexpandarg\exploregroups\StrSubstitute{#4}{UU}{\alertNoH{ 11}{h(x)}}\noexploregroups\expandarg}\right)}%
\alertNoH{ 9-10}{\left( \fcAnswerUncoverNoH{7}{10}{#5}\right)}%
}\\%
& \uncover<11-| handout:0>{=} \uncover<11-| handout:0>{%
 \noexpandarg \exploregroups \StrSubstitute{#6}{UU}{\alertNoH{ 11}{#2}}.\noexploregroups \expandarg%
}%
\end{align*}
\end{example}
}

%
%  Similar to chainrulefofx but in different style.
%  It looks as follows:
%
%  Recall the chain rule (...).
%******************************
%  Differentiate f(x) = #1.
%  h(x) = #2
%  Let g(u) = #3
%  Then g'(u)=#4
%  Then f(x) = g(u)
%  f'(x) = g'(u)h'(x)
%  = (#4, with h(x) substituted for UU)(#5)
%  = #6, with #2 substituted for UU
%
%  #7 appears in the subtitle of the example.
%
\newcommand{\chainruleStyleOne}[7]{%
{\renewcommand{\arraystretch}{1.2}
$
\begin{array}{rclll}
\alertNoH{1-}{\left(g(h(x))\right)'}&\alertNoH{1-}{=}&\alertNoH{1-}{g'(h(x))\cdot  h'(x)}&& \text{(notation 1)} {~~~~~~~~~~~~~~~~~~~~~~~~~~~~~~~~~~~~} \\
(g(u))'&\alertNoH{0}{=}&g'(u) u'&\text{where } u=h(x)& \text{(notation 2)}\\
\displaystyle\frac{\diff y}{\diff x} &\alertNoH{0}{=}& \displaystyle\frac{\diff y}{\diff u}  \frac{\diff u}{\diff x} &\text{where } y=g(u)& \text{(notation 3)}\quad.\\
\end{array}
$
}
\begin{example}[Chain Rule, Notation 1%
\ifthenelse{\equal{#7}{0}}%
{}%
{, #7}%
]%
\[
\begin{array}{rrcl}
\text{Differentiate } & f(x) & =& #1.\\%
\uncover<2->{%
\text{Let}&\alertNoH{2-3,9-11}{h(x)}%
}%
&\uncover<2-| handout:0>{\alertNoH{2-3, 9-11}{ = }} &\displaystyle \uncover<2-| handout:0>{%
\alertNoH{2-3,9-11}{ \fcAnswerNoH{3}{#2.}}%
}\\%
\uncover<2->{%
\text{Let}&\alertNoH{4-5,7-8}{g(u)}%
}
&\uncover<2->{\alertNoH{4-5,7-8}{=}}&\displaystyle
\uncover<2->{\alertNoH{4-5,7-8}{ \fcAnswerUncover{2}{5}{\uncover<5-| handout:0>{#3.}}}%
}\\%
\uncover<2-| handout:0>{%
\text{Then}& f(x)
}%
&\uncover<2-| handout:0>{{=}}&\uncover<2-| handout:0>{%
 g(h(x)).%
}\\%
\uncover<6->{%
\text{Chain Rule:} &
f'(x)%
}%
&\uncover<6->{=}& \uncover<6->{%
 \alertNoH{ 7-8}{g'(h(x))}%
\alertNoH{ 9-10}{h'(x)}%
}\\%
&&\uncover<7->{=}& \displaystyle
\uncover<7->{\alertNoH{ 7-8}{ \left( \fcAnswerUncoverNoH{7}{8}{\noexpandarg \exploregroups \StrSubstitute{#4}{UU}{\alertNoH{ 11}{h(x)}} \noexploregroups\expandarg}\right)}%
\alertNoH{ 9-10}{\left( \fcAnswerUncoverNoH{7}{10}{#5}\right)}%
}\\%
&&\uncover<11-| handout:0>{=}&\displaystyle \uncover<11-| handout:0>{%
 \noexpandarg \exploregroups \StrSubstitute{#6}{UU}{\alertNoH{ 11}{#2}}.\noexploregroups \expandarg%
}%
\end{array}
\]
\end{example}
}

%
%  Similar to chainrulefofx but in different style.
%  It looks as follows:
%
%  Recall the chain rule (...).
%******************************
%  Differentiate f(x) = #1.
%  Let u= #2
%  Let g(u) = #3
%  Then g'(u)=#4
%  Then f(x) = g(u)
%  f'(x) = g'(u)h'(x)
%  = (#4, with h(x) substituted for UU)(#5)
%  = #6, with #2 substituted for UU
%
%  #7 appears in the subtitle of the example.
%
\newcommand{\chainruleStyleTwo}[7]{%
{\renewcommand{\arraystretch}{1.2}
$
\begin{array}{rclll}
\alertNoH{0}{\left(g(h(x))\right)'}&\alertNoH{0}{=}&g'(h(x))  \cdot  h'(x)&& \text{(notation 1)} {~~~~~~~~~~~~~~~~~~~~} \\
\alertNoH{1-}{(g(u))'}&\alertNoH{1-}{=}&\alertNoH{1-}{g'(u) u'}&\text{where } u=h(x)& \text{(notation 2)}\\
\displaystyle\frac{\diff y}{\diff x} &\alertNoH{0}{=}& \displaystyle\frac{\diff y}{\diff u}  \frac{\diff u}{\diff x} &\text{where } y=g(u)& \text{(notation 3)}\quad.\\
\end{array}
$
}
\begin{example}[Chain Rule, Notation 2%
\ifthenelse{\equal{#7}{0}}%
{}%
{, #7}%
]%
\[
\begin{array}{rrcl}
\text{Differentiate } & f(x) & =& #1.\\%
\uncover<2->{%
\text{Let}&\alertNoH{2-3,9-11}{u}%
}%
&\uncover<2->{\alertNoH{2-3,9-11}{=}}&\displaystyle \uncover<2->{%
\alertNoH{2-3,9-11}{ \fcAnswerNoH{3}{#2.}}%
}\\%
\uncover<2->{%
\text{Let}&\alertNoH{4-5,7-8}{g(u)}%
}
&\uncover<2->{\alertNoH{4-5,7-8}{=}}&\displaystyle
\uncover<2->{\alertNoH{4-5,7-8}{\fcAnswerUncoverNoH{2}{5}{ #3.}}%
}\\%
\uncover<2->{%
\text{Then}& f(x)
}%
&\uncover<2->{{=}}&\uncover<2->{%
 g(u).%
}\\%
\uncover<6->{%
\text{Chain Rule:} &
f'(x)%
}%
&\uncover<6->{=}& \uncover<6->{%
 \alertNoH{ 7-8}{g'(u)}%
\alertNoH{ 9-10}{u'}%
}\\%
&& \uncover<7-|handout:0>{=}&\displaystyle \uncover<7-|handout:0>{\alertNoH{7-8}{\left( \fcAnswerUncoverNoH{7}{8}{\noexpandarg\exploregroups\StrSubstitute{#4}{UU}{\alertNoH{11}{u}}\noexploregroups\expandarg}\right)}%
\alertNoH{9-10}{\left( \fcAnswerUncoverNoH{7}{10}{#5}\right)}%
}\\%
&& \uncover<11-|handout:0>{ = }&\displaystyle \uncover<11-| handout:0>{%
 \noexpandarg \exploregroups \StrSubstitute{#6}{UU}{\alertNoH{11}{#2}}.\noexploregroups \expandarg%
}%
\end{array}
\]
\end{example}
}


%
%  Similar to chainrulefofx but in different style.
%  It looks as follows:
%
%  Recall the chain rule (...).
%******************************
%  Differentiate f(x) = #1.
%  h(x) = #2
%  Let g(u) = #3
%  Then f(x) = g(u)
%  f'(x) = g'(u)h'(x)
%  = (#4, with h(x) substituted for UU)(#5)
%  = #6, with #2 substituted for UU
%
%  #7 appears in the subtitle of the example.
%
\newcommand{\chainruleStyleThree}[7]{%
{\renewcommand{\arraystretch}{1.2}
$
\begin{array}{rclll}
\alertNoH{0}{\left(g(h(x))\right)'}&\alertNoH{0}{=}&g'(h(x))  \cdot  h'(x)&& \text{(notation 1)} {~~~~~~~~~~~~~~~~~~~~} \\
(g(u))'&\alertNoH{0}{=}&g'(u) u'&\text{where } u=h(x)& \text{(notation 2)}\\
\displaystyle\alertNoH{1-}{\frac{\diff y}{\diff x}}&\alertNoH{1-}{=}&\displaystyle\alertNoH{1-}{\frac{\diff y}{\diff u}  \frac{\diff u}{\diff x}} &\text{where } y=g(u)& \text{(notation 3)}\quad.\\
\end{array}
$
}
\begin{example}[Chain Rule, Notation 3%
\ifthenelse{\equal{#7}{0}}%
{}%
{, #7}%
]%
\[
\begin{array}{rrcl}
\text{Differentiate } & y & =& #1.\\%
\uncover<2->{%
\text{Let}&\alertNoH{2-3,9-11}{u}%
}%
&\uncover<2->{\alertNoH{2-3,9-11}{=}}& \displaystyle \uncover<2->{%
\alertNoH{2-3,9-11}{ \fcAnswerNoH{3}{#2.}}%
}\\%
\uncover<2->{%
\text{Then}&\alertNoH{4-5,7-8}{y}%
}
&\uncover<2->{\alertNoH{4-5,7-8}{=}}&\displaystyle
\uncover<2->{\alertNoH{4-5,7-8}{\fcAnswerUncoverNoH{2}{5}{ #3.}}%
}\\%
\uncover<6->{%
\text{Chain Rule:} &
\displaystyle \frac{\diff y}{\diff x}%
}%
&\uncover<6->{=}&\displaystyle  \uncover<6->{%
 \alertNoH{7-8}{\frac{\diff y}{\diff u}}%
\alertNoH{9-10}{\frac{\diff u}{\diff x}}%
}\\%
&& \uncover<7->{ =&\displaystyle  \alertNoH{7-8}{ \left( \fcAnswerUncoverNoH{7}{8}{\noexpandarg \exploregroups \StrSubstitute{#4}{UU}{\alertNoH{ 11}{u}} \noexploregroups\expandarg}\right)}%
\alertNoH{9-10}{\left( \fcAnswerUncoverNoH{7}{10}{#5}\right)}}%
\\%
&&\uncover<11->{=}&\displaystyle \uncover<11-| handout:0>{%
\noexpandarg \exploregroups \StrSubstitute{#6}{UU}{\alertNoH{ 11}{#2}}.\noexploregroups \expandarg%
}%
\end{array}
\]
\end{example}
}

%
%  An example of an infinite limit calculation.
%  There are nine arguments to the function.  The ninth is a string of six
%  plus and minus signs.  Let AA, BB, CC, DD, EE, and FF denote these plus
%  and minus signs.  Then the output of the function looks as follows:
%
%  Find lim_{x \to #1^AA} (#2, with x substituted for UU)/(#3, with x substituted for UU).
%  Plug in #1.
%  (#2, with (#1) substituted for UU)/(#3, with (#1) substituted for UU) = #4/0.
%  The numerator is non-zero and the denominator is zero.
%  Therefore the answer is DNE, infty, or -infty.
%  Factor: (#3, with x substituted for UU)/(#4, with x substituted for UU) = (#5 #6)/(#7 #8)
%  \to ((BB)(CC))/((DD)(EE))
%  = (FF).
%  Therefore lim_{x \to #1^AA} (#2, with x substituted for UU)/(#3, with x substituted for UU) = FF infty.
%
\newcommand{\infinitelimit}[9]{%
\begin{example}[Infinite Limit]%
\abovedisplayskip=0pt
\belowdisplayskip=0pt
\abovedisplayshortskip=0pt
\belowdisplayshortskip=0pt
\begin{align*}
\text{Find}\quad \lim_{x\to #1^{\StrMid{#9}{1}{1}}}
\frac%
{\noexpandarg\StrSubstitute{#2}{UU}{x}\expandarg}%
{\noexpandarg\StrSubstitute{#3}{UU}{x}\expandarg}%
& \\%
\uncover<2->{%
\text{Plug in $#1$:}\quad%
\frac%
{\alertNoH{ 2-3}{\noexpandarg\StrSubstitute{#2}{UU}{(#1)}\expandarg}}%
{\alertNoH{ 4-5}{\noexpandarg\StrSubstitute{#3}{UU}{(#1)}\expandarg}}%
}%
& \uncover<2->{= \frac{\fcAnswer{3}{#4}}{ \fcAnswerUncover{2}{5}{ 0}}}%
\uncover<6->
Therefore the answer is DNE, $\infty$, or $-\infty$.}
}%
\uncover<7->{%
\text{Factor:}\quad
}%
\uncover<7->{%
\lim_{x\to #1^{\StrMid{#9}{1}{1}}}%
\frac%
{\alertNoH{ 8-9}{\noexpandarg\StrSubstitute{#2}{UU}{x}\expandarg}}%
{\alertNoH{ 10-11}{\noexpandarg\StrSubstitute{#3}{UU}{x}\expandarg}}%
}%
& \uncover<8->{%
 = \lim_{x\to #1^{\StrMid{#9}{1}{1}}}%
\frac%
{%
\fcAnswer{9}{%
\alertNoH{ 12-13}{%
#5%
}%
\alertNoH{ 14-15}{%
#6%
}%
}%
}{%
\fcAnswerUncover{8}{11}{%
\alertNoH{ 16-17}{%
#7%
}%
\alertNoH{ 18-19}{%
#8%
}%
}%
}%
}\\%
& \uncover<12->{%
 \to \alertNoH{ 20-21}{\frac%
{%
\alertNoH{ 12-13}{( \fcAnswerUncover{12}{13}{%
\StrMid{#9}{2}{2}%
})}%
\alertNoH{ 14-15}{(\fcAnswerUncover{12}{15}{%
\StrMid{#9}{3}{3}%
})}%
}{%
\alertNoH{ 16-17}{(\fcAnswerUncover{12}{17}{%
\StrMid{#9}{4}{4}%
})}%
\alertNoH{ 18-19}{(\fcAnswerUncover{12}{19}{%
\StrMid{#9}{5}{5}%
})}%
}%
}%
}\\%
& \uncover<20->{\alertNoH{ 20-21}{ = \fcAnswer{21}{(\alertNoH{22}{ \StrMid{#9}{6}{6}})}}}\\%
\uncover<22->{%
\text{Therefore}\quad\lim_{x\to #1^{\StrMid{#9}{1}{1}}}%
\frac%
{\noexpandarg\StrSubstitute{#2}{UU}{x}\expandarg}%
{\noexpandarg\StrSubstitute{#3}{UU}{x}\expandarg}%
}%
& \uncover<22->{ = } \uncover<handout:0| 22->{ \alertNoH{ 22}{\StrMid{#9}{6}{6}}\infty.}
\end{align*}
\end{example}
}




%
%  An example of a limit calculation with factoring.
%
%  It looks as follows.
%
%  Find lim_{x \to #1} (#2, with x substituted for UU)/(#3, with x substituted for UU).
%  Plug in #1.
%  (#2, with (#1) substituted for UU)/(#3, with (#1) substituted for UU) = 0/0.
%  Zero over zero gives no information.
%  Factor: (#2, with x substituted for UU)/(#3, with x substituted for UU) = ((#4, with x substituted for UU) #6)/((#5, with x substituted for UU) #6)
%  = (#4, with x substituted for UU)/(#5, with x substituted for UU)
%  Plug in #1: = (#4, with (#1) substituted for UU)/(#5, with (#1) substituted for UU)
%  = #7
%  = #8
%
\newcommand{\limitfactor}[8]{%
\begin{example}[Limit with Factoring]%
\abovedisplayskip=0pt
\belowdisplayskip=0pt
\abovedisplayshortskip=0pt
\belowdisplayshortskip=0pt
\begin{align*}
\text{Find}\quad \lim_{x\to #1}
\frac%
{\noexpandarg\StrSubstitute{#2}{UU}{x}\expandarg}%
{\noexpandarg\StrSubstitute{#3}{UU}{x}\expandarg}%
& \\%
\uncover<2->{%
\text{Plug in $#1$:}\quad%
\frac%
{\alertNoH{2-3}{\noexpandarg\StrSubstitute{#2}{UU}{(#1)}\expandarg}}%
{\alertNoH{4-5}{\noexpandarg\StrSubstitute{#3}{UU}{(#1)}\expandarg}}%
}%
& \uncover<2->{%
= \frac%
{\fcAnswerUncoverNoH{2}{3}{0}}%
{\fcAnswerUncoverNoH{2}{5}{0}}%
}%
\uncover<6->{%
\intertext{Zero over zero is undefined, so we can't use direct substitution.}
}%
\uncover<7->{%
\text{Factor:}\quad%
\lim_{x\to #1} \frac%
{\alertNoH{8-9}{\noexpandarg\StrSubstitute{#2}{UU}{x}\expandarg}}%
{\alertNoH{10-11}{\noexpandarg\StrSubstitute{#3}{UU}{x}\expandarg}}%
}%
& \uncover<8->{%
 = \lim_{x\to #1} \frac%
{%
\fcAnswerUncoverNoH{8}{9}{%
(\noexpandarg\StrSubstitute{#4}{UU}{x}\expandarg)%
\fcCancel{12}{#6}%
}%
}{%
\fcAnswerUncoverNoH{8}{11}{%
(\noexpandarg\StrSubstitute{#5}{UU}{x}\expandarg)%
\fcCancel{12}{#6}%
}%
}%
}\\%
& \uncover<12->{%
 = \lim_{x\to #1} \frac%
{\uncover<handout:0| 12->{\noexpandarg\StrSubstitute{#4}{UU}{\alertNoH{ 13}{x}}\expandarg}}%
{\uncover<handout:0| 12->{\noexpandarg\StrSubstitute{#5}{UU}{\alertNoH{ 13}{x}}\expandarg}}%
}\\%
\uncover<13->{%
\text{Plug in $#1$:}\quad%
\lim_{x\to #1} \frac%
{\noexpandarg\StrSubstitute{#2}{UU}{x}\expandarg}%
{\noexpandarg\StrSubstitute{#3}{UU}{x}\expandarg}%
}%
& \uncover<13->{%
 = \frac%
{\uncover<handout:0| 13->{\noexpandarg\StrSubstitute{#4}{UU}{(\alertNoH{ 13}{#1})}\expandarg}}%
{\uncover<handout:0| 13->{\noexpandarg\StrSubstitute{#5}{UU}{(\alertNoH{ 13}{#1})}\expandarg}}%
}\\%
& \uncover<14->{%
= \uncover<handout:0| 14->{#7}%
}\\%
& \uncover<15->{%
= \uncover<handout:0| 14->{#8.}%
}%
\end{align*}
\end{example}
}




%
%  An example of a limit calculation with a conjugate radical.
%
%  It looks as follows.
%
%  Find lim_{x \to #1} (#2, with x substituted for UU)/(#3, with x substituted for UU).
%  Plug in #1.
%  (#2, with (#1) substituted for UU)/(#3, with (#1) substituted for UU) = 0/0.
%  Zero over zero gives no information.
%  Factor: (#2, with x substituted for UU)/(#3, with x substituted for UU) = ((#4, with x substituted for UU) #6)/((#5, with x substituted for UU) #6)
%  = (#4, with x substituted for UU)/(#5, with x substituted for UU)
%  Plug in #1: = (#4, with (#1) substituted for UU)/(#5, with (#1) substituted for UU)
%  = #7
%  = #8
%
\newcommand{\limitradical}[9]{%
\begin{example}[Limit with Conjugate Radical]%
\abovedisplayskip=0pt
\belowdisplayskip=0pt
\abovedisplayshortskip=0pt
\belowdisplayshortskip=0pt
\begin{align*}
& \text{Find}\quad \lim_{x\to #1}
\frac%
{\noexpandarg\StrSubstitute{#2}{UU}{x}\expandarg}%
{\noexpandarg\StrSubstitute{#3}{UU}{x}\expandarg}%
 \\%
\uncover<2->{%
& \text{Plug in $#1$:}\quad%
\frac%
{\alertNoH{ 2-3}{\noexpandarg\StrSubstitute{#2}{UU}{(#1)}\expandarg}}%
{\alertNoH{ 4-5}{\noexpandarg\StrSubstitute{#3}{UU}{(#1)}\expandarg}}%
}%
 \uncover<2->{%
= \frac%
{\uncover<3->{\alertNoH{ 3}{0}}}%
{\uncover<5->{\alertNoH{ 5}{0}}}%
}%
\uncover<6->{%
\intertext{Zero over zero gives no information.  Use a conjugate radical.}
}%
& \uncover<7->{%
\lim_{x\to #1} \frac%
{\noexpandarg\StrSubstitute{#2}{UU}{x}\expandarg}%
{\alertNoH{ 7-8}{\noexpandarg\StrSubstitute{#3}{UU}{x}\expandarg}}%
\cdot %
\frac%
{\uncover<8->{\alert<8>{\noexpandarg\StrSubstitute{#4}{UU}{x}\expandarg}}}%
{\uncover<8->{\alert<8>{\noexpandarg\StrSubstitute{#4}{UU}{x}\expandarg}}}%
}\\%
& \uncover<9->{%
 = \lim_{x\to #1} \frac%
{(\noexpandarg\StrSubstitute{#2}{UU}{x}\expandarg)%
\left(\noexpandarg\StrSubstitute{#4}{UU}{x}\expandarg\right)}%
{#5}%
}\\%
& \uncover<10->{%
 = \lim_{x\to #1} \frac%
{(\alert<11-12>{\noexpandarg\StrSubstitute{#2}{UU}{x}\expandarg})%
\left(\noexpandarg\StrSubstitute{#4}{UU}{x}\expandarg\right)}%
{\alert<13-14>{#6}}%
}\\%
\uncover<11->{%
\text{Factor:}\quad%
}%
& \uncover<11->{%
 = \lim_{x\to #1} \frac%
{\uncover<12->{\alert<12>{(\noexpandarg\StrSubstitute{#7}{UU}{x}\expandarg)(x-#1)}}%
\left(\noexpandarg\StrSubstitute{#4}{UU}{x}\expandarg\right)}%
{\uncover<14->{\alert<14>{(\noexpandarg\StrSubstitute{#8}{UU}{x}\expandarg)(x-#1)}}}%
}\\%
& \uncover<15->{%
 = \lim_{x\to #1} \frac%
{(\noexpandarg\StrSubstitute{#7}{UU}{x}\expandarg)%
\left(\noexpandarg\StrSubstitute{#4}{UU}{x}\expandarg\right)}%
{\noexpandarg\StrSubstitute{#8}{UU}{x}\expandarg}%
}\\%
\uncover<16->{%
\text{Plug in $#1$:}\quad%
}%
& \uncover<16->{%
 = \frac%
{(\noexpandarg\StrSubstitute{#7}{UU}{(#1)}\expandarg)%
\left(\noexpandarg\StrSubstitute{#4}{UU}{(#1)}\expandarg\right)}%
{\noexpandarg\StrSubstitute{#8}{UU}{(#1)}\expandarg}%
}\\%
& \uncover<17->{%
#9.
}%
\end{align*}
\end{example}
}


%
%  An example of a limit calculation with direct substitution.
%
%  It looks as follows.
%
%  Find lim_{x \to #1} (#2, with x substituted for UU)/(#3, with x substituted for UU).
%  Plug in #1.
%  (#2, with (#1) substituted for UU)/(#3, with (#1) substituted for UU) = 0/0.
%  Zero over zero gives no information.
%  Factor: (#2, with x substituted for UU)/(#3, with x substituted for UU) = ((#4, with x substituted for UU) #6)/((#5, with x substituted for UU) #6)
%  = (#4, with x substituted for UU)/(#5, with x substituted for UU)
%  Plug in #1: = (#4, with (#1) substituted for UU)/(#5, with (#1) substituted for UU)
%  = #7
%  = #8
%
\newcommand{\limitsub}[7]{%
\begin{example}[%
\ifthenelse{\equal{#6}{0}}%
{Limit in Which Direct Substitution Doesn't Work}%
{Limit with Direct Substitution}%
]%
\abovedisplayskip=0pt
\belowdisplayskip=0pt
\abovedisplayshortskip=0pt
\belowdisplayshortskip=0pt
\begin{align*}
\text{Find}\quad \lim_{x\to #1}
\frac%
{\noexpandarg\StrSubstitute{#2}{UU}{x}\expandarg}%
{\noexpandarg\StrSubstitute{#3}{UU}{x}\expandarg}%
& \\%
\uncover<2->{%
\text{Plug in $#1$:}\quad%
\frac%
{\alertNoH{ 2-3}{\noexpandarg\StrSubstitute{#2}{UU}{(#1)}\expandarg}}%
{\alertNoH{ 4-5}{\noexpandarg\StrSubstitute{#3}{UU}{(#1)}\expandarg}}%
}%
& \uncover<2->{%
= \frac%
{\uncover<3->{\alertNoH{ 3}{#4}}}%
{\uncover<5->{\alertNoH{ 5}{#5}}}%
}\\%
\ifthenelse{\equal{#6}{0}}%
{ }%
{&}%
\uncover<6->{%
\ifthenelse{\equal{#6}{0}}%
{\intertext{Dividing by zero is undefined, so we can't use direct substitution.}}%
{ = #7.}%
}%
\ifthenelse{\equal{#6}{0}}%
{ }%
{ \text{Therefore}= #7.}%
\end{align*}
\end{example}
}



%
%  An example Newton's Method.
%
%  It looks as follows.
%
%  Starting with x_1 = #1, find the third approximation x_3 to the root of the equation #2.
%
%  f(x) = (#3, with x substituted for UU).
%  f'(x) = (#4, with x substituted for UU).
%  Newton's Method: x_{n+1} = x_n - f(x_n)/f'(x_n) = x_n - (#3, with x_n substituted for UU)/(#4, with x_n substituted for UU).
%
%  x_2 = x_1 - (#3, with x_1 substituted for UU)/(#4, with x_1 substituted for UU)     x_3 = x_2 - (#3, with x_2 substituted for UU)/(#4, with x_2 substituted for UU)
%   = (#1) - (#3, with (#1) substituted for UU)/(#4, with (#1) substituted for UU)     = (#5) - (#3, with (#5) substituted for UU)/(#4, with (#5) substituted for UU)
%  = #5.      = #6.
%
\newcommand{\newtonsmethod}[8]{%
\begin{example}[Newton's Method%
\ifthenelse{\equal{#8}{0}}%
{}%
{, #8}%
]%
\ifthenelse{\equal{#7}{0}}%
{%
Starting with $x_1 = #1$, find the third approximation $x_3$ to the root of the equation $#2$.
}%
{#7}%
\abovedisplayskip=0pt
\belowdisplayskip=10pt
\abovedisplayshortskip=0pt
\belowdisplayshortskip=0pt
\begin{align*}
\uncover<2->{%
\alertNoH{ 2-3,7}{f(x)}%
& \alertNoH{ 2-3,7}{ = \uncover<3->{\noexpandarg \exploregroups \StrSubstitute{#3}{UU}{x}.\noexploregroups \expandarg}}%
}\\%
\uncover<4->{%
\alertNoH{ 4-5,8}{f'(x)}%
& \alertNoH{ 4-5,8}{ = \uncover<5->{\noexpandarg \exploregroups \StrSubstitute{#4}{UU}{x}.\noexploregroups \expandarg}}%
}\\%
\uncover<6->{%
\text{Newton's Method:}\quad %
x_{n+1} & = x_n - \frac{\alertNoH{ 7}{f(x_n)}}{\alertNoH{ 8}{f'(x_n)}}%
}
\uncover<7->{%
 = x_n - \frac%
{\alertNoH{ 7}{\noexpandarg \exploregroups \StrSubstitute{#3}{UU}{x_n}\noexploregroups \expandarg}}%
{\alertNoH{ 8}{\uncover<8->{\noexpandarg \exploregroups \StrSubstitute{#4}{UU}{x_n}\noexploregroups \expandarg}}}%
}
\end{align*}
\begin{align*}
\uncover<9->{%
x_2 %
}%
& \uncover<9->{%
 = \alertNoH{ 10}{x_1} - \frac%
{\noexpandarg \exploregroups \StrSubstitute{#3}{UU}{\alertNoH{ 10}{x_1}}\noexploregroups \expandarg}%
{\noexpandarg \exploregroups \StrSubstitute{#4}{UU}{\alertNoH{ 10}{x_1}}\noexploregroups \expandarg}%
}%
& \uncover<12->{%
x_3 %
}%
& \uncover<12->{%
 = \alertNoH{ 13}{x_2} - \frac%
{\noexpandarg \exploregroups \StrSubstitute{#3}{UU}{\alertNoH{ 13}{x_2}}\noexploregroups \expandarg}%
{\noexpandarg \exploregroups \StrSubstitute{#4}{UU}{\alertNoH{ 13}{x_2}}\noexploregroups \expandarg}%
}\\%
& \uncover<10->{%
 = \alertNoH{ 10}{(#1)} - \frac%
{\noexpandarg \exploregroups \StrSubstitute{#3}{UU}{\alertNoH{ 10}{(#1)}}\noexploregroups \expandarg}%
{\noexpandarg \exploregroups \StrSubstitute{#4}{UU}{\alertNoH{ 10}{(#1)}}\noexploregroups \expandarg}%
}%
& %
& \uncover<13->{%
 = \alertNoH{ 13}{(#5)} - \frac%
{\noexpandarg \exploregroups \StrSubstitute{#3}{UU}{\alertNoH{ 13}{(#5)}}\noexploregroups \expandarg}%
{\noexpandarg \exploregroups \StrSubstitute{#4}{UU}{\alertNoH{ 13}{(#5)}}\noexploregroups \expandarg}%
}\\%
& \uncover<11->{%
 = #5.%
}%
& %
& \uncover<14->{%
 = #6.
}%
\end{align*}
\end{example}
}


%
%  An example of a derivative using the Chain Rule twice, using dy/dx.
%  It looks as follows:
%
%  Differentiate: y = #1.
%		  dy\dx  = d\dx(#1)
%  Chain Rule:     = (#2) (d/dx)(#3)
%  Chain Rule:     = (#2)(#4) d/dx(#5)
%  #7 [optional]    = (#2)(#3)(#6)
%                             = (#8)
%                             = (#9)    [optional]
%

\newcommand{\chainruletwice}[9]{%
\begin{example}[Using the Chain Rule twice]%
\abovedisplayskip=0pt
\belowdisplayskip=0pt
\abovedisplayshortskip=0pt
\belowdisplayshortskip=0pt
\begin{align*}
\text{Differentiate:}\quad y & = #1.\\%
\uncover<2->{\frac{\diff y}{\diff x} & = \alertNoH{3-5}{\frac{\diff}{\diff x}\left( #1\right)}}\\%
\uncover<4->{\text{Chain Rule:} \ \ \quad &= \alertNoH{4-5}{\left(\fcAnswerNoH{5}{#2} \right)\alertNoH{6-8}{\frac{\diff}{\diff x} \left(\uncover<4-| handout:0>{#3}\right)}}} \\%
\uncover<7->{\text{Chain Rule:} \ \ \quad &= \left(\uncover<7-| handout:0>{#2}\right) \alertNoH{7-8}{\left(\fcAnswerNoH{8}{#4}\right) \alertNoH{9-10}{\frac{\diff}{\diff x}\left( \uncover<7-| handout:0>{#5} \right)}}}\\%
\uncover<9->{\uncover<10->{\ifthenelse{\equal{#7}{}}{}{\text{#7 :} \ \ \quad}}& = \left(\uncover<9-| handout:0>{#2} \right) \left(\uncover<9-| handout:0>{#4}\right)\alertNoH{9-10}{\left( \fcAnswerNoH{10}{#6} \right) }} \\%
\uncover<11->{& = \uncover<11-| handout:0>{#8 \ifthenelse{\equal{#9}{}}{.}{\\}}}%
\ifthenelse{\equal{#9}{}}{}{\uncover<12->{& = \uncover<12-| handout:0>{#9.}}}
\end{align*}
\end{example}
}

\ProvidesPackage{pstricks-commands}
\usepackage{etex, ifthen}
\usepackage{auto-pst-pdf}
\usepackage{pst-plot}
\usepackage{pst-math}
%WARNING THE FOLLOWING PACKAGE IS BROKEN use only with EXTREME CAUTION
%\usepackage{pst-3dplot}

\newcommand{\fcXLabel}{$x$}
\newcommand{\fcYLabel}{$y$}
\newcommand{\fcZLabel}{$z$}
\newcommand{\fcDelta}{0.5}
\newcommand{\fcStartXIId}{0}
\newcommand{\fcStartYIId}{0}
\newcommand{\fcIterationsX}{9\space}
\newcommand{\fcIterationsY}{9\space}
\newcommand{\fcScreenStyle}{z}
\newcommand{\fcLineColor}{black}
\newcommand{\fcArrows}{}

\makeatletter %needed for define@key command.
\define@key{pstricks,pst-plot}{xLabel}[]{}
\define@key{pstricks,pst-plot}{yLabel}[]{}
\define@key{pstricks,pst-plot}{zLabel}[]{}
\define@key{fcGraphics}{Delta}[\renewcommand{\fcDelta}{1}]{\renewcommand{\fcDelta}{#1}}
\define@key{fcGraphics}{startX}[\renewcommand{\fcStartXIId}{0}]{\renewcommand{\fcStartXIId}{#1}}
\define@key{fcGraphics}{startY}[\renewcommand{\fcStartYIId}{0}]{\renewcommand{\fcStartYIId}{#1}}
\define@key{fcGraphics}{iterationsX}[\renewcommand{\fcIterationsX}{9\space}]{\renewcommand{\fcIterationsX}{#1\space}}
\define@key{fcGraphics}{iterationsY}[\renewcommand{\fcIterationsY}{9\space}]{\renewcommand{\fcIterationsY}{#1\space}}
\define@key{fcGraphics}{screenStyle}[\renewcommand{\fcScreenStyle}{z}]{\renewcommand{\fcScreenStyle}{#1}}
\define@key{fcGraphics}{xLabel}[\renewcommand{\fcXLabel}{$x$}]{\renewcommand{\fcXLabel}{#1}}
\define@key{fcGraphics}{yLabel}[\renewcommand{\fcYLabel}{$y$}]{\renewcommand{\fcYLabel}{#1}}
\define@key{fcGraphics}{zLabel}[\renewcommand{\fcZLabel}{$z$}]{\renewcommand{\fcZLabel}{#1}}
\define@key{fcGraphics}{linecolor}[\renewcommand{\fcLineColor}{black}]{\renewcommand{\fcLineColor}{#1}}
\define@key{fcGraphics}{arrows}[\renewcommand{\fcArrows}{}]{\renewcommand{\fcArrows}{#1}}
\makeatother %undoes \makeatletter.


\newcommand{\fcHollowDot}[2]{
\pscircle*[fillcolor=white, linecolor=red](#1, #2){0.07}
\pscircle*[fillcolor=white, linecolor=white](#1, #2){0.04}
}

\newcommand{\fcFullDot}[3][linecolor=red]{
\pscircle*[#1](! #2 #3){0.07}
}

\newcommand{\fcHollowDotBlue}[2]{
\pscircle*[fillcolor=white, linecolor=blue](#1, #2){0.07}
\pscircle*[fillcolor=white, linecolor=white](#1, #2){0.04}
}
\newcommand{\fcFullDotBlack}[2]{
\pscircle*[fillcolor=white, linecolor=black](#1, #2){0.07}
}
\newcommand{\fcFullDotBlue}[2]{
\pscircle*[fillcolor=white, linecolor=blue](#1, #2){0.07}
}
\newcommand{\fcXTickColored}[2]{\psline[linecolor=#1](#2, -0.1)(#2,0.1)}

\newcommand{\fcXTick}[1]{\psline(#1, -0.1)(#1,0.1)}
\newcommand{\fcYTick}[1]{\psline(-0.1, #1)(0.1, #1)}
\newcommand{\fcXYTick}[2]{\fcXTick{#1} \fcYTick{#2}}

\newcommand{\fcXTickWithLabel}[2]{\fcXTick{#1}\rput[t](#1,-0.2){#2}}
\newcommand{\fcYTickWithLabel}[2]{\fcYTick{#1}\rput[r](-0.2,#1){#2}}

\newcommand{\fcLabelNumberXaxis}[1]{\fcXTickWithLabel{#1}{#1}}
\newcommand{\fcLabelNumberYaxis}[1]{\fcYTickWithLabel{#1}{#1}}

\newcommand{\fcLabelNumberXYaxes}[2]{\fcLabelNumberXaxis{#1} \fcLabelNumberYaxis{#2} }

\newcommand{\fcLabelXOne}{\fcLabelNumberXaxis{1} }
\newcommand{\fcLabelYOne}{\fcLabelNumberYaxis{1} }

\newcommand{\fcLabelOnXaxis}[2]{\fcXTick{#1}\rput[t](#1,-0.2){#2}}
\newcommand{\fcLabelOnYaxis}[2]{\fcYTick{#1}\rput[r](-0.2, #1){#2}}

\newcommand{\fcLabels}[1][$x$]{%
  \def\ArgpsXAxisLabel{{#1}}%
  \fcLabelsRelay
}
\newcommand\fcLabelsRelay[3][$y$]{\rput[t](! #2 -0.1){\ArgpsXAxisLabel}\rput[r](! -0.1 #3){#1}}

\newcommand{\fcLabelsWithOnes}[2]{\psline(1, -0.1)(1,0.1) \rput[t](1, -0.2 ) { $1$} \psline(-0.1, 1)(0.1, 1) \rput[r](-0.2, 1 ) { $1$} \fcLabels{#1}{#2}}

\newcommand{\fcDefaultXLabel}{$x$}
\newcommand{\fcDefaultYLabel}{$y$}

\newcommand{\fcBoundingBox}[4]{%
\psframe*[linecolor=white](! #1\space #2)(! #3\space #4)%
\psline[linecolor=black!1](! #1 #2 )(! #1 #2 0.01 add)%
\psline[linecolor=black!1](! #3 #4 )(! #3 #4 0.01 add)%
}
\newcommand{\fcAxesStandardNoFrame}[4]{%
\psaxes[ticks=none, labels=none]{<->}(0,0)(#1,#2)(#3,#4) \fcLabels[\fcDefaultXLabel][\fcDefaultYLabel]{#3}{#4}
}%

\newcommand{\fcAxesStandard}[4]{%
\psframe*[linecolor=white](! #1\space #2)(! #3 \space 0.1 add #4 \space 0.1 add)%
\fcAxesStandardNoFrame{#1}{#2}{#3}{#4}
}%
\newcommand{\fcColorTangent}{blue}
\newcommand{\fcColorGraph}{red}
\newcommand{\fcColorAreaUnderGraph}{cyan}
\newcommand{\fcColorNegativeAreaUnderGraph}{orange}

\newcommand{\fcMachine}[2]{
\pscustom*[linecolor=#2]{
\psline(1,1.1)(1,0.1)(1.5,0.1)(2, 0.6)(2.5, 0.6)(2.5, -0.6)(2, -0.6)(1.5,-0.1)(1,-0.1)(1,-1.1)(-1,-1.1)(-1,-0.1)(-1.5,-0.1)(-2, -0.6)(-2.5, -0.6)(-2.5, 0.6)(-2, 0.6)(-1.5,0.1)(-1,0.1)(-1,1.1)
}
\pscircle*[linecolor=white](0,0){0.3}
\rput(0,0){#1}
}

%command format
%first argument gives you formula for the direction field in
%postscript notation, for example x y add.
%second and third argument give the starting x,y coordinates
\newcommand{\fcDirectionFieldOneTangent}[6]{%
\pstVerb{%
3 dict begin%
/x #2 \space def%
/y #3 \space def%
/F #1 \space def%
}%
\psline[#6](! x F ATAN 57.295 mul cos #4 mul sub y F ATAN 57.295 mul sin #4 mul sub)(! x F ATAN 57.295 mul cos #4 mul add y F ATAN 57.295 mul sin #4 mul add)%
\pscircle*[linecolor=red!60](! x y){#5}%
\pstVerb{%
end%
}%
}

\newcommand{\fcDirectionFieldOneTangentDefault}[3]{%
\fcDirectionFieldOneTangent{#1}{#2}{#3}{0.3}{0.03}{linecolor=blue}%
}

%command format
%first argument gives you formula for the direction field in
%postscript notation, for example x y add.
%second and third argument give the starting x,y coordinates
%fourth coordinate gives the delta x=delta y
%fifth argument gives the number of iterations delta x
%sixth argument gives the number of iterations delta y
%seventh argument gives the length of the vector
%eighth  argument gives the circle radius
%ninth argument gives the arguments of the psline command
\newcommand{\fcDirectionFieldFull}[9]{%
\multido{\ra=#2+#4}{#5}{%
\multido{\rb=#3+#4}{#6}{%
\fcDirectionFieldOneTangent{#1}{\ra}{\rb}{#7}{#8}{#9}%
}%end multido
}%end multido
}%end newcommand

\newcommand{\fcDirectionFieldDefault}[5]{%
\fcDirectionFieldFull{#1}{#2}{#3}{#4}{#5}{#5}{0.2}{0.02}{linecolor=blue}%
}%
\newcommand{\fcDirectionFieldDefaultRange}[1]{%
\fcDirectionFieldFull{#1}{-4}{-4}{0.5}{21}{21}{0.2}{0.02}{linecolor=blue}%
}

\newcommand{\fcVectorProjectOntoVector}{%
\fcVectorNormalize dup 3 1 roll \fcVectorScalarVector \fcVectorTimesScalar%
} %

\newcommand{\fcAngleIIId}[4][]{%
\pstVerb{%
3 dict begin%
/firstV #2 \fcVectorNormalize def%
/orthonormalV #3 dup firstV  \fcVectorProjectOntoVector \fcVectorMinusVector \fcVectorNormalize def%
/theAngle firstV #3\space \fcVectorNormalize \fcVectorScalarVector arccos def%
}%
\parametricplot[#1]{0}{theAngle}{firstV t cos #4 mul \fcVectorTimesScalar orthonormalV t sin #4 mul \fcVectorTimesScalar \fcVectorPlusVector \fcGetDrawCoords}%
\pstVerb{end}%
}

\makeatletter
\newcommand{\fcAngle}[5][linecolor=\fcColorGraph]{%
\ifPst@algebraic{%
\parametricplot[#1, algebraic=true]{#2}{#3}{#4*cos(t)| #4*sin(t)}%
\rput(! #2\space #3\space add 2 div 57.29578 mul cos #4\space 0.2 add mul #2\space #3\space add 2 div 57.29578 mul sin #4\space 0.2 add mul){#5}%
}%
\else%
\parametricplot[#1, algebraic=false]{#2}{#3}{t 57.29578 mul cos #4\space mul t 57.29578 mul sin #4\space mul}%
\rput(! #2\space #3\space add 2 div 57.29578 mul cos #4\space 0.2 add mul #2\space #3\space add 2 div 57.29578 mul sin #4\space 0.2 add mul){#5}%
\fi%
}
\makeatother

\newcommand{\fcLengthIndicator}[5]{
\psline[arrows=<-, linecolor=red](! #1 #2)(! #1 0.58 mul #3 0.42 mul add #2 0.58 mul #4 0.42 mul add)
\psline[arrows=->, linecolor=red]{->}(! #1 0.42 mul #3 0.58 mul add #2 0.42 mul #4 0.58 mul add)(! #3 #4)
\rput(! #1 #3 add 0.5 mul #2 #4 add 0.5 mul){ #5}
}

\makeatletter
\newcommand{\fcDrawPolar}[4][linecolor=\fcColorGraph]{%
\ifPst@algebraic{%
\parametricplot[#1]{#2}{#3}{(#4) *cos(t) | (#4) * sin(t)}%
}%
\else%
\parametricplot[#1]{#2}{#3}{#4 t 57.29578 mul cos mul #4 t 57.29578 mul sin mul}%
\fi%
}
\makeatother

\newcommand{\fcPolarCurveEvaluateX}[2]{
1 dict begin /t #1 def #1 57.29578 mul cos #2 mul end
}

\newcommand{\fcPolarCurveEvaluateY}[2]{
1 dict begin /t #1 def #1 57.29578 mul sin #2 mul end
}

\newcommand{\fcPolarCurveEvaluateXY}[2]{
\fcPolarCurveEvaluateX{#1}{#2} \fcPolarCurveEvaluateY{#1}{#2}
}

\newcommand{\fcPolarWedge}[3]{%
\ifPst@algebraic{%
\rput(0,0){Set algebraic to FALSE}%
}%
\else%
\pstVerb{%
%/firstX 1 dict begin /t #1 def #1 57.29578 mul sin #2 mul end def%
/firstX \fcPolarCurveEvaluateX{#1}{#3} def%
/firstY \fcPolarCurveEvaluateY{#1}{#3} def%
/secondX \fcPolarCurveEvaluateX{#2}{#3} def%
/secondY \fcPolarCurveEvaluateY{#2}{#3} def%
}%
\pscustom[fillcolor=\fcColorAreaUnderGraph, fillstyle=solid, linecolor=blue]{%
\psline(0,0)(! \fcPolarCurveEvaluateXY{#1}{#3} )(! \fcPolarCurveEvaluateXY{#2}{#3})(0,0)%
}%
\fi%
}%

\newcommand{\fcPolarWedgeSequence}[4]{%
\multido{\ra=#1+#2}{#3}{%
\fcPolarWedge{\ra}{\ra\space #2 add}{#4}
}%
}

\newcommand{\fcRegularNgon}[3][linecolor=\fcColorGraph]{%
\multido{\ra=0+1}{#2}{%
\psline[#1](! \ra \space #2 div 360 mul cos #3 mul \ra \space #2 div 360 mul sin #3 mul)(! \ra \space 1 add #2 div 360 mul cos #3 mul \ra \space 1 add #2 div 360 mul sin #3 mul)%
}%end multido
}

\newcommand{\fcEvaluateT}[2]{%
1 dict begin /t #1 def #2 end
}

\newcommand{\fcPolylineAlongCurve}[5][linecolor=\fcColorGraph]{%
\multido{\ra=0+1}{#2}{%
\psline[#1](! \fcEvaluateT{\ra\space #2 div #3 mul 1 \ra \space #2 div sub #4 mul add}{#5})(! \fcEvaluateT{\ra\space 1 add #2 div #3 mul 1 \ra \space 1 add #2 div sub #4 mul add}{#5})%
\rput(! \fcEvaluateT{\ra\space #2 div #3 mul 1 \ra \space #2 div sub #4 mul add}{#5}){\fcFullDot{0}{0}}%
}%
\rput(! \fcEvaluateT{#3}{#5}){\fcFullDot{0}{0}}%
}

\newcommand{\fcPolylineAlongCurveWithLabels}[6][linecolor=\fcColorGraph]{%
\fcPolylineAlongCurve[#1]{#2}{#3}{#4}{#5}%
\multido{\ia=0+1}{#2}{%
\rput[b](! \fcEvaluateT{\ia\space #2 div #3 mul 1 \ia \space #2 div sub #4 mul add}{#5} 0.1 add){${#6}_{\ia}$}%
}%
\rput[b](! \fcEvaluateT{#3}{#5}){${#6}_{#2}$}%
}

\newcommand{\fcVectorNormalize}{ %
1 dict begin %
/theV exch def % theV is our vector
theV 1 theV \fcVectorNorm div \fcVectorTimesScalar %
end %
} %pushes elements of array onto the stack

\newcommand{\fcArrayToStack}{ %
\space %
1 dict begin %
/theArray exch def %put array in var.
0 1 theArray length 1 sub %loop parameters
{ theArray exch get %get array member
} for %
end \space%
} %pushes elements of array onto the stack

\newcommand{\fcSpliceArrayOperationArray}{ %
5 dict begin %
/theOp exch def %
/secondV exch def %
/firstV exch def %
/counter 0 def %
/dimension firstV length def %
[dimension {firstV counter get secondV counter get theOp /counter counter 1 add def } repeat] %
end %
} %splices two arrays and operation, for example [a b] [c d] {op} -> [a c op b d op]

\newcommand{\fcSpliceArrayOperation}{ %
4 dict begin %
/theOp exch def %
/firstV exch def %
/counter 0 def %
/dimension firstV length def %
[ dimension {firstV counter get theOp /counter counter 1 add def } repeat ] %
end %
} %splices array with operation. [a b] {op} -> [a op b op]

\newcommand{\fcArrayOperation}{ %
4 dict begin %
/theOp exch def %
/firstV exch def %
/counter 0 def%
/dimension firstV length def %
dimension {firstV counter get /counter counter 1 add def} repeat %
dimension 1 sub {theOp} repeat %
end %
} %applies operation n-1 times to array. Example: [a b c] {op} -> a b c op op

\newcommand{\fcVectorScalarVector}{%
{mul} \fcSpliceArrayOperationArray {add}\fcArrayOperation
} %Scalar product two vectors

\newcommand{\fcVectorPlusVector}{%
{add} \fcSpliceArrayOperationArray %
} %Adds two vectors

\newcommand{\fcVectorMinusVector}{%
{sub} \fcSpliceArrayOperationArray %
} %Adds two vectors

\newcommand{\fcVectorTimesScalar}{ %
2 dict begin %
/theScalar exch def %
/theV exch def %
theV {theScalar mul} \fcSpliceArrayOperation %
end %
} %

\newcommand{\fcVectorCrossVector}{ %
8 dict begin %
/vectB exch def %
/vectA exch def %
vectA \fcArrayToStack %
/a3 exch def %The three coordinates of Vector a
/a2 exch def %
/a1 exch def %
vectB \fcArrayToStack %
/b3 exch def %The three coordinates of Vector b
/b2 exch def %
/b1 exch def %
[ a2 b3 mul a3 b2 mul sub a3 b1 mul a1 b3 mul sub a1 b2 mul a2 b1 mul sub] %the cross product of a and b
end %
}

\newcommand{\fcVectorNorm}{%
dup \fcVectorScalarVector sqrt %
} %

\newcommand{\fcVectorNormSquared}{%
dup \fcVectorScalarVector %
} %

\newcommand{\fcProjectOntoScreen}{%
%(calling project onto plane with arguments:) == %
%dup == %
3 dict begin %
\fcScreenWithSpace %
/theD exch def %
/theNormal exch def %
/theV exch def %
theV theNormal theD theV theNormal \fcVectorScalarVector sub theNormal \fcVectorNormSquared div \fcVectorTimesScalar \fcVectorPlusVector %
end %
} %Projection of point onto a plane. First argument is point, second argument is plane normal, third argument is the scalar product you need to have with the normal to be in the plane. Format: [1 2 3] [4 5 6] 7, corresponds to projecting the point (1,2,3) onto the plane 4x+5y+6z=7

\newcommand{\fcGetDrawCoords}{%
5 dict begin %
/theV exch def %
/theVprojected theV \fcProjectOntoScreen [0 0 0] \fcProjectOntoScreen  \fcVectorMinusVector def%
/theNormalizedNormal \fcScreenWithSpace pop \fcVectorNormalize def %
(\fcScreenStyle) (z) eq %
{ %
/theYUnitV [0 0 1] \fcProjectOntoScreen [0 0 0] \fcProjectOntoScreen \fcVectorMinusVector \fcVectorNormalize def %
/theXUnitV theNormalizedNormal theYUnitV \fcVectorCrossVector def %
} %
{ %
(\fcScreenStyle) (x) eq %
{
/theXUnitV [1 0 0] \fcProjectOntoScreen [0 0 0] \fcProjectOntoScreen \fcVectorMinusVector \fcVectorNormalize def %
/theYUnitV theXUnitV theNormalizedNormal \fcVectorCrossVector def%
}
{
/theYUnitV \fcScreenStyle \fcProjectOntoScreen [0 0 0] \fcProjectOntoScreen \fcVectorMinusVector \fcVectorNormalize def%
/theXUnitV theNormalizedNormal theYUnitV \fcVectorCrossVector def% 
} ifelse%
}%
ifelse %
%(normalized normal: ) == theNormalizedNormal ==
%(y unit v) == theYUnitV ==
%(x unit v: ) == theXUnitV ==
theVprojected theXUnitV \fcVectorScalarVector theVprojected theYUnitV \fcVectorScalarVector
end %
}

\newcommand{\fcScreen}{[-1 1 -0.5] -1} %default projection plane. Renew this command to change projection plane.
\newcommand{\fcScreenWithSpace}{\fcScreen\space } %Darned LaTeX...

\newcommand{\fcBoxIIId}[5][]{%
\pstVerb{%
4 dict begin%
/visibleCorner #2 def%
/vectorOne #3 #2 \fcVectorMinusVector def%
/vectorTwo #4 #2 \fcVectorMinusVector def%
/vectorThree #5 #2 \fcVectorMinusVector def%
}%
\fcPolyLineIIId[#1]{visibleCorner dup vectorOne \fcVectorPlusVector dup vectorTwo \fcVectorPlusVector dup vectorOne \fcVectorMinusVector dup vectorTwo \fcVectorMinusVector visibleCorner}%
\fcPolyLineIIId[#1]{visibleCorner dup vectorOne \fcVectorPlusVector dup vectorThree \fcVectorPlusVector dup vectorOne \fcVectorMinusVector dup vectorThree \fcVectorMinusVector}%
\fcPolyLineIIId[#1]{visibleCorner vectorTwo \fcVectorPlusVector dup vectorThree \fcVectorPlusVector dup vectorTwo \fcVectorMinusVector}%
\fcPolyLineIIId[#1, linestyle=dashed]{visibleCorner vectorOne  vectorTwo vectorThree \fcVectorPlusVector \fcVectorPlusVector \fcVectorPlusVector dup vectorOne \fcVectorMinusVector}%
\fcPolyLineIIId[#1, linestyle=dashed]{visibleCorner vectorOne  vectorTwo vectorThree \fcVectorPlusVector \fcVectorPlusVector \fcVectorPlusVector dup vectorTwo \fcVectorMinusVector}%
\fcPolyLineIIId[#1, linestyle=dashed]{visibleCorner vectorOne  vectorTwo vectorThree \fcVectorPlusVector \fcVectorPlusVector \fcVectorPlusVector dup vectorThree \fcVectorMinusVector}%
\pstVerb{end}%
}

\newcommand{\fcBoxIIIdFilled}[5][]{%
\pscustom*[#1]{%
\fcPolyLineIIId{4 dict begin%
/visibleCorner #2 def%
/vectorOne #3 #2 \fcVectorMinusVector def%
/vectorTwo #4 #2 \fcVectorMinusVector def%
/vectorThree #5 #2 \fcVectorMinusVector def %
visibleCorner vectorOne \fcVectorPlusVector dup vectorTwo \fcVectorPlusVector dup vectorOne \fcVectorMinusVector dup vectorThree \fcVectorPlusVector dup vectorTwo \fcVectorMinusVector dup vectorOne \fcVectorPlusVector visibleCorner vectorOne \fcVectorPlusVector end %
}%
}%
}

\newcommand{\fcParallelogramIIId}[4][linecolor=cyan!30]{%
\pscustom*[#1]{%
\fcParallelogramHollowIIId{#2}{#3}{#4}%
}%
}

\newcommand{\fcParallelogramHollowIIId}[4][]{ %
\fcPolyLineIIId[#1]{3 dict begin /corner #2 def /vectorOne #3 #2 \fcVectorMinusVector def /vectorTwo #4 #2 \fcVectorMinusVector def corner dup vectorOne \fcVectorPlusVector dup vectorTwo \fcVectorPlusVector dup vectorOne \fcVectorMinusVector corner end
}%
}

\newcommand{\fcParallelogramHalfVisibleIIId}[4][]{%
\pstVerb{3 dict begin /corner #2 def /vectorOne #3 #2 \fcVectorMinusVector def /vectorTwo #4 #2 \fcVectorMinusVector def}%
\fcPolyLineIIId[#1]{corner vectorOne \fcVectorPlusVector corner dup vectorTwo \fcVectorPlusVector}%
\fcPolyLineIIId[#1,linestyle=dashed]{corner vectorOne \fcVectorPlusVector dup vectorTwo \fcVectorPlusVector dup vectorOne \fcVectorMinusVector}%
\pstVerb{end}%
}

\newcommand{\fcPolyLineIIId}[2][linecolor=black]{%
\listplot[#1]{ [#2] {\fcGetDrawCoords} \fcSpliceArrayOperation \fcArrayToStack}%
}

\newcommand{\fcLineIIId}[3][linecolor=black]{
\psline[#1](! #2 \space \fcGetDrawCoords)(! #3 \space \fcGetDrawCoords )
}
\newcommand{\fcAxesIIIdFull}[4][linecolor=black, arrows=->]{ %
\fcAxesIIId[#1]{#2}{#3}{#4} %
\fcLineIIId[#1]{[0 0 0]}{[#2\space -1 mul 0 0]} %
\fcLineIIId[#1]{[0 0 0]}{[0 #3\space -1 mul 0]} %
\fcLineIIId[#1]{[0 0 0]}{[0 0 #4\space -1 mul]} %
} %

\newcommand{\fcAxesIIId}[4][linecolor=black, arrows=->]{
\setkeys{fcGraphics}{#1}
\fcLineIIId[#1]{[0 0 0]}{[#2 0 0]}
\rput(! [#2 0 0] \fcGetDrawCoords){$~~x$}
\fcLineIIId[#1]{[0 0 0]}{[0 #3 0]}
\rput(! [0 #3 0] \fcGetDrawCoords){$~~y$}
\fcLineIIId[#1]{[0 0 0]}{[0 0 #4]}
\rput(! [0 0 #4] \fcGetDrawCoords){$~~z$}
}
\newcommand{\fcDotIIId}[2][linecolor=\fcColorGraph]{ %
\pscircle*[#1](! #2 \fcGetDrawCoords){0.07} %
} %

\newcommand{\fcPutIIId}[3][]{ %
\rput[#1](! #2 \fcGetDrawCoords) {#3} %
} %

\newcommand{\fcCurveIIId}[4][linecolor=\fcColorGraph]{%
\parametricplot[#1]{#2}{#3}{ %
[#4]%
\fcGetDrawCoords %
}
}

\newcommand{\fcZeroVector}{%
[ exch { 0 } repeat ]
}

\newcommand{\fcPerpendicularComputeHeel}[3]{%
\pstVerb{%
7 dict begin%
/thePoint #1 def%
/heelSize #3 def %
mark #2 %
counttomark 1 eq {%
/directionUnitVector exch \fcVectorNormalize def%
/basePoint thePoint length \fcZeroVector def%
}{%
/basePoint exch def%
/directionUnitVector exch basePoint \fcVectorMinusVector \fcVectorNormalize def%
} ifelse %
pop%
/heel directionUnitVector thePoint basePoint \fcVectorMinusVector directionUnitVector \fcVectorScalarVector \fcVectorTimesScalar basePoint \fcVectorPlusVector def%
%heel == %
/perpendicularUnitVector thePoint heel \fcVectorMinusVector \fcVectorNormalize def %
%perpendicularUnitVector == %
/polyLineInput {% 
heel directionUnitVector heelSize \fcVectorTimesScalar \fcVectorMinusVector %
%dup ==
dup perpendicularUnitVector heelSize \fcVectorTimesScalar \fcVectorPlusVector %
heel perpendicularUnitVector heelSize \fcVectorTimesScalar \fcVectorPlusVector%
} def%
}%
}

\newcommand{\fcPerpendicular}[4][]{%
\fcPerpendicularComputeHeel{#2}{#3}{#4}%
\psline[#1](! thePoint \fcArrayToStack)(! heel \fcArrayToStack)%
\listplot[linecolor=red]{ [polyLineInput] {\fcArrayToStack} \fcSpliceArrayOperation \fcArrayToStack}%
\pstVerb{end}%
}

\newcommand{\fcPerpendicularIIId}[4][]{%
\fcPerpendicularComputeHeel{#2}{#3}{#4}
\fcLineIIId[#1]{thePoint}{heel}%
\fcPolyLineIIId[linecolor=red]{polyLineInput}%
\pstVerb{end}%
}%

\newcommand{\fcPlotIIId}[7][]{%
\fcPlotIIIdXconst[#1]{#2}{#3}{#4}{#5}{#6}{#7}%
\fcPlotIIIdYconst[#1]{#2}{#3}{#4}{#5}{#6}{#7}%
}
\newcommand{\fcPlotIIIdXconst}[7][]{%
\setkeys{fcGraphics}{#2}%
\multido{\ra=0+1}{\fcIterationsX}{%
\pstVerb{%
3 dict begin %
/x \ra \space #3 mul \fcIterationsX \space \ra \space sub 1 sub  #5\space mul add \fcIterationsX\space 1 sub div def%
/ymin #4 def%
/ymax #6 def%
}%
\parametricplot[#1]{ymin}{ymax}{% 
1 dict begin /y t def  [x y #7] \fcGetDrawCoords end%
}%
\pstVerb{end}%
}%end multido
}

\newcommand{\fcPlotIIIdYconst}[7][]{%
\setkeys{fcGraphics}{#2}%
\multido{\ra=0+1}{\fcIterationsY}{%
\pstVerb{%
3 dict begin%
/y \ra \space #4 mul \fcIterationsY \space \ra \space sub 1 sub  #6\space mul add \fcIterationsY\space 1 sub div def%
/xmin #3 def%
/xmax #5 def%
}%
\parametricplot[#1]{xmin}{xmax}{% 
1 dict begin /x t def  [x y #7] \fcGetDrawCoords end%
}%
\pstVerb{end}%
}%end multido
}

\newcommand{\fcVectorField}[3][linecolor=blue]{%
\setkeys{fcGraphics}{#2}%
\multido{\ra=\fcStartXIId+\fcDelta}{\fcIterationsX}{%
\multido{\rb=\fcStartYIId+\fcDelta}{\fcIterationsY}{%
\pstVerb{%
4 dict begin%
/x \ra\space def%
/y \rb\space def %
#3\space%
/vY exch def%
/vX exch def%
}%
\psline[#1](! x vX 2 div sub y vY 2 div sub)(! x vX 2 div add y vY 2 div add)%
\pscircle*[linecolor=red](! x y){0.02}%
\pstVerb{end}%
}%end multido
}%end multido
}%



\usepackage{auto-pst-pdf}
\usepackage{pst-plot}
%\usepackage{pstricks-add}

% Or whatever. Note that the encoding and the font should match. If T1
% does not look nice, try deleting the line with the fontenc.


\graphicspath{{../../modules/}}

\newtheoremstyle{partialproof}{3pt}{3pt}{}{}{}{.}{.5em}{}
\theoremstyle{partialproof} \newtheorem{partialproof}[theorem]{Proof.}
%\DeclareMathOperator{\diff}{d}
\setbeamertemplate{navigation symbols}{}

\includeonlylecture{1}

\newcommand{\lect}[3]{
  \date{#1}
  \lecture[#1]{#2}{#3}
}

\setbeamertemplate{footline}
{
  \leavevmode%
  \hbox{%
  \begin{beamercolorbox}[wd=.333333\paperwidth,ht=2.25ex,dp=1ex,center]{author in head/foot}%
    \usebeamerfont{author in head/foot}\insertshortauthor
  \end{beamercolorbox}%
  \begin{beamercolorbox}[wd=.333333\paperwidth,ht=2.25ex,dp=1ex,center]{title in head/foot}%
    \usebeamerfont{title in head/foot}\insertshorttitle
  \end{beamercolorbox}%
  \begin{beamercolorbox}[wd=.333333\paperwidth,ht=2.25ex,dp=1ex,center]{date in head/foot}%
    \usebeamerfont{date in head/foot}\insertshortdate{}
  \end{beamercolorbox}}%
  \vskip0pt%
}

% If you have a file called "university-logo-filename.xxx", where xxx
% is a graphic format that can be processed by latex or pdflatex,
% resp., then you can add a logo as follows:

%\pgfdeclareimage[height=0.8cm]{logo}{bluelogo}
%\logo{\pgfuseimage{logo}}
\renewcommand{\Arcsin}{\arcsin}
\renewcommand{\Arccos}{\arccos}
\renewcommand{\Arctan}{\arctan}
\renewcommand{\Arccot}{\text{arccot\hspace{0.03cm}}}
\renewcommand{\Arcsec}{\text{arcsec\hspace{0.03cm}}}
\renewcommand{\Arccsc}{\text{arccsc\hspace{0.03cm}}}



\begin{document}

\AtBeginLecture{%

\title[\insertlecture]{FreeCalc}
\subtitle{\insertlecture}
\author[FreeCalc]{}
\institute[UMass Boston]{University of Massachusetts Boston}
\date{\insertshortlecture}
\begin{frame}
  \titlepage
\end{frame}
}%

% begin lecture
\lect{\today}{Sample}{1}{
%\begin{frame}
\begin{example}[Volume of toroid]
\begin{columns}
\column{0.53\textwidth} 
\begin{pspicture}(-1,-1)(1,1)
\tiny
\renewcommand{\fcScreen}{[-1 1 -0.75] -1}
\fcBoundingBox{-2.9}{-1.8}{3.4}{2.2}
\pstVerb{%
/theMajorRadius 2 def
/theMinorRadius 0.8 def
/theT {[5 dict begin 
/R theMajorRadius def
/r theMinorRadius def
/A R r v cos mul add def
u cos A mul 
u sin A mul 
v sin r mul
end]}
def
/phiP 60 def
/thetaP 60 def
/theP 2 dict begin /u phiP def /v thetaP def theT end def
/theH [theP \fcArrayToStack pop 0] def
/theC [phiP cos theMajorRadius mul phiP sin theMajorRadius mul 0] def
}%
\fcStartIIIdScene
\only<1-4>{%
\fcSurfaceInScene[iterationsU=24, iterationsV=12 ]{0}{0}{360}{360}{
[5 dict begin 
/R 2 def
/r 0.8 def
/A R r v cos mul add def
u cos A mul 
u sin A mul 
v sin r mul
end]
}{}%
}%
\only<5>{%
\fcSurfaceInScene[iterationsU=3, iterationsV=12 ]{60}{0}{105}{360}{
[5 dict begin 
/R 2 def
/r 0.8 def
/A R r v cos mul add def
u cos A mul 
u sin A mul 
v sin r mul
end]
}{}%
}%
\only<4->{%
\fcPatchInScene[colorUV=cyan, colorVU=cyan]{[0 0 -1.2]}{[0.5 3 sqrt 2 div -0.4] 3 \fcVectorTimesScalar }{[0 0 1.1] }%
}%
\fcAxesIIIdInScene{4}{4}{2.2}%
\fcFinishIIIdScene[true]%
\uncover<6->{%
\fcDotIIId{[0 0 0]}%
\fcPutIIId[rt]{[0 0 0]}{$O~~$}%
\fcCurveIIId{0}{360}{2 dict begin /u phiP def  /v t def theT end}%
}%
\uncover<3->{%
\fcDotIIId{theP}%
\fcPutIIId[l]{theP}{$~~P(x,y,z)$}%
}%
\uncover<9->{%
\fcLineIIId{[0 0 0]}{theH}%
\fcLineIIId{theC}{theP}%
\fcAngleIIId[linecolor=red]{[1 0 0]}{theC}{0.2}%
\fcPutIIId{[0.2 0.1 0]}{$~~~\phi$}
\fcPutIIId{theC}{\fcAngleIIId[linecolor=red]{theC}{theP theC \fcVectorMinusVector }{0.2}}%
\fcPutIIId[l]{theC [0 0 0.2] \fcVectorPlusVector}{$~~~\theta$}
}%
\uncover<7->{%
\fcDotIIId{theH}%
\fcPerpendicularIIId{theP}{[0 0 0] theH}{0.2}%
\fcPutIIId[t]{theH [0 0 -0.1] \fcVectorPlusVector}{$H$}%
}%
\uncover<8->{%
\fcDotIIId{theC}%
\fcPutIIId[t]{theC [0 0 -0.1] \fcVectorPlusVector}{$C$}%
}%
\uncover<10,11>{%
\fcLineIIId[linecolor=red, linewidth=1.5pt]{[0 0 0]}{theC}%
}%
\uncover<12,13>{%
\fcLineIIId[linecolor=red, linewidth=1.5pt]{theP}{theC}%
}%
\uncover<14,15>{%
\fcLineIIId[linecolor=red, linewidth=1.5pt]{theP}{theH}%
}%
\uncover<16,17>{%
\fcLineIIId[linecolor=red, linewidth=1.5pt]{[0 0 0]}{theH}%
}%
\uncover<18->{%
\fcPerpendicularIIId{theH}{[0 0 0] [1 0 0]}{0.2}%
}%
\uncover<18,19>{%
\fcLineIIId[linecolor=red, linewidth=1.5pt]{[0 0 0]}{[theH \fcArrayToStack pop pop 0 0]}%
}%
\uncover<20,21>{%
\fcLineIIId[linecolor=red, linewidth=1.5pt]{theH}{[theH \fcArrayToStack pop pop 0 0]}%
}%
\uncover<22,23>{%
\fcLineIIId[linecolor=red, linewidth=1.5pt]{theH}{theP}%
}%
\end{pspicture}
\column{0.47\textwidth} 
Find the volume of a toroid $T$ (the inside of a torus $S$) with major radius $R$ and minor radius $r$.

\uncover<18->{%
$
S:\left|\begin{array}{r@{}c@{}l}
\fcQuestion{18}{x}&\fcQuestion{18}{=}&\fcAnswer{19}{(R+r\cos \theta) \cos \phi} \\
\alert<20,21>{y}&\alert<20,21>{=}& \fcAnswer{21} {(R+r\cos \theta) \sin \phi}\\
\alert<22,23>{z}&\alert<22,23>{=}& \fcAnswer{23} {r\sin \theta}
\end{array}\right.
$
}
\end{columns}
\uncover<2->{Suppose the toroid sits in space as drawn.} \uncover<3->{Let $P(x,y,z)\in S$.} \uncover<4->{ Let $\mathcal P$ be the plane through the $z$-axis and $P$.} 

\uncover<7->{ Let $H$ be the heel of the perpendicular from $P$ to the $x,y$-plane.} \uncover<8->{Let $C$ be the center of the circle cross-section of $\mathcal P$ with $T$. } \uncover<9->{ Let $\phi$ and $\theta$ be the indicated angles.} \uncover<10->{ We have
$\begin{array}{rcl}
\fcQuestion{10}{|OC|} & \fcQuestion{10}{=}& \fcAnswer{11}{R}\\
\fcQuestion{12}{|PC|} & \fcQuestion{12}{=}& \fcAnswer{13}{r}\\
\fcQuestion{14}{|PH|} & \fcQuestion{14}{=}& \fcAnswer{15}{r\sin\theta}\\
\fcQuestion{16}{ \alert<18-21>{|OH|}} & \fcQuestion{16}{\alert<18-21>{=}}& \fcAnswer{17}{\alert<18-21>{R+r\cos\theta}}\\
\end{array}.
$  }
\end{example}
\vskip 10cm 
\end{frame}
%

\begin{frame}
\begin{example}[Volume of toroid]
\begin{columns}
\column{0.53\textwidth} 
\begin{pspicture}(-2.9,-1.8)(3.4,2.2)
\tiny
\renewcommand{\fcScreen}{[-1 1 -0.75] -1}
\fcBoundingBox{-2.9}{-1.8}{3.4}{2.2}
\pstVerb{%
/theMajorRadius 2 def
/theMinorRadius 0.8 def
/theT {[5 dict begin 
/R theMajorRadius def
/r theMinorRadius def
/A R r v cos mul add def
u cos A mul 
u sin A mul 
v sin r mul
end]}
def
/phiP 60 def
/thetaP 60 def
/theP 2 dict begin /u phiP def /v thetaP def theT end def
/theH [theP \fcArrayToStack pop 0] def
/theC [phiP cos theMajorRadius mul phiP sin theMajorRadius mul 0] def
}%
\fcStartIIIdScene
\only<2>{%
\fcSurfaceInScene[linecolor=black, arrows=->, iterationsU=8, iterationsV=12 ]{0}{0}{120}{360}{
[5 dict begin 
/R theMajorRadius def
/r theMinorRadius def
/A R r v cos mul add def
u cos A mul 
u sin A mul 
v sin r mul
end]
}{}%
}%
\only<3>{%
\fcSurfaceInScene[linecolor=black, arrows=->, iterationsU=16, iterationsV=12 ]{0}{0}{240}{360}{
[5 dict begin 
/R theMajorRadius def
/r theMinorRadius def
/A R r v cos mul add def
u cos A mul 
u sin A mul 
v sin r mul
end]
}{}%
}%
\only<4,5,9,10,13->{%
\fcSurfaceInScene[linecolor=black, arrows=->, iterationsU=24, iterationsV=12 ]{0}{0}{360}{360}{
[5 dict begin 
/R theMajorRadius def
/r theMinorRadius def
/A R r v cos mul add def
u cos A mul 
u sin A mul 
v sin r mul
end]
}{}%
}%
\fcAxesIIIdInScene{4}{4}{2.2}%
\only<1>{%
\fcCurveIIIdInScene[linewidth=2, linecolor=red]{0}{360}{2 dict begin /u 0 def /v t def theT end}%
}%
\only<2>{%
\fcCurveIIIdInScene[linewidth=2, linecolor=red]{0}{360}{2 dict begin /u 120 def /v t def theT end}%
}%
\only<3>{%
\fcCurveIIIdInScene[linewidth=2, linecolor=red]{0}{360}{2 dict begin /u 240 def /v t def theT end}%
}%
\only<4>{%
\fcCurveIIIdInScene[linewidth=2, linecolor=red]{0}{360}{2 dict begin /u 0 def /v t def theT end}%
}%
\only<7>{%
\fcSurfaceInScene[linecolor=black, linewidth=1, arrows=->, iterationsU=24, iterationsV=4 ]{0}{0}{360}{120}{
[5 dict begin 
/R theMajorRadius def
/r theMinorRadius def
/A R r v cos mul add def
u cos A mul 
u sin A mul 
v sin r mul
end]
}{}%
}%
\only<8>{%
\fcSurfaceInScene[linecolor=black, linewidth=1, arrows=->, iterationsU=24, iterationsV=8 ]{0}{0}{360}{240}{
[5 dict begin 
/R theMajorRadius def
/r theMinorRadius def
/A R r v cos mul add def
u cos A mul 
u sin A mul 
v sin r mul
end]
}{}%
}%
\only<11>{%
\fcSurfaceInScene[linecolor=black, linewidth=1, arrows=->, iterationsU=24, iterationsV=12 ]{0}{0}{360}{360}{
[5 dict begin 
/R theMajorRadius def
/r theMinorRadius 0.33 mul def
/A R r v cos mul add def
u cos A mul 
u sin A mul 
v sin r mul
end]
}{}%
}%
\only<12>{%
\fcSurfaceInScene[linecolor=black, linewidth=1, arrows=->, iterationsU=24, iterationsV=12 ]{0}{0}{360}{360}{
[5 dict begin 
/R theMajorRadius def
/r theMinorRadius 0.66 mul def
/A R r v cos mul add def
u cos A mul 
u sin A mul 
v sin r mul
end]
}{}%
}%
\only<8>{\fcCurveIIIdInScene[linewidth=2, linecolor=red]{0}{360}{2 dict begin /u t def /v 240 def theT end}}%
\only<9>{\fcCurveIIIdInScene[linewidth=2, linecolor=red]{0}{360}{2 dict begin /u t def /v 0 def theT end}}%
\fcFinishIIIdScene[fastsort=true]%
\only<7>{\fcCurveIIId[linewidth=2pt, linecolor=red]{0}{360}{2 dict begin /u t def /v 120 def theT end}}%
\only<6>{\fcCurveIIId[arrows=->,linewidth=2pt, linecolor=red]{0}{360}{2 dict begin /u t def /v 0 def theT end}}%
\end{pspicture}
\column{0.47\textwidth} 
Find the volume of a toroid $T$ (the inside of a torus $S$) with major radius $R$ and minor radius $r$.

$
S:\left|\begin{array}{r@{}c@{}l}
x&=&(R+r\cos \theta) \cos \phi \\
y&=&(R+r\cos \theta) \sin \phi \\
z&=& r\sin \theta
\end{array}\right.
$
\end{columns}
$
T:\left|\begin{array}{r@{}c@{}l}
x&=&(R+\rho\cos \theta) \cos \phi \\
y&=&(R+\rho\cos \theta) \sin \phi \\
z&=& \rho\sin \theta
\end{array}\right. \rho\in[0, \uncover<1-13>{\alert<10-13>{\textbf{?}}} \uncover<14->{\alert<14>{r}}], \phi\in [0,\uncover<1-4>{\alert<1-4>{\textbf{?}}} \uncover<5->{\alert<5>{2\pi}}), \theta\in [0,\uncover<1-8>{\alert<6-8>{\textbf{?}}}\uncover<9->{\alert<9>{2\pi}}). 
$

\uncover<15->{Let $\fcv f$ be the map participating in the parametrization of $T$.}

\alert<18>{ 
$\displaystyle \uncover<16->{\Vol (T)= \int \limits_{{\fcQuestion{16}{\theta =} \fcAnswer{17}{0}}}^{{\fcQuestion{16}{\theta=}\fcAnswer{17}{2\pi} }} \int \limits_{{ \fcQuestion{16}{\phi= }\fcAnswer{17}{0} }}^{{\fcQuestion{16}{ \phi=} \fcAnswer{17}{2\pi}}} \int\limits_{{\fcQuestion{16}{\rho =} \fcAnswer{17}{0} }}^{{\fcQuestion{16}{\rho=}\fcAnswer{17}{r}}} \det\left( J_{\fcv f}\right) \diff \rho \diff \phi \diff \theta }
$
}
\end{example}


\vskip 10cm 
\end{frame}
%\begin{frame}
\begin{example}
\begin{columns}
\column{0.55\textwidth} 
\begin{pspicture}(-1,-1)(1,1)
\tiny
\renewcommand{\fcScreen}{[-1 1 -0.75] -1}
\fcBoundingBox{-2.9}{-1.8}{3.4}{2.2}
\pstVerb{%
/theMajorRadius 2 def
/theMinorRadius 0.8 def
/theT {[5 dict begin 
/R theMajorRadius def
/r theMinorRadius def
/A R r v cos mul add def
u cos A mul 
u sin A mul 
v sin r mul
end]}
def
/phiP 60 def
/thetaP 60 def
/theP 2 dict begin /u phiP def /v thetaP def theT end def
/theH [theP \fcArrayToStack pop 0] def
/theC [phiP cos theMajorRadius mul phiP sin theMajorRadius mul 0] def
}%
\fcStartIIIdScene
\fcSurfaceInScene[linecolor=black, arrows=none, iterationsU=24, iterationsV=12 ]{0}{0}{360}{360}{
[5 dict begin 
/R theMajorRadius def
/r theMinorRadius def
/A R r v cos mul add def
u cos A mul 
u sin A mul 
v sin r mul
end]
}{}%
\fcAxesIIIdInScene{4}{4}{2.2}%
\fcFinishIIIdScene[fastsort=true]%
\end{pspicture}
\column{0.45\textwidth} 
Find the volume of a toroid $T$ (the inside of a torus $S$) with major radius $R$ and minor radius $r$.

$
S:\left|\begin{array}{r@{}c@{}l}
x&=&(R+r\cos \theta) \cos \phi \\
y&=&(R+r\cos \theta) \sin \phi \\
z&=& r\sin \theta
\end{array}\right.
$
\end{columns}
$
T:\left|\begin{array}{r@{}c@{}l}
x&=&(R+\rho\cos \theta) \cos \phi \\
y&=&(R+\rho\cos \theta) \sin \phi \\
z&=& \rho\sin \theta
\end{array}\right.
$

\end{example}


\vskip 10cm 
\end{frame}
\begin{frame}
\begin{example}[Volume of toroid]
\begin{columns}
\column{0.53\textwidth} 
\begin{pspicture}(-1,-1)(1,1)
\tiny
\renewcommand{\fcScreen}{[-1 1 -0.75] -1}
\fcBoundingBox{-2.9}{-1.8}{3.4}{2.2}
\pstVerb{%
/theMajorRadius 2 def
/theMinorRadius 0.8 def
/theT {[5 dict begin 
/R theMajorRadius def
/r theMinorRadius def
/A R r v cos mul add def
u cos A mul 
u sin A mul 
v sin r mul
end]}
def
/phiP 60 def
/thetaP 60 def
/theP 2 dict begin /u phiP def /v thetaP def theT end def
/theH [theP \fcArrayToStack pop 0] def
/theC [phiP cos theMajorRadius mul phiP sin theMajorRadius mul 0] def
}%
\fcStartIIIdScene
\fcSurfaceInScene[linecolor=black, arrows=none, iterationsU=24, iterationsV=12 ]{0}{0}{360}{360}{
[5 dict begin 
/R theMajorRadius def
/r theMinorRadius def
/A R r v cos mul add def
u cos A mul 
u sin A mul 
v sin r mul
end]
}{}%
\fcAxesIIIdInScene{4}{4}{2.2}%
\fcFinishIIIdScene[fastsort=true]%
\end{pspicture}

\column{0.47\textwidth} 
Find volume of toroid $T$, major radius $R$, minor radius $r$.

$
\fcv f:\left|\begin{array}{r@{}c@{}l}
x&\alert<0>{=}&(R+\rho\cos \theta) \cos \phi \\
y&\alert<0>{=}&(R+\rho\cos \theta) \sin \phi \\
z&\alert<0>{=}& \rho\sin \theta
\end{array}\right.
$

$ 
J_{\fcv f}=\uncover<0->{
\left(\begin{array}{ccc}
\frac{\partial x}{ \partial \rho}& \frac{\partial x}{\partial \phi} &\frac{\partial x}{\partial \theta}\\
\frac{\partial y}{ \partial \rho}& \frac{\partial y}{\partial \phi} &\frac{\partial y}{\partial \theta}\\
\frac{\partial z}{ \partial \rho}& \frac{\partial z}{\partial \phi} &\frac{\partial z}{\partial \theta}\\
\end{array}\right)
}
$
\end{columns}
$
\begin{array}{r@{}c@{}l}
\displaystyle \alert<0>{\Vol (T)}&\alert<0>{=}&\displaystyle \alert<0>{ \int \limits_{\theta = 0}^{\theta=2\pi} \int \limits_{\phi= 0}^{\phi=2\pi} \int\limits_{\rho =0}^{\rho=r} \det \left(\alert<0>{J_{\fcv f}}\right) \diff \rho \diff \phi \diff \theta } = \int \limits_{ 0}^{2\pi} \int \limits_{0}^{2\pi} \int\limits_{0}^{r} \alert<1>{\rho(R+\rho\cos \theta) } \diff \rho \diff \phi \diff \theta  \\
\uncover<2->{&=& \int \limits_{ 0}^{2\pi} \int \limits_{0}^{2\pi} {\left[ \fcAnswer{3}{\frac{R\rho^2}{2}+\frac{\rho^3}{3}\cos \theta} \right]}_{{\rho=0}}^{{\rho=r}} \diff \phi \diff \theta \uncover<4->{ = \int \limits_{ 0}^{2\pi} \int \limits_{0}^{2\pi}\left( \frac{Rr^2}{2}+ \frac{r^3}{3}\cos \theta \right) \diff \phi \diff \theta  }}\\
\uncover<4->{ &=&2\pi \alert<5>{\int_{0}^{2\pi}} \left( \frac{Rr^2}{2}+ \frac{r^3}{3} \alert<5>{\cos \theta} \right) \alert<5>{\diff \theta} }\uncover<5->{=2\pi \int_{0}^{2\pi} \frac{Rr^2}{2} \diff \theta }\uncover<6->{= 2Rr^2\pi^2}
\end{array}
$

\end{example}


\vskip 10cm 
\end{frame}

%\begin{frame}
\begin{example}
\begin{columns}
\column{0.55\textwidth} 
\begin{pspicture}(-1,-1)(1,1)
\tiny
\renewcommand{\fcScreen}{[-1 1 -0.75] -1}
\fcBoundingBox{-2.9}{-1.8}{3.4}{2.2}
\pstVerb{%
/theMajorRadius 2 def
/theMinorRadius 0.8 def
/theT {[5 dict begin 
/R theMajorRadius def
/r theMinorRadius def
/A R r v cos mul add def
u cos A mul 
u sin A mul 
v sin r mul
end]}
def
/phiP 60 def
/thetaP 60 def
/theP 2 dict begin /u phiP def /v thetaP def theT end def
/theH [theP \fcArrayToStack pop 0] def
/theC [phiP cos theMajorRadius mul phiP sin theMajorRadius mul 0] def
}%
\fcStartIIIdScene
\only<1-4>{%
\fcSurfaceInScene[iterationsU=24, iterationsV=12 ]{0}{0}{360}{360}{
[5 dict begin 
/R 2 def
/r 0.8 def
/A R r v cos mul add def
u cos A mul 
u sin A mul 
v sin r mul
end]
}{}%
}%
\only<5>{%
\fcSurfaceInScene[iterationsU=3, iterationsV=12 ]{60}{0}{105}{360}{
[5 dict begin 
/R 2 def
/r 0.8 def
/A R r v cos mul add def
u cos A mul 
u sin A mul 
v sin r mul
end]
}{}%
}%
\only<4->{%
\fcPatchInScene[colorUV=cyan, colorVU=cyan]{[0 0 -1.2]}{[0.5 3 sqrt 2 div -0.4] 3 \fcVectorTimesScalar }{[0 0 1.1] }%
}%
\fcAxesIIIdInScene{4}{4}{2.2}%
\fcFinishIIIdScene[true]%
\uncover<6->{%
\fcDotIIId{[0 0 0]}%
\fcPutIIId[rt]{[0 0 0]}{$O~~$}%
\fcCurveIIId{0}{360}{2 dict begin /u phiP def  /v t def theT end}%
}%
\uncover<3->{%
\fcDotIIId{theP}%
\fcPutIIId[l]{theP}{$~~P(x,y,z)$}%
}%
\uncover<9->{%
\fcLineIIId{[0 0 0]}{theH}%
\fcLineIIId{theC}{theP}%
\fcAngleIIId[linecolor=red]{[1 0 0]}{theC}{0.2}%
\fcPutIIId{[0.2 0.1 0]}{$~~~\phi$}
\fcPutIIId{theC}{\fcAngleIIId[linecolor=red]{theC}{theP theC \fcVectorMinusVector }{0.2}}%
\fcPutIIId[l]{theC [0 0 0.2] \fcVectorPlusVector}{$~~~\theta$}
}%
\uncover<7->{%
\fcDotIIId{theH}%
\fcPerpendicularIIId{theP}{[0 0 0] theH}{0.2}%
\fcPutIIId[t]{theH [0 0 -0.1] \fcVectorPlusVector}{$H$}%
}%
\uncover<8->{%
\fcDotIIId{theC}%
\fcPutIIId[t]{theC [0 0 -0.1] \fcVectorPlusVector}{$C$}%
}%
\uncover<10,11>{%
\fcLineIIId[linecolor=red, linewidth=1.5pt]{[0 0 0]}{theC}%
}%
\uncover<12,13>{%
\fcLineIIId[linecolor=red, linewidth=1.5pt]{theP}{theC}%
}%
\uncover<14,15>{%
\fcLineIIId[linecolor=red, linewidth=1.5pt]{theP}{theH}%
}%
\uncover<16,17>{%
\fcLineIIId[linecolor=red, linewidth=1.5pt]{[0 0 0]}{theH}%
}%
\uncover<18->{%
\fcPerpendicularIIId{theH}{[0 0 0] [1 0 0]}{0.2}%
}%
\uncover<18,19>{%
\fcLineIIId[linecolor=red, linewidth=1.5pt]{[0 0 0]}{[theH \fcArrayToStack pop pop 0 0]}%
}%
\uncover<20,21>{%
\fcLineIIId[linecolor=red, linewidth=1.5pt]{theH}{[theH \fcArrayToStack pop pop 0 0]}%
}%
\uncover<22,23>{%
\fcLineIIId[linecolor=red, linewidth=1.5pt]{theH}{theP}%
}%
\end{pspicture}
\column{0.45\textwidth} 
Find the volume of a toroid $T$ (the inside of a torus $S$) with major radius $R$ and minor radius $r$.

\uncover<18->{%
$
S:\left|\begin{array}{r@{}c@{}l}
\fcQuestion{18}{x}&\fcQuestion{18}{=}&\fcAnswer{19}{(R+r\cos \theta) \cos \phi} \\
\alert<20,21>{y}&\alert<20,21>{=}& \fcAnswer{21} {(R+r\cos \theta) \sin \phi}\\
\alert<22,23>{z}&\alert<22,23>{=}& \fcAnswer{23} {r\sin \theta}
\end{array}\right.
$
}
\end{columns}
\uncover<2->{Suppose the toroid sits in space as drawn.} \uncover<3->{Let $P(x,y,z)\in S$.} \uncover<4->{ Let $\mathcal P$ be the plane through the $z$-axis and $P$.} 

\uncover<7->{ Let $H$ be the heel of the perpendicular from $P$ to the $x,y$-plane.} \uncover<8->{Let $C$ be the center of the circle cross-section of $\mathcal P$ with $T$. } \uncover<9->{ Let $\phi$ and $\theta$ be the indicated angles.} \uncover<10->{ We have
$\begin{array}{rcl}
\fcQuestion{10}{|OC|} & \fcQuestion{10}{=}& \fcAnswer{11}{R}\\
\fcQuestion{12}{|PC|} & \fcQuestion{12}{=}& \fcAnswer{13}{r}\\
\fcQuestion{14}{|PH|} & \fcQuestion{14}{=}& \fcAnswer{15}{r\sin\theta}\\
\fcQuestion{16}{ \alert<18-21>{|OH|}} & \fcQuestion{16}{\alert<18-21>{=}}& \fcAnswer{17}{\alert<18-21>{R+r\cos\theta}}\\
\end{array}.
$  }
\end{example}
\vskip 10cm 
\end{frame}

\begin{frame}
\begin{example}
\begin{columns}
\column{0.55\textwidth} 
\begin{pspicture}(-1,-1)(1,1)
\tiny
\renewcommand{\fcScreen}{[-1 1 -0.75] -1}
\fcBoundingBox{-2.9}{-1.8}{3.4}{2.2}
\pstVerb{%
/theMajorRadius 2 def
/theMinorRadius 0.8 def
/theT {[5 dict begin 
/R theMajorRadius def
/r theMinorRadius def
/A R r v cos mul add def
u cos A mul 
u sin A mul 
v sin r mul
end]}
def
/phiP 60 def
/thetaP 60 def
/theP 2 dict begin /u phiP def /v thetaP def theT end def
/theH [theP \fcArrayToStack pop 0] def
/theC [phiP cos theMajorRadius mul phiP sin theMajorRadius mul 0] def
}%
\fcStartIIIdScene
\only<2>{%
\fcSurfaceInScene[linecolor=black, arrows=->, iterationsU=8, iterationsV=12 ]{0}{0}{120}{360}{
[5 dict begin 
/R theMajorRadius def
/r theMinorRadius def
/A R r v cos mul add def
u cos A mul 
u sin A mul 
v sin r mul
end]
}{}%
}%
\only<3>{%
\fcSurfaceInScene[linecolor=black, arrows=->, iterationsU=16, iterationsV=12 ]{0}{0}{240}{360}{
[5 dict begin 
/R theMajorRadius def
/r theMinorRadius def
/A R r v cos mul add def
u cos A mul 
u sin A mul 
v sin r mul
end]
}{}%
}%
\only<4,5,9,10,13->{%
\fcSurfaceInScene[linecolor=black, arrows=->, iterationsU=24, iterationsV=12 ]{0}{0}{360}{360}{
[5 dict begin 
/R theMajorRadius def
/r theMinorRadius def
/A R r v cos mul add def
u cos A mul 
u sin A mul 
v sin r mul
end]
}{}%
}%
\fcAxesIIIdInScene{4}{4}{2.2}%
\only<1>{%
\fcCurveIIIdInScene[linewidth=2, linecolor=red]{0}{360}{2 dict begin /u 0 def /v t def theT end}%
}%
\only<2>{%
\fcCurveIIIdInScene[linewidth=2, linecolor=red]{0}{360}{2 dict begin /u 120 def /v t def theT end}%
}%
\only<3>{%
\fcCurveIIIdInScene[linewidth=2, linecolor=red]{0}{360}{2 dict begin /u 240 def /v t def theT end}%
}%
\only<4>{%
\fcCurveIIIdInScene[linewidth=2, linecolor=red]{0}{360}{2 dict begin /u 0 def /v t def theT end}%
}%
\only<7>{%
\fcSurfaceInScene[linecolor=black, linewidth=1, arrows=->, iterationsU=24, iterationsV=4 ]{0}{0}{360}{120}{
[5 dict begin 
/R theMajorRadius def
/r theMinorRadius def
/A R r v cos mul add def
u cos A mul 
u sin A mul 
v sin r mul
end]
}{}%
}%
\only<8>{%
\fcSurfaceInScene[linecolor=black, linewidth=1, arrows=->, iterationsU=24, iterationsV=8 ]{0}{0}{360}{240}{
[5 dict begin 
/R theMajorRadius def
/r theMinorRadius def
/A R r v cos mul add def
u cos A mul 
u sin A mul 
v sin r mul
end]
}{}%
}%
\only<11>{%
\fcSurfaceInScene[linecolor=black, linewidth=1, arrows=->, iterationsU=24, iterationsV=12 ]{0}{0}{360}{360}{
[5 dict begin 
/R theMajorRadius def
/r theMinorRadius 0.33 mul def
/A R r v cos mul add def
u cos A mul 
u sin A mul 
v sin r mul
end]
}{}%
}%
\only<12>{%
\fcSurfaceInScene[linecolor=black, linewidth=1, arrows=->, iterationsU=24, iterationsV=12 ]{0}{0}{360}{360}{
[5 dict begin 
/R theMajorRadius def
/r theMinorRadius 0.66 mul def
/A R r v cos mul add def
u cos A mul 
u sin A mul 
v sin r mul
end]
}{}%
}%
\only<7>{\fcCurveIIId[linewidth=2pt, linecolor=red]{0}{360}{2 dict begin /u t def /v 120 def theT end}}%
\only<8>{\fcCurveIIIdInScene[linewidth=2, linecolor=red]{0}{360}{2 dict begin /u t def /v 240 def theT end}}%
\fcFinishIIIdScene[fastsort=true]%
\only<6,9>{%
\fcCurveIIId[arrows=->,linewidth=2pt, linecolor=red]{0}{360}{2 dict begin /u t def /v 0 def theT end}%
}%
\end{pspicture}
\column{0.45\textwidth} 
Find the volume of a toroid $T$ (the inside of a torus $S$) with major radius $R$ and minor radius $r$.

$
S:\left|\begin{array}{r@{}c@{}l}
x&=&(R+r\cos \theta) \cos \phi \\
y&=&(R+r\cos \theta) \sin \phi \\
z&=& r\sin \theta
\end{array}\right.
$
\end{columns}
$
T:\left|\begin{array}{r@{}c@{}l}
x&=&(R+\rho\cos \theta) \cos \phi \\
y&=&(R+\rho\cos \theta) \sin \phi \\
z&=& \rho\sin \theta
\end{array}\right. \rho\in[0, \uncover<1-13>{\alert<10-13>{\textbf{?}}} \uncover<14->{\alert<14>{r}}], \phi\in [0,\uncover<1-4>{\alert<1-4>{\textbf{?}}} \uncover<5->{\alert<5>{2\pi}}), \theta\in [0,\uncover<1-8>{\alert<6-8>{\textbf{?}}}\uncover<9->{\alert<9>{2\pi}}). 
$

\end{example}


\vskip 10cm 
\end{frame}
%\begin{frame}
\begin{columns}
\column{0.4\textwidth}
$
\fcv f:\left|\begin{array}{r@{~}c@{~}l}
\alert<2>{x_1} &=& \alert<2>{f_1}( y_1, \dots, y_n)\\
&\vdots&\\
\alert<2>{x_n} &=& \alert<2>{f_n} ( y_1, \dots, y_n) \quad .
\end{array}\right.
$
\column{0.6\textwidth}
\begin{pspicture}(-0.5, -0.5)(2,2)
\tiny
\fcAxesStandard{-0.5}{-0.5}{2}{2}
\fcLabels[$y_1$][$y_2$]{2}{2}
\multido{\na=6+1}{5}{%
\pstVerb{/y2 \na\space 6 sub 0.3 mul 0.3 add def}%
\psline(! 0.2 y2)(! 1.6 y2)%
\psline(! -0.05 y2)(! 0.05 y2)%
}%
\multido{\na=11+1}{5}{%
\pstVerb{/y1 \na\space 11 sub 0.3 mul 0.3 add def}%
\psline(! y1 0.2)(! y1 1.6)%
\psline(!  y1 -0.05)(! y1 0.05)%
}%
\uncover<3->{\psline[linecolor=red, linewidth=2pt](0.2, 0.6)(1.6, 0.6)}
\uncover<6->{\psline[linecolor=red, linewidth=2pt](1.2, 0.2)(1.2, 1.6)}
\end{pspicture} ~
 \raisebox{1.25cm}{$\stackrel{\fcv f}{\to} $}  ~
\begin{pspicture}(-0.5, -0.5)(2,2)
\tiny
\fcAxesStandard{-0.5}{-0.5}{2.5}{2}
\fcLabels[$x_1$][$x_2$]{2.5}{2}
\pstVerb{
/theF { theta 57.295779513 mul cos r mul theta 57.295779513 mul sin r mul  } def
}
\multido{\na=6+1}{5}{%
\pstVerb{/y2 \na\space 6 sub 0.3 mul 0.3 add def}%
\parametricplot{0.2 }{1.6}{2 dict begin /r t def /theta y2 def theF end} }%
\multido{\na=11+1}{5}{%
\pstVerb{/y1 \na\space 11 sub 0.3 mul 0.3 add def}%
\parametricplot{0.2 }{1.6}{2 dict begin /r y1 def /theta t def theF end}%
}%
\uncover<4->{\parametricplot[linecolor=red, linewidth=2pt]{0.2 }{1.6}{2 dict begin /r t def /theta 0.6 def theF end}}
\uncover<6->{\parametricplot[linecolor=red, linewidth=2pt]{0.2 }{1.6}{2 dict begin /r 1.2 def /theta t def theF end}}
\uncover<6->{%
\fcFullDot{0.6 57.295779513 mul cos 1.2 mul }{0.6 57.295779513 mul sin 1.2 mul}
\psline[linecolor=blue, linewidth=2pt, arrows=->](!
0.6 57.295779513 mul cos 1.2 mul 
0.6 57.295779513 mul sin 1.2 mul
)(!
0.6 57.295779513 mul cos 1.2 mul 
0.6 57.295779513 mul sin 1.2 mul sub
0.6 57.295779513 mul sin 1.2 mul
0.6 57.295779513 mul cos 1.2 mul add
)}%
\uncover<6->{%
\rput[l](0.5, 1.5){$\alert<6>{\left( \frac{\partial x_1 }{\partial y_2},\dots , \frac{\partial x_n}{\partial y_2} \right)}$}
}%
\uncover<5->{%
\psline[linecolor=blue, linewidth=2pt, arrows=->](!
0.6 57.295779513 mul cos 1.2 mul 
0.6 57.295779513 mul sin 1.2 mul
)(!
0.6 57.295779513 mul cos 2.2 mul 
0.6 57.295779513 mul sin 2.2 mul
)}%
\uncover<5->{%
\rput[b](1.4, -0.5){$\alert<5>{\left( \frac{\partial x_1 }{\partial y_1},\dots , \frac{\partial x_n}{\partial y_1} \right)}$}
\psline[linestyle=dotted, arrows=->](1.4, -0.25)(! 0.6 57.295779513 mul cos 1.7 mul 0.6 57.295779513 mul sin 1.7 mul)
}%
\end{pspicture} 
\end{columns}
\begin{definition}[Jacobian matrix]
The Jacobian matrix of a variable change $\fcv f$ is defined as the matrix 
\[
J_{\fcv f}=\left( \begin{array}{ccc} \frac{\partial \alert<2>{f_1}}{\partial y_1} & \cdots & \frac{\partial \alert<2>{f_1}}{\partial y_n} \\ \vdots & \ddots & \vdots \\ \frac{\partial \alert<2>{f_n}}{\partial y_1} & \cdots & \frac{\partial \alert<2>{f_n}}{\partial y_n} \end{array}\right) \uncover<2->{= \left( \begin{array}{ccc} \alert<5>{ \frac{\partial \alert<2>{x_1} }{\partial y_1}} & \cdots & \frac{\partial \alert<2>{x_1} }{\partial y_n} \\ \vdots & \ddots & \vdots \\ \alert<5>{ \frac{\partial \alert<2>{x_n}}{\partial y_1} } & \cdots & \frac{\partial \alert<2>{x_n}}{\partial y_n} \end{array}\right) }
\]
\end{definition}
\begin{itemize}
\item<3-> Consider \alert<4>{curve given by $\fcv f$} with \alert<3>{parameter $y_1$} (other $y_j$'s-fixed).
\item<5-> Then the tangent vector of that curve is $\alert<5>{\left( \frac{\partial x_1 }{\partial y_1},\dots , \frac{\partial x_n}{\partial y_1} \right)}$.
\item<6-> Similar considerations hold for $\alert<6>{y_2}\uncover<7->{\alert<7>{,\dots, y_n}.}$
\end{itemize}


\end{frame}

%\begin{frame}
\frametitle{Properties of determinants}
\begin{itemize}
\item Multiplying a column of a matrix by a number changes multiplies the determinant by the same number. In precise notation:
\begin{lemma}
$\left|\begin{array}{ccccc}
a_{11} &   \dots & x a_{1k} & \dots & a_{1n}\\
a_{21} &   \dots & x a_{2k} & \dots & a_{2n}\\
\vdots \\
a_{n1} &   \dots & x a_{nk} & \dots & a_{nn}
\end{array} \right| = 
x \left|\begin{array}{ccccc}
a_{11} &   \dots & a_{1k} & \dots & a_{1n}\\
a_{21} &   \dots & a_{2k} & \dots & a_{2n}\\
\vdots \\
a_{n1} &   \dots & a_{nk} & \dots & a_{nn}
\end{array} \right| $
\end{lemma}

\end{itemize}
\end{frame}
%\begin{frame}
\frametitle{Variable change in multivariable integrals}
\begin{columns}
\column{0.4\textwidth}
$
\fcv f:\left|\begin{array}{r@{~}c@{~}l}
x_1 &=& f_1( y_1, \dots, y_n)\\
&\vdots&\\
x_n &=& f_n ( y_1, \dots, y_n) 
\end{array}\right.
$

$J_{\fcv f}=\left( \begin{array}{ccc}  \frac{\partial x_1 }{\partial y_1} & \cdots & \frac{\partial x_1 }{\partial y_n} \\ \vdots & \ddots & \vdots \\  \frac{\partial x_n}{\partial y_1}  & \cdots & \frac{\partial x_n}{\partial y_n} \end{array}\right) $
\column{0.6\textwidth}
\begin{pspicture}(-0.5, -0.5)(2,2)
\tiny%
\fcAxesStandard{-1}{-1}{2}{2}%
\fcLabels[$y_1$][$y_2$]{2}{2}%
\pstVerb{%
/Delta 0.3 def
/base1 1.2 def
/base2 0.6 def
}%
\uncover<4->{%
\psline*[linecolor=pink](! base1 base2 )(! base1 Delta add base2 )(! base1 Delta add base2 Delta add )(! base1 base2 Delta add)(! base1 base2)%
}%
\multido{\na=6+1}{5}{%
\pstVerb{/y2 \na\space 6 sub Delta mul Delta add def}%
\psline(! 0.2 y2)(! 1.6 y2)%
\psline(! -0.05 y2)(! 0.05 y2)%
}%
\multido{\na=11+1}{5}{%
\pstVerb{/y1 \na\space 11 sub Delta mul Delta add def}%
\psline(! y1 0.2)(! y1 1.6)%
\psline(!  y1 -0.05)(! y1 0.05)%
}%
\only<1>{%
\psline[arrows=->, linecolor=red, linewidth=1.5pt](0,0)(1,0)%
\psline[arrows=->, linecolor=red, linewidth=1.5pt](0,0)(0,1)%
\rput[t](0.5, -0.1){$\alert<1>{\fcv e_1}$}%
\rput[r](-0.05, 0.5){$\alert<1>{\fcv e_2~}$}%
}%
\uncover<2->{\fcFullDot[linecolor=brown]{base1}{base2}}%
\uncover<5->{%
\fcFullDot[linecolor=brown]{base1 Delta add}{base2}%
\fcFullDot[linecolor=brown]{base1}{base2 Delta add}%
}%
\uncover<3->{%
\psline[arrows=|-|] (0.3, -0.3)(0.6,-0.3)%
\rput[t](0.45, -0.45){$\Delta y_1$}%
\psline[arrows=|-|] (-0.3,0.3)(-0.3,0.6)%
\rput[r]( -0.45, 0.45){$\Delta y_2$}%
}%
\uncover<4->{%
\rput[b](1.6,0.7){$~B$}%
}%
\end{pspicture} ~
 \raisebox{1.25cm}{$\stackrel{\fcv f}{\to} $}  ~
\begin{pspicture}(-0.5, -0.5)(2,2)
\tiny
\fcAxesStandard{-0.5}{-1}{2.5}{2}
\fcLabels[$x_1$][$x_2$]{2.5}{2}
\pstVerb{
/theF { theta 57.295779513 mul cos r mul theta 57.295779513 mul sin r mul  } def
/Delta 0.3 def
/base1 1.2 def
/base2 0.6 def
}
\uncover<7->{%
\pscustom*[linecolor=blue]{%
\parametricplot{base1}{base1 Delta add}{2 dict begin /r t def /theta base2 def theF end}%
\parametricplot{base2}{base2 Delta add}{2 dict begin /r base1 Delta add  def /theta t def theF end}%
\parametricplot{base1 Delta add}{base1}{2 dict begin /r t def /theta base2 Delta add def theF end}%
\parametricplot{base2 Delta add}{base2}{2 dict begin /r base1 def /theta t def theF end}%
}%
}%
\multido{\na=6+1}{5}{%
\pstVerb{/y2 \na\space 6 sub 0.3 mul 0.3 add def}%
\parametricplot{0.2 }{1.6}{2 dict begin /r t def /theta y2 def theF end} }%
\multido{\na=11+1}{5}{%
\pstVerb{/y1 \na\space 11 sub 0.3 mul 0.3 add def}%
\parametricplot{0.2 }{1.6}{2 dict begin /r y1 def /theta t def theF end}%
}%
\uncover<8->{%
\fcFullDot[linecolor=magenta]{2 dict begin /r base1 def /theta base2 def theF pop end}{2 dict begin /r base1 def /theta base2 def theF exch pop end}%
\fcFullDot[linecolor=magenta]{2 dict begin /r base1 Delta add def /theta base2 def theF pop end}{2 dict begin /r base1 Delta add def /theta base2 def theF exch pop end}%
\fcFullDot[linecolor=magenta]{2 dict begin /r base1 def /theta base2 Delta add def theF pop end}{2 dict begin /r base1 def /theta base2 Delta add def theF exch pop end}%
}%
\uncover<8->{%
\pscustom*[linecolor=green]{%
\psline
(! 2 dict begin /r base1 def /theta base2 def theF end)
(! 2 dict begin /r base1 Delta add def /theta base2 def theF end)
(! [2 dict begin /r base1 def /theta base2 def theF end]
   [2 dict begin /r base1 Delta add def /theta base2 def theF end]
   [2 dict begin /r base1 def /theta base2 Delta add def theF end]
\fcVectorPlusVector exch \fcVectorMinusVector \fcArrayToStack 
)
(! 2 dict begin /r base1 def /theta base2 Delta add def theF end)
(! 2 dict begin /r base1 def /theta base2 def theF end)
}
}%
\uncover<7->{\rput[bl](1.1,1.1){$\uncover<8->{{\color{green}{E}}}\uncover<9->{\approx}{\color{blue}C}$}}%
\end{pspicture} 

\end{columns}

 
\begin{itemize}
\item Let $\fcv e_1,\dots, \fcv e_n$ be the basis vectors. \uncover<2->{Fix a point $\fcv y=(y_1,\dots, y_n)$.}
\item<3-> Let $\Delta y_1,\dots \Delta y_n$ be small numbers. \uncover<4->{Construct a small box $B$ with corner $\fcv y$ spanned by the vectors $\Delta y_1 \fcv e_1,\dots, \Delta y_n \fcv e_n$.} 
\item<5-> Knowing the point $\fcv y$ and the corners $\fcv y +\Delta y_1 \fcv e_1,\dots, \fcv y+\Delta y_n\fcv e_n$ suffices to identify $B$. Mark those corners.
\item<6-> $\Vol (B)=\Delta y_1, \dots, \Delta y_n$. 

\item<7-> Let the image of $B$ be $f(B)={\color{blue}C}$. ${\color{blue}C} $ is a ``curvilinear box''. 
\item<8-> Let ${\color{green}E}$ be the parrallelotope at $\fcv f(\fcv y)$ spanned by images of the marked corners of $B$. \uncover<9,10->{\alert<10>{Then $\Vol ({\color{blue}C})\approx \Vol_n ({\color{green}E})$.}}
\end{itemize}

\vskip 10cm
\end{frame}



%\begin{frame}
\frametitle{Variable change in multivariable integrals}
\begin{columns}
\column{0.4\textwidth}
$
\fcv f:\left|\begin{array}{r@{~}c@{~}l}
x_1 &=& f_1( y_1, \dots, y_n)\\
&\vdots&\\
x_n &=& f_n ( y_1, \dots, y_n) \quad .
\end{array}\right.
$

$J_{\fcv f}=\left( \begin{array}{ccc}  \frac{\partial x_1 }{\partial y_1} & \cdots & \frac{\partial x_1 }{\partial y_n} \\ \vdots & \ddots & \vdots \\  \frac{\partial x_n}{\partial y_1}  & \cdots & \frac{\partial x_n}{\partial y_n} \end{array}\right) $
\column{0.6\textwidth}
\begin{pspicture}(-0.5, -0.5)(2,2)
\tiny%
\fcAxesStandard{-1}{-1}{2}{2}%
\fcLabels[$y_1$][$y_2$]{2}{2}%
\pstVerb{%
/Delta 0.3 def
/base1 1.2 def
/base2 0.6 def
}%
\psline*[linecolor=pink](! base1 base2 )(! base1 Delta add base2 )(! base1 Delta add base2 Delta add )(! base1 base2 Delta add)(! base1 base2)%
\multido{\na=6+1}{5}{%
\pstVerb{/y2 \na\space 6 sub Delta mul Delta add def}%
\psline(! 0.2 y2)(! 1.6 y2)%
\psline(! -0.05 y2)(! 0.05 y2)%
}%
\multido{\na=11+1}{5}{%
\pstVerb{/y1 \na\space 11 sub Delta mul Delta add def}%
\psline(! y1 0.2)(! y1 1.6)%
\psline(!  y1 -0.05)(! y1 0.05)%
}%
\fcFullDot[linecolor=brown]{base1}{base2}%
\uncover<4>{\fcFullDot[linewidth=0.09, linecolor=red]{base1}{base2}}%
\fcFullDot[linecolor=brown]{base1 Delta add}{base2}%
\uncover<6>{\fcFullDot[linewidth=0.09, linecolor=red]{base1 Delta add}{base2}}%
\fcFullDot[linecolor=brown]{base1}{base2 Delta add}%
\psline[arrows=|-|] (0.3, -0.3)(0.6,-0.3)%
\rput[t](0.45, -0.45){$\Delta y_1$}%
\psline[arrows=|-|] (-0.3,0.3)(-0.3,0.6)%
\rput[r]( -0.45, 0.45){$\Delta y_2$}%
\rput[b](1.6,0.7){$~B$}%
\end{pspicture} ~
 \raisebox{1.25cm}{$\stackrel{\fcv f}{\to} $}  ~
\begin{pspicture}(-0.5, -0.5)(2,2)
\tiny
\fcAxesStandard{-0.5}{-1}{2.5}{2}
\fcLabels[$x_1$][$x_2$]{2.5}{2}
\pstVerb{
/theF { theta 57.295779513 mul cos r mul theta 57.295779513 mul sin r mul  } def
/Delta 0.3 def
/base1 1.2 def
/base2 0.6 def
/basePoint [2 dict begin /r base1 def /theta base2 def theF end] def
/tangent2 [base2 57.295779513 mul sin -1 mul base1 mul base2 57.295779513 mul cos base1 mul] def
/tangent1 [base2 57.295779513 mul cos base2 57.295779513 mul sin] def
}
\pscustom*[linecolor=blue]{%
\parametricplot{base1}{base1 Delta add}{2 dict begin /r t def /theta base2 def theF end}%
\parametricplot{base2}{base2 Delta add}{2 dict begin /r base1 Delta add  def /theta t def theF end}%
\parametricplot{base1 Delta add}{base1}{2 dict begin /r t def /theta base2 Delta add def theF end}%
\parametricplot{base2 Delta add}{base2}{2 dict begin /r base1 def /theta t def theF end}%
}%
\multido{\na=6+1}{5}{%
\pstVerb{/y2 \na\space 6 sub 0.3 mul 0.3 add def}%
\parametricplot{0.2 }{1.6}{2 dict begin /r t def /theta y2 def theF end} }%
\multido{\na=11+1}{5}{%
\pstVerb{/y1 \na\space 11 sub 0.3 mul 0.3 add def}%
\parametricplot{0.2 }{1.6}{2 dict begin /r y1 def /theta t def theF end}%
}%
\fcFullDot[linecolor=magenta]{2 dict begin /r base1 def /theta base2 def theF pop end}{2 dict begin /r base1 def /theta base2 def theF exch pop end}%
\fcFullDot[linecolor=magenta]{2 dict begin /r base1 Delta add def /theta base2 def theF pop end}{2 dict begin /r base1 Delta add def /theta base2 def theF exch pop end}%
\fcFullDot[linecolor=magenta]{2 dict begin /r base1 def /theta base2 Delta add def theF pop end}{2 dict begin /r base1 def /theta base2 Delta add def theF exch pop end}%
\pscustom*[linecolor=green]{%
\psline
(! 2 dict begin /r base1 def /theta base2 def theF end)
(! 2 dict begin /r base1 Delta add def /theta base2 def theF end)
(! [2 dict begin /r base1 def /theta base2 def theF end]
   [2 dict begin /r base1 Delta add def /theta base2 def theF end]
   [2 dict begin /r base1 def /theta base2 Delta add def theF end]
\fcVectorPlusVector exch \fcVectorMinusVector \fcArrayToStack 
)
(! 2 dict begin /r base1 def /theta base2 Delta add def theF end)
(! 2 dict begin /r base1 def /theta base2 def theF end)
}
\rput[bl](1.1,1.1){${\color{green}{E}}\approx{\color{blue}C} \uncover<16->{ \approx } \uncover<15->{{\color{red}J}}$}%
\uncover<2>{%
\psline[linecolor=red, arrows=->, linewidth=1.5pt]
(! 2 dict begin /r base1 def /theta base2 def theF end)
(! 2 dict begin /r base1 Delta add def /theta base2 def theF end)%
}%
\uncover<3-14>{%
\psline[linecolor=magenta, arrows=->, linewidth=1.5pt]
(! 2 dict begin /r base1 def /theta base2 def theF end)
(! 2 dict begin /r base1 Delta add def /theta base2 def theF end)%
}%
\uncover<5>{%
\fcFullDot[linecolor=red, linewidth=0.09pt]{2 dict begin /r base1 def /theta base2 def theF pop end}{2 dict begin /r base1 def /theta base2 def theF exch pop end}%
}%
\uncover<7>{%
\fcFullDot[linecolor=red, linewidth=0.09pt]{2 dict begin /r base1 Delta add def /theta base2 def theF pop end}{2 dict begin /r base1 Delta add def /theta base2 def theF exch pop end}%
}%
\uncover<12-14>{%
\psline[linecolor=magenta, arrows=->, linewidth=1.5pt]
(! 2 dict begin /r base1 def /theta base2 def theF end)
(! 2 dict begin /r base1 def /theta base2 Delta add def theF end)%
}%
\uncover<13-14>{%
\psline[linecolor=blue, linewidth=2pt, arrows=->](! basePoint \fcArrayToStack)
(! basePoint tangent2 \fcVectorPlusVector \fcArrayToStack)%
}%
\uncover<14>{%
\psline[linecolor=red, linewidth=2pt, arrows=->](! basePoint \fcArrayToStack)
(! basePoint tangent2 Delta \fcVectorTimesScalar \fcVectorPlusVector \fcArrayToStack)%
}%
\uncover<10-14>{%
\psline[linecolor=blue, linewidth=2pt, arrows=->](! basePoint \fcArrayToStack)(! basePoint tangent1 \fcVectorPlusVector \fcArrayToStack)}%
\uncover<11-14>{%
\psline[linecolor=red, linewidth=2pt, arrows=->](! basePoint \fcArrayToStack)(! basePoint tangent1 Delta \fcVectorTimesScalar \fcVectorPlusVector \fcArrayToStack)}%
\uncover<15->{%
\psline*[linecolor=red]
(! basePoint \fcArrayToStack)
(! basePoint tangent1 Delta \fcVectorTimesScalar \fcVectorPlusVector \fcArrayToStack)
(! basePoint tangent1 tangent2 \fcVectorPlusVector Delta \fcVectorTimesScalar \fcVectorPlusVector \fcArrayToStack)
(! basePoint tangent2 Delta \fcVectorTimesScalar \fcVectorPlusVector \fcArrayToStack)
(! basePoint \fcArrayToStack)
}%
\end{pspicture} 
\end{columns}

\begin{itemize}
\item $\alert<1>{\Vol ({\color{blue}C})\approx \Vol_n({\color{green}E})}$.
\item<2-> The first edge of ${\color{green}E}$ \uncover<3->{corresponds to the vector }
$\begin{array}{r@{}c@{}l}
\uncover<3->{\alert<7>{\fcv f(\alert<6>{ \fcv y+\Delta y_1 \fcv e_1})} - \alert<5>{\fcv f( \alert<4>{\fcv y} )}} &\uncover<8->{\approx}&\uncover<8->{ \Delta y_1 \alert<10>{ \left(D_{\fcv e_1}\left( \fcv f(\fcv y)\right)\right)}} \uncover<9->{=\Delta y_1 \alert<10>{\frac{\partial \fcv f}{\partial y_1}}}\\
\uncover<10->{&=& \Delta y_1 \alert<10>{\left(\frac{\partial x_1}{\partial y_1},\dots, \frac{ \partial x_n}{\partial y_1}  \right)}} \uncover<11->{ = \alert<11>{\left(\Delta y_1 \frac{ \partial x_1}{\partial y_1},\dots, \Delta y_1 \frac{\partial x_n}{\partial y_1}\right)}.}
\end{array}
$
\item<12-> Similar considerations holds for the other edges of $E$.
\item<15-> \alert<17>{ Let ${\color{red}J}$ be the parallelotope at $\fcv f(\fcv y)$ spanned by the vectors $\Delta y_1 \left( \frac{\partial x_1}{\partial y_1},\dots, \frac{\partial x_n}{ \partial y_1} \right)$,\dots $\Delta y_n \left(\frac{\partial x_1}{\partial y_n},\dots, \frac{ \partial x_n}{\partial y_n}\right) $.}
\item<16-> Then $\alert<17>{\Vol ({\color{blue}C}) \approx \Vol_n ( {\color{green}E} ) \approx \Vol_n({\color{red}J})}$.
\end{itemize}


\vskip 10cm
\end{frame}

%

\begin{frame}
\frametitle{Variable change in multivariable integrals}
\begin{columns}
\column{0.4\textwidth}
$
\fcv f:\left|\begin{array}{r@{~}c@{~}l}
x_1 &=& f_1( y_1, \dots, y_n)\\
&\vdots&\\
x_n &=& f_n ( y_1, \dots, y_n) \quad .
\end{array}\right.
$

$J_{\fcv f}=\left( \begin{array}{ccc}  \frac{\partial x_1 }{\partial y_1} & \cdots & \frac{\partial x_1 }{\partial y_n} \\ \vdots & \ddots & \vdots \\  \frac{\partial x_n}{\partial y_1}  & \cdots & \frac{\partial x_n}{\partial y_n} \end{array}\right) $
\column{0.6\textwidth}
\begin{pspicture}(-1, -1)(2.1,2.1)
\tiny%
\fcAxesStandard{-1}{-1}{2}{2}%
\fcLabels[$y_1$][$y_2$]{2}{2}%
\pstVerb{%
/Delta 0.3 def
/base1 1.2 def
/base2 0.6 def
}%
\psline*[linecolor=pink](! base1 base2 )(! base1 Delta add base2 )(! base1 Delta add base2 Delta add )(! base1 base2 Delta add)(! base1 base2)%
\multido{\na=6+1}{5}{%
\pstVerb{/y2 \na\space 6 sub Delta mul Delta add def}%
\psline(! 0.2 y2)(! 1.6 y2)%
\psline(! -0.05 y2)(! 0.05 y2)%
}%
\multido{\na=11+1}{5}{%
\pstVerb{/y1 \na\space 11 sub Delta mul Delta add def}%
\psline(! y1 0.2)(! y1 1.6)%
\psline(!  y1 -0.05)(! y1 0.05)%
}%
%\fcFullDot[linecolor=brown]{base1}{base2}%
%\fcFullDot[linecolor=brown]{base1 Delta add}{base2}%
%\fcFullDot[linecolor=brown]{base1}{base2 Delta add}%
\psline[arrows=|-|] (0.3, -0.3)(0.6,-0.3)%
\rput[t](0.45, -0.45){$\Delta y_1$}%
\psline[arrows=|-|] (-0.3,0.3)(-0.3,0.6)%
\rput[r]( -0.45, 0.45){$\Delta y_2$}%
\rput[b](1.6,0.7){$~B$}%
\end{pspicture} ~
 \raisebox{1.25cm}{$\stackrel{\fcv f}{\to} $}  ~
\begin{pspicture}(-0.5, -1)(2.25,2.1)
\tiny
\fcAxesStandard{-0.5}{-1}{2.2}{2}
\fcLabels[$x_1$][$x_2$]{2.2}{2}
\pstVerb{
/theF { theta 57.295779513 mul cos r mul theta 57.295779513 mul sin r mul  } def
/Delta 0.3 def
/base1 1.2 def
/base2 0.6 def
/basePoint [2 dict begin /r base1 def /theta base2 def theF end] def
/tangent2 [base2 57.295779513 mul sin -1 mul base1 mul base2 57.295779513 mul cos base1 mul] def
/tangent1 [base2 57.295779513 mul cos base2 57.295779513 mul sin] def
}
\pscustom*[linecolor=blue]{%
\parametricplot{base1}{base1 Delta add}{2 dict begin /r t def /theta base2 def theF end}%
\parametricplot{base2}{base2 Delta add}{2 dict begin /r base1 Delta add  def /theta t def theF end}%
\parametricplot{base1 Delta add}{base1}{2 dict begin /r t def /theta base2 Delta add def theF end}%
\parametricplot{base2 Delta add}{base2}{2 dict begin /r base1 def /theta t def theF end}%
}%
\multido{\na=6+1}{5}{%
\pstVerb{/y2 \na\space 6 sub 0.3 mul 0.3 add def}%
\parametricplot{0.2 }{1.6}{2 dict begin /r t def /theta y2 def theF end} }%
\multido{\na=11+1}{5}{%
\pstVerb{/y1 \na\space 11 sub 0.3 mul 0.3 add def}%
\parametricplot{0.2 }{1.6}{2 dict begin /r y1 def /theta t def theF end}%
}%
%\fcFullDot[linecolor=magenta]{2 dict begin /r base1 def /theta base2 def theF pop end}{2 dict begin /r base1 def /theta base2 def theF exch pop end}%
%\fcFullDot[linecolor=magenta]{2 dict begin /r base1 Delta add def /theta base2 def theF pop end}{2 dict begin /r base1 Delta add def /theta base2 def theF exch pop end}%
%\fcFullDot[linecolor=magenta]{2 dict begin /r base1 def /theta base2 Delta add def theF pop end}{2 dict begin /r base1 def /theta base2 Delta add def theF exch pop end}%
\pscustom*[linecolor=green]{%
\psline
(! 2 dict begin /r base1 def /theta base2 def theF end)
(! 2 dict begin /r base1 Delta add def /theta base2 def theF end)
(! [2 dict begin /r base1 def /theta base2 def theF end]
   [2 dict begin /r base1 Delta add def /theta base2 def theF end]
   [2 dict begin /r base1 def /theta base2 Delta add def theF end]
\fcVectorPlusVector exch \fcVectorMinusVector \fcArrayToStack 
)
(! 2 dict begin /r base1 def /theta base2 Delta add def theF end)
(! 2 dict begin /r base1 def /theta base2 def theF end)
}
\rput[bl](1.1,1.1){${\color{green}{E}}\approx{\color{blue}C} \approx{\color{red}J}$}%
\psline*[linecolor=red]
(! basePoint \fcArrayToStack)
(! basePoint tangent1 Delta \fcVectorTimesScalar \fcVectorPlusVector \fcArrayToStack)
(! basePoint tangent1 tangent2 \fcVectorPlusVector Delta \fcVectorTimesScalar \fcVectorPlusVector \fcArrayToStack)
(! basePoint tangent2 Delta \fcVectorTimesScalar \fcVectorPlusVector \fcArrayToStack)
(! basePoint \fcArrayToStack)
\end{pspicture} 
\end{columns}
\begin{itemize}
\item \alert<1>{Let $J$ be the parallelotope at $\fcv f(\fcv y)$ spanned by the vectors $\alert<2>{\Delta y_1 \left( \frac{\partial x_1}{\partial y_1},\dots, \frac{\partial x_n}{\partial y_1} \right)}$, \dots, $\alert<3>{ \Delta y_n \left(\frac{\partial x_1}{\partial y_n},\dots, \frac{\partial x_n}{\partial y_n}\right) }$.} \uncover<7->{Suppose $\det J_{\fcv f}\geq 0$. }
\item \alert<1>{$\alert<8>{ \Vol ({\color{blue}C}) \approx} \Vol_n ( {\color{green}E} ) \approx \Vol_n({\color{red}J}) \uncover<8->{= \alert<8>{ \det J_{\fcv f} \Delta y_1\dots \Delta y_n}}$}
\end{itemize}
$
\begin{array}{r@{}c@{}l}
\uncover<2->{\Vol_n(J)&=& \alert<7>{\pm} \left|\begin{array}{ccc}  \alert<2>{\alert<4>{ \Delta y_1} \frac{\partial x_1 }{\partial y_1}} & \cdots &\alert<3>{ \alert<5>{\Delta y_n} \frac{\partial x_1 }{\partial y_n}} \\ \vdots & \ddots & \vdots \\ \alert<2>{ \alert<4>{\Delta y_1}  \frac{\partial x_n}{\partial y_1}}  & \cdots &\alert<3>{ \alert<5>{\Delta y_n} \frac{\partial x_n}{\partial y_n}} \end{array} \right|}\! \uncover<4->{=\!\alert<7>{\pm} \alert<4>{ \Delta y_1} \dots \alert<5>{\Delta y_n} \alert<6>{ \left| \begin{array}{ccc}  \frac{ \partial x_1 }{\partial y_1} & \cdots & \frac{\partial x_1 }{\partial y_n} \\ \vdots & \ddots & \vdots \\  \frac{\partial x_n}{\partial y_1}  & \cdots & \frac{\partial x_n}{\partial y_n} \end{array}\right|}} \\
\uncover<6->{&=& \alert<7>{\pm} \alert<6>{\det \left(J_{\fcv f}\right)} \Delta y_1\dots \Delta y_n \quad .}
\end{array}
$

\vskip 10cm
\end{frame}
%

\begin{frame}
\frametitle{Variable change in multivariable integrals}
\begin{columns}
\column{0.4\textwidth}
$
\fcv f:\left|\begin{array}{r@{~}c@{~}l}
x_1 &=& f_1( y_1, \dots, y_n)\\
&\vdots&\\
x_n &=& f_n ( y_1, \dots, y_n) \quad .
\end{array}\right.
$

$J_{\fcv f}=\left( \begin{array}{ccc}  \frac{\partial x_1 }{\partial y_1} & \cdots & \frac{\partial x_1 }{\partial y_n} \\ \vdots & \ddots & \vdots \\  \frac{\partial x_n}{\partial y_1}  & \cdots & \frac{\partial x_n}{\partial y_n} \end{array}\right) $
\column{0.6\textwidth}
\begin{pspicture}(-0.5, -0.5)(2,2)
\newcommand{\numIterations}{5}%
\newcommand{\numIterationsMinusOne}{4}%
\only<9>{%
\renewcommand{\numIterations}{8}%
\renewcommand{\numIterationsMinusOne}{7}%
}%
\only<10->{%
\renewcommand{\numIterations}{12}%
\renewcommand{\numIterationsMinusOne}{11}%
}%
\tiny%
\fcAxesStandard{-1}{-1}{2}{2}%
\fcLabels[$y_1$][$y_2$]{2}{2}%
\pstVerb{%
/Delta 1.2 \numIterationsMinusOne\space div def
/base1 1.2 def
/base2 0.6 def
}%
\uncover<1-2>{%
\psline*[linecolor=pink](! base1 base2 )(! base1 Delta add base2 )(! base1 Delta add base2 Delta add )(! base1 base2 Delta add)(! base1 base2)
}%
\uncover<3->{%
\multido{\na=0+1}{\numIterationsMinusOne}{%
\pstVerb{/y1 \na\space Delta mul 0.3 add def}%
\multido{\na=0+1}{\numIterationsMinusOne}{%
\pstVerb{/y2 \na\space Delta mul 0.3 add def}%
\pscustom*[linecolor=pink]{%
\psline(! y1 y2)(! y1 Delta add y2)(! y1 Delta add y2 Delta add)(! y1 y2 Delta add)(! y1 y2)%
}%
}%
}%
}%
\uncover<5->{%
\rput[b](1, 1.7){$\mathcal R$}%
}%
\multido{\na=0+1}{\numIterations}{%
\pstVerb{/y2 \na\space Delta mul 0.3 add def}%
\psline(! 0.2 y2)(! 1.6 y2)%
\psline(! -0.05 y2)(! 0.05 y2)%
}%
\multido{\na=0+1}{\numIterations}{%
\pstVerb{/y1 \na\space Delta mul 0.3 add def}%
\psline(! y1 0.2)(! y1 1.6)%
\psline(!  y1 -0.05)(! y1 0.05)%
}%
\uncover<3-7>{%
\multido{\na=0+1}{\numIterationsMinusOne}{%
\pstVerb{/y1 \na\space Delta mul 0.3 add def}%
\multido{\na=0+1}{\numIterationsMinusOne}{%
\pstVerb{/y2 \na\space Delta mul 0.3 add def}%
\fcFullDot{y1}{y2}
}%
}%
}%
\uncover<2>{%
\fcFullDot{base1}{base2}%
}%
%\fcFullDot[linecolor=brown]{base1}{base2}%
%\fcFullDot[linecolor=brown]{base1 Delta add}{base2}%
%\fcFullDot[linecolor=brown]{base1}{base2 Delta add}%
\psline[arrows=|-|] (0.3, -0.3)(!0.3 Delta add -0.3)%
\rput[t](! 0.3 Delta 2 div add -0.45){$\Delta y_1$}%
\psline[arrows=|-|] (-0.3,0.3)(!-0.3 0.3 Delta add)%
\rput[r](! -0.45 0.3 Delta 2 div add){$\Delta y_2$}%
\uncover<1-7>{\rput[b](1.6,0.7){$~B$}}%
\end{pspicture} ~
 \raisebox{1.25cm}{$\stackrel{\fcv f}{\to} $}  ~
\begin{pspicture}(-0.5, -0.5)(2,2)
\newcommand{\numIterations}{5}%
\newcommand{\numIterationsMinusOne}{4}%
\only<9>{%
\renewcommand{\numIterations}{8}%
\renewcommand{\numIterationsMinusOne}{7}%
}%
\only<10->{%
\renewcommand{\numIterations}{12}%
\renewcommand{\numIterationsMinusOne}{11}%
}%
\tiny
\fcAxesStandard{-0.5}{-1}{2.5}{2}
\fcLabels[$x_1$][$x_2$]{2.5}{2}
\pstVerb{
/theF { theta 57.295779513 mul cos r mul theta 57.295779513 mul sin r mul  } def
/Delta 1.2 \numIterationsMinusOne\space div def
/base1 1.2 def
/base2 0.6 def
/basePoint [2 dict begin /r base1 def /theta base2 def theF end] def
/tangent2 [base2 57.295779513 mul sin -1 mul base1 mul base2 57.295779513 mul cos base1 mul] def
/tangent1 [base2 57.295779513 mul cos base2 57.295779513 mul sin] def
}
\pscustom*[linecolor=blue]{%
\parametricplot{base1}{base1 Delta add}{2 dict begin /r t def /theta base2 def theF end}%
\parametricplot{base2}{base2 Delta add}{2 dict begin /r base1 Delta add  def /theta t def theF end}%
\parametricplot{base1 Delta add}{base1}{2 dict begin /r t def /theta base2 Delta add def theF end}%
\parametricplot{base2 Delta add}{base2}{2 dict begin /r base1 def /theta t def theF end}%
}%
\uncover<5->{%
\pscustom*[linecolor=blue]{%
\parametricplot{0.3}{0.3 1.2 add}{2 dict begin /r t def /theta 0.3 def theF end}%
\parametricplot{0.3}{0.3 1.2 add}{2 dict begin /r 0.3 1.2 add  def /theta t def theF end}%
\parametricplot{0.3 1.2 add}{0.3}{2 dict begin /r t def /theta 0.3 1.2 add def theF end}%
\parametricplot{0.3 1.2 add}{0.3}{2 dict begin /r 0.3 def /theta t def theF end}%
}%
}%
\multido{\na=0+1}{\numIterations}{%
\pstVerb{/y2 \na\space Delta mul 0.3 add def}%
\parametricplot{0.2 }{1.6}{2 dict begin /r t def /theta y2 def theF end}% 
}%
\multido{\na=0+1}{\numIterations}{%
\pstVerb{/y1 \na\space Delta mul 0.3 add def}%
\parametricplot{0.2 }{1.6}{2 dict begin /r y1 def /theta t def theF end}%
}%
\uncover<1-5>{%
\rput[bl](1.1,1.1){$\phantom{E\approx} {\color{blue}C} \approx{\color{red}J}$}%
}%
\uncover<5->{%
\rput[lb](0.6, 1.6){$ \mathcal S =\fcv f(\mathcal R)$}
}%
\uncover<1-5>{%
\psline*[linecolor=red]
(! basePoint \fcArrayToStack)
(! basePoint tangent1 Delta \fcVectorTimesScalar \fcVectorPlusVector \fcArrayToStack)
(! basePoint tangent1 tangent2 \fcVectorPlusVector Delta \fcVectorTimesScalar \fcVectorPlusVector \fcArrayToStack)
(! basePoint tangent2 Delta \fcVectorTimesScalar \fcVectorPlusVector \fcArrayToStack)
(! basePoint \fcArrayToStack)
}%
\only<6->{%
\multido{\na=0+1}{\numIterationsMinusOne}{%
\pstVerb{/y2 \na\space Delta mul 0.3 add def}%
\multido{\nb=0+1}{\numIterationsMinusOne}{%
\pstVerb{/y1 \nb\space Delta mul 0.3 add def}%
\pstVerb{%
/r y1 def 
/theta y2 def
/basePoint [theF] def
/tangent1 basePoint 1 r div \fcVectorTimesScalar def
/tangent2 [theF -1 mul exch ] def
}%
\psline*[linecolor=red]
(! basePoint \fcArrayToStack)
(! basePoint tangent1 Delta \fcVectorTimesScalar \fcVectorPlusVector \fcArrayToStack)
(! basePoint tangent1 tangent2 \fcVectorPlusVector Delta \fcVectorTimesScalar \fcVectorPlusVector \fcArrayToStack)
(! basePoint tangent2 Delta \fcVectorTimesScalar \fcVectorPlusVector \fcArrayToStack)
(! basePoint \fcArrayToStack)
}%
}%
}%
\end{pspicture} 
\end{columns}
\begin{itemize}
\item \alert<1>{
$
\begin{array}{rcl}
\displaystyle \uncover<5->{\alert<5,7>{ \Vol(\mathcal S) =}} \alert<5>{ \uncover<4->{ \sum_{\fcv y}}  \Vol ({\color{blue}C\uncover<2->{ \alert<2>{ ( \fcv y)}} }  ) }&\approx& \displaystyle  \alert<6>{ \uncover<4->{ \sum_{\fcv y}} \det \left( J_{\fcv f} \uncover<2->{\alert<2>{(\fcv y)}}\right) \Delta y_1\dots \Delta y_n}\\
\displaystyle \uncover<7->{\alert<7,13>{ \int_{\mathcal S} 1\cdot \diff x_1 \dots \diff x_n}} &\uncover<12->{\alert<12,13>{=}} &\displaystyle \uncover<11->{\alert<13>{ \int \dots\int_{\mathcal R} \det\left( J_{\fcv f} (y) \right)~\diff y_1\dots \diff y_n } }
\end{array}
$
}
\item<2-> \alert<2>{Regard $\fcv y$ as a variable} \uncover<3->{\alert<3>{and let it traverse a rectangular mesh.}}
\item<4-> Sum over the rectangular mesh.
\item<8-> Let $\Delta y_1\to 0, \dots, \Delta y_n\to 0$.\uncover<11>{}
\end{itemize}

\vskip 10cm
\end{frame}
%\begin{frame}
\frametitle{Variable change in multivariable integrals}
\begin{columns}
\column{0.4\textwidth}
$
\fcv f:\left|\begin{array}{r@{~}c@{~}l}
x_1 &=& f_1( y_1, \dots, y_n)\\
&\vdots&\\
x_n &=& f_n ( y_1, \dots, y_n) \quad .
\end{array}\right.
$

$J_{\fcv f}=\left( \begin{array}{ccc}  \frac{\partial x_1 }{\partial y_1} & \cdots & \frac{\partial x_1 }{\partial y_n} \\ \vdots & \ddots & \vdots \\  \frac{\partial x_n}{\partial y_1}  & \cdots & \frac{\partial x_n}{\partial y_n} \end{array}\right) $
\column{0.6\textwidth}
\begin{pspicture}(-1, -1)(2.1,2.1)
\newcommand{\numIterations}{12}%
\newcommand{\numIterationsMinusOne}{11}%
\tiny%
\fcAxesStandard{-1}{-1}{2}{2}%
\fcLabels[$y_1$][$y_2$]{2}{2}%
\pstVerb{%
/Delta 1.2 \numIterationsMinusOne\space div def
/base1 1.2 def
/base2 0.6 def
}%
\multido{\na=0+1}{\numIterationsMinusOne}{%
\pstVerb{/y1 \na\space Delta mul 0.3 add def}%
\multido{\na=0+1}{\numIterationsMinusOne}{%
\pstVerb{/y2 \na\space Delta mul 0.3 add def}%
\pscustom*[linecolor=pink]{%
\psline(! y1 y2)(! y1 Delta add y2)(! y1 Delta add y2 Delta add)(! y1 y2 Delta add)(! y1 y2)%
}%
}%
}%
\rput[b](1, 1.7){$\mathcal R$}%
\multido{\na=0+1}{\numIterations}{%
\pstVerb{/y2 \na\space Delta mul 0.3 add def}%
\psline(! 0.2 y2)(! 1.6 y2)%
\psline(! -0.05 y2)(! 0.05 y2)%
}%
\multido{\na=0+1}{\numIterations}{%
\pstVerb{/y1 \na\space Delta mul 0.3 add def}%
\psline(! y1 0.2)(! y1 1.6)%
\psline(!  y1 -0.05)(! y1 0.05)%
}%
\psline[arrows=|-|] (0.3, -0.3)(!0.3 Delta add -0.3)%
\rput[t](! 0.3 Delta 2 div add -0.45){$\Delta y_1$}%
\psline[arrows=|-|] (-0.3,0.3)(!-0.3 0.3 Delta add)%
\rput[r](! -0.45 0.3 Delta 2 div add){$\Delta y_2$}%
\end{pspicture} ~
 \raisebox{1.25cm}{$\stackrel{\fcv f}{\to} $}  ~
\begin{pspicture}(-0.5, -0.5)(2,2)
\newcommand{\numIterations}{12}%
\newcommand{\numIterationsMinusOne}{11}%
\tiny
\fcAxesStandard{-0.5}{-1}{2.5}{2}
\fcLabels[$x_1$][$x_2$]{2.5}{2}
\pstVerb{
/theF { theta 57.295779513 mul cos r mul theta 57.295779513 mul sin r mul  } def
/Delta 1.2 \numIterationsMinusOne\space div def
/base1 1.2 def
/base2 0.6 def
/basePoint [2 dict begin /r base1 def /theta base2 def theF end] def
/tangent2 [base2 57.295779513 mul sin -1 mul base1 mul base2 57.295779513 mul cos base1 mul] def
/tangent1 [base2 57.295779513 mul cos base2 57.295779513 mul sin] def
}
\pscustom*[linecolor=blue]{%
\parametricplot{base1}{base1 Delta add}{2 dict begin /r t def /theta base2 def theF end}%
\parametricplot{base2}{base2 Delta add}{2 dict begin /r base1 Delta add  def /theta t def theF end}%
\parametricplot{base1 Delta add}{base1}{2 dict begin /r t def /theta base2 Delta add def theF end}%
\parametricplot{base2 Delta add}{base2}{2 dict begin /r base1 def /theta t def theF end}%
}%
\pscustom*[linecolor=blue]{%
\parametricplot{0.3}{0.3 1.2 add}{2 dict begin /r t def /theta 0.3 def theF end}%
\parametricplot{0.3}{0.3 1.2 add}{2 dict begin /r 0.3 1.2 add  def /theta t def theF end}%
\parametricplot{0.3 1.2 add}{0.3}{2 dict begin /r t def /theta 0.3 1.2 add def theF end}%
\parametricplot{0.3 1.2 add}{0.3}{2 dict begin /r 0.3 def /theta t def theF end}%
}%
\multido{\na=0+1}{\numIterations}{%
\pstVerb{/y2 \na\space Delta mul 0.3 add def}%
\parametricplot{0.2 }{1.6}{2 dict begin /r t def /theta y2 def theF end}% 
}%
\multido{\na=0+1}{\numIterations}{%
\pstVerb{/y1 \na\space Delta mul 0.3 add def}%
\parametricplot{0.2 }{1.6}{2 dict begin /r y1 def /theta t def theF end}%
}%
\rput[lb](0.6, 1.6){$ \mathcal S =\fcv f(\mathcal R)$}
\multido{\na=0+1}{\numIterationsMinusOne}{%
\pstVerb{/y2 \na\space Delta mul 0.3 add def}%
\multido{\nb=0+1}{\numIterationsMinusOne}{%
\pstVerb{/y1 \nb\space Delta mul 0.3 add def}%
\pstVerb{%
/r y1 def 
/theta y2 def
/basePoint [theF] def
/tangent1 basePoint 1 r div \fcVectorTimesScalar def
/tangent2 [theF -1 mul exch ] def
}%
\psline*[linecolor=red]
(! basePoint \fcArrayToStack)
(! basePoint tangent1 Delta \fcVectorTimesScalar \fcVectorPlusVector \fcArrayToStack)
(! basePoint tangent1 tangent2 \fcVectorPlusVector Delta \fcVectorTimesScalar \fcVectorPlusVector \fcArrayToStack)
(! basePoint tangent2 Delta \fcVectorTimesScalar \fcVectorPlusVector \fcArrayToStack)
(! basePoint \fcArrayToStack)
}%
}%
\end{pspicture} 
\end{columns}

\begin{theorem}[Variable change in multivariable integrals]
Let $\fcv f$ be a smooth one to one variable change. Let $\fcv f(\mathcal R)= \mathcal S$. \uncover<2->{\alert<2>{Let $h$ be an integrable function.}} Then 

\medskip
\small 
\noindent $\displaystyle
\alert<1>{ \idotsint\limits_{\mathcal S}\uncover<2->{ \alert<2>{ h(x_1,\dots, x_n) } } \diff x_1 \dots \diff x_n   =\idotsint\limits_{\mathcal R} \uncover<2->{\alert<2>{ h(f_1,\dots, f_n)}} \det \left(J_{\fcv f}(\fcv y)\right) \diff y_1 \dots \diff y_n} ,
$

\normalsize
provided that $\det \left( J_{ \fcv f} (\fcv y)\right) \geq 0$ for all $\fcv y\in \mathcal R$.
\end{theorem}
\vskip 10cm 
\end{frame} 
%\begin{frame}
\begin{theorem}[Variable change in multivariable integrals]
$f$ - smooth, \alert<7>{one-to-one}, $\alert<4>{f(\mathcal R)= \mathcal S}$, $\alert<6>{ \det \left( J_{ \fcv f} (\fcv y)\right) \geq 0}$.

\medskip
\small 
\noindent $\displaystyle
\idotsint\limits_{\alert<4>{\mathcal S}} h(x_1,\dots, x_n) \diff x_1 \dots \diff x_n   =\idotsint\limits_{\alert<4>{\mathcal R} }  h(f_1,\dots, f_n) \alert<3>{\det \left(J_{\fcv f}(\fcv y)\right) } \diff y_1 \dots \diff y_n ,
$
\end{theorem}
\begin{itemize}
\only<1-11>{
\item<2-> One-variable subst. rule: $\displaystyle \int_{{\alert<4>{ f( a )}}}^{{\alert<4>{f(b)}}} h(x)\diff x= \int_{{\alert<4>{a}}}^{{\alert<4>{b}}} h(f(y)) \alert<3>{ f'(y)} \diff y $.
\item<5-> The one-variable substitution rule is valid 
\begin{itemize}
\item<6-> \alert<6>{without positivity requirements} (arranged by compensating with minus sign when changing boundaries of integration)
\item<7-> and \alert<7>{without requiring that $f$ be one to one} (compensated by neutralizing contributions arising from sign changes of $f'(y)$).
\end{itemize} 
}
\item<8-> \alert<11,12>{Similarly integration can be generalized so multivar. subst. holds }
\begin{itemize}
\item<9-> \alert<11,12>{without positivity of $\det \left(J_{\fcv f} \right)$ (arranged by compensating with minus sign when changing orientation of spaces),}
\item<10-> \alert<11,12>{without requiring that $\fcv f$ be one to one (compensated by neutralizing contributions arising from sign changes of $\det J_{\fcv f}$).}
\end{itemize}
\only<13->{
\item<13->  When using the above generalization of $\int$, one writes 
$\displaystyle
\idotsint\limits_{\mathcal S} h(x_1,\dots, x_n) \diff x_1 \alert<14>{\wedge} \dots \alert<14>{\wedge} \diff x_n\quad .
$
\item<14-> The \alert<14>{wedge sign $\wedge $ stands for exterior product}.
\item<15-> We will skip the theoretical details of the above generalization.
\item<16-> However we will learn to compute with $\wedge$ in practice.
}
\end{itemize}

\vskip 10cm
\end{frame} 

%\begin{frame}
\frametitle{Integral and Algebraic Volume Definitions Agree}
\begin{columns}
\column{0.25\textwidth}
\psset{xunit=1.4cm, yunit=1.4cm}
\begin{pspicture}
\tiny
\fcBoundingBox{-0.8}{-0.2}{2}{2}%
\renewcommand{\fcScreenStyle}{x}%
\pstVerb{10 dict begin 
/ver1 [1 0 0.05] def 
/ver2 [0 1 0.05] def 
/ver3 [0.3 0.3 1] def  
/normal ver1 ver2 \fcVectorCrossVector def 
/perpFoot ver3 normal normal ver3 \fcVectorScalarVector normal normal \fcVectorScalarVector div \fcVectorTimesScalar \fcVectorMinusVector def
/perpendicular ver3 perpFoot \fcVectorMinusVector def
/innerCorner ver1 ver2 \fcVectorPlusVector 0.1 \fcVectorTimesScalar def
/innerCornerBoxes ver1 ver2 ver3 \fcVectorPlusVector \fcVectorPlusVector 0.2 \fcVectorTimesScalar def
}%
\uncover<3->{%
\fcStartIIIdScene%
\fcBoxIIIdInScene[linewidth=1, colorUV=cyan]{[0 0 0]}{ver1}{ver2}{ver3}%
\multido{\ra=0+0.2}{5}{%
\fcBoxIIIdInScene[linewidth=0.5, colorUV={1 0.5 0.5}, forceForeground=true] {innerCorner ver3 \ra \space \fcVectorTimesScalar \fcVectorPlusVector}{innerCorner ver3 \ra \space \fcVectorTimesScalar \fcVectorPlusVector ver1 0.8 \fcVectorTimesScalar \fcVectorPlusVector }{ innerCorner ver3 \ra \space \fcVectorTimesScalar \fcVectorPlusVector ver2 0.8 \fcVectorTimesScalar \fcVectorPlusVector}{innerCorner ver3 \ra \space \fcVectorTimesScalar \fcVectorPlusVector perpendicular 0.2 \fcVectorTimesScalar \fcVectorPlusVector}%
}%
\fcAxesIIIdInScene[linewidth=1, arrows=->, xLabel={$x_1$}, yLabel={$x_2$}, zLabel={$x_3$}]{2}{2.5}{2}%
\fcFinishIIIdScene[true]%
}%
\uncover<1,2>{%
\fcAxesIIId[linewidth=1pt, arrows=->]{2}{2.5}{2}%
}%
\uncover<1,2>{%
\fcBoxIIIdFilledNew[linewidth=1, dashes={[0.5 3] 0}, colorUV=cyan]{[0 0 0]}{ver1}{ver2}{ver3}%
}%
\uncover<2>{%
\pscustom{%
\code{%
\fcSetUpGraphicsToScreen
30 dict begin
/boxSides
[ [ver1 ver2 \fcVectorCrossVector 0] 
  [ver1 ver2 \fcVectorCrossVector -1 \fcVectorTimesScalar dup ver3 \fcVectorScalarVector]
  [ver2 ver3 \fcVectorCrossVector 0] 
  [ver2 ver3 \fcVectorCrossVector -1 \fcVectorTimesScalar dup ver1 \fcVectorScalarVector]
  [ver3 ver1 \fcVectorCrossVector 0] 
  [ver3 ver1 \fcVectorCrossVector -1 \fcVectorTimesScalar dup ver2 \fcVectorScalarVector]
]
def
/isInBox {/currentPoint exch def true boxSides{
\fcArrayToStack exch currentPoint \fcVectorScalarVector gt {pop false exit}if 
}forall } def
/numIt 8 def %
ver1 ver2 ver3 \fcVectorPlusVector \fcVectorPlusVector \fcArrayToStack
/DeltaZ exch numIt div def %
/DeltaY exch numIt div def %
/DeltaX exch numIt div def %
/DeltaXvector [DeltaX 0 0] def %
/DeltaYvector [0 DeltaY 0] def %
/DeltaZvector [0 0 DeltaZ] def %
/counter1 -1 def
[
numIt {
/counter1 counter1 1 add def
/counter2 -1 def
numIt {
/counter2 counter2 1 add def
/counter3 -1 def
numIt {
/counter3 counter3 1 add def
/startingCorner [DeltaX counter1 mul DeltaY counter2 mul DeltaZ counter3 mul ] def 
/theCorners 
[startingCorner 
 startingCorner DeltaXvector \fcVectorPlusVector
 startingCorner DeltaYvector \fcVectorPlusVector
 startingCorner DeltaZvector \fcVectorPlusVector
 startingCorner DeltaXvector DeltaYvector \fcVectorPlusVector \fcVectorPlusVector
 startingCorner DeltaYvector DeltaZvector \fcVectorPlusVector \fcVectorPlusVector
 startingCorner DeltaZvector DeltaXvector \fcVectorPlusVector \fcVectorPlusVector
 startingCorner DeltaXvector DeltaYvector DeltaZvector \fcVectorPlusVector \fcVectorPlusVector \fcVectorPlusVector 
 ]
def
true theCorners{isInBox not{pop false exit}if }forall
{[
 startingCorner 
 startingCorner DeltaXvector \fcVectorPlusVector
 startingCorner DeltaYvector \fcVectorPlusVector
 startingCorner DeltaZvector \fcVectorPlusVector
 [\fcGetColorCode{black}] [1 0.5 0.5] true false [[0.5 3] 0]
 ]
}if
}repeat
}repeat
}repeat
]
/LeftGreaterThanRight {0 get \fcScreen\space pop \fcVectorScalarVector exch 0 get \fcScreen\space pop \fcVectorScalarVector gt} def
\fcMergeSort
dup
{\fcArrayToStack \fcBoxIIIdFilledCode }forall
{\fcArrayToStack 3 1 roll pop pop false true 3 -1 roll \fcBoxIIIdFilledCode}forall
end
}%
}%
}%
\uncover<1,2>{%
\fcBoxIIIdHollowNew[dashes={[0.5 3] 0}, colorUV=cyan]{[0 0 0]}{ver1}{ver2}{ver3}%
\fcAxesIIId[linewidth=1pt, xLabel={$x_1$}, yLabel={$x_2$}, zLabel={$x_3$}, linestyle=dotted]{2}{2.5}{2}%
}%
\pstVerb{end}%
\end{pspicture}

\column{0.75\textwidth}
\begin{itemize}
\item Let $ \fcv v_1=(v_{11}, \dots, v_{1n})$, $\dots$, $\fcv v_n=(v_{n1},\dots, v_{nn} )$ be $n$-vectors in $n$-dimensional space.
\item Let $\mathcal R_k$ be the parallelotope spanned by $\fcv v_1, \dots, \fcv v_k$.
\item Let $h_k$ be the height of $\mathcal R_k$ with base $\mathcal R_{k-1}$.
\end{itemize}
\end{columns}
\begin{theorem}
$\Vol_n(\mathcal R_n)=h_n \Vol_{n-1}(\mathcal R_{n-1})= \displaystyle {\int\dots \int}_{ \mathcal R_n} 1 \cdot \alert<2>{ \diff x_1\dots \diff x_n} $.
\end{theorem}
\begin{itemize}
\item<2-> \alert<2>{Right hand side: approx. vol. with boxes, sides along coord. axes.}
\item<3-> \alert<3>{ Left hand side: approximate volume with slabs parallel to base.}
\item<4-> Theorem is fully intuitive but its proof is surprisingly laborious.
\end{itemize}

\end{frame}
%\begin{frame}
%\frametitle{Parallelotope definition}
\begin{itemize}
\item Let $\fcv o$ be a marked point. If omitted, we assume $\fcv o$ is the origin. 
\item<2-> Let $\fcv v_1= (v_{11},\dots, v_{1n}),\dots, \fcv v_{k}= (v_{k1}, \dots, v_{kn})$ be $k$ vectors in $n$-dimensional space, $k\leq n$.
\item<3-> Let $\mathcal R$ be region \alert<3-10>{spanned by} the vectors at $\fcv o$, coefficients  in $[0,1]$.
\item<4-> That is, $\mathcal R=\left\{ \alert<8-10>{ \alert<5-7>{\fcv o+ t_1 \fcv v_1} +t_2 \fcv v_2} +\dots+ t_k\fcv v_k | \alert<5-7>{t_1\in [0,1]}, \dots, t_k\in [0,1]\right\}$.
\uncover<11->{
\begin{definition}[parallelotope at $\fcv o$]
We call a region $\mathcal R$ of the above form a $k$-dimensional parallelotope at the point $\fcv o$ in $n$-dimensional space.
\end{definition}
}
\end{itemize}
\begin{columns}
\column{0.4\textwidth}
\begin{pspicture}
\tiny
\fcBoundingBox{-0.5}{-0.5}{2}{3.2}
\renewcommand{\fcScreenStyle}{x}%
\fcAxesIIId{2}{4}{2}%
\only<1>{\fcDotIIId{[0 0 0]}}%
\pstVerb{5 dict begin /ver1 [1 0 0.2] def /ver2 [0 1 0.2] def /ver3 [1 1 2] def}%
\only<2>{%
\fcPutIIId[bl]{ver1}{$~\fcv v_1$}%
\fcPutIIId[br]{ver2}{$\fcv v_2~$}%
}%
\uncover<3-12,17,18>{%
\fcParallelogramIIId[linecolor=cyan]{[0 0 0]}{ver1}{ver2}%
\fcParallelogramHollowIIId{[0 0 0]}{ver1}{ver2}%
}%
\uncover<2-12>{%
\fcLineIIId[arrows=->]{[0 0 0]}{ver1}%
\fcLineIIId[arrows=->]{[0 0 0]}{ver2}%
}%
\only<5>{\fcLineIIId[arrows=->, linecolor=red, linewidth=1.5pt]{[0 0 0]}{ver1 0.4 \fcVectorTimesScalar}
\fcDotIIId{ver1 0.4 \fcVectorTimesScalar}%
}%
\only<6>{\fcLineIIId[arrows=->, linecolor=red, linewidth=1.5pt]{[0 0 0]}{ver1 0.5 \fcVectorTimesScalar}
\fcDotIIId{ver1 0.5 \fcVectorTimesScalar}%
}%
\only<7-10>{\fcLineIIId[arrows=->, linecolor=red, linewidth=1.5pt]{[0 0 0]}{ver1 0.6 \fcVectorTimesScalar}
\fcDotIIId{ver1 0.6 \fcVectorTimesScalar}%
}%
\only<8>{ %
\fcLineIIId[arrows=->, linecolor=red, linewidth=1.5pt]{ver1 0.6 \fcVectorTimesScalar}{ver1 0.6 \fcVectorTimesScalar ver2 0.4 \fcVectorTimesScalar \fcVectorPlusVector}%
\fcDotIIId{ver1 0.6 \fcVectorTimesScalar ver2 0.4 \fcVectorTimesScalar \fcVectorPlusVector}%
}%
\only<9>{ %
\fcLineIIId[arrows=->, linecolor=red, linewidth=1.5pt]{ver1 0.6 \fcVectorTimesScalar}{ver1 0.6 \fcVectorTimesScalar ver2 0.5 \fcVectorTimesScalar \fcVectorPlusVector}%
\fcDotIIId{ver1 0.6 \fcVectorTimesScalar ver2 0.5 \fcVectorTimesScalar \fcVectorPlusVector}%
}%
\only<10>{ %
\fcLineIIId[arrows=->, linecolor=red, linewidth=1.5pt]{ver1 0.6 \fcVectorTimesScalar}{ver1 0.6 \fcVectorTimesScalar ver2 0.6 \fcVectorTimesScalar \fcVectorPlusVector}%
\fcDotIIId{ver1 0.6 \fcVectorTimesScalar ver2 0.6 \fcVectorTimesScalar \fcVectorPlusVector}%
}%
\uncover<11,12,19,20>{%
\fcBoxIIIdFilledNew[colorUV=cyan, dashes={[0.5 3] 0}]{[0 0 0]}{ver1}{ver2}{ver3}%
}%
\uncover<13,14>{%
\fcLineIIId[linewidth=1.5pt]{[0 0 0]}{ver1}%
}%
\uncover<15,16>{%
\fcParallelogramIIId[linecolor=cyan]{[0 0 0]}{[1 0 0]}{[1 1 0]}%
\fcParallelogramHollowIIId{[0 0 0]}{[1 0 0]}{[1 1 0]}%
}%
\fcAxesIIId[linestyle=dotted]{2}{4}{2}
\pstVerb{end}
\end{pspicture}
\column{0.6\textwidth}
\begin{itemize}
\item<12-> When $k,n, \fcv o$ are clear from context we can omit them.
\uncover<13->{
\begin{tabular}{c|c|c}
$k$& $n$ & parallelotope name\\\hline 
\fcQuestion{13}{$1$}& \fcQuestion{13}{any} & \fcAnswer{14}{segment (in $n$-dim space)}\\
\fcQuestion{14}{$2$}& \fcQuestion{15}{$2$} & \fcAnswer{16}{parallelogram} \\
\fcQuestion{17}{$2$}& \fcQuestion{17}{$3$} & \fcAnswer{18}{parallelogram in space} \\
\fcQuestion{19}{$3$}& \fcQuestion{19}{$3$} & \fcAnswer{20}{parallelepiped}
\end{tabular}
}
\end{itemize}
\end{columns}
\end{frame}

%\begin{frame}
\frametitle{Example: Moment of Inertia}
\begin{itemize}
\item Problem: compute the moment of inertia $I$
\begin{itemize}
\item \alert<2>{of a rectangular box with sides $2a$, $2b$, and $2c$}
\item \alert<3>{rotating about axis $L$ through center that is perpendicular to a face.}
\item \alert<4>{The box has constant density $\rho$.} \uncover<11->{Therefore it's mass is $m=8\rho abc $.}
\end{itemize}
\item<5-> Coord. system: rotation axis = $z$-axis, $x,y$ axes along box sides.
\uncover<6->{
\[
I = \iiint_{\cR} \rho\, \text{dist}^2(P,L) \diff V = \iiint_{\cR} \rho (x^2+y^2)\, \diff x\diff y \diff z\; .
\]
}
\item<7-> Decompose into slices as follows.
\begin{itemize}
\item<7-> Project $\cR$ onto the $z-$axis \alert<8>{to get segment from $z=-c$ to $z=c$.}
\[
\iiint_{\cR} \rho (x^2+y^2) \diff x \diff y \diff z = \alert<8>{\int_{z=-c}^{z=c}} \left( \iint_{S_z} \rho (x^2+y^2)\, \diff x \diff y \right) \diff z
\]    
\item<9-> For a fixed $z$, the slice $S_z$ is: $-a \leq x \leq a$, $-b \leq  y \leq  b$.
\[
I_L = \alert<10,11>{ \int_{z=-c}^{z=c} \left(\int_{x=-a}^{x=a} \left( \int_{y=-b}^{y=b} \rho (x^2+y^2)  \diff y \right)  \diff x\right)  \diff z} \uncover<10->{\alert<10,11>{=}} \fcAnswer{11}{\frac{m(a^2+b^2)}{3}} \uncover<11->{.}
\]
%

\end{itemize}
  \end{itemize}
\end{frame}

%\begin{frame}
%\frametitle{Parallelotope definition}
\begin{itemize}
\item Let $\fcv o$ be a marked point. If omitted, we assume $\fcv o$ is the origin. 
\item<2-> Let $\fcv v_1= (v_{11},\dots, v_{1n}),\dots, \fcv v_{k}= (v_{k1}, \dots, v_{kn})$ be $k$ vectors in $n$-dimensional space, $k\leq n$.
\item<3-> Let $\mathcal R$ be region \alert<3-10>{spanned by} the vectors at $\fcv o$, coefficients  in $[0,1]$.
\item<4-> That is, $\mathcal R=\left\{ \alert<8-10>{ \alert<5-7>{\fcv o+ t_1 \fcv v_1} +t_2 \fcv v_2} +\dots+ t_k\fcv v_k | \alert<5-7>{t_1\in [0,1]}, \dots, t_k\in [0,1]\right\}$.
\uncover<11->{
\begin{definition}[parallelotope at $\fcv o$]
We call a region $\mathcal R$ of the above form a $k$-dimensional parallelotope at the point $\fcv o$ in $n$-dimensional space.
\end{definition}
}
\end{itemize}
\begin{columns}
\column{0.4\textwidth}
\begin{pspicture}
\tiny
\fcBoundingBox{-0.5}{-0.5}{2}{3.2}
\renewcommand{\fcScreenStyle}{x}%
\fcAxesIIId{2}{4}{2}%
\only<1>{\fcDotIIId{[0 0 0]}}%
\pstVerb{5 dict begin /ver1 [1 0 0.2] def /ver2 [0 1 0.2] def /ver3 [1 1 2] def}%
\only<2>{%
\fcPutIIId[bl]{ver1}{$~\fcv v_1$}%
\fcPutIIId[br]{ver2}{$\fcv v_2~$}%
}%
\uncover<3-12,17,18>{%
\fcParallelogramIIId[linecolor=cyan]{[0 0 0]}{ver1}{ver2}%
\fcParallelogramHollowIIId{[0 0 0]}{ver1}{ver2}%
}%
\uncover<2-12>{%
\fcLineIIId[arrows=->]{[0 0 0]}{ver1}%
\fcLineIIId[arrows=->]{[0 0 0]}{ver2}%
}%
\only<5>{\fcLineIIId[arrows=->, linecolor=red, linewidth=1.5pt]{[0 0 0]}{ver1 0.4 \fcVectorTimesScalar}
\fcDotIIId{ver1 0.4 \fcVectorTimesScalar}%
}%
\only<6>{\fcLineIIId[arrows=->, linecolor=red, linewidth=1.5pt]{[0 0 0]}{ver1 0.5 \fcVectorTimesScalar}
\fcDotIIId{ver1 0.5 \fcVectorTimesScalar}%
}%
\only<7-10>{\fcLineIIId[arrows=->, linecolor=red, linewidth=1.5pt]{[0 0 0]}{ver1 0.6 \fcVectorTimesScalar}
\fcDotIIId{ver1 0.6 \fcVectorTimesScalar}%
}%
\only<8>{ %
\fcLineIIId[arrows=->, linecolor=red, linewidth=1.5pt]{ver1 0.6 \fcVectorTimesScalar}{ver1 0.6 \fcVectorTimesScalar ver2 0.4 \fcVectorTimesScalar \fcVectorPlusVector}%
\fcDotIIId{ver1 0.6 \fcVectorTimesScalar ver2 0.4 \fcVectorTimesScalar \fcVectorPlusVector}%
}%
\only<9>{ %
\fcLineIIId[arrows=->, linecolor=red, linewidth=1.5pt]{ver1 0.6 \fcVectorTimesScalar}{ver1 0.6 \fcVectorTimesScalar ver2 0.5 \fcVectorTimesScalar \fcVectorPlusVector}%
\fcDotIIId{ver1 0.6 \fcVectorTimesScalar ver2 0.5 \fcVectorTimesScalar \fcVectorPlusVector}%
}%
\only<10>{ %
\fcLineIIId[arrows=->, linecolor=red, linewidth=1.5pt]{ver1 0.6 \fcVectorTimesScalar}{ver1 0.6 \fcVectorTimesScalar ver2 0.6 \fcVectorTimesScalar \fcVectorPlusVector}%
\fcDotIIId{ver1 0.6 \fcVectorTimesScalar ver2 0.6 \fcVectorTimesScalar \fcVectorPlusVector}%
}%
\uncover<11,12,19,20>{%
\fcBoxIIIdFilledNew[colorUV=cyan, dashes={[0.5 3] 0}]{[0 0 0]}{ver1}{ver2}{ver3}%
}%
\uncover<13,14>{%
\fcLineIIId[linewidth=1.5pt]{[0 0 0]}{ver1}%
}%
\uncover<15,16>{%
\fcParallelogramIIId[linecolor=cyan]{[0 0 0]}{[1 0 0]}{[1 1 0]}%
\fcParallelogramHollowIIId{[0 0 0]}{[1 0 0]}{[1 1 0]}%
}%
\fcAxesIIId[linestyle=dotted]{2}{4}{2}
\pstVerb{end}
\end{pspicture}
\column{0.6\textwidth}
\begin{itemize}
\item<12-> When $k,n, \fcv o$ are clear from context we can omit them.
\uncover<13->{
\begin{tabular}{c|c|c}
$k$& $n$ & parallelotope name\\\hline 
\fcQuestion{13}{$1$}& \fcQuestion{13}{any} & \fcAnswer{14}{segment (in $n$-dim space)}\\
\fcQuestion{14}{$2$}& \fcQuestion{15}{$2$} & \fcAnswer{16}{parallelogram} \\
\fcQuestion{17}{$2$}& \fcQuestion{17}{$3$} & \fcAnswer{18}{parallelogram in space} \\
\fcQuestion{19}{$3$}& \fcQuestion{19}{$3$} & \fcAnswer{20}{parallelepiped}
\end{tabular}
}
\end{itemize}
\end{columns}
\end{frame}

%\input{../../modules/TESTING/test-paraboloid-part}



%\begin{frame}
\begin{example}[Decomposition into rods]
\begin{columns}
\column{0.35\textwidth}
\begin{pspicture}(-0.9,-0.3)(3.3,2.4)
\tiny%
\renewcommand{\fcScreen}{[-1 3 -1] 0}%
\renewcommand{\fcDashes}{[0.5 2] 0}%
\fcBoundingBox{-0.9}{-0.3}{3.3}{2.4}%
\fcStartIIIdScene%
\fcTriangleInScene{[0 0 0]}{[1 0.5 0]}{[0 0 2]}%
\fcTriangleInScene{[0 0 0]}{[0 1 0]}{[0 0 2]}%
\fcTriangleInScene{[0 0 0]}{[1 0.5 0]}{[0 1 0]}%
\fcLineIIIdInScene{[0 1 0]}{[1 0.5 0]}%
\fcAxesIIIdInScene[linecolor=black, linestyle=normal, arrows=->]{2}{3}{2.3}%
\only<3->{%
\fcTriangleInScene[colorUV=blue, linestyle=none, forceForeground=true]{[0 0 0]}{[1 0.5 0]}{[0 1 0]}%
}%
\fcFinishIIIdScene%
\fcDotIIId[linecolor=red]{[0 0 0]}
\fcDotIIId[linecolor=red]{[0 1 0]}
\fcDotIIId[linecolor=red]{[1 0.5 0]}
\fcDotIIId[linecolor=red]{[0 0 2]}%
\uncover<5->{%
\fcDotIIId[linecolor=green]{[0.5 0.5 0]}%
\fcDotIIId[linecolor=green]{[0.5 0.5 0.5]}%
\fcLineIIId[linecolor=green, linewidth=1.5pt]{[0.5 0.5 0]}{[0.5 0.5 0.5]}
}%
\uncover<9>{%
\fcDotIIId[linecolor=green]{[1 0 0] }%
\fcDotIIId[linecolor=green]{[0 0 0] }%
\fcLineIIId[linecolor=green,linewidth=1.5pt]{[0 0 0]}{[1 0 0]}%
}%
\uncover<10->{%
\fcDotIIId[linecolor=red]{[1 0 0]}%
}%
\uncover<10->{%
\fcDotIIId[linecolor=green]{[0.5 0 0]}%
\fcLineIIId[linecolor=green,linewidth=1.5pt]{[0.5 0.25 0]}{[0.5 0.75 0]}%
}%
%\fcPutIIId[l]{[2.05 0 0]}{$x$}
%\fcPutIIId[l]{[0 3 0]}{$y$}
%\fcPutIIId[b]{[0 0 2.4]}{$z$}
\end{pspicture}
\column{0.65\textwidth}
\vbox to 2.7cm{


Compute the volume of the region $\cR$ bounded by $x+2y+z=2$, $x=2y$, $x=0$, $z=0$.

\hfil $\displaystyle\text{vol}(\cR) = \iiint_{\cR} 1\cdot \diff V$

$\cR$ is a tetrahedron with vertices at $(0,0,0)$, $(0,1,0)$, $(0,0,2)$, and $\left(1, \frac{1}{2}, 0\right)$.

}
\end{columns}
$
\begin{array}{r@{}c@{}l}
\uncover<2->{\text{vol}(\cR)} &\uncover<2->{=}&\displaystyle \uncover<2->{ \iiint_{ \cR } 1\cdot \diff V } \uncover<5->{ = \iint_D \left( \int_{{\fcAnswer{6}{z= 0}} }^{ {\fcAnswer{6}{ z=2 -x -2y}}} 1\cdot \diff z \right)  \diff x \diff y} \\
\uncover<7->{ &=&\displaystyle \iint_D (2-x-2y)  \diff x \diff y} \uncover<10->{=  \int_{\alertNoH{10}{x=0}}^{\alertNoH{9}{x=1}} \left( \int_{{ \uncover<12->{\alertNoH{12}{y= \frac{ x}{2}}} }}^{{\uncover<12->{\alertNoH{12}{ y=1-\frac{x}{2}}}}} (2-x-2y) \diff y \right) \diff x }\\
\uncover<13->{&=&\displaystyle \alertNoH{15,16}{ \int_{0}^{1} \left( \left[ \fcAnswer{14}{ ( 2-x)y-y^2}\right]_{y=\frac{x}{2}}^{ y =1 -\frac{x}{2}} \right) \diff x}} \uncover<15->{\alertNoH{15,16}{=}} \only<15>{\alertNoH{15}{\textbf{?}}} \only<16->{\alertNoH{16}{\int_0^1 (x^{2}-2 x+1)\diff x= \frac{1}{3}.} }
\end{array}
$
\only<1-7>{
\uncover<3->{Project the region onto the $xy-$plane to get triangle $D$ with vertices \fcAnswer{4}{$(0,0,0)$}, \fcAnswer{4}{$(0,1,0)$} and \fcAnswer{4}{$ \left( 1 ,\frac{1}{2},0\right)$}.} \uncover<5->{Fix $(x,y)\in D$; the vertical rod is segment with endpoints $z=\fcAnswer{6}{0}$ and $z=\fcAnswer{6}{2-x-2y} $.}
}
\only<9->{
Project $D$ on the $x-$axis to get segment from $x=0$ to $x=1$.  \uncover<11->{Fix $x$ in that range; the slice is the segment from $y=\fcAnswer{12}{ \frac{x}{2}}$ to $y=\fcAnswer{12}{1- \frac{ x}{2}}$.}
}
\end{example}

\vskip 5cm
\end{frame}

%\begin{frame}
%\frametitle{Parallelotope definition}
\begin{itemize}
\item Let $\fcv o$ be a marked point. If omitted, we assume $\fcv o$ is the origin. 
\item<2-> Let $\fcv v_1= (v_{11},\dots, v_{1n}),\dots, \fcv v_{k}= (v_{k1}, \dots, v_{kn})$ be $k$ vectors in $n$-dimensional space, $k\leq n$.
\item<3-> Let $\mathcal R$ be region \alert<3-10>{spanned by} the vectors at $\fcv o$, coefficients  in $[0,1]$.
\item<4-> That is, $\mathcal R=\left\{ \alert<8-10>{ \alert<5-7>{\fcv o+ t_1 \fcv v_1} +t_2 \fcv v_2} +\dots+ t_k\fcv v_k | \alert<5-7>{t_1\in [0,1]}, \dots, t_k\in [0,1]\right\}$.
\uncover<11->{
\begin{definition}[parallelotope at $\fcv o$]
We call a region $\mathcal R$ of the above form a $k$-dimensional parallelotope at the point $\fcv o$ in $n$-dimensional space.
\end{definition}
}
\end{itemize}
\begin{columns}
\column{0.4\textwidth}
\begin{pspicture}
\tiny
\fcBoundingBox{-0.5}{-0.5}{2}{3.2}
\renewcommand{\fcScreenStyle}{x}%
\fcAxesIIId{2}{4}{2}%
\only<1>{\fcDotIIId{[0 0 0]}}%
\pstVerb{5 dict begin /ver1 [1 0 0.2] def /ver2 [0 1 0.2] def /ver3 [1 1 2] def}%
\only<2>{%
\fcPutIIId[bl]{ver1}{$~\fcv v_1$}%
\fcPutIIId[br]{ver2}{$\fcv v_2~$}%
}%
\uncover<3-12,17,18>{%
\fcParallelogramIIId[linecolor=cyan]{[0 0 0]}{ver1}{ver2}%
\fcParallelogramHollowIIId{[0 0 0]}{ver1}{ver2}%
}%
\uncover<2-12>{%
\fcLineIIId[arrows=->]{[0 0 0]}{ver1}%
\fcLineIIId[arrows=->]{[0 0 0]}{ver2}%
}%
\only<5>{\fcLineIIId[arrows=->, linecolor=red, linewidth=1.5pt]{[0 0 0]}{ver1 0.4 \fcVectorTimesScalar}
\fcDotIIId{ver1 0.4 \fcVectorTimesScalar}%
}%
\only<6>{\fcLineIIId[arrows=->, linecolor=red, linewidth=1.5pt]{[0 0 0]}{ver1 0.5 \fcVectorTimesScalar}
\fcDotIIId{ver1 0.5 \fcVectorTimesScalar}%
}%
\only<7-10>{\fcLineIIId[arrows=->, linecolor=red, linewidth=1.5pt]{[0 0 0]}{ver1 0.6 \fcVectorTimesScalar}
\fcDotIIId{ver1 0.6 \fcVectorTimesScalar}%
}%
\only<8>{ %
\fcLineIIId[arrows=->, linecolor=red, linewidth=1.5pt]{ver1 0.6 \fcVectorTimesScalar}{ver1 0.6 \fcVectorTimesScalar ver2 0.4 \fcVectorTimesScalar \fcVectorPlusVector}%
\fcDotIIId{ver1 0.6 \fcVectorTimesScalar ver2 0.4 \fcVectorTimesScalar \fcVectorPlusVector}%
}%
\only<9>{ %
\fcLineIIId[arrows=->, linecolor=red, linewidth=1.5pt]{ver1 0.6 \fcVectorTimesScalar}{ver1 0.6 \fcVectorTimesScalar ver2 0.5 \fcVectorTimesScalar \fcVectorPlusVector}%
\fcDotIIId{ver1 0.6 \fcVectorTimesScalar ver2 0.5 \fcVectorTimesScalar \fcVectorPlusVector}%
}%
\only<10>{ %
\fcLineIIId[arrows=->, linecolor=red, linewidth=1.5pt]{ver1 0.6 \fcVectorTimesScalar}{ver1 0.6 \fcVectorTimesScalar ver2 0.6 \fcVectorTimesScalar \fcVectorPlusVector}%
\fcDotIIId{ver1 0.6 \fcVectorTimesScalar ver2 0.6 \fcVectorTimesScalar \fcVectorPlusVector}%
}%
\uncover<11,12,19,20>{%
\fcBoxIIIdFilledNew[colorUV=cyan, dashes={[0.5 3] 0}]{[0 0 0]}{ver1}{ver2}{ver3}%
}%
\uncover<13,14>{%
\fcLineIIId[linewidth=1.5pt]{[0 0 0]}{ver1}%
}%
\uncover<15,16>{%
\fcParallelogramIIId[linecolor=cyan]{[0 0 0]}{[1 0 0]}{[1 1 0]}%
\fcParallelogramHollowIIId{[0 0 0]}{[1 0 0]}{[1 1 0]}%
}%
\fcAxesIIId[linestyle=dotted]{2}{4}{2}
\pstVerb{end}
\end{pspicture}
\column{0.6\textwidth}
\begin{itemize}
\item<12-> When $k,n, \fcv o$ are clear from context we can omit them.
\uncover<13->{
\begin{tabular}{c|c|c}
$k$& $n$ & parallelotope name\\\hline 
\fcQuestion{13}{$1$}& \fcQuestion{13}{any} & \fcAnswer{14}{segment (in $n$-dim space)}\\
\fcQuestion{14}{$2$}& \fcQuestion{15}{$2$} & \fcAnswer{16}{parallelogram} \\
\fcQuestion{17}{$2$}& \fcQuestion{17}{$3$} & \fcAnswer{18}{parallelogram in space} \\
\fcQuestion{19}{$3$}& \fcQuestion{19}{$3$} & \fcAnswer{20}{parallelepiped}
\end{tabular}
}
\end{itemize}
\end{columns}
\end{frame}

%\begin{frame}
\begin{columns}
\column{0.4\textwidth}
$
\fcv f:\left|\begin{array}{r@{~}c@{~}l}
\alert<2>{x_1} &=& \alert<2>{f_1}( y_1, \dots, y_n)\\
&\vdots&\\
\alert<2>{x_n} &=& \alert<2>{f_n} ( y_1, \dots, y_n) \quad .
\end{array}\right.
$
\column{0.6\textwidth}
\begin{pspicture}(-0.5, -0.5)(2,2)
\tiny
\fcAxesStandard{-0.5}{-0.5}{2}{2}
\fcLabels[$y_1$][$y_2$]{2}{2}
\multido{\na=6+1}{5}{%
\pstVerb{/y2 \na\space 6 sub 0.3 mul 0.3 add def}%
\psline(! 0.2 y2)(! 1.6 y2)%
\psline(! -0.05 y2)(! 0.05 y2)%
}%
\multido{\na=11+1}{5}{%
\pstVerb{/y1 \na\space 11 sub 0.3 mul 0.3 add def}%
\psline(! y1 0.2)(! y1 1.6)%
\psline(!  y1 -0.05)(! y1 0.05)%
}%
\uncover<3->{\psline[linecolor=red, linewidth=2pt](0.2, 0.6)(1.6, 0.6)}
\uncover<6->{\psline[linecolor=red, linewidth=2pt](1.2, 0.2)(1.2, 1.6)}
\end{pspicture} ~
 \raisebox{1.25cm}{$\stackrel{\fcv f}{\to} $}  ~
\begin{pspicture}(-0.5, -0.5)(2,2)
\tiny
\fcAxesStandard{-0.5}{-0.5}{2.5}{2}
\fcLabels[$x_1$][$x_2$]{2.5}{2}
\pstVerb{
/theF { theta 57.295779513 mul cos r mul theta 57.295779513 mul sin r mul  } def
}
\multido{\na=6+1}{5}{%
\pstVerb{/y2 \na\space 6 sub 0.3 mul 0.3 add def}%
\parametricplot{0.2 }{1.6}{2 dict begin /r t def /theta y2 def theF end} }%
\multido{\na=11+1}{5}{%
\pstVerb{/y1 \na\space 11 sub 0.3 mul 0.3 add def}%
\parametricplot{0.2 }{1.6}{2 dict begin /r y1 def /theta t def theF end}%
}%
\uncover<4->{\parametricplot[linecolor=red, linewidth=2pt]{0.2 }{1.6}{2 dict begin /r t def /theta 0.6 def theF end}}
\uncover<6->{\parametricplot[linecolor=red, linewidth=2pt]{0.2 }{1.6}{2 dict begin /r 1.2 def /theta t def theF end}}
\uncover<6->{%
\fcFullDot{0.6 57.295779513 mul cos 1.2 mul }{0.6 57.295779513 mul sin 1.2 mul}
\psline[linecolor=blue, linewidth=2pt, arrows=->](!
0.6 57.295779513 mul cos 1.2 mul 
0.6 57.295779513 mul sin 1.2 mul
)(!
0.6 57.295779513 mul cos 1.2 mul 
0.6 57.295779513 mul sin 1.2 mul sub
0.6 57.295779513 mul sin 1.2 mul
0.6 57.295779513 mul cos 1.2 mul add
)}%
\uncover<6->{%
\rput[l](0.5, 1.5){$\alert<6>{\left( \frac{\partial x_1 }{\partial y_2},\dots , \frac{\partial x_n}{\partial y_2} \right)}$}
}%
\uncover<5->{%
\psline[linecolor=blue, linewidth=2pt, arrows=->](!
0.6 57.295779513 mul cos 1.2 mul 
0.6 57.295779513 mul sin 1.2 mul
)(!
0.6 57.295779513 mul cos 2.2 mul 
0.6 57.295779513 mul sin 2.2 mul
)}%
\uncover<5->{%
\rput[b](1.4, -0.5){$\alert<5>{\left( \frac{\partial x_1 }{\partial y_1},\dots , \frac{\partial x_n}{\partial y_1} \right)}$}
\psline[linestyle=dotted, arrows=->](1.4, -0.25)(! 0.6 57.295779513 mul cos 1.7 mul 0.6 57.295779513 mul sin 1.7 mul)
}%
\end{pspicture} 
\end{columns}
\begin{definition}[Jacobian matrix]
The Jacobian matrix of a variable change $\fcv f$ is defined as the matrix 
\[
J_{\fcv f}=\left( \begin{array}{ccc} \frac{\partial \alert<2>{f_1}}{\partial y_1} & \cdots & \frac{\partial \alert<2>{f_1}}{\partial y_n} \\ \vdots & \ddots & \vdots \\ \frac{\partial \alert<2>{f_n}}{\partial y_1} & \cdots & \frac{\partial \alert<2>{f_n}}{\partial y_n} \end{array}\right) \uncover<2->{= \left( \begin{array}{ccc} \alert<5>{ \frac{\partial \alert<2>{x_1} }{\partial y_1}} & \cdots & \frac{\partial \alert<2>{x_1} }{\partial y_n} \\ \vdots & \ddots & \vdots \\ \alert<5>{ \frac{\partial \alert<2>{x_n}}{\partial y_1} } & \cdots & \frac{\partial \alert<2>{x_n}}{\partial y_n} \end{array}\right) }
\]
\end{definition}
\begin{itemize}
\item<3-> Consider \alert<4>{curve given by $\fcv f$} with \alert<3>{parameter $y_1$} (other $y_j$'s-fixed).
\item<5-> Then the tangent vector of that curve is $\alert<5>{\left( \frac{\partial x_1 }{\partial y_1},\dots , \frac{\partial x_n}{\partial y_1} \right)}$.
\item<6-> Similar considerations hold for $\alert<6>{y_2}\uncover<7->{\alert<7>{,\dots, y_n}.}$
\end{itemize}


\end{frame}

}
\end{document}
