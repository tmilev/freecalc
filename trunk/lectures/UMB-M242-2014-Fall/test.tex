\documentclass%
%[handout]
{beamer}
% % % % % % % %
% % % % % % % %
% % % % % % % %
%IMPORTANT
%compiles with
%pdflatex -shell-escape
%IMPORTANT
% % % % % % % %
% % % % % % % %
% % % % % % % %
\mode<presentation>
{
\useinnertheme{rounded}
\useoutertheme{infolines}
\usecolortheme{orchid}
\usecolortheme{whale}
}

\usepackage[english]{babel}
\usepackage[latin1]{inputenc}
\usepackage{times}
\usepackage[T1]{fontenc}
\usepackage{../example-templates}
\usepackage{../pstricks-commands}

\usepackage{auto-pst-pdf}
\usepackage{pst-plot}
%\usepackage{pstricks-add}

% Or whatever. Note that the encoding and the font should match. If T1
% does not look nice, try deleting the line with the fontenc.


\graphicspath{{../../modules/}}

\newtheoremstyle{partialproof}{3pt}{3pt}{}{}{}{.}{.5em}{}
\theoremstyle{partialproof} \newtheorem{partialproof}[theorem]{Proof.}
%\DeclareMathOperator{\diff}{d}
\setbeamertemplate{navigation symbols}{}

\includeonlylecture{1}

\newcommand{\lect}[3]{
  \date{#1}
  \lecture[#1]{#2}{#3}
}

\setbeamertemplate{footline}
{
  \leavevmode%
  \hbox{%
  \begin{beamercolorbox}[wd=.333333\paperwidth,ht=2.25ex,dp=1ex,center]{author in head/foot}%
    \usebeamerfont{author in head/foot}\insertshortauthor
  \end{beamercolorbox}%
  \begin{beamercolorbox}[wd=.333333\paperwidth,ht=2.25ex,dp=1ex,center]{title in head/foot}%
    \usebeamerfont{title in head/foot}\insertshorttitle
  \end{beamercolorbox}%
  \begin{beamercolorbox}[wd=.333333\paperwidth,ht=2.25ex,dp=1ex,center]{date in head/foot}%
    \usebeamerfont{date in head/foot}\insertshortdate{}
  \end{beamercolorbox}}%
  \vskip0pt%
}

% If you have a file called "university-logo-filename.xxx", where xxx
% is a graphic format that can be processed by latex or pdflatex,
% resp., then you can add a logo as follows:

%\pgfdeclareimage[height=0.8cm]{logo}{bluelogo}
%\logo{\pgfuseimage{logo}}
\renewcommand{\Arcsin}{\arcsin}
\renewcommand{\Arccos}{\arccos}
\renewcommand{\Arctan}{\arctan}
\renewcommand{\Arccot}{\text{arccot\hspace{0.03cm}}}
\renewcommand{\Arcsec}{\text{arcsec\hspace{0.03cm}}}
\renewcommand{\Arccsc}{\text{arccsc\hspace{0.03cm}}}



\begin{document}

\AtBeginLecture{%

\title[\insertlecture]{FreeCalc}
\subtitle{\insertlecture}
\author[FreeCalc]{}
\institute[UMass Boston]{University of Massachusetts Boston}
\date{\insertshortlecture}
\begin{frame}
  \titlepage
\end{frame}
}%

% begin lecture
\lect{\today}{Sample}{1}{
\begin{frame}\frametitle{Riemann sum over rectangular regions}
\begin{pspicture}(-1,-1)(1,1)
\renewcommand{\fcScreen}{[-1 -1 -1] 0}
\fcAxesIIId{2}{2}{2}
\fcBoxIIIdFilledNew{[0 0 0]}{[1 0 0]}{[0 1 0]}{[0 0 1]}
\end{pspicture}
\end{frame}
 
%

\begin{frame}
\frametitle{Midpoint Rule}
\small
\begin{columns}
\column{0.3\textwidth}
\begin{pspicture}(-1,-1)(1,1)
\tiny
\pstVerb{%
/theRFf {x 2 sub dup mul y 2 sub dup mul add 2 div} def%
}%
\fcBoundingBox{-2}{-1.8}{2}{3.3}
\renewcommand{\fcScreen}{[-1 -1.1 -1] 0}
\fcAxesIIId{3}{3}{4.3}
\uncover<3-11>{%
\fcLineIIId[linewidth =0.5pt]{[-0.1 0 0]}{[2.1 0 0]}%
\fcLineIIId[linewidth =0.5pt]{[-0.1 1 0]}{[2.1 1 0]}%
\fcLineIIId[linewidth =0.5pt]{[-0.1 2 0]}{[2.1 2 0]}%
\fcLineIIId[linewidth =0.5pt]{[0 -0.1 0]}{[0 2.1 0]}%
\fcLineIIId[linewidth =0.5pt]{[1 -0.1 0]}{[1 2.1 0]}%
\fcLineIIId[linewidth =0.5pt]{[2 -0.1 0]}{[2 2.1 0]}%
}%
\uncover<3->{%
\fcPutIIId[l]{[0 0 0]}{$~~(a,c)$}
\fcPutIIId[lb]{[1.9 2.2 0]}{$~~(b,d)$}
}%
\uncover<9>{%
\pscustom*[linecolor=cyan]{\fcPolyLineIIId{[1 0 0] [2 0 0] [2 1 0] [1 1 0] [1 0 0]}}%
}%
\uncover<5-9>{%
\fcPolyLineIIId[linecolor=cyan]{[1 0 0] [2 0 0] [2 1 0] [1 1 0] [1 0 0]}
\fcDotIIId{[1.5 0.5 0]}%
}%
\uncover<10>{%
\fcBoxIIIdFilledNew[linecolor=blue, colorUV=cyan]{[2 0 0]}{[2 1 0]}{[1 0 0]}{[2 0 2 dict begin /x 1.5 def /y 0.5 def theRFf end]}%
}%
\uncover<3-9>{%
\fcPutIIId[r]{[1.5 -0.1 0]}{\alert<3>{$\Delta x$}}%
}%
\uncover<3-10>{%
\fcPutIIId[bl]{[-0.1 1.5 0]}{\alert<4>{$\Delta y$}}%
}%
\uncover<6>{%
\fcDotIIId{[0.5 0.5 0]}%
\fcDotIIId{[0.5 1.5 0]}%
\fcDotIIId{[1.5 1.5 0]}%
}%
\uncover<7-10>{%
\fcPutIIId[r]{[2 dict begin /x 1.5 def /y 0.5 def x y theRFf end ]}{$f(P_{s,t})~~$}%
}%
\uncover<7-9>{%
\fcLineIIId[linecolor=red]{[1.5 0.5 0]}{[2 dict begin /x 1.5 def /y 0.5 def x y theRFf end]}
\fcDotIIId{[2 dict begin /x 1.5 def /y 0.5 def x y theRFf end]}
}%
\multido{\na=11+1}{10}{%
\only<\na>{%
\fcRectangularRiemannSum[colorUV=cyan, linecolor=blue]{0}{0}{2}{2}{\na \space 10 sub 3 mul 1 sub}{\na\space 10 sub 3 mul 1 sub}{theRFf}
}%
}%
\uncover<21->{%
\fcRectangularRiemannSum[colorUV=cyan, linecolor=blue]{0}{0}{2}{2}{10 3 mul 1 sub}{10 3 mul 1 sub}{theRFf}%
\fcStartIIIdScene%
\fcSurfaceInScene[linecolor=blue, colorUV=cyan]{0}{0}{2}{2}{[2 dict begin /x u def /y v def x y theRFf end]}{}%
\fcFinishIIIdScene%
}
\end{pspicture}
\column{0.7\textwidth}
\begin{itemize}
\item Suppose region of integration $\mathcal R$ is rectangle, i.e., $\mathcal{R} = [a,b] \times [c,d]$, integration w.r.t. $\diff A = \diff x \diff y$.
$\iint_{\mathcal{R}} f(P) \diff A = \iint_{[a,b] \times [c,d]} f(x,y) \; \diff x\, \diff y$.
\item<2-> If the integral exists, it can be approximated by any fine enough Riemann sum.
\item<3-> Simplest partition coming to mind: divide $\mathcal R$ into $n\times n$ equal pieces, sides $\alert<3>{\Delta x = \frac{b-a}{n}}$, $\alert<4>{\Delta y = \frac{d-c}{n}} $.
\item<5-> For $(s,t)^{th}$ rectangle $D_{st}$, choose midpoint as sampling point, \uncover<6->{ i.e., sample at $P_{s,t}= \left( a+ \left( s - \frac{1}{2} \right) \Delta x, c+\left( t-\frac{1}{2}\right)\Delta y\right) $.}
\end{itemize}
\end{columns}
\uncover<7->{
\[
\begin{array}{r@{~}c@{~}l}
\displaystyle \alert<21>{ \iint\limits_{\mathcal R} f(x,y)  \diff x \diff y }  &=&\displaystyle \alert<12-20>{\lim\limits_{n \to \infty}  \alert<11>{\sum_{1\leq s,t \leq n} \alert<10>{ \alert<8>{ f \left(P_{s,t} \right)} \alert<9,22>{ \text{area}(D_{st})} }} }\\
\uncover<22->{ &\approx&\displaystyle \sum_{1\leq i,j \leq n} f\left(P_{s,t} \right)  \alert<22>{ \Delta x  \Delta y } \quad .}
\end{array}
\]
}
\end{frame}
 

%\begin{frame}
\begin{example}
\begin{columns}
\column{0.3 \textwidth}
\psset{xunit=0.3cm, yunit=0.3cm}
\begin{pspicture}(-4,-4)(4,5.5)%
\tiny%
\renewcommand{\fcScreen}{[-1 -1.1 -3] 0}%
\fcBoundingBox{-4}{-4}{4}{5.5}%
\fcAxesIIId{5}{5}{11}%
\uncover<3->{%
\fcLineIIId[linewidth=0.6pt]{[4 -0.1 0]}{[4 2.1 0]}%
\fcLineIIId[linewidth=0.6pt]{[2 -0.1 0]}{[2 2.1 0]}%
\fcLineIIId[linewidth=0.6pt]{[0 -0.1 0]}{[0 2.1 0]}%
\fcLineIIId[linewidth=0.6pt]{[-0.1 0 0]}{[4.1 0 0]}%
\fcLineIIId[linewidth=0.6pt]{[-0.1 1 0]}{[4.1 1 0]}%
\fcLineIIId[linewidth=0.6pt]{[-0.1 2 0]}{[4.1 2 0]}%
}%
\uncover<5->{%
\fcDotIIId{[1 0.5 0]}%
\fcDotIIId{[1 1.5 0]}%
\fcDotIIId{[3 0.5 0]}%
\fcDotIIId{[3 1.5 0]}%
}%
\uncover<7->{%
\fcRectangularRiemannSum[colorUV=cyan, linecolor=blue]{0}{0}{4}{2}{2}{2}{x x mul y add}}%
\end{pspicture}
\column{0.7\textwidth}
Use the Midpoint Rule to approximate $\displaystyle \iint_{[0,4]\times [0,2]} x^2y \diff x\diff y,$ with each side divided into $n=2$ pieces.

\uncover<2->{\alertNoH{2,3}{The small rectangles have dimensions}} \fcAnswer{3}{ $\frac{4-0}{2} \cdot \frac{2-0}{2} = 2\cdot 1$ and area $2$.} \uncover<4->{\alertNoH{4,5}{The midpoints are}}
\end{columns}
\fcAnswer{5}{$\displaystyle
P_{11} = \left(1,\frac{1}{2}\right), \quad P_{12} = \left(1,\frac{3}{2}\right), \quad P_{21} = \left(3,\frac{1}{2}\right),  \quad P_{22} = \left(3,\frac{3}{2}\right)\; .
$}
\[
\begin{array}{r@{~}c@{~}l}
\displaystyle \fcQuestion{6}{\iint\limits_{[0,4]\times [0,2]} x^2y\; \diff x \diff y}  &\fcQuestion{6}{\approx} &  \displaystyle \fcAnswer{7}{ 2\left(f \left(1,\frac{1}{2 }\right)  +  f\left(3,\frac{1}{2}\right) + f\left(1,\frac{3}{2}\right)   + f\left(3,\frac{3}{2}\right) \right)} \\
\uncover<8->{& =&\displaystyle   1\cdot \frac{1}{2}\cdot 2 + 9 \cdot \frac{1}{2} \cdot 2 + 1\cdot \frac{3}{2} \cdot 2 + 9 \cdot \frac{3}{2} \cdot 2} \\
\uncover<9->{&=&\displaystyle  1+9+3+27 = 40\; .}
\end{array}
\]

\end{example}
\end{frame}
 

%\begin{frame}
\frametitle{Properties}
\[
\iint_{\mathcal{R}} f(P) \; \diff A = \lim_{\max_k(\text{diam}D_k) \to 0} \sum_k f(P_k) \; \text{area}(D_k)
\]
\begin{itemize}
\item<2-> If $f$ is bounded and continuous, except maybe on a finite number of smooth curves, then the limit exists and is finite.
\item  Linearity with respect to function
\[
\iint_{\mathcal{R}} [\lambda f(P) +\mu g(P)] \, \diff A = \lambda \, \iint_{\mathcal{R}} f(P) \, \diff A +\mu \, \iint_{\mathcal{R}} g(P) \, \diff A \; .
\]
\item Domain additivity: if $\mathcal{R}_1$ and $\mathcal{R}_2$ intersect only along boundaries:
\[
\iint_{\mathcal{R}_1\cup \mathcal{R}_2} f(P)\, \diff A = \iint_{\mathcal{R}_1} f(P)\, \diff A + \iint_{\mathcal{R}_2} f(P)\, \diff A 
\]
\item \pause Monotonicity property: If $m \leq f(P) \leq M$ for all $P$ in $\mathcal{R}$, then
\[
m \, \text{area}(\mathcal{R}) \leq \iint_{\mathcal{R}} f(P)\, \diff A \leq M \, \text{area} (\mathcal{R})\; .
\]
\end{itemize}
\end{frame} 
%\begin{frame}
\frametitle{Applications}
\begin{itemize}
\item Average value of $f$ on $\mathcal{R}$: constant value that would produce the same accumulation.
\[
\iint _{\mathcal{R}} f(P) \, \diff A =
\iint_{\mathcal{R}} (\text{average value}) \, \diff A =  (\text{average value})
\cdot \text{area}(\mathcal{R}) \; \Longrightarrow 
\]
\[
\text{average value of }f \text{ on } \mathcal{R} = \frac{1}{\text{area}(\mathcal{R})} \iint_{\mathcal{R}} f(P) \, \diff A\; .
\]
\end{itemize}
\end{frame} 
%\begin{frame}
\begin{theorem}[Mean Value Theorem]
If $f$ is continuous on $\mathcal{R}$, then there exists $P_0$ in $\mathcal{R}$ such that
\[
f(P_0) = \frac{1}{\text{area}(\mathcal{R})} \iint_{\mathcal R} f(Q) \,\diff A
\]
\end{theorem}
\begin{theorem}[Analog of Fundamental Theorem of Calculus]
If $f$ is continuous around $P$, then
\[
\lim_{D \to \{ P \} } \frac{1}{\text{area}(D)} \iint_D f(Q) \diff A = f(P)
\]
\end{theorem}
\end{frame} 
%\begin{frame}
  \frametitle{Vectorial Integrals}

    $$\int\!\!\int_{\mathcal{R}} \fcv{F}(P) \; dA = \lim_{\text{maxdiam}(\mathcal{D}) \to 0} \sum_k \fcv{F}(P_k) \; \text{area}(D_k)$$
%
    The definition can be extended to functions with vectorial output.

\underline{Example}: Electric force on a lamina

    \begin{itemize}
      \item Charge $Q$, at a fixed point
      \item Charge $q$, uniformly distributed on a planar lamina $\mathcal{R}$
      \item Total force on $Q$?
    \end{itemize}
    %
    $$dq = (\text{density of charge}) \, dA = \frac{q}{A(\mathcal{R})}\, dA$$
    %
    $$d\fcv{F} = \frac{\epsilon Q dq}{|\fcv{r}|^3} \; \fcv{r}  =
    \left( \epsilon \frac{Q q}{A(\mathcal{R}) |\fcv{r}|^3} \, \fcv{r}\right)\, dA$$
    %
    $$\fcv{F} = \int\!\!\!\int_{\mathcal{R}} d\fcv{F} =
    \int\!\!\!\int_{\mathcal{R}} \left( \epsilon \frac{Q q}{A(\mathcal{R}) |\fcv{r}|^3} \, \fcv{r}\right)\, dA$$

    \pause Constant Multiple Rule:
    %
    $$\int\!\!\int_{\mathcal{R}} c \fcv{F} \; dA =
    c \int\!\!\int_{\mathcal{R}} \fcv{F} \; dA$$
\end{frame} 
%\begin{frame}
\frametitle{Iterated Integrals}
\begin{columns}
\column{0.25\textwidth}
\psset{xunit=0.6cm, yunit=0.6cm}
\begin{pspicture}(-1,-1)(1,1)
\tiny
\renewcommand{\fcScreen}{[-1 -1.1 -1] 0}
\fcAxesIIId{3}{3}{4.5}%
\uncover<1>{\fcRectangularRiemannSum[colorUV=cyan, linecolor=blue]{0}{0}{2}{2} {25}{25}{ x 2 sub dup mul y 2 sub dup mul add 2 div}
\fcStartIIIdScene%
\fcSurfaceInScene[linecolor=blue, colorUV=blue]{0}{0}{2}{2}{[2 dict begin /x u def /y v def x y  x 2 sub dup mul y 2 sub dup mul add 2 div end]}{}%
\fcFinishIIIdScene%
}%
\uncover<2->{%
\fcRectangularRiemannSum[colorUV=cyan, linecolor=blue]{0}{0}{2}{0.4}{5}{1}{x 2 sub dup mul y 2 sub dup mul add 2 div}%
\uncover<4,5,10>{%
\fcRectangularRiemannSum[colorUV=pink, linecolor=red]{0}{0}{2}{0.4}{5}{1}{x 2 sub dup mul y 2 sub dup mul add 2 div}%
}
\fcRectangularRiemannSum[colorUV=cyan, linecolor=blue]{0}{0.4}{2}{0.8}{5}{1}{x 2 sub dup mul y 2 sub dup mul add 2 div}%
\uncover<6>{%
\fcRectangularRiemannSum[colorUV=pink, linecolor=red]{0}{0.4}{2}{0.8}{5}{1}{x 2 sub dup mul y 2 sub dup mul add 2 div}%
}
\fcRectangularRiemannSum[colorUV=cyan, linecolor=blue]{0}{0.8}{2}{1.2}{5}{1}{x 2 sub dup mul y 2 sub dup mul add 2 div}%
\uncover<7>{%
\fcRectangularRiemannSum[colorUV=pink, linecolor=red]{0}{0.8}{2}{1.2}{5}{1}{x 2 sub dup mul y 2 sub dup mul add 2 div}%
}
\fcRectangularRiemannSum[colorUV=cyan, linecolor=blue]{0}{1.2}{2}{1.6}{5}{1}{x 2 sub dup mul y 2 sub dup mul add 2 div}%
\uncover<8>{%
\fcRectangularRiemannSum[colorUV=pink, linecolor=red]{0}{1.2}{2}{1.6}{5}{1}{x 2 sub dup mul y 2 sub dup mul add 2 div}%
}
\fcRectangularRiemannSum[colorUV=cyan, linecolor=blue]{0}{1.6}{2}{2}{5}{1}{x 2 sub dup mul y 2 sub dup mul add 2 div}%
\uncover<9>{%
\fcRectangularRiemannSum[colorUV=pink, linecolor=red]{0}{1.6}{2}{2}{5}{1}{x 2 sub dup mul y 2 sub dup mul add 2 div}%
}
}%
\fcCurveIIId[linecolor=black]{0}{2}{ [2 dict begin /x t def /y 0 def x y  x 2 sub dup mul y 2 sub dup mul add 2 div end]}
\fcCurveIIId[linecolor=black]{0}{2}{ [2 dict begin /x t def /y 2 def x y  x 2 sub dup mul y 2 sub dup mul add 2 div end]}
\fcCurveIIId[linecolor=black]{0}{2}{ [2 dict begin /x 0 def /y t def x y  x 2 sub dup mul y 2 sub dup mul add 2 div end]}
\fcCurveIIId[linecolor=black]{0}{2}{ [2 dict begin /x 2 def /y t def x y  x 2 sub dup mul y 2 sub dup mul add 2 div end]}
\fcPolyLineIIId{[2 0 2] [2 0 0] [2 2 0] [0 2 0] [0 2 2] }
\end{pspicture}
\column{0.75\textwidth}
\[
\begin{array}{rcl}
\displaystyle \iint\limits_{[a,b] \times [c,d]} f(x,y)  \diff x  \diff y &\uncover<2->{\alert<2>{\approx}}&\displaystyle \uncover<2->{ \alert<2>{ \alert<3>{ \sum_{1 \leq i ,j \leq n}}  f ( x_i,y_j)   \Delta x  \Delta y }} \\
&\uncover<3->{=}&\displaystyle  \uncover<3->{\alert<3,5-9>{ \sum \limits_{ j = 1 }^n } \left( \alert<4,10,11>{ \alert<3>{\sum_{i=1}^n} f(x_i,y_j) \Delta x  } \right) \Delta y}\; .
\end{array}
\]
\end{columns}
\uncover<10->{ The $j^{th}$ summand is a Riemann sum for $\displaystyle \alert<10,11>{ g(y_j) = \int_{x=a}^{x=b} f(x,y_j) \diff x }\quad.$}
\[
\begin{array}{r@{~}c@{~}l}
\uncover<11->{\displaystyle \alert<14>{\sum_{j=1}^n}\left( \alert<11>{\alert<14>{\sum_{i=1}^n} f(x_i,y_j) \alert<14>{\Delta x}} \right) \alert<14>{\Delta y}} &\uncover<11->{\approx}&\displaystyle  \uncover<11->{ \alert<15>{\sum_{j=1}^n} \alert<11>{ g(y_j)} \alert<15>{\Delta y}} \uncover<11->{\alert<11>{\approx \int_{y=c}^{y=d} \alert<16>{ g(y)}  \diff y}}
\\
\uncover<13->{ \displaystyle \alert<14>{ \iint\limits_{[a,b] \times [c,d]}} f(x,y)  \alert<14>{\diff x \diff y }&=& \displaystyle  \alert<15>{\int\limits_{y=c}^{y=d}} g(y) \alert<15>{ \diff y }= \int\limits_{y=c}^{y=d} \left( \alert<16>{\int\limits_{x=a}^{x=b} f(x,y)  \diff x} \right)  \diff y}
\end{array}
\]
\end{frame} 
%\begin{frame}
\begin{theorem}
If  $f$ is continuous the double integral $\displaystyle \iint _{ [a,b]\times [c,d]} f(x,y)\diff x \diff y$ exists.
\end{theorem}
\uncover<2->{
\begin{theorem}[Fubini's Theorem]
Suppose the double integral of $f$ exists. Then, except at a \alertNoH{4}{set of measure $0$}, the iterated integrals exist and
$\begin{array}{rcl}
\displaystyle
\iint\limits_{[a,b]\times [c,d]} f(x,y) \; \diff x\diff y &=&\displaystyle \int \limits^{{\color{red}{y=d}}}_{{\color{red}{y=c}}} \left( \int \limits_{{\color{blue}{x=a}}}^{{\color{blue}{x=b}}} f(x,y) \; {\color{blue}{\diff x}} \right)  {\color{red}{\diff y}} \\
&=&\displaystyle \int\limits_{{\color{blue}{x=a }}}^{{\color{blue}{x=b}}}
\left( \int \limits_{{\color{red}{y=c}}}^{{\color{red}{y=d}}} f(x,y)  {\color{red}{\diff y}} \right) {\color{blue}{\diff x}}\; .
\end{array}$
\end{theorem}
}
\uncover<3->{This theorem allows to integrate non-continuous functions.}
\uncover<4->{The term ``\alertNoH{4}{set of measure 0}'' is too technical to define here; usually studied in the subject(s) ``Real Analysis/Measure Theory''.}
\end{frame}
 
%\begin{frame}
\begin{example}
Use iterated integrals to compute the integral $\displaystyle \iint \limits_{ [1,2]\times [2,3]}(2x+3y^2) \; \diff x\diff y$.
\uncover<2->{For $(x,y)$ in $[1,2]\times [2,3]$, \alert<2,3>{$\alert<8>{y} $ takes values between}} \fcAnswer{3}{$ \alert<8>{ c=2}$ and $\alert<8>{d=3}$.}\uncover<4->{For a fixed value \alert<4,5>{$y=y_0$, $\alert<7>{x}$ takes values between}} \fcAnswer{5}{$\alert<7>{a=1}$ and $\alert<7>{b=2}$.}
\[
\begin{array}{r@{~}c@{~}l} 
\uncover<6->{ \iint\limits_{[1,2]\times [2,3]} (2x+3y^2) \diff x \diff y &=& \int \limits_{{\alert<8>{y=2}}}^{{\alert<8>{y=3}}} \left( \alert<9-14>{ \int \limits_{{ \alert<7>{x=1}}}^{{\alert<7>{x=2}}} (2x+3y^2) {{\diff x}}} \right) {{\diff y} } }\\
\uncover<9->{ &=& \int \limits_{{{y=2}}}^{{{y=3}}} {\left[ \fcAnswer{10}{ \alert<15,16>{x^2+3y^2x} } \right]}_{ {\uncover<9-11>{\alert<11>{ \textbf{?}}} \uncover<12->{\alert<12,16>{x =1}}}}^{{ \uncover<9-13>{\alert<13>{\textbf{?}}} \uncover<14->{ \alert<14,15>{x = 2 } }}} ~~ { \diff y }} \\
\uncover<15->{ &=& \int \limits_{{{y=2}}}^{{{y=3}}} \left(\alert<15>{ ( 4+ 6y^2 )} -\alert<16>{(1 + 3 y^2)} \right) {{\diff y}}} \\
\uncover<17->{&=& \alert<18,19>{\int\limits_{{{y=2}}}^{{{y=3}}} (3+3y^2) {{\diff y}}} \uncover<18->{\alert<18>{=} {\left[\fcAnswer{19}{3y+y^3} \right]}_{y=2}^{y=3} }}\\
\uncover<20->{&=&36-14=22\; .}
\end{array}
\]
\end{example}
\end{frame} 
%\begin{frame}
\frametitle{More General Regions}
What makes iterated integrals work over rectangular regions? \uncover<2->{Slices with respect to one variable are intervals in the other.} \uncover<3->{If variable is $x$:}


\begin{columns}
\column{0.3\textwidth}
\begin{pspicture}(-0.5,-0.5)(3,2.7)
\tiny
\fcAxesStandard{-0.5}{-0.5}{3}{2.7}
\pstVerb{
/f {x 170 mul sin 0.4 mul 0.6 add} def 
/g {x 190 mul cos 0.4 mul 2 add x 4 div sub} def
}
\pscustom*[linecolor=\fcColorAreaUnderGraph]{
\psplot{0.2 }{2}{f }
\psplot{2 }{0.2}{g}
}
\multido{\na=4+1}{4}{%
\pstVerb{1 dict begin /x \na\space 4 sub 0.5 mul 0.3 add def}%
\only<\na->{\psline[linecolor=red, linewidth=2pt](! x f)(! x g)}%
\only<\na->{%
\psline[linestyle=dotted](! x f)(! x 0)%
\rput[t](! x -0.1){$\alert<8>{x_{\fcEvalToInt{\na-3}}}$}%
}%
\pstVerb{end}%
}%
\uncover<3>{\rput[t](! 0.3 -0.1){$x_1$}}
\psplot{0.2}{2}{f}
\rput[l](2.1, 2.1){$y=g(x)$}
\psplot{2}{0.2}{g}
\rput[l](2, 0.2){$y=f(x)$}
\end{pspicture}
\column{0.7\textwidth}
\begin{itemize}
\item<3-> fix $x$,
\item<4-> integrate with respect to $y$,
\item<8-> to obtain function that depends only on $x$,
\item<9-> then integrate the so obtained function in $x$.
\end{itemize}
\end{columns}
\uncover<10->{So far used rectangular regions;} \uncover<11->{this also works if \alert<11>{slices are intervals whose endpoints depend continuously} on the location of the slice.} 
\begin{itemize}
\item<12-> Regions of type I: vertical slices are segments.
\item<12-> Regions of type II: horizontal slices are segments.
\end{itemize}
\uncover<12->{We call such regions curvilinear trapezoids.}
\end{frame} 
}
\end{document}
