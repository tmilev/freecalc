\documentclass%
%[handout]
{beamer}
% % % % % % % %
% % % % % % % %
% % % % % % % %
%IMPORTANT
%compiles with
%pdflatex -shell-escape
%IMPORTANT
% % % % % % % %
% % % % % % % %
% % % % % % % %
\mode<presentation>
{
\useinnertheme{rounded}
\useoutertheme{infolines}
\usecolortheme{orchid}
\usecolortheme{whale}
}

\usepackage[english]{babel}
\usepackage[latin1]{inputenc}
\usepackage{times}
\usepackage[T1]{fontenc}
\usepackage{../example-templates}
\usepackage{../pstricks-commands}

\usepackage{auto-pst-pdf}
\usepackage{pst-plot}
%\usepackage{pstricks-add}

% Or whatever. Note that the encoding and the font should match. If T1
% does not look nice, try deleting the line with the fontenc.


\graphicspath{{../../modules/}}

\newtheoremstyle{partialproof}{3pt}{3pt}{}{}{}{.}{.5em}{}
\theoremstyle{partialproof} \newtheorem{partialproof}[theorem]{Proof.}
%\DeclareMathOperator{\diff}{d}
\setbeamertemplate{navigation symbols}{}

\includeonlylecture{1}

\newcommand{\lect}[3]{
  \date{#1}
  \lecture[#1]{#2}{#3}
}

\setbeamertemplate{footline}
{
  \leavevmode%
  \hbox{%
  \begin{beamercolorbox}[wd=.333333\paperwidth,ht=2.25ex,dp=1ex,center]{author in head/foot}%
    \usebeamerfont{author in head/foot}\insertshortauthor
  \end{beamercolorbox}%
  \begin{beamercolorbox}[wd=.333333\paperwidth,ht=2.25ex,dp=1ex,center]{title in head/foot}%
    \usebeamerfont{title in head/foot}\insertshorttitle
  \end{beamercolorbox}%
  \begin{beamercolorbox}[wd=.333333\paperwidth,ht=2.25ex,dp=1ex,center]{date in head/foot}%
    \usebeamerfont{date in head/foot}\insertshortdate{}
  \end{beamercolorbox}}%
  \vskip0pt%
}

% If you have a file called "university-logo-filename.xxx", where xxx
% is a graphic format that can be processed by latex or pdflatex,
% resp., then you can add a logo as follows:

%\pgfdeclareimage[height=0.8cm]{logo}{bluelogo}
%\logo{\pgfuseimage{logo}}
\renewcommand{\Arcsin}{\arcsin}
\renewcommand{\Arccos}{\arccos}
\renewcommand{\Arctan}{\arctan}
\renewcommand{\Arccot}{\text{arccot\hspace{0.03cm}}}
\renewcommand{\Arcsec}{\text{arcsec\hspace{0.03cm}}}
\renewcommand{\Arccsc}{\text{arccsc\hspace{0.03cm}}}



\begin{document}

\AtBeginLecture{%

\title[\insertlecture]{FreeCalc}
\subtitle{\insertlecture}
\author[FreeCalc]{}
\institute[UMass Boston]{University of Massachusetts Boston}
\date{\insertshortlecture}
\begin{frame}
  \titlepage
\end{frame}
}%

% begin lecture
\lect{\today}{Sample}{1}{
%\begin{frame}
\frametitle{More General Regions}
What makes iterated integrals work over rectangular regions? \uncover<2->{Slices with respect to one variable are intervals in the other.} \uncover<3->{If variable is $x$:}


\begin{columns}
\column{0.3\textwidth}
\begin{pspicture}(-0.5,-0.5)(3,2.7)
\tiny
\fcAxesStandard{-0.5}{-0.5}{3}{2.7}
\pstVerb{
/f {x 170 mul sin 0.4 mul 0.6 add} def 
/g {x 190 mul cos 0.4 mul 2 add x 4 div sub} def
}
\pscustom*[linecolor=\fcColorAreaUnderGraph]{
\psplot{0.2 }{2}{f }
\psplot{2 }{0.2}{g}
}
\multido{\na=4+1}{4}{%
\pstVerb{1 dict begin /x \na\space 4 sub 0.5 mul 0.3 add def}%
\only<\na->{\psline[linecolor=red, linewidth=2pt](! x f)(! x g)}%
\only<\na->{%
\psline[linestyle=dotted](! x f)(! x 0)%
\rput[t](! x -0.1){$\alert<8>{x_{\fcEvalToInt{\na-3}}}$}%
}%
\pstVerb{end}%
}%
\uncover<3>{\rput[t](! 0.3 -0.1){$x_1$}}
\psplot{0.2}{2}{f}
\rput[l](2.1, 2.1){$y=g(x)$}
\psplot{2}{0.2}{g}
\rput[l](2, 0.2){$y=f(x)$}
\end{pspicture}
\column{0.7\textwidth}
\begin{itemize}
\item<3-> fix $x$,
\item<4-> integrate with respect to $y$,
\item<8-> to obtain function that depends only on $x$,
\item<9-> then integrate the so obtained function in $x$.
\end{itemize}
\end{columns}
\uncover<10->{So far used rectangular regions;} \uncover<11->{this also works if \alert<11>{slices are intervals whose endpoints depend continuously} on the location of the slice.} 
\begin{itemize}
\item<12-> Regions of type I: vertical slices are segments.
\item<12-> Regions of type II: horizontal slices are segments.
\end{itemize}
\uncover<12->{We call such regions curvilinear trapezoids.}
\end{frame} 
%\begin{frame}
\frametitle{Strategy: Curvilinear Trapezoids (Type I)}
\begin{columns}
\column{0.3\textwidth}
\begin{pspicture}(-1,-1)(2,1)
\tiny
\fcAxesStandard{-0.5}{-0.5}{3}{2.7}
\pstVerb{ 10 dict begin
/f1 {x 170 mul sin 0.4 mul 0.6 add} def 
/f2 {x 190 mul cos 0.4 mul 2 add x 4 div sub} def
}
\pscustom*[linecolor=\fcColorAreaUnderGraph]{%
\psplot{0.2 }{2}{f1 }%
\psplot{2 }{0.2}{f2}%
}%
\psplot{0.2}{2}{f1}%
\psplot{2}{0.2}{f2}%
\uncover<7>{%
\psplot[linewidth=2pt, linecolor=red]{0.2}{2}{f1}%
\psplot[linewidth=2pt, linecolor=red]{2}{0.2}{f2}%
}%
\pstVerb{/x 0.2 def}
\psline(! x f1)(! x f2)
\uncover<2>{\psline[linewidth=2pt, linecolor=red](! x f1)(! x f2)}
\uncover<2->{%
\psline[linestyle=dotted](! x f1)(! x 0)%
\rput[lt](! x -0.2 ){$~a$}
}%
\pstVerb{/x 2 def}
\psline(! x f1)(! x f2)
\uncover<3>{\psline[linewidth=2pt, linecolor=red](! x f1)(! x f2)}
\uncover<3->{%
\psline[linestyle=dotted](! x f1)(! x 0)%
\rput[lt](! x -0.2 ){$~b$}
}%
\pstVerb{/x 1 def}
\uncover<4>{\psline[linewidth=2pt, linecolor=red](! x f1)(! x f2)}
\uncover<5->{\psline(! x f1)(! x f2)}
\uncover<4->{%
\psline[linestyle=dotted](! x f1)(! x 0)%
\rput[lt](! x -0.2){$~x$}
}%
\uncover<6,7>{
\psline[linecolor=red, linewidth=2pt](0.2, -0.5)(0.2, 2.7)
\psline[linecolor=red, linewidth=2pt](2, -0.5)(2, 2.7)
}
\uncover<5->{%
\fcFullDot{x}{f1}%
\fcFullDot{x}{f2}%
\rput[lb](! x f1){$~~(x,f(x))$}%
\rput[lt](! x f2){$~~(x,g(x))$}%
}%
\rput[l](2.1, 2.1){$y=g(x)$}
\rput[l](2.1, 0.2){$y=f(x)$}
\pstVerb{end}%
\end{pspicture}
\column{0.7\textwidth}
\begin{itemize}
\item<2-> Identify the \alert<2>{leftmost point(s), with $x$-coordinate $x=a$} and \alert<3>{the rightmost point(s), $x=b$}.
\item<4-> Draw a vertical slice at a value $x$ between $a$ and $b$.
\item<5-> Find the lowest point on that slice, $(x,f(x))$ and the highest point, $(x,g(x))$.
\end{itemize}
\end{columns}
\uncover<6->{
The region is the region bounded by:
\begin{itemize}
\item<6-> the vertical lines $\alert<6>{x=a}$ and $\alert<6>{x=b}$;
\item<7-> the graphs of $\alert<7>{y=f(x)}$ and $\alert<7>{y=g(x)}$, with $f,g \colon [a,b] \to \mathbb{R}$.
 \end{itemize}
}
\uncover<8->{
\[
\mathcal{R} = \{(x,y)  | {\color{blue}{a \leq x \leq b}} ,  {\color{red}{f(x) \leq y \leq g(x)}} \}  .
\]
}
\uncover<9->{
\[
\iint_{\mathcal{R}} f(x,y) \; \diff x\diff y = \int_{{\color{blue}{x=a}}}^{{\color{blue}{x=b}}} \left( \int_{{\color{red}{y=f(x)}}}^{{\color{red}{y=g(x)}}} f({\color{blue}{x}},{\color{red}{y}})  {\color{red}{\diff y}} \right) \; \color{blue}{\diff x}
\]
}
\end{frame} 
\begin{frame}
\frametitle{Strategy: Curvilinear Trapezoids (Type II)}
\begin{columns}
\column{0.3\textwidth}
\begin{pspicture}(-0.5,-0.5)(3.2,2.7)
\tiny
\fcAxesStandard{-0.5}{-0.5}{3.2}{2.7}
\pstVerb{ 10 dict begin
/f1 {t 170 mul sin 0.4 mul 0.6 add} def 
/f2 {t 190 mul cos 0.4 mul 2 add t 4 div sub} def
}
\pscustom*[linecolor=\fcColorAreaUnderGraph]{%
\parametricplot{0.2 }{2}{f1 t}%
\parametricplot{2 }{0.2}{f2 t}%
}%
\parametricplot{0.2}{2}{f1 t}%
\parametricplot{2}{0.2}{f2 t}%
\uncover<7>{%
\parametricplot[linewidth=2pt, linecolor=red]{0.2}{2}{f1 t}%
\parametricplot[linewidth=2pt, linecolor=red]{2}{0.2}{f2 t}%
}%
\pstVerb{/t 0.2 def}
\psline(! f1 t)(! f2 t)
\uncover<2>{\psline[linewidth=2pt, linecolor=red](! f1 t)(! f2 t)}
\uncover<2->{%
\psline[linestyle=dotted](! f1 t)(! 0 t)%
\rput[lb](! -0.2 t 0.1 add){$c~$}
}%
\pstVerb{/t 2 def}
\psline(! f1 t)(! f2 t)
\uncover<3>{\psline[linewidth=2pt, linecolor=red](! f1 t)(! f2 t)}
\uncover<3->{%
\psline[linestyle=dotted](! f1 t)(! 0 t)%
\rput[rb](! -0.2 t 0.1 add){$d~$}
}%
\pstVerb{/t 1 def}
\uncover<4>{\psline[linewidth=2pt, linecolor=red](! f1 t)(! f2 t)}
\uncover<5->{\psline(! f1 t)(!  f2 t)}
\uncover<4->{%
\psline[linestyle=dotted](! f1 t)(! 0 t)%
\rput[rb](! -0.2 t 0.1 add){$x~$}
}%
\uncover<6,7>{
\psline[linecolor=red, linewidth=2pt](-0.5, 0.2)(2.7, 0.2)
\psline[linecolor=red, linewidth=2pt](-0.5,2)(2.7, 2)
}
\uncover<5->{%
\fcFullDot{f1}{t}%
\fcFullDot{f2}{t}%
\rput[b](! f1 t 0.2 add){$(f(y),y)$}%
\rput[lt](! f2 t){$~~(g(y),y)$}%
}%
\rput[bl](2.2, 0.3){$x=g(y)$}
\rput[br](0.9, 0.3){$x=f(y)$}
\pstVerb{end}%
\end{pspicture}
\column{0.7\textwidth}

\begin{itemize}
\item<2-> Identify the \alert<2>{lowest point(s), with $y$-coordinate $y=c$} and \alert<3>{the topmost point(s), $y=d$}.
\item<4-> Draw a generic horizontal slice at some value $y$ between $c$ and $d$.
\item<5-> Find the lowest point on that slice, $(f(y),y)$ and the topmost point, $(g(y),y)$.
\end{itemize}
\end{columns}
\uncover<6->{
The region is bounded by:
\begin{itemize}
\item<6-> horizontal lines $\alert<6>{y=c}$ and $\alert<6>{y=d}$
\item<7-> graphs of $\alert<7>{x=f(y)}$ and $\alert<7>{x=g(y)}$, with  $f,g \colon [c,d] \to \mathbb{R}$:
\end{itemize}
}
\uncover<8->{
\[
\mathcal{R} = \{(x,y) \; | \; \color{red}{c \leq y \leq d}\; , \; \color{blue}{f(y) \leq x \leq g(y)} \} \; .
\]
}
\uncover<9->{
\[
\iint_{\mathcal{R}} f(x,y) \; \diff x\diff y = \int_{{\color{red}{y=c}}}^{{\color{red}{y=d}}} \left( \int_{{\color{blue}{x=f(y)}}}^{{\color{blue}{x=g(y)}}} f({\color{blue}{x}},{\color{red}{y}}) {\color{blue}{\diff x}} \right) \; \color{red}{\diff y}
\]
}
\end{frame} 
\begin{frame}
  \frametitle{Examples}

  \begin{itemize}
    \item $\mathcal{R}_1$: region bounded by $y=2x$ and $y=x^2$. Compute
%
$$\int\!\!\!\int_{\mathcal{R}_1} (x^2+y^2) \; dxdy$$

\item $\mathcal{R}_2$: region bounded by $y=x-1$ and $y^2=2x+6$. Compute
%
$$ \int\!\!\!\int_{\mathcal{R}_2} xy \; dxdy$$

\item $\mathcal{R}$: region bounded by
$y=(x+1)^2$, $x=y-y^3$, the line $x=-1$ and the line $y=-1$. Set-up iterated integrals for
%
$$\int\!\!\!\int_{\mathcal{R}} f(x,y) \, dA \; .$$

\item How do we compute
%
$$\int_0^1 \int_{3y}^3 e^{x^2} \, dx \; dy \; \qquad , \qquad  \int\!\!\!\int_{[0,\infty) \times [0,\infty)} e^{-x-y} \, dxdy $$

  \end{itemize}
\end{frame} 
\section{Triple Integrals}
\begin{frame}
  \frametitle{Density to Mass}
  \underline{Question}: If we know the density at every point, can we find the mass?

  \underline{Answer}: Yes.

\begin{itemize}
  \item Partition the region $\cR$ into regions with small diameter; let $\cD=\{D_1,\ldots,D_N\}$ be such a partition, and let
      %
      $$\text{maxdiam}(\cD) = \max_{k} \text{diam}(D_k)\; .$$
      %
  \item For each region $D_k$, choose a sample point $P_k$ inside $D_k$:
      %
      $$\text{mass}(D_k) \simeq \rho(P_k) \text{vol}(D_k)\; .$$
      %
      \item Mass is approximated by the sum of the masses of the subregions:
      %
      $$\text{mass}(\cR) \simeq \sum_{k=1}^{N} \rho(P_k)\text{vol}(D_k)\; .$$
      %
      \item Take partitions with diameter closer and closer to 0:
      %
      $$\text{mass}(\cR) = \lim_{\text{maxdiam}(\cD) \to 0}  \sum_{k=1}^{N} \rho(P_k)\text{vol}(D_k)\; .$$
\end{itemize}
\end{frame} 
\begin{frame}
  \frametitle{Triple Integrals}

  $f$ defined on region $\cR$, \emph{scalar} or \emph{vectorial} function.

  \begin{definition}
  If the limit
%
$$\lim_{\text{maxdiam}(\cD) \to 0}  \sum_{k=1}^{N} f(P_k)\text{vol}(D_k)$$
%
exists and is finite, its value is called
\begin{quote}
  \emph{the integral of $f$ on $\cR$ with respect to volume}
\end{quote}
 and is denoted by
%
$$\int\!\!\!\int\!\!\!\int_{\cR} f(P) \; dV\; .$$
  \end{definition}

\begin{itemize}
  \item If $f$ is a scalar function, then the value of the integral is a scalar;
  \item If $f$ is a vectorial function, then the value of the integral is a vector.
\end{itemize}

The limit does not always exist. It exists if the function $f$ is continuous.

\end{frame} 
\begin{frame}
\frametitle{Theoretical Examples}
\begin{itemize}
\item The volume of a region is defined via a triple integral.
\[
\text{vol}(\cR) = \iiint_{\cR} 1 \cdot \diff V
\]
\item<2-> The mass of a body can be computed via a triple integral.
\[
\text{mass}(\cR) = \iiint_{\cR} \text{density}(P) \cdot \diff V\; .
\]
\item<3-> Average value of function $f$ (with respect to volume) is given by:
\[
\text{average value of }f = \frac{1}{\text{vol}(\cR)} \iiint_{\cR} f(P) \cdot \diff V .
\]
\item<4-> The average value of a function $f$ with respect to mass distribution:
\[
\text{av. value of }f = \frac{1}{\text{m}(\cR)} \iiint_{\cR} f(P)\,  \diff m = \frac{1}{\text{m}(\cR)} \iiint_{\cR} f(P)\rho(P)\, \diff V\; .
\]
\end{itemize}

\end{frame} 
\begin{frame}
\frametitle{Iterated Integrals}
\begin{itemize}
\item To compute a triple integral over $\cR$ one reduces to iterated integrals.
\item<2-> One reduces to 
\begin{itemize}
\item<2-> a \alert<4>{single integral of a double integral}
\item<3-> or \alert<8>{double integral of a single integral}.
\end{itemize} 
\item<4-> Single integral of a double integral: \alert<4>{decomposition into slices}.
\uncover<5->{
\begin{itemize}
\item \alert<6>{Project the body on an axis.}
\item \alert<7>{Look at 2D slices perpendicular to that axis (CT-scan).}
\end{itemize}
\[
\iiint_{\cR}f(P)  \diff V = \alert<6>{\int_{\text{location of slice}} } \left(\alert<7>{\iint_{\text{slice}} f(P) \diff A} \right) \alert<6>{\diff h}
\]
}
\item<8-> Double integral of a single integral: decomposition into rods.
\begin{itemize}
\item\alert<9>{Project the body on a plane}.
\item\alert<10>{Look at 1D slices perpendicular to that plane (rods)}.
\end{itemize}
\[
\iiint_{\cR}f(P) \diff V = \alert<9>{\iint_{\text{location of rod}}}  \left( \alert<10>{\int_{\text{rod}} f(P)  \diff h} \right)  \diff A
\]
\end{itemize}
\end{frame} 
\begin{frame}
\frametitle{Example: Moment of Inertia}
\begin{itemize}
\item Problem: compute the moment of inertia $I$
\begin{itemize}
\item \alert<2>{of a rectangular box with sides $2a$, $2b$, and $2c$}
\item \alert<3>{rotating about axis $L$ through center that is perpendicular to a face.}
\item \alert<4>{The box has constant density $\rho$.} \uncover<11->{Therefore it's mass is $m=8\rho abc $.}
\end{itemize}
\item<5-> Coord. system: rotation axis = $z$-axis, $x,y$ axes along box sides.
\uncover<6->{
\[
I = \iiint_{\cR} \rho\, \text{dist}^2(P,L) \diff V = \iiint_{\cR} \rho (x^2+y^2)\, \diff x\diff y \diff z\; .
\]
}
\item<7-> Decompose into slices as follows.
\begin{itemize}
\item<7-> Project $\cR$ onto the $z-$axis \alert<8>{to get segment from $z=-c$ to $z=c$.}
\[
\iiint_{\cR} \rho (x^2+y^2) \diff x \diff y \diff z = \alert<8>{\int_{z=-c}^{z=c}} \left( \iint_{S_z} \rho (x^2+y^2)\, \diff x \diff y \right) \diff z
\]    
\item<9-> For a fixed $z$, the slice $S_z$ is: $-a \leq x \leq a$, $-b \leq  y \leq  b$.
\[
I_L = \alert<10,11>{ \int_{z=-c}^{z=c} \left(\int_{x=-a}^{x=a} \left( \int_{y=-b}^{y=b} \rho (x^2+y^2)  \diff y \right)  \diff x\right)  \diff z} \uncover<10->{\alert<10,11>{=}} \fcAnswer{11}{\frac{m(a^2+b^2)}{3}} \uncover<11->{.}
\]
%

\end{itemize}
  \end{itemize}
\end{frame} 
\begin{frame}
\begin{example}
\begin{columns}
\column{0.3\textwidth} 
\begin{pspicture}(-1,-1)(1,1)
\tiny
\renewcommand{\fcScreen}{[-1 3 -1] 0}
\fcBoundingBox{-0.4}{-1.8}{3.3}{2.4}
\fcStartIIIdScene%
\only<5->{\fcPatchInScene{[0 0 0]}{[0 1 0]}{[0 0 2]}}%
\only<4->{\fcPatchInScene{[0 0 0]}{[1 0 0]}{[0 1 0]}}%
\only<6->{\fcPatchInScene{[0 0 0]}{[1 0.5 0]}{[0 0 2]}}%
\only<7->{%
\fcTriangleInScene{[0 0 2]}{[1 0.5 0]}{[0 1 0]}%
\fcTriangleInScene{[1 1.5 -2]}{[1 0.5 0]}{[0 1 0]}%
\fcLineIIIdInScene{[0 1 0]}{[1 0.5 0]}%
}%
\fcAxesIIIdInScene{3}{3}{2.3}%
\fcFinishIIIdScene%
\uncover<8->{
\fcDotIIId{[0 0 0]}
\fcDotIIId{[0 1 0]}
\fcDotIIId{[1 0.5 0]}
\fcDotIIId{[0 0 2]}
}
\fcPutIIId[l]{[3 0 0]}{$x$}
\fcPutIIId[l]{[0 3 0]}{$y$}
\fcPutIIId[b]{[0 0 2.4]}{$z$}
\end{pspicture}
\column{0.7\textwidth}
Compute the volume of the \alert<3-7>{region $\cR$ bounded by} \alert<7>{$x+2y+z=2$}, \alert<6>{$x=2y$}, \alert<5>{$x=0$}, \alert<4>{$z=0$}.
\uncover<2->{
\[
\text{vol}(\cR) = \iiint_{\cR} 1\cdot \diff V\; .
\]
}
\uncover<3->{\alert<3-7>{$\cR$ is}} \uncover<3-7>{\alert<3-7>{\textbf{?}}} \uncover<8->{\alert<8>{a tetrahedron with vertices at $(0,0,0)$, $(0,1,0)$, $(0,0,2)$, and $\left(1, \frac{1}{2}, 0\right)$}.}
\end{columns}
\only<1-13>{ 
\uncover<9->{Project $\cR$ onto the $z-$axis to get segment from $z=0$ to $z=2$.} \fcQuestion{10}{Fix a value for $z$ to get slice $S_z$ equal to} \fcAnswer{11}{the triangle $(0,0,z)$, $(0,1-\frac{z}{2},z)$, $(1-z, \frac{1}{2}-\frac{z}{2},z)$}
}
\uncover<12->{
\alert<13,14>{
$\iiint_{\cR} 1\cdot \diff V = \int_{z=0}^{z=2} \left( \iint_{S_z} 1\cdot \diff x \diff y \right)  \diff z
$
}
}

\uncover<15->{
Project $S_z$ onto the $x-$axis to get segment from $x=0$ to $x=1-z$.} \uncover<16->{ Fix $x$ in that range. Then the vertical slice is a segment from $y=\frac{x}{2}$ to $y=1-\frac{z}{2} - \frac{x}{2}$.}
\uncover<17->{
\[
\text{vol}(\cR) = \iiint_{\cR} 1\cdot \diff V = \int_{z=0}^{z=2} \left(\int_{x=0}^{x=1-z} \left( \int_{y=\frac{x}{2}}^{y=1-\frac{z}{2}-\frac{x}{2}} 1 \cdot \diff y \right)  \diff x\right) \diff z .
\]
}
\end{example}
\end{frame} 
\begin{frame}
\underline{Decomposition by rods}:
\begin{itemize}
  \item Projection of the region onto the $xy-$plane:
  \begin{itemize}
    \item triangle $D$ with vertices $(0,0,0)$, $(0,1,0)$, and $(1,\frac{1}{2},0)$
  \end{itemize}
  \item For a generic location $(x,y)$ within this region, the vertical rod is
  \begin{itemize}
    \item segment with endpoints $z=0$ and $z=2-x-2y$
  \end{itemize}
 %
$$\iiint_{\cR} 1\cdot dV = \iint_D \left( \int_{z=0}^{z=2-x-2y} 1\cdot dz \right) \, dxdy = \iint_D (2-x-2y) \; dxdy$$
%
\item Projection of $D$ over the $x-$axis:
\begin{itemize}
  \item segment from $x=0$ to $x=1$;
\end{itemize}
%
\item For generic $x$ in that range, the slice is
\begin{itemize}
  \item segment from $y=\frac{x}{2}$ to $y=1- \frac{x}{2}$
\end{itemize}
%
$$\iint_{D} f(x,y) \, dxdy = \int_{x=0}^{x=1-z} \left( \int_{y=x/2}^{y=1-x/2} f(x,y)\, dy \right) \, dx$$
%
$$\text{vol}(\cR) = \iiint_{\cR} 1\cdot dV = \int_{x=0}^{x=1-z} \left( \int_{y=x/2}^{y=1-x/2} \left( \int_{z=0}^{z=2-x-2y} 1\cdot dz \right)dy \right) \, dx \; .$$
\end{itemize}

\end{frame}  
}
\end{document}
