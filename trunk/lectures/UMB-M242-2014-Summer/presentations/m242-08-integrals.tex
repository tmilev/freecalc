\begin{frame}
  \frametitle{A Cheaper Census}

  To find the total population over a region $\mathcal{R}$:\pause
  \begin{itemize}
    \item Decomposition into smaller regions, $\mathcal{D} = (D_k)_k$\pause
    %
\begin{itemize}
  \item states;
  \item counties;
  \item finer division.
\end{itemize}
%
\pause
$$\text{population}(\mathcal{R}) = \sum_k \text{population}(D_k) = \sum_k \text{density}(D_k) \cdot \text{area}(D_k)$$

\item \pause Density of $D_k$: \pause sampling
\begin{itemize}
  \item $\mathcal{P} = (P_k)_k$, collection of sample points, one for each subregion;
  \item $\text{density}(P_k)$: density in a small region around $P_k$;
  \item $\text{density}(D_k) \simeq \text{density}(P_k)$
\end{itemize}
\end{itemize}
\pause
$$\text{population}(\mathcal{R}) = \sum \text{population}(D_k) \simeq \sum \text{density}(P_k) \, \text{area}(D_k)\; .$$
%
\pause
Cheaper census, \pause but illegal for US Census ...

\end{frame}

\begin{frame}
  \frametitle{Double Integrals}

  Let $\mathcal{R}$ be a compact (closed, bounded) region in the plane, and let $f \colon \mathcal{R} \to \mathbb{R}$ be a function on $\mathcal{R}$.
  \pause
  Let $\mathcal{D} = (D_k)_k$ be a finite covering of $\mathcal{R}$
%
\begin{itemize}
  \item $D_k$ compact set, for all $k$;
  %
  \item the boundary of $D_k$ is a collection of smooth curves, for all $k$;
  %
  \item two regions $D_i$ and $D_j$ may overlap only on their boundaries.
\end{itemize}

\pause Let $\mathcal{P} = (P_k)_k$ be a collection of sample points, with $P_k \in D_k$ for all $k$.

\pause
The \emph{Riemann sum} defined by such data is
%
$$\Sigma_{f,\mathcal{D}, \mathcal{P}} = \sum_k f(P_k) \; \text{area}(D_k)\; .$$

\pause \underline{Definition}: If the limit
%
$$\lim_{\text{maxdiam}(\mathcal{D}) \to 0} \sum_k f(P_k) \; \text{area}(D_k)$$
%
exists and is finite, then its value is called the \emph{double integral of $f$ over $\mathcal{R}$ with respect to area}, and is denoted by
%
$$\int\!\!\!\int_{\mathcal{R}} f(P) \, dA \; .$$
\end{frame}


 \begin{frame}
 \frametitle{Examples}

\begin{itemize}
  %
  \item \pause The total population over a region $\mathcal{R}$ is:
%
$$\text{population}(\mathcal{R}) = \int\!\!\!\int_{\mathcal{R}} \text{density}(P) \, dA \simeq \sum_k \text{density}(P_k) \, \text{area}(D_k) \; .$$

  \item \pause Mass is the double integral of density with respect to area:
  %
  $$\text{mass}(\mathcal{R}) = \int\!\!\!\int_{\mathcal{R}} \text{density}(P) \, dA\; .$$

    \item \pause Volume under the graph of $h\colon \mathcal{R} \to [0,\infty)$
%
$$\text{Volume} = \int\!\!\!\int_{\mathcal{R}} h(P) \, dA \; .$$

    \item \pause Area of a region:
    %
    $$\text{Area}(\mathcal{R}) = \int\!\!\!\int_{\mathcal{R}} 1 \; dA$$
\end{itemize}
\end{frame}

\begin{frame}
  \frametitle{Properties}
    $$\int\!\!\int_{\mathcal{R}} f(P) \; dA = \lim_{\text{maxdiam}(\mathcal{D}) \to 0} \sum_k f(P_k) \; \text{area}(D_k)$$
%

 \begin{itemize}
    \item \pause If $f$ is bounded and continuous, except maybe on a finite number of smooth curves,
  then the limit exists and is finite.

  \item \pause Linearity with respect to function
  %
  $$\int\!\!\!\int_{\mathcal{R}} [\lambda f(P) +\mu g(P)] \, dA = \lambda \, \int\!\!\!\int_{\mathcal{R}} f(P) \, dA +\mu \, \int\!\!\!\int_{\mathcal{R}} g(P) \, dA \; .$$
  %
  \item \pause Additivity with respect to domain: If $\mathcal{R}_1$ and $\mathcal{R}_2$ intersect only along the boundaries, then
      %
      $$\int\!\!\!\int_{\mathcal{R}_1\cup \mathcal{R}_2} f(P)\, dA = \int\!\!\!\int_{\mathcal{R}_1} f(P)\, dA + \int\!\!\!\int_{\mathcal{R}_2} f(P)\, dA $$
  %
  \item \pause Monotonicity property: If $m \leqslant f(P) \leqslant M$ for all $P$ in $\mathcal{R}$, then
      %
      $$m \, \text{area}(\mathcal{R}) \leqslant \int\!\!\!\int_{\mathcal{R}} f(P)\, dA \leqslant M \, \text{area} (\mathcal{R})\; .$$
\end{itemize}
\end{frame}

\begin{frame}
  \frametitle{Applications}

  \begin{itemize}
    \item \pause Average value of $f$ on $\mathcal{R}$: \pause constant value that would produce the same accumulation.\pause
%
$$\int\!\!\!\int_{\mathcal{R}} f(P) \, dA =
\int\!\!\!\int_{\mathcal{R}} (\text{average value}) \, dA =  (\text{average value})
\cdot \text{area}(\mathcal{R}) \; \Longrightarrow $$
%
$$\text{average value of }f \text{ on } \mathcal{R} = \frac{1}{\text{area}(\mathcal{R})} \int\!\!\!\int_{\mathcal{R}} f(P) \, dA\; .$$
%
\item \pause \underline{Mean Value Theorem}:

If $f$ is continuous on $\mathcal{R}$, then there exists $P_0$ in $\mathcal{R}$ such that
%
$$f(P_0) = \frac{1}{\text{area}(\mathcal{R})} \int\!\!\!\int_{\mathcal R} f(Q) \,dA$$

\item \pause \underline{Fundamental Theorem of Calculus} (sort of ..):

If $f$ is continuous around $P$, then
%
$$\lim_{D \to \{ P \} } \frac{1}{\text{area}(D)} \int\!\!\!\int_D f(Q) dA = f(P)$$

  \end{itemize}
\end{frame}


\begin{frame}
  \frametitle{Vectorial Integrals}

    $$\int\!\!\int_{\mathcal{R}} \textbf{F}(P) \; dA = \lim_{\text{maxdiam}(\mathcal{D}) \to 0} \sum_k \textbf{F}(P_k) \; \text{area}(D_k)$$
%
    The definition can be extended to functions with vectorial output.

\underline{Example}: Electric force on a lamina

    \begin{itemize}
      \item Charge $Q$, at a fixed point
      \item Charge $q$, uniformly distributed on a planar lamina $\mathcal{R}$
      \item Total force on $Q$?
    \end{itemize}
    %
    $$dq = (\text{density of charge}) \, dA = \frac{q}{A(\mathcal{R})}\, dA$$
    %
    $$d\textbf{F} = \frac{\epsilon Q dq}{|\textbf{r}|^3} \; \textbf{r}  =
    \left( \epsilon \frac{Q q}{A(\mathcal{R}) |\textbf{r}|^3} \, \textbf{r}\right)\, dA$$
    %
    $$\textbf{F} = \int\!\!\!\int_{\mathcal{R}} d\textbf{F} =
    \int\!\!\!\int_{\mathcal{R}} \left( \epsilon \frac{Q q}{A(\mathcal{R}) |\textbf{r}|^3} \, \textbf{r}\right)\, dA$$

    \pause Constant Multiple Rule:
    %
    $$\int\!\!\int_{\mathcal{R}} c \textbf{F} \; dA =
    c \int\!\!\int_{\mathcal{R}} \textbf{F} \; dA$$
\end{frame}

\begin{frame}
  \frametitle{Midpoint Rule}

  $\mathcal{R}$: rectangle \pause $\Longrightarrow$ rectangular coordinate system.

  $\mathcal{R} = [a,b] \times [c,d]$, element of area $dA = dx\, dy$,

%
$$\int\!\!\!\int_{\mathcal{R}} f(P) \, dA = \int\!\!\!\int_{[a,b] \times [c,d]} f(x,y) \; dx\, dy\; .$$

\pause If the double integral exists, then we approximate its value by a Riemann sum for a fine enough partition and a consistent choice of sample points.
%
\begin{itemize}
  \item \pause Divide each side into $n$ equal pieces, and obtain a covering of $\mathcal{R}$ by $n^2$ smaller rectangles, of sides $\Delta x = \frac{b-a}{n}$ and $\Delta y = \frac{d-c}{n}$.
  %
  \item \pause Choose the sample points as the midpoints of the smaller rectangles: take $P_{ij}$ to be the midpoint of the rectangle $D_{ij}$ on the $i^{th}$ row and $j^{th}$ column (counting from bottom left corner).
\end{itemize}
%
\pause
$$\int\!\!\!\int_{[a,b] \times [c,d]} f(x,y) \; dx\, dy\;  = \lim_{n \to \infty} \sum_{1\leqslant i,j \leqslant n} f(P_{ij}) \, \text{area}(D_{ij})\; ,$$
%
$$\int\!\!\!\int_{[a,b] \times [c,d]} f(x,y) \; dx\, dy\;  \simeq \sum_{1\leqslant i,j \leqslant n} f(P_{ij}) \, \Delta x \, \Delta y\; .$$
%
\end{frame}

\begin{frame}
  \frametitle{Example}

  Use the Midpoint Rule to approximate
%
$$\int\!\!\!\int_{[0,4]\times [0,2]} x^2y\; dxdy\; ,$$
%
with each side divided into $n=2$ pieces.

\pause
The smaller rectangles have size $\frac{4-0}{2} \times \frac{2-0}{2} = 2\times 1$ and area $2$.

\pause
The midpoints are
%
$$P_{11} = (1,1/2), \quad P_{12} = (1,3/2), \quad P_{21} = (3,1/2),  \quad P_{22} = (3,3/2)\; .$$
%
\pause
\begin{align*}
  \int\!\!\!&\int_{[0,4]\times [0,2]} x^2y\; dxdy  \simeq \pause \\
  & \simeq f(1,1/2) \cdot 2 +  f(3,1/2) \cdot 2 + f(1,3/2) \cdot 2  + f(3,3/2) \cdot 2 = \\
  %
  & = \left(1\cdot \frac{1}{2}\cdot 2 + 9 \cdot \frac{1}{2} \cdot 2\right) + \left(1\cdot \frac{3}{2} \cdot 2 + 9 \cdot \frac{3}{2} \cdot 2\right) = \\
  &= 1+9+3+27 = 40\; .
\end{align*}
\end{frame}

\begin{frame}
  \frametitle{Iterated Integrals}
%
\begin{align*}
  \int\!\!\!\int_{[a,b] \times [c,d]} f(x,y) \; dx\, dy \simeq \sum_{1\leqslant i,j \leqslant n} f(P_{ij}) \, \Delta x \, \Delta y = \\
  = \sum_{1\leqslant i,j \leqslant n} f(x_i,y_j) \, \Delta x\, \Delta y = \sum_{j=1}^n \left( \sum_{i=1}^n f(x_i,y_j) \Delta x \right)\, \Delta y\; .
\end{align*}

\pause
The sum with index $i$ is a Riemann sum for computing the integral
%
$$g(y_j) = \int_{x=a}^{x=b} f(x,y_j) \; dx$$
%
\pause
$$\sum_{j=1}^n\left( \sum_{i=1}^n f(x_i,y_j) \Delta x \right) \Delta y \simeq \sum_{j=1}^n g(y_j) \Delta y \simeq \int_{y=c}^{y=d} g(y) \; dy$$
%
\pause
$$\int\!\!\!\int_{[a,b] \times [c,d]} f(x,y) \; dx\, dy = \int_{y=c}^{y=d} g(y) \; dy = \int_{y=c}^{y=d} \left( \int_{x=a}^{x=b} f(x,y) \; dx \right) \; dy$$
%
\end{frame}

\begin{frame}
    \frametitle{Fubini's Theorem}
Let $f \colon [a,b]\times[c,d] \to \mathbb{R}$ be a function that is
\begin{itemize}
  \item bounded
  \item continuous, except maybe on a finite number of smooth curves.
\end{itemize}
If the iterated integrals exist, then
%
\begin{align*}
  \int&\!\!\!\int_{[a,b]\times [c,d]}\!\!\!\! f(x,y) \; dxdy = \int_{\textcolor[rgb]{1.00,0.00,0.00}{y=c}}^{\textcolor[rgb]{0.98,0.00,0.00}{y=d}} \left( \int_{\textcolor[rgb]{0.00,0.00,1.00}{x=a}}^{\textcolor[rgb]{0.00,0.00,1.00}{x=b}} f(x,y) \; \textcolor[rgb]{0.00,0.00,1.00}{dx} \right) \; \textcolor[rgb]{0.98,0.00,0.00}{dy} = \\
  & = \int_{\textcolor[rgb]{0.00,0.00,1.00}{x=a}}^{\textcolor[rgb]{0.00,0.00,1.00}{x=b}} \left( \int_{\textcolor[rgb]{0.98,0.00,0.00}{y=c}}^{\textcolor[rgb]{0.98,0.00,0.00}{y=d}} f(x,y) \; \textcolor[rgb]{0.98,0.00,0.00}{dy} \right) \; \textcolor[rgb]{0.00,0.00,1.00}{dx}\; .
\end{align*}

The iterated integrals exists if $f$ is continuous.
\end{frame}

\begin{frame}
  \frametitle{Example}

Use iterated integrals to compute the following integral:
%
$$\int\!\!\!\int_{[1,2]\times [2,3]}\!\!\!\!\!\! (2x+3y^2) \; dxdy$$
%
\begin{itemize}
  \item \pause For $(x,y)$ in $[1,2]\times [2,3]$, $\textcolor[rgb]{0.98,0.00,0.00}{y}$ takes values \pause between $\textcolor[rgb]{0.98,0.00,0.00}{c=2}$ and $\textcolor[rgb]{0.98,0.00,0.00}{d=3}$.
  \item \pause For a fixed value $\textcolor[rgb]{0.98,0.00,0.00}{y=y_0}$, $\textcolor[rgb]{0.00,0.00,1.00}{x}$ takes values \pause between $\textcolor[rgb]{0.00,0.00,1.00}{a=1}$ and $\textcolor[rgb]{0.00,0.00,1.00}{b=2}$.
\end{itemize}
\pause
Then the double integral can be set up as the following iterated integral:
%
$$\int\!\!\!\int_{[1,2]\times [2,3]}\!\!\!\!\!\! (2x+3y^2) \; dxdy = \int_{\textcolor[rgb]{0.98,0.00,0.00}{y=2}}^{\textcolor[rgb]{0.98,0.00,0.00}{y=3}} \left( \int_{\textcolor[rgb]{0.00,0.00,1.00}{x=1}}^{\textcolor[rgb]{0.00,0.00,1.00}{x=2}} (2x+3y^2) \; \textcolor[rgb]{0.00,0.00,1.00}{dx} \right) \; \textcolor[rgb]{0.98,0.00,0.00}{dy} $$
\pause
Integral with respect to $x$ \pause $\Longrightarrow$ $y$ is a constant:
%
$$g(y) = \int_{x=1}^{x=2} (2x+3y^2) \; dx = \left. (x^2+3y^2x) \right|_{x=1}^{x=2} = (4+6y^2)-(1+3y^2) = 3+3y^2\; .$$
%
\pause Continuing the computation we get\pause
%
$$\int\!\!\!\int_{[1,2]\times [2,3]}\!\!\!\!\!\! (2x+3y^2) \; dxdy = \int_{y=2}^{y=3} (3+3y^2)\; dy = \left. (3y+y^3) \right|_{y=2}^{y=3} = 36-14=22\; .$$
\end{frame}

\begin{frame}
  \frametitle{More General Regions}

  What makes the iterated integral method work over rectangular regions?\pause

\begin{quote}
  For all values of one variable, all slices with respect to that variable are intervals in the other variable.
\end{quote}
\pause
If the variable is $y$, then:
\begin{itemize}
  \item we can integrate over each slice with respect to $x$,
  \item obtaining a function that depends only on the location of the slice,
  \item given by the $y-$value,
  \item then we integrate the result again with respect to $y$.
\end{itemize}
\pause
We can apply the same procedure if the slices are intervals, with endpoints depending continuously on the location of the slice.\pause
%
\begin{itemize}
  \item Regions of type I: vertical slices are segments.
  \item Regions of type II: horizontal slices are segments.
\end{itemize}
\end{frame}

\begin{frame}
  \frametitle{Strategy: Type I Regions}
%
\begin{itemize}
  \item \pause Identify the leftmost point, $(a,*)$, and the rightmost point, $(b,*)$.
  \item \pause Draw a generic vertical slice at some value $x$ between $a$ and $b$.
  \item \pause Find the lowest point on that slice, $(x,g_1(x))$ and the highest point, $(x,g_2(x))$.
\end{itemize}
\pause
The region is the region bounded by:
 \begin{itemize}
   \item \pause vertical lines $\textcolor[rgb]{0.00,0.00,1.00}{x=a}$ and $\textcolor[rgb]{0.00,0.00,1.00}{x=b}$;
   \item \pause the graphs of $\textcolor[rgb]{0.98,0.00,0.00}{y=g_1(x)}$ and $\textcolor[rgb]{0.98,0.00,0.00}{y=g_2(x)}$, with $g_1,g_2 \colon [a,b] \to \mathbb{R}$:
 \end{itemize}
%
\pause
$$\mathcal{R} = \{(x,y) \; | \; \textcolor[rgb]{0.00,0.00,1.00}{a \leqslant x \leqslant b}\; , \; \textcolor[rgb]{0.98,0.00,0.00}{g_1(x) \leqslant y \leqslant g_2(x)} \} \; .$$
%
\pause
$$\int\!\!\!\int_{\mathcal{R}} f(x,y) \; dxdy = \int_{\textcolor[rgb]{0.00,0.00,1.00}{x=a}}^{\textcolor[rgb]{0.00,0.00,1.00}{x=b}} \left( \int_{\textcolor[rgb]{0.98,0.00,0.00}{y=g_1(x)}}^{\textcolor[rgb]{0.98,0.00,0.00}{y=g_2(x)}} f(\textcolor[rgb]{0.00,0.00,1.00}{x},\textcolor[rgb]{0.98,0.00,0.00}{y}) \; \textcolor[rgb]{0.98,0.00,0.00}{dy}\right) \; \textcolor[rgb]{0.00,0.00,1.00}{dx}$$
\end{frame}

\begin{frame}
  \frametitle{Strategy: Type II Regions}
%
\begin{itemize}
  \item Identify the lowest point, $(*,c)$, and the highest point, $(*,d)$.
  \item Draw a generic horizontal slice at some value $y$ between $c$ and $d$.
  \item Find the leftmost point on that slice, $(h_1(y),y)$ and the rightmost point, $(h_2(y),y)$.
\end{itemize}

The region is bounded by:
 \begin{itemize}
   \item horizontal lines $\textcolor[rgb]{0.98,0.00,0.00}{y=c}$ and $\textcolor[rgb]{0.98,0.00,0.00}{y=d}$
   \item graphs of $\textcolor[rgb]{0.00,0.00,1.00}{x=h_1(y)}$ and $\textcolor[rgb]{0.00,0.00,1.00}{x=h_2(y)}$, with  $h_1,h_2 \colon [c,d] \to \mathbb{R}$:
 \end{itemize}
%
$$\mathcal{R} = \{(x,y) \; | \; \textcolor[rgb]{0.98,0.00,0.00}{c \leqslant y \leqslant d}\; , \; \textcolor[rgb]{0.00,0.00,1.00}{h_1(y) \leqslant x \leqslant h_2(y)} \} \; .$$
%
$$\int\!\!\!\int_{\mathcal{R}} f(x,y) \; dxdy = \int_{\textcolor[rgb]{0.98,0.00,0.00}{y=c}}^{\textcolor[rgb]{0.98,0.00,0.00}{y=d}} \left( \int_{\textcolor[rgb]{0.00,0.00,1.00}{x=h_1(y)}}^{\textcolor[rgb]{0.00,0.00,1.00}{x=h_2(y)}} f(\textcolor[rgb]{0.00,0.00,1.00}{x},\textcolor[rgb]{0.98,0.00,0.00}{y}) \; \textcolor[rgb]{0.00,0.00,1.00}{dx}\right) \; \textcolor[rgb]{0.98,0.00,0.00}{dy}$$
\end{frame}

\begin{frame}
  \frametitle{Examples}

  \begin{itemize}
    \item $\mathcal{R}_1$: region bounded by $y=2x$ and $y=x^2$. Compute
%
$$\int\!\!\!\int_{\mathcal{R}_1} (x^2+y^2) \; dxdy$$

\item $\mathcal{R}_2$: region bounded by $y=x-1$ and $y^2=2x+6$. Compute
%
$$ \int\!\!\!\int_{\mathcal{R}_2} xy \; dxdy$$

\item $\mathcal{R}$: region bounded by
$y=(x+1)^2$, $x=y-y^3$, the line $x=-1$ and the line $y=-1$. Set-up iterated integrals for
%
$$\int\!\!\!\int_{\mathcal{R}} f(x,y) \, dA \; .$$

\item How do we compute
%
$$\int_0^1 \int_{3y}^3 e^{x^2} \, dx \; dy \; \qquad , \qquad  \int\!\!\!\int_{[0,\infty) \times [0,\infty)} e^{-x-y} \, dxdy $$

  \end{itemize}
\end{frame}

\begin{frame}
  \frametitle{Density to Mass}
  \underline{Question}: If we know the density at every point, can we find the mass?

  \underline{Answer}: Yes.

\begin{itemize}
  \item Partition the region $\cR$ into regions with small diameter; let $\cD=\{D_1,\ldots,D_N\}$ be such a partition, and let
      %
      $$\text{maxdiam}(\cD) = \max_{k} \text{diam}(D_k)\; .$$
      %
  \item For each region $D_k$, choose a sample point $P_k$ inside $D_k$:
      %
      $$\text{mass}(D_k) \simeq \rho(P_k) \text{vol}(D_k)\; .$$
      %
      \item Mass is approximated by the sum of the masses of the subregions:
      %
      $$\text{mass}(\cR) \simeq \sum_{k=1}^{N} \rho(P_k)\text{vol}(D_k)\; .$$
      %
      \item Take partitions with diameter closer and closer to 0:
      %
      $$\text{mass}(\cR) = \lim_{\text{maxdiam}(\cD) \to 0}  \sum_{k=1}^{N} \rho(P_k)\text{vol}(D_k)\; .$$
\end{itemize}
\end{frame}

\begin{frame}
  \frametitle{Triple Integrals}

  $f$ defined on region $\cR$, \emph{scalar} or \emph{vectorial} function.

  \begin{definition}
  If the limit
%
$$\lim_{\text{maxdiam}(\cD) \to 0}  \sum_{k=1}^{N} f(P_k)\text{vol}(D_k)$$
%
exists and is finite, its value is called
\begin{quote}
  \emph{the integral of $f$ on $\cR$ with respect to volume}
\end{quote}
 and is denoted by
%
$$\int\!\!\!\int\!\!\!\int_{\cR} f(P) \; dV\; .$$
  \end{definition}

\begin{itemize}
  \item If $f$ is a scalar function, then the value of the integral is a scalar;
  \item If $f$ is a vectorial function, then the value of the integral is a vector.
\end{itemize}

The limit does not always exist. It exists if the function $f$ is continuous.

\end{frame}


\begin{frame}
\frametitle{Examples}

\begin{itemize}
  \item Volume of a region:
  %
  $$\text{vol}(\cR) = \int\!\!\!\!\int\!\!\!\!\int_{\cR} 1 \cdot dV$$
  %
  \item Mass of a body:
  %
  $$\text{mass}(\cR) = \int\!\!\!\!\int\!\!\!\!\int_{\cR} \text{density}(P) \cdot dV\; .$$
  %
  \item The average value of a function $f$ with respect to volume:
%
$$\text{average value of }f = \frac{1}{\text{vol}(\cR)} \int\!\!\!\!\int\!\!\!\!\int_{\cR} f(P) \cdot dV\; .$$
  %
  \item The average value of a function $f$ with respect to mass distribution:
%
$$\text{average value of }f = \frac{1}{\text{m}(\cR)} \int\!\!\!\!\int\!\!\!\!\int_{\cR} f(P)\,  dm = \frac{1}{\text{m}(\cR)} \int\!\!\!\!\int\!\!\!\!\int_{\cR} f(P)\rho(P)\, dV\; .$$
\end{itemize}

\end{frame}

\begin{frame}
  \frametitle{Iterated Integrals}

  There are two ways of slicing a 3D region $\cR$:
  \begin{itemize}
    \item By slices:
    \begin{itemize}
      \item Project the body on an axis;
      \item Look at 2D slices perpendicular to that axis (CT-scan)
    \end{itemize}
      %
      $$\iiint_{\cR}f(P) \, dV = \int_{\text{location of slice}} \left(\iint_{\text{slice}} f(P) \, dA\right) \, dh$$
      %
      \item By rods:
      \begin{itemize}
        \item Project the body on a plane;
        \item Look at 1D slices perpendicular to that plane (rods)
      \end{itemize}
      %
      $$\iiint_{\cR}f(P) \, dV = \iint_{\text{location of rod}} \left(\int_{\text{rod}} f(P) \, dh\right) \, dA$$
      %
  \end{itemize}
  %
  \begin{itemize}
    \item Reduce the computation to computations of
    \begin{itemize}
      \item  a single integral and
      \item a double integral
    \end{itemize}
  \end{itemize}
\end{frame}

\begin{frame}
  \frametitle{Example: Moment}
  
  \begin{itemize}
    \item Moment of a rectangular box with sides $2a$, $2b$, and $2c$;
    \item Constant density $\rho$;
    \item With respect to an axis through the center, perpendicular to a face;
    \item Coordinate system $Oxyz$, $L=Oz$;
    $$I_L = \int\!\!\!\int\!\!\!\int_{\cR} \rho\, \text{dist}^2(P,L)\, dV = \int\!\!\!\int\!\!\!\int_{\cR} \rho (x^2+y^2)\, dxdydz\; .$$
    \item \underline{Decomposition by slices}:
    \begin{itemize}
    \item Projection of $\cR$ onto the $z-$axis: segment from $z=-c$ to $z=c$;
%
$$\iiint_{\cR} \rho (x^2+y^2)\, dxdydz = \int_{z=-c}^{z=c} \left( \iint_{S_z} \rho (x^2+y^2)\, dxdy \right) \, dz$$
%    
    \item For a generic location $z$, the slice $S_z$: $-a \leqslant x \leqslant a$, $-b \leqslant y \leqslant b$;
%
$$I_L = \int_{z=-c}^{z=c} \left(\int_{x=-a}^{x=a} \left( \int_{y=-b}^{y=b} \rho (x^2+y^2) \, dy \right)  dx\right)  dz\; = \frac{m(a^2+b^2)}{3} \; .$$
%

\end{itemize}
  \end{itemize}
\end{frame}

\begin{frame}
  \frametitle{Example: Volume}

Volume of region $\cR$ bounded by  $x+2y+z=2$, $x=2y$, $x=0$, and $z=0$:
%
$$\text{vol}(\cR) = \iiint_{\cR} 1\cdot dV\; .$$
%
$\cR$: tetrahedron with vertices at $(0,0,0)$, $(0,1,0)$, $(0,0,2)$, and $(1, \frac{1}{2}, 0)$.

\underline{Decomposition by slices}:
\begin{itemize}
  \item Projection of $\cR$ onto the $z-$axis: the segment from $z=0$ to $z=2$;
  \item For a generic location $z$, the slice $S_z$:
   \begin{itemize}
     \item triangle $(0,0,z)$, $(0,1-\frac{z}{2},z)$, $(1-z, \frac{1}{2}-\frac{z}{2},z)$
   \end{itemize}
%
$$\iiint_{\cR} 1\cdot dV = \int_{z=0}^{z=2} \left( \iint_{S_z} 1\cdot dxdy \right) \, dz$$
%
\item Projection of $S_z$ onto the $x-$axis: segment from $x=0$ to $x=1-z$.
\item For a generic $x$ in that range, the vertical slice is
\begin{itemize}
  \item segment from $y=\frac{x}{2}$ to $y=1-\frac{z}{2} - \frac{x}{2}$
\end{itemize}
%
$$\text{vol}(\cR) = \iiint_{\cR} 1\cdot dV = \int_{z=0}^{z=2} \left(\int_{x=0}^{x=1-z} \left( \int_{y=x/2}^{y=1-z/2-x/2} 1 \cdot dy \right) \, dx\right) \, dz\; .$$
%
\end{itemize}
\end{frame}

\begin{frame}
\underline{Decomposition by rods}:
\begin{itemize}
  \item Projection of the region onto the $xy-$plane:
  \begin{itemize}
    \item triangle $D$ with vertices $(0,0,0)$, $(0,1,0)$, and $(1,\frac{1}{2},0)$
  \end{itemize}
  \item For a generic location $(x,y)$ within this region, the vertical rod is
  \begin{itemize}
    \item segment with endpoints $z=0$ and $z=2-x-2y$
  \end{itemize}
 %
$$\iiint_{\cR} 1\cdot dV = \iint_D \left( \int_{z=0}^{z=2-x-2y} 1\cdot dz \right) \, dxdy = \iint_D (2-x-2y) \; dxdy$$
%
\item Projection of $D$ over the $x-$axis:
\begin{itemize}
  \item segment from $x=0$ to $x=1$;
\end{itemize}
%
\item For generic $x$ in that range, the slice is
\begin{itemize}
  \item segment from $y=\frac{x}{2}$ to $y=1- \frac{x}{2}$
\end{itemize}
%
$$\iint_{D} f(x,y) \, dxdy = \int_{x=0}^{x=1-z} \left( \int_{y=x/2}^{y=1-x/2} f(x,y)\, dy \right) \, dx$$
%
$$\text{vol}(\cR) = \iiint_{\cR} 1\cdot dV = \int_{x=0}^{x=1-z} \left( \int_{y=x/2}^{y=1-x/2} \left( \int_{z=0}^{z=2-x-2y} 1\cdot dz \right)dy \right) \, dx \; .$$
\end{itemize}

\end{frame} 