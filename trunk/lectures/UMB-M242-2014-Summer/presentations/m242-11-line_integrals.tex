\begin{frame}
  \frametitle{Integrals}

  We have so far studied accumulations (integrals) over:
  \begin{itemize}
    \item intervals on a line;
    \item planar regions in the plane;
    \item solid regions in space.
  \end{itemize}
  Regions that have the same dimension as their ambient space.

  \bigskip
  What if the region has a lower dimension then the ambient space?
  \begin{itemize}
    \item curve (1D region) embedded in a plane (2D) or in space (3D)
    \item surface (2D region) embedded in space (3D).
  \end{itemize}
\end{frame}

\begin{frame}
  \frametitle{Example}
  Accumulations along a curve \textcolor[rgb]{0.98,0.00,0.00}{with respect to arclength}

  \begin{itemize}
    \item $C$ be a continuous curve, with both endpoints included;
    \item $C$ has a piecewise smooth regular parametrization $\textbf{r}\colon [a,b]\to C$;
    \item Element of arclength: $ds= |\textbf{r}'(t)|\, dt$.
  \end{itemize}

\pause   Example:
  \begin{itemize}
    \item Mass from linear density:
    \begin{itemize}
      \item \pause pointwise linear density: $\rho(P)$;
      \item \pause infinitesimal element of mass (scalar): $dm = \pause \rho\, ds$;
      \item \pause the mass $m$ of the wire: accumulation of $dm$.
    \end{itemize}
  \end{itemize}
\end{frame}

\begin{frame}
  \frametitle{Riemann Sums}
  Given:
  \begin{itemize}
    \item piecewise smooth curve (with endpoints included) $C$ in space;
    \item scalar  or vectorial function $f$ defined on $C$;
  \end{itemize}

\pause  Accumulation of $f$ on $C$ with respect to arclength:

  \begin{itemize}
    \item \pause Divide $C$ into pieces $D_1$, \ldots, $D_N$ with non-overlapping interiors;
    \item \pause Pick a sample point $P_k$ in $D_k$. Assume that $f \equiv f(P_k)$ is constant on $D_k$;
    \item \pause The accumulation on $D_k$ is approximated by
    %
    $$f(P_k) \cdot \text{length}(D_k)$$
    \item \pause The total accumulation is \textcolor[rgb]{0.98,0.00,0.00}{approximated} by the Riemann sum
  %
  $$\sum_{k=1}^N f(P_k) \cdot \text{length}(D_k)$$
  %
  \end{itemize}
\end{frame}

\begin{frame}
    \frametitle{Line Integrals}

If the limit
%
$$\lim_{\text{maxlength} \to 0} \sum_{k=1}^N f(P_k) \cdot \text{length}(D_k)$$
%
exists and is finite, then:

\begin{itemize}
   \item \pause the \emph{line integral of $f$ on $C$ with respect to arclength} exists;
   %
   \item \pause we denote this line integral by $\int_C f(P) \, ds$ \pause and assign it the value
   %
   $$\int_C f(P) \, ds =\lim_{\text{maxlength} \to 0} \sum_{k=1}^N f(P_k) \cdot \text{length}(D_k)\; .$$
 \end{itemize}
%
$$m = \int_C \rho \, ds$$
%

\pause The line integral always exists if
\begin{itemize}
  \item \pause $f$ is a continuous functions, or
  \item \pause $f$ is bounded and continuous except at a finite number of points.
\end{itemize}

\end{frame}

\begin{frame}
  \frametitle{Parametrizations and Computations}


$\textbf{r} \colon [a,b] \to C$: regular, piecewise smooth parametrization of $C$
%
$$ds = | \textbf{r}'(t) | dt$$
%
$$\int_{\textcolor[rgb]{0.98,0.00,0.00}{C}} \textcolor[rgb]{0.00,0.00,1.00}{f(P)} \, \textcolor[rgb]{0.00,0.59,0.00}{ds} = \int_{\textcolor[rgb]{0.98,0.00,0.00}{a}}^{\textcolor[rgb]{0.98,0.00,0.00}{b}} \textcolor[rgb]{0.00,0.00,1.00}{f(\textbf{r}(t))} \textcolor[rgb]{0.00,0.59,0.00}{|\textbf{r}'(t)| \, dt}\; .$$
%

\pause The result is independent of the parametrization of $C$ we use.

\pause In rectangular coordinates:
%
$$\textbf{r}\colon [a,b] \to \RR^2\quad , \quad \textbf{r}(t) = \langle x(t), y(t) \rangle$$
%
$$ds = |\textbf{r}'(t)| dt = \sqrt{(x'(t))^2 + (y'(t))^2} \, dt $$
%
$$
  \int_{\textcolor[rgb]{0.98,0.00,0.00}{C}} \textcolor[rgb]{0.00,0.00,1.00}{f(P)} \, \textcolor[rgb]{0.00,0.59,0.00}{ds}  = \int_C f(x,y) \, ds  = \int_{\textcolor[rgb]{0.98,0.00,0.00}{a}}^{\textcolor[rgb]{0.98,0.00,0.00}{b}} \textcolor[rgb]{0.00,0.00,1.00}{f(x(t),y(t))}\textcolor[rgb]{0.00,0.59,0.00}{\sqrt{(x'(t))^2 + (y'(t))^2} \, dt}\; .
$$
\end{frame}

\begin{frame}
  \frametitle{Example}

  $C$: first quadrant part of the circle of radius 2 centered at origin.
  %
  $$\int_C x^2y \; ds$$
  %
  \pause Parametrization of $C$: \pause $\textbf{r}(t) = \langle 2\cos{t} , 2\sin{t} \rangle$, $\textcolor[rgb]{0.98,0.00,0.00}{0 \leqslant t \leqslant \frac{\pi}{2}}$

\pause The element of arclength is\pause
%
$$\textcolor[rgb]{0.00,0.59,0.00}{ds} = |\textbf{r}'(t)|\, dt = |\langle -2\sin{t}, 2\cos{t} \rangle|\, dt \textcolor[rgb]{0.00,0.59,0.00}{ = 2 dt}$$
%
\pause The function $f(x,y) = x^2y$ becomes\pause
%
$$\textcolor[rgb]{0.00,0.00,1.00}{f(x(t), y(t))} = (x(t))^2y(t) = \textcolor[rgb]{0.00,0.00,1.00}{8\cos^2{t} \sin{t}} \; .$$
%
\begin{align*}
  \int_{\textcolor[rgb]{0.98,0.00,0.00}{C}} \textcolor[rgb]{0.00,0.00,1.00}{x^2y} \; \textcolor[rgb]{0.00,0.59,0.00}{ds} & = \int_{t=0}^{t=\pi/2} f(x(t), y(t)) |\textbf{r}'(t)|\, dt = \int_{\textcolor[rgb]{0.98,0.00,0.00}{t=0}}^{\textcolor[rgb]{0.98,0.00,0.00}{t=\pi/2}} \textcolor[rgb]{0.00,0.00,1.00}{8\cos^2{t} \sin{t}} \; \textcolor[rgb]{0.00,0.59,0.00}{2 dt} = \\
  & = 16 \left. \frac{-\cos^3{t}}{3}\right|_{t=0}^{t=\pi/2} = \frac{16}{3}\; .
\end{align*}

\end{frame}

\begin{frame}
  \frametitle{Line Integrals from Vector Fields}

 \textcolor[rgb]{0.98,0.00,0.00}{scalar} integrals on \textcolor[rgb]{0.98,0.00,0.00}{oriented} curves
  %
  \begin{itemize}
    \item $C$: piecewise smooth, \emph{oriented} curve;
    \item $ds$: element of arclength;
    \item \pause $\textbf{F}$ is a continuous vector field defined on $C$;
    \item \pause $\textbf{T}$: unit tangent vector field on $C$ compatible with the orientation;
    \item \pause $\textbf{N}$: unit vector field on $C$ and normal to $C$ (only for planar curves);
  \end{itemize}

  \begin{itemize}
    \item \pause Line integral of $\textbf{F}$ along $C$ (in all dimensions):
%
$$\int_C \textbf{F} \cdot \textbf{dr} = \int_{\textcolor[rgb]{0.98,0.00,0.00}{C}} \textcolor[rgb]{0.00,0.00,1.00}{\textbf{F} \cdot \textbf{T}} \; \textcolor[rgb]{0.00,0.59,0.00}{ds} \; ,\quad \text{ with } \quad \textbf{dr} = \textbf{T}\, ds \; .$$
%
Work done by a force field $\textbf{F}$: $dW = \textbf{F} \cdot \textbf{dr} = \textbf{F} \cdot \textbf{T} \, ds$
 %
 \item \pause Line integral of $\textbf{F}$ across $C$:
 %
$$\int_{\textcolor[rgb]{0.98,0.00,0.00}{C}} \textcolor[rgb]{0.00,0.00,1.00}{\textbf{F} \cdot \textbf{N}} \; \textcolor[rgb]{0.00,0.59,0.00}{ds} = \int_C \textbf{F} \cdot \textbf{dn} \; , \quad \text{ with } \quad \textbf{dn} = \textbf{N}\, ds \; .$$

Flux across a membrane: $\textbf{F}\cdot \textbf{N}$ is the normal component~of~$\textbf{F}$.
 \end{itemize}

\end{frame}

\begin{frame}
  \frametitle{Computations}
$\textbf{r} \colon [a,b] \to C$: regular parametrization compatible with the orientation.\pause
%
$$\textbf{T}(\textbf{r}(t)) = \frac{1}{|\textbf{r}'(t)|}\textbf{r}'(t), \quad  ds = |\textbf{r}'(t)| \; dt \; \Longrightarrow \textbf{T} \, ds = \textbf{r}'(t) \; dt = \textbf{dr}\; .$$
%
$$\int_C \textbf{F} \cdot \textbf{dr} = \int_C \textbf{F} \cdot \textbf{T} \; ds$$
%
\pause In rectangular coordinates:
%
$$\textbf{F}(x,y) = P(x,y) \textbf{i} + Q(x,y) \textbf{j}\; .$$
%
$$\textbf{r} \colon [a,b] \to \RR^2\quad ,\quad  \textbf{r}(t) = \langle x(t), y(t) \rangle$$
%
$$\textbf{r}'(t) = \langle x'(t), y'(t) \rangle = x'(t) \textbf{i} + y'(t) \textbf{j}$$
%
$$
  \textbf{F} \cdot \textbf{dr}  = \textbf{F}(\textbf{r}(t)) \cdot \textbf{r}'(t) \, dt = \left(P(x(t),y(t)) x'(t) + Q(x(t),y(t)) y'(t)\right) \, dt \; .
$$
%
$$\int_C \textbf{F} \cdot \textbf{dr} = \int_{t=a}^{t=b} \left(P(x(t),y(t)) x'(t) + Q(x(t),y(t)) y'(t)\right) \, dt$$
\end{frame}

\begin{frame}
  \frametitle{Example}

  Work done by $\textbf{F}= \langle x, -y \rangle = x \, \textbf{i} - y\, \textbf{j}$ in moving a particle from $(1,0)$ to $(0,1)$ along the quarter of the unit circle contained in the first quadrant.\pause

A parametrization of $C$ compatible with the given orientation is
%
$$ \textbf{r}\colon \left[0, \frac{\pi}{2}\right]\to \RR^2, \quad \textbf{r}(t) = \langle \cos{t}, \sin{t} \rangle = \cos{t} \, \textbf{i} + \sin{t} \, \textbf{j}\; .$$
%
\pause
%
$$W = \int_C \textbf{F} \cdot \textbf{dr} = \int_{t=0}^{t=\pi/2} \textbf{F}(\textbf{r}(t)) \cdot \textbf{r}'(t) \, dt$$
%
\begin{align*}
  \textbf{F}(\textbf{r}(t)) = \cos{t}\, \textbf{i} -\sin{t}\, \textbf{j} \quad & , \quad
  %
  \textbf{r}'(t)  = -\sin{t}\, \textbf{i} + \cos{t}\, \textbf{j} \\
  %
  \textbf{F}(\textbf{r}(t)) \cdot \textbf{r}'(t) & = -2\sin{t}\cos{t} \; ,
\end{align*}
%
\pause
%
$$W = \int_{t=0}^{t=\pi/2} -2\sin{t}\cos{t}\, dt = \left. \cos^2{t} \right|_{t=0}^{t=\pi/2} = -1\; . $$
%
\pause What if the parametrization is \emph{not} compatible with the orientation?
\end{frame}


\begin{frame}
  \frametitle{1-Forms}

  We return to $\textbf{F} \cdot \textbf{dr}$.
Since $\textbf{r} =x\, \textbf{i} + y\, \textbf{j}$, we have $\textbf{dr} =dx\, \textbf{i} + dy\, \textbf{j}$.

If $\textbf{F}(x,y) = P(x,y) \, \textbf{i} + Q(x,y)\, \textbf{j}$, then
%
$$\textbf{F} \cdot \textbf{dr} = P(x,y) \, dx + Q(x,y) \, dy\; .$$
%
\pause An expression of the type
%
\begin{align*}
  \omega & = P(x) \, dx \hspace{6.5cm} \text{(in 1D)}\\
  %
  \omega & = P(x,y) \, dx + Q(x,y) \, dy \hspace{4cm} \text{(in 2D)}\\
  %
  \omega & = P(x,y,z) \, dx + Q(x,y,z) \, dy+ R(x,y,z)\, dz \qquad \text{(in 3D)}
\end{align*}
%
is called a \emph{1-form}.
\begin{itemize}
  \item \pause $\int_C \textbf{F} \cdot \textbf{dr}$ is the integral of a 1-form over the oriented curve $C$.

  \item \pause The definite integral $\int_a^b f(x) \,dx$ actually means

  \pause \emph{the integral of the 1-form $\omega = f(x) \, dx$} \pause

  \emph{on the segment with endpoints $a$ and $b$} \pause

  \emph{oriented from $a$ to $b$.}
\end{itemize}
\end{frame}

\begin{frame}
  \frametitle{Integrals of 1-Forms}

  \begin{itemize}
    \item $\omega = P(x,y) \, dx + Q(x,y) \, dy$ is a 1-form;
    \item $C$ is an \textcolor[rgb]{0.98,0.00,0.00}{oriented} curve,
  \end{itemize}
%
$$\int_{\textcolor[rgb]{0.98,0.00,0.00}{C}} \textcolor[rgb]{0.00,0.00,1.00}{\omega} = \int_{\textcolor[rgb]{0.98,0.00,0.00}{C}} \textcolor[rgb]{0.00,0.00,1.00}{P(x,y) \, dx + Q(x,y) \, dy} $$
%
is computed using an orientation-compatible parametrization\pause

$\textbf{r} \colon [a,b]\to C$, $\textbf{r}(t) = \langle x(t), y(t) \rangle$:
%
\begin{align*}
  dx = x'(t)\, dt\qquad &, \qquad dy = y'(t) \, dt \\
  %
  P(x,y) \, dx + Q(x,y) \, dy  &= \left(P(x(t),y(t))x'(t) + Q(x(t),y(t))y'(t)\right) \, dt
\end{align*}
%
$$\int_{\textcolor[rgb]{0.98,0.00,0.00}{C}} \textcolor[rgb]{0.00,0.00,1.00}{P(x,y) \, dx + Q(x,y) \, dy} = \int_{\textcolor[rgb]{0.98,0.00,0.00}{a}}^{\textcolor[rgb]{0.98,0.00,0.00}{b}} \textcolor[rgb]{0.00,0.00,1.00}{\left(P(x(t),y(t))x'(t) + Q(x(t),y(t))y'(t)\right) \, dt}$$

\pause Re-parametrization of the interval $\simeq$ Substitution rule for integrals
\end{frame}

\begin{frame}
  \frametitle{Examples}

$C$: circle of radius $R$ centered at the origin, oriented counterclockwise.
%
$$\oint_C \frac{x}{x^2+y^2} \, dy - \frac{y}{x^2+y^2}\, dx = ?$$
%

\pause Parametrization: $x(t) = R\cos{t}$, $y(t) = R\sin{t}$, $0 \leqslant t \leqslant 2\pi$.

\pause Then  $dx= -R\sin{t} \, dt$, $dy = R\cos{t} \, dt$, hence
%
$$\frac{x}{x^2+y^2} \, dy - \frac{y}{x^2+y^2}\, dx = \frac{R\cos{t} (R\cos{t} \, dt)}{R^2} - \frac{R\sin{t} (-R\sin{t} \, dt)}{R^2} = dt\; ,$$
%
\pause and therefore
%
$$\oint_C \frac{x}{x^2+y^2} \, dy - \frac{y}{x^2+y^2}\, dx = \int_{0}^{2\pi} dt = 2\pi\; .$$

\pause Compute
%
$$\oint_C \frac{x}{x^2+y^2} \, dx + \frac{y}{x^2+y^2}\, dy $$
%
\end{frame}

\begin{frame}
  \frametitle{1-Forms in Polar Coordinates}

  In polar coordinates:
  %
  $$x = r\cos{\theta} \Longrightarrow dx = \cos{\theta} dr - r\sin{\theta} d\theta$$
  $$y = r\sin{\theta} \Longrightarrow dy = \sin{\theta} dr + r\cos{\theta} d\theta$$
  %
  \pause Then:
  %
  $$ \frac{x}{x^2+y^2} \, dy - \frac{y}{x^2+y^2}\, dx = d\theta$$
  %
  \pause The curve is $r(t) = R$, $\theta(t) = t$, $0 \leqslant t \leqslant 2\pi$.
  %
  $$\oint_C \frac{x}{x^2+y^2} \, dy - \frac{y}{x^2+y^2}\, dx = \oint_C d\theta = 2\pi$$
  %
  \pause Similarly:
  %
  $$\frac{x}{x^2+y^2} \, dx + \frac{y}{x^2+y^2}\, dy =
  \frac{1}{r} \, dr =  d(\ln{r}) $$
  %
  $$\oint_C \frac{x}{x^2+y^2} \, dx + \frac{y}{x^2+y^2}\, dy = \oint_C d(\ln{r}) = 0$$
\end{frame}


\begin{frame}
  \frametitle{Inverse Square Distance Fields}

  \begin{itemize}
    \item $O$: fixed point in space
    \item $\textbf{F}$: vector field (defined away from $O$) given by
%
$$\textbf{F}(A) = \frac{1}{|OA|^2} \widehat{\textbf{AO}}\,  = -\frac{1}{|OA|^3} \, \textbf{OA}\; .$$
    \item $C$ be a smooth curve with endpoints $A$ and $B$.
    \item Work done by the field $\textbf{F}$ in moving a particle from $A$ to $B$ along $C$
%
$$W = \int_C \textbf{F} \cdot \textbf{dr}\; .$$
  \end{itemize}

\pause  $\textbf{r} \colon [a,b] \to C$: parametrization of $C$, with $A=\textbf{r}(a)$ and $B=\textbf{r}(b)$
 %
$$
  W  = \int_C \textbf{F} \cdot \textbf{dr} = \int_a^b \textbf{F}(\textbf{r}(t)) \cdot \textbf{r}'(t) \, dt = \int_a^b - \frac{1}{|\textbf{r}(t)|^3} \, \textbf{r}(t) \cdot \textbf{r}'(t) \, dt \; .
$$
\end{frame}

\begin{frame}
    \frametitle{Independence of Path}
$$
  W  = \int_C \textbf{F} \cdot \textbf{dr} = \int_a^b \textbf{F}(\textbf{r}(t)) \cdot \textbf{r}'(t) \, dt = \int_a^b - \frac{1}{|\textbf{r}(t)|^3} \, \textbf{r}(t) \cdot \textbf{r}'(t) \, dt \; .
$$
%
\pause
%
$$\textbf{r}(t) \cdot \textbf{r}'(t)  = \frac{1}{2} \frac{d}{dt}(\textbf{r}(t) \cdot \textbf{r}(t)) = \frac{1}{2} \frac{d}{dt}|\textbf{r}(t)|^2$$
%
\pause Change of variable $u=|\textbf{r}(t)|^2$, $du = 2 \textbf{r}(t) \cdot \textbf{r}'(t)\, dt$
%
%$$du =  \, dt = \frac{d}{dt}(\textbf{r}(t) \cdot \textbf{r}(t)) \, dt =  2 \textbf{r}(t) \cdot \textbf{r}'(t) \, dt$$
%
\begin{align*}
  W & = \int_{u=|\textbf{r}(a)|^2}^{u=|\textbf{r}(b)|^2} -\frac{1}{u^{3/2}} \, \frac{1}{2} \, du = \left. u^{-1/2} \right|_{u=|\textbf{r}(a)|^2}^{u=|\textbf{r}(b)|^2} = \\
  & = \frac{1}{|\textbf{r}(b)|} - \frac{1}{|\textbf{r}(a)|} = \frac{1}{|OB|} - \frac{1}{|OA|}
\end{align*}

\pause The line integral:
 \begin{itemize}
   \item \pause depends only on the endpoints $A$ and $B$;
   \item \pause does not depend on the path $C$ from $A$ to $B$.
 \end{itemize}
\end{frame}

\begin{frame}
  \frametitle{Conservative Fields}

  \underline{Definition}: A vector field $\textbf{F}$ is called \emph{conservative} if for every pair of points $A$ and $B$ and every pair of paths $C_1$ and $C_2$ joining $A$ to $B$ we have
%
$$\int_{C_1} \textbf{F} \cdot \textbf{dr} = \int_{C_2} \textbf{F}\cdot \textbf{dr}\; .$$

\pause \underline{Equivalent definition}: A vector field $\textbf{F}$ is conservative if for every point $A$ every path $C$ starting and ending at $A$ we have
%
$$\oint_{C} \textbf{F} \cdot \textbf{dr} = 0\; .$$

\pause Equivalence: $C = C_1 \cup (-C_2)$ starts and ends at $A$ and
%
$$\int_{C} \textbf{F} \cdot \textbf{dr} = \int_{C_1} \textbf{F} \cdot \textbf{dr} - \int_{C_2} \textbf{F} \cdot \textbf{dr}\; .$$
\end{frame}

\begin{frame}
  \frametitle{Conservative Field = Gradient Field}

  \begin{itemize}
    \item Let $\textbf{F}$ be conservative field. \pause Fix a point $A$ and define $f$ by
%
$$f(B)= f(\textbf{r}_B) = \int_C \textbf{F} \cdot \textbf{dr} \; ,$$
%
where $C$ is \emph{any} piecewise smooth curve joining $A$ to $B$.

\pause Then $\textbf{F} = \nabla f$ $\Longrightarrow$ $\textbf{F}$ is a gradient field.

Any function $f$ such that $\textbf{F}= \nabla f$ is called a \emph{scalar potential} for~$\textbf{F}$.

\item \pause
\textbf{Fundamental Theorem of Calculus} (Line Integral version):
%
$$\int_C (\nabla f) \cdot \textbf{dr} = \int_a^b \frac{d}{dt} (f(\textbf{r}(t)) \, dt = f(B) - f(A) \; ,$$
%
for every curve $\textbf{r} \colon [a,b] \to C$ joining $A$ to $B$.

 \item \pause Let $\textbf{F} = \nabla f$ be gradient field. \pause For a curve $C$ joining points $A$ and $B$
 %
 $$\int_C \textbf{F} \cdot \textbf{dr} =  \int_C (\nabla f )\cdot \textbf{dr}= f(B) - f(A)\;$$
 %
\pause depends only on $A$ and $B$, but not on $C$ $\Longrightarrow$ $\textbf{F}$ is conservative.
  \end{itemize}

\end{frame}


\begin{frame}
  \frametitle{A Criterion in Rectangular Coordinates}

$\textbf{F}(x,y) = P(x,y) \textbf{i} + Q(x,y) \textbf{j}$: smooth field

If $\textbf{F}$ is a gradient field, then $\textbf{F} = \nabla f$, hence
%
$$P(x,y) = f_x(x,y), \qquad Q(x,y) = f_y(x,y)\, .$$
%
\pause Since mixed partial derivatives are equal, it follows that
%
$$P_y = (f_x)_y = f_{xy} = f_{yx} = (f_y)_x = Q_x \; .$$
%
\pause If $\textbf{F} = P \textbf{i} + Q \textbf{j}$ is a gradient field, then
%
$$P_y = Q_x \; .$$

\pause If $\textbf{F} = P\, \textbf{i} + Q\, \textbf{j}+ R\, \textbf{k}$ is a gradient field, then
%
$$P_y = Q_x, \quad P_z = R_x, \quad Q_z=R_y\; .$$
%

\end{frame}

\begin{frame}
  \frametitle{Simply Connected Regions}

  If $P_y(x,y) \neq Q_x(x,y)$, then \pause $\textbf{F}$ is not a gradient field.

  \pause But if $P_y(x,y) = Q_x(x,y)$, is $\textbf{F}$ necessarily a gradient field?

  \pause Unfortunately, NO!
%
$$\textbf{F}(x,y) = \frac{-y}{x^2+y^2} \, \textbf{i} + \frac{x}{x^2+y^2} \, \textbf{j} = P(x,y) \, \textbf{i} + Q(x,y)\, \textbf{j}\; ,$$
%
$$P_y(x,y) = \frac{y^2-x^2}{(x^2+y^2)^2} = Q_x(x,y) \; .$$
%
$$\oint_C \textbf{F} \cdot \textbf{dr} = \oint_C \frac{x}{x^2+y^2} \, dy - \frac{y}{x^2+y^2}\, dx = 2\pi \neq 0 \; .$$

\pause \underline{Definition}: A domain $D$ is called \emph{simply connected} if every closed loop in $D$ can be deformed (``lassoed'') into a point inside $D$.

\medskip

If $\textbf{F}(x,y)=P(x,y) \, \textbf{i} + Q(x,y) \, \textbf{j}$ is defined on a simply connected domain $D$ and $P_y(x,y)=Q_x(x,y)$ over $D$, then $\textbf{F}$ is a gradient field.

Similar for $\textbf{F} = P \, \textbf{i} + Q\, \textbf{j} + R\, \textbf{k}$
\end{frame}

\begin{frame}
  \frametitle{Finding a Scalar Potential}

  Let $\textbf{F}(x,y) = (3+2xy) \, \textbf{i} + (x^2-3y^2)\, \textbf{j}$, defined on the entire plane. \pause

  Then $P(x,y) = 3+2xy$ and $Q(x,y) = x^2-3y^2$ and $P_y = 2x=Q_x$.

  \pause $P_y=Q_x$ and $\RR^2$ is simply connected $\Longrightarrow$ $\textbf{F}$ is a gradient field, $\textbf{F} = \nabla f$.

  \begin{itemize}
    \item \pause The scalar potential is, up to a constant, given by
%
$$f(x,y) = \int_C \textbf{F} \cdot \textbf{dr} \; .$$
%
    \item \pause Partially integrate $P$ and $Q$:
    %
    $$f_x = P(x,y) = 3+2xy \Longrightarrow f(x,y) = 3x+x^2y + c(y)$$
    %
    \pause
    %
    $$x^2-3y^2  = Q(x,y) = f_y = x^2+c'(y) \Longrightarrow c(y) = -y^3+a$$
    %
    \pause
    %
    $$f(x,y) = 3x+x^2y + c(y) = 3x+x^2y-y^3+a\; .$$
  \end{itemize}

\pause $C$: any smooth curve joining the points $(1,0)$ and $(0,1)$.
%
$$\int_C \textbf{F} \cdot \textbf{dr} = \int_C (\nabla f) \cdot \textbf{dr} = f(1,0) - f(0,1) = (3+a) - (-1+a) = 4\; .$$

\end{frame}

\begin{frame}
  \frametitle{Exact 1-Forms}

  Let $\textbf{F}(x,y) = P(x,y) \, \textbf{i} + Q(x,y)\, \textbf{j}$ be a vector field, and $\omega = \textbf{F} \cdot \textbf{dr} = P(x,y) dx + Q(x,y) dy$ the corresponding 1-form. \pause
%
$$\textbf{F} = \nabla f \Longleftrightarrow P=f_x, Q=f_y \Longleftrightarrow Pdx + Qdy = f_xdx+f_ydy \Longleftrightarrow \omega = df$$
%
\pause 1-forms that are total differentials of functions are called \emph{exact}. Then
%
\begin{center}
  $\textbf{F}$ is a gradient field $\Longleftrightarrow$ the 1-form $\omega = \textbf{F} \cdot \textbf{dr}$ is exact
\end{center}

\pause
\textbf{Net Change Theorem} (Line Integral version):

If $C$ is a curve joining points $A$ and $B$, then
%
$$\int_C df = \int_C (\nabla f) \cdot \textbf{dr} = f(B) - f(A)\; .$$

\pause Compare to: $df = f'(x)\, dx$
%
$$\int_a^b df = \int_a^b f'(x)\, dx = f(b)-f(a)\; .$$
\end{frame}

\begin{frame}
  \frametitle{Conservation of Energy}

  What does a conservative field $\textbf{F}$ conserve?

  \pause Particle of mass $m$ in a force field $\textbf{F} = \nabla f$

  \pause Moves from from $A$ to $B$, with $\textbf{r}(t)$ the position vector at time $t$.

  \begin{itemize}
    \item \pause $K = \frac{1}{2}\, mv^2 = \frac{1}{2}\, m|\textbf{r}'(t)|^2$ is the kinetic energy;
    \item \pause $V=-f$ is called the \emph{potential} energy;
    \item \pause $E=K+V$ is the
\emph{total} energy.
  \end{itemize}
  %
  $$V(A)-V(B) = f(B) - f(A) = \int_C (\nabla f) \cdot \textbf{dr} = \int_C \textbf{F} \cdot \textbf{dr} = K(B)-K(A)$$
  %
  \pause Therefore the total energy is conserved:
  %
  $$E(B) = K(B)+V(B) = K(A)+V(A) = E(A)\; .$$

\pause
%
$$\text{A scalar potential for gravitational field } \textbf{F}(P) = -\frac{KMm}{|\textbf{r}|^3}\, \textbf{r} \text{ is } f(P) = \frac{KMm}{|\textbf{r}|}  $$
%
$$\text{Potential energy should be } V = -\frac{KMm}{R+h} \text{ and not } V = mgh\; .$$
\pause How do you reconcile the two formulas?
\end{frame}

\begin{frame}
  \frametitle{Planar Vector Fields}

    \begin{itemize}
      \item $\textbf{F}$: smooth vector field defined on $\mathcal{R}$;
      \item $\textbf{p}$: point in the interior of $\mathcal{R}$;
      \item Question: How does $\textbf{F}$ impact the space around $p$?
      \item \pause Method: study the effect of $\textbf{F}$ on the boundary $C =\partial D$ of a small region $D$ around $p$
    \end{itemize}

  \pause There are two fundamental effects:

\begin{enumerate}
  \item \pause Normal component of $\textbf{F}$ $\Longrightarrow$ moves matter across the boundary. Accumulation: \emph{flux across $C$};

  \item \pause Tangential component of $\textbf{F}$ $\Longrightarrow$ moves matter along the boundary. Accumulation: \emph{circulation along $C$}.
\end{enumerate}

\end{frame}

\begin{frame}
  \frametitle{Flux and Divergence}

    \begin{itemize}
      \item Convention: we measure the \emph{outward} flux;
      \item $\textbf{N}$: unit vector, normal to $C$ and pointing outwards from $D$;
    \end{itemize}
%
$$\text{Outward Flux } = \oint_C \textbf{F} \cdot \textbf{N} ds = \oint_C \textbf{F} \cdot \textbf{dn} \; ,$$
%
where $\textbf{dn} =\textbf{N} ds$ is the normal vectorial element of arclength

\medskip

\pause \emph{Divergence} of $\textbf{F}$ at $p$: the density of flux
%
\begin{align*}
  (\divg \textbf{F})(p) = & \lim_{D \to \{p\}} \frac{\text{Outward flux across boundary}}{\text{Area}(D)} = \\ = & \lim_{D\to \{p\}} \frac{1}{\text{Area}(D)} \oint_C \textbf{F} \cdot \textbf{dn}
\end{align*}
\end{frame}



\begin{frame}
  \frametitle{Circulation and Flatland Curl}

    \begin{itemize}
      \item Convention: the plane has a predetermined orientation.
      \item a regular parametrization of $C$ is \emph{positive} if $C$ is followed in a counterclockwise direction $\Longleftrightarrow$ $\{ \textbf{N}, \textbf{T}\}$ is positively oriented
    \end{itemize}

%
$$\text{Circulation } = \oint_C \textbf{F} \cdot \textbf{T} ds = \oint_C \textbf{F} \cdot \textbf{dr}\; ,$$
%
where $\textbf{dr} = \textbf{r}'(t) dt = \textbf{T} ds$ is the tangent vectorial element of arclength.

\medskip

\pause The \emph{Flatland curl} of $\textbf{F}$ at $p$:  the density of circulation
%
\begin{align*}
(\curl_{\,\textbf{k}} \textbf{F})(p)= & \lim_{D \to \{p\}} \frac{\text{Circulation along boundary}}{\text{Area}(D)} =  \\ = &
      \lim_{D\to \{p\}} \frac{1}{\text{Area}(D)} \oint_C \textbf{F} \cdot \textbf{dr} \; .
\end{align*}

\end{frame}

\begin{frame}
  \frametitle{Computations in Rectangular Coordinates}
\begin{itemize}
  \item $\textbf{F} (x,y) = P(x,y) \textbf{i} + Q(x,y) \textbf{j}$
  \item $\textbf{r} \colon [a,b] \to \RR^2$ be a positive, piecewise smooth parametrization by arclength of the boundary $C = \partial D$
%
$$\textbf{r}(t) = x(t) \textbf{i} + y(t) \textbf{j} \; .$$
%
$$
  \textbf{T} (t)  = x'(t) \textbf{i} + y'(t) \textbf{j} \qquad \text{ and } \qquad
  \textbf{N} (t)  = y'(t) \textbf{i} - x'(t) \textbf{j}
$$
%
\end{itemize}

Then
%
\begin{align*}
  \oint_C \textbf{F} \cdot \textbf{dn} = & \int_a^b \left[P(x,y) y' - Q(x,y) x'\right]\, dt  = \oint_C P(x,y) dy - Q(x,y) dx \\
%
  \oint_C \textbf{F} \cdot \textbf{dr} = & \int_a^b \left[P(x,y) x'+ Q(x,y) y'\right]\, dt  = \oint_C P(x,y) dx + Q(x,y) dy
\end{align*}

\end{frame}

\begin{frame}
  \frametitle{}

\begin{align*}
  (\divg F )(p) = & \lim_{D\to \{p\}} \frac{1}{\text{Area}(D)} \oint_C \textbf{F} \cdot \textbf{dn} = \\ = & \lim_{D\to \{p\}} \frac{1}{\text{Area}(D)} \oint_C P(x,y) dy - Q(x,y) dx \\
  (\curl_{\,\textbf{k}} \textbf{F})(p)= & \lim_{D\to \{p\}} \frac{1}{\text{Area}(D)} \oint_C \textbf{F} \cdot \textbf{dr} = \\ = & \lim_{D\to \{p\}} \frac{1}{\text{Area}(D)} \oint_C P(x,y) dx + Q(x,y) dy
\end{align*}

\pause \underline{Key fact}: If $f$ is a continuous function defined around $p$, then
%
$$\lim_{D \to \{p\}} \frac{1}{\text{Area}(D)} \iint_{D} f(q) dA = f(p)\; .$$

\pause We need to convert the line integrals into double integrals.
\end{frame}

\begin{frame}
  \frametitle{\textcolor[rgb]{0.00,1.00,0.00}{Green's Theorem}}

\begin{theorem}[Green]
  {\rm Let $D$ be a region whose boundary $C$ is piecewise smooth and positively oriented. If $P$ and $Q$ have continuous partial derivatives in an open region around $D$, then}
  %
  $$
    \oint_C P(x,y) dx + Q(x,y) dy = \iint_D \left[\frac{\partial Q}{\partial x}(x,y) - \frac{\partial P}{\partial y}(x,y) \right] dxdy \; .
  $$
\end{theorem}
%
Companion formula:
%
  $$
    \oint_C P(x,y) dy - Q(x,y) dx = \iint_D \left[\frac{\partial P}{\partial x}(x,y) + \frac{\partial Q}{\partial y}(x,y) \right] dxdy \; .
  $$
\end{frame}

\begin{frame}
  \frametitle{Divergence and Flatland Curl}

  \begin{align*}(\curl_{\textbf{k}} \textbf{F})(p) = &  \lim_{D \to \{ p \}}  \frac{1}{\text{Area}(D)} \oint_{\partial D} \textbf{F} \cdot \textbf{dr} = \\
  =& \lim_{D \to \{ p \}}  \frac{1}{\text{Area}(D)} \oint_{\partial D} P(x,y) dx+Q(x,y)dy = \\
  = & \lim_{D \to \{ p\}}  \frac{1}{\text{Area}(D)} \iint_{D} \left[\frac{\partial Q}{\partial x} (x,y) - \frac{\partial P}{\partial y} (x,y) \right] \, dxdy = \\
  = & \frac{\partial Q}{\partial x} (x_0,y_0) - \frac{\partial P}{\partial y} (x_0,y_0)
  \end{align*}
%
$$
  (\divg \textbf{F})(p) = \lim_{D \to \{p\}} \frac{1}{\text{Area}(D)} \oint_{\partial D} \textbf{F} \cdot \textbf{dn}   = \frac{\partial P}{\partial x} (x_0,y_0) + \frac{\partial Q}{\partial y} (x_0,y_0) \; .
$$

\pause Coordinate-free formulations of Green's theorem:
%
$$\oint_{\partial D} \textbf{F} \cdot \textbf{dr} = \iint_D \curl_{\textbf{k}} \textbf{F} \, dA \qquad \text{ and } \qquad
\oint_{\partial D} \textbf{F} \cdot \textbf{dn} = \iint_{D} \divg \textbf{F} \, dA$$

\end{frame}

\begin{frame}
  \frametitle{Interpretation of Divergence}

$$\oint_{\partial D} \textbf{F} \cdot \textbf{dn} = \iint_{D} \divg \textbf{F} \, dA$$

  \begin{itemize}
    \item If $\divg \textbf{F}(p)>0$:
    \begin{itemize}
      \item \pause for small enough regions around $p$, $\textbf{F}$ carries more outside than it brings inside through the boundary;
      \item \pause $p$ acts as a source.
    \end{itemize}
    \item If $\divg \textbf{F}(p)<0$:
    \begin{itemize}
      \item \pause for small enough regions around $p$, $\textbf{F}$ carries less outside than it brings inside through the boundary;
      \item \pause $p$ acts as a sink
    \end{itemize}
    \item If the divergence is identically zero on a region $D$:
    \begin{itemize}
      \item \pause the outward flux through any closed curve is zero;
      \item \pause the amount pushed inside in some places is equal to the amount pushed outside somewhere else;
      \item The field $\textbf{F}$ is \emph{incompressible} or \emph{solenoidal}.
    \end{itemize}
  \end{itemize}
\end{frame}

\begin{frame}
  \frametitle{Interpretation of Curl}

  $$\oint_{\partial D} \textbf{F} \cdot \textbf{dr} = \iint_D \curl_{\textbf{k}} \textbf{F} \, dA $$

  \begin{itemize}
    \item If $\curl_{\textbf{k}} \textbf{F}(p)>0$:
    \begin{itemize}
      \item \pause $\textbf{F}$ tends to rotate points counter-clockwise around $p$;
    \end{itemize}

    \item If $\curl_{\textbf{k}} \textbf{F}(p)<0$:
    \begin{itemize}
      \item \pause $\textbf{F}$ tends to rotate points clockwise around $p$
    \end{itemize}

    \item If $\curl_{\textbf{k}} \textbf{F} \equiv 0$ on a region $D$:
    \begin{itemize}
      \item \pause $\textbf{F}$ is an \emph{irrotational field}
      \item \pause the circulation along any closed curve $C$ that bounds a subregion of $D$ is zero.
      \item \pause If $D$ is simply connected:
      \begin{itemize}
        \item \pause any closed curve in $D$ bounds a subregion of $D$;
        \item \pause a zero circulation field is conservative.
      \end{itemize}
    \end{itemize}
  \end{itemize}

\pause On simply connected regions:
\begin{center}
  Irrotational field $\Longleftrightarrow$ Conservative field
\end{center}

\end{frame}

\begin{frame}
  \frametitle{Divergence and Curl in Polar Coordinates}

Suppose that in polar coordinates $(r, \theta)$, the field $\textbf{F}$ is given by
%
$$\textbf{F}(r, \theta) = P(r, \theta) \textbf{e}_{r} + Q(r, \theta) \textbf{e}_{\theta} \; ,$$
%
where $\textbf{e}_{r} = \textbf{e}_{r}(r, \theta)$ and $\textbf{e}_{\theta}=\textbf{e}_{\theta}(r, \theta)$ are the unit polar coordinate vectors.

%
$$\divg \textbf{F} =  \frac{1}{r} \frac{\partial (r P)}{\partial r} + \frac{1}{r} \frac{\partial Q}{\partial \theta} \quad \text{ and } \quad
   \curl_{\textbf{k}} \textbf{F} =  \frac{1}{r} \frac{\partial (r Q)}{\partial r} - \frac{1}{r} \frac{\partial P}{\partial \theta}
$$

\begin{itemize}
  \item \pause If $\textbf{F}$ is a radial field given in polar coordinates by $\textbf{F} = f(r,\theta)\textbf{e}_r$, then
%
$$\curl_{\textbf{k}} \textbf{F} = \frac{1}{r} \frac{\partial (r 0)}{\partial r} - \frac{1}{r}\frac{\partial f(r,\theta)}{\partial \theta} = - \frac{1}{r} \frac{\partial f}{\partial \theta} \, .$$
\item \pause Radial fields $\textbf{F}(r,\theta) = f(r)\textbf{e}_r$ are irrotational.
\end{itemize}

\end{frame}


\begin{frame}
  \frametitle{Examples}

\begin{itemize}
  \item $\textbf{F}$: a vector field that
  \begin{itemize}
    \item rotates points around $p$,
    \item with constant angular velocity $\omega$.
  \end{itemize}
  \item \pause In rectangular coordinates centered at $p$:
  %
  $$\textbf{F}(x,y) = -\omega y \textbf{i} + \omega x \textbf{j}$$
  %
  $$\curl_{\textbf{k}} \textbf{F} = \frac{\partial (\omega x)}{\partial x} - \frac{\partial (-\omega y)}{\partial y} = 2\omega \; ;$$
  %
  \item \pause In polar coordinates:
  %
  $$\textbf{F}(r, \theta) = \omega r \textbf{e}_{\theta}$$
  %
$$\curl_{\textbf{k}} \textbf{F} = \frac{1}{r}\frac{\partial (r \omega r)}{\partial r} - \frac{1}{r}\frac{\partial 0}{\partial \theta} = 2\omega \; .$$
\end{itemize}

\pause The divergence of this field is identically zero, so the field is incompressible.

\end{frame}

\begin{frame}
   \begin{itemize}
     \item $\textbf{F}$: a field that
     \begin{itemize}
       \item pushes points away from a central source,
       \item with magnitude inverse proportional to the square of the distance.
     \end{itemize}

     \item \pause In rectangular coordinates:
%
$$\textbf{F}(x,y) = \frac{ax}{(x^2+y^2)^{3/2}}\, \textbf{i} + \frac{ay}{(x^2+y^2)^{3/2}} \, \textbf{j}\, .$$
%
\item \pause In polar coordinates:
%
$$\textbf{F}(r,\theta) = ar^{-2} \textbf{e}_r$$
%
$$\divg \textbf{F} = \frac{1}{r} \frac{\partial (r a r^{-2})}{\partial r} + \frac{1}{r} \frac{\partial 0}{\partial \theta} = -\frac{a}{r^3}\; .$$
%
$$\curl_{\textbf{k}} \textbf{F} = 0\; .$$
   \end{itemize}

\end{frame}

\begin{frame}
  \frametitle{Average Values}

\begin{itemize}
\item $C$:  curve; $L(C)$: the length of $C$.
\item $f$: a continuous function (with scalar of vectorial values) defined on $C$
\end{itemize}

The average value of $f$ on $C$ with respect to arclength is
%
$$\text{Average} = \frac{1}{L(C)} \int_C f(p)\,  ds$$
%

\underline{Example:} Average value of $f(x) = x^2$ on the circle of radius $R$ centered at the origin 
%
$$\text{Average} = \frac{1}{2\pi R} \int_C x^2 \, ds \; .$$
%
\pause
%
\begin{itemize}
\item Parametrization:  $x=R\cos{t}$, $y=R\sin{t}$, with $0 \leqslant t \leqslant 2\pi$;
\item Element of arclength: $ds = R\, dt$
\end{itemize}
 %
\begin{align*}
  \text{Average} = & \frac{1}{2\pi R} \int_C x^2 \, ds = \frac{1}{2\pi R} \int_{t=0}^{t=2\pi} R^2\cos^2{t}\,  R \, dt = \frac{R^3}{2\pi R} \int_0^{2\pi} \cos^2{t}\, dt \\
  %
  = & \frac{R^2}{2\pi} \int_0^{2\pi} \frac{1-\cos{(2t)}}{2}\, dt = \frac{R^2}{2\pi} \; \pi = \frac{R^2}{2}\; .
\end{align*}
\end{frame}

\begin{frame}
  \frametitle{Centroid and Center of Mass}

\begin{itemize}
\item Centroid:
	$$x_c = \frac{1}{L} \int_C x\, ds \quad , \quad  y_c = \frac{1}{L} \int_C y\, ds \quad ,  \ldots $$
	
	\item Center of Mass:
	$$x_{cm} = \frac{1}{M} \int_C x\, dm = \frac{1}{M} \int_C x \rho\, ds\quad , \quad  y_{cm}= \frac{1}{M} \int_C y\, dm = \frac{1}{M} \int_C y \rho\, ds\quad , \ldots $$
	\end{itemize}

  \underline{Example:} Semicircle $C$ of radius $R$.
  \pause
  \begin{itemize}
  \item Upper half of the circle of radius $R$ centered at the origin;
  \item Symmetry $\to$ $x_c = 0$;
  \item $y_c$: average of the $y-$values with respect to arclength:
  \end{itemize}
%
$$y_C = \frac{1}{\pi R} \int_C y\, ds = \frac{1}{\pi R}\int_{0}^{\pi} R\sin{t}\, R\, ds = \frac{R}{\pi} \left. (-\cos{t})\right|_0^\pi = \frac{2R}{\pi} \; .$$
\end{frame}

\begin{frame}
  \frametitle{Areas using Green's Theorem}

  $$\oint_{C=\partial D} P(x,y) dx + Q(x,y) dy = \iint_D (Q_x(x,y) - P_y(x,y)) \, dx\, dy$$
  
\begin{itemize}
  \item compute a double integral by computing a line integral;
  \item compute a line integral by computing a double integral.
\end{itemize}

Example:
%
\begin{align*}
\text{Area}(D) & = \iint_D 1 dA = \iint_D (Q_x(x,y) - P_y(x,y)) \, dx\, dy \\
& = \oint_{C=\partial D} P(x,y) dx + Q(x,y) dy\; .
\end{align*}
%
$Q_x - P_y = 1$: There are many such pairs. For example:
%
\begin{itemize}
  \item $P(x,y) = -y$ and $Q(x,y) = 0$,
  \item $P(x,y) = 0$ and $Q(x,y) = y$,
  \item $P(x,y) = -y/2$ and $Q(x,y) = x/2$.
\end{itemize}
%
$$\text{Area}(D) = \iint_D 1 dA = \oint_C -ydx = \oint_C xdy = \frac{1}{2}\oint_C xdy-ydx\; .$$
\end{frame}

\end{small}

\end{document}

\begin{frame}
  \frametitle{}
 
  The first line integral should look (almost) familiar! If $D$ is the region bounded by the horizontal segment from $(0,a)$ to $(0,b)$ (with $a<b$), the vertical lines $y=a$ and $y=b$, and the graph of a positive function $y=f(x)$ defined on $[a,b]$, then the boundary has four pieces. On the vertical lines the line integral is zero because $dx=0$ and on the horizontal segment the integral is also zero because $y=0$. The only piece with non-zero contribution is the graph. A parametrization consistent with its orientation is $x=t$, $y=f(t)$, with $t$ from $b$ to $a$. Hence
%
$$\text{Area}(D) = \oint_C -ydx = \int_b^a -f(t)dt = \int_a^b f(t) dt$$
%
hence
%
$$\int_a^b f(t) dt = \text{Area}(D)\; .$$
%
But if $f$ is negative on the interval $[a,b]$, then the parametrization of the graph consistent with its orientation is $x=t$, $y=f(t)$, with $t$ from $a$ to $b$. Then
%
$$\text{Area}(D) = \oint_C -ydx = \int_a^b -f(t)dt = - \int_a^b f(t) dt$$
%
hence
%
$$\int_a^b f(t) dt = - \text{Area}(D) \; .$$
%
This explains the convention for definite integrals as \emph{signed} areas and the reason why signed areas of regions below the horizontal axis $y=0$ are considered negative.

  \underline{Example} We use Green's theorem to compute the area of the domain $D$ enclosed by ellipse $C$: $\frac{x^2}{a^2}+\frac{y^2}{b^2}=1$. The area is given by
%
$$\text{Area} = \int\!\!\!\int_D 1\; dA = \int_C x\, dy\; .$$
%
A parametrization of the ellipse $C$ is given by $x=a\cos{t}$, $y= b\sin{t}$, with $0 \leqslant t \leqslant 2\pi$. Therefore
%
$$\text{Area} = \int_C x\, dy = \int_{t=0}^{t=2\pi} a\cos{t}\, b\cos{t}\, dt = ab\int_0^{2\pi} \cos^2{t} \; dt = \pi ab\; .$$
\end{frame}

\begin{frame}
  \frametitle{Line Integrals using Double Integrals}

  Let
%
$$\textbf{X} = (y^2+e^{\sqrt{1+x^2}}) \, \textbf{i} + (x-\sin{\frac{1}{1+y^2}})\, \textbf{j}$$
%
and let $C$ be the boundary of a square of side $a$ centered at the origin. We want to compute the circulation of $\textbf{X}$ along $C$. We have two alternatives:

\begin{itemize}
  \item Parametrize $C$ and then change the line integral into a sum of integrals over intervals in $\RR$.
  \item Use Green's Theorem and change the line integral into an integral over the square region.
\end{itemize}

Given the hideous aspect of the field, the first method is not too appealing, and it is not recommended in this case. But the second one is definitely nicer, since the monster terms bring no contribution to the curl:
%
$$\curl_{\bm{k}} \textbf{X} = \partial_x (x-\sin{\frac{1}{1+y^2}}) - \partial_y (y^2+e^{\sqrt{1+x^2}}) = 1-2y \; .$$
%
Therefore
%
$$\int_C \textbf{X} \cdot \textbf{dr} = \int\!\!\!\int_D \curl_{\textbf{k}} \textbf{X} \, dA = \int\!\!\!\int_D (1-2y) dA = \text{Area}(D) = a^2\; ,$$
%
since the integral of $2y$ is zero due to symmetry.
\end{frame}

\begin{frame}
  \frametitle{}

  We proved that the radial irrotational and incompressible fields are of the form
%
$$\textbf{X} = ar^{-1} \; \textbf{e}_r\; .$$
%
Let $C$ be a closed curve enclosing the domain $D$. If $D$ does not contain the origin, then we can use Green's theorem to conclude that
%
$$\oint_C \textbf{X} \cdot \textbf{dr} = \int_D \curl_{\bm{k}} \textbf{X} \, dA = 0$$
%
and
%
$$\oint_C \textbf{X} \cdot \textbf{ds} = \int_D \divg \textbf{X} \, dA = 0$$

But if $D$ contains the origin, then we can't use Green's theorem \emph{like above}, because $\textbf{X}$ is not defined everywhere in $D$! But we can still use Green's theorem to reduce the computations to more manageable curves!

Let $C_R$ be a circle of radius $R$ centered at the origin and contained in $D$, and let $E$ be the region between the circle $C_R$ and the curve $C$. We can now use Green's theorem on $E$, because $X$ is defined everywhere on $E$. However, extra caution is needed for the line integral on $C_R$, because the exterior from $E$ normal to $C_R$ orients the curve \emph{clockwise}! Therefore we get
%
$$\oint_C \textbf{X} \cdot \textbf{ds} - \oint_{C_R} \textbf{X} \cdot \textbf{ds} = \int_E \divg \textbf{X} \, dA = 0 \Longrightarrow \oint_C \textbf{X} \cdot \textbf{ds} = \oint_{C_R} \textbf{X} \cdot \textbf{ds}\;.$$
%
The last integral is easy to compute because $\textbf{X}$ is a constant multiple of a unit normal:
%
$$\oint_{C_R} \textbf{X} \cdot \textbf{ds} = \int_{C_R} aR^{-1} ds = \int_{0}^{2\pi} aR^{-1} \, R \, ds = 2\pi a \; .$$
%
Therefore the flux across any curve that encloses the origin is $2\pi a$.

Similarly
%
%
$$\oint_C \textbf{X} \cdot \textbf{dr} - \oint_{C_R} \textbf{X} \cdot \textbf{dr} = \int_E \curl_{\bm{k}} \textbf{X} \, dA = 0 \Longrightarrow \oint_C \textbf{X} \cdot \textbf{dr} = \oint_{C_R} \textbf{X} \cdot \textbf{dr} = 0\;,$$
%
because $\textbf{X}$ is normal to $C_R$. Hence the circulation along any curve is still 0, which is not really that surprising given that any field of the form $f(r)\textbf{e}_r$ is a gradient field.
\end{frame}












