\begin{frame}
  \frametitle{Surfaces}

 \underline{Surface}:  A two dimensional object in space.

\medskip

 Locally: each point $P$ of $S$ has a patch around it that (shape-wise) looks like an open patch in the plane.

\underline{Examples}:
 
\begin{overlayarea}{\textheight}{7cm}
 \only<2>{
Graph surfaces:

Let $D \subset \RR^2$ be an open set and $f \colon D \to \RR$ a continuous function. Then
%
$$S=\{ (x,y,z) \; | \; (x,y) \in D, z=f(x,y)\}$$
%
is a surface: for every point $(x_0,y_0,z_0)$ on $S$, there exists an open disk $B$ centered at $(x_0,y_0)$ and included in $D$. Then the patch
%
$$E= \{ (x,y,z) \; | \; (x,y) \in B, z=f(x,y)\}$$
%
is an open patch on $S$ around $P$ and $f$ identifies $E$ with $B$.}

\only<3>{
Planes:

$S$: plane through $P_0$ and parallel to directions $\textbf{u}$ and $\textbf{v}$
%
$$\textbf{r} = \textbf{r}_0 + s \textbf{u} + t \textbf{v}\; ,$$
%
Function $\textbf{F} \colon \RR^2 \to \RR^3$,
%
$$\textbf{F}(s,t) = \textbf{r}_0 + s \textbf{u} + t \textbf{v}$$
%
globally identifies $S$ with the plane $\RR^2$.
%
\begin{itemize}
  \item  $S \ni P$ $\Longleftrightarrow$  pair of coordinates $(s,t)$;
  \item The vectors $\textbf{F}_s = \textbf{u}$ and $\textbf{F}_t = \textbf{v}$ are non-collinear.
\end{itemize}
}
\end{overlayarea}
\end{frame}


\begin{frame}
  \frametitle{Local parametrizations}

  A \emph{(differentiable) local parametrization}\index{parametrization!surface}:
   function $\textbf{F} \colon D \to \RR^3$,
   %
   \begin{itemize}
     \item defined on an open subset $D \subset \RR^2$;
     \item given by
%
$$\textbf{F}(s,t) = ( f_1(s,t), f_2(s,t), f_3(s,t) )$$
   \end{itemize}
%
such that
%
\begin{itemize}
  \item $\textbf{F}$ is 1-1: each point in the image has a unique pair of coordinates;
  %
  \item The components $f_1, f_2, f_3$ are differentiable functions from $D$ to $\RR$.
  %
  \item The vectors $\textbf{F}_s$ and $\textbf{F}_t$ are non-collinear.
\end{itemize}

\medskip

Image of $\textbf{F}$: surface in $\RR^3$.
\end{frame}

\begin{frame}
  \frametitle{Example}

  Let $R$ be fixed and
  $\textbf{F} \colon (0,\pi)\times (0,2\pi) \to \RR^3$ be given by
%
$$\textbf{F}(s,t) = \langle R\sin{s}\cos{t}, R\sin{s}\sin{t}, R\cos{s}\rangle \; .$$
%
Image of $\textbf{F}$: \pause open subset on the sphere $S_R(O)$.\pause

\begin{itemize}
  \item  $\textbf{F}$ is defined on the open subset \pause $D= (0,\pi) \times (0,2\pi) \subset \RR^2$;\pause
  \item $\textbf{F}$ is $1-1$: \pause $\textbf{F}(s_1,t_1) = \textbf{F}(s_2,t_2) \Longleftrightarrow (s_1,t_1) = (s_2,t_2)$;\pause
  \item Components of $\textbf{F}$ are differentiable:\pause
  %
  $$f_1(s,t) = R\sin{s}\cos{t}, \quad
  f_2(s,t) = R\sin{s}\sin{t}, \quad
  f_3(s,t) = R\cos{s}\; .$$
  %
  \item \pause $\textbf{F}_s$ and $\textbf{F}_t$ are non-collinear:\pause
  %
  \begin{align*}
    \textbf{F}_s & = \langle R\cos{s}\cos{t}, R\cos{s}\sin{t}, -R\sin{s} \rangle \\
    \textbf{F}_t & = \langle -R\sin{s}\sin{t}, R\sin{s}\cos{t}, 0 \rangle
  \end{align*}
%
$$|\textbf{F}_s \times \textbf{F}_t| = R^2\sin{s} \neq 0 $$
\end{itemize}

\end{frame}


\begin{frame}
  \frametitle{Parametrized surfaces}

  A subset $S$ in space is a \emph{smooth surface}\index{surface!smooth} if
%
\begin{itemize}
  \item For every point $P$ of $S$, there is a smooth local parametrization of an open patch on $S$ around $P$;
  %
  \item When images of two local parametrizations overlap on $S$, the two local parametrizations should be compatible.
\end{itemize}

\noindent \underline{Examples}:
\begin{itemize}
  \item \pause The plane $ax+by+cz=d$ is a surface;
  %
  \item \pause The sphere $x^2+y^2+z^2=R^2$ is a surface.
\end{itemize}

\pause \underline{Non-example}:

The cone $z^2= x^2+y^2$ is not globally a surface\pause , because there is no patch around $(0,0,0)$ that looks like an open set in the plane.
\end{frame}

\begin{frame}
  \frametitle{Level Surfaces as Parametrized Surfaces}

Let $f \colon \RR^3 \to R$ be a differentiable function.
\begin{itemize}
  \item $P$ is a \emph{critical point} if $(\nabla f)(P) = 0$;
  \item $c \in \RR$ is a \emph{critical value} if
%
$$f^{-1}(c) = \{ (x,y,z) \; | \; f(x,y,z) = c\}$$
%
contains a critical point.
\item If $c$ is not a critical value then it is a \emph{regular} value.
\end{itemize}

Implicit Function Theorem $\Longrightarrow$

If $c$ is a regular value taken by $f$, then $f^{-1}(c)$ is a smooth surface.

\underline{Examples}:
 \begin{overlayarea}{\textheight}{5cm}
 \only<2>{
 For $f(x,y,z) = ax+by+cz$ with $a$, $b$, $c$ not all zero,
 \begin{itemize}
 \item All points $P$ are regular points
   \item All values $d$ are regular;
   \item $f^{-1}(d)$ is a smooth surface: the plane $ax+by+cz=d$.
 \end{itemize}
 }

 \only<3>{
 For $f(x,y,z) = x^2+y^2-z^2$,
 \begin{itemize}
   \item $(0,0,0)$ is the only critical point $\Longrightarrow$ $0=f(0,0,0)$ is the only critical value.
   \item $H=f^{-1}(1)$ is a smooth surface: $x^2+y^2-z^2=1$, a hyperboloid with one sheet.
 \end{itemize}
 }
 \end{overlayarea}
\end{frame}

\begin{frame}
  \frametitle{Surfaces of Revolution}

\begin{itemize}
  \item $C$ be a curve in the $(x,z)-$plane;
  \item $S$ be the surface obtained by revolving $C$ around the $z-$axis.
\end{itemize}

\begin{center}
  Parametrization of $C$ $\Longrightarrow$ parametrization of $S$
\end{center}

\pause
$I \ni u\to (f(u),g(u))$: smooth, regular parametrization of $C$

\pause
$\textbf{F} \colon I \times (0,2\pi) \to \RR^3$ given by
%
$$\textbf{F}(u,v) = (f(u)\cos{v}, f(u)\sin{v}, g(u))$$
%
parametrizes the part of $S$ not in the $(x,z)-$plane. \pause
%
$$|\textbf{F}_u \times \textbf{F}_v| = |f|\sqrt{(f')^2+(g')^2}\; ,$$
%
hence $\textbf{F}$ is a smooth parametrization if $f(u) \neq 0$ for all $u$.

\pause
Geometrically: the curve $C$ shouldn't intersect the $z-$axis.
\end{frame}

\begin{frame}
  \frametitle{Example: Torus}

  Let $C$ be the circle
  \begin{itemize}
    \item in the $(x,z)-$plane,
    \item of radius $r$,
    \item with center at $(R,0,0)$  such that $R > r$.
  \end{itemize}

  Parametrization of $C$: \pause $u \to (R+r\cos{u}, r\sin{u})$; $0 \leqslant u \leqslant 2\pi$.

  Corresponding surface of revolution: $S$\pause , \emph{torus} (the surface of a doughnut)

  Parametrization of an open patch on the torus:\pause
%
$$\textbf{F}(u,v) = ((R+r\cos{u})\cos{v}, (R+r\cos{u})\sin{v}, r\sin{u})\; .$$
%
\pause To make $\textbf{F}$ to be 1-1 we restrict the domain to $(0,2\pi) \times (0,2\pi)$.

The image of $F$ is \pause $S$ minus two circles:
\begin{itemize}
  \item the vertical circle $x^2+z^2=1, y=0$ ($u \neq 0$)
  \item the horizontal circle $x^2+y^2 = (R+r)^2, z=0$ ($v \neq 0$).
\end{itemize}

\end{frame}

\begin{frame}
  \frametitle{Tangent Plane}

  Let $\textbf{F} \colon D \to \RR^3$, $\textbf{F}(u,v) = (f(u,v), g(u,v), h(u,v))$ be a local parametrization of a surface $S$,

  $P = \textbf{F}(u_0,v_0)$: point on $S$.

  The \emph{tangent plane} to $S$ at $P$:
  \begin{itemize}
    \item passes through $P$;
    \item contains vectors tangent at $P$ to curves on $S$ passing through $P$.
  \end{itemize}
\begin{overlayarea}{\textheight}{5cm}
\only<2>{
  We get curves on $S$ from curves in $D$, using $\textbf{F}$:
%
$$\gamma(t) = \textbf{F}(u(t), v(t)) = (f(u(t),v(t)), g(u(t),v(t)), h(u(t),v(t)))$$
%
$\gamma(0) = P$ $\Longrightarrow$ $u(0)=u_0$ and $v(0)=v_0$
%
$$\bm{\gamma}'(0) = u'(0)\textbf{F}_u(u_0,v_0) + v'(0) \textbf{F}_v(u_0,v_0)$$
%
Tangent vector at $P$: linear combination of $\textbf{F}_u$ and $\textbf{F}_v$.
}
\only<3>{
Tangent plane: parallel to $\textbf{F}_u(u_0,v_0)$ and $\textbf{F}_v(u_0,v_0)$.

Normal to the plane tangent to $S$ at $P=\textbf{F}(u_0,v_0)$:
%
$$\textbf{N}(u_0,v_0) = \textbf{F}_u(u_0,v_0) \times \textbf{F}_v(u_0,v_0)$$
%
Equation for the tangent plane:
%
$$(\textbf{r}-\textbf{F}(u_0,v_0)) \cdot \textbf{N}(u_0,v_0) = 0 \; .$$
%
Vectorial parametric equation for the line normal to $S$ at $P$:
%
$$\textbf{r}(t) = \textbf{F}(u_0,v_0) + t \textbf{N}(u_0,v_0)\; .$$
}
\end{overlayarea}

\end{frame}

\begin{frame}
  \frametitle{Example}

  Let $\textbf{F}(u,v) = (u\cos{v}, u\sin{v}, e^{2u-4})$ and $P=\textbf{F}(2, \pi/3)$. \pause
%
\begin{align*}
  \textbf{F}_u(u,v) =  \langle \cos{v}, \sin{v}, 2e^{2u-4}\rangle & \Longrightarrow \textbf{F}_u(1,\pi/3) = \langle \frac{1}{2}, \frac{\sqrt{3}}{2}, 2\rangle \\
  %
  \textbf{F}_v(u,v) = \pause \langle -u\sin{v}, u\cos{v}, 0 \rangle & \Longrightarrow \textbf{F}_v(1,\pi/3) = \langle -\sqrt{3}, 1, 0 \rangle
\end{align*}
%
\pause
%
$$\textbf{N}(2,\pi/3) = \pause \textbf{F}_u(2,\pi/3) \times \textbf{F}_v(2,\pi/3)  \Longrightarrow \textbf{N}(2,\pi/3) = -2 \, \textbf{i} - 2\sqrt{3}\, \textbf{j} + 2\,\textbf{k}$$
%
and since $\textbf{F}(2,\pi/3) = \langle 1, \sqrt{3},1\rangle$, an equation for the tangent plane is\pause
%
$$(\langle x,y,z\rangle - \langle 1, \sqrt{3},1\rangle) \cdot \langle -2, -2\sqrt{3}, 2\rangle = 0$$
%
$$-2(x-1) -2\sqrt{3}(y-\sqrt{3}) +2(z-1) = 0 \Longleftrightarrow x + \sqrt{3} y -z = 3\; .$$
\pause
Parametric equations for the normal line are\pause
%
$$x= 1-2t, \quad y=\sqrt{3} -2\sqrt{3} t,\quad z = 1+2t\; .$$
\end{frame}

\begin{frame}
  \frametitle{Tangent Plane to Level Surface}

Given:
  \begin{itemize}
    \item a differentiable function $F$,
    \item a level surface $S=\{ Q\; | \; F(Q) = k\}$,
    \item point $P$ on this level set
  \end{itemize}

\underline{Question}:

\begin{center}
  What should be the tangent plane/line to $S$ at $P$?
\end{center}


\pause
\underline{Intuitive answer}:

Analogy to graph surfaces: the tangent plane should contain tangent vectors at $P$ to curves through $P$ and contained in $S$.
\end{frame}

\begin{frame}
  \frametitle{Regular Points}

 Let $\textbf{r}=\textbf{r}(t)$ be a curve that
%
\begin{itemize}
  \item passes through $P$ for $t=0$: $\textbf{r}(0) = \textbf{r}_0$, the position vector of $P$;
  %
  \item is contained in the level set $S$: $F(\textbf{r}(t)) \equiv k$ for all $t$.
\end{itemize}

Then
%
$$ \frac{d(F(\textbf{r}(t))}{dt} \equiv 0 \Longrightarrow (\nabla F)(\textbf{r}(t)) \cdot \textbf{r}'(t) \equiv 0 \Longrightarrow (\nabla F)_P \cdot \textbf{r}'(0) =0$$

\pause
If $(\nabla F)_P \neq \textbf{0}$, then $\textbf{n}= (\nabla F)_P$ is normal to the tangent plane/line.

\pause
\medskip
\underline{Definition}: A point $P$ is called
\begin{itemize}
  \item a \emph{regular} point for $F$ if $(\nabla F)_P \neq \textbf{0}$;
  \item a \emph{critical} point for $F$ if $(\nabla F)_P = \textbf{0}$.
\end{itemize}

\pause
\medskip
\underline{Fact}: At a regular point, the gradient is orthogonal to the level set through that point.
\end{frame}

\begin{frame}
  \frametitle{Equation of Tangent Plane}

Equation of the tangent plane to the level surface of $F$ that contains the regular point $P$:
%
$$(\nabla F)_P \cdot (\textbf{r}-\textbf{r}_0) = 0\; .$$

\pause
In a rectangular coordinate system $Oxyz$ we get:

\begin{itemize}
  \item Function: $F = F(x,y,z)$;
  \item Point: $P=P(x_0,y_0,z_0)$, hence $\textbf{r}_0 = \langle x_0, y_0 , z_0 \rangle$;
  \item Level surface $S$ through $P$: $F(x,y,z) = k$, where $k = F(x_0,y_0,z_0)$;
  \item \pause Gradient of $F$ at $P$:
  %
  $$(\nabla F)_P = \langle \frac{\partial F}{\partial x}(x_0,y_0,z_0), \frac{\partial F}{\partial y}(x_0,y_0,z_0), \frac{\partial F}{\partial z}(x_0,y_0,z_0) \rangle$$
  %
  \item \pause Equation of tangent plane to $S$ at $P$:
  %
  $$\frac{\partial F}{\partial x}(x_0,y_0,z_0)(x-x_0) + \frac{\partial F}{\partial y}(x_0,y_0,z_0) (y-y_0) + \frac{\partial F}{\partial z}(x_0,y_0,z_0)(z-z_0) = 0$$
\end{itemize}
\end{frame}

\begin{frame}
  \frametitle{Example}
  If
  %
  $$F(x,y,z) = x^2 + y^2-z^2 \; ,$$
  %
  then the level surface through $P(1,2,3)$ is the surface $S$ with equation
%
$$x^2+y^2 -z^2 = F(1,2,3) \Longleftrightarrow x^2+y^2 -z^2 = -4 \; .$$
%
\pause The gradient of $F$ at $P$ is \pause
%
$$(\nabla F)_P = \left. \langle 2x, 2y, -2z \rangle \right|_{1,2,3} = \langle 2,4,-6 \rangle \; ,$$
%
\pause hence $P$ is a regular point.

\pause The equation of the tangent plane at $P(1,2,3)$ is\pause
%
$$\langle 2,4,-6 \rangle \cdot \langle x-1, y-2, z-3\rangle = 0 \Longrightarrow$$
%
$$2(x-1) + 4(y-2) -6(z-3) = 0  \Longleftrightarrow x+2y-3z=-4$$
\end{frame}

\begin{frame}
  \frametitle{Tangent Planes to Quadric Surfaces}

  Let $F(x,y,z) = Ax^2+By^2+Cz^2$ such that not all of $A$, $B$, $C$ are zero.

  \medskip
  \underline{Fact}: If $Q(a,b,c)$ is a critical point, then
  %
  $$Aa=Bb=Cc=0 \Longrightarrow F(a,b,c) = 0$$

\pause
\underline{Consequence}: If $k \neq 0$, then every point on the surface $S$: $F(x,y,z) = k$ is a regular point.

\pause
\medskip
Suppose $k\neq 0$. Let $P(x_0,y_0,z_0)$ be a point on $S$. Then
\begin{itemize}
  \item $P$ is necessarily a regular point of $F$;
  \item $Ax_0^2+By_0^2+Cz_0^2=k$;
  \item \pause The gradient of $F$ at $P$ is
%
$$(\nabla F)(x_0,y_0,z_0) = \langle 2Ax_0, 2By_0,2Cz_0 \rangle\; .$$
\item \pause The equation of the tangent plane is
%
$$2Ax_0(x-x_0) + 2By_0(y-y_0) + 2Cz_0 (z-z_0) = 0 \Longrightarrow Ax_0x + B y_0y + Cz_0z = k$$
\end{itemize}
\end{frame}

\begin{frame}
  \frametitle{Graph Surface vs. Level Surface}

A surface $S$ can be given in two different ways:

\begin{itemize}
  \item \pause Explicit form, as a \emph{graph surface}:
  %
  $$z=f(x,y)$$
%
  \item \pause Implicit form, as a \emph{level surface}:
  %
  $$F(x,y,z) = k$$
  %
\end{itemize}

\pause
A graph surface $z=f(x,y)$ can always be represented as a level surface:
%
$$z=f(x,y) \Longleftrightarrow F(x,y,z) =0 \quad \text{ for } \quad F(x,y,z) = z-f(x,y)\; .$$

\pause
\underline{Question}: Can a given level surface be represented as a graph surface?
\end{frame}

\begin{frame}

\underline{Question}: Can a given level surface be represented as a graph surface?

\pause
\medskip

\underline{Remark}: A graph surface passes the \textbf{Vertical Line Test}.

\pause
\medskip
\underline{Answers}:
\begin{itemize}
  \item \pause Bad news: In general, \emph{globally}, NO.

  \noindent Think $x^2+y^2+z^2= 1$.

  One can't solve for $z$ globally; there is the $\pm\sqrt{1-x^2-y^2}$ issue;
  \item \pause Good news: In a lot of situations, \emph{locally}, YES.

  \noindent Around $P(0,0,1)$, the surface is the graph surface of $z = f(x,y)$ with $f(x,y) =\sqrt{1-x^2-y^2}$.
\end{itemize}

\pause
How about the level surface

$$x^3+y^3+z^3+6xyz =-4$$

around the point $(1,2,-1)$?
\end{frame}

\begin{frame}
\frametitle{Implicit Functions}

Given:
\begin{itemize}
  \item a function $F$,
  \item a point $P(x_0,y_0,z_0)$ in the domain of $F$,
  \item the level surface $S$ of $F$ through $P$:
  %
  $$F(x,y,z) = k \qquad \text{ with } k=F(x_0,y_0,z_0)$$
\end{itemize}

\pause
$S$ is a graph surface around $P$ if there is a function $z=f(x,y)$ such that:
%
\begin{itemize}
  \item $f$ is defined on an open disk $D$ around $(x_0,y_0)$;
  \item $f(x_0,y_0) = z_0$;
  \item $F(x,y,f(x,y)) = k$ for all $(x,y)$ in the disk $D$.
\end{itemize}

\pause
The equation $F(x,y,z) = k$ \emph{implicitly} defines the function $f$ satisfying
%
\begin{itemize}
  \item $F(x,y,f(x,y)) = k$ for all $(x,y)$ in the disk $D$.
  \item $f(x_0,y_0) = z_0$.
\end{itemize}
\end{frame}

\begin{frame}
  \frametitle{Examples}
The equation $x^2+y^2+z^2 = 1$ implicitly defines $z=\sqrt{1-x^2-y^2}$ as the unique function $z=f(x,y)$ such that
%
\begin{itemize}
  \item $x^2+ y^2+(f(x,y))^2 = 1$ for all $(x,y)$ in a disk around $(0,0)$;
  \item $f(0,0) = 1$.
\end{itemize}

\pause
The equation $x^2+y^2+z^2 = 1$ implicitly defines $z=-\sqrt{1-x^2-y^2}$ as the unique function $z=f(x,y)$ such that
%
\begin{itemize}
  \item $x^2+ y^2+(f(x,y))^2 = 1$ for all $(x,y)$ in a disk around $(0,0)$;
  \item $f(0,0) = -1$.
\end{itemize}

\pause
In general, the implicit function question is quite tricky!

\pause
But for differentiable functions things are somehow simpler.
\end{frame}

\begin{frame}
  \frametitle{Implicit Function Theorem}

    \begin{itemize}
      \item $F$: function
      \begin{itemize}
        \item defined for all points in a ball around the point $P(x_0,y_0,z_0)$,
        \item with continuous partial derivatives.
      \end{itemize}
      %
      \item $S$: the level surface of $F$ through $P$, for the level $k = F(x_0,y_0,z_0)$.
    \end{itemize}

\pause If $F_z(x_0,y_0,z_0) \neq 0$, then there exists a function $z=f(x,y)$
\begin{itemize}
  \item defined on an open disk $D$ centered at $(x_0,y_0)$,
  %
  \item such that $F(x,y,f(x,y)) = k$ for all $(x,y)$ in $D$,
  %
  \item and such that $f(x_0,y_0) = z_0$.
\end{itemize}

\pause Moreover:
\begin{itemize}
  \item For each fixed disk $D$ the function $z=f(x,y)$ is unique.
  \item The function $z=f(x,y)$ has partial derivatives and
%
$$z_x(x,y) = -\frac{F_x(x,y,z)}{F_z(x,y,z)}\quad \text{and}\quad z_y(x,y) = -\frac{F_y(x,y,z)}{F_z(x,y,z)}$$
\end{itemize}
\end{frame}

\begin{frame}
  \frametitle{Consequences}

If $F_z(x_0,y_0,z_0) \neq 0$, then the level surface $S$
%
$$F(x,y,z) = F(x_0,y_0,z_0)$$
%
can be locally represented as a graph surface
%
$$z=f(x,y)$$
%
around the point $(x_0,y_0,z_0)$. \pause  Nothing special about $z$: if
%
$$(\nabla F)(x_0,y_0,z_0) = \langle F_x(x_0,y_0,z_0) , F_y(x_0,y_0,z_0), F_z(x_0,y_0,z_0)\rangle  \neq \textbf{0}\; ,$$
%
then the level surface $S$ can be represented as a graph surface
\begin{itemize}
  \item $z=f(x,y)$, if $F_x(x_0,y_0,z_0) \neq 0$;
  \item $y=g(x,z)$, if $F_y(x_0,y_0,z_0) \neq 0$;
  \item $x=h(y,z)$, if $F_z(x_0,y_0,z_0) \neq 0$;
\end{itemize}
\end{frame}

\begin{frame}
    \frametitle{Implicit Differentiation}

If
%
$$F(x,y,f(x,y)) = 0$$
%
for all $(x,y)$ in an open disk, then, by differentiating with respect to $x$ and using the chain rule, we get:
%
$$F_x(x,y,z) + F_z(x,y,z) f_x(x,y) = 0$$
%
Since $F_z(x,y,z)$ is continuous
%
$$F_z(x_0,y_0,z_0) \neq 0 \Longrightarrow F_z(x,y,z) \neq 0$$
%
for all $(x,y,z)$ in an open ball around $(x_0,y_0,z_0)$. For such points we get
%
$$\frac{\partial z}{\partial x} = f_x(x,y) = -\frac{F_x(x,y,z)}{F_z(x,y,z)} \; ,$$
%
and similarly for the partial derivative with respect to $y$.

\end{frame}



\begin{frame}
  \frametitle{Example}

  Show that the equation $x^3+y^3+z^3+6xyz =-4$ implicitly defines a function $z=f(x,y)$ around the point $(1,2)$, such that $f(1,2) = -1$. Compute the partial derivatives of this function at $(1,2)$.

\pause
\medskip

Let $F(x,y,z) = x^3+y^3+z^3+6xyz$. Then
%
$$F_z(x,y,z) = 3z^2 + 6xy  \Longrightarrow F_z(1,2,-1) = 15$$
%
\pause Since $F_z(1,2,-1) \neq 0$, it follows that $F(x,y,z) = F(1,2,-1) = 1$ defines $z$ implicitly in terms of $(x,y)$ around $(1,2)$, and
%
$$z_x(x,y) = -\frac{F_x(x,y,z)}{F_z(x,y,z)} = -\frac{3x^2 + 6yz}{3z^2 + 6xy} \Longrightarrow z_x(1,2) = \frac{3}{5}\; .$$
%
\pause Similarly
%
$$z_y(x,y) = -\frac{F_y(x,y,z)}{F_z(x,y,z)} = -\frac{3y^2 + 6xz}{3z^2 + 6xy} \Longrightarrow z_y(1,2) = -\frac{2}{5}\; .$$

\end{frame}

\begin{frame}
  \frametitle{Tangent Plane}

  Equation of tangent plane to
  $$x^3+y^3+z^3+6xyz =-4$$
  %
  at $(1,2,-1)$.

  \begin{itemize}
    \item \pause As level surface of $F(x,y,z) = x^3+y^3+z^3+6xyz$:
    %
    $$F_x(1,2,-1) (x-1) + F_y(1,2,-1) (y-2) + F_z(1,2,-1) (z+1) = 0$$
    %
    $$-9(x-1) + 6(y-2) + 15 (z+1) = 0$$

    \item \pause As graph of $z=f(x,y)$:
    %
    $$z = f(1,2) + f_x(1,2) (x-1) + f_y(1,2) (y-2)$$
    %
    $$z = -1 + \frac{3}{5} (x-1) - \frac{2}{5} (y-2)$$
  \end{itemize}
\end{frame}

\begin{frame}
  \frametitle{Intuitive Argument}

  Actual equation:
  %
  $$F(x,y,z) = F(x_0,y_0,z_0) \Longrightarrow z= f(x,y)$$

  \pause Linearized equation:
  %
  $$L_{F,(x_0,y_0,z_0)}(x,y,z) = F(x_0,y_0,z_0) \Longrightarrow z = L_{f,(x_0,y_0)}(x,y)$$

%
$$L_{F,(x_0,y_0,z_0)}(x,y,z) = F(x_0,y_0,z_0) \Longleftrightarrow$$
%
$$F_x(x_0,y_0,z_0) (x-x_0) + F_y(x_0,y_0,z_0) (y-y_0)+ F_z(x_0,y_0,z_0) (z-z_0)=0$$
%
\pause $F_z(x_0,y_0,z_0) \neq 0$ $\Longrightarrow$ \emph{can} solve $L_{F,(x_0,y_0,z_0)}(x,y,z) = F(x_0,y_0,z_0)$ for $z$:
%
$$z = z_0 - \frac{F_x(x_0,y_0,z_0)}{F_z(x_0,y_0,z_0)}(x-x_0) - \frac{F_y(x_0,y_0,z_0)}{F_z(x_0,y_0,z_0)}(y-y_0) = L_{f,(x_0,y_0)}(x,y)$$

Then
%
$$z_x(x_0,y_0) = - \frac{F_x(x_0,y_0,z_0)}{F_z(x_0,y_0,z_0)} \quad \text{ and } \quad z_y(x_0,y_0) =- \frac{F_y(x_0,y_0,z_0)}{F_z(x_0,y_0,z_0)} \;.$$
\end{frame}
 
 \begin{frame}
  \frametitle{Motivation}

  With $K$ units of capital and $L$ units of labor, the production is given by the following Cobb-Douglas function:
%
$$P(K,L) = L^{1/4} K^{3/4}\; .$$
%
\pause With unlimited resources $(K,L)$ \pause any level of production can be achieved. \pause

\underline{Constraint}: \pause Capital and labor are not free, and the budget is limited.\pause

Suppose
\begin{itemize}
  \item one  unit of $L$ costs 2 units of money;
  \item and one unit of $K$ costs 3 units of money.
\end{itemize}

\pause Cost function:\pause
%
$$C(K,L) = 3K +2L\; .$$
%
\pause \underline{Question}: What is the maximal production that can be achieved with a budget of 16 units of money?
\end{frame}

\begin{frame}
  \frametitle{Economics Method}

  At the point $(K_0,L_0)$ of maximal production:
 %
$$\Delta P \simeq P_K(K_0,L_0) \Delta K + P_L(K_0,L_0) \Delta L$$
%
\pause The last unit of money spent on capital
\begin{itemize}
  \item \pause increased the amount of capital by $\Delta K = 1/3$ units;
  \item \pause increased the production by
%
$$\Delta P \simeq P_K(K_0,L_0) \Delta K \simeq \frac{P_K(K_0,L_0)}{3}\; .$$
\end{itemize}

\pause The last unit of money spent on labor:
\begin{itemize}
  \item \pause increased the amount of labor by $\Delta L = 1/2$ units;
  \item \pause increased the production by
%
$$\Delta P \simeq P_L(K_0,L_0) \Delta L \simeq \frac{P_L(K_0,L_0)}{2}\; .$$
%
\end{itemize}

\pause If these two ratios were not equal, then \pause we can achieve a higher level of production by redistributing the last unit of money.\pause Therefore \pause
%
$$\frac{P_K(K_0,L_0)}{3} = \frac{P_L(K_0,L_0)}{2}\;  .$$
\end{frame}

\begin{frame}
  \frametitle{Geometric Method}

  At the point $(K_0,L_0)$ of maximal production:\pause

\begin{itemize}
  \item Level curves $P(K,L) = P(K_0,L_0)$ and $C(K,L) = C(K_0,L_0)$ are tangent;\pause
  \item Gradients of $P$ and $C$ at $(K_0,L_0)$ are  collinear.
\end{itemize}

\pause There exists a scalar $\lambda_0$ such that
%
\begin{align*}
  (\nabla P)_{(K_0,L_0)} = & \lambda_0 (\nabla C)_{(K_0,L_0)} \pause \Longrightarrow
\left\{ \begin{array}{ll}
  P_K(K_0,L_0) = & \lambda_0 C_K(K_0,L_0) \\
  %
  P_L(K_0,L_0) = & \lambda_0 C_L(K_0,L_0)
\end{array}
\right. \\
\Longrightarrow & \left\{ \begin{array}{ll}
  P_K(K_0,L_0) = & 3 \lambda_0 \\
  %
  P_L(K_0,L_0) = & 2 \lambda_0
\end{array}
\right. \; \Longrightarrow \frac{P_K(K_0,L_0)}{3}=\frac{P_L(K_0,L_0)}{2}\; .
\end{align*}
%
\end{frame}

\begin{frame}
  The triple $(K_0,L_0,\lambda_0)$ is a solution of the system
%
$$\left\{ \begin{array}{ll}
  P_K(K,L) = & 3\lambda\\
  %
  P_L(K,L) = & 2\lambda  \\
  %
  C(K,L) = & 16
\end{array} \right.
%
\Longleftrightarrow
%
\left\{ \begin{array}{ll}
  \frac{3}{4} L^{1/4}K^{-1/4} = & 3\lambda\\
  %
  & \\
  %
  \frac{1}{4} L^{-3/4}K^{3/4} = & 2\lambda  \\
  %
  & \\
  %
  3K+2L = & 16
\end{array} \right.
$$

\pause
The system is not linear, and there is no general rule for solving a nonlinear system. \pause

But in this particular case:
%
$$\frac{1}{4} L^{1/4}K^{-1/4} = \lambda = \frac{1}{8} L^{-3/4}K^{3/4} \Longrightarrow \frac{K}{L}=2 \Longrightarrow K = 2L\; .$$
%
From $K=2L$ and $3K+2L=16$ we get \pause $L=2$ and $K=4$.\pause

How do we know that $K=4$ and $L=2$ corresponds to a maximum? \pause

Geometric argument using the direction of the gradient $(\nabla P)(4,2)$.

\end{frame}

\begin{frame}
  \frametitle{Method of Lagrange Multipliers}

  General problem:

We want to optimize an objective function $f(x,y)$

subject to some constraint $g(x,y)=0$.

\begin{overlayarea}{\textheight}{6cm}
\only<2-4>{
At an optimal point $(x_0,y_0)$ the gradients of $f$ and $g$ are collinear:
%
$$(\nabla f)(x_0,y_0) = \lambda (\nabla g)(x_0,y_0)$$}
%
\only<3-4>{ Together with the constraint $g(x,y) =0$, we get the system
%
$$\left\{\begin{array}{ll}
 f_x(x,y) = & \lambda g_x(x,y) \\
%
f_y(x,y) = & \lambda g_y(x,y) \\
%
g(x,y) = & 0
\end{array}
 \right.$$}
 %
\only<4>{Consider the \emph{Lagrange function}
%
$$F(x,y,\lambda) = f(x,y) - \lambda g(x,y) \; ,$$
%

New variable $\lambda$: \emph{Lagrange multiplier}.
 }

\only<5->{
$$F(x,y,\lambda) = f(x,y) - \lambda g(x,y) \; ,$$
%
$$
\text{Points } (x,y) \text{ where } f \text{ attains an extreme, subject to } g(x,y) =0: $$
%
$$\text{Critical points of the function } F(x,y,\lambda) = f(x,y) - \lambda g(x,y)\; .$$}

\only<6->{Remarks:
 \begin{itemize}
   \item  not all critical points of $F$ give points of extreme for $f$;
   \item to decide whether a critical point gives a min, max or neither we use:
    \begin{itemize}
      \item (harder) a special second derivative test;
      \item (easier) the direction of the gradient of $f$.
    \end{itemize}
 \end{itemize}
 }
 \end{overlayarea}

 \end{frame}

\begin{frame}
  \frametitle{Example}

Find the maximum and the minimum values of $f(x,y) = xy$ on the region $D = \{ (x,y) \; | \; |x|+|y| \leqslant 2 \}$.
\pause

Region: \pause closed square of vertices $(2,0)$, $(0,2)$, $(-2,0)$, and $(0,-2)$.

\pause

The function is continuous and the domain is bounded and closed.

Extreme Value Theorem $\Longrightarrow$ $f$ has global minimum and maximum points.

\pause
\underline{Strategy}:
%
\begin{itemize}
  \item Find critical points in the interior of the disk;
  \item Find extreme points on the boundary of the disk;
  \item Compare the values.
\end{itemize}

\pause
Since $f$ is differentiable everywhere, the interior extreme points are among the solutions of the system
%
$$\left\{
\begin{array}{ll}
  f_x(x,y) = &  0 \\
  %
  f_y(x,y) = &  0
\end{array}
\right.
%
\Longleftrightarrow
%
\left\{
\begin{array}{ll}
  y = &  0 \\
  %
  x= &  0
\end{array}
\right.
$$
\end{frame}

\begin{frame}
  Find the maximum and the minimum values of $f(x,y) = xy$ on the region $D = \{ (x,y) \; | \; |x|+|y| \leqslant 2 \}$.

  Extreme points on the boundary: check each of the four sides.

  For the segment joining $(2,0)$ with $(0,2)$ we get:
%
\begin{align*}
  \text{Find min/max of } & f(x,y) = xy \\
  %
  \text{Subject to } & g(x,y) = x+y-2 =0
\end{align*}


\pause
The Lagrange function is
%
$$F(x,y,\lambda) = f(x,y) - \lambda g(x,y) = xy - \lambda(x+y-2)$$

\pause
The critical points of $F$ are the solutions of the system
%
$$\left\{
\begin{array}{ll}
  F_x(x,y,\lambda) & = 0 \\
  F_y(x,y,\lambda) & =0\\
  F_\lambda(x,y,\lambda) & =0
\end{array}
\right. \Longleftrightarrow
\left\{
\begin{array}{ll}
  y-\lambda & = 0 \\
  x-\lambda & =0\\
  x+y=2 & =0
\end{array}
\right. \Longleftrightarrow
\left\{
\begin{array}{ll}
  x & = 1 \\
  y & = 1\\
  \lambda & =1
\end{array}
\right.
 $$
\end{frame}

\begin{frame}
    \frametitle{Gradient Analysis}

    \begin{itemize}
      \item \pause $(\nabla f)_{(1,1)} = \langle y, x \rangle|_{x=1,y=1} = \langle 1,1\rangle$.
      \item \pause If we move along the direction of the gradient at $(1,1)$:
      \begin{itemize}
        \item the value of the objective would increase;
        \item the level curves of $f$ we cross no longer intersect the constraint
        \item those levels of $f$ are unattainable on the constraint set $x+y=2$.
      \end{itemize}
      \item \pause The point $(1,1)$ corresponds to a local maxim.
    \end{itemize}

\pause
Three more critical points on the boundary: $(-1,1)$, $(-1,-1)$, $(1,-1)$.

\pause
Compare the values at all points:
\begin{itemize}
  \item the global maximum is 1, attained at $(1,1)$ and $(-1,-1)$;
  \item the global minimum is -1, attained at $(1,-1)$ and $(-1,1)$;
  \item the critical point $(0,0)$ is a saddle point.
\end{itemize}

\end{frame}

\begin{frame}
  \frametitle{Multiple Constraints}

\begin{align*}
  \text{ Find } & \min/\max f(x,y,z) \\
  \text{ Subject to } & g(x,y,z) = 0 \\
                      & h(x,y,z) = 0
\end{align*}
%
\pause
Each constraint defines a surface $\Longrightarrow$ their intersection defines a curve.

\pause
Condition: $(\nabla g)_P$$(\nabla h)_P$ are non-collinear for each intersection point $P$

The level surface of $f$ through a point of extreme $P_0$ is tangent to the constraint curve, so $(\nabla f)(P_0)$ is perpendicular to the curve at $P_0$.

Constraint curve included in both surfaces $\Longrightarrow$

$(\nabla g)(P_0)$ and $(\nabla h)(P_0)$ are perpendicular to the curve $\Longrightarrow$

\pause
there exist constants $\lambda$ and $\mu$ such that
%
$$(\nabla f)(P_0) = \lambda (\nabla g)(P_0) + \mu (\nabla h)(P_0)\; .$$

\pause
The Lagrange function is in this case
%
$$F(x,y,z,\lambda,\mu) = f(x,y,z) - \lambda g(x,y,z) - \mu h(x,y,z)$$

\end{frame}

\begin{frame}
  \frametitle{Example}

  Find the extreme points of $x+2y$ on the intersection of the the cylinder $y^2+z^2=5$ and the plane $x+y+z=1$.

\begin{itemize}
  \item \pause Objective function: \pause $f(x,y,z) = x+2y$.
  %
  \item \pause Constraints: \pause $g(x,y,z) = y^2+z^2-5$ and $h(x,y,z) = x+y+z-1$.
  %
  \item \pause Lagrange function:\pause
  %
  $$F(x,y,z,\lambda,\mu) = x+2y - \lambda (y^2+z^2-5)- \mu (x+y+z-1)\; .$$
  %
  \item \pause Critical points of $F$: \pause $(1,\sqrt{5/2},-\sqrt{5/2})$ and $(1,-\sqrt{5/2},\sqrt{5/2})$
  %
\item \pause Values of objective function at these points:\pause
  %
  $$f(1,\sqrt{5/2},-\sqrt{5/2}) = 1+2\sqrt{5/2}, \quad  f(1,-\sqrt{5/2},\sqrt{5/2}) = 1-2\sqrt{5/2}\; .$$
  %
  \item \pause Constraint set is bounded and closed, function $f$ is continuous $\Longrightarrow$
       $f$ attains its extreme on the constraint $\Longrightarrow$\\
       $(1,-\sqrt{5/2},\sqrt{5/2})$ corresponds to an absolute minimum and $(1,\sqrt{5/2},-\sqrt{5/2})$ corresponds to an absolute maximum.
  %
  \item The minimum value is $f(1,-\sqrt{5/2},\sqrt{5/2}) = 1-2\sqrt{5/2}$ and the maximal value is $f(1,\sqrt{5/2},-\sqrt{5/2})=1+2\sqrt{5/2}$.
\end{itemize}

\end{frame}