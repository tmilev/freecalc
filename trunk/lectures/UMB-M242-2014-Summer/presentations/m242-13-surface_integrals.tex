\begin{frame}
  \frametitle{Motivation}

    \begin{itemize}
      \item $S$ be a surface in space;
      %
      $$dS = \text{element of surface area}$$
      \item \pause $f$ be a continuous function of $S$
      \begin{itemize}
        \item \pause with scalar values: For a point $P$ on $S$, let $\rho(P)$ be the density of mass with respect to surface area. Then $dm = \rho(P) dS$ is the element of mass, and the total mass is
%
$$M = \int\!\!\!\int_S dm = \int\!\!\!\int_S \rho(P) \, dS \; .$$
%
        \item \pause with vectorial values: let $\textbf{p}(P)$ be the pressure at $P$, which is the density of force with respect to surface area. Then the element of force is $d\textbf{F} = \textbf{p}(P) dS$ and the total force is
%
$$\textbf{F} = \int\!\!\!\int_S d\textbf{F} = \int\!\!\!\int_S \textbf{p}(P) \, dS \; .$$
      \end{itemize}
    \item \pause How do we compute surface integrals?
    \end{itemize}
\end{frame}

\begin{frame}
  \frametitle{Surface Area}

\begin{itemize}
  \item $\varphi \colon D \to \RR^3$ be a local parametrization;
  \item $(u_0,v_0)$ a point in the parameter space and $P = \varphi(u_0,v_0)$;
  \item $B = [u_0,u_0 + \Delta u] \times [v_0,v_0+\Delta v]$: small rectangle in the parameter space;
  \item $E=\varphi(B)$ the corresponding patch on the surface;
\end{itemize}

\pause
$E \simeq$ curvilinear box on $S$ with one vertex at $\bm{\varphi}(u_0,v_0)$ and directions
%
\begin{align*}
  \bm{\varphi}(u_0+\Delta u,v_0) - \bm{\varphi}(u_0,v_0) & \simeq  \bm{\varphi}_u (u_0,v_0) \Delta u \\
  %
  \bm{\varphi}(u_0,v_0+\Delta v) - \bm{\varphi}(u_0,v_0) & \simeq  \bm{\varphi}_v (u_0,v_0) \Delta v
\end{align*}
%
$$\text{area}(E) \simeq |\bm{\varphi}_u(u_0,v_0) \Delta u \times \bm{\varphi}_v(u_0,v_0)\Delta v| = |\bm{\varphi}_u(u_0,v_0) \times \bm{\varphi}_v(u_0,v_0)| \Delta u \Delta v\; .$$
%
$$dS = |\bm{\varphi}_u(u,v) \times \bm{\varphi}_v(u,v)| \, du\, dv\, .$$
%
\pause If $B$ is a compact (bounded and closed) subset of $D$, then $S=\varphi(B)$ is a compact surface in space and
%
$$\text{Area}(S) = \int\!\!\!\int_{S} \, 1\,dS = \int\!\!\!\int_B |\bm{\varphi}_u(u,v) \times \bm{\varphi}_v(u,v)| \, du\, dv\; .$$
%
\end{frame}

\begin{frame}
  \frametitle{Example}

  If $\varphi \colon D \to \RR^3$ is a graph surface $\varphi(u,v) = (u,v,f(u,v))$, then
%
\pause $$\bm{\varphi}_u(u,v) = \langle 1, 0 , f_u(u,v)\rangle$$
%
\pause $$\bm{\varphi}_v(u,v) = \langle 0,1,f_v(u,v) \rangle$$
%
\pause $$dS = |\bm{\varphi}_u \times \bm{\varphi}_v|\,du\,dv = \sqrt{1+(f_u(u,v))^2 + (f_v(u,v))^2} \, du\, dv$$
%
\pause Area of the surface $S = \varphi(B)$:
%
$$\text{Area}(S) = \int\!\!\!\int_B \sqrt{1+(f_u(u,v))^2 + (f_v(u,v))^2} \, du\, dv$$
%
\end{frame}

\begin{frame}
  \frametitle{Surfaces of Revolution}

\begin{itemize}
  \item $C$: parametrized curve in the $x>0$ half plane of the $xz-$plane;
  \item Parametrization: $x=f(u)$, $z=g(u)$, $a \leqslant u \leqslant b$;
  \item $S$ be the surface obtained by revolving the curve about the $z-$axis;
  \item \pause Parametrization:
  %
  $$x=f(u)\cos{v}, \quad y=f(u)\sin{v}, \quad z=g(u), \qquad a \leqslant u \leqslant b, 0 \leqslant v \leqslant 2\pi$$
\end{itemize}
%
\pause Then
%
$$dS = |\bm{\varphi}_u \times \bm{\varphi}_v| \, du\, dv = f(u) \sqrt{(f'(u)^2+(g'(u))^2} \; du\, dv$$
%
The area of the surface is
%
\begin{align*}
  \text{Area}(S) = &  \int\!\!\!\int_{[a,b]\times [0,2\pi]} f(u) \sqrt{(f'(u)^2+(g'(u))^2} \; du\, dv = \\
   = & 2\pi \int_{[a,b]} f(u) \sqrt{(f'(u)^2+(g'(u))^2} \; du = 2\pi \int_C x \,ds
\end{align*}

\end{frame}

\begin{frame}
  \frametitle{Pappus' First Centroid Theorem}

$$\text{Area}(S) = L(C)\, 2\pi \,\frac{1}{L(C)}\int_C x \,ds$$
\pause
\begin{theorem}
   The area of a surface of revolution is the product between the distance traveled by the centroid of the curve and the length of the revolved curve.
 \end{theorem}

  \begin{itemize}
    \item \pause For a torus
     \begin{itemize}
       \item \pause the length of the revolved circle is $2\pi r$;
       \item \pause the centroid is at $(R,0)$;
       \item \pause hence the surface area of a torus is $4\pi^2 Rr$.
     \end{itemize}
    \item \pause $C$: semicircle of radius $R$, rotated about axis joining endpoints
         \begin{itemize}
           \item \pause resulting surface is a sphere of radius $R$ and area $4\pi R^2$.
           \item \pause length of $C$ is $\pi R$;
           \item \pause the centroid travels a distance of $4R$;
           \item \pause the centroid is at a distance of $\frac{2R}{\pi}$ from the axis.
         \end{itemize}
  \end{itemize}
\end{frame}

\begin{frame}
  \frametitle{Surface Integrals}

\begin{itemize}
  \item $S$ be a surface in space;
  \item $\varphi \colon D \to S \subset \RR^3$, global parametrization of $S$;
  \item use $\varphi$ to move $\iint_S$ to $\iint_D$.
\end{itemize}

\pause
More precisely, if $f$ is a continuous (scalar or vectorial) function on $S$, then
%
$$\int\!\!\!\int_{\textcolor[rgb]{0.98,0.00,0.00}{S=\varphi(D)}} \textcolor[rgb]{0.00,0.00,1.00}{f(P)} \, \textcolor[rgb]{0.00,0.59,0.00}{dS} = \int\!\!\!\int_{\textcolor[rgb]{0.98,0.00,0.00}{D}} \textcolor[rgb]{0.00,0.00,1.00}{f(\varphi(u,v))}\, \textcolor[rgb]{0.00,0.59,0.00}{| \bm{\varphi}_u \times \bm{\varphi}_v| \, du\, dv}\; .$$
%
\pause If $S$ doesn't have a global parametrization:
\begin{itemize}
  \item \pause divide it into smaller pieces $S_1$, \ldots $S_N$ with non-overlapping interiors;
  \item such that each piece  $S_k$ has a local parametrization;
  \item then
%
$$\int\!\!\!\int_{S} f(P)\, dS = \int\!\!\!\int_{S_1} f(P)\; dS + \dotsb + \int\!\!\!\int_{S_N} f(P)\; dS\; .$$
\end{itemize}
\end{frame}

\begin{frame}
  \frametitle{Example}

The centroid of a hemisphere $S$ of radius $R$. \pause The $z-$coordinate is
%
$$z_C = \frac{1}{\text{area}(S)}\, \int\!\!\!\int_S z \, dS\; .$$
%
\pause To compute the surface integral:
 \begin{itemize}
   \item \pause parametrize the hemisphere as a graph surface;
   \item \pause parametrize the hemisphere as a surface of revolution;
   \item \pause use spherical coordinates.
 \end{itemize}

\pause $S$ has the parametrization $\varphi \colon D \to \RR^3$,
%
$$\varphi(u,v) = (u,v,\sqrt{R^2-u^2-v^2}) \; .$$
%
\pause The element of area is
%
$$|\bm{\varphi}_u \times \bm{\varphi}_v| \, du\, dv = \sqrt{1+h_u^2+h_v^2} \, du\, dv = \frac{R}{\sqrt{R^2-u^2-v^2}} \; du\, dv$$
%
\pause The surface integral becomes
%
$$\int\!\!\!\int_{\textcolor[rgb]{0.98,0.00,0.00}{S}} \textcolor[rgb]{0.00,0.00,1.00}{z} \, \textcolor[rgb]{0.00,0.59,0.00}{dS} = \int\!\!\!\int_{\textcolor[rgb]{0.98,0.00,0.00}{D}} \textcolor[rgb]{0.00,0.00,1.00}{\sqrt{R^2-u^2-v^2}}\, \textcolor[rgb]{0.00,0.59,0.00}{\frac{R}{\sqrt{R^2-u^2-v^2}} \; du\, dv} = \int\!\!\!\int_D R\, du\, dv = \pi R^3 \; .$$
\end{frame}

\begin{frame}

The hemisphere is also the surface obtained by revolving the quarter of a circle  $(x,z) = (f(u), g(u)) = (R\cos{u}, R\sin{u})$, $0 \leqslant u \leqslant \pi/2$ about the $z-$axis. \pause
%
$$\varphi(u,v) = (R\cos{u}\cos{v}, R\cos{u}\sin{v}, R\sin{u}) \; ,$$
%
with $(u,v)-$parameter space $D=$\pause $[0,\pi/2]\times [0,2\pi]$. \pause The element of area is
%
$$dS = |f|\sqrt{|f'|^2+|g'|^2} = R^2\cos{u}\;$$
%
\pause The surface integral becomes
%
\begin{align*}
  \int\!\!\!\int_{\textcolor[rgb]{0.98,0.00,0.00}{S}} \textcolor[rgb]{0.00,0.00,1.00}{z} \, \textcolor[rgb]{0.00,0.59,0.00}{dS} = & \int\!\!\!\int_{\textcolor[rgb]{0.98,0.00,0.00}{D}} \textcolor[rgb]{0.00,0.00,1.00}{R\sin{u}}\, \textcolor[rgb]{0.00,0.59,0.00}{R^2\cos{u}\; du\, dv} =  R^3 \left(\int_{u=0}^{u=\pi/2} \!\!\!\!\!\!\sin{u}\cos{u}\, du\right) \left( \int_{v=0}^{v=2\pi} \!\!\!\! dv \right) = \\
 = &2\pi R^3 \left. \frac{\sin^2{u}}{2} \right|_{u=0}^{u=\pi/2} =\pi R^3\; .
\end{align*}
%
\pause The $z-$coordinate of the centroid is
%
$$z_C = \frac{1}{\text{area}(S)}\, \int\!\!\!\int_S z \, dS = \frac{1}{2\pi R^2} \, \pi R^3 = \frac{R}{2}\; .$$
\end{frame}

\begin{frame}
  \frametitle{Vectorial Integrals}

\begin{itemize}
  \item \pause If the density of force acting on a surface is variable, then the total force is expressed as a surface integral of a vectorial quantity.
\end{itemize}

\begin{itemize}
  \item \pause $S$: the surface of an inflated balloon.
  \item \pause pressure inside the balloon is $p_1$ and the the pressure outside is $p_2 < p_1$;
  \item \pause $Q$: point on the surface of the balloon;
  \item On an infinitesimal region around $Q$:
  \begin{itemize}
    \item \pause difference in pressure $p = p_1-p_2$ determines a force:
        \begin{itemize}
          \item \pause magnitude $p\, dS$;
          \item \pause direction $\textbf{N}$: unit normal to $S$ at $Q$, pointing outward.
        \end{itemize}
    \item \pause infinitesimal force $d\textbf{F} = p \textbf{N} \, dS$
  \end{itemize}
  \item \pause total force
%
$$\bm{F} =  \int\!\!\!\int_S \, d\textbf{F} = \int\!\!\!\int_S p \textbf{N} \, dS = \int\!\!\!\int_S p\, \bm{dS} \; .$$
\end{itemize}
\end{frame}

\begin{frame}
  \frametitle{Vector Fields in Space}

  \begin{itemize}
    \item $\textbf{X}$: smooth vector field defined on an open region $D$ in space;
    \item $p$: point in $D$.
    \item \underline{Questions}:
    \begin{itemize}
      \item What is the effect of $\textbf{X}$ around $p$?
      \item What is the infinitesimal effect of $\textbf{X}$ at $p$?
    \end{itemize}
    \item Infinitesimal: limit of effect on regions shrinking to $p$.
    \item \pause Two types of effects:
    \begin{itemize}
      \item radial effect
      \item rotational effect
    \end{itemize}
  \end{itemize}

  \pause Radial effect:
  \begin{itemize}
    \item $S$: a piecewise smooth surface, bounding a region $B$ around $p$;
    \item Questions:
    \begin{itemize}
      \item How much matter does $\textbf{X}$ carry across $S$?
      \item What happens when $B \to \{p\}$?
    \end{itemize}
  \end{itemize}
\end{frame}


\begin{frame}
  \frametitle{Orientations of Surfaces}

We first need to decide \emph{which way} do we measure: \emph{inward} or \emph{outward}? \pause

  \begin{itemize}
    \item $S$: smooth surface, not necessarily boundary of a domain;
    \item We need a consistent choice of normal direction at each point of $S$
    \item $\textbf{N}$: a continuous unit vector field normal to $S$.
    \item Such a normal field doesn't always exist! (M\"oebius band)
  \end{itemize}

\pause \underline{Definitions}:
\begin{itemize}
  \item $S$ is \emph{orientable} if it has a continuous normal unit vector field.
  \item Each choice of such a normal field endows $S$ with an \emph{orientation};
  \item An \emph{oriented surface} is a surface with a predetermined orientation.
\end{itemize}

\pause If the surface $S$ bounds a domain in space:
\begin{itemize}
  \item outward normal gives the \emph{positive} orientation;
  \item inward normal gives the \emph{negative} orientation.
\end{itemize}
\end{frame}

\begin{frame}
  \frametitle{Flux and Divergence}

\begin{itemize}
  \item $S$: an oriented surface, orientation given by the unit normal field $\textbf{N}$;
  \item $\textbf{X}$: smooth vector field on $S$
\end{itemize}
\underline{Definition}: The flux of $\textbf{X}$ across $S$ is
%
$$\int\!\!\!\int_S \textbf{X}\cdot \textbf{N} \, dS = \int\!\!\!\int_S \textbf{X} \cdot \textbf{dS}\; .$$
%
\pause Radial effect of $\textbf{X}$ at $p$:
\begin{itemize}
  \item $D$: region around $p$,
  \item Boundary of $D$: $S=\partial D$, piecewise smooth parametrized surface.
\end{itemize}

\underline{Definition:} The \emph{divergence}\index{divergence} of $\textbf{X}$ at $p$ is the density of flux\index{density!of flux} at $p$:
%
$$(\divg \textbf{X})(p) = \lim_{D\to \{p\}} \frac{1}{\text{vol}(D)} \int\!\!\!\int_S \bm{X} \cdot \bm{N}dS \; ,$$
%
if the limit exists.
\end{frame}

\begin{frame}
\frametitle{Computations Using Parametrizations}
\begin{itemize}
\item $\varphi\colon D \to S$, $P=\varphi(u,v)$: smooth parametrization of $S$;
    \item $\bm{\varphi}_u$ and $\bm{\varphi}_v$ are tangent vectors;
    \item $\bm{\varphi}_u \times \bm{\varphi}_v$ is a normal vector; may point in the direction of $\textbf{N}$ or not;
    \item parametrization $\varphi$ is \emph{compatible} with the orientation given by $\textbf{N}$ if $\bm{\varphi}_u \times \bm{\varphi}_v$ and $\textbf{N}$ point in the same direction
    \item Equivalently:  the frame $\{ \textbf{N}, \bm{\varphi}_u, \bm{\varphi}_v\}$ is right hand oriented.
    %
    \item If $\varphi$ is a parametrization compatible with the orientation, then
    %
    $$\textbf{N} = \frac{\bm{\varphi}_u \times \bm{\varphi}_v}{|\bm{\varphi}_u \times \bm{\varphi}_v|}$$
    %
$$\textbf{dS} = \textbf{N} \, dS = \frac{\bm{\varphi}_u \times \bm{\varphi}_v}{|\bm{\varphi}_u \times \bm{\varphi}_v|}  \, \cdot \,|\bm{\varphi}_u \times \bm{\varphi}_v| \, du\,dv = \bm{\varphi}_u \times \bm{\varphi}_v\, du\, dv$$
%
and therefore
%
$$\int\!\!\!\int_S \textbf{X}\cdot \textbf{N} \, dS = \int\!\!\!\int_S \textbf{X} \cdot \textbf{dS} = \int\!\!\!\int_D \textbf{X}(\varphi(u,v)) \cdot (\bm{\varphi}_u \times \bm{\varphi}_v) \, du\, dv\; .$$
\end{itemize}
\end{frame}


\begin{frame}
  \frametitle{Example}

\begin{itemize}
  \item Compute the flux of  $\textbf{X} = ax\, \textbf{i}$ across $S$
  \item $S$: sphere of radius $R$ centered at the origin, positively oriented
\end{itemize}

\pause A parametrization of $S$ is $\varphi \colon [0,\pi] \times [0,2\pi]\to \RR^3$,
%
$$\varphi(u,v) = (R\sin{u}\cos{v}, R\sin{u}\sin{v}, R\cos{u}) \; .$$
%
\pause Then
%
\begin{align*}
  \bm{\varphi}_u =& \langle R\cos{u}\cos{v}, R\cos{u}\sin{v}, -R\sin{u}\rangle \\
  %
  \bm{\varphi}_v = & \langle -R\sin{u}\sin{v}, R\sin{u}\cos{v}, 0 \rangle
\end{align*}
%
$$\bm{\varphi}_u \times \bm{\varphi}_v = \langle R^2\sin^2{u}\cos{v}, R^2\sin^2{u}\sin{v}, R^2\sin{u}\cos{u} \rangle = R\sin{u} \, \textbf{r} = R^2\sin{u}\, \textbf{N}$$
%
\begin{align*}
  \int\!\!\!\int_S \textbf{X} \cdot \textbf{dS}= & \int\!\!\!\int_D \textbf{X}(\varphi(u,v)) \cdot (\bm{\varphi}_u \times \bm{\varphi}_v) \, du\, dv = \\
  = & \int_{u=0}^{u=\pi} \int_{v=0}^{v=2\pi} a\, R\sin{u}\cos{v}\, R^2\sin^2{u}\cos{v}\; du\, dv = \\
  %
  = & aR^3 \left( \int_{u=0}^{u=\pi} \sin^3{u} \, du \right) \left( \int_{v=0}^{v=2\pi} \cos^2{v}\, dv \right) = aR^3 \cdot \frac{4}{3} \cdot \pi = \frac{4\pi aR^3}{3}\; .
\end{align*}
\end{frame}

\begin{frame}
\frametitle{Example: Continued}

$$\int\!\!\!\int_S ax\textbf{i} \cdot \textbf{dS}= \frac{4\pi aR^3}{3} ,\quad
\int\!\!\!\int_S by\textbf{j} \cdot \textbf{dS}= \frac{4\pi bR^3}{3},\quad
\int\!\!\!\int_S cz\textbf{k} \cdot \textbf{dS}= \frac{4\pi cR^3}{3}
 $$

\pause If $\textbf{X} = ax\, \textbf{i} + by\, \textbf{j} + cz\, \textbf{k}$, then
%
$$\int\!\!\!\int_S \textbf{X} \cdot \textbf{dS} = \frac{4\pi R^3}{3} (a+b+c) = (a+b+c) \text{vol}(B)\; ,$$
%
where $B$ is the volume of the ball enclosed by the sphere $S$.
\medskip

\pause If the divergence of $\textbf{X}$ at 0 exists, then:
%
$$\divg \textbf{X} (0) = \lim_{R\to 0} \frac{3}{4\pi R^3} \int\!\!\!\int_{S_R(0)} \textbf{X} \cdot \textbf{dS} = \lim_{R \to 0} (a+b+c) = a+b+c\; .$$
\end{frame}

\begin{frame}
  \frametitle{Another Example}

  Let $S$ be the part of the paraboloid $z=4-x^2-y^2$ above the $xy-$plane, oriented upward, and $\textbf{X} = a\, \textbf{i} + b\, \textbf{j} + c\, \textbf{k}$. Compute
%
$$\int\!\!\!\int_S \textbf{X} \cdot \textbf{dS} \; .$$

\pause Parametrization: $\varphi \colon B \to \RR^3$, $\varphi(u,v) = (v,u,4-u^2-v^2)$
%
$$\bm{\varphi}_u \times \bm{\varphi}_v = \langle 0,1,-2v\rangle \times \langle 1,0,-2u\rangle = \left| \begin{array}{ccc}
i & j & k \\
0 & 1 & -2v\\
1 & 0 & -2u
\end{array}\right| = -2u \, \textbf{i} - 2v\, \textbf{j} - \textbf{k }\; .$$
%
$$\textbf{N} = -\frac{\bm{\varphi}_u \times \bm{\varphi}_v}{|\bm{\varphi}_u \times \bm{\varphi}_v|}\; .$$
%
\begin{align*}
  \textbf{X} &\cdot \textbf{dS}  = \textbf{X} \cdot \textbf{n} \, dS =  \textbf{X} \cdot \left( -\frac{\bm{\varphi}_u \times \bm{\varphi}_v}{|\bm{\varphi}_u \times \bm{\varphi}_v|}\right) \, |\bm{\varphi}_u \times \bm{\varphi}_v| \, du\, dv = \\
  %
  & = (a\, \textbf{i} +b\, \textbf{j} +c\, \textbf{k}) \cdot (2u \, \textbf{i} + 2v\, \textbf{j} + \textbf{k}) \, du\, dv = (2au +2bv +c)\, du\, dv\; ,
\end{align*}
%
hence
%
$$\iint_S \textbf{X} \cdot \textbf{dS} = \int\!\!\!\int_B (2au +2bv +c)\, du\, dv = c\,\int\!\!\!\int_B \, du\, dv = c \cdot 4\pi = 4\pi\, c\; .$$
\end{frame}

\begin{frame}
  \frametitle{Divergence}

\underline{Recall}: The \emph{divergence}\index{divergence} of $\textbf{X}$ at $p$ is the density of flux\index{density!of flux}:
%
$$(\divg \textbf{X})(p) = \lim_{D\to \{p\}} \frac{1}{\text{vol}(D)} \int\!\!\!\int_S \bm{X} \cdot \bm{N}dS \; ,$$
%
if the limit exists. \pause The limit does exist when $X$ is reasonably smooth.

\pause We have seen similar limits when we talked about average values:
%
$$f(P) = \lim_{D\to\{p\}} \frac{1}{\text{vol}(D)} \int\!\!\!\!\int\!\!\!\!\int_D f(Q)\, dV\; .$$
%
\pause However:
\begin{itemize}
  \item the definition of $\divg \textbf{X}$ involves a surface integral
  \item the definition of average involves a triple integral
\end{itemize}

\pause We should somehow transform the surface integral into a triple integral.
\end{frame}

\begin{frame}
  \frametitle{Divergence Theorem}

\begin{theorem}{\rm
  Let $D$ be a compact set in space with boundary $S$ a piecewise smooth parametrized surface, oriented by the outward normal, and let
%
$$\textbf{X}(x,y,z) = P(x,y,z) \, \textbf{i} + Q(x,y,z) \, \textbf{j} + R(x,y,z)\, \textbf{k}$$
%
be a smooth vector field defined on $D$. Then
%
$$\int\!\!\!\int_S \bm{X} \cdot \bm{dS} = \int\!\!\!\!\!\int\!\!\!\!\!\int_D \left(\frac{\partial P}{\partial x}+ \frac{\partial Q}{\partial y} + \frac{\partial R}{\partial z} \right) \, dV$$
}\end{theorem}
%
\pause \underline{Consequence}:
%
\begin{align*}
  (\divg \textbf{X}) (p) = &
\lim_{D\to \{P\}} \frac{1}{\text{vol}(D)}\int\!\!\!\!\!\int_S \textbf{X} \cdot \textbf{dS}  = \\
= & \lim_{D\to \{P\}} \frac{1}{\text{vol}(D)}\int\!\!\!\!\!\int\!\!\!\!\!\int_D \left(\frac{\partial P}{\partial x}+ \frac{\partial Q}{\partial y} + \frac{\partial R}{\partial z} \right) \, dV =
  \frac{\partial P}{\partial x}+ \frac{\partial Q}{\partial y} + \frac{\partial R}{\partial z}\; .
\end{align*}


\end{frame}

\begin{frame}
  \frametitle{Divergence Theorem}

%
If $\textbf{X} = P\, \textbf{i} + Q\, \textbf{j} +R\, \textbf{k}$, then
%
$$\divg \textbf{X} = \frac{\partial P}{\partial x}+ \frac{\partial Q}{\partial y} + \frac{\partial R}{\partial z}  = ``\langle \partial_x, \partial_y, \partial_z \rangle \cdot \langle P, Q, R\rangle" =  ``\nabla \cdot \textbf{X}"\; .$$

Intuitive notation:
%
$$\divg \textbf{X} = \nabla \cdot \textbf{X}\, .$$

\pause Confirmed by the example of $\textbf{X} = ax \, \textbf{i} + by\, \textbf{j} + cz\, \textbf{k}$ $\Longrightarrow$ $\divg \textbf{X} = a+b+c$.\pause

\begin{theorem}
  %
$$\int\!\!\!\int_S \bm{X} \cdot \bm{dS} = \int\!\!\!\!\!\int\!\!\!\!\!\int_D \divg \textbf{X} \, dV$$
\end{theorem}

\begin{itemize}
  \item \pause If $(\divg \textbf{X})(p)>0$, then \pause $p$ acts as a source;
  \item \pause If $(\divg \textbf{X})(p)<0$, then \pause $p$ acts as a sink;
  \item \pause If $\divg \textbf{X} \equiv 0$ on some domain $D$, then \pause $\textbf{X}$ is incompressible on $D$.
\end{itemize}

\end{frame}

\begin{frame}
  \frametitle{Example}

$S$ is the part of the paraboloid $z=4-x^2-y^2$ above the $xy-$plane, oriented \emph{upward}; $\textbf{X} = a\, \textbf{i} + b\, \textbf{j} + c\, \textbf{k}$. Use the Divergence Theorem to compute
%
$$\int\!\!\!\int_{S\uparrow} \textbf{X} \cdot \textbf{dS} \; .$$
%

\begin{itemize}
  \item \pause \underline{Problem}: The surface $S$ does not enclose a region in space;
  \item \pause We add the disk $D$ of radius 2 centered at the origin in the plane $z=0$;
  %
$$\int\!\!\!\int_{S\uparrow \cup D\downarrow} \textbf{X} \cdot \textbf{dS} = \int\!\!\!\int\!\!\!\int_R\; \divg \textbf{X} \; dV = 0\; ,$$
%
\item \pause $R$ orients $D$ with the \emph{downward} normal, hence
        %
%
$$\int\!\!\!\int_{S\uparrow} \textbf{X} \cdot \textbf{dS} = \int\!\!\!\int_{D\uparrow} \textbf{X} \cdot \textbf{dS}$$
%
\end{itemize}

\pause The upward normal to $D$ is $\textbf{k}$, hence $\textbf{X}\cdot \textbf{dS}= \textbf{X}\cdot \textbf{k} \, dS= c\,dS$. Therefore
%
$$\int\!\!\!\int_{S} \textbf{X} \cdot \textbf{dS} = \int\!\!\!\int_{D} \textbf{X} \cdot \textbf{dS} = \int\!\!\!\int_D c\, dS = c\cdot\text{area}(D) = 4\pi c\, .$$
\end{frame}

\begin{frame}
  \frametitle{Application}

\begin{itemize}
  \item \pause $\textbf{F}$: total force due to the difference in pressure between the interior of an inflated balloon and the exterior;
%
$$\bm{F} =  \int\!\!\!\int_S \, d\textbf{F} = \int\!\!\!\int_S p \textbf{N} \, dS = \int\!\!\!\int_S p\, \bm{dS} \; .$$
%
\item \pause For every unit vector $\textbf{u}$ we have
%
\begin{align*}
  \textbf{F} \cdot \textbf{u} = \left(\int\!\!\!\int_S p \textbf{N} \, dS \right) \cdot \textbf{u} = \int\!\!\!\int_S p\, \textbf{u} \cdot  \textbf{N} \, dS = \int\!\!\!\int\!\!\!\int_D \divg (p\, \textbf{u}) dV = 0
\end{align*}
%
because $\divg (p \textbf{u}) = 0$ since the vector field $\textbf{X}= p\, \textbf{u}$ is constant on $D$.
\item Therefore $\textbf{F}\cdot \textbf{u} = 0$ for every unit vector $\textbf{u}$;
    \item \pause Which implies $\textbf{F}=\textbf{0}$.
\end{itemize}

\end{frame}

\begin{frame}
  \frametitle{Another Application}

A solid body is submerged into a tank containing a liquid of constant density $\rho$. What is the buoyant force?

\begin{itemize}
  \item \pause Body occupies a region $D$, exterior boundary $S$;
  \item Unit outward normal field $\textbf{N}$;
  \item \pause Magnitude of pressure at depth $a$ below the surface is $p_0+\rho a g$, where
   \begin{itemize}
     \item $g$ is the magnitude of the gravitational acceleration.
     \item $p_0$ is the pressure at surface of liquid
   \end{itemize}
  \item \pause Infinitesimal force acting on $S$ is $d\textbf{F} = -(p_0+\rho a g) \, \textbf{N} \, dS$,
  \item  \pause The total force is
%
$$\bm{F} = \int\!\!\!\int_S \, d\textbf{F} = \int\!\!\!\int_S -(p_0+\rho a g) \, \textbf{N} \, dS = \int\!\!\!\int_S -\rho a g \, \bm{N}\,dS = \int\!\!\!\int_S \rho g z \, \bm{N}\,dS$$
%
\item \pause EC: Use the Divergence Theorem to show that $\textbf{F} = \rho V g \, \textbf{k}$

    ($V$: volume of the region enclosed by $S$.)

\end{itemize}

\end{frame}

\begin{frame}
  \frametitle{Rotational Effect}

  \begin{itemize}
    \item $\textbf{X}$ a smooth vector field defined on an open region;
    \item $p$: a point in the region
    \item Effect of $\textbf{X}$ around $p$:
    \begin{itemize}
      \item Radial effect $\leftrightsquigarrow$ measured by divergence,
      $(\divg \textbf{X})(p)$
      \item Rotational effect $\leftrightsquigarrow$ ...
    \end{itemize}
    \item \pause A rotation in space is determined by
    \begin{itemize}
      \item axis;
      \item orientation;
      \item angular velocity
    \end{itemize}
    \item \pause Information encoded in a vector $\textbf{Y}$
    \item \pause 2D rotations are easier to understand
  \end{itemize}


\end{frame}

\begin{frame}
  \frametitle{Components}

  \begin{itemize}
    \item $\textbf{n}$: unit vector; $\textbf{Y} \cdot \textbf{n}$: scalar component of $\textbf{Y}$ onto $\textbf{n}$;
    \item \pause ``Wish list": for each $\textbf{n}$, the component $\textbf{Y} \cdot \textbf{n}$:
    \begin{itemize}
      \item measures rotational effect in the plane $S$ normal to $\textbf{n}$.
      \item is the Flatland curl of the projection of $\textbf{X}$ onto such a plane:
  %
  $$\textbf{Y} \cdot \textbf{n} = \text{curl}_{\textbf{n}} (\orth_{\textbf{n}} \textbf{X})(p)$$
  %
    \end{itemize}
\item \pause Infinitely long and, technically, not even a list!
  \end{itemize}

\pause \underline{Questions}:
\begin{itemize}
  \item \pause What does $\text{curl}_{\textbf{n}}$ mean?
  \item \pause IS there such a vector?
\end{itemize}

\pause \underline{Fact}:
\begin{itemize}
  \item If such a vector $\textbf{Y}$ exists, it is unique.
\end{itemize}

If $\textbf{A}$ and $\textbf{B}$ are vectors and $\textbf{A}\cdot \textbf{u} = \textbf{B}\cdot \textbf{u}$ for all unit vectors $\textbf{u}$, then $\textbf{A}=\textbf{B}$.

\end{frame}


\begin{frame}
  \frametitle{Induced Orientation on a Surface}

  \begin{itemize}
    \item $S$: plane in space (an instance of Flatland); unit normal vector $\textbf{n}$;
    \item $\text{curl}_{\textbf{n}}$: density of circulation;
    \item \pause To measure circulation along curves in $S$
    \begin{itemize}
      \item we must consistently orient closed curves in $S$.
      \item we must consistently orient $S$;
    \end{itemize}
    \item \pause We will do that not just for planes, but for more general surfaces.
  \end{itemize}

  \begin{itemize}
    \item $S$: smooth surface;
    \item $\textbf{n}$: smooth unit vector field normal to $S$;
    \item $\textbf{n}$ orients $S$ by orienting every tangent plane to $S$:
    \begin{itemize}
      \item $\textbf{u}$, $\textbf{v}$ are non-collinear vectors tangent to $S$ at a point $p$;
      \item \underline{Definition}: $\{\textbf{u},\textbf{v}\}$ is a positively oriented frame for the tangent plane if $\{\textbf{n},\textbf{u},\textbf{v}\}$ is positively oriented in space.
    \end{itemize}
  \end{itemize}
\end{frame}

\begin{frame}
  \frametitle{Induced Orientation on a Curve}
  \begin{itemize}
    \item $S$: smooth surface, oriented by unit normal vector $\textbf{n}$;
    \item $D$: region in $S$, bounded by a curve $C=\partial D$;
    \item $\textbf{N}$ the unit vector field on $C$
    \begin{itemize}
      \item tangent to $S$;
      \item normal to $C$;
      \item pointing outward of $D$.
    \end{itemize}
    \item \pause $\textbf{N}$ orients $C$ by orienting tangents:
    \begin{itemize}
      \item $\textbf{T}$: unit tangent to $C$ (and hence tangent to $S$)
      \item \underline{Definition}: $\textbf{T}$ is \emph{positively oriented} if $\{\textbf{N}, \textbf{T}\}$ is positively oriented in the plane tangent to $S$
      \item $\Longleftrightarrow$ $\{\textbf{n}, \textbf{N}, \textbf{T}\}$ is positively oriented in space $\Longleftrightarrow$ $\textbf{T}=\textbf{n}\times \textbf{N}$
    \end{itemize}
  \end{itemize}

\pause \underline{Recap}:
\begin{itemize}
  \item Orientation of space + unit normal to surface $\Longrightarrow$ orientation on surface
  \item Orientation of surface + unit normal to curve $\Longrightarrow$ orientation of curve
\end{itemize}
\end{frame}

\begin{frame}
  \frametitle{Example}

\begin{itemize}
  \item $S$: the unit sphere $x^2+y^2+z^2 = 1$;
  \item oriented by the outward normal $\textbf{n}$;
  \item $D = S \cap \{z\geqslant 0\}$ be the upper hemisphere;
  \item Boundary of $D$: the circle $C= S \cap \{z=0\}$.
\end{itemize}

\pause At the point $p=(1,0,0)$
\begin{itemize}
  \item the normal $\textbf{n}$ is the vector \pause $\textbf{i}$;
  \item the normal $\textbf{N}$ is the vector \pause $-\textbf{k}$;
  \item positively oriented tangent to $C$:\pause
%
$$\textbf{T}= \textbf{n} \times \textbf{N} = \textbf{i} \times (-\textbf{k}) = \textbf{k}\times \textbf{i} = \textbf{j} \;,$$
%
\item \pause which orients $C$ counterclockwise in the $\{\textbf{i},\textbf{j}\}-$plane.
\end{itemize}

\pause If instead we use the lower hemisphere to orient $C$: at $p=(1,0,0)$
 \begin{itemize}
   \item $\textbf{n}=\textbf{i}$, $\textbf{N}=\textbf{k}$, $\textbf{T}=\textbf{n}\times \textbf{N} = \textbf{i} \times \textbf{k} = -\textbf{j}$;
   \item which orients $C$ clockwise in the $\{\textbf{i},\textbf{j}\}-$plane.
 \end{itemize}
\end{frame}

\begin{frame}
  \frametitle{Components of Curl in Rectangular Coordinates}

      %
$$\textbf{Y} \cdot \textbf{n} = \text{curl}_{\textbf{n}} (\orth_{\textbf{n}} \textbf{X})(p)$$

If $\textbf{X}=P \, \textbf{i} + Q\, \textbf{j} + R\, \textbf{k}$, then
\begin{itemize}
  \item The $\textbf{k}$ normal orients the $xy-$plane as $\{ \textbf{i}, \textbf{j}\}$, because $\{\textbf{k}, \textbf{i}, \textbf{j}\} \sim \{\textbf{i}, \textbf{j}, \textbf{k}\}$
%
$$\textbf{Y} \cdot \textbf{k} = \text{curl}_{\textbf{k}} (\orth_{\textbf{k}} \textbf{X}) = \text{curl}_{\textbf{k}} (P\, \textbf{i} + Q\, \textbf{j}) = \partial_x Q - \partial_y P = Q_x - P_y\; .$$
%
\item \pause The $\textbf{j}$ normal orients the $xz-$plane as \pause $\{ \textbf{k}, \textbf{i}\}$, because $\{\textbf{j}, \textbf{k}, \textbf{i}\} \sim \{\textbf{i}, \textbf{j}, \textbf{k}\}$\pause
%
$$\textbf{Y} \cdot \textbf{j} = \text{curl}_{\textbf{j}} (R\, \textbf{k} + P\, \textbf{i}) = \partial_z P - \partial_x R = P_z - R_x\; .$$
%
\item \pause The $\textbf{i}$ normal orients the $yz-$plane as \pause $\{ \textbf{j}, \textbf{k}\}$, because $\{\textbf{i}, \textbf{j}, \textbf{k}\} \sim \{\textbf{i}, \textbf{j}, \textbf{k}\}$\pause
%
$$\textbf{Y} \cdot \textbf{i} = \text{curl}_{\textbf{i}} (Q\, \textbf{j} + R\, \textbf{k}) = \partial_y R - \partial_z Q = R_y - Q_z\; .$$
\end{itemize}

\pause If $\textbf{Y} = (\partial_y R - \partial_z Q) \, \textbf{i} +  (\partial_z P - \partial_x R) \, \textbf{j} + (\partial_x Q - \partial_y P) \, \textbf{k}$,
then
%
$$\textbf{Y} \cdot \textbf{n} = \text{curl}_{\textbf{n}} (\orth_{\textbf{n}} \textbf{X})(p) \; $$
%
is true for $\textbf{n} = \textbf{i}$, $\textbf{j}$, and $\textbf{k}$. \pause Is it true for \emph{all} $\textbf{n}$?

\end{frame}

\begin{frame}

\begin{theorem}[Stokes]{\rm
Let $S$ be a smooth surface in space, oriented by the unit normal field $\textbf{n}$. Let $D$ be a region on $S$, bounded by the piecewise smooth curve $C=\partial D$, with unit tangent $\textbf{T}$ positively oriented by the outward pointing normal $\textbf{N}$. Let
$\textbf{X}=P \, \textbf{i} + Q\, \textbf{j} + R\, \textbf{k}$ be a smooth vector field on $S$ and
%
$$  \textbf{Y} = (\partial_y R - \partial_z Q) \, \textbf{i} +  (\partial_z P - \partial_x R) \, \textbf{j} + (\partial_x Q - \partial_y P) \, \textbf{k} \; .
$$
%
Then
%
$$\oint_C \textbf{X} \cdot \textbf{dr} = \int\!\!\! \int_D \textbf{Y} \cdot \textbf{dS}\, \; .$$}
\end{theorem}

Recall: $\textbf{X}\cdot \textbf{dr} = \textbf{T}\, ds$ and $\textbf{Y} \cdot \textbf{dS} = \textbf{n} \; dS$.

\pause Idea of proof:
\begin{itemize}
  \item \pause Use a parametrization of $S$ to move the integrals to the parameter plane;
  \item \pause Apply Green's Theorem in the parameter plane.
\end{itemize}

\end{frame}

\begin{frame}
  \frametitle{Consequences}

  \begin{itemize}
    \item $\textbf{n}$: unit vector
    \item $S$: plane through $p$ normal to $\textbf{n}$;
    \item $C$: simple curve in $S$ around $p$;
    \item $D$: domain enclosed by $C$.\pause
    %
    \begin{align*}
    \textbf{X} = & \orth_{\textbf{n}} \textbf{X} + \proj_{\textbf{n}} \textbf{X} \Longrightarrow \textbf{X}\cdot \textbf{T} = \orth_{\textbf{n}} \textbf{X} \cdot \textbf{T} \pause \\
      \text{curl}_{\textbf{n}} &(\orth_{\textbf{n}} \textbf{X}) (p) = \lim_{D\to\{p\}} \frac{1}{\text{area}(D)} \oint_C \orth_{\textbf{n}} \textbf{X} \cdot \textbf{dr} = \\
      = & \lim_{D\to\{p\}} \frac{1}{\text{area}(D)} \oint_C
      \textbf{X}\cdot \textbf{dr} = \lim_{D\to\{p\}} \frac{1}{\text{area}(D)} \int\!\!\!\int_D \textbf{Y}\cdot \textbf{n} \, dS = \textbf{Y}\cdot \textbf{n}
    \end{align*}
    \item \pause $\textbf{Y} = (\partial_y R - \partial_z Q) \, \textbf{i} +  (\partial_z P - \partial_x R) \, \textbf{j} + (\partial_x Q - \partial_y P) \, \textbf{k}$ works for all $\textbf{n}$ \, !!
  \end{itemize}
\end{frame}

\begin{frame}
\frametitle{Curl of a Vector Field}

\underline{Definition}:

The \emph{curl} of a smooth field $X$ is the unique vector field $\curl X$ that satisfies
%
$$\textbf{curl}\, \textbf{X}\cdot \textbf{n} = \text{curl}_{\textbf{n}} (\orth_{\textbf{n}} \textbf{X})$$

\pause In rectangular coordinates:

If $\textbf{X}=P(x,y,z) \, \textbf{i} + Q(x,y,z)\, \textbf{j} + R(x,y,z)\, \textbf{k}$, then
%
$$
  \textbf{curl}\, \textbf{X} = (\partial_y R - \partial_z Q) \, \textbf{i} +  (\partial_z P - \partial_x R) \, \textbf{j} + (\partial_x Q - \partial_y P) \, \textbf{k} \; ,
$$
%
$$\textbf{curl}\, \textbf{X} = \left| \begin{array}{ccc}
  \textbf{i} & \textbf{j} & \textbf{k} \\
  %
  \partial_x & \partial_y & \partial_z \\
  %
  P & Q & R
\end{array}\right| = \nabla \times \textbf{X}\; .$$
%

\pause Stokes' Theorem:
%
$$\oint_C \textbf{X}\cdot \textbf{dr} = \int\!\!\!\int_D \curl \textbf{X} \cdot \textbf{dS}\; .$$
\end{frame}

\begin{frame}
  \frametitle{Example}
\end{frame}



\begin{frame}
  \frametitle{Vectorial Potential}

  Given a smooth vector field $\textbf{X}$, one can ask:
%
\begin{itemize}
  \item Is $\textbf{X}$ the $\curl$ of a vector field?
  \item \pause Any field $\textbf{G}$ such that $\textbf{X} = \textbf{curl}\, \textbf{G}$ is called a \emph{vectorial potential}\index{potential!vectorial} for $\textbf{X}$.
  \item \pause If $\textbf{X}=\nabla \times \textbf{G}$ is a curl field, then $\divg \textbf{X} = 0$.
  \item \pause Two vectorial potentials differ by a gradient field.
\end{itemize}
\pause We can use Stokes' theorem to:
\begin{itemize}
  \item \pause Evaluate line integrals by computing a surface integral, or
  %
  \item \pause Evaluate a surface integral by computing a line integral.
\end{itemize}

\end{frame}

\begin{frame}
  \frametitle{Application}
Surface $D$ the part of the paraboloid $z=4-x^2-y^2$ above the $xy-$plane, oriented upward, $\textbf{X} = a\, \textbf{i} +b\, \textbf{j} +c\, \textbf{k}$
%
$$\iint_D \textbf{X} \cdot \textbf{dS} $$
%

\pause
$\divg \textbf{X} =0$, hence $\textbf{X}$ may be the curl of a vector field $\textbf{G}=P\, \textbf{i} + Q\, \textbf{j} + R\, \textbf{k}$.\pause
%
$$Q_x - P_y = c, \qquad P_z-R_x = b, \qquad R_y-Q_z= a\; .$$
%
\pause One solution is $Q=cx$, $P=bz$, $R=ay$, hence $\textbf{G} = bz\, \textbf{i} + cx\, \textbf{j} + ay\, \textbf{k}$ is a vector potential for $\textbf{X}$. Then $\textbf{X}= \textbf{curl}\, \textbf{G}$ and therefore
%
$$\iint_D \textbf{X} \cdot \textbf{dS} = \iint_D \textbf{curl}\, \textbf{G}\, \cdot \textbf{dS} = \oint_C \textbf{G} \cdot \textbf{dr} = \oint_C bz\, dx + cx\, dy + ay\, dz\; ,$$
%
where $C =\partial D$, the circle of radius 2 centered at the origin, oriented counterclockwise;\pause $x=2\cos{t}$, $y=2\sin{t}$, $z=0$, with $0 \leqslant t \leqslant 2\pi$ is an orientation-compatible parametrization of $C$
%
$$\oint_{C} \textbf{G} \cdot \textbf{dr} = \int_{0}^{2\pi} 2c\cos{t} \, (2\cos{t}) \, dt = 4c\, \int_{0}^{2\pi} \cos^2{t} \, dt = 4\pi\,c \; .$$
\end{frame}

\begin{frame}
  \frametitle{Div, Curl, Grad}

$$\divg (\textbf{curl} \, \textbf{X})  = \nabla \cdot (\nabla \times \textbf{X}) = 0 \; .$$
%
\pause $B$: ball centered at $p$, with boundary a sphere $S$ centered at $p$.
%
$$\iiint_B \divg (\textbf{curl}\,\textbf{X}) \, dV = \iint_{S=\partial B} \textbf{curl}\, \textbf{X} \cdot \textbf{dS} = \oint_{\partial S} \textbf{X} \cdot \textbf{dr} = 0\; ,$$
%
$$\divg (\textbf{curl}\, \textbf{X}) (p) = \lim_{B \to \{p\}} \frac{1}{\text{vol}(B)} \iiint_B \divg (\textbf{curl}\, \textbf{X}) \, dV = 0\; .$$
\pause
%
$$\textbf{curl} ( \textbf{grad} f ) = \nabla \times (\nabla f) = \textbf{0}\; .$$

\pause $D$: disk centered at $p$, in the plane normal to $\textbf{n}$ at $p$, and $C=\partial D$
%
$$\iint_D \textbf{curl} \,(\textbf{grad} f) \cdot \textbf{n}\, dS = \iint_D \textbf{curl}\, (\textbf{grad} f) \cdot \textbf{dS} = \oint_C \textbf{grad} f \cdot \textbf{dr} = 0\; ,$$
%
$$\textbf{curl} \,(\textbf{grad} f) (p) \cdot \textbf{n} = \lim_{D\to \{p\}} \frac{1}{\text{area}(D)} \iint_D \textbf{curl} (\textbf{grad} f) \cdot \textbf{n}\, dS = 0\; ; $$
%
since this is valid for all unit vectors $\textbf{n}$, we conclude that $\textbf{curl} \,(\textbf{grad} f) (p)= \textbf{0}$.
\end{frame}

\begin{frame}
  \frametitle{A Unifying Theme}

 \begin{center}
   \boxed{
\begin{tabular}{ccc}
   Accumulation of a quantity &  & Accumulation of a  \\
   over the boundary of & = & derived quantity \\
   a closed domain &  &  over the entire domain
\end{tabular}
}
 \end{center}
\begin{itemize}
  \item the domain is oriented;
  \item the orientation of the domain induces an orientation of the boundary.
\end{itemize}


\end{frame}

\begin{frame}
  \frametitle{Domains of Dimension One}

\underline{Fundamental Theorem of Line Integrals}:
\begin{itemize}
  \item $C$: smooth curve, joining points $A$ and $B$, oriented from $A$ to $B$;
  \item $\partial C = \{A,B\}$: boundary of $C$;
  \item Orientation of $\partial C$: $A$, weight -1; $B$, weight +1;
  \item $f$ a differentiable function defined on (an open neighborhood of) $C$.
\end{itemize}
%
$$f(B)-f(A) =\int_C df \quad \Longleftrightarrow \quad \int_{\partial C} f = \int_C df\;.$$

$\textbf{r} \colon [a,b] \to \RR^n$, smooth parametrization of $C$ with $\textbf{r}(a) = A$ and $\textbf{r}(b)=B$:
%
$$
  f(\textbf{r}(b)) - f(\textbf{r}(a)) = \int_C \nabla f \cdot  \textbf{dr}\; .
$$


\underline{Net Change Theorem}: If $f \colon [a,b] \to \RR$ is a differentiable function, then
%
$$
   f(b)-f(a) = \int_{a}^b f'(x) \; dx\; .
$$


\end{frame}

\begin{frame}
  \frametitle{Domains of Dimension Two}

\underline{Stokes' Theorem}:
\begin{itemize}
  \item $S$: surface, oriented by unit normal field $\textbf{n}$; $D$: domain on $S$;
  \item $C=\partial D$: boundary of $D$, oriented by the outward unit normal $\textbf{N}$;
  \item $\textbf{X}$: is a smooth field on $D$.
\end{itemize}
$$
  \oint_{\partial D} \textbf{X} \cdot \textbf{dr} = \iint_D \textbf{curl} \textbf{X} \cdot \textbf{dS}
$$

If $\textbf{X} = P(x,y,z)\; \textbf{i} + Q(x,y,z)\, \textbf{j} + R(x,y,z)\; \textbf{k}$, then
%
$$ \textbf{curl} \textbf{X} =\langle \partial_y R - \partial_z Q, \partial_z P - \partial_x R, \partial_x Q - \partial_y P\rangle = \left| \begin{array}{ccc}
  \textbf{i} & \textbf{j} & \textbf{k} \\
  %
  \partial_x & \partial_y & \partial_z \\
  %
  P & Q & R
\end{array}\right| = \nabla \times \textbf{X}\; .$$

\underline{Green's Theorem}: Particular case when $S$ is a plane, oriented by $\textbf{k}$.
$$
  \oint_{\partial D} P(x,y) dx + Q(x,y) dy = \iint_D \left( Q_x - P_y\right) dx\,dy
$$
$$
  \oint_{\partial D} \textbf{X} \cdot \textbf{dr} = \iint_D \text{curl}_{\textbf{k}} \textbf{X}\, dA \quad , \quad
  \oint_{\partial D} \textbf{X} \cdot \textbf{n}\,ds = \iint_D \divg  \textbf{X}\, dA
$$
\end{frame}

\begin{frame}
  \frametitle{Domains of Dimension Three}

  \underline{Divergence Theorem}:
   \begin{itemize}
     \item $D$, domain in $\RR^3$;
     \item $\partial D$: boundary of $D$, oriented by the outward unit normal $\textbf{n}$;
     \item $X$: smooth vector field on $D$.
   \end{itemize}
%
$$
  \iint_{\partial D} \textbf{X} \cdot \textbf{dS} = \iiint_D \text{div} \textbf{X} \;dV \; ,
$$
%
where $\textbf{X} \cdot \textbf{dS} = \textbf{X} \cdot \textbf{n} \, dS$.

\medskip
If $\textbf{X} = P(x,y,z)\; \textbf{i} + Q(x,y,z)\, \textbf{j} + R(x,y,z)\; \textbf{k}$, then
%
$$\divg \textbf{X} = P_x + Q_y + R_z = \nabla \cdot \textbf{X}.$$
%
$$\textbf{X}=\grad f \Longrightarrow \curl \textbf{X} = \curl (\grad f) = \textbf{0} \Longrightarrow \text{condition for scalar potential}$$
%
$$\textbf{X}=\curl \textbf{G} \Longrightarrow \divg \textbf{X} = \divg (\curl \textbf{G}) = 0 \Longrightarrow \text{condition for vector potential}$$

\end{frame}

\begin{frame}
\frametitle{Higher Dimensional Domains}

 \begin{center}
   \boxed{
\begin{tabular}{ccc}
   Accumulation of a quantity &  & Accumulation of a  \\
   over the boundary of & = & derived quantity \\
   a closed domain &  &  over the entire domain
\end{tabular}
}
 \end{center}

\begin{itemize}
  \item the domain is oriented;
  \item the orientation of the domain induces an orientation of the boundary.
\end{itemize}



\underline{General Stokes Theorem}:
%
$$\boxed{\int_{\partial M} \omega = \int_M d\omega}$$
%

\begin{itemize}
  \item It would take too long to explain here what all that means,
  \item Will be happy to do so in a future course, \emph{Analysis on Manifolds}.
\end{itemize}

\end{frame}






