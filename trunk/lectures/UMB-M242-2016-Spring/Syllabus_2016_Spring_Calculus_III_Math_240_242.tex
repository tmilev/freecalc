\documentclass{article}
\usepackage{amsmath, amsfonts, amssymb, verbatim, hyperref}
\usepackage{enumitem}
\usepackage{pst-plot}
\usepackage{pstricks}
\usepackage{lscape}

\addtolength{\hoffset}{-3.5cm}
\addtolength{\textwidth}{6.8cm}
\addtolength{\voffset}{-3.3cm}
\addtolength{\textheight}{6.3cm}

\newcounter{topicsCounter}
\newcounter{topicsSubCounter}[topicsCounter]
\newcounter{topicsSubSubCounter}[topicsSubCounter]

\newcommand{\skipped}{\begin{tabular}{@{}c}optional \\reading \\not on test \end{tabular}}

\usepackage{longtable}
\usepackage{xr}
%\externaldocument{../../homework/UMB-All-Problems-By-Course/Calc-I-MasterProblemSheet}
%\externaldocument{./CalcIMasterProblemSheetOneFile}

\newcommand{\refBad}[1]{%
\ifthenelse{\equal{\ref{#1}}{??}}%
{(n/a)}%
{\ref{#1}}%
}%example usage: \refBad{\ref{eqMacLaurinDef}}{their definition}{their definition (Definition \ref{eqMacLaurinDef})}


\newcommand{\counterTopic}{\refstepcounter{topicsCounter}\thetopicsCounter. &&&}
\newcommand{\counterSubTopic}{&\refstepcounter{topicsSubCounter}\thetopicsCounter.\thetopicsSubCounter.  &&}
\newcommand{\counterSubSubTopic}{&&\refstepcounter{topicsSubSubCounter}\thetopicsCounter.\thetopicsSubCounter.\thetopicsSubSubCounter. &}

\newcommand{\apex}{A\kern -1pt \lower -2pt\hbox{P}\kern -4pt \lower .7ex\hbox{E}\kern -1pt X}


\newcommand{\websitebase}{https://piazza.com/umb/spring2016/math240242}

\usepackage{pdfpages}

\title{\vskip -2cm 
Math 242  (Calculus III)\\
Multivariable and Vector Calculus \\ Spring 2016}
\date{}
\begin{document}
%\color{green}
\maketitle

%\noindent\textbf{Time and place.}
%Monday, Wednesday, Friday 10-10:50, McCormack, Room 417, first floor. Monday 11:00-11:50,

\noindent \textbf{Instructor.} 
\begin{tabular}{ll}
Todor Milev & \href{mailto:todor.milev@gmail.com}{\nolinkurl{todor.milev@gmail.com}}  
\end{tabular}

\medskip
\noindent \textbf{Office hours. } \begin{tabular}{p{12cm}}
By appointment, or walk-in office hours: Tuesday, Thursday 12:00-14:00 (if not in the office I may be off to lunch).  Room: S-03-65.\\
\end{tabular}





\medskip \noindent \textbf{Lecture slides. }  \url{\websitebase/resources}

\medskip\noindent The lecture slides may be updated as the course progresses.


%\medskip \noindent \textbf{Master Problem Sheet. }  \url{\websitebase/resources} 

%\medskip\noindent The master problem sheet contains a collection of Calculus III problems. 

\medskip
\noindent \textbf{Homework.} You will be assigned homework, which will be posted on

\url{\websitebase/resources} \quad \quad \quad .

\noindent You will be expected to complete the homework in written form in a convenient for you format (notebook, folder, etc.). However, \textbf{the instructor(s) will not check/collect/proofread your homework.} 
 
\medskip
\noindent \textbf{Quizzes.} You will be given quizzes in class. \textbf{The time of the quiz will be announced in class}. Quizzes may be announced from one lecture day to the next. Your quiz problem will be one of your homework problems, verbatim (no number changes).



\medskip\noindent \textbf{Textbook. } The student should choose one of three textbooks for this course. The textbooks are recommended; they are considered supplementary materials to the lecture slides and should be used as a reference for solving homework problems in case the lecture slides are not sufficient. The order of exposition will follow the lecture slides and will not match exactly any of the recommended textbooks.

\begin{itemize}
\item Option I. The textbook \apex{} 3.0, Volume 3, Chapters 10-13. To download a free pdf file of the textbook, or to buy a physical copy of the textbook, visit the following site.

\url{http://www.apexcalculus.com/downloads/} 
\item Option II. James Stewart, Multivariable Calculus: Concepts,...  $4^{th}$ edition (older/newer editions are also acceptable).

\item Option III. James Stewart, Calculus Early Transcendentals, $6^{th}$ edition, Chapters 12-16 (older/newer editions are also acceptable).
\end{itemize}

%\medskip
%\noindent \textbf{Prerequisite. } A standard pre-calculus course or equivalent.


\medskip
\noindent \textbf{Grades.} Your grade will consist of two tests, a comprehensive final exam, and a number of quizzes. 
\begin{itemize}
\item The quizzes will account for 20\% of your total grade.
\item The 2 tests will account for 45\% (22.5\% each) of your total grade.
\item The final will account for 35\% of your total grade.
\end{itemize}
Please note that missed tests can not be made up, unless there is a valid medical reason accompanied with an official signed document from a medical doctor. Letter grades will be assigned according to the scale currently recommended by the Math Department Curriculum Committee. 

\begin{center}
\begin{tabular}{lc|lc}
A & 93-100 & C  & 73-75 \\
A-& 90-92  & C- & 70-72 \\
B+& 86-99  & D+ & 66-69 \\
B & 83-85  & D  & 63-65\\
B-& 80-82  & D- & 60-62\\
C+& 76-79  & F  & below 60\\
\end{tabular}

\end{center}

No books, notes, calculators or any other electronic device (such as mobile phones) are allowed during any exam unless otherwise stated.

\medskip
\noindent \textbf{Student conduct.} Students  are required to adhere the University Policy on Academic Standards and Cheating, to the University Statement of Plagiarism and the Documentation of Written Work, and to the Code of Student Conduct as described in the catalog of Undergraduate programs, pages 44-45 and 48-52. The code is available at the following web-page.

\noindent\url{http://www.umb.edu/life_on_campus/policies/code/}


%\begin{landscape}

\noindent \begin{longtable}{|@{}r@{}l@{}l@{~}l@{~}c c|}\hline
\multicolumn{6}{|c|}{\textbf{List of topics.} Some topics are marked as optional or review and may be covered only briefly.
}\\\hline
&&& Topic &  \begin{tabular}{l}Lecture  \end{tabular} & \begin{tabular}{l} Expected \\ Week \\ (total 14) \end{tabular}  \\\hline
\counterTopic Space, coordinates and coordinate systems. (Review) \\
\counterSubTopic Polar coordinates. \\
\counterSubTopic Cylindrical coordinates. \\
\counterSubTopic Spherical coordinates. \\
\counterTopic  Vectors.\\
\counterSubTopic Scalar (dot) product.\\
\counterSubTopic Projections.\\
\counterSubTopic Cross product.\\
\counterSubTopic Scalar triple product.\\
\counterSubTopic Space orientation.\\
\counterTopic Lines, Planes.\\
\counterTopic Functions of several variables.\\
\counterSubTopic Descriptions (graphical, numerical, analytical).\\
\counterSubTopic Slices and level curves.\\
\counterSubTopic Vector fields.\\
\counterTopic Quadratic surfaces.\\
\counterTopic Implicit functions.\\
\counterTopic Surface Parametrizations.\\
\counterTopic Parametrized curves.\\
\counterSubTopic Parametric curve equations.\\
\counterSubTopic Tangents.\\
\counterSubTopic Curve integrals.\\
\counterSubTopic Curve length.\\
\counterSubTopic Parametrization by arc length.\\
\counterTopic Limits of functions of several variables.\\
\counterTopic Differentiation of functions of several variables. \\
\counterSubTopic Directional derivatives. \\
\counterSubTopic Partial derivatives.\\
\counterSubTopic Tangent plane.\\
\counterSubTopic Differentiability.\\
\counterSubTopic Differentials (total).\\
\counterSubTopic Multivariable chain rule.\\
\counterSubTopic Critical points, minima and maxima.\\
\counterTopic Double integrals.\\
\counterTopic Triple integrals, volumes\\
\counterSubTopic Iterated integrals.\\
\counterSubTopic Fubini's Theorem.\\
\counterTopic Curvilinear integrals.\\
\counterTopic Parametrized surfaces.\\
\counterSubTopic Surface integrals.\\
\counterTopic Green's Theorem.\\
\counterTopic The Divergence Theorem.\\
\counterTopic Stoke's Theorem.
\end{longtable}
%\end{landscape}
\end{document}


