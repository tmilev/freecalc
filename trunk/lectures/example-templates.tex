\usepackage[latin1]{inputenc}
\usepackage{etex}
\usepackage{ifthen}
\usepackage[strings]{underscore}
\usepackage{tikz}
\usetikzlibrary{calc}
\usepackage{bbding}
\let\Cross\relax
\let\Square\relax
\usepackage{amsmath}
\usepackage{amssymb}
\usepackage{cancel}
\usepackage{comment}
\usepackage{multirow}
\usepackage{psfrag}
\usepackage{rotating}
\usepackage{fp}
\usepackage{calc}
\usepackage{bm}
\usepackage[all,cmtip]{xy}
\RequirePackage{xstring}
\usepackage{times}
\usepackage[english]{babel}


\newcommand{\autopstpdfConflictResolutionTemporary}{
\usepackage[
dvips={-o -Ppdf}, 
pspdf={
-dNOSAFER
%-dAutoRotatePages=/None %<-breaks in windows:%
}, 
pdfcrop={}, 
%crop=off%without crop=off breaks in windows
]{auto-pst-pdf}
}


%%%%%%%%%%%%%%%%%%%%%%%%%%%%%%%%%%%%%%%%%%
%
% List of commands in this document
%
%
% \logdiffbaseandexp
% \logdifftwouponedown
% \productrulefofx
% \quotientruley
% \limitradical  (broken)
% \limitsub
% \chainruley
% \chainrulefofx
% \chainruleStyleOne
% \chainruleStyleTwo
% \chainruleStyleThree
% \infinitelimit
% \limitfactor
% \newtonsmethod
% \constantmultiple
% \chainruletwice
% \youWillNotBeTested
% \optionalDisplay  %Dummy command needed for compatibility with Calculus notes.
% \Arcsin
% \Arccos
% \Arctan
% \Arccot
% \diff
%%%%%%%%%%%%%%%%%%%%%%%%%%%%%%%%%%%%%%%%%%

\newcommand{\diff}{{\normalfont \text{d}}}
\newcommand{\ds}{\displaystyle}
\newtheorem{question}{Question}
\newtheorem{emptyTheorem}{}
\newtheorem{observation}{Observation}
\newtheorem{proposition}{Proposition}
\newtheorem{remark}{Remark}
\newcommand{\youWillNotBeTested}{\begin{frame}You will not be tested on the material in the following slide.\end{frame}}
\DeclareMathOperator{\Vol}{Vol}

\DeclareMathOperator{\Arcsin}{\sin^{-1}}
\DeclareMathOperator{\Arccos}{\cos^{-1}}
\DeclareMathOperator{\Arctan}{\tan^{-1}}
\DeclareMathOperator{\Arccot}{{\cot^{-1}}}
\DeclareMathOperator{\Arcsec}{{\sec^{-1}}}
\DeclareMathOperator{\Arccsc}{{\csc^{-1}}}
\DeclareMathOperator{\sech}{sech}
\DeclareMathOperator{\csch}{csch}

\DeclareMathOperator{\maclaurin}{{\normalfont{Mc}}}
\newcommand{\taylor}{{\normalfont{T}}}

\newcommand{\optionalDisplay}[1]{#1}
\renewcommand{\Im}{\mathrm{Im}}
\renewcommand{\Re}{\mathrm{Re}}

%\DeclareMathOperator{\Re}{Re}
%\DeclareMathOperator{\Im}{Im}
\newcommand{\fcv}[1]{{\bf #1}} %this command stands for freecalc Vector
\DeclareMathOperator{\curl}{\fcv{curl}}
\DeclareMathOperator{\divg}{div}
\DeclareMathOperator{\proj}{\fcv{proj}}
\DeclareMathOperator{\orth}{\fcv{orth}}
\DeclareMathOperator{\grad}{\fcv{grad}}
\newcommand{\RR}{{\mathbb{R}}}
\newcommand{\cR}{{\mathcal{R}}}
\newcommand{\cD}{{\mathcal{D}}}
\newcommand{\cP}{{\mathcal{P}}}
\newcommand{\fcUncoverAlert}[2]{\uncover<#1->{\alert<#1>{#2}}}
\newcommand{\alertNoH}[2]{\alert<handout:0|#1>{#2}}
\newcommand{\rectangle}{{%
  \ooalign{$\sqsubset\mkern2mu$\cr$\mkern1mu\sqsupset$\cr}%
}}
\newcommand{\worksheet}[1]{\uncover<1->{#1}}
\newcommand{\turnOnWorksheetMode}{\renewcommand{\worksheet}[1]{\uncover<handout:0|1->{##1}}}
%Code from user cfr from stackexchange, from discussion:
%http://tex.stackexchange.com/questions/259138/beamer-class-creating-3-modes-presentation-handout-and-custom
%\makeatletter
%\newcommand{\turnOnWorksheetMode}{%
%\gdef\beamer@currentmode{worksheet}%
%\def\animate<##1>{\transduration<##1| handout:0| worksheet:0| trans:0>{0}}%
%}
%\makeatother
\newcommand{\fcAnswerNoH}[2]{
\FPeval{\fcResult}{clip(#1-1)}
\uncover<handout:0|\fcResult>{\alertNoH{\fcResult}{\textbf{?} }} \worksheet{\uncover<handout:0| #1->{\alertNoH{#1}{\!\!\!#2}}}
}
\newcommand{\fcAnswer}[2]{%
%\FPeval{\fcResult}{clip(#1-1)}%
\uncover<handout:0|\the\numexpr#1-1\relax>{\alertNoH{\the\numexpr#1-1\relax}{\textbf{?}}}\worksheet{\uncover<#1->{\alertNoH{#1}{\!\!\!#2}}}%
}%
\newcommand{\fcAnswerUncover}[3]{%
\FPeval{\fcResult}{clip(#2-1)}%
\uncover<handout:0|#1-\fcResult>{\alertNoH{\fcResult}{\textbf{?}}}\worksheet{\uncover<#2->{\alertNoH{#2}{\!\!\!#3}}}
%\makebox[\widthof{#3}][c]{\only<handout:0|#1-\fcResult>{\alertNoH{\fcResult}{\textbf{?}}} \only<#2->{\alertNoH{#2}{\!\!\!#3}}}%
}
\newcommand{\fcAnswerUncoverNoH}[3]{
\FPeval{\fcResult}{clip(#2-1)}
\uncover<handout:0|#1-\fcResult>{\alertNoH{\fcResult}{\textbf{?}}}\worksheet{\uncover<handout:0|#2->{\alertNoH{#2}{\!\!\!#3}}}
}

\newcommand{\fcQuestion}[2]{%
\FPeval{\fcResult}{clip(#1+1)}%
\uncover<#1->{\alertNoH{ #1,\fcResult}{#2}}%
}
\newcommand{\fcEvalToInt}[1]{\FPeval{\fcResult}{clip(#1)}\fcResult}
\newcommand{\refBad}[3]{%
\ifthenelse{\equal{#1}{??}}%
{#2}%
{#3}%
}%example usage: \refBad{\ref{eqMacLaurinDef}}{their definition}{their definition (Definition \ref{eqMacLaurinDef})}
\newcommand{\fcCancel}[2]{
\FPeval{\fcResult}{clip(#1-1)}
\only<handout:0|-\fcResult>{#2} \only<#1->{\alertNoH{#1}{\cancel{\alertNoH{0}{#2}}}}
\vphantom{\cancel{#2}}
}
%<-WARNING: the superflous-looking \alertNoH{0} is needed:
% for some unknown to me reason it causes LaTeX to add the correct amount of spacing.

%code blocks regular expression that replaces all strings of the form \alert<handout:0| a> by \alertNoH{a}:
%Find:
%\\alert<[^|^0]*0|\([^>]*\)>
%Replace:
%\\alertNoH{\1}
%code blocks regular expression that replaces all strings of the form \alert<a> but not containing | by \alertNoH{a}:
%Find:
%\\alert<\([^|^>]*\)>
%Replace:
%\\alertNoH{\1}

\newcommand{\fcLicenseContent}{
These lecture slides and their \LaTeX{} source code are licensed to you under the Creative Commons license CC BY 3.0. You are free
\begin{itemize}
\item to Share - to copy, distribute and transmit the work,
\item to Remix - to adapt, change, etc., the work,
\item to make commercial use of the work,
\end{itemize}
as long as you reasonably acknowledge the original project.
\begin{itemize}
\item Latest version of the .tex sources of the slides: \url{https://sourceforge.net/p/freecalculus/code/HEAD/tree/}
\item Should the link be outdated/moved, search for  ``freecalc project''.
\item Creative Commons license CC BY 3.0:
\url{https://creativecommons.org/licenses/by/3.0/us/}
and the links therein.
\end{itemize}
}

\newcommand{\fcLicense}{
\begin{frame}
\frametitle{License to use and redistribute}
\fcLicenseContent
\end{frame}
}
\newcommand{\onlyNoH}[2]{\only<handout:0|#1>{#2}}

%\newcommand{\fcUncover}[2]{{\newcommand{\tempExpander}{#1}\uncover<\tempExpander>{#2}}

%Produces a picture which illustrates graphically the solution of
%sin x = sin a (a=known angle).
%argument1: the number of the first frame
%argument2: the angle a
%argument3: the label of the segment representing sin a
%argument4: the label of the angle a
%argument5: the label of the other angle that solves the problem
%Example:
%\uncover<10>{}% to ensure uncovering of the whole picture
%\fcPictureSolvingSinXequalsSinA{2}{240}{$-\frac{\sqrt{3}}{2}$}{$240^\circ$}{$300^\circ$}
\newcommand{\fcPictureSolvingSinXequalsSinA}[5]{
\begin{pspicture}(-1.2, -1.2)(1.2, 1.2)
\tiny
\pstVerb{25 dict begin}
\pstVerb{%
/theBaseAngle #2\space def
/theOtherAngle theBaseAngle 360 mod 180 gt{540 theBaseAngle sub}{180 theBaseAngle sub}ifelse def
/theYcoord theBaseAngle sin def
/theXbase theBaseAngle cos def
/theXother theOtherAngle cos def
}%
\fcAxesStandard{-1.2}{-1.2}{1.2}{1.2}%
\parametricplot[linecolor=\fcColorGraph]{0}{360}{t cos t sin}%
\uncover<#1->{%
\psline[linecolor=blue, linewidth=2pt](0,0)(! 0 theYcoord )%
\rput[l](! 0 theYcoord 2 div){#3}%
}%
%\uncover<#3, \the\numexpr#3+1\relax>{\psline[linecolor=blue, linewidth=2pt](0,0)(! 0 theYcoord)}%
\uncover<handout:0|\the\numexpr#1+4\relax>{%
\parametricplot[linecolor=blue, linewidth=2pt, plotpoints=500, arrows=->]{0}{theBaseAngle 360 add}{t cos 0.35 t 3000 div add mul t sin 0.35 t 3000 div add mul}%
}%
\uncover<\the\numexpr#1+2\relax,\the\numexpr#1+3\relax>{\parametricplot[arrows=->, linecolor=purple, linewidth=2pt]{0}{theBaseAngle }{t cos 0.3 mul t sin 0.3 mul}}%
\uncover<\the\numexpr#1+2\relax->{%
%\fcPerpendicular{[theXbase theYcoord]}{[0 1]}{0.1}%
%\fcPerpendicular{[theXother theYcoord]}{[0 1]}{0.1}%
\psline[arrows=->](0,0)(! theXbase 1.2 mul theYcoord 1.2 mul)%
\parametricplot[arrows=->, linecolor=purple]{0}{theBaseAngle }{t cos 0.3 mul t sin 0.3 mul}%
\rput(! theBaseAngle 2 div dup cos 0.4 mul exch sin 0.4 mul){\fcAnswer{\the\numexpr#1+3\relax}{#4}}%
}%
\uncover<handout:0|\the\numexpr#1+7\relax>{%
\parametricplot[linecolor=brown, linewidth=2pt, plotpoints=500, arrows=->]{0}{theOtherAngle 360 add}{t cos 0.55 t 3000 div add mul t sin 0.55 t 3000 div add mul}%
}%
\uncover<\the\numexpr#1+5\relax,\the\numexpr#1+6\relax>{\parametricplot[arrows=->, linewidth=2pt, linecolor=orange]{0}{theOtherAngle}{t cos 0.5 mul t sin 0.5 mul}}%
\uncover<\the\numexpr#1+5\relax->{%
%\fcPerpendicular{[theXbase theYcoord]}{[0 1]}{0.1}%
%\fcPerpendicular{[theXother theYcoord]}{[0 1]}{0.1}%
\psline[arrows=->](0,0)(! theXother 1.2 mul theYcoord 1.2 mul)%
\parametricplot[arrows=->, linecolor=orange]{0}{theOtherAngle}{t cos 0.5 mul t sin 0.5 mul}%
\rput(! theOtherAngle 2 div dup cos 0.6 mul exch sin 0.6 mul){\fcAnswer{\the\numexpr#1+6\relax}{#5}}%
}%
\uncover<\the\numexpr#1+1\relax->{%
%\fcPerpendicular{[theXbase theYcoord]}{[0 1]}{0.1}%
%\fcPerpendicular{[theXother theYcoord]}{[0 1]}{0.1}%
\psline[linecolor=green, linewidth=2pt](! theXother theYcoord)(! theXbase theYcoord)%
\fcFullDot{theXother}{theYcoord}%
\fcFullDot{theXbase}{theYcoord}%
}%
\pstVerb{end}%
\end{pspicture}
}

%Produces a picture which illustrates graphically the solution of
%sin x = sin a (a=known angle).
%argument1: the number of the first frame
%argument2: the angle a
%argument3: the label of the segment representing sin a
%argument4: the label of the angle a
%argument5: the label of the other angle that solves the problem
\newcommand{\fcPictureSolvingCosXequalsCosA}[5]{
\begin{pspicture}(-1.2, -1.2)(1.2, 1.2)
\tiny
\pstVerb{25 dict begin}
\pstVerb{%
/theBaseAngle #2\space def
/theOtherAngle theBaseAngle -1 mul def
/theXcoord theBaseAngle cos def
/theYbase theBaseAngle sin def
/theYother theOtherAngle sin def
}%
\fcAxesStandard{-1.2}{-1.2}{1.2}{1.2}%
\parametricplot[linecolor=\fcColorGraph]{0}{360}{t cos t sin}%
\uncover<#1->{%
\psline[linecolor=green, linewidth=2pt](0,0)(! theXcoord 0)%
\rput[t](! theXcoord 2 div -0.02){#3}%
}%
%\uncover<#3, \the\numexpr#3+1\relax>{\psline[linecolor=blue, linewidth=2pt](0,0)(! 0 theYcoord)}%
\uncover<handout:0|\the\numexpr#1+4\relax>{%
\parametricplot[linecolor=blue, linewidth=2pt, plotpoints=500, arrows=->]{0}{theBaseAngle 360 add}{t cos 0.35 t 3000 div add mul t sin 0.35 t 3000 div add mul}%
}%
\uncover<\the\numexpr#1+2\relax,\the\numexpr#1+3\relax>{\parametricplot[arrows=->, linecolor=purple, linewidth=2pt]{0}{theBaseAngle }{t cos 0.3 mul t sin 0.3 mul}}%
\uncover<\the\numexpr#1+2\relax->{%
%\fcPerpendicular{[theXbase theYcoord]}{[0 1]}{0.1}%
%\fcPerpendicular{[theXother theYcoord]}{[0 1]}{0.1}%
\psline[arrows=->](0,0)(! theXcoord 1.2 mul theYbase 1.2 mul)%
\parametricplot[arrows=->, linecolor=purple]{0}{theBaseAngle }{t cos 0.3 mul t sin 0.3 mul}%
\rput(! theBaseAngle 2 div dup cos 0.4 mul exch sin 0.4 mul){\fcAnswer{\the\numexpr#1+3\relax}{#4}}%
}%
\uncover<handout:0|\the\numexpr#1+7\relax>{%
\parametricplot[linecolor=brown, linewidth=2pt, plotpoints=500, arrows=->]{0}{theOtherAngle 360 sub}{t cos 0.55 t 3000 div add mul t sin 0.55 t 3000 div add mul}%
}%
\uncover<handout:0|\the\numexpr#1+8\relax>{%
\parametricplot[linecolor=brown, linewidth=2pt, plotpoints=500, arrows=->]{0}{theOtherAngle 360 add}{t cos 0.55 t 3000 div add mul t sin 0.55 t 3000 div add mul}%
}%
\uncover<\the\numexpr#1+5\relax,\the\numexpr#1+6\relax>{\parametricplot[arrows=->, linewidth=2pt, linecolor=orange]{0}{theOtherAngle}{t cos 0.5 mul t sin 0.5 mul}}%
\uncover<\the\numexpr#1+5\relax->{%
%\fcPerpendicular{[theXbase theYcoord]}{[0 1]}{0.1}%
%\fcPerpendicular{[theXother theYcoord]}{[0 1]}{0.1}%
\psline[arrows=->](0,0)(! theXcoord 1.2 mul theYother 1.2 mul)%
\parametricplot[arrows=->, linecolor=orange]{0}{theOtherAngle}{t cos 0.5 mul t sin 0.5 mul}%
\rput(! theOtherAngle 2 div dup cos 0.63 mul exch sin 0.63 mul){\fcAnswer{\the\numexpr#1+6\relax}{#5}}%
}%
\uncover<\the\numexpr#1+1\relax->{%
%\fcPerpendicular{[theXbase theYcoord]}{[0 1]}{0.1}%
%\fcPerpendicular{[theXother theYcoord]}{[0 1]}{0.1}%
\psline[linecolor=blue, linewidth=2pt](! theXcoord theYother)(! theXcoord theYbase)%
\fcFullDot{theXcoord}{theYother}%
\fcFullDot{theXcoord}{theYbase}%
}%
\pstVerb{end}%
\end{pspicture}
}

%
%  An example of logarithmic differentiation of a function with a
%  variable base and exponent.
%  #1 is the base.
%  #2 is the exponent.
%  #3 is the derivative of the natural logarithm of the base.
%  #4 is the derivative of the exponent.
%  #5 is (base)(exponent)' + (exponent)(base)' after simplification.
%
\newcommand{\logdiffbaseandexp}[5]{
\begin{example}[Variable base and exponent]
\abovedisplayskip=0pt
\belowdisplayskip=0pt
\abovedisplayshortskip=0pt
\belowdisplayshortskip=0pt
\begin{align*}
\text{Differentiate}\quad \alertNoH{ 13}{y} %
 & \alertNoH{ 13}{=} %
\alertNoH{ 13}{%
#1^{#2}%
}.%
\uncover<2->{%
\intertext{
Take logarithms of both sides:%
}
}%
\uncover<2->{%
\ln y
}%
 & \uncover<2->{ = } %
\uncover<2->{%
\ln #1^{\alertNoH{ 3}{#2}}%
}\\%
\uncover<3->{%
\alertNoH{ 4-5}{\ln y}%
}%
 & \uncover<3->{ = } %
\uncover<3->{%
\alertNoH{ 6-7}{%
\alertNoH{ 3}{#2} \ln #1%
}.}%
\uncover<4->{%
\intertext{
Differentiate implicitly with respect to $x$:%
}%
}%
\fcAnswer{5}{\frac{1}{y} y'}%
 & \uncover<4->{ = } %
\fcAnswerUncover{4}{7}{%
\left( #2 \right) \alertNoH{ 8-9}{\frac{\diff}{\diff x} \left( \ln #1 \right)} + \left( \ln #1 \right)\alertNoH{ 10-11}{\frac{\diff}{\diff x}\left( #2 \right)} %
}\\%
\uncover<8->{%
\frac{1}{\alertNoH{12}{y}} y'%
}%
 & \uncover<8->{ = } %
\uncover<8->{%
( #2 ) \alertNoH{8-9}{\left( \fcAnswerUncover{8}{9}{ #3 }\right)} + \left( \ln #1 \right) \alertNoH{ 10-11}{ \left( \fcAnswerUncover{8}{11}{ #4 } \right) }
}\\%
\uncover<12->{%
y'%
}%
 & \uncover<12->{ = } %
\uncover<12->{%
\alertNoH{ 12-13}{y} \left( #5 \right)%
}\\%
 & \uncover<13->{ = } %
\uncover<13->{%
\alertNoH{ 13}{#1^{#2}} \left( #5 \right).%
}%
\end{align*}
\end{example}
}


%
%  An example of logarithmic differentiation of a function.
%  It looks as follows:
%
%  Differentiate y = (#1 #2)/#3.
%  Take logarithms of both sides:
%  ln y = ln((#1 #2)/#3)
%  ln y = ln#1 + ln#2 - ln#3
%  ln y = #4 + #5 - #6
%  Differentiate implicitly with respect to x:
%  (1/y)y' = #7 + #8 - #9
%  y' = y(#7 + #8 - #9)
%  y' = ((#1 #2)/#3)(#7 + #8 - #9)
%
\newcommand{\logdifftwouponedown}[9]{
\begin{example}[Logarithmic Differentiation%
]
\abovedisplayskip=0pt
\belowdisplayskip=0pt
\abovedisplayshortskip=0pt
\belowdisplayshortskip=0pt
\begin{align*}
\text{Differentiate}\quad \alertNoH{ 18}{y} %
 & \alertNoH{ 18}{=} %
\alertNoH{ 18}{%
\frac{#1 #2}{#3}%
}.%
\uncover<2->{%
\intertext{
Take logarithms of both sides:%
}
}%
\uncover<2->{%
\ln y
}%
 & \uncover<2->{ = } %
\uncover<2->{%
\ln \frac{\alertNoH{ 3-4}{#1}\alertNoH{ 5-6}{#2}}{\alertNoH{ 7-8}{#3}}%
}\\%
\uncover<2->{%
\ln y
}%
 & \uncover<2->{ = } %
\uncover<2->{%
\ln \alertNoH{ 3-4}{#1} + \ln \alertNoH{ 5-6}{#2} -  \ln \alertNoH{ 7-8}{#3}%
}\\%
\uncover<3->{%
\alertNoH{ 9-10}{\ln y}%
}%
 & \uncover<3->{ = } %
\uncover<3->{%
\alertNoH{ 3-4,11-12}{%
\left( \uncover<4->{#4}\right) %
}%
\alertNoH{ 5-6}{%
\uncover<6->{+} \alertNoH{ 13-14}{\left( \uncover<6->{#5}\right)} %
}%
\alertNoH{ 7-8}{%
\uncover<8->{-} \alertNoH{ 15-16}{\left( \uncover<8->{#6}\right)} %
}%
}%
\uncover<9->{%
\intertext{
Differentiate implicitly with respect to $x$:%
}%
}%
\uncover<10->{%
\alertNoH{ 10}{\frac{1}{\alertNoH{ 17}{y}} y'}%
}%
 & \uncover<9->{ = } %
\uncover<9->{%
\alertNoH{ 11-12}{\left( \uncover<12->{#7} \right)} + %
\alertNoH{ 13-14}{\left( \uncover<14->{#8} \right)} - %
\alertNoH{ 15-16}{\left( \uncover<16->{#9} \right)} %
}\\%
\uncover<17->{%
y'%
}%
 & \uncover<17->{ = } %
\uncover<17->{%
\alertNoH{ 17-18}{y} \left( #7 + #8 - #9 \right)%
}\\%
 & \uncover<18->{ = } %
\uncover<18->{%
\alertNoH{ 18}{\frac{#1 #2}{#3}} \left( #7 + #8 - #9 \right)%
}%
\end{align*}
\end{example}
}


%
%  An example of a derivative with the Product Rule, using the symbol f(x).
%  It looks as follows:
%
%  Differentiate f(x) = #1 #2.
%  Product Rule: f'(x) = (#1)(d/dx)(#2) + (#2)(d/dx)(#1)
%   = (#1)(#4) + (#2)(#3)
%   = #5.
%
%  #6 appears in the subtitle of the example.
%
\newcommand{\productrulefofx}[6]{%
\begin{example}[Product Rule%
\ifthenelse{\equal{#6}{0}}%
{}%
{, #6}%
]%
\abovedisplayskip=0pt
\belowdisplayskip=0pt
\abovedisplayshortskip=0pt
\belowdisplayshortskip=0pt
\begin{align*}
\text{Differentiate}\quad f(x) & = \alertNoH{2}{ #1}\alertNoH{3}{ #2.}\\%
\uncover<2->{%
\text{Product Rule:}\quad f'(x)%
}%
& \uncover<2->{%
 =  \alertNoH{ 6-7}{\frac{\diff}{\diff x}\left( \alertNoH{2}{#1} \right)}\left( \alertNoH{3}{#2} \right)+\left( \alertNoH{2}{#1} \right) \alertNoH{ 4-5}{\frac{\diff}{\diff x}\left( \alertNoH{3}{#2} \right)} %
}\\%
& \uncover<4->{%
 = \alertNoH{ 6-7}{\left( \fcAnswerUncover{4}{7}{#3} \right)}\left( #2 \right)+ \left( #1 \right) \alertNoH{ 4-5}{\left(\fcAnswer{5}{ #4 }\right)}  %
}\\%
& \uncover<8->{%
 = #5.%
}%
\end{align*}
\end{example}
}


%
%  An example of a derivative with the Constant Multiple Rule.
%  It looks as follows:
%
%  Find the derivative of #1 = #2.
%   #1 = (#3)(#4).
%   d#1/dx = (d/dx)((#3)(#4))
% Constant Multiple Rule: = (#3)(d/dx)(#4)
%   = (#3)(#5)
%   = #6.
%
%  #7 appears in the subtitle of the example.
%
\newcommand{\constantmultiple}[7]{%
\begin{example}[Constant Multiple Rule%
\ifthenelse{\equal{#7}{0}}%
{}%
{, #7}%
]%
\abovedisplayskip=0pt
\belowdisplayskip=0pt
\abovedisplayshortskip=0pt
\belowdisplayshortskip=0pt
\begin{align*}
\text{Find the derivative of}\quad #1 & = #2.\\%
\uncover<2->{%
#1 %
}%
& \uncover<2->{%
 = \left( #3\right)\left( #4\right).
}\\%
\uncover<3->{%
\frac{\diff #1}{\diff x} %
}%
& \uncover<3->{%
 = \frac{\diff}{\diff x}\left[ \alertNoH{ 4}{\left( #3\right)}\left( #4\right)\right]
}\\%
\uncover<4->{%
\text{Constant Multiple Rule:}\quad %
}%
& \uncover<4->{%
 =  \alertNoH{ 4}{\left( #3\right)}\alertNoH{ 5-6}{\frac{\diff}{\diff x}\left( #4\right)}
}\\%
& \uncover<5->{%
 =  \left( #3\right)\alertNoH{ 5-6}{\left( \fcAnswer{6}{#5}\right)}
}\\%
& \uncover<7->{%
 =  #6.
}%
\end{align*}
\end{example}
}


%
%  An example of a derivative with the Quotient Rule, using the symbol y.
%  It looks as follows:
%
%  Differentiate y = #1 / #2.
%  Quotient Rule: dy/dx = ((#2)(d/dx)(#1)-(#1)(d/dx)(#2))/(#2)^2
%   = ((#2)(#3)-(#1)(#4))/(#2)^2
%   = #5
%   = #6.
%
%  #7 appears in the subtitle of the example.
%
\newcommand{\quotientruley}[7]{%
\begin{example}[Quotient Rule%
\ifthenelse{\equal{#7}{0}}%
{}%
{, #7}%
]%
\abovedisplayskip=0pt
\belowdisplayskip=0pt
\abovedisplayshortskip=0pt
\belowdisplayshortskip=0pt
\begin{align*}
\text{Differentiate}\quad y & = \frac{\alertNoH{2}{ #1}}{\alertNoH{3}{#2}}.%
\uncover<2->{%
\intertext{Quotient Rule:}%
}%
%&\\%
\uncover<2->{%
\frac{\diff y}{\diff x}%
}%
& \uncover<2->{%
 = \frac%
{ \alertNoH{ 4-5}{\frac{\diff}{\diff x}\left( \alertNoH{2}{ #1} \right)}\left( \alertNoH{3}{#2} \right) - \left( \alertNoH{2}{#1} \right) \alertNoH{ 6-7}{\frac{\diff}{\diff x}\left( \alertNoH{3}{#2} \right)}}%
{\left( \alertNoH{3}{#2}\right)^2}%
}\\%
& \uncover<4->{%
 = \frac%
{\alertNoH{ 4-5}{\left(\fcAnswer{5}{ #3 }\right)}\left( #2 \right)  - \left( #1 \right) \alertNoH{ 6-7}{\left( \fcAnswerUncover{4}{7}{#4} \right)}}%
{\left( #2\right)^2}%
}\\%
& \uncover<8->{%
 = #5%
}\\%
& \uncover<9->{%
 = #6.%
}%
\end{align*}
\end{example}
}

%
%  An example of an indefinite integral with the Substitution Rule.
%  It looks as follows:
%
%  Find \int (#1, with nothing substituted for UU and VV).
%  Let u = #2
%  Then du = #3.
%  Therefore #4 = #5.
%  Substitute: \int (#1, with the alert command for u and du
%          substituted for UU and VV respectively)
%  = \int (#6, with the alert command for u and du substituted for UU and VV)
%  = (#7, with u substituted for UU) + C
%  = (#8, with #2 substituted for UU) + C
%
%  #9 appears in the subtitle of the example.
%
\newcommand{\subrule}[9]{%
\begin{example}[Substitution Rule%
\ifthenelse{\equal{#9}{0}}%
{}%
{, #9}%
]%
\abovedisplayskip=0pt
\belowdisplayskip=0pt
\abovedisplayshortskip=0pt
\belowdisplayshortskip=0pt
\begin{align*}
\text{Find}\quad \int %
 \noexpandarg\exploregroups\StrSubstitute{\StrSubstitute{#1}{UU}{3}}{VV}{6-7}\noexploregroups\expandarg. & \\%
\uncover<2->{%
\text{Let}\quad\alertNoH{2-3, 8, 14}{u}%
}%
& \uncover<2->{%
\alertNoH{ 2-3,8,14}{ = \fcAnswer{3}{#2.}}%
}\\%
\uncover<4->{%
\text{Then}\quad \alertNoH{ 4-5,7}{\diff u}%
}%
& \uncover<4->{%
\alertNoH{ 4-5}{ = \fcAnswer{5}{#3}}%
}\\%
\uncover<6->{%
\alertNoH{ 6-7,10}{#4}%
}%
& \uncover<6->{%
\alertNoH{ 6-7,10}{ = \fcAnswer{7}{#5.}}%
}\\%
\uncover<8->{%
\text{Substitute:}\quad \int%
 \noexpandarg\exploregroups\StrSubstitute{\StrSubstitute{#1}{UU}{8}}{VV}{9,10}\noexploregroups\expandarg}%
& \uncover<8->{= \alertNoH{ 11-12}{\int\noexpandarg\exploregroups\StrSubstitute{\StrSubstitute{#6}{UU}{8}}{VV}{10}\noexploregroups\expandarg %
}}\\%
& \uncover<11->{%
= \fcAnswer{12}{\noexpandarg\exploregroups \StrSubstitute{#7}{UU}{\alertNoH{ 14}{u}}\noexploregroups\expandarg} \uncover<13->{\alertNoH{ 13}{+C}}%
}\\%
& \uncover<14->{%
 = \noexpandarg\exploregroups \StrSubstitute{#8}{UU}{\alertNoH{14}{#2}}\noexploregroups\expandarg +C.%
}%
\end{align*}
\end{example}
}

%
%  An example of a definite integral with the Substitution Rule.
%  There are nine arguments to the function.  The ninth is a string of four
%  groups of the form {AA}{BB}{CC}{DD} where AA is the lower limit of
%  integration, BB is the upper limit of integration, CC is the lower limit
%  of integration with respect to u, and DD is the upper limit of integration
%  with respect to u.
%  It looks as follows:
%
%  Find \int_{AA}^{BB} (#1, with nothing substituted for UU and VV).
%  Let u = #2
%  Then du = #3.
%  #4 = #5.
%  When x = AA, u = CC.
%  When x = BB, u = DD.
%  Substitute: \int_{AA}^{BB} (#1, with the alert command for u and du
%          substituted for UU and VV respectively)
%  = \int_{CC}^{DD} (#6, with the alert command for u and du substituted for UU and VV)
%  = [#7, with u substituted for UU]_{CC}^{DD}
%  = #8.
%
%
\newcommand{\subruledefbounds}[9]{%
\begin{example}[Substitution Rule, Definite Integral%
]%
\abovedisplayskip=0pt
\belowdisplayskip=0pt
\abovedisplayshortskip=0pt
\belowdisplayshortskip=0pt
\begin{align*}
\text{Find}\quad \int%
_{\StrMid{#9}{1}{1}}%
^{\StrMid{#9}{2}{2}} %
 \noexpandarg\exploregroups\StrSubstitute{\StrSubstitute{#1}{UU}{3}}{VV}{6-7}\noexploregroups\expandarg. & \\%
\uncover<2->{%
\text{Let}\quad\alertNoH{ 2-3,8-12}{u}%
}%
& \uncover<2->{%
\alertNoH{ 2-3,8-12}{ = \uncover<3->{#2.}}%
}\\%
\uncover<4->{%
\text{Then}\quad \alertNoH{ 4-5}{\diff u}%
}%
& \uncover<4->{%
\alertNoH{ 4-5}{ = \uncover<5->{#3}}%
}\\%
\uncover<6->{%
\alertNoH{ 6-7,13}{#4}%
}%
& \uncover<6->{%
\alertNoH{ 6-7,13}{ = \uncover<7->{#5.}}%
}\\%
\uncover<8->{%
\alertNoH{ 8-9,14}{\text{When } x = \StrMid{#9}{1}{1}, \quad u }%
}%
& \uncover<8->{%
\alertNoH{ 8-9,14}{ = \uncover<9->{\StrMid{#9}{3}{3}.}}%
}\\%
\uncover<10->{%
\alertNoH{ 10-11,15}{\text{When } x = \StrMid{#9}{2}{2}, \quad u }%
}%
& \uncover<10->{%
\alertNoH{ 10-11,15}{ = \uncover<11->{\StrMid{#9}{4}{4}.}}%
}\\%
\uncover<12->{%
\text{Substitute:}\quad \int%
_{\alertNoH{ 14}{\StrMid{#9}{1}{1}}}%
^{\alertNoH{ 15}{\StrMid{#9}{2}{2}}} %
 \noexpandarg\exploregroups\StrSubstitute{\StrSubstitute{#1}{UU}{12}}{VV}{13}\noexploregroups\expandarg}%
& \uncover<12->{= \alertNoH{ 16-17}{{\int}%
_{\uncover<14->{\alertNoH{ 14}{
\StrMid{#9}{3}{3}}}}%
^{\uncover<15->{
\alertNoH{ 15}{
\StrMid{#9}{4}{4}}}} %
\noexpandarg\exploregroups\StrSubstitute{\StrSubstitute{#6}{UU}{12}}{VV}{13}\noexploregroups\expandarg %
}}\\%
& \uncover<16->{\alertNoH{ 16-17}{%
 = {\left[ \uncover<17->{%
\noexpandarg\exploregroups\StrSubstitute{#7}{UU}{u}\noexploregroups\expandarg %
}\right]}_{\StrMid{#9}{3}{3}}^{\StrMid{#9}{4}{4}}%
}}\\%
& \uncover<18->{%
 = #8.
}%
\end{align*}
\end{example}
}


%
%  An example of a definite integral with the Substitution Rule.
%  There are nine arguments to the function.  The ninth is a string of two
%  groups of the form {AA}{BB} where AA is the lower limit of
%  integration and BB is the upper limit of integration.
%  It looks as follows:
%
%  Find \int_{AA}^{BB} (#1, with nothing substituted for UU and VV).
%  Let u = #2
%  Then du = #3.
%  #4 = #5.
%  Substitute: \int (#1, with the alert command for u and du
%          substituted for UU and VV respectively)
%  = \int (#6, with the alert command for u and du substituted for UU and VV)
%  = #7, with u substituted for UU
%  = #8.
%  Therefore int_{AA}^{BB} (#1, with nothing substituted for UU and VV)
%      = [#8]_{AA}^{BB}
%  = #9.
%
%
\newcommand{\subruledefvar}[9]{%
\begin{example}[Substitution Rule, Definite Integral%
]%
\abovedisplayskip=0pt
\belowdisplayskip=0pt
\abovedisplayshortskip=0pt
\belowdisplayshortskip=0pt
\begin{align*}
\text{Find}\quad \int%
_{\StrMid{#9}{1}{1}}%
^{\StrMid{#9}{2}{2}} %
 \noexpandarg\exploregroups\StrSubstitute{\StrSubstitute{#1}{UU}{3}}{VV}{6-7}\noexploregroups\expandarg. & \\%
\uncover<2->{%
\text{Let}\quad\alertNoH{ 2-3,8,12}{u}%
}%
& \uncover<2->{%
\alertNoH{ 2-3,8,12}{ = \uncover<3->{#2.}}%
}\\%
\uncover<4->{%
\text{Then}\quad \alertNoH{ 4-5}{\diff u}%
}%
& \uncover<4->{%
\alertNoH{ 4-5}{ = \uncover<5->{#3}}%
}\\%
\uncover<6->{%
\alertNoH{ 6-7,9}{#4}%
}%
& \uncover<6->{%
\alertNoH{ 6-7,9}{ = \uncover<7->{#5.}}%
}\\%
\uncover<8->{%
\text{Substitute:}\quad \int%
 \noexpandarg\exploregroups\StrSubstitute{\StrSubstitute{#1}{UU}{8}}{VV}{9}\noexploregroups\expandarg}%
& \uncover<8->{= \alertNoH{ 10-11}{{\int}%
\noexpandarg\exploregroups\StrSubstitute{\StrSubstitute{#6}{UU}{8}}{VV}{9}\noexploregroups\expandarg %
}}\\%
& \uncover<10->{%
 \alertNoH{ 10-11}{ = \uncover<11->{%
\noexpandarg\exploregroups{\StrSubstitute{#7}{UU}{\alertNoH{ 12}{u}}}\noexploregroups\expandarg%
}}%
  \uncover<12->{%
 = \noexpandarg\exploregroups{\StrSubstitute{#7}{UU}{\alertNoH{ 12}{#2}}}\noexploregroups\expandarg.%
}%
}\\%
\uncover<13->{%
\text{Therefore}\quad \int%
_{\StrMid{#9}{1}{1}}%
^{\StrMid{#9}{2}{2}} %
 \noexpandarg\exploregroups\StrSubstitute{\StrSubstitute{#1}{UU}{0}}{VV}{0}\noexploregroups\expandarg}%
& \uncover<13->{%
 = \left[%
 \noexpandarg\exploregroups{\StrSubstitute{#7}{UU}{#2}}\noexploregroups\expandarg%
\right]%
_{\StrMid{#9}{1}{1}}%
^{\StrMid{#9}{2}{2}} %
}\\%
& \uncover<14->{%
 = #8.
}%
\end{align*}
\end{example}
}

%
%  An example of a derivative with the Chain Rule, using the symbol y.
%  It looks as follows:
%
%  Differentiate y = #1.
%  Let u = #2
%  Then y = #3
%  Chain Rule: dy/dx = (dy/du)(du/dx)
%  = (#4, with u substituted for UU)(#5)
%  = #6, with #2 substituted for UU
%
%  #7 appears in the subtitle of the example.
%
\newcommand{\chainruley}[7]{%
\begin{example}[Chain Rule%
\ifthenelse{\equal{#7}{0}}%
{}%
{, #7}%
]%
\abovedisplayskip=0pt
\belowdisplayskip=0pt
\abovedisplayshortskip=0pt
\belowdisplayshortskip=0pt
\begin{align*}
\text{Differentiate}\quad y & = #1.\\%
\uncover<2->{%
\text{Let}\quad\alertNoH{ 2-3,8-10}{u}%
}%
& \uncover<2->{%
\alertNoH{ 2-3,8-10}{ = \fcAnswer{3}{\worksheet{#2}.}}%
}\\%
\uncover<4->{%
\text{Then}\quad \alertNoH{ 6-7}{y}%
}%
& \uncover<4->{%
\alertNoH{ 6-7}{ = \fcAnswer{4}{\worksheet{#3}.}}%
}\\%
\uncover<5->{%
\text{Chain Rule:}\quad%
\frac{\diff y}{\diff x}%
}%
& \uncover<5->{%
 = \alertNoH{ 6-7}{\frac{\diff y}{\diff u}}%
\alertNoH{ 8-9}{\frac{\diff u}{\diff x}}%
}\\%
& \uncover<6->{%
 = \alertNoH{ 6-7}{\left( \fcAnswer{7}{\worksheet{  \noexpandarg\exploregroups\StrSubstitute{#4}{UU}{\alertNoH{ 10}{u}}\noexploregroups\expandarg}} \right)}%
\alertNoH{ 8-9}{\left( \fcAnswer{9}{\worksheet{#5}}\right)}%
}\\%
& \uncover<10->{ =  \worksheet{ \noexpandarg\exploregroups \StrSubstitute{#6}{UU}{\alertNoH{ 10}{#2}}.\noexploregroups\expandarg}}%
\end{align*}
\end{example}
}





%
%  An example of a derivative with the Chain Rule, using the symbol f(x).
%  It looks as follows:
%
%  Differentiate f(x) = #1.
%  Let h(x) = #2
%  Let g(x) = #3
%  Then f(x) = g(h(x))
%  f'(x) = g'(h(x))h'(x)
%  = (#4, with h(x) substituted for UU)(#5)
%  = #6, with #2 substituted for UU
%
%  #7 appears in the subtitle of the example.
%
\newcommand{\chainrulefofx}[7]{%
\begin{example}[Chain Rule%
\ifthenelse{\equal{#7}{0}}%
{}%
{, #7}%
]%
\abovedisplayskip=0pt
\belowdisplayskip=0pt
\abovedisplayshortskip=0pt
\belowdisplayshortskip=0pt
\begin{align*}
\text{Differentiate}\quad f(x) & = #1.\\%
\uncover<2->{%
\text{Let}\quad\alertNoH{ 2-3,9-11}{h(x)}%
}%
& \uncover<2->{%
\alertNoH{ 2-3,9-11}{ = \fcAnswerNoH{3}{#2.}}%
}\\%
\uncover<2->{%
\text{Let}\quad\alertNoH{ 4-5,7-8}{g(x)}%
}%
& \uncover<2->{%
\alertNoH{ 4-5,7-8}{ = \fcAnswerUncover{2}{5}{#3.}}%
}\\%
\uncover<2-| handout:0>{%
\text{Then}\quad f(x)%
}%
& \uncover<2-| handout:0>{%
 = g(h(x)).%
}\\%
\uncover<6-| handout:0>{%
\text{Chain Rule:}\quad%
f'(x)%
}%
& \uncover<6-| handout:0>{%
 = \alertNoH{ 7-8}{g'(h(x))}%
\alertNoH{ 9-10}{h'(x)}%
}\\%
& \uncover<7-| handout:0>{%
=}\uncover<7-| handout:0>{\alertNoH{ 7-8}{\left( \fcAnswerNoH{8}{\noexpandarg\exploregroups\StrSubstitute{#4}{UU}{\alertNoH{ 11}{h(x)}}\noexploregroups\expandarg}\right)}%
\alertNoH{ 9-10}{\left( \fcAnswerUncoverNoH{7}{10}{#5}\right)}%
}\\%
& \uncover<11-| handout:0>{=} \uncover<11-| handout:0>{%
 \noexpandarg \exploregroups \StrSubstitute{#6}{UU}{\alertNoH{ 11}{#2}}.\noexploregroups \expandarg%
}%
\end{align*}
\end{example}
}

%
%  Similar to chainrulefofx but in different style.
%  It looks as follows:
%
%  Recall the chain rule (...).
%******************************
%  Differentiate f(x) = #1.
%  h(x) = #2
%  Let g(u) = #3
%  Then g'(u)=#4
%  Then f(x) = g(u)
%  f'(x) = g'(u)h'(x)
%  = (#4, with h(x) substituted for UU)(#5)
%  = #6, with #2 substituted for UU
%
%  #7 appears in the subtitle of the example.
%
\newcommand{\chainruleStyleOne}[7]{%
{\renewcommand{\arraystretch}{1.2}
$
\begin{array}{rclll}
\alertNoH{1-}{\left(g(h(x))\right)'}&\alertNoH{1-}{=}&\alertNoH{1-}{g'(h(x))\cdot  h'(x)}&& \text{(notation 1)} {~~~~~~~~~~~~~~~~~~~~~~~~~~~~~~~~~~~~} \\
(g(u))'&\alertNoH{0}{=}&g'(u) u'&\text{where } u=h(x)& \text{(notation 2)}\\
\displaystyle\frac{\diff y}{\diff x} &\alertNoH{0}{=}& \displaystyle\frac{\diff y}{\diff u}  \frac{\diff u}{\diff x} &\text{where } y=g(u)& \text{(notation 3)}\quad.\\
\end{array}
$
}
\begin{example}[Chain Rule, Notation 1%
\ifthenelse{\equal{#7}{0}}%
{}%
{, #7}%
]%
\[
\begin{array}{rrcl}
\text{Differentiate } & f(x) & =& #1.\\%
\uncover<2->{%
\text{Let}&\alertNoH{2-3,9-11}{h(x)}%
}%
&\uncover<2-| handout:0>{\alertNoH{2-3, 9-11}{ = }} &\displaystyle \uncover<2-| handout:0>{%
\alertNoH{2-3,9-11}{ \fcAnswerNoH{3}{#2.}}%
}\\%
\uncover<2->{%
\text{Let}&\alertNoH{4-5,7-8}{g(u)}%
}
&\uncover<2->{\alertNoH{4-5,7-8}{=}}&\displaystyle
\uncover<2->{\alertNoH{4-5,7-8}{ \fcAnswerUncover{2}{5}{\uncover<5-| handout:0>{#3.}}}%
}\\%
\uncover<2-| handout:0>{%
\text{Then}& f(x)
}%
&\uncover<2-| handout:0>{{=}}&\uncover<2-| handout:0>{%
 g(h(x)).%
}\\%
\uncover<6->{%
\text{Chain Rule:} &
f'(x)%
}%
&\uncover<6->{=}& \uncover<6->{%
 \alertNoH{ 7-8}{g'(h(x))}%
\alertNoH{ 9-10}{h'(x)}%
}\\%
&&\uncover<7->{=}& \displaystyle
\uncover<7->{\alertNoH{ 7-8}{ \left( \fcAnswerUncoverNoH{7}{8}{\noexpandarg \exploregroups \StrSubstitute{#4}{UU}{\alertNoH{ 11}{h(x)}} \noexploregroups\expandarg}\right)}%
\alertNoH{ 9-10}{\left( \fcAnswerUncoverNoH{7}{10}{#5}\right)}%
}\\%
&&\uncover<11-| handout:0>{=}&\displaystyle \uncover<11-| handout:0>{%
 \noexpandarg \exploregroups \StrSubstitute{#6}{UU}{\alertNoH{ 11}{#2}}.\noexploregroups \expandarg%
}%
\end{array}
\]
\end{example}
}

%
%  Similar to chainrulefofx but in different style.
%  It looks as follows:
%
%  Recall the chain rule (...).
%******************************
%  Differentiate f(x) = #1.
%  Let u= #2
%  Let g(u) = #3
%  Then g'(u)=#4
%  Then f(x) = g(u)
%  f'(x) = g'(u)h'(x)
%  = (#4, with h(x) substituted for UU)(#5)
%  = #6, with #2 substituted for UU
%
%  #7 appears in the subtitle of the example.
%
\newcommand{\chainruleStyleTwo}[7]{%
{\renewcommand{\arraystretch}{1.2}
$
\begin{array}{rclll}
\alertNoH{0}{\left(g(h(x))\right)'}&\alertNoH{0}{=}&g'(h(x))  \cdot  h'(x)&& \text{(notation 1)} {~~~~~~~~~~~~~~~~~~~~} \\
\alertNoH{1-}{(g(u))'}&\alertNoH{1-}{=}&\alertNoH{1-}{g'(u) u'}&\text{where } u=h(x)& \text{(notation 2)}\\
\displaystyle\frac{\diff y}{\diff x} &\alertNoH{0}{=}& \displaystyle\frac{\diff y}{\diff u}  \frac{\diff u}{\diff x} &\text{where } y=g(u)& \text{(notation 3)}\quad.\\
\end{array}
$
}
\begin{example}[Chain Rule, Notation 2%
\ifthenelse{\equal{#7}{0}}%
{}%
{, #7}%
]%
\[
\begin{array}{rrcl}
\text{Differentiate } & f(x) & =& #1.\\%
\uncover<2->{%
\text{Let}&\alertNoH{2-3,9-11}{u}%
}%
&\uncover<2->{\alertNoH{2-3,9-11}{=}}&\displaystyle \uncover<2->{%
\alertNoH{2-3,9-11}{ \fcAnswerNoH{3}{#2.}}%
}\\%
\uncover<2->{%
\text{Let}&\alertNoH{4-5,7-8}{g(u)}%
}
&\uncover<2->{\alertNoH{4-5,7-8}{=}}&\displaystyle
\uncover<2->{\alertNoH{4-5,7-8}{\fcAnswerUncoverNoH{2}{5}{ #3.}}%
}\\%
\uncover<2->{%
\text{Then}& f(x)
}%
&\uncover<2->{{=}}&\uncover<2->{%
 g(u).%
}\\%
\uncover<6->{%
\text{Chain Rule:} &
f'(x)%
}%
&\uncover<6->{=}& \uncover<6->{%
 \alertNoH{ 7-8}{g'(u)}%
\alertNoH{ 9-10}{u'}%
}\\%
&& \uncover<7-|handout:0>{=}&\displaystyle \uncover<7-|handout:0>{\alertNoH{7-8}{\left( \fcAnswerUncoverNoH{7}{8}{\noexpandarg\exploregroups\StrSubstitute{#4}{UU}{\alertNoH{11}{u}}\noexploregroups\expandarg}\right)}%
\alertNoH{9-10}{\left( \fcAnswerUncoverNoH{7}{10}{#5}\right)}%
}\\%
&& \uncover<11-|handout:0>{ = }&\displaystyle \uncover<11-| handout:0>{%
 \noexpandarg \exploregroups \StrSubstitute{#6}{UU}{\alertNoH{11}{#2}}.\noexploregroups \expandarg%
}%
\end{array}
\]
\end{example}
}


%
%  Similar to chainrulefofx but in different style.
%  It looks as follows:
%
%  Recall the chain rule (...).
%******************************
%  Differentiate f(x) = #1.
%  h(x) = #2
%  Let g(u) = #3
%  Then f(x) = g(u)
%  f'(x) = g'(u)h'(x)
%  = (#4, with h(x) substituted for UU)(#5)
%  = #6, with #2 substituted for UU
%
%  #7 appears in the subtitle of the example.
%
\newcommand{\chainruleStyleThree}[7]{%
{\renewcommand{\arraystretch}{1.2}
$
\begin{array}{rclll}
\alertNoH{0}{\left(g(h(x))\right)'}&\alertNoH{0}{=}&g'(h(x))  \cdot  h'(x)&& \text{(notation 1)} {~~~~~~~~~~~~~~~~~~~~} \\
(g(u))'&\alertNoH{0}{=}&g'(u) u'&\text{where } u=h(x)& \text{(notation 2)}\\
\displaystyle\alertNoH{1-}{\frac{\diff y}{\diff x}}&\alertNoH{1-}{=}&\displaystyle\alertNoH{1-}{\frac{\diff y}{\diff u}  \frac{\diff u}{\diff x}} &\text{where } y=g(u)& \text{(notation 3)}\quad.\\
\end{array}
$
}
\begin{example}[Chain Rule, Notation 3%
\ifthenelse{\equal{#7}{0}}%
{}%
{, #7}%
]%
\[
\begin{array}{rrcl}
\text{Differentiate } & y & =& #1.\\%
\uncover<2->{%
\text{Let}&\alertNoH{2-3,9-11}{u}%
}%
&\uncover<2->{\alertNoH{2-3,9-11}{=}}& \displaystyle \uncover<2->{%
\alertNoH{2-3,9-11}{ \fcAnswerNoH{3}{#2.}}%
}\\%
\uncover<2->{%
\text{Then}&\alertNoH{4-5,7-8}{y}%
}
&\uncover<2->{\alertNoH{4-5,7-8}{=}}&\displaystyle
\uncover<2->{\alertNoH{4-5,7-8}{\fcAnswerUncoverNoH{2}{5}{ #3.}}%
}\\%
\uncover<6->{%
\text{Chain Rule:} &
\displaystyle \frac{\diff y}{\diff x}%
}%
&\uncover<6->{=}&\displaystyle  \uncover<6->{%
 \alertNoH{7-8}{\frac{\diff y}{\diff u}}%
\alertNoH{9-10}{\frac{\diff u}{\diff x}}%
}\\%
&& \uncover<7->{ =&\displaystyle  \alertNoH{7-8}{ \left( \fcAnswerUncoverNoH{7}{8}{\noexpandarg \exploregroups \StrSubstitute{#4}{UU}{\alertNoH{ 11}{u}} \noexploregroups\expandarg}\right)}%
\alertNoH{9-10}{\left( \fcAnswerUncoverNoH{7}{10}{#5}\right)}}%
\\%
&&\uncover<11->{=}&\displaystyle \uncover<11-| handout:0>{%
\noexpandarg \exploregroups \StrSubstitute{#6}{UU}{\alertNoH{ 11}{#2}}.\noexploregroups \expandarg%
}%
\end{array}
\]
\end{example}
}

%
%  An example of an infinite limit calculation.
%  There are nine arguments to the function.  The ninth is a string of six
%  plus and minus signs.  Let AA, BB, CC, DD, EE, and FF denote these plus
%  and minus signs.  Then the output of the function looks as follows:
%
%  Find lim_{x \to #1^AA} (#2, with x substituted for UU)/(#3, with x substituted for UU).
%  Plug in #1.
%  (#2, with (#1) substituted for UU)/(#3, with (#1) substituted for UU) = #4/0.
%  The numerator is non-zero and the denominator is zero.
%  Therefore the answer is DNE, infty, or -infty.
%  Factor: (#3, with x substituted for UU)/(#4, with x substituted for UU) = (#5 #6)/(#7 #8)
%  \to ((BB)(CC))/((DD)(EE))
%  = (FF).
%  Therefore lim_{x \to #1^AA} (#2, with x substituted for UU)/(#3, with x substituted for UU) = FF infty.
%
\newcommand{\infinitelimit}[9]{%
\begin{example}[Infinite Limit]%
\abovedisplayskip=0pt
\belowdisplayskip=0pt
\abovedisplayshortskip=0pt
\belowdisplayshortskip=0pt
\begin{align*}
\text{Find}\quad \lim_{x\to #1^{\StrMid{#9}{1}{1}}}
\frac%
{\noexpandarg\StrSubstitute{#2}{UU}{x}\expandarg}%
{\noexpandarg\StrSubstitute{#3}{UU}{x}\expandarg}%
& \\%
\uncover<2->{%
\text{Plug in $#1$:}\quad%
\frac%
{\alertNoH{ 2-3}{\noexpandarg\StrSubstitute{#2}{UU}{(#1)}\expandarg}}%
{\alertNoH{ 4-5}{\noexpandarg\StrSubstitute{#3}{UU}{(#1)}\expandarg}}%
}%
& \uncover<2->{= \frac{\fcAnswer{3}{#4}}{ \fcAnswerUncover{2}{5}{ 0}}}%
\uncover<6->
Therefore the answer is DNE, $\infty$, or $-\infty$.}
}%
\uncover<7->{%
\text{Factor:}\quad
}%
\uncover<7->{%
\lim_{x\to #1^{\StrMid{#9}{1}{1}}}%
\frac%
{\alertNoH{ 8-9}{\noexpandarg\StrSubstitute{#2}{UU}{x}\expandarg}}%
{\alertNoH{ 10-11}{\noexpandarg\StrSubstitute{#3}{UU}{x}\expandarg}}%
}%
& \uncover<8->{%
 = \lim_{x\to #1^{\StrMid{#9}{1}{1}}}%
\frac%
{%
\fcAnswer{9}{%
\alertNoH{ 12-13}{%
#5%
}%
\alertNoH{ 14-15}{%
#6%
}%
}%
}{%
\fcAnswerUncover{8}{11}{%
\alertNoH{ 16-17}{%
#7%
}%
\alertNoH{ 18-19}{%
#8%
}%
}%
}%
}\\%
& \uncover<12->{%
 \to \alertNoH{ 20-21}{\frac%
{%
\alertNoH{ 12-13}{( \fcAnswerUncover{12}{13}{%
\StrMid{#9}{2}{2}%
})}%
\alertNoH{ 14-15}{(\fcAnswerUncover{12}{15}{%
\StrMid{#9}{3}{3}%
})}%
}{%
\alertNoH{ 16-17}{(\fcAnswerUncover{12}{17}{%
\StrMid{#9}{4}{4}%
})}%
\alertNoH{ 18-19}{(\fcAnswerUncover{12}{19}{%
\StrMid{#9}{5}{5}%
})}%
}%
}%
}\\%
& \uncover<20->{\alertNoH{ 20-21}{ = \fcAnswer{21}{(\alertNoH{22}{ \StrMid{#9}{6}{6}})}}}\\%
\uncover<22->{%
\text{Therefore}\quad\lim_{x\to #1^{\StrMid{#9}{1}{1}}}%
\frac%
{\noexpandarg\StrSubstitute{#2}{UU}{x}\expandarg}%
{\noexpandarg\StrSubstitute{#3}{UU}{x}\expandarg}%
}%
& \uncover<22->{ = } \uncover<handout:0| 22->{ \alertNoH{ 22}{\StrMid{#9}{6}{6}}\infty.}
\end{align*}
\end{example}
}




%
%  An example of a limit calculation with factoring.
%
%  It looks as follows.
%
%  Find lim_{x \to #1} (#2, with x substituted for UU)/(#3, with x substituted for UU).
%  Plug in #1.
%  (#2, with (#1) substituted for UU)/(#3, with (#1) substituted for UU) = 0/0.
%  Zero over zero gives no information.
%  Factor: (#2, with x substituted for UU)/(#3, with x substituted for UU) = ((#4, with x substituted for UU) #6)/((#5, with x substituted for UU) #6)
%  = (#4, with x substituted for UU)/(#5, with x substituted for UU)
%  Plug in #1: = (#4, with (#1) substituted for UU)/(#5, with (#1) substituted for UU)
%  = #7
%  = #8
%
\newcommand{\limitfactor}[8]{%
\begin{example}[Limit with Factoring]%
\abovedisplayskip=0pt
\belowdisplayskip=0pt
\abovedisplayshortskip=0pt
\belowdisplayshortskip=0pt
\begin{align*}
\text{Find}\quad \lim_{x\to #1}
\frac%
{\noexpandarg\StrSubstitute{#2}{UU}{x}\expandarg}%
{\noexpandarg\StrSubstitute{#3}{UU}{x}\expandarg}%
& \\%
\uncover<2->{%
\text{Plug in $#1$:}\quad%
\frac%
{\alertNoH{2-3}{\noexpandarg\StrSubstitute{#2}{UU}{(#1)}\expandarg}}%
{\alertNoH{4-5}{\noexpandarg\StrSubstitute{#3}{UU}{(#1)}\expandarg}}%
}%
& \uncover<2->{%
= \frac%
{\fcAnswerUncoverNoH{2}{3}{0}}%
{\fcAnswerUncoverNoH{2}{5}{0}}%
}%
\uncover<6->{%
\intertext{Zero over zero is undefined, so we can't use direct substitution.}
}%
\uncover<7->{%
\text{Factor:}\quad%
\lim_{x\to #1} \frac%
{\alertNoH{8-9}{\noexpandarg\StrSubstitute{#2}{UU}{x}\expandarg}}%
{\alertNoH{10-11}{\noexpandarg\StrSubstitute{#3}{UU}{x}\expandarg}}%
}%
& \uncover<8->{%
 = \lim_{x\to #1} \frac%
{%
\fcAnswerUncoverNoH{8}{9}{%
(\noexpandarg\StrSubstitute{#4}{UU}{x}\expandarg)%
\fcCancel{12}{#6}%
}%
}{%
\fcAnswerUncoverNoH{8}{11}{%
(\noexpandarg\StrSubstitute{#5}{UU}{x}\expandarg)%
\fcCancel{12}{#6}%
}%
}%
}\\%
& \uncover<12->{%
 = \lim_{x\to #1} \frac%
{\uncover<handout:0| 12->{\noexpandarg\StrSubstitute{#4}{UU}{\alertNoH{ 13}{x}}\expandarg}}%
{\uncover<handout:0| 12->{\noexpandarg\StrSubstitute{#5}{UU}{\alertNoH{ 13}{x}}\expandarg}}%
}\\%
\uncover<13->{%
\text{Plug in $#1$:}\quad%
\lim_{x\to #1} \frac%
{\noexpandarg\StrSubstitute{#2}{UU}{x}\expandarg}%
{\noexpandarg\StrSubstitute{#3}{UU}{x}\expandarg}%
}%
& \uncover<13->{%
 = \frac%
{\uncover<handout:0| 13->{\noexpandarg\StrSubstitute{#4}{UU}{(\alertNoH{ 13}{#1})}\expandarg}}%
{\uncover<handout:0| 13->{\noexpandarg\StrSubstitute{#5}{UU}{(\alertNoH{ 13}{#1})}\expandarg}}%
}\\%
& \uncover<14->{%
= \uncover<handout:0| 14->{#7}%
}\\%
& \uncover<15->{%
= \uncover<handout:0| 14->{#8.}%
}%
\end{align*}
\end{example}
}




%
%  An example of a limit calculation with a conjugate radical.
%
%  It looks as follows.
%
%  Find lim_{x \to #1} (#2, with x substituted for UU)/(#3, with x substituted for UU).
%  Plug in #1.
%  (#2, with (#1) substituted for UU)/(#3, with (#1) substituted for UU) = 0/0.
%  Zero over zero gives no information.
%  Factor: (#2, with x substituted for UU)/(#3, with x substituted for UU) = ((#4, with x substituted for UU) #6)/((#5, with x substituted for UU) #6)
%  = (#4, with x substituted for UU)/(#5, with x substituted for UU)
%  Plug in #1: = (#4, with (#1) substituted for UU)/(#5, with (#1) substituted for UU)
%  = #7
%  = #8
%
\newcommand{\limitradical}[9]{%
\begin{example}[Limit with Conjugate Radical]%
\abovedisplayskip=0pt
\belowdisplayskip=0pt
\abovedisplayshortskip=0pt
\belowdisplayshortskip=0pt
\begin{align*}
& \text{Find}\quad \lim_{x\to #1}
\frac%
{\noexpandarg\StrSubstitute{#2}{UU}{x}\expandarg}%
{\noexpandarg\StrSubstitute{#3}{UU}{x}\expandarg}%
 \\%
\uncover<2->{%
& \text{Plug in $#1$:}\quad%
\frac%
{\alertNoH{ 2-3}{\noexpandarg\StrSubstitute{#2}{UU}{(#1)}\expandarg}}%
{\alertNoH{ 4-5}{\noexpandarg\StrSubstitute{#3}{UU}{(#1)}\expandarg}}%
}%
 \uncover<2->{%
= \frac%
{\uncover<3->{\alertNoH{ 3}{0}}}%
{\uncover<5->{\alertNoH{ 5}{0}}}%
}%
\uncover<6->{%
\intertext{Zero over zero gives no information.  Use a conjugate radical.}
}%
& \uncover<7->{%
\lim_{x\to #1} \frac%
{\noexpandarg\StrSubstitute{#2}{UU}{x}\expandarg}%
{\alertNoH{ 7-8}{\noexpandarg\StrSubstitute{#3}{UU}{x}\expandarg}}%
\cdot %
\frac%
{\uncover<8->{\alert<8>{\noexpandarg\StrSubstitute{#4}{UU}{x}\expandarg}}}%
{\uncover<8->{\alert<8>{\noexpandarg\StrSubstitute{#4}{UU}{x}\expandarg}}}%
}\\%
& \uncover<9->{%
 = \lim_{x\to #1} \frac%
{(\noexpandarg\StrSubstitute{#2}{UU}{x}\expandarg)%
\left(\noexpandarg\StrSubstitute{#4}{UU}{x}\expandarg\right)}%
{#5}%
}\\%
& \uncover<10->{%
 = \lim_{x\to #1} \frac%
{(\alert<11-12>{\noexpandarg\StrSubstitute{#2}{UU}{x}\expandarg})%
\left(\noexpandarg\StrSubstitute{#4}{UU}{x}\expandarg\right)}%
{\alert<13-14>{#6}}%
}\\%
\uncover<11->{%
\text{Factor:}\quad%
}%
& \uncover<11->{%
 = \lim_{x\to #1} \frac%
{\uncover<12->{\alert<12>{(\noexpandarg\StrSubstitute{#7}{UU}{x}\expandarg)(x-#1)}}%
\left(\noexpandarg\StrSubstitute{#4}{UU}{x}\expandarg\right)}%
{\uncover<14->{\alert<14>{(\noexpandarg\StrSubstitute{#8}{UU}{x}\expandarg)(x-#1)}}}%
}\\%
& \uncover<15->{%
 = \lim_{x\to #1} \frac%
{(\noexpandarg\StrSubstitute{#7}{UU}{x}\expandarg)%
\left(\noexpandarg\StrSubstitute{#4}{UU}{x}\expandarg\right)}%
{\noexpandarg\StrSubstitute{#8}{UU}{x}\expandarg}%
}\\%
\uncover<16->{%
\text{Plug in $#1$:}\quad%
}%
& \uncover<16->{%
 = \frac%
{(\noexpandarg\StrSubstitute{#7}{UU}{(#1)}\expandarg)%
\left(\noexpandarg\StrSubstitute{#4}{UU}{(#1)}\expandarg\right)}%
{\noexpandarg\StrSubstitute{#8}{UU}{(#1)}\expandarg}%
}\\%
& \uncover<17->{%
#9.
}%
\end{align*}
\end{example}
}


%
%  An example of a limit calculation with direct substitution.
%
%  It looks as follows.
%
%  Find lim_{x \to #1} (#2, with x substituted for UU)/(#3, with x substituted for UU).
%  Plug in #1.
%  (#2, with (#1) substituted for UU)/(#3, with (#1) substituted for UU) = 0/0.
%  Zero over zero gives no information.
%  Factor: (#2, with x substituted for UU)/(#3, with x substituted for UU) = ((#4, with x substituted for UU) #6)/((#5, with x substituted for UU) #6)
%  = (#4, with x substituted for UU)/(#5, with x substituted for UU)
%  Plug in #1: = (#4, with (#1) substituted for UU)/(#5, with (#1) substituted for UU)
%  = #7
%  = #8
%
\newcommand{\limitsub}[7]{%
\begin{example}[%
\ifthenelse{\equal{#6}{0}}%
{Limit in Which Direct Substitution Doesn't Work}%
{Limit with Direct Substitution}%
]%
\abovedisplayskip=0pt
\belowdisplayskip=0pt
\abovedisplayshortskip=0pt
\belowdisplayshortskip=0pt
\begin{align*}
\text{Find}\quad \lim_{x\to #1}
\frac%
{\noexpandarg\StrSubstitute{#2}{UU}{x}\expandarg}%
{\noexpandarg\StrSubstitute{#3}{UU}{x}\expandarg}%
& \\%
\uncover<2->{%
\text{Plug in $#1$:}\quad%
\frac%
{\alertNoH{ 2-3}{\noexpandarg\StrSubstitute{#2}{UU}{(#1)}\expandarg}}%
{\alertNoH{ 4-5}{\noexpandarg\StrSubstitute{#3}{UU}{(#1)}\expandarg}}%
}%
& \uncover<2->{%
= \frac%
{\uncover<3->{\alertNoH{ 3}{#4}}}%
{\uncover<5->{\alertNoH{ 5}{#5}}}%
}\\%
\ifthenelse{\equal{#6}{0}}%
{ }%
{&}%
\uncover<6->{%
\ifthenelse{\equal{#6}{0}}%
{\intertext{Dividing by zero is undefined, so we can't use direct substitution.}}%
{ = #7.}%
}%
\ifthenelse{\equal{#6}{0}}%
{ }%
{ \text{Therefore}= #7.}%
\end{align*}
\end{example}
}



%
%  An example Newton's Method.
%
%  It looks as follows.
%
%  Starting with x_1 = #1, find the third approximation x_3 to the root of the equation #2.
%
%  f(x) = (#3, with x substituted for UU).
%  f'(x) = (#4, with x substituted for UU).
%  Newton's Method: x_{n+1} = x_n - f(x_n)/f'(x_n) = x_n - (#3, with x_n substituted for UU)/(#4, with x_n substituted for UU).
%
%  x_2 = x_1 - (#3, with x_1 substituted for UU)/(#4, with x_1 substituted for UU)     x_3 = x_2 - (#3, with x_2 substituted for UU)/(#4, with x_2 substituted for UU)
%   = (#1) - (#3, with (#1) substituted for UU)/(#4, with (#1) substituted for UU)     = (#5) - (#3, with (#5) substituted for UU)/(#4, with (#5) substituted for UU)
%  = #5.      = #6.
%
\newcommand{\newtonsmethod}[8]{%
\begin{example}[Newton's Method%
\ifthenelse{\equal{#8}{0}}%
{}%
{, #8}%
]%
\ifthenelse{\equal{#7}{0}}%
{%
Starting with $x_1 = #1$, find the third approximation $x_3$ to the root of the equation $#2$.
}%
{#7}%
\abovedisplayskip=0pt
\belowdisplayskip=10pt
\abovedisplayshortskip=0pt
\belowdisplayshortskip=0pt
\begin{align*}
\uncover<2->{%
\alertNoH{ 2-3,7}{f(x)}%
& \alertNoH{ 2-3,7}{ = \uncover<3->{\noexpandarg \exploregroups \StrSubstitute{#3}{UU}{x}.\noexploregroups \expandarg}}%
}\\%
\uncover<4->{%
\alertNoH{ 4-5,8}{f'(x)}%
& \alertNoH{ 4-5,8}{ = \uncover<5->{\noexpandarg \exploregroups \StrSubstitute{#4}{UU}{x}.\noexploregroups \expandarg}}%
}\\%
\uncover<6->{%
\text{Newton's Method:}\quad %
x_{n+1} & = x_n - \frac{\alertNoH{ 7}{f(x_n)}}{\alertNoH{ 8}{f'(x_n)}}%
}
\uncover<7->{%
 = x_n - \frac%
{\alertNoH{ 7}{\noexpandarg \exploregroups \StrSubstitute{#3}{UU}{x_n}\noexploregroups \expandarg}}%
{\alertNoH{ 8}{\uncover<8->{\noexpandarg \exploregroups \StrSubstitute{#4}{UU}{x_n}\noexploregroups \expandarg}}}%
}
\end{align*}
\begin{align*}
\uncover<9->{%
x_2 %
}%
& \uncover<9->{%
 = \alertNoH{ 10}{x_1} - \frac%
{\noexpandarg \exploregroups \StrSubstitute{#3}{UU}{\alertNoH{ 10}{x_1}}\noexploregroups \expandarg}%
{\noexpandarg \exploregroups \StrSubstitute{#4}{UU}{\alertNoH{ 10}{x_1}}\noexploregroups \expandarg}%
}%
& \uncover<12->{%
x_3 %
}%
& \uncover<12->{%
 = \alertNoH{ 13}{x_2} - \frac%
{\noexpandarg \exploregroups \StrSubstitute{#3}{UU}{\alertNoH{ 13}{x_2}}\noexploregroups \expandarg}%
{\noexpandarg \exploregroups \StrSubstitute{#4}{UU}{\alertNoH{ 13}{x_2}}\noexploregroups \expandarg}%
}\\%
& \uncover<10->{%
 = \alertNoH{ 10}{(#1)} - \frac%
{\noexpandarg \exploregroups \StrSubstitute{#3}{UU}{\alertNoH{ 10}{(#1)}}\noexploregroups \expandarg}%
{\noexpandarg \exploregroups \StrSubstitute{#4}{UU}{\alertNoH{ 10}{(#1)}}\noexploregroups \expandarg}%
}%
& %
& \uncover<13->{%
 = \alertNoH{ 13}{(#5)} - \frac%
{\noexpandarg \exploregroups \StrSubstitute{#3}{UU}{\alertNoH{ 13}{(#5)}}\noexploregroups \expandarg}%
{\noexpandarg \exploregroups \StrSubstitute{#4}{UU}{\alertNoH{ 13}{(#5)}}\noexploregroups \expandarg}%
}\\%
& \uncover<11->{%
 = #5.%
}%
& %
& \uncover<14->{%
 = #6.
}%
\end{align*}
\end{example}
}


%
%  An example of a derivative using the Chain Rule twice, using dy/dx.
%  It looks as follows:
%
%  Differentiate: y = #1.
%		  dy\dx  = d\dx(#1)
%  Chain Rule:     = (#2) (d/dx)(#3)
%  Chain Rule:     = (#2)(#4) d/dx(#5)
%  #7 [optional]    = (#2)(#3)(#6)
%                             = (#8)
%                             = (#9)    [optional]
%

\newcommand{\chainruletwice}[9]{%
\begin{example}[Using the Chain Rule twice]%
\abovedisplayskip=0pt
\belowdisplayskip=0pt
\abovedisplayshortskip=0pt
\belowdisplayshortskip=0pt
\begin{align*}
\text{Differentiate:}\quad y & = #1.\\%
\uncover<2->{\frac{\diff y}{\diff x} & = \alertNoH{3-5}{\frac{\diff}{\diff x}\left( #1\right)}}\\%
\uncover<4->{\text{Chain Rule:} \ \ \quad &= \alertNoH{4-5}{\left(\fcAnswerNoH{5}{#2} \right)\alertNoH{6-8}{\frac{\diff}{\diff x} \left(\uncover<4-| handout:0>{#3}\right)}}} \\%
\uncover<7->{\text{Chain Rule:} \ \ \quad &= \left(\uncover<7-| handout:0>{#2}\right) \alertNoH{7-8}{\left(\fcAnswerNoH{8}{#4}\right) \alertNoH{9-10}{\frac{\diff}{\diff x}\left( \uncover<7-| handout:0>{#5} \right)}}}\\%
\uncover<9->{\uncover<10->{\ifthenelse{\equal{#7}{}}{}{\text{#7 :} \ \ \quad}}& = \left(\uncover<9-| handout:0>{#2} \right) \left(\uncover<9-| handout:0>{#4}\right)\alertNoH{9-10}{\left( \fcAnswerNoH{10}{#6} \right) }} \\%
\uncover<11->{& = \uncover<11-| handout:0>{#8 \ifthenelse{\equal{#9}{}}{.}{\\}}}%
\ifthenelse{\equal{#9}{}}{}{\uncover<12->{& = \uncover<12-| handout:0>{#9.}}}
\end{align*}
\end{example}
}
