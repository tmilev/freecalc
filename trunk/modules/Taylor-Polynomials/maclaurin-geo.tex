% begin module maclaurin-series-ex4
\begin{frame}
\begin{example} %[Example 4, p. 775]
Find the Maclaurin series of $f(x) = \frac{1}{1+x}$.
\abovedisplayskip=2pt
\belowdisplayskip=2pt
\[
\begin{array}{rcl@{\qquad}rcl}
\uncover<2->{%
f(x)%
}%
&%
\uncover<2->{%
=%
}%
&%
\uncover<2->{%
(1+x)^{-1}%
}%
&%
\uncover<2->{%
f(0)%
}%
&%
\uncover<2->{%
=%
}%
&%
\uncover<3->{%
\color{blue}{1}%
}\\%

\uncover<2->{%
f'(x)%
}%
&%
\uncover<2->{%
=%
}%
&%
\uncover<5->{%
-(1+x)^{-2}%
}%
&%
\uncover<2->{%
f'(0)%
}%
&%
\uncover<2->{%
=%
}%
&%
\uncover<7->{%
\color{blue}{-1}%
}\\%

\uncover<2->{%
f''(x)%
}%
&%
\uncover<2->{%
=%
}%
&%
\uncover<9->{%
2(1+x)^{-3}%
}%
&%
\uncover<2->{%
f''(0)%
}%
&%
\uncover<2->{%
=%
}%
&%
\uncover<11->{%
\color{blue}{2}%
}\\%

\uncover<2->{%
f'''(x)%
}%
&%
\uncover<2->{%
=%
}%
&%
\uncover<13->{%
-6(1+x)^{-4}%
}%

&%
\uncover<2->{%
f'''(0)%
}%
&%
\uncover<2->{%
=%
}%
&%
\uncover<15->{%
\color{blue}{-6}=-3!%
}%
\\%

\uncover<2->{%
f^{(4)}(x)%
}%
&%
\uncover<2->{%
=%
}%
&%
\uncover<17->{%
24(1+x)^{-5}%
}%

&%
\uncover<2->{%
f^{(4)}(0)%
}%
&%
\uncover<2->{%
=%
}%
&%
\uncover<19->{%
\color{blue}{24}=4!%
}%
\end{array}
\]
\uncover<20->{%
The Maclaurin series is
\abovedisplayskip=2pt
\belowdisplayskip=2pt
\begin{align*}
\frac{1}{1+x} & =\sum_{n=0}^\infty \frac{f^{(n)}(0)}{n!}x^n \\ %
\uncover<21->{%
& = \frac{\color{blue}{1}}{0!} +\frac{\color{blue}{-1}}{1!}x + \frac{\color{blue}{2}}{2!}x^2 +\frac{\color{blue}{-6}}{3!}x^3 +\frac{\color{blue}{24}}{4!} x^4 \cdots\\
} %
\uncover<22->{%
& = 1-x+x^2-x^3+x^4-x^5\cdots \\   
}%
\uncover<22->{%
& \sum_{i=0}^\infty (-x)^n   
}%
\end{align*}
}%
\uncover<24->{%
(This is a geometric series with $ a=1 $, and $ r=-x $.)
}
\end{example}
\end{frame}
% end module maclaurin-series-ex4
