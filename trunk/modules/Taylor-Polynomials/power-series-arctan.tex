% begin module power-series-as-function-ex1
\begin{frame}
\begin{example} %[Example 1, p. 765]
Find the Maclaurin series for $\frac{1}{1+x^2}$.
\uncover<2->{%
\abovedisplayskip=0pt
\belowdisplayskip=0pt
\[
\frac{1}{1+x^2} = \frac{1}{1+(x^2)}%
\]
}%
\uncover<3->{%
\abovedisplayskip=0pt
\belowdisplayskip=0pt
\[
\begin{array}{c@{\  }c@{\  }l@{\ }c@{\  }c@{\  }c@{  }c@{  }c@{  }c@{  }c@{  }c@{  }c@{  }}
\displaystyle \frac{1}{1+\alert<handout:0| 4>{x}} &%
 = & %
\displaystyle \sum_{n=0}^\infty \alert<handout:0| 4>{(-x)}^n &%
 = & %
1 &%
- &%
x &%
+ &%
x^2 &%
- &%
x^3 &%
%+ &%
%x^4 &%
+ \ \cdots \\%
% & \uncover<3->{%
%\displaystyle \frac{1}{1+x^2}%
%} &&&&&&&&&&&&&\\%
\uncover<4->{%
\displaystyle \frac{1}{1+(\alert<handout:0| 4>{x^2})}%
} &%
\uncover<4->{ = } &%
\uncover<4->{%
\displaystyle \sum_{n=0}^\infty (\alert<handout:0| 4>{-x^2})^n%
} &%
\uncover<5->{ = } &%
\uncover<5->{1} &%
\uncover<5->{+} &%
\uncover<5->{(-x^2)} &%
\uncover<5->{+} &%
\uncover<5->{(-x^2)^2} &%
\uncover<5->{+} &%
\uncover<5->{(-x^2)^3} &%
%\uncover<5->{+} &%
%\uncover<5->{(-x^2)^4} &%
\uncover<5->{+ \ \cdots} \\%
&&&%
\uncover<6->{ = } &%
\uncover<6->{1} &%
\uncover<6->{-} &%
\uncover<6->{x^2} &%
\uncover<6->{+} &%
\uncover<6->{x^4} &%
\uncover<6->{-} &%
\uncover<6->{x^6} &%
%\uncover<6->{+} &%
%\uncover<6->{x^8} &%
\uncover<6->{+ \ \cdots} %
%
\end{array}
\]
}%
\begin{itemize}
\item<7->  Another way to write the series is $\sum_{n=0}^\infty (-x^2)^n = \sum_{n=0}^\infty (-1)^nx^{2n}$.
%\item<8->  This converges if $|-x^2| < 1$, that is, if $x^2 < 1$, or $|x| < 1$.
%\item<9->  Therefore the interval of convergence is $(-1, 1)$.
\end{itemize}
\end{example}
\end{frame}
% end module power-series-as-function-ex1


\begin{frame}
\[
\frac{1}{1+x^2} = 1-x^2+x^3-x^4\cdots 
\]
\pause 
Integrating both sides gives \pause 
\begin{align*}
\int \frac{1}{1+x^2} = \int (1-x^2+x^3-x^4+x^5-x^6\cdots)dx 
\end{align*}
\pause  So
\begin{align*}
\tan^{-1}(x) & = x-\frac{x^3}{3}+\frac{x^5}{5}-\frac{x^7}{7}+\cdots\\
& = \sum_{n=0}^\infty (-1)^n\frac{x^{2n+1}}{2n+1} 
\end{align*}
\pause 
Note: This looks very similar to the expansion for sine, but without the factorials.
\end{frame}