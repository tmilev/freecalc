% begin module Taylor-ex1
\begin{frame}
\begin{example}[]
Find the 8'th Taylor polynomial $ T_8 $ of $f(x) = \cos x$ at $ x=\frac{\pi}{2} $. Based on the pattern, what does the Taylor series look like?
\abovedisplayskip=2pt
\belowdisplayskip=2pt
\[
\begin{array}{rcl@{\qquad}rcl}
\uncover<3->{%
f(x)%
}%
&%
\uncover<3->{%
=%
}%
&%
\uncover<3->{%
\cos x%
}%

&%
\uncover<3->{%
\alert<handout:0| 2-3,21>{%
f\left(\frac{\pi}{2}\right)%
}%
}%
&%
\uncover<3->{%
\alert<handout:0| 2-3,21>{%
=%
}%
}%
&%
\alert<handout:0| 3,21>{%
\uncover<3->{%
0%
}%
}\\%

\uncover<3->{%
\alert<handout:0| 4-5>{%
f'(x)%
}%
}%
&%
\uncover<3->{%
\alert<handout:0| 4-5>{%
=%
}%
}%
&%
\alert<handout:0| 4-5>{%
\uncover<5->{%
-\sin x%
}%
}%

&%
\uncover<3->{%
\alert<handout:0| 6-7,22-23>{%
f'\left(\frac{\pi}{2}\right)%
}%
}%
&%
\uncover<3->{%
\alert<handout:0| 6-7,22-23>{%
=%
}%
}%
&%
\alert<handout:0| 6-7,22-23>{%
\uncover<7->{%
-1%
}%
}\\%

\uncover<3->{%
\alert<handout:0| 8-9>{%
f''(x)%
}%
}%
&%
\uncover<3->{%
\alert<handout:0| 8-9>{%
=%
}%
}%
&%
\alert<handout:0| 8-9>{%
\uncover<9->{%
-\cos x%
}%
}%

&%
\uncover<3->{%
\alert<handout:0| 10-11,24>{%
f''\left(\frac{\pi}{2}\right)%
}%
}%
&%
\uncover<3->{%
\alert<handout:0| 10-11,24>{%
=%
}%
}%
&%
\alert<handout:0| 10-11,24>{%
\uncover<11->{%
0%
}%
}\\%

\uncover<3->{%
\alert<handout:0| 12-13>{%
f'''(x)%
}%
}%
&%
\uncover<3->{%
\alert<handout:0| 12-13>{%
=%
}%
}%
&%
\alert<handout:0| 12-13>{%
\uncover<13->{%
\sin x%
}%
}%

&%
\uncover<3->{%
\alert<handout:0| 14-15,25-26>{%
f'''\left(\frac{\pi}{2}\right)%
}%
}%
&%
\uncover<3->{%
\alert<handout:0| 14-15,25-26>{%
=%
}%
}%
&%
\alert<handout:0| 14-15,25-26>{%
\uncover<15->{%
1%
}%
}\\%

\uncover<3->{%
\alert<handout:0| 16-17>{%
f^{(4)}(x)%
}%
}%
&%
\uncover<3->{%
\alert<handout:0| 16-17>{%
=%
}%
}%
&%
\alert<handout:0| 16-17>{%
\uncover<17->{%
\cos x%
}%
}%

&%
\uncover<3->{%
\alert<handout:0| 18-19,27>{%
f^{(4)}\left(\frac{\pi}{2}\right)%
}%
}%
&%
\uncover<3->{%
\alert<handout:0| 18-19,27>{%
=%
}%
}%
&%
\alert<handout:0| 18-19,27>{%
\uncover<19->{%
0%
}%
}%
\end{array}
\]
\uncover<20->{%
We see a  pattern already, so the 8'th Taylor polynomial is
\abovedisplayskip=2pt
\belowdisplayskip=2pt
} %
\uncover<2->{%
\[
T_8=\sum_{n=0}^8 \frac{f^{(n)}\left(\frac{\pi}{2}\right)}{n!}\left(x-\frac{\pi}{2}\right)^n = %
\uncover<20->{%
\uncover<21->{%
0
} %
\uncover<23->{%
\alert<handout:0| 23>{%
\alert<handout:0| 33-34>{-(x-\frac{\pi}{2})}%
}%
}%
}%
\uncover<24->{%
+0
} %
\uncover<26->{%
\alert<handout:0| 26>{%
\alert<handout:0| 31-32>{+} \frac{(x-\frac{\pi}{2})^{\alert<handout:0| 33-34>{3}}}{\alert<handout:0| 33-34>{3}!}%
}%
}%
\uncover<27->{%
+0
} %
\uncover<29->{%
\alert<handout:0| 29>{%
\alert<handout:0| 31-32>{-} \frac{(x-\frac{\pi}{2})^{\alert<handout:0| 33-34>{5}}}{\alert<handout:0| 33-34>{5}!}%
}%
}%
\uncover<30->{%
\alert<handout:0| 30>{%
\alert<handout:0| 31-32>{+0+} \frac{(x-\frac{\pi}{2})^{\alert<handout:0| 33-34>{7}}}{\alert<handout:0| 33-34>{7}!} \alert<handout:0| 31-32>{ }  %
}%
}%
\]
}%
\uncover<31->{%
Note that this is the 8'th Taylor polynomial, even though it has degree only 7. \\ \pause 
The Taylor series is
\begin{align*}
\sum_{n=0}^{\infty} \frac{f^{(n)}\left(\frac{\pi}{2}\right)}{n!}(x-\frac{\pi}{2})^n & = \sum_{n=0}^\infty \uncover<35->{\alert<handout:0| 35>{(-1)^{n+1}}}\frac{\uncover<34->{(x-\frac{\pi}{2})^{\alert<handout:0| 34>{2n+1}}}}{\uncover<34->{(\alert<handout:0| 34>{2n+1})!}}%
\end{align*}
 
}%

\end{example}
\end{frame}
% end module Taylor-ex1
