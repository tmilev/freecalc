% begin module rectilinear-motion-intro
\begin{frame}
\frametitle{Rectilinear Motion}
\begin{itemize}
\item  Suppose a particle is moving in a straight line, with position function $s(t)$.
\item<2->  Its velocity is \alert<handout:0| 2-3,7>{$v(t) =$ \uncover<3->{$s'(t)$.}}
\item<4->  Its acceleration is \alert<handout:0| 4-5,9>{$a(t) =$ \uncover<5->{$v'(t)$.}}
\item<6-| alert@6-7>  Position is the antiderivative of \uncover<7->{velocity.}
\item<6-| alert@8-9>  Velocity is the antiderivative of \uncover<9->{acceleration.}
\item<10->  If we know the acceleration and the initial values $s(0)$ and $v(0)$ for position and velocity, then we can find $s(t)$ by antidifferentiating twice.
\end{itemize}
\end{frame}
% end module rectilinear-motion-intro
