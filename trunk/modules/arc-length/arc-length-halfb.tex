\begin{frame}
% begin module arc-length-half

Sometimes an arc length problem can be simplified significantly if the function $y = f(x)$ has a certain form.\\
In particular, if $y' = a - b$, where $2ab = \frac{1}{2}$, then  $ 1+(y')^2 $ is a perfect square, simplifying the arc length formula. 
\begin{align*}
1+(y')^2 
&= 1 + (a-b)^2\\
& =  1 + \left( a^2 - \frac{1}{2} + b^2\right)\\
& =  a^2 + \frac{1}{2} + b^2 \\
& =  a^2 + 2ab + b^2 \\
& =  (a+b)^2.
\end{align*}
There are several exercises like this in the textbook (such as numbers 9 and 10) and on WebWork (question 2 from this section) of this type.\\
Consider the following examples:
\end{frame}

\begin{frame}
\frametitle{Example: Find the arc length of $\displaystyle y = \frac{x^7}{14} + \frac{1}{10x^5}$ between $x = 1$ and $x = 2$.}

\textbf{Solution:}
Arc length formula: $L = \int_1^2 \sqrt{1 + (y')^2}\diff x$, so first find $y'$.
\begin{eqnarray*}
y' & = & 7\frac{x^6}{14} + (-5)\frac{1}{10x^6}\\
& = & \frac{1}{2} x^6 - \frac{1}{2}x^{-6}
\end{eqnarray*} \pause

Observe that if $a = \frac{1}{2}x^6$ and $b = \frac{1}{2}x^{-6}$, then $ y'=a-b $, and  $2ab = 2\left( \frac{1}{2}x^6\right) \left(\frac{1}{2}x^{-6}\right) = \frac{1}{2}$, so
\[
1+(y')^2 = \left( \frac{1}{2}x^6 + \frac{1}{2}x^{-6}\right)^2
\]

Now apply this to the arc length formula:
\end{frame}

\begin{frame}
\begin{align*}
L  =  \int_1^2 \sqrt{1 + (y')^2}\diff x
& =  \int_1^2 \sqrt{\left(\frac{1}{2}x^6 + \frac{1}{2}x^{-6}\right)^2}\diff x\\
\uncover<2->{%
& =  \int_1^2 \left( \frac{1}{2}x^6 + \frac{1}{2}x^{-6}\right)\diff x\\
} %
\uncover<3->{%
& =  \left[ \frac{1}{2}\cdot \frac{1}{7}x^7 + \frac{1}{2} \cdot \frac{1}{-5}x^{-5}\right]_1^2\\
} %
\uncover<4->{%
& =  \left[ \frac{1}{14}x^7 - \frac{1}{10}x^{-5}\right]_1^2\\
} %
\uncover<5->{%
& =  \left( \frac{128}{14} - \frac{1}{10}\cdot \frac{1}{32}\right) - \left( \frac{1}{14} - \frac{1}{10}\right) 
\uncover<6->{%
 =  \frac{20537}{2240}.}
 } %
\end{align*}

\end{frame}


% end module arc-length-half
