% begin module arc-length-intro
\begin{frame}
\frametitle{(9.1) Arc Length}
\begin{center}
\ \only<-2>{%
\includegraphics[height=4cm]{arc-length/pictures/09-01-circlea.pdf}%
}%
\only<handout:0| 3>{%
\includegraphics[height=4cm]{arc-length/pictures/09-01-circleb.pdf}%
}%
\only<handout:0| 4>{%
\includegraphics[height=4cm]{arc-length/pictures/09-01-circlec.pdf}%
}%
\only<handout:0| 5>{%
\includegraphics[height=4cm]{arc-length/pictures/09-01-circled.pdf}%
}%
\only<handout:0| 6>{%
\includegraphics[height=4cm]{arc-length/pictures/09-01-circlee.pdf}%
}%
\only<handout:0| 7->{%
\includegraphics[height=4cm]{arc-length/pictures/09-01-circlef.pdf}%
}%
\end{center}
\begin{itemize}
\item  What do we mean by the length of a curve?
\item<2->  The length of a polygon is easy to compute: add up the length of the line segments that form the polygon.
\item<3->  If the curve is a circle, approximate it by a polygon.
\item<4->  Then take the limit as the number of segments of the polygon goes to $\infty$.
\end{itemize}
\end{frame}
% end module arc-length-intro
