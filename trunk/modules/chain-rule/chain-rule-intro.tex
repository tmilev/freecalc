% begin module chain-rule-intro
\begin{frame}
\frametitle{The Chain Rule}
\begin{itemize}
\item  What is the derivative of $F(x) = \sqrt{x^2 + 1}$?
\item<2->  The formulas we have learned don't tell us how to solve this.
\item<3->  $F$ is a composite function $f\circ g$:
\item<3-| alert@4-5,9,11-12>  $y = f(u) = \uncover<5->{\sqrt{u}}$.
\item<3-| alert@6-8,13-14>  $u = g(x) = \uncover<7->{x^2+1}$.
\item<3->  Then $y = F(x) = f(\alert<handout:0| 8>{g(x)}) = \uncover<8->{\alert<handout:0| 9>{f(\alert<handout:0| 8>{x^2+1}) =}}  \uncover<9->{\alert<handout:0| 9>{\sqrt{x^2+1}}.}$
\item<10->  We know the derivatives of $f$ and $g$:
\item<10-| alert@11-12>  $f'(u) = \uncover<12->{\frac{1}{2}u^{-1/2}}$.
\item<10-| alert@13-14>  $g'(x) = \uncover<14->{2x}$.
\item<15->  It would be nice if we could find the derivative of $F$ in terms of the derivatives of $y$ and $u$.
\item<16->  It turns out that the derivative of the composition $f\circ g$ is the product of the derivative of $f$ and the derivative of $g$.
\item<17->  This important fact is called the Chain Rule.
\end{itemize}
\end{frame}
% end module chain-rule-intro
