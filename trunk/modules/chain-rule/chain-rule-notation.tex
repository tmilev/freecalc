%begin module chain-rule-notation
\begin{frame}
\frametitle{Chain rule notations}
\begin{itemize}
\item As we saw, the chain rule can be written using a number of notations:
\[
\begin{array}{rclll}
\left(g(h(x))\right)' &=&g'(h(x))  \cdot  h'(x)&& \text{(notation 1)}  \\ 
\left(g(u)\right)'&=&g'(u) u'&\text{where } u=h(x)& \text{(notation 2)}\\
\displaystyle\frac{\diff y}{\diff x} &=& \displaystyle\frac{\diff y}{\diff u}  \frac{\diff u}{\diff x} &\text{where } y=g(u)& \text{(notation 3)}\quad.\\
\end{array}
\]
\item<2-> The three notations are all accepted and can be used interchangeably.
\item<3-> Most authors tend to prefer one of these notations over the others.
\item<4-> In order to exercise ourselves we shall use all three notations throughout our course.
\item<5-> There are additional notations (not covered here) used in practice.
\item<6-> Whenever in doubt about derivative notation, if possible, request clarification.
\end{itemize}
\end{frame}
%end module chain-rule-notation