\begin{frame}
\begin{definition}[Real exponent, Definition I]
Let $z\in \mathbb R$. The real exponent $e^z$ is defined as $\displaystyle \lim\limits_{\substack{p\to x \\ p\text{ is rational}}} e^p$.
\end{definition}
\uncover<2->{
\begin{definition}[Exponent, Definition II]
Let $z\in \mathbb C$. The complex exponent $e^z$ is defined by $\displaystyle e^z=e(z)= \sum \limits_{ n=0 }^\infty \frac{z^n}{n!}.$
\end{definition}
}
\only<handout:1|1-8>{
\begin{itemize}
\item<3-> For real $z$, $e^z$ may be defined via Definition I. 
\item<4-> For complex $z$, $e^z$ is defined via Definition II. 
\item<5-> Real numbers are complex numbers (with zero imaginary part). Thus Definition II is also valid when $z$ is a real number, and therefore Definition II is more general.
\item<6-> \alert<handout:0|8>{A calculus course may be built by presenting Definition II first and proving Definition I as a theorem.}
\item<7-> \alert<handout:0|8>{Alternatively, a calculus course may be built by first presenting Definition I, and then expanding it to Definition II.}
\end{itemize}
}
\only<handout:2|9->{
\begin{theorem}
\alert<9>{When $z\in \mathbb R$, Definition I and Definition II are equivalent.}
\end{theorem}
\only<handout:2|10-12>{
\begin{proof}[Sketch of Proof. Definition I implies Definition II over $\mathbb R$]
Under Definition I the Maclaurin series of $e^z$ was computed to be $\sum\limits_{n=0}^\infty \frac{z^n}{n!}$. \uncover<11->{Under Definition I, it can be shown that $e^z$ equals its Maclaurin series,} \uncover<12->{which is the defining expression for Definition II.}
\end{proof}
}
\only<handout:3|13->{
\begin{proof}[Sketch of Proof. Definition II implies Definition I over $\mathbb R$]
We showed that $e(z+w)=e(z)e(w)$. \uncover<14->{Using that statement alone, one can show that the two definitions agree over the rational numbers.} \uncover<15->{Two continuous functions are equal if they are equal over the rationals,} \uncover<16->{and the theorem follows.}
\end{proof}
} %only
} %only
\vskip 10cm
\end{frame}