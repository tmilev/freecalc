\begin{frame}
\begin{definition}[Real exponent, Definition I]
Let $x\in \mathbb R$. The real exponent $e^x$ is defined as $\displaystyle \lim\limits_{\substack{p\to x \\ p\text{ is rational}}} e^p$.
\end{definition}

\begin{definition}[Exponent, Definition II]
Let $z\in \mathbb C$. The complex exponent $e^z$ is defined by $\displaystyle e^z=e(z)= \sum\limits_{n=0}^\infty \frac{z^n}{n!}.$
\end{definition}

\end{frame}

\begin{frame}
\begin{theorem}
When $z\in \mathbb R$, Definition I and Definition II are equivalent.
\end{theorem}
\begin{proof}
Under Definition I the Maclaurin series of $e^z$ was computed to be $e^z=\sum\limits_{n=0}^\infty \frac{z^n}{n!}$. If Under Definition I, it can be shown that $e^z$ equals its Maclaurin series, which is the defining expression for Definition II.
\end{proof}
\end{frame}

\begin{frame}
\begin{definition}[Real exponent, Definition I]
Let $x\in \mathbb R$. The real exponent $e^x$ is defined as $\displaystyle \lim\limits_{\substack{p\to x \\ p\text{ is rational}}} e^p$.
\end{definition}

\begin{definition}[Exponent, Definition II]
Let $z\in \mathbb C$. The complex exponent $e^z$ is defined by $\displaystyle e^z=e(z)= \sum\limits_{n=0}^\infty \frac{z^n}{n!}.$
\end{definition}

\begin{itemize}
\item For real $x$, $e^x$ may be defined via Definition I. 
\item For complex $z$, $e^z$ is defined via Definition II. 
\item Since Definition II is valid when $z$ is a real number, it is a more general definition.
\item A calculus course may be built by directly presenting Definition II. 
\item If we assume Definition II, Definition I can be proce
\end{itemize}

\vskip 10cm

\end{frame}