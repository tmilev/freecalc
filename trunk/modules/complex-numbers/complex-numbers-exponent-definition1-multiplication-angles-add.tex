\begin{frame}
\begin{columns}[t]
\column{0.5\textwidth}
\begin{definition}[Real exponent]
Let $\rho\in \mathbb R$. The real exponent $e^\rho$ is defined as $\displaystyle \lim\limits_{\substack{p\to \rho \\ p\text{ is rational}}} e^p$.
\end{definition}
\column{0.5\textwidth}
\uncover<2->{
\begin{definition}[Extension to $\mathbb{C}$]
Let $\rho,\theta\in \mathbb R$. Define the complex exponent $e^{\rho+i\theta} $ via ${\alert<3>{e}}^{\rho+\alert<3>{i\theta} } = e^{\rho} ( \alert<3>{\cos \theta + i\sin \theta})$
\end{definition}
}
\end{columns}
\begin{itemize}
\item For the duration of this slide, assume Definition I of real exponent.
\item<2-> Extend this def. to complex numbers \uncover<3->{(motivation: \alert<3,6,7,14>{Euler's f-la}).} 
\uncover<4->{
\begin{theorem}
(a) Let $\alpha, \beta\in \mathbb R $. Then $e^{i\alpha} e^{ i \beta} = e^{ i\alpha+i\beta}= e^{i(\alpha+\beta) }$.

(b) Let $z,w\in \mathbb C$. Then $e^{z }e^{w}=e^{z+w}$.
\end{theorem}
}
\uncover<4->{
\begin{proof}[Proof of (a)]
$\uncover<5->{\alert<6>{e^{i\alpha}} \alert<7>{e^{i\beta}} = (\alert<6>{\alert<8,10>{ \cos \alpha} +\alert<9,11>{i \sin \alpha}})(\alert<7>{ \alert<8,11>{\cos \beta} + \alert<9,10>{ i\sin \beta}} )} \uncover<8->{ = ( \alert<12,15>{ \alert<8>{ \cos \alpha\cos \beta} \alert<9>{ - \sin\alpha \sin \beta} } ) + \alert<10,11>{i} ( \alert<13,15>{ \alert<10>{\cos \alpha \sin \beta }+ \alert<11>{ \sin \alpha \cos \beta}}) } \uncover<12->{ =\alert<14>{ \alert<12,15>{\cos (\alpha +\beta) }+ i \alert<13,15>{ \sin (\alpha+\beta) } }} \uncover<14->{ \alert<14>{ = e^{i(\alpha+ \beta)} }}
$.
\end{proof}
}
\item<15-> The trig. f-las used above need separate (relatively long) proof.
\end{itemize}
\end{frame}