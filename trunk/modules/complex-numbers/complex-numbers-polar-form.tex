\begin{frame}
\begin{columns}
\column{0.4\textwidth}
\uncover<4->{
\begin{pspicture}(-2,-2)(2,2)
\tiny
\fcAxesStandard{-1.3}{-1.3}{2}{2}
\fcLabels[\Re][\Im]{2}{2}
\uncover<7->{%
\psparametricplot{0}{360}{t cos t sin}%
}%
\rput[br](-0.03,1.03){$i$}
\rput[tl](1.1,-0.1){$1$}
\fcFullDot{0}{1}
\fcFullDot{1}{0}
\uncover<8->{\fcAngle{0}{1.047197551}{0.1}{$\theta$}}%
\uncover<6->{%
\psline(0,0)(! 0.5 3 sqrt 2 div)%
\psline[linestyle=dotted](0,0)(! 1 3 sqrt)%
}%
\rput(1.1,1.81){$~~z$}
\fcFullDot{1}{3 sqrt}
\uncover<5->{%
\fcFullDot{0.5}{3 sqrt 2 div}%
\rput[b](0.47,1){$\frac{z}{|z|}$}%
}%
\uncover<handout:0|5>{%
\psline[linewidth=2pt, arrows=->, linecolor=red](0,0)(! 1 3 sqrt )%
}%
\uncover<9->{\fcPerpendicular{[0.5 3 sqrt 2 div]}{[0 0 ] [1 0]}{0.1}}%
\uncover<handout:0|10>{\psline[linecolor=red, linewidth=2pt](0,0)(0.5, 0)}%
\uncover<handout:0|11>{\psline[linecolor=red, linewidth=2pt](! 0.5 3 sqrt 2 div)(0.5, 0)}%
\end{pspicture}
}%
\column{0.6\textwidth}
\uncover<12->{
\begin{theorem}[Euler's formula]
$e^{i\theta}=\cos \theta+i\sin \theta$.
\end{theorem}
} 
\begin{lemma}
$\alert<handout:0|1>{\left|\frac{z}{|z|}\right| } \uncover<2->{\alert<2>{=\frac{|z|}{|z|}}} \uncover<3->{ \alert<3>{ =1 }.}$
\end{lemma}
\end{columns}
\begin{itemize}
\item<4-> Let $z=x+iy$ be a non-zero complex number.
\item<5-> Then $0$, $z$, $\frac{z}{|z|}$ lie on a ray \uncover<7->{and $\frac{z}{|z|}$ lies on the unit circle.}
\item<8-> Let $\theta$ - angle between the real axis and the ray between $0$ and $z$.
\item<9-> Then $\alert<15>{\frac{z}{|z|}= \alert<10>{\cos \theta} +i \alert<11>{\sin \theta}}$.
\item<12-> This gives a geometric proof/motivation for Euler's formula when $\theta $ a real number. \uncover<13->{Euler's f-la does hold over all complex numbers.}
\item<14-> Let $\alert<16>{\rho = \ln |z|}= \ln \sqrt{x^2+y^2}=\frac{1}{2}\ln(x^2+y^2)$.
\item<15-> Then $\alert<15>{z=\alert<16>{|z|}(\cos\theta+i\sin\theta)}\uncover<16->{= \alert<16>{ e^{\rho}}( \cos \theta +i\sin \theta)}$.
\end{itemize}
\uncover<17->{
\begin{definition}
Let $z\neq 0$. Then $z=e^{\rho}(\cos\theta+i\sin \theta)$ is called polar form of $z $.
\end{definition}
}
\end{frame}