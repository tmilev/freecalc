\begin{frame}
\begin{columns}
\column{0.25\textwidth}
\begin{pspicture}(-1,-1)(1,1)
\tiny
\fcAxesStandard{-0.7}{-0.6}{2.5}{2.5}
\fcLabels[$\Re$][$\Im$]{2.5}{2.5}
\uncover<handout:1|8-10>{%
\fcPerpendicular{[0.5 1]}{[1 0] [0 0]}{0.1}%
\fcPerpendicular{[0.5 1]}{[0 1] [0 0]}{0.1}%
}%
\uncover<8->{%
\fcFullDot{0.5}{1}%
}%
\uncover<handout:3|14>{%
\rput[tl](1,1.9 ){$c(x+yi)$}%
\fcFullDot{1}{2}%
\psline[arrows=->](0,0)(1,2)%
}%
\uncover<handout:3|15>{%
\rput[tl](0.25,0.4){$c(x+yi), 0<c<1$}%
\fcFullDot{0.25}{0.5}%
\psline[arrows=->](0,0)(0.25,0.5)%
}%
\uncover<handout:3|16>{%
\rput[bl](-0.25,-0.4 ){$~~~~~c(x+yi), c<0$}%
\fcFullDot{-0.25}{-0.5}%
\psline[arrows=->](0,0)(-0.25,-0.5)%
}%
\uncover<handout:2|10->{\psline[arrows=->](0,0)(0.5,1)}%
\uncover<handout:2|11-13>{%
\fcFullDot{1}{0.5}%
\psline[arrows=->](0,0)(1,0.5)%
}%
\uncover<handout:2|12-13>{%
\fcFullDot{1.5}{1.5}%
\psline[arrows=->](0.5,1)(1.5,1.5)%
\psline[arrows=->](1,0.5)(1.5,1.5)%
}%
\uncover<handout:2|12>{\rput[b](1.5,1.6){$\alertNoH{12}{x+u+(y+w)i}$}}
\uncover<handout:0|13>{\rput[b](1.5,1.6){$\alertNoH{13}{(x+u,y+w)}$}}
\uncover<handout:0|8,13>{\rput[lb](0,1.1){$\alertNoH{8,13}{(x,y)}$}}
\uncover<handout:2|11,12>{\rput[tl](1,0.5){$\alertNoH{11,12}{~u+wi}$}}
\uncover<handout:0|13>{\rput[tl](1,0.5){$\alertNoH{13}{~(u,w)}$}}
\uncover<9-12,14->{\rput[lb](0,1.1){$\alertNoH{9,11,12}{x+yi}$}}
\end{pspicture}
\column{0.75\textwidth}
\begin{definition}[Complex numbers]
The complex numbers are the set $\{x+yi|x,y\in\alertNoH{2}{\mathbb R}\}$.
\end{definition}
\begin{itemize}
\item<2-> Real numbers are usually denoted by $\mathbb R$.
\item<3-> Complex numbers are usually denoted by $\mathbb C$.
\end{itemize}
\end{columns}
\uncover<4->{Consider $z= \alertNoH{5}{x}+\alertNoH{6}{y}i$.}
\begin{itemize}
\item<5-> \alertNoH{5,7}{$x$} is called the \alertNoH{5,7}{real part} of $z$, \uncover<6->{ \alertNoH{6,7}{$y$} is called the \alertNoH{6,7}{imaginary part} of $z$.} \uncover<7->{We write $ \alertNoH{7}{x=\Re z} =\Re(x+yi)$, $\alertNoH{7}{y=\Im z} = \Im (x+yi)$.}
\item<8-> Real \& imaginary part of $z$ can be used as $x,y$-coords. to depict $z$.
\item<10-> In this way we view complex number $x+iy$ as the point (position vector) $(x,y)$ in a two-dimensional space.
\item<11-> The addition of complex numbers corresponds to \alertNoH{11-13}{vector addition}.
\item<14-> \alertNoH{14,15,16}{Multiplication by a real number $c$ } corresponds to \alertNoH{14,15,16}{vector scalar multiplication by $c$ \alertNoH{14,15,16}{(scaling)}}.
\item<17-> The space the complex numbers is referred to as the \alertNoH{17}{complex plane} (sometimes alternatively called the complex line).
\end{itemize}
\uncover<handout:1,2,3|1->{}
\end{frame}
