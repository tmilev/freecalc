\begin{frame}
\begin{theorem}
Let $\alert<2>{e(z)= \sum\limits_{n=0}^{\infty}\frac{z^n }{ n!}}$, \uncover<2->{\alert<2>{$z\in \mathbb C$}.} \uncover<3->{Then $e(z ) e ( w ) =e(z+w)$.}
\end{theorem}
\uncover<2->{\alert<2>{Power series over $\mathbb C$ are defined similarly to power series over $\mathbb R$. }} \uncover<4->{\alert<4>{The following proof lies outside of scope Calc II. Details are omitted and get filled in a course of Complex Analysis. You will not be tested on it. }}
\uncover<4->{
\begin{proof}
\uncover<5->{
$
\begin{array}{rcll|l}
\displaystyle e(z)e(w)&=&\displaystyle  \sum_{n=0}^{\infty} \frac{z^n}{n!} \sum_{m=0}^{\infty} \frac{w^m}{m!} =  \displaystyle  \sum_{s=0}^{ \infty} \sum_{ k=0 }^s \frac{z^{k}w^{s-k}}{k! (s-k)!} \\ &=& \displaystyle  \sum_{s=0}^{\infty} \alert<6>{ \sum_{k=0 }^s  } \frac{\alert<6>{ z^{k} w^{s -k} }}{ s!}\alert<6>{ \frac{s!}{k! (s-k)! }} =\displaystyle  \sum_{s=0}^{\infty} \frac{ \alert<6>{ (z+w)^s} }{s!}=e(z+w).
\end{array}
$
} %uncover
\end{proof}
} %uncover
\uncover<6->{
\begin{lemma}[Newton Binomial formula]
\alert<6>{$(z+w)^s=\sum_{k=0 }^s z^{k}w^{s-k} \frac{s!}{k! (s-k)!}$}.
\end{lemma}
}
\end{frame}