\begin{frame}
If $\fcv r =\langle a,b \rangle$ is a vector, by $f(\fcv r) $ we understand $f(a,b) $ (i.e., define $f(\fcv r)$ via the vector-point identification).
\begin{definition}
\begin{itemize}
\item<2-> Let $f: D\to \mathbb R$, where $D$ is a region in the plane;
\item<3-> let  $f$ be defined near $P$ with position vector $\fcv r$; $f$ is not necessarily defined at $P$;
\item<4-> let $\fcv u$ be an arbitrary vector.
\end{itemize}
\uncover<5->{ 
We say that the one-variable limit
\[
\lim\limits_{t\to 0} f(\fcv{r}+t\fcv{u})
\]
is the limit of $f$ along the direction $\fcv u$.
}
\end{definition}
\uncover<6->{
\begin{theorem}
If the limit $\lim\limits_{Q\to P}f(Q) $ exists, then every directional limit $\lim\limits_{t\to 0 } f(\fcv r+t\fcv u ) $ exists and all directional limits are equal.
\end{theorem}
}
\end{frame}