\begin{frame}
\frametitle{Rectangular/Cartesian Coordinates}
\begin{columns}
\column{0.3\textwidth}
\psset{xunit=2cm, yunit=2cm}
\begin{pspicture}(-1 ,-1)(1, 1)
\fcBoundingBox{-1}{-1}{1}{1}
\tiny
\psline[arrows=->](0,-1)(0,1)
\psline[arrows=->](-1,0)(1,0)
\rput[t](1, -0.1){$x$}
\rput[l](-0.1, 1){$y$}
\fcFullDot{0}{0}
\end{pspicture}


\column{0.7\textwidth}
\begin{itemize}
\item The Cartesian (rectangular) coordinate system is a way to represent points on the plane.
\item To introduce Cartesian coordinates, fix:
\begin{itemize}
\item<2-> a point $O$ (called the origin),
\item<3-> 2 pairwise perpendicular lines intersecting at the origin, called axes,
\item<4-> a direction in each of the coordinate axis.
\end{itemize}
\item<5-> The axes are labeled as $x$-axis and $y$-axis. 
\item<6-> The $x$ axis is drawn horizontal with direction pointing from left to right.
\item<7-> The $y$ axis is drawn vertical, pointing up.
\item<8-> The Cartesian coordinate system is named after Ren\'e Descartes (1596-1650) (Latinized name: Cartesius).
\end{itemize}

\vskip 3cm
\end{columns}
\end{frame}