%\begin{comment}
\begin{frame}
\frametitle{Rectangular/Cartesian Coordinates}
\begin{columns}
\column[t]{0.3\textwidth}
\psset{xunit=0.7cm, yunit=0.7cm}
\begin{pspicture}(-0.5 ,-2)(4.5, 4)
\fcBoundingBox{-0.5}{-2}{4.5}{4}
\tiny
\fcAxesIIId{3}{3}{3}
\fcPutIIId[r]{[0 0 0]}{$O~$}
\end{pspicture}


\vfill
\column{0.7\textwidth}
\begin{itemize}
\item<1-> A Cartesian coordinate system is given by fixing:
\begin{itemize}
\item<2-> a point $O$ (called the origin),
\item<3-> 3 pairwise perpendicular lines intersecting at the origin,
\item<4-> a direction in each of the coordinate axis.
\end{itemize}
\item<5-> The three lines are labeled as $x$-axis, $y$-axis and $z$-axis.
\end{itemize}

\vskip 3cm
\end{columns}


%
\end{frame}

\begin{frame}
\frametitle{Rectangular/Cartesian Coordinates}
\begin{columns}
\column[t]{0.3\textwidth}
\psset{xunit=0.7cm, yunit=0.7cm}
\begin{pspicture}(-0.5 ,-2)(4.5, 4)
\fcBoundingBox{-0.5}{-2}{4.5}{4}
\tiny
\renewcommand{\fcScreen}{[-1 1.1 -0.5] 0}
\fcAxesIIId{3}{3}{3}
\fcPutIIId[r]{[0 0 0]}{$O~$}
\fcPutIIId[b]{[2.5 2.5 2.8]}{$P(x_P,y_P,z_P)$}
\uncover<9->{
\fcLineIIId[linecolor=gray]{[2.5 2.5 2.5]}{[2.5 2.5 0]}
\fcLineIIId[linecolor=gray]{[2.5 2.5 2.5]}{[2.5 0 2.5]}
\fcLineIIId[linecolor=gray]{[2.5 2.5 2.5]}{[0 2.5 2.5]}
\fcLineIIId[linecolor=gray]{[0 2.5 2.5]}{[0 0 2.5]}
\fcLineIIId[linestyle=dashed, linecolor=gray]{[0 2.5 2.5]}{[0 2.5 0]}
\fcLineIIId[linecolor=gray]{[2.5 0 2.5]}{[0 0 2.5]}
\fcLineIIId[linecolor=gray]{[2.5 0 2.5]}{[2.5 0 0]}
\fcLineIIId[linestyle=dashed, linecolor=gray]{[2.5 2.5 0]}{[0 2.5 0]}
\fcLineIIId[linecolor=gray]{[2.5 2.5 0]}{[2.5 0 0]}
}
\fcDotIIId{[2.5 2.5 2.5]}
\uncover<4-8>{%
\fcPerpendicularIIId{[2.5 2.5 2.5]}{[1 0 0]}{0.3}%
}%
\uncover<4->{\fcPutIIId[t]{[1.25 0 -0.1]}{$x_P$}}
\uncover<6-8>{%
\fcPerpendicularIIId{[2.5 2.5 2.5]}{[0 1 0]}{0.3}%
\fcPerpendicularIIId{[2.5 2.5 2.5]}{[0 0 1]}{0.3}%
}%
\uncover<6->{ %
\fcPutIIId[b]{[0 1.25 0.1]}{$y_P$}
\fcPutIIId[r]{[0 0 1.25]}{$z_P~~$}
}
\end{pspicture}


\vfill
\column{0.7\textwidth}
\begin{itemize}
\item<1-> $P$ -point. We assign to it triple $(x_P,y_P,z_P)$.
\item<2-> Assignment will be such that distinct points are assigned distinct triples.
\item<3-> $Q=$ base of perpendicular from $P$ to $x$-axis.
\item<4-> Define $x_P$ as \alert<5>{signed distance b-n $O$ and $Q$}.
\item<5-> Take distance with \alert<5>{$+$ sign if $OQ$ points in direction of $x$-axis, $-$ sign else}.
\item<6-> Definitions of $y_P$, $z_P$ are similar.
\item<7-> $(x_P,y_P,z_P)$ = Cartesian coordinates of $P$. 
\item<8-> $x_P$ is called the $x$-coordinate of $P$,  and so on for other axes.
\item<9-> $(x_P, y_P, z_P)$ = singed lengths of edges of the rectangular box indicated in the picture.
\end{itemize}

\vfill
\end{columns}

\vskip 5cm

\end{frame}
%\end{comment}