\solution{\ref{problemAsymptotesy=(-5x^2-3x+5)/(x^2-2x-3)}

\textbf{Vertical asymptotes.} The function is rational, and therefore has a finite limit (and therefore no vertical asymptote) at every point it its domain. The function is not defined for $x^2-2x-3=0$, which has two solutions, $x=-1$ and $x=3$. These are precisely the vertical asymptotes: indeed, 
\[
\begin{array}{rcll|l}
\displaystyle 
\lim\limits_{x\to -1^+} \frac{-5x^2-3x+5}{x^2-2x-3}&=&\displaystyle  \lim\limits_{x\to -1^+} \frac{-5x^2-3x+5}{(x+1) \left(x-3\right)} = -\infty &&\text{Limit of form }\frac{(+)}{(+)(-)}\\
\displaystyle 
\lim\limits_{x\to -1^-} \frac{-5x^2-3x+5}{x^2-2x-3}&=&\displaystyle  \lim\limits_{x\to -1^-} \frac{-5x^2-3x+5}{(x+1) \left(x-3\right)} = \infty &&\text{Limit of form }\frac{(+)}{(-)(-)}\\
\end{array}
\]
and
\[
\begin{array}{rcll|l}
\displaystyle 
\lim\limits_{x\to 3^+} \frac{-5x^2-3x+5}{x^2-2x-3}&=&\displaystyle  \lim\limits_{x\to 3^+} \frac{-5x^2-3x+5}{(x+1) \left(x-3\right)} = -\infty &&\text{Limit of form }\frac{(-)}{(+)(+)}\\
\displaystyle 
\lim\limits_{x\to 3^-} \frac{-5x^2-3x+5}{x^2-2x-3}&=&\displaystyle  \lim\limits_{x\to 3^-} \frac{-5x^2-3x+5}{(x+1) \left(x-3\right)} = \infty &&\text{Limit of form }\frac{(-)}{(+)(-)}\\
\end{array}
\]

\textbf{Horizontal asymptotes.} 
\[
\begin{array}{rcll|l}\renewcommand{\arraystretch}{1.6}
\displaystyle \lim\limits_{x\to \pm\infty} \frac{-5x^2-3x+5}{x^2-2x-3} &=&\displaystyle \lim\limits_{x\to \pm\infty} \frac{\left(-5x^2-3x+5\right)\frac{1}{x^2}}{\left(x^2-2x-3\right)\frac{1}{x^2}}&&\text{Divide by highest term in den.}\\
&=&\displaystyle  \displaystyle \lim\limits_{x\to \pm\infty} \frac{-5-\frac{3}{x}+\frac{5}{x^2}}{1-\frac{2}{x}-\frac{3}{x^2}} \\
&=&\displaystyle  \displaystyle  \frac{-\lim\limits_{x\to \pm\infty}5-\lim\limits_{x\to \pm\infty}\frac{3}{x}+\lim\limits_{x\to \pm\infty}\frac{5}{x^2}}{\lim\limits_{x\to \pm\infty}1-\lim\limits_{x\to \pm\infty}\frac{2}{x}-\lim\limits_{x\to \pm\infty}\frac{3}{x^2}}&&\text{Step may be skipped}\\
&=& \displaystyle \frac{-5-0+0}{1-0-0}\\
&=&\displaystyle -5.\\
\end{array}
\]

Therefore $y=-5$ is the only horizontal asymptote, valid in both directions ($x\to \pm \infty$). 


A computer generated graph confirms our computations.

\psset{xunit=0.2cm, yunit=0.2cm}
\begin{pspicture}(-16, -20.9)(16,17.2)
\tiny
\fcAxesStandard{-15}{-20.8}{15}{17.5}
\fcXTick{10}
\newcommand{\theFun}{x x -5 mul mul -3 x mul 5 add add x x mul -2 x mul -3 add add div\space}
\psplot[linecolor=\fcColorGraph, plotpoints=1000]{-15}{-1.04}{\theFun}
\psplot[linecolor=\fcColorGraph, plotpoints=1000]{-0.96}{2.95}{\theFun}
\psplot[linecolor=\fcColorGraph, plotpoints=1000]{3.05}{15}{\theFun}
\psline[linestyle=dotted](-1,-19.3)(-1,16.1)
\psline[linestyle=dotted](3,-19.3)(3,16.1)
\psline[linestyle=dashed, linecolor=blue](-15, -5)(15, -5)
\rput[t](-8, -5.2){$y=-5$}
\rput[bl](5,2){$y= \frac{-5x^2-3x+5}{x^2-2x-3}$}
\rput[l](3.2,-8){$x=3$}
\rput[r](-0.6,8){$x=-1$}
\end{pspicture}
}