% begin module limit-at-infinity-ex5
\begin{frame}
\begin{example}
\begin{columns}[c]
\column{.45\textwidth}
Evaluate $\lim\limits_{x\to \infty} \sqrt{x^2+1}-x$. %

\psset{xunit=0.7cm, yunit=0.7cm}
\begin{pspicture}(-2,-0.5)(6.2,4.7)
\psframe*[linecolor=white](-2,-0.5)(6.2,4.7)
\psaxes[ticks=none, labels=none]{<->}(0,0)(-2,-0.5)(6,4.5)
\fcLabelXOne
\fcLabelYOne
%Function formula: - (x)+sqrt{}((x)^{2}+1)
\uncover<12->{
\psline[linecolor=blue](-1.95, 0)(5.95, 0)
}
\uncover<13->{\psplot[linecolor=red, plotpoints=1000]{-1.95}{5.95}{1 x 2 exp add sqrt x -1 mul add }
\rput[lb](1,1){\footnotesize $y=\sqrt{x^2+1}-x$}}
\end{pspicture}
\begin{itemize}
\item<2->  $\sqrt{x^2+1}\to \infty$ and $x\to \infty$ as $x\to \infty$.
\item<2->  It isn't clear what happens to the difference.
\item<7-12,13-| alert@7>  Divide top \& bottom by $x$.
\end{itemize}
\column{.55\textwidth}
\begin{itemize}
\item<3-| alert@3-4>  Standard approach: multiply top and bottom by $\pm$conjugate radical.
\end{itemize}
\abovedisplayskip=0pt
\belowdisplayskip=0pt
\begin{eqnarray*}
&&%
\uncover<2->{%
\lim_{x\to \infty} \left( \alertNoH{5}{ \sqrt{x^2+1}-x}\right) \uncover<4->{\alertNoH{ 4,5}{\frac{\sqrt{x^2+1}+x}{\sqrt{x^2+1}+x}}}%
}%
\\%
&&%
\uncover<5->{%
 = \lim_{x\to \infty} \frac{(\fcCancel{6}{x^2}+1)-\fcCancel{6}{x^2} }{\sqrt{x^2+1}+x}%
}%
\\%
&&%
\uncover<6->{%
= \lim_{x\to \infty} \frac{1}{\left(\sqrt{ \alertNoH{10}{x^2} +\alertNoH{9}{1}}+ \alertNoH{8}{x}\right)}\uncover<7->{\alertNoH{ 7}{\cdot \frac{\frac{1}{x}}{ \alertNoH{8,9,10}{\frac{1}{x}}}}}%
}%
\\%
&&%
\uncover<8->{%
= \lim_{x\to \infty} \frac{\frac{1}{x}}{\sqrt{\alertNoH{10}{1}+\alertNoH{9}{\frac{1}{x^2}}}+\alertNoH{8}{1}}%
}%
\\%
&&%
\uncover<11->{ = \frac{0}{\sqrt{1+0}+1}}\uncover<12->{=0}%
\end{eqnarray*}
\end{columns}
\end{example}
\end{frame}
% end module limit-at-infinity-ex5
