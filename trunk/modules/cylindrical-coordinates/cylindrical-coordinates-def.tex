\begin{frame}
 \frametitle{Cylindrical coordinates}
\begin{columns}
\column{0.4\textwidth}
\psset{xunit=2.5cm, yunit=2.5cm}
\begin{pspicture}(-0.5, -1)(2,2)
\tiny
\renewcommand{\fcScreen}{[-3 -1 -1] 0}
\fcAxesIIId{1.5}{1.5}{1.5}
\fcLineIIId{[0 0 0]}{[1 1 1]}%
\fcPerpendicularIIId[linestyle=dashed]{[1 1 1]}{[0 0 1]}{0.2}%
\fcPerpendicularIIId{[1 1 1]}{[1 1 0]}{0.2}%
\fcPutIIId[bl]{[1 1 1]}{$~~P(x_P, y_P, z_P)$}
\fcPutIIId[tl]{[1 1 0]}{$~~\alert<6>{Q}$}
\fcPerpendicularIIId[linestyle=dashed]{[1 1 0]}{[1 0 0]}{0.2}%
\fcPerpendicularIIId[linestyle=dashed]{[1 1 0]}{[0 1 0]}{0.2}%
\fcLineIIId{[0 0 0]}{[1 1 0]}
\fcPutIIId[r]{[0.5 0 0]}{$x_P~~$}
\fcPutIIId[b]{[0 0.5 0]}{$y_P$}
\fcPutIIId[r]{[0 0 0.5]}{$\alert<5>{z_P~~}$}
\uncover<5>{%
\fcLineIIId[linecolor=red, linewidth=2pt]{[0 0 0]}{[0 0 1]}%
\fcLineIIId[linecolor=red, linewidth=2pt]{[1 1 1]}{[1 1 0]}%
}%
\fcDotIIId{[1 1 1]}%
\fcDotIIId{[1 1 0]}%
\fcPutIIId[tr]{[0.5 0.5 0]}{$\alert<6>{r_P}$}%
\fcAngleIIId[arrows=->, linecolor=red]{[1 0 0]}{[1 1 0]}{0.3}%
\uncover<7>{\fcAngleIIId[arrows=->, linecolor=red, linewidth=2pt]{[1 0 0]}{[1 1 0]}{0.3}}%
\uncover<6>{\fcLineIIId[linecolor=red, linewidth=2pt]{[0 0 0]}{[1 1 0]}}%
\fcPutIIId[t]{[0.4 0.2 0]}{$\alert<7>{\theta_P}$}
\end{pspicture}


\column{0.6\textwidth}

\begin{itemize}
\item In Cartesian coordinates, a point $P$ is given by triple $(x_P, y_P, z_P)$.
\item<2-> We introduce alternative cylindrical coordinates $(r_P, \theta_P ,z_P)$.
\item<3-> Cylindrical coordinates are obtained by ``adding a $z$-coordinate'' to the ($2$-dimensional) polar coordinates.
\item<4-> More precisely, to $P$ we assign triple $(\alert<6>{r_P}, \alert<7>{\theta_P}, \alert<5>{z_P})$, where:
\begin{itemize}
\item<5-> $\alert<5>{z_P}$ equals the $z$-coordinate of $P$,
\item<6-> $\alert<6>{r_P}$ is the distance $|OQ|$, where $Q$ is the projection of $P$ in the $xy$-plane and $O$-origin,
\item<7-> $\alert<7>{\theta_P}$ is an angle between the $x$-axis and $\fcv {OQ} $.
\end{itemize}
\end{itemize}
\end{columns}

\end{frame}