% begin module derivatives-rules-summary
\begin{frame}
\frametitle{Rules of differentiation.}
We studied the basic rules of differentiation.
\begin{itemize}
\item<1->\alert<14>{ $f (g(x))'=f'(g(x)) g'(x) $ (Chain rule). }
\item<2->\alert<14>{ $(f*g)'=f'g+fg'$ (Product rule).}
\item<3->\alert<14>{ $(f+g)'=f'+g'$ (Sum rule). }
\item<4->\alert<14>{ $x'=1$. }
\item<5->\alert<14>{ $(c)'=0$ if $c$ is a constant (Constant derivative rule).}
\end{itemize}
\uncover<6->{We studied additional differentiation rules.}
\begin{itemize}
\item<6->\alert<13>{ $(e^x)'=e^x$.}
\item<7->\alert<13>{ $\left(\frac{f}{g}\right)'=\frac{f' g-f g' }{g^2}$ (Quotient rule).}
\item<8->\alert<13>{ $(x^r)'=rx^{r-1} $, $r$-arbitrary real number (Power rule).}
\item<9->\alert<13>{ $(\ln x)'=\frac{1}x$.}
\item<10->\alert<13>{ $(\log_a x)'=\frac{1}{x\ln a}$.}
\item<11->\alert<13>{ $(\sin x)'=\cos x$, $(\cos x)'=-\sin x$}
\end{itemize}

\uncover<12->{These rules are related to one another: we explore how.}

\uncover<13->{The \alert<13>{second set of rules} can be derived from the \alert<14>{first set}.}

\end{frame}
\begin{frame}
We need to remember the basic 5 rules:
\begin{itemize}
\item{ $f (g(x))'=f'(g(x)) g'(x) $ (Chain rule). }
\item{ $(f*g)'=f'g+fg'$ (Product rule).}
\item{ $(f+g)'=f'+g'$ (Sum rule). }
\item{ $x'=1$. }
\item{ $(c)'=0$ if $c$ is a constant (Constant derivative rule).}
\end{itemize}

We derived the remaining rules in the previous lectures using various considerations. 

In the present lecture we derive the remaining rules using algebra alone.

\end{frame}
% end module derivatives-rules-summary
