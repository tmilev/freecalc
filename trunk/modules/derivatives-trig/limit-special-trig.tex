% begin module limit-laws-ex5
\begin{frame}
\begin{example}[Special Trigonometric Limits Example]%
\abovedisplayskip=0pt
\belowdisplayskip=-15pt
\abovedisplayshortskip=0pt
\belowdisplayshortskip=0pt
\begin{align*}
\text{Find}\quad \lim_{h\to 0}{\sin 4h \over h}%
& \\%
\uncover<2->{%
\text{Plug in 0:}\quad%
\frac%
{\alertNoH{ 2-3}{\sin(0)}}%
{\alertNoH{ 4}{(0)}}
}%
& \uncover<2->{%
= \frac%
{\uncover<3->{\alertNoH{ 3}{0}}}%
{\uncover<4->{\alertNoH{ 4}{0}}}%
}\\%
\uncover<5->{%
\intertext{Zero over zero is undefined, so we can't use direct substitution.}
}%
&\\ %
\uncover<6->{%
\intertext{To evaluate the limit we rewrite it in the form of
$\displaystyle \frac{\sin \theta}{\theta}$ so that we may use the fact that   $\displaystyle \lim_{\theta \to 0} \frac{\sin \theta}{\theta} =1$.
Therefore, we set $\theta = 4h$ and write}
}%
&\\ %
\end{align*}
\end{example}
\end{frame}

\begin{frame}
\begin{example}[Special Trigonometric Limits Example]%
\abovedisplayskip=0pt
\belowdisplayskip=-15pt
\abovedisplayshortskip=0pt
\belowdisplayshortskip=0pt
\begin{align*}
\lim_{h\to 0}\frac{\sin 4h }{h}%
\uncover<2->{%
= \lim_{h\to 0} \frac{4}{4}\cdot\frac{\sin4h}{h}%
}%
 \uncover<3->{%
&= \lim_{h\to 0} 4\cdot \frac{\sin {\alertNoH{ 4}{4h}}}{{\alertNoH{ 5}{4h}}}
}\\ %
& \uncover<4->{%
= 4\cdot \frac{\sin({\alertNoH{ 4}{\theta}})}{{\alertNoH{ 5}{\theta}}}%
}\\%
\end{align*}
\uncover<6->{%
The new variable $\theta$ approaches zero as $h\to 0$ since $\theta$
 is a constant multiple of $h$.
 Therefore, we may change the limit as $h\to 0$ into a limit as $\theta \to 0$
to obtain
} %
\begin{align*}
\uncover<6->{%
\lim_{h\to 0}{\frac{\sin 4h}{h}}
} %
& \uncover<6->{%
=\lim_{\theta\to 0} 4\cdot \frac{\sin(\theta)}{\theta}
}\\%
& \uncover<7->{%
 =4\cdot {\alertNoH{ 8-8}{\lim_{\theta\to 0} \frac{\sin(\theta)}{\theta}}}
}\\%
& \uncover<8->{%
 =4\cdot {\alertNoH{ 8-8}{1}}
}\\%
& \uncover<9->{%
=4
}
\end{align*}
\end{example}
\end{frame}
% end module limit-laws-ex5
