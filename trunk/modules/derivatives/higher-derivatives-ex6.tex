% begin module higher-derivatives-ex6
\begin{frame}
\begin{example}[Example 6, p. 130]
If $f(x) = x^3-x$, find $f''(x)$.
\begin{columns}[c]
\column{.4\textwidth}
\ \only<handout:0| 1>{%
\includegraphics[height=4.5cm]{derivatives/pictures/03-02-ex6a.pdf}%
}%
\only<handout:0| 2-7>{%
\includegraphics[height=4.5cm]{derivatives/pictures/03-02-ex6b.pdf}%
}%
\only<8->{%
\includegraphics[height=4.5cm]{derivatives/pictures/03-02-ex6c.pdf}%
}%
\column{.6\textwidth}
\uncover<2->{%
In Example 2 we found that the first derivative is $f'(x) = 3x^2 - 1$.
}%
\abovedisplayskip=0pt
\belowdisplayskip=0pt
\begin{eqnarray*}
& & \uncover<3->{f''(x)}\\%
 & \uncover<3->{ = } & %
\uncover<3->{\lim_{h\rightarrow 0}\frac{f'(x+h)-f'(x)}{h}}\\%
 & \uncover<4->{ = } & %
\uncover<4->{\lim_{h\rightarrow 0}\frac{[3(x+h)^2-1]-[3x^2-1]}{h}}\\%
 & \uncover<5->{ = } & %
\uncover<5->{\lim_{h\rightarrow 0}\frac{3x^2 + 6xh + 3h^2 -1 - 3x^2 +1}{h}}\\%
 & \uncover<6->{ = } & %
\uncover<6->{\lim_{h\rightarrow 0}\frac{6xh + 3h^2}{h}}\\%
 & \uncover<7->{ = } & %
\uncover<7->{\lim_{h\rightarrow 0}(6x + 3h)}\uncover<8->{ = 6x}\\%
\end{eqnarray*}
\end{columns}
\end{example}
\end{frame}
% end module higher-derivatives-ex6
