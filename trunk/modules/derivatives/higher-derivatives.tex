% begin module higher-derivatives
\begin{frame}
\frametitle{Higher Derivatives}
If $f$ is a differentiable function, then $f'$ is also a function, so $f'$ might have a derivative of its own, denoted by $(f')' = f''$.  This new function $f''$ is called the second derivative of $f$.

In Leibniz notation, the second derivative of $y = f(x)$ is written
\[
\frac{\diff}{\diff x}\left( \frac{\diff y}{\diff x}\right) = \frac{\diff^2 y}{\diff x^2}.
\]

\uncover<2->{%
We can interpret $f''(x)$ as a rate of change of a rate of change.  The most familiar example is acceleration, which is the instantaneous rate of change of velocity with respect to time.
}%

\uncover<3->{%
The third derivative of $f$ is the derivative of the second derivative, and is written $f'''$.
}%

\uncover<4->{%
The fourth derivative is denoted by $f^{(4)}$, and for $n > 3$ the $n$th derivative is denoted by $f^{(n)}$.
}%
\end{frame}
% end module higher-derivatives
