\begin{frame}
\frametitle{Square matrices}
\begin{itemize}
\item Let $A$ be $n\times n$ (square) table of numbers.
\item<2-> Technical term: $A$ is a (square) \emph{matrix}.
\item<3-> Matrices are often denoted by surrounding with $\left(\right)$-parenthesis:
\[
A=\left(
\begin{array}{ccccc}
a_{\alertNoH{7}{1}\alertNoH{10}{1}} & a_{\alertNoH{7}{1}\alertNoH{11}{2}} & a_{\alertNoH{7}{1}3}& \dots & a_{\alertNoH{7}{1}\alertNoH{12}{n}}\\
a_{\alertNoH{8}{2}\alertNoH{10}{1}} & a_{\alertNoH{8}{2}\alertNoH{11}{2}} & a_{\alertNoH{8}{2}3}& \dots & a_{\alertNoH{8}{2}\alertNoH{12}{n}}\\
\vdots \\
a_{\alertNoH{9}{n}\alertNoH{10}{1}} & a_{\alertNoH{9}{n}\alertNoH{11}{2}} & a_{\alertNoH{9}{n}3}& \dots & a_{\alertNoH{9}{n}\alertNoH{12}{n}}
\end{array}
\right).
\]
\vphantom{ $n^{th}$ Second First row}
\only<7>{\alertNoH{7}{First row}}
\only<8>{\alertNoH{8}{Second row}}
\only<9>{\alertNoH{9}{$n^{th}$ row}}
\only<10>{\alertNoH{10}{First column}}
\only<11>{\alertNoH{11}{Second column}}
\only<12>{\alertNoH{12}{$n^{th}$ column}}

\item<4-> Most common convention for matrix notation:
\begin{itemize}
\item $(i,j)^{th}$ entry of a matrix = denoted by letter with indices $i,j$, such as $a_{ij}$
\item<5-> no comma between indices $i,j$ in $a_{ij}$
\item<6-> first index stands for row, second - for column.
\end{itemize}
\item<13-> Non-square matrices: used \& important but we discuss them elsewhere.
\end{itemize}
\end{frame}



\begin{frame}

\begin{itemize}
\item The determinant $\det A$ of a square matrix $A$ is a number written as:
\[
\det A= \left|\begin{array}{ccccc}
a_{11} & a_{12} & a_{13}& \dots & a_{1n}\\
a_{21} & a_{22} & a_{23}& \dots & a_{2n}\\
\vdots \\
a_{n1} & a_{n2} & a_{n3}& \dots & a_{nn}
\end{array} \right|
\]
\item<2-> The formula for the determinant is:

\[\det A=\sum_{\text{all permutations }\alertNoH{3}{\sigma}} a_{\alertNoH{4}{1\alertNoH{3}{\sigma}(1)}} a_{\alertNoH{4}{2\alertNoH{3}{\sigma}(2)}} \dots a_{\alertNoH{4}{n\alertNoH{3}{\sigma}(n)}} \alertNoH{6}{ sign(\alertNoH{3}{\sigma})} \quad .
\]
\item<3-> For every permutation $\sigma$ we have one summand.
\item<4-> Every pair $(k,\sigma(k))$ can be identified with a peaceful of a rook placement (as described in previous slides/lectures).
\item<5-> For each rook placement we have a summand obtained by multiplying the numbers on which the rooks are standing.
\item<6-> The sign of each summand is determined by the sign of the permutation.
\end{itemize}

\vskip 10cm
\end{frame}
