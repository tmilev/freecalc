\begin{frame}
\frametitle{Square matrices}
\begin{itemize}
\item Let $A$ be $n\times n$ (square) table of numbers.
\item<2-> Technical term: $A$ is a (square) \emph{matrix}. 
\item<3-> Matrices are often denoted by surrounding with $\left(\right)$-parenthesis:
\[
A=\left(
\begin{array}{ccccc}
a_{\alert<7>{1}\alert<10>{1}} & a_{\alert<7>{1}\alert<11>{2}} & a_{\alert<7>{1}3}& \dots & a_{\alert<7>{1}\alert<12>{n}}\\
a_{\alert<8>{2}\alert<10>{1}} & a_{\alert<8>{2}\alert<11>{2}} & a_{\alert<8>{2}3}& \dots & a_{\alert<8>{2}\alert<12>{n}}\\
\vdots \\
a_{\alert<9>{n}\alert<10>{1}} & a_{\alert<9>{n}\alert<11>{2}} & a_{\alert<9>{n}3}& \dots & a_{\alert<9>{n}\alert<12>{n}}
\end{array}
\right).
\] 
\vphantom{ $n^{th}$ Second First row}
\only<7>{\alert<7>{First row}}
\only<8>{\alert<8>{Second row}}
\only<9>{\alert<9>{$n^{th}$ row}}
\only<10>{\alert<10>{First column}}
\only<11>{\alert<11>{Second column}}
\only<12>{\alert<12>{$n^{th}$ column}}

\item<4-> Most common convention for matrix notation: 
\begin{itemize}
\item $(i,j)^{th}$ entry of a matrix = denoted by letter with indices $i,j$, such as $a_{ij}$
\item<5-> no comma between indices $i,j$ in $a_{ij}$
\item<6-> first index stands for row, second - for column. 
\end{itemize}
\item<13-> Non-square matrices: used \& important but we discuss them elsewhere.
\end{itemize}
\end{frame}



\begin{frame}

\begin{itemize}
\item The determinant $\det A$ of a square matrix $A$ is a number written as:
\[
\det A= \left|\begin{array}{ccccc}
a_{11} & a_{12} & a_{13}& \dots & a_{1n}\\
a_{21} & a_{22} & a_{23}& \dots & a_{2n}\\
\vdots \\
a_{n1} & a_{n2} & a_{n3}& \dots & a_{nn}
\end{array} \right|
\]
\item<2-> The formula for the determinant is:

\[\det A=\sum_{\text{all permutations }\alert<3>{\sigma}} a_{\alert<4>{1\alert<3>{\sigma}(1)}} a_{\alert<4>{2\alert<3>{\sigma}(2)}} \dots a_{\alert<4>{n\alert<3>{\sigma}(n)}} \alert<6>{ sign(\alert<3>{\sigma})} \quad .
\]
\item<3-> For every permutation $\sigma$ we have one summand.
\item<4-> Every pair $(k,\sigma(k))$ can be identified with a placement of a rook placement (as described in previous slides/lectures).
\item<5-> For each rook placement we have a summand obtained by multiplying the numbers on which the rooks are standing.
\item<6-> The sign of each summand is determined by the sign of the permutation.
\end{itemize}

\vskip 10cm
\end{frame}
