\begin{frame}

\frametitle{$2\times 2$ determinants}
\[
\det \left|
\begin{array}{cc}
\only<1,3,5->{ a_{11}}\only<handout:0|2,4>{\alertNoH{2,4}{\bullet}} & \only<1,2,4,6->{a_{12}}\only<handout:0|3,5>{\alertNoH{3,5}{\bullet}} \\
\only<1,2,4,6->{ a_{21}}\only<handout:0|3,5>{\alertNoH{3,5}{\bullet}} & \only<1,3,5->{ a_{22}}\only<handout:0|2,4>{\alertNoH{2,4}{\bullet}}
\end{array}
\right| = \uncover<4->{\alertNoH{4}{ a_{11} a_{22}}} \uncover<6>{\alertNoH{6}{-}}\uncover<5->{ \alertNoH{5}{a_{12}a_{21}}}
\]
\begin{itemize}
\item We specialize the $n\times n$ determinant formula to the case $n=2$.
\item<2-> There are two peaceful rook placements for a $2\times 2$ chessboard.
\item<4-> For each peaceful rook placement we got one summand.
\item<6-> The permutation $(\sigma(1), \sigma(2))=(2,1)$ is odd, so one of the summands comes with negative sign.
\end{itemize}\end{frame}
