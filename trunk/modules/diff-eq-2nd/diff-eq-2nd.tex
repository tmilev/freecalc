% begin module diff-eq-2nd
\begin{frame}{Second Order Homogeneous Linear Equations with Constant Coefficients}
 
 We shall now discuss the problem of solving the second order  homogeneous equation 
\[ 
ay^{\prime \prime }+by^{\prime }+cy=0\ 
\] 
 where $a,b$ and $c$ are constants and $a\neq 0$.\\
 
 \vsp  
 
 Try $x$, and powers of $x$: These are no  good. \\ \pause  
 Try $\ln x$: also no good.\\ \pause 
 Try $e^{\lambda x}$: \\ \pause 
  If $y=e^{\lambda x}$ is a solution of   then
\[
ay^{\prime \prime }+by^{\prime }+cy=a\lambda ^{2}e^{\lambda x} + b\lambda e^{\lambda
 x}+ce^{\lambda x}=e^{\lambda x} (a\lambda ^{2}+b\lambda +c)=0. \] 
\pause  
It follows that if this is a solution, then
 $\ds   a\lambda ^{2}+b\lambda +c=0$.
   
 This equation for $\lambda $ is called the \textit{auxiliary} or \textit{%
 characteristic} equation.
 
 \end{frame}
 
 
 
 \begin{frame}
 The characteristic equation
$\ds   a\lambda ^{2}+b\lambda +c=0$ has the solution 
\[\lambda =\dfrac{-b\pm \sqrt{b^{2}-4ac}}{2a}=\dfrac{-b\pm \sqrt{\Delta }}{2a},\;\;\;\; \Delta =b^{2}-4ac 
 \]
 
 \pause 
 
 There are three possibilities:
\begin{itemize}
\item[1.] $\Delta >0$  two real, distinct roots \pause  
\item[2.] $\Delta =0$  one real root, repeated \pause 
\item[3.] $\Delta <0$ \ two imaginary roots which are the complex conjugates of  each other, i.e. $\lambda _{1}=\alpha +i\beta $ and  $\lambda_{2}=\alpha -i\beta$.
\end{itemize} 
  
 We shall now discuss the three cases in detail.
 
 \end{frame}
 
 
 
 \begin{frame} \frametitle{Case 1. \ $\Delta >0.$} 
   Here there are two real distinct real roots $\lambda_{1},\lambda _{2},$ \ where $\lambda _{1}\neq \lambda_{2}$ 
 \begin{equation*}
 \lambda_{1}=\dfrac{-b+\sqrt{b^{2}-4ac}}{2a},\;\;\;\;\;\;\lambda_{2}=\dfrac{-b-%
 \sqrt{b^{2}-4ac}}{2a}
 \end{equation*}
 
It follows that $e^{\lambda _{1}x}$ and $e^{\lambda _{2}x}$ are both  solutions of the differential equation. These functions are linearly independent (LI), and the general solution is
 
 \begin{equation*}
 y=c_{1}e^{\lambda _{1}x}+c_{2}e^{\lambda _{2}x}.
 \end{equation*}
 
 where $\lambda_{1}$ and $\lambda_{2}$ are both real and $\lambda_{1}\neq \lambda_{2}.$
 
 \end{frame}
 
 
 
 \begin{frame} 
 \begin{example}
 Solve: $2y^{\prime \prime }-y^{\prime }-3y=0$
 \end{example}
 \pause 
 $\vspace{1pt}$
 
Characteristic Equation:  \[2\lambda ^{2}-\lambda -3=0\] \pause 
 
 \vspace{1pt} Equivalently: 
 
 \[(2\lambda -3)(\lambda +1)=0\] 
 
 \pause
\[ \Rightarrow  \lambda _{1}=-1,\;\; \lambda
 _{2}=\frac{3}{2}\]
\pause
\[
\Rightarrow  y=c_{1}e^{-x}+c_{2}e^{+\frac{3}{2}x}.
\] 
 
 
 \vspace{1pt}
  \end{frame}
  
  
  
  \begin{frame} \frametitle{Case 2. $\Delta =0$, $\lambda _{1}=-%
   \dfrac{b}{2a}$ repeated root}
In this case  $e^{\lambda _{1}x}$ is a solution.\\
To find a  second LI solution (method of variation of
 parameters) we seek a solution of the form  $\ds   y=v(x)e^{\lambda _{1}x}$ where $v(x)$ is a function to be determined. \pause Then \ 
 
 \begin{equation*}
 y^{\prime }=v^{\prime }e^{\lambda _{1}x}+v\lambda _{1}e^{\lambda _{1}x}
 \end{equation*}
  and 
\[
 y^{\prime \prime }=v^{\prime \prime }e^{\lambda _{1}x}+2v^{\prime }\lambda
 _{1}e^{\lambda _{1}x}+v\lambda _{1}^{2}e^{\lambda _{1}x}
 \]
 \pause 
Then $ay''+by'+cy=0 $  gives 
 \begin{equation*}
 av^{\prime \prime }(x)e^{\lambda _{1}x}+2av^{\prime }\lambda _{1}e^{\lambda
 _{1}x}\ +av\lambda _{1}^{2}e^{\lambda _{1}x}+bv^{\prime }e^{\lambda
 _{1}x}+bv\lambda _{1}e^{\lambda _{1}x}+cve^{\lambda _{1}x}=0
 \end{equation*}
 \pause 
Which simplifies to 
 \begin{equation*}
 av^{\prime \prime }+\underbrace{(2a\lambda _{1}+b)}v^{\prime }\ \ +%
 \underbrace{(a\lambda _{1}^{2}+b\lambda _{1}+c)}v=0
 \end{equation*}%
 
 \[\Rightarrow \;\;\; v^{\prime \prime }=0  \textrm{ why?} \]
 \end{frame}
% % % % % % % % % % % % % % % % % % % % % % % % % % %
  
\begin{frame}
  $v^{\prime \prime }=0$, integration then gives  
 $
  v=c_{1}+c_{2}x$\\ \vspace*{3mm}
  
 \pause 
 
 Putting this all together we get
  \begin{equation*}
  y=ve^{\lambda _{1}x}=c_{1}e^{\lambda _{1}x}+c_{2}xe^{\lambda _{1}x}
  \end{equation*}
  
 is a solution of the differential equation in the case where there exists
 one repeated root $\lambda _{1}$. \\
 
 Since $e^{\lambda _{1}x}$ and $xc^{\lambda
 _{1}x}$ are LI this is the general solution.
 
 \end{frame}  
  
  \begin{frame}
 \begin{example}
 Solve:  $y^{\prime \prime }-4y^{\prime }+4y=0$
 \end{example}
 
 Characteristic Equation: $\lambda ^{2}-4\lambda +4=0$ \pause  or $(\lambda -2)^{2}=0$ \ \pause  $%
 \Rightarrow $ \ one real, repeated root $\lambda =2$. \pause 
 
Therefore the general solution is
 \begin{equation*}
 \ y=c_{1}e^{2x}+c_{2}\;xe^{2x}.
 \end{equation*}
 
 \end{frame}
 
 
 
 \begin{frame} 
\frametitle{Case 3. $\Delta <0$ \ 2 complex roots} 
 
 \begin{equation*}
 \lambda =\frac{-b\pm \sqrt{b^{2}-4ac}}{2a}\ =-\frac{b}{2a}\pm \;\frac{i\sqrt{%
 4ac-b^{2}}}{2a} = \alpha \pm i\beta 
 \end{equation*}
 
Say $ \lambda_1=\alpha + i \beta$, and  $ \lambda_2=\alpha - i \beta$.
\\ \vsp 
 
Hence, there are two complex solutions: \\
 \[e^{\lambda_1x} =e^{(\alpha +i\beta )x} =e^{\alpha x}(\cos \beta x+i\sin \beta x)\] 
 and 
 \[
e^{\lambda_2x} = e^{(\alpha -i\beta
  )x}=e^{\alpha x}(\cos \beta x-i\sin \beta x)
 \]
\pause
 
 Since the differential equation has $\underline{\text{real}}$ coefficients, both the real and imaginary parts of the above are solutions, i.e.,
  $e^{\alpha x}\cos \beta x$ and\ \ \ $e^{\alpha x}\sin \beta x$ are both
  solutions in this case.\ These are LI functions.
   \end{frame}
  
  
  
  \begin{frame}
 
 
 The solution of the differential equation is therefore
 
 \begin{equation*}
 y=C_1e^{\alpha x}\cos \beta x+C_2e^{\alpha x}\sin \beta x \;\text{\ where }C_1,C_2\text{ \ \
 real constants.}
 \end{equation*}
 
 \vspace{1pt}
  \end{frame}
  
  
  
  \begin{frame}
 \begin{example}
 \begin{equation*}
Solve:  32y^{\prime \prime }-40y^{\prime }+17y=0
 \end{equation*}
 \end{example}
 Solution: 
 \pause 
 \begin{equation*}
 \text{ Characteristic Equation: } 32\lambda ^{2}-40\lambda +17=0
 \end{equation*}
 \pause
 \begin{align*} 
 \lambda & =\dfrac{40\pm \sqrt{1600-4(32)(17)}}{2(32)} = \dfrac{40\pm 
 \sqrt{1600-2176}}{2(32)}=\dfrac{40\pm i\sqrt{576}}{2(32)}\\ & =\dfrac{5\pm 
 i\sqrt{9}}{8} =\dfrac{5}{8}\pm \dfrac{3}{8}i=\alpha \pm i\beta  
 \end{align*}  \pause 
 
 Thus the solutions are
 
 \begin{align*}
 y & = C_1e^{\alpha x}\cos \beta x+C_2e^{\alpha x}\sin \beta x\\ 
  & =C_1e^{\frac{5}{8}x}\cos\left( \frac{3}{8}x\right)+C_2 e^{\frac{5}{8}x}\sin\left( \frac{3}{8}x\right). 
 \end{align*}
 
 
  \end{frame}
  
  
  
  \begin{frame}
  \frametitle{Summary: Solutions to $ ay''+by'+cy=0 $}
  The solutions fall into one of the following three cases depending on the solutions to the \textit{characteristic equation} $ a\lambda^2+b\lambda +c = 0 $.
  
  \begin{enumerate}
  \item \textit{Distinct Real Zeros}. If $ \lambda_1\ne \lambda_2 $ are distinct real zeros of the characteristic equation, then the general solution is \[
  y=C_1 e^{\lambda_1 x}+ C_2 e^{\lambda_2 x}
  \]
    \item \textit{Single Real Zero}. If $ \lambda_1= \lambda_2 $ are equal real zeros of the characteristic equation, then the general solution is \[
    y=C_1 e^{\lambda_1 x}+ C_2 xe^{\lambda_1 x}
    \]
    \item \textit{Distinct Complex Zeros}. If $ \lambda_1=\alpha+i\beta$ and  $ \lambda_2=\alpha-i\beta $ are complex zeros of the characteristic equation, then the general solution is \[
    y=C_1 e^{\alpha x}\cos(\beta x)+ C_2 e^{\alpha x}\sin(\beta x)
    \]  
  \end{enumerate}
  
  
  
  \end{frame}
  
  
  
  
  \begin{frame}
 \begin{example}
 Write down a second order homogeneous linear differential equation with real  constant coefficients whose solutions are $ \frac{1}{2}e^{-2x}\cos 3x\text{ \ and\ \ }\frac{3e^{-2x}}{4}\sin 3x $
 \end{example}
 Solution:\\ \pause 
 

$\ds  \frac{1}{2}e^{-2x}\cos 3x$ and $ \ds \frac{3e^{-2x}}{4}\sin 3x$ are equal to $ C_1e^{\alpha x}\cos(\beta x) $, and $ C_2e^{\alpha x}\cos(\beta x) $.\\ \vsp 

By observation we get $\alpha =-2$,and $\beta =3$ so that the roots of the characteristic equation must be  $\alpha + i\beta =-2+3i$ and 
   $\alpha-i\beta =-2-3i.$ \pause 
   
 Therefore, the characteristic equation is (equivalent to) 
 \begin{eqnarray*}
 p(\lambda ) &=&[\lambda -(-2+3i)][\lambda -(-2-3i)] \\
 &=&[\lambda +2-3i][\lambda +2+3i] \\
 &=&\lambda ^{2}+(2+3i)\lambda +(2-3i)\lambda +4+9 \\
 &=&\lambda ^{2}+4\lambda +13
 \end{eqnarray*}
 \pause 
% (Check: $\lambda =\dfrac{-4\pm \sqrt{16-4(1)(13)}}{2}$ \ $=\dfrac{-4\pm bi}{2%}$ \ $=-2\pm 3i$ ) \pause 
Therefore, an ODE satisfying our conditions is 
  \begin{equation*}
  y^{\prime \prime }+4y^{\prime }+13y=0
  \end{equation*}
 
  \end{frame}
  
  
  



% end module diff-eq-natural-growth-solution
