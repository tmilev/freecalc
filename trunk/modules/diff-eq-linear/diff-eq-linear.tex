% begin module diff-eq-natural-growth-solution
\begin{frame}{First Order Linear ODE}
 The general first order linear ODE (Ordinary Diff'l Equation) has the form
$$a_0(x)y'+a_1(x)y=b(x)$$
where $a_0(x),\ a_(x),\ b(x)$ are continuous functions of $x$ on some interval.
To bring it to normal form $y'=f(x,y)$  we have to divide both sides of the equation by $a_0(x)$. This is possible only for those $x$ where $a_0(x)\ne 0$.
After possibly shrinking the interval we assume that $a_0(x)\ne 0$. So our equation
has the form (\textbf{standard form} or \textbf{normal form})
$$y'+p(x)y=q(x)$$
with $p(x)=a_1(x)/a_0(x)$ and $q(x)=b(x)/a_0(x)$. 
%Solving for
%$y'$ we get the normal form for a linear first order ODE, namely
%$$y'+p(x)y=q(x).$$
\end{frame}


\begin{frame}{Integrating Factor}
We now introduce the function,
$$I(x)=e^{\int p(x)dx}$$
It has the property $I'(x)=p(x)I(x)$ and $I(x)\ne 0$ for all
$x$. Hence our differential equation is equivalent (has the same
solutions) to the equation
$$I(x)y'+I(x)p(x)y=I(x)q(x).$$ \pause 
Since the left hand side of this equation is the derivative of $I(x)y$, it
can be written in the form
$$\frac{d}{dx}(I(x)y)=I(x)q(x).$$ \pause 
Integrating both sides, we get
$$I(x)y=\int I(x)q(x)\; dx + C$$
with $C$ an arbitrary constant.
\end{frame}


\begin{frame}
Thus we have 
$$I(x)y=\int I(x)q(x)\; dx + C=\int I(x)q(x)\; dx$$ \pause 
 Solving for $y$, we get
$$y=\frac{1}{I(x)}\int I(x)q(x)\;dx.$$
The function $I$ is called an {\bf integrating factor} for the
given linear ODE.
\end{frame}



\begin{frame}
In summary:
\begin{theorem}
For the first order linear differential equation
\[
  \frac{dy}{dx}+P(x)y=Q(x)
\]
An integrating factor is $\ds I(x)=e^{\int P(x)\;dx}$, and the solution of the differential equation is  
\[
y=\frac{1}{I(x)}\int I(x)Q(x)\;dx
\]  
\end{theorem}

Note: replacing $I(x)$ with $-I(x)$ gives the same result. So, if $e^{\int P(x)}=|g(x)|$ for some $g(x)$ then $g(x)$ will be a suitable integrating factor.  (i.e. the absolute value signs may be dropped).\\                                      


%Note that for any pair of scalars $a,b$ with $a$
%in $I$, there is a unique scalar $C$ such that $y(a)=b$.
%Geometrically, this means that the solution curves $y=y(x)$ are a
%family of non-intersecting curves which fill the region
%$I\times\R$.
\end{frame}



\begin{frame}
\begin{example}[Solve: $y'+y=x$.]
\textbf{Solution:} \pause 

This is a linear first order ODE in normal/standard
 form $ y'+p(x)y-q(x) $ with $p(x)= \pause 1$, $q(x)= \pause x$. 
 
The integrating factor is $$I(x)=e^{\int 1\,dx}=e^x.$$ \pause
Hence, the solution to the ODE is
$$y=\frac{1}{I(x)}\int I(x)q(x)\;dx = \frac{1}{e^x}\int xe^x\;dx $$
\pause
which yields (IBP or Tabular Integration)
$$y=\frac{1}{e^x}(xe^x-e^x+C)\pause = x-1+Ce^{-x} $$
\end{example}  
\end{frame}

\begin{frame}
\begin{example}[Solve the Initial Value Problem: $y'+y=x$, and\\ (a) $y(0)=1$ (b) $y(0)=k$.]
\textbf{Solution:} 
\pause 

In the previous example we saw that the solution to the ODE is:
$$y= x-1+Ce^{-x} $$
\pause 
\begin{enumerate}
\item[(a)] If $ y(0)=1 $, then substituting $ x=0 $ and $ y=1 $ into the ODE we get
\[
y(0)=1 \Rightarrow 0-1+Ce^0=1  \pause  \Rightarrow C=2
\]
\pause 
Hence, the solution is $y=x-1+2e^x$. \pause 
\item[(b)] If $ y(0)=k $, then substituting $ x=0 $ and $ y=k $ into the ODE we get
\[
y(0)=k \Rightarrow C-1=k \Rightarrow C=k+1
\]
\pause 
Hence, the solution is $y=x-1+(k+1)e^{-x}$
\end{enumerate}
\end{example}  
\end{frame}


\begin{frame}


\begin{example}[ Solve $xy'-2y=x^3\sin(x)$]
\textbf{Solution:} \pause 
We begin by putting the equation into normal form by
dividing by $x$:
$$y'-\frac{2}{x}y=x^2\sin(x)$$
\pause 
This now has the form $ y'+p(x)y-q(x) $ with $p(x)= \pause -\frac{2}{x}$, $q(x)= \pause x^2\sin(x)$. 
 
The integrating factor is $$I(x)=\pause e^{\int -2/x\; dx}=\pause e^{-2\ln|x|}\pause = e^{\ln(1/x^2)} \pause =1/x^2.$$
\pause 
Hence, the solution to the ODE is
$$y=\frac{1}{I(x)}\int I(x)q(x)\;dx = x^2\int \sin(x)\;dx \pause =-x^2\cos(x)+Cx^2. $$
\pause 
{\scriptsize{Note: This is the general solution of the given ODE for $x\ge0$ or for $x\le0$ by continuity.\\
Observe also that if $a\ne0$, there is a unique
solution satisfying $y(a)=b$ for any constant $b$ while all
solutions satisfy the initial condition $y(0)=0$.\\
(To see why, substitute $ x=0 $ into the original ODE.)

%This non-uniqueness is due to the fact the DE in normal form is not well behaved at $x=0$. However, $y=-x^2\cos(x)+Cx^2$ is not the general solution of the given ODE since different $C's$ are
%possible for $x\ge 0$ and $x\le 0$ due to the fact that the
%one-sided derivatives at $x=0$ are zero for all $C$. 

%It is thegeneral solution if $y$ if $y''$ is required to exist at $x=0$.
}}

\end{example}
 
\end{frame}

\begin{frame}


\begin{example}[Mixing Problem--Again]
{\small Note: \textit{If the rates of flow into and out of the system are
different, then the volume is not constant and the resulting
differential equation is linear but not separable.}\\} \vspace*{2mm}

A tank contains $400$ litres of water.
A solution with a chlorine concentration of $.05$ grams per litre is added at a rate of $10$ litres per second.\\ 
The solution is constantly mixed and is drained from the tank at a rate of $5$  litres per second.\\  
Find the amount (in grams) of chlorine in the tank after $20$ seconds. 

\textbf{Solution:} \pause 
Let $y(t)$ be the amount (grams) of chlorine in the tank after $ t $ seconds  \pause 
\begin{align*}
\frac{dy}{dt}& ={\color{green}{(\text{input rate})}}-{\color{red}{(\text{output rate})}}\\
&\textrm{\color{blue}{\small Note: after $t$ seconds, there will be $400+ 5t$ liters of water in the tank.}}\\ \uncover<4->{                                  
&={\color{green}{\left(10\; l/s \right) \cdot   \left(.05 \;g/l \right)}}- {\color{red}{\left( 5 \;l/s \right) \cdot  \left( \frac{y}{400+5t}\; g/l \right)}}\\
} % 
\uncover<5->{
&=\frac12-\frac{1}{80+t}y
} %
\end{align*}

\end{example}
 
\end{frame}

\begin{frame}

\begin{example}[cont'd]
 So we get:  $ \ds \frac{dy}{dt}+\frac{1}{80+t}\;y=\frac12.
  $
\uncover<2->{  \[
 \text{Recall: }\;\frac{dy}{dt}+P(t)y=Q(t)\;\Rightarrow\; y=\frac{1}{I(t)}\int Q(t)I(t) dt
 \]  
 } %
\uncover<3->{ Find an integrating factor:}
 \begin{align*}
 \uncover<4->{e^{\int P(t)dt}& =e^{\int \frac{1}{80+t}dt}=e^{ \ln|80+t|+C}\\}
\uncover<5->{ &=|80+t|=80+t.\;\;\;\uncover<6->{\textrm{ So take } I(t)=80+t.}}
 \end{align*}
\uncover<7->{ 
 Then} 
 \begin{align*}
\uncover<7->{ y& =\frac{1}{80+t}\int(80+t)\frac12 dt} \uncover<8->{=\frac{1}{80+t}\cdot\left(40t+\frac{t^2}{4}+C\right)}
 \end{align*}
\uncover<9->{  
 To solve for $C$, we note that initially there was no chlorine in the tank.  So $y(0)=0$.} 
\end{example}

\end{frame}



\begin{frame}

$\ds 
y=\frac{1}{80+t}\cdot\left(40t+\frac{t^2}{4}+C\right)
$. \\ 
$y(0)=0$, gives:                                     

\[
0=\frac{1}{80}\cdot C\;\Rightarrow\; C=0
\]      
\pause So
\[
 y=\frac{1}{80+t}\cdot\left(40t+\frac{t^2}{4}\right)
\]                                     
\pause 
 and after 20 seconds 
\begin{align*}
y& =\frac{1}{80+20}\cdot\left(40(20)+\frac{(20)^2}{4}\right)\\
&=\frac{900}{100}=9
\end{align*} 
\pause                                        
So after 20 seconds, there will be $9$ grams of chlorine in the tank.    

\end{frame}


