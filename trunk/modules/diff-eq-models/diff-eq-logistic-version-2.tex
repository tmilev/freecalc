% begin module diff-eq-logistic
\begin{frame}
\frametitle{Law of Natural Growth: $ \ds \frac{dP}{dt}=kP$}
\begin{itemize}
%\item  In other words, if a certain number of bacteria produce a certain number of offspring in a certain time, then ten times that many bacteria produce ten times that many offspring in the same time.
\item  This model works well under ideal conditions. This is plausible when the population has unlimited food and environment and no restrictions on its size.
\item  In real life, most populations are constrained by the environment, the amount of food, etc.
\item  Many populations start by increasing exponentially, but then level off when they approach some upper bound, called the carrying capacity $K$.
\item<2->  To take this into account, make two assumptions:
\begin{itemize}
\item<2->  $\frac{\diff P}{\diff t} \approx kP$ if $P$ is small (Initially, the growth rate is proportional to $P$).
\item<2->  $\frac{\diff P}{\diff t} < 0$ if $P > K$ ($P$ decreases if it ever exceeds $K$).
\end{itemize}
\item<3->  Here is an expression that takes both assumptions into account:
\[
\uncover<3->{%
\frac{\diff P}{\diff t} = kP\left( 1 - \frac{P}{K}\right)
}%
\]
\item<3->  This is called the \textit{logistic differential equation}.
\end{itemize}
\end{frame}
% end module diff-eq-logistic
