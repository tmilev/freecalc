% begin module natural-growth-reminder
\begin{frame}
\frametitle{The Law of Natural Growth}
\begin{itemize}
\item  Recall (from Calc I) that differential equations can be used to model population growth.
\item  The Law of Natural Growth works in ideal cases, where populations are unconstrained by lack of food, or the environment.\\
For example, under ideal conditions, if a certain number of bacteria produce a certain number of offspring in a certain time, then ten times that many bacteria produce ten times that many offspring in the same time.
\item  Let $P(t)$ be the population at time $t$.
\item  Then the \textbf{Law of Natural Growth} says:
\end{itemize}
\[
\frac{\diff P}{\diff t} = kP
\]

\begin{itemize}
\item  The proportionality constant $k$ is sometimes called the \textit{\textbf{relative growth rate}}.

\end{itemize}
\end{frame}
% end module natural-growth-reminder
