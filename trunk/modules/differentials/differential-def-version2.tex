% begin module differential-def
\begin{frame}
\frametitle{Differentials}
\begin{itemize}
\item<1-> Recall $\Delta y, \Delta x$ stand for change of $x,y$. Recall: \alert<12>{$\displaystyle \Delta y\approx \frac{\diff y}{\diff x} \Delta x$}
\item $\displaystyle \only<1-2>{\Delta y} 
\only<3->{\alert<3,6> {\alert<7>{\diff}y}} \only<1-3>{\approx}
\only<4->{\alert<3> =} \only<1->{\frac{\diff y}{\alert<5>{\diff x}}}
\only<1-3>{ \Delta x} 
\only<4->{\alert<4,5,8>{\alert<7>{\diff}x}}
\only<6->{=\alert<6>{\alert<7>{\diff}y} }
$
\item<2-> If we substitute \alert<3>{$\Delta y $ by the formal expression $\diff y$} and \alert<4>{$\Delta x$ by the formal expression $\diff x$}, the expression \alert<5>{$\diff x$ appears to ``cancel''} to give a \alert<6>{formal identity}.
\item<7-> Define the \alert<7,11>{\emph{differential $\diff$}} %\uncover<8->
{ and the \alert<8,10>{\emph{differential forms $\diff x$, $\diff(f(x))$}}} %\uncover<9->
{by requesting that \alert<9>{$\diff$ and $\diff x$ satisfy the transformation law} 
\[
\alert<9>{\alert<8>{\alert<7>{\diff}(f(x))}=f'(x) \alert<8>{\alert<7>{\diff}x}}
\] 
for any differentiable function $f(x)$.} In abbreviated notation:
\[ 
\alert<9>{ \alert<7>{\diff}f = f' \alert<8>{\alert<7>{\diff}x}}
\]
\uncover<10->{Expressions containing expression of the form $\alert<10>{\alert<11>{\diff}(something)}$ are called \alert<10>{differential forms}.}
\end{itemize}
\end{frame}
\begin{frame}
\begin{itemize}
\item $\alert<4,9,12>{\alert<3>{ \alert<2>{\diff} f(x)}= f'(x) \alert<3>{\alert<2>{\diff}x}}$.
\item<2-> On the previous slide we stated the \alert<2>{differential $\diff$} and the \alert<3>{differential forms} $\alert<3>{\diff x, \diff f(x)}$ are \alert<4,8>{formal expressions related by a transformation law}.
\item<5-> The precise definitions of differential forms and differentials are outside of the scope of Calculus I and II. 
\item<6-> Differential forms ``encode'' linear approximations which in turn ``encode'' ``infinitesimal'' lengths of segments.
\item<7-> Courses such as ``Integration and Manifolds'' or ``Differential geometry'' usually give precise definitions and fill in the details.
\item<8-> Nonetheless, \alert<8>{what we studied} is \alert<9>{completely sufficient} for practical purposes and \alert<9>{carrying out computations}.
\item<10-> \alert<10,11>{\textbf{Do not confuse differentials with derivatives.}} \uncover<12->{\alert<12>{The correct equality is this.}}

\[
\only<10>{\alert<10,11>{ \diff f(x) = f'(x)}} \only<11->{\alert<10,11>{ \xcancel{ \diff f(x) = f'(x)}}}
\quad \quad \quad\quad \quad \uncover<12>{\alert<12>{\diff f(x)=f'(x)\diff x}}
\]
\end{itemize}
\end{frame}
