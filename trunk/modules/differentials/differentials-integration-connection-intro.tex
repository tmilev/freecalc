%begin module differentials-integration-connection-intro
\begin{frame}
\begin{columns}

\column{0.25\textwidth}
\begin{pspicture}(-0.2,-0.2)(3,1)%
\tiny%
\fcBoundingBox{-0.3}{-0.4}{3}{2}
\pstVerb{
5 dict begin
/theFunctioN {x 0.7 sub dup mul 0.3 add x div} def
/leftEndPt 0.4 def
/rightEndPt 2.8 def
/fa 1 dict begin /x leftEndPt def theFunctioN end def
/fb 1 dict begin /x rightEndPt def theFunctioN end def
}%
\uncover<1-2>{%
\pscustom*[linecolor=\fcColorAreaUnderGraph]{%
\psplot{leftEndPt}{rightEndPt}{theFunctioN}%
\psline(! rightEndPt fb)(! rightEndPt 0)(! leftEndPt 0) (! leftEndPt fa)%
}%
\psplot[linecolor=\fcColorGraph]{leftEndPt}{rightEndPt}{theFunctioN}%
}%
\only<3>{%
\fcRiemannSum{0.4}{2.8}{theFunctioN}{4}{0}%
\fcLengthIndicatorTwo[t]{leftEndPt}{-0.2}{1}{-0.2}{$\vphantom{\overline{X}} \Delta x$}%
}%
\only<4>{%
\fcRiemannSum{leftEndPt}{rightEndPt}{theFunctioN}{8}{0}%
\fcLengthIndicatorTwo[t]{leftEndPt}{-0.2}{0.7}{-0.2}{$\vphantom{\overline{X}} \Delta x$}%
}%
\only<5>{%
\fcRiemannSum{leftEndPt}{rightEndPt}{theFunctioN}{12}{0}%
\fcLengthIndicatorTwo[t]{leftEndPt}{-0.2}{0.6}{-0.2}{$\vphantom{\overline{X}} \Delta x$}%
}%
\only<6->{%
\fcRiemannSum{leftEndPt}{rightEndPt}{theFunctioN}{24}{0}%
\fcLengthIndicatorTwo[t]{leftEndPt}{-0.2}{0.5}{-0.2}{$\vphantom{\overline{X}} \Delta x$}%
}%
\fcAxesStandardNoFrame{-0.2}{-0.2}{3}{2}%
\only<7>{\psline[linecolor=red, linewidth=2pt](! leftEndPt 0)(! 0.5 0)}
\only<8>{\psline[linecolor=red, linewidth=2pt](! leftEndPt 0)(! leftEndPt fa)}
\pstVerb{end}%
\end{pspicture}
\column{0.75\textwidth}
\begin{itemize}
\item<1->  $\displaystyle \only<3>{\color{red} } \int \limits^{{\only<3>{\color{black} } b}}_{{\only<3>{\color{black} } a}} \only<3>{\color{black}}  \alert<5>{f( x) }\alert<7>{ \diff x} $ is the definite integral of $f$.
\item<2-> $\displaystyle\alert<3>{ \int} f(x) \diff x$ = corresponding anti-derivative.
\item<3-> $\alert<3-6>{\int}$ stands for the \alert<3-6>{limit of a Riemann sum} (sum of \alert<7,8,9>{approximating rectangles}).
\end{itemize}
\end{columns}
\begin{itemize}

\item<7-> $\alert<7>{\diff x}$ ``encodes''  \alert<3>{the base length} of ``\alert<5>{infinitesimally small}'' approximating rectangle\uncover<8->{, $\alert<8>{f(x)}$ is the \alert<8>{height}.}
\item<9-> $f(x) \diff x$ is a differential form as discussed already.
\item<10-> We postponed a formal definition of differential form to another course, but we showed how to compute with those. 
\item<11-> \alert<11>{This is consistent:} \alert<12>{integrals of equal differential forms are equal} \uncover<13->{(follows from Net Change Theorem (subst. rule)).}
\end{itemize}
\end{frame}

%end module differentials-integratino-connection-intro