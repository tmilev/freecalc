\begin{frame}
\begin{question}
Can we plot the graph of an arbitrary equation, $F(x,y)=G(x,y)$?
\end{question}
\begin{itemize}
\item<2-> For sufficiently well behaved equations the answer is yes.
\item<3-> We illustrate one computer algorithm for doing this.
\item<4-> The theory behind plotting arbitrary equations, even when they are well behaved, is well beyond the scope of the present course. 
\item<5-> In particular, while computer algorithms plot graphs of well-behaved equations relatively well, it is not clear why those algorithms work.
\item<6-> When, using algebra, we can express one variable in terms of the other, it is easy to produce the graph of the equation.
\end{itemize}
\end{frame}

\begin{frame}
\begin{question}
Can we plot the graph of an equation of the form $y=f(x)$ or $x=h(y)$ (for continuous $h,f$)? Yes.
\end{question}
\begin{columns}
\column{0.3\textwidth}
\psset{xunit=0.2cm, yunit=0.2cm}
\begin{pspicture}(-1,-1)(1,1)%
\tiny%
\newcommand{\theFun}{-1 3 div x mul 1 add\space}%
\newcommand{\xMax}{9\space}%
\newcommand{\xMin}{-9\space}%
\fcAxesStandard{-10}{-10}{10}{10}%
\only<handout:1|13->{\renewcommand{\numPoints}{10}}%
\only<handout:0|14->{\renewcommand{\numPoints}{19}}%
\only<handout:2|15->{\renewcommand{\numPoints}{37}}%
\only<handout:3,4|16->{\renewcommand{\theFun}{1 9 div x x mul mul 0.5 x mul sub 4 sub\space }}
\only<handout:3|16->{\renewcommand{\numPoints}{10}}%
\only<handout:0|17->{\renewcommand{\numPoints}{19}}%
\only<handout:4|18->{\renewcommand{\numPoints}{37}}%
\only<handout:5,6|19->{\renewcommand{\theFun}{-1 9 div x x mul mul 0.5 x mul sub 6 add\space }}
\only<handout:3|19->{\renewcommand{\numPoints}{10}}%
\only<handout:0|20->{\renewcommand{\numPoints}{19}}%
\only<handout:4|21->{\renewcommand{\numPoints}{37}}%
\rput[l](-10, -5){\uncover<7-12>{$\alertNoH{5}{n=3}$,} \uncover<8-12>{$\alertNoH{6}{\Delta = \frac{9-(-9)}{3}=6}$,} \uncover<9-12>{$\alertNoH{7}{x_j=-9+6j}$}}%
\newcommand{\numPoints}{4}%
\uncover<5->{%
\psline(9, -0.4) (9, 0.4)%
\rput[t](9, -0.5){$9$}%
}%
\uncover<4->{%
\psline(-9, -0.4)(-9, 0.4)%
\rput[t](-9, -0.5){$-9$}%
}%
\pstVerb{40 dict begin %
/xMax \xMax def %
/xMin \xMin def %
/numPoints \numPoints\space def %
/Delta xMax xMin sub numPoints 1 sub div def %
}%
\multido{\na=0+1}{\numPoints}{%
\pstVerb{%
/counter \na \space def
/theX counter Delta mul xMin add def
/theXnext theX Delta add def
theXnext xMax gt{/theXnext theX def}if
/theY 1 dict begin /x theX def \theFun end def
/theYnext 1 dict begin /x theXnext def \theFun end def
}%
\uncover<10-12>{%
\psline(! theX -0.4)(! theX 0.4)%
\rput[b](! theX 0.5){$x_{\na}$}%
}%
\uncover<11->{\fcFullDot{theX}{theY}}%
\uncover<12->{\psline[linecolor=\fcColorGraph](! theX theY)(! theXnext theYnext)}%
}%
\pstVerb{end}%
\end{pspicture}
\uncover<3->{Demonstration of the algorithm for \only<handout:1,2|-15>{$\alertNoH{3}{y=-\frac{1}{3}x+1}$}
\only<handout:3,4|16-18>{$\alertNoH{16}{y=\frac{1}{9}x^2-\frac{1}{2}x+4}$}
\only<handout:3,4|19-21>{$\alertNoH{19}{y=-\frac{1}{9}x^2-\frac{1}{2}x+6}$}
} 

\uncover<4->{from $\alertNoH{4}{x=-9}$ to $\alertNoH{5}{x=9}$.} 
\column{0.7\textwidth}
\begin{itemize}
\item<2-> Suppose $y=f(x)$. 
\item<4-> To plot the graph from $\alertNoH{4}{x=a}$ to $\alertNoH{5}{x=b}$, select $n+1$ points $x_0, x_1, \dots, x_n$ on $[a,b]$. 
\begin{itemize}
\item<6-> Usually we choose the points to be evenly spaced with $x_0=a$ and $x_n=b$.
\item<7-> $n+1$ points split $[a,b]$ into $n$ intervals. 
\item<8-> Each interval has length $\alertNoH{6}{\Delta=\frac{b-a}{n}}$.
\item<9-> The formula for the $j^{th}$ point is then $\alertNoH{7}{x_j=a+\Delta\cdot j}$.
\end{itemize}
\item<11-> The points $(x_0, f(x_0)), \dots, (x_n, f(x_n))$ all lie on the graph. 
\item<12-> Connect them with straight lines. 
\item<13-> Repeat for increasing number of segments $n$.
\end{itemize}
\end{columns}
\end{frame}