\begin{frame}
\begin{example}
Most of the carbon in a living organism is the stable isotope ${}^{12}C $ but there is a near constant ratio of the radioactive isotope ${}^{14}C$. After death, the radioactive carbon gradually decays, leaving a smaller ratio of radioactive carbon compared to living organisms. An approximate law measuring the ratio of ${}^{14}C $ to ${}^{12}C $ (by mass) is given by the formula $10^{-12} e^{-\frac{t}{8223}}$, where $t$ is the number of years after death. A fossil has ratio of ${}^{14}C $ to ${}^{12}C $ equal to $0.5\cdot 10^{-13}$. Estimate the age of the fossil. 

\uncover<2->{
$
\begin{array}{r@{}c@{}l}
\displaystyle 10^{-12} e^{-\frac{t}{8223}} &=&\displaystyle \frac{1}{2}\cdot 10^{-13} \\
\displaystyle  e^{-\frac{t}{8223}}&=&\displaystyle \frac{1}{2}\cdot 10^{-1}=\frac{1}{20} \\
\displaystyle -\frac{t}{8223}&=&\displaystyle \ln \left(\frac{1}{20}\right)\\
t&=&\displaystyle -8223\ln \left(\frac{1}{20}\right)=8223\ln \left(20\right)\\
t&\approx & 24633.
\end{array}
$

We estimate the fossil's age to be between $24000$ and $25000$ years.
}
\end{example}
\end{frame}