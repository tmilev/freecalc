% begin module exponential-def-various-approaches
\begin{frame}
\frametitle{Exponents overview}
\begin{itemize}
\item<1-> For integer $x$, we know how to compute $a^{x}$ as a function of $a$. 
\item<2-> How do we compute $f(x)=a^x$ when $x$ is not an integer?
\item<3-> We need to go back to the definition of $a^x$ (for $x$ non-integer).
\item<4-> In what follows we give/recall an elementary way to define exponent.
\item<5-> Then we give an alternative second definition. 
\item<6-> The second definition will be studied in sufficient depth only much later. 
\item<7-> The two definitions are equivalent: if we choose one definition the other becomes a theorem and the other way round.
\item<8-> Choosing one definition makes some statements easier to prove and others more difficult.
\item<9-> We shall discuss pros and cons of the two. In a nutshell: 
\begin{itemize}
\item<10-> the first elementary definition is easier to motivate;
\item<11-> the second alternative definition is easier to compute with.
\end{itemize}
\end{itemize}
\end{frame}
\begin{frame}
\frametitle{Exponent definition using limits (approach I)}
\begin{itemize}
\item<1-> For integer $p$ we know to compute $a^p$.
\item<2-> Therefore for integer $q$ we know to compute $a^{\frac{1}{q}}= \sqrt[q]{a}=\max\{x|\text{~for~which~} x^q\leq a\}$.
\item<3-> Therefore we know to compute $a^{\frac{p}{q}}$ for all rational $\frac{p}{q}$.
\item<4-> We can then define
\[
a^x = \lim\limits_{\substack{y \to x \\ y\text{-rational}}} a^y 
\]
For example, $a^\pi$ would be defined as the limit of the sequence $a^{3.14}, a^{3.141}, a^{3.1415}, \dots$.
\item<5-> Cons: not computationally effective; not how computers compute.
\item<6-> Pros: for non-integer $x$ and $y$, it is very easy to prove that $a^{x+y}=a^xa^y$ - this follows from the definition of limit above.
\item<7->\alertNoH{7}{This is the definition assumed in many elementary courses.}
\end{itemize}
\end{frame}
\begin{frame}
\frametitle{Exponent definition using series (approach II)}
\begin{itemize}
\item<1-> The following formula (studied much later) can be used as alternative definition.
\[
e^{x}=\alertNoH{4}{\sum_{n=0}^{\infty}} \frac{x^n}{\alertNoH{2}{n!}}= 1+ x+\frac{x^2}{2!}+\frac{x^3}{3!}+\dots + \frac{x^{n}}{\alertNoH{2}{n!}}+\dots
\]
\uncover<2->{\alertNoH{2}{Here $n!=1\cdot 2\cdot 3\cdot\dots \cdot(n-1)\cdot n$ and is read ``$n$ factorial''. }}
\item<3-> For $|x|<1$ define 
\[
\ln (1+x)=\alertNoH{4}{\sum_{n=1}^{\infty}} (-1)^{n+1}\frac{x^n}{n}=  x-\frac{x^2}{2}+\frac{x^3}{3}-\dots + \frac{(-1)^{n+1}x^{n}}{n}+\dots
\]
\uncover<4->{\alertNoH{4}{Infinite sum studied much later.}} 
\item<5-> For arbitrary $a>0$ define $a^x$ as $a^x=e^{x\ln a}$. 
\item<6-> Cons: more difficult to prove $e^{x+y}= e^{x} e^y$ and $e^{\ln(1+x)}=1+x$, proof done in later.
\item<7-> Pros: this is how $e^x$ and $a^x$ are actually computed (by modern computers and by humans in the past).
\end{itemize}
\end{frame}

% end module exponential-def-various-approaches
