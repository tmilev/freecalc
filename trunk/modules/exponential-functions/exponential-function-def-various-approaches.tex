% begin module exponential-def-various-approaches
\begin{frame}
\frametitle{Exponents overview}
\begin{itemize}
\item<1-> Previously, for fixed $x$, we studied $a^{x}$ as a function of $a$. 
\item<2-> In present lecture we study $f(x)=a^x$ as a function of $x$.
\item<3-> A construction of $a^x$ was previously promised.
\item<4-> There are several equivalent ways of defining $a^x$. 
\item<5-> We give the easiest definition. 
\item<6-> We give a second equivalent definition. The second definition is studied in detail in Calculus II. 
\item<7->We discuss pros and cons.
\end{itemize}
\end{frame}
\begin{frame}
\frametitle{Exponent definition using limits (approach I)}
\begin{itemize}
\item<1-> For integer $p$ we know to compute $a^p$.
\item<2-> Therefore for integer $q$ we know to compute $a^{\frac{1}{q}}= \sqrt[q]{a}=\max\{x|\text{~for~which~} x^q\leq a\}$.
\item<3-> Therefore we know to compute $a^{\frac{p}{q}}$ for all rational $\frac{p}{q}$.
\item<4-> We can then define
\[
a^x = \lim\limits_{\substack{y \to x \\ y\text{-rational}}} a^y 
\]
\item<5-> Not computationally effective. It is not how computers compute.
\item<6-> However is the easiest approach. $a^{x+y}=a^xa^y$ is easiest to prove (follows directly from the $\varepsilon, \delta$-definition of $\lim$).
\item<7->\alert<7->{This is the definition assumed in Calculus I.}
\end{itemize}
\end{frame}
\begin{frame}
\frametitle{Exponent definition using series (approach II)}
\begin{itemize}
\item<1-> The Calc II formula can be used as alternative definition.
\[
e^{x}=\alert<4>{\sum_{n=0}^{\infty}} \frac{x^n}{\alert<2>{n!}}= 1+ x+\frac{x^2}{2!}+\frac{x^3}{3!}+\dots + \frac{x^{n}}{\alert<2>{n!}}+\dots
\]
\uncover<2->{\alert<2>{Here $n!=1*2*3*\dots *(n-1)*n$ and is read ``$n$ factorial''. }}
\item<3-> For $|x|<1$ define 
\[
\ln (1+x)=\alert<4>{\sum_{n=1}^{\infty}} (-1)^{n+1}\frac{x^n}{n}=  x-\frac{x^2}{2}+\frac{x^3}{3}-\dots + \frac{(-1)^{n+1}x^{n}}{n}+\dots
\]
\uncover<4->{\alert<4>{Infinite sum studied in Calc II.}} 
\item<5-> For arbitrary $a>0$ define $a^x$ as $a^x=e^{x\ln a}$. 
\item<6-> Disadvantage: more difficult to prove $e^{x+y}=e^{x}e^y$ and $e^{\ln(1+x)}=1+x$, proof done in Calculus II.
\item<7-> This is how computers compute $e^x$ and $a^x$.
\end{itemize}
\end{frame}

% end module exponential-def-various-approaches
