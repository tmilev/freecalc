\begin{frame}
\frametitle{Exponential growth/decay model}
\begin{itemize}
\item Exponential growth models can be applied to model (the initial stage of) growth in biological systems where the environment limitations are negligible.
\item Exponential decay models use similar formulas but have negative coefficients in the exponent. Exponential decay models are used (among other applications) for carbon dating.
\end{itemize}

\psset{xunit=0.6cm, yunit=0.6cm}
\begin{pspicture}(-1,-1)(1,1)
\tiny 
\fcAxesStandard{-3}{-1}{2.5}{3}
\psplot[linecolor=\fcColorGraph]{-3}{1}{2.718281828 x exp}
\rput[tl](0.2,1){$\begin{array}{l}f(x)=be^{ax}\\ a>0 \end{array}$}
\rput[bl](-1.85,-0.9){\text{Exponential growth}}
\end{pspicture}
\psset{xunit=0.6cm, yunit=0.6cm}
\begin{pspicture}(-1,-1)(1,1)
\tiny 
\fcAxesStandard{-1.2}{-1}{3.2}{3}
\psplot[linecolor=\fcColorGraph]{-1}{3}{2.718281828 x -1 mul exp}
\rput[tl](0.2,1.8){$\begin{array}{l}f(x)=be^{-ax}\\ a>0 \end{array}$}
\rput[bl](0.05,-0.9){\text{Exponential decay}}
\end{pspicture}

\end{frame}