\begin{frame}
\begin{example}
In a hypothetical experiment, the number of E. Coli bacteria cells $E(t)$ is modeled with a logistic curve $\displaystyle E(t)= \frac{2.6\times 10^{11}}{1+ (3.94\times 10^9) e^{-1.387t }} $, where $t$ measures time in hours since the start of the experiment. Provided that the model is accurate, answer the following.
\begin{itemize}
\item Approximately how many cells were there at the start of the experiment?
\item How many hours are needed for the number of cells to be approximately $3.3\times 10^{9}$?
\end{itemize}
\uncover<2->{
According to the model, the number of cells in the beginning of the experiment must have been $\displaystyle E(0)=\frac{4\times 10^{9}}{1+ (6.06\times 10^7) e^{-1.387\cdot 0 }}=  \frac{4\times 10^{9}}{1+ (6.06\times 10^7) e^{0}}=  \frac{4\times 10^{9}}{1+ (6.06\times 10^7)} \approx  \frac{4\times 10^{9}}{6.06\times 10^7}= \frac{4\times 10^2}{6.06}\approx \frac{400}{6}\approx  66.7 \approx 67 $ cells.
}
\end{example}

\end{frame}