\solution{\ref{problemEvenOrOdd:(1-x)/(1+x)+(1+x)/(1-x)}.

To check whether a function $f$ is even, odd or neither, we need to compare $ f(x)$ to $f(-x)$. We have that

\[
\begin{array}{rcl}
\displaystyle f (x)&=&\displaystyle \frac{1-x}{1+x}+\frac{1+x}{1-x}\\
&=&\displaystyle \frac{(1-x)(1-x)}{(1+x)(1-x)}+\frac{(1+x)(1+x)}{(1-x)(1+x)}\\
&=&\displaystyle \frac{(1-x)^2+(1+x)^2}{(1+x)(1-x)}\\
&=&\displaystyle \frac{(1-\cancel{2x}+x^2)+(1+\cancel{2x}+x^2)}{1-x^2}\\
&=&\displaystyle \frac{2+2x^2}{1-x^2}\\
\multicolumn{3}{c}{\text{Therefore}}\\
f(-x)&=& \displaystyle \frac{2+2(-x)^2}{1-(-x)^2}\\
&=& \displaystyle \frac{2+2x^2}{1-x^2}.\\
\end{array}
\]
Thus we computed that $f(-x)=f(x)$, which shows that the function is even. Even functions have graphs that are symmetric across the $y$ axis; a computer-generated plot of $f$ confirms this symmetry.

\psset{xunit=0.3cm, yunit=0.3cm}
\begin{pspicture}(-1,-1)(1,1)
\tiny
\newcommand{\theFun}{x x mul 1 add 2 mul 1 x x mul sub div}
\fcAxesStandard{-10}{-10}{10}{10}
\fcLabels{10}{10}
\fcXTickWithLabel{1}{$1$}
\fcYTickWithLabel{1}{$1$}
\psline[linestyle=dashed, linewidth=0.2pt](-1,-10)(-1,10)
\psline[linestyle=dashed, linewidth=0.2pt](1,-10)(1,10)
\psplot[linecolor=\fcColorGraph]{-10}{-1.23}{\theFun}
\psplot[linecolor=\fcColorGraph]{-0.81}{0.81}{\theFun}
\psplot[linecolor=\fcColorGraph]{1.23}{10}{\theFun}
\end{pspicture}
}

\solution{\ref{problemEvenOrOdd5x+4ifx>0and5x-4ifx<0}.

We will show that this piecewise defined function is odd, although each of the individual pieces ($5x+4$ and $5x-4$), viewed over the entire real line, is neither even nor odd.

This problem can be solved both via algebra and graphically.

\textbf{Solution via algebra. }
Recall that a function is even when $f(x)=f(-x)$ and odd when $f(-x)=-f(x)$. We have 
\[
f(x)=\begin{cases}
5x+4 & \text{ if } x>0 \\ 
5x-4 & \text{ if } x<0 \end{cases}
\]
and therefore 
\[
\begin{array}{rcll|l}
f(-x)&=&\begin{cases}
-5x+4 & \text{ if } -x>0 \\ 
-5x-4 & \text{ if } -x<0\end{cases} && \begin{array}{l}\text{Multiplying}\\
\text{inequalities by }-1\\ \text{reverses their direction} \end{array}\\
&=&\begin{cases}
-5x+4 & \text{ if } x<0 \\ 
-5x-4 & \text{ if } x>0\end{cases} &&\begin{array}{l}\text{swap the order} \\ \text{of writing the cases}\end{array}\\
&=&\begin{cases}
-5x-4 & \text{ if } x>0 \\
-5x+4 & \text{ if } x<0  
\end{cases} \\
&=&\begin{cases}
-(5x+4) & \text{ if } x>0 \\
-(5x-4) & \text{ if } x<0  
\end{cases} \\
&=&-f(x).
\end{array}
\]
This shows that the function is odd. 


\textbf{Solution via plotting the function.} This graphical solution is slightly informal, but shows a good understanding of the subject and is acceptable (and well perceived by graders) when taking exams. 

We recall that a function is even if its graph is symmetric across the $y$ axis and odd if its graph has a half-turn symmetry about the origin of the coordinate system.  Plotting  $f(x)=\begin{cases}
5x+4 & \text{ if } x>0 \\ 
5x-4 & \text{ if } x<0 \end{cases}$ results in the following graph:
\begin{center}
\psset{xunit=0.1cm, yunit=0.1cm}
\begin{pspicture}(-1, -1)
\fcAxesStandard{-20}{-20}{20}{20}
\psplot[linecolor=\fcColorGraph, plotpoints=5]{0}{3.2}{5 x mul 4 add}
\psplot[linecolor=\fcColorGraph, plotpoints=5]{-3.2}{0}{5 x mul -4 add}
\fcFullDot{0}{4}
\fcFullDot{0}{-4}
\end{pspicture}
\end{center}
The graph is symmetric relative to rotation at $180^{\circ}$ around the origin so $f(x)$ is an odd function.


}