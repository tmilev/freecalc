% begin homework implicit-inverse-trig2
The variables $x$ and $y$ are related by
\[
x^2y+xy^2+\arcsin x = \frac{\pi}{6}.
\]

\begin{enumerate}
\item   Find all points on the graph of this relation for which $x = \frac{1}{2}$.  

\solution{%
Set $x = \frac{1}{2}$ and solve for $y$.  
\begin{align*}
\left(\frac{1}{2}\right)^2y+\frac{1}{2}y^2 + \arcsin \frac{1}{2} & = \frac{\pi}{6} \\
\frac{1}{4}y + \frac{1}{2}y^2 + \frac{\pi}{6} & = \frac{\pi}{6} \\
\frac{1}{4}y + \frac{1}{2}y^2  & = 0 \\
\frac{1}{4}y(1 + 2y)  & = 0,
\end{align*}
so $y = 0$ or $y = -\frac{1}{2}$.  
Therefore $(\frac{1}{2},0)$ and $(\frac{1}{2},-\frac{1}{2})$ are the points on the graph of the relation for which $x = \frac{1}{2}$.  
}%

\item   Find $\frac{\diff y}{\diff x}$ in terms of $x$ and $y$.  

\solution{%
Differentiate implicitly.
\begin{align*}
\left((x^2)\frac{\diff}{\diff x}(y) + (y)\frac{\diff}{\diff x}(x^2)\right) + \left( (x)\frac{\diff}{\diff x}(y^2) + (y^2)\frac{\diff}{\diff x}(x)\right) + \frac{1}{\sqrt{1-x^2}} & = 0 \\
x^2\frac{\diff y}{\diff x} + y(2x) + x(2y)\frac{\diff y}{\diff x} + y^2 + \frac{1}{\sqrt{1-x^2}} & = 0 \\
x^2\frac{\diff y}{\diff x} + 2xy + 2xy\frac{\diff y}{\diff x} + y^2 + \frac{1}{\sqrt{1-x^2}} & = 0.
\end{align*}
Rearrange to isolate $\frac{\diff y}{\diff x}$ on one side.  
\begin{align*}
x^2\frac{\diff y}{\diff x} + 2xy\frac{\diff y}{\diff x} & = -y^2-2xy-\frac{1}{\sqrt{1-x^2}} \\
(x^2+2xy)\frac{\diff y}{\diff x} & = -\left(y^2+2xy+\frac{1}{\sqrt{1-x^2}}\right) \\
\frac{\diff y}{\diff x} & = -\frac{y^2+2xy+\frac{1}{\sqrt{1-x^2}}}{x^2+2xy}.
\end{align*}

}%

\item   Find the equation of the tangent to the graph at each of the points you found in the first part.  

\solution{%
To find the slope of the tangent at $(\frac{1}{2},0)$, plug in $x=\frac{1}{2},y=0$ to the formula for $\frac{\diff y}{\diff x}$.  

\begin{align*}
\frac{\diff y}{\diff x} & = -\frac{(0)^2+2(\frac{1}{2})(0) + \frac{1}{\sqrt{1-(\frac{1}{2})^2}}}{(\frac{1}{2})^2+2(\frac{1}{2})(0)} \\
& = -\frac{0+0+\frac{1}{\sqrt{\frac{3}{4}}}}{\frac{1}{4} + 0} \\
& = -\frac{\frac{1}{\frac{\sqrt{3}}{2}}}{\frac{1}{4}} \\
& = -\frac{2}{\sqrt{3}}\cdot \frac{4}{1} \\
& = -\frac{8}{\sqrt{3}}.
\end{align*}

Now use the point $(\frac{1}{2},0)$ to find an equation for the tangent line.  
\begin{align*}
y - 0 & = -\frac{8}{\sqrt{3}}\left(x-\frac{1}{2}\right) \\
y & = -\frac{8}{\sqrt{3}}x +\frac{4}{\sqrt{3}}.
\end{align*}

This is the equation for one of the tangent lines.  

To find the slope of the tangent at $\left(\frac{1}{2},-\frac{1}{2}\right)$, plug in $x=\frac{1}{2},y=-\frac{1}{2}$ to the formula for $\frac{\diff y}{\diff x}$.  

\begin{align*}
\frac{\diff y}{\diff x} & = -\frac{(-\frac{1}{2})^2+2(\frac{1}{2})(-\frac{1}{2}) + \frac{1}{\sqrt{1-(\frac{1}{2})^2}}}{(\frac{1}{2})^2+2(\frac{1}{2})(-\frac{1}{2})} \\
& = -\frac{\frac{1}{4}-\frac{1}{2}+\frac{1}{\sqrt{\frac{3}{4}}}}{\frac{1}{4} -\frac{1}{2}} \\
& = -\frac{-\frac{1}{4}+\frac{2}{\sqrt{3}}}{-\frac{1}{4}} \\
& = \frac{-\frac{1}{4}+\frac{2}{\sqrt{3}}}{\frac{1}{4}} \\
& = 4\left(-\frac{1}{4}+\frac{2}{\sqrt{3}}\right) \\
& = -1 + \frac{8}{\sqrt{3}}.
\end{align*}

Now use the point $(\frac{1}{2},-\frac{1}{2})$ to find an equation for the tangent line.  
\begin{align*}
y - \left(-\frac{1}{2}\right) & = \left(-1 + \frac{8}{\sqrt{3}}\right) \left(x - \frac{ 1}{2}\right) \\
y  & = \left(-1 + \frac{8}{\sqrt{3}}\right)x +\frac{1}{2} - \frac{4}{\sqrt{3}} - \frac{1}{2} \\
y  & = \left(-1 + \frac{8}{\sqrt{3}}\right)x - \frac{4}{\sqrt{3}},
\end{align*}
and this is the equation of the other tangent line.  
}%


\end{enumerate}
% end homework implicit-inverse-trig2
