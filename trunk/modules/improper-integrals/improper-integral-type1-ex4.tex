% begin module improper-integral-type1-ex4
\begin{frame}
\begin{example}[Example 4, p. 547]
For what values of $p$ is the integral $\int_1^\infty \frac{1}{x^p}\diff x$ convergent?
\begin{itemize}
\item<2->  We know from Example 1 that if $p = 1$, the integral is divergent.
\item<3->  Assume $p\neq 1$.
\end{itemize}
\abovedisplayskip=0pt
\belowdisplayskip=0pt
\[
\uncover<4->{%
\int_1^\infty \frac{1}{x^p} \diff x%
}%
 \uncover<4->{ = }  %
\uncover<4->{%
\lim_{t\rightarrow \infty}\int_1^t \frac{1}{x^p} \diff x%
}%
 \uncover<5->{ = }  %
\uncover<5->{%
\lim_{t\rightarrow \infty}\left[ \frac{x^{-p+1}}{-p+1}\right]_1^t%
}%
 \uncover<6->{ = }  %
\uncover<6->{%
\lim_{t\rightarrow \infty}\frac{ \frac{1}{t^{p-1}} - 1}{1-p}%
}%
\]
\begin{itemize}
\item<7->  If $p > 1$, then $p - 1 > 0$, so as $t\rightarrow \infty$, $t^{p-1}\rightarrow \infty$ and $1/t^{p-1}\rightarrow 0$.
\item<8->  Therefore $\int_1^\infty \frac{1}{x^p}\diff x = \frac{1}{p-1}$ if $p > 1$, and so the integral is convergent.
\item<9->  If $p < 1$, then $p - 1 < 0$, so $\frac{1}{t^{p-1}} = t^{1-p} \rightarrow \infty$ as $t\rightarrow \infty$.
\item<10->  Therefore $\int_1^\infty \frac{1}{x^p}\diff x$ is divergent if $p < 1$.
\end{itemize}
\end{example}
\uncover<11->{%
\begin{theorem}
$\int_1^\infty \frac{1}{x^p}\diff x$ converges if $p > 1$ and diverges if $p \leq 1$.
\end{theorem}
}%
\end{frame}
% end module improper-integral-type1-ex4
