\begin{frame}
\small
  \frametitle{Conservation of Energy}

  What does a conservative field $\fcv{F}$ conserve?

  \pause Particle of mass $m$ in a force field $\fcv{F} = \nabla f$

  \pause Moves from from $A$ to $B$, with $\fcv{r}(t)$ the position vector at time $t$.

  \begin{itemize}
    \item \pause $K = \frac{1}{2}\, mv^2 = \frac{1}{2}\, m|\fcv{r}'(t)|^2$ is the kinetic energy;
    \item \pause $V=-f$ is called the \emph{potential} energy;
    \item \pause $E=K+V$ is the
\emph{total} energy.
  \end{itemize}
  %
  $$V(A)-V(B) = f(B) - f(A) = \int_C (\nabla f) \cdot \fcv{\diff r} = \int_C \fcv{F} \cdot \fcv{\diff r} = K(B)-K(A)$$
  %
  \pause Therefore the total energy is conserved:
  %
  $$E(B) = K(B)+V(B) = K(A)+V(A) = E(A)\; .$$

\pause
%
$$\text{A scalar potential for gravitational field } \fcv{F}(P) = -\frac{KMm}{|\fcv{r}|^3}\, \fcv{r} \text{ is } f(P) = \frac{KMm}{|\fcv{r}|}  $$
%
$$\text{Potential energy should be } V = -\frac{KMm}{R+h} \text{ and not } V = mgh\; .$$
\pause How do you reconcile the two formulas?
\end{frame}