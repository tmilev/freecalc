\begin{frame}
\frametitle{Conservative Field $ \Rightarrow$ Gradient Field}
\begin{columns}
\column{0.2\textwidth}
\psset{xunit=0.8cm, yunit=0.8cm}
\begin{pspicture}(-0.2,-0.3)(1.5,2.3)
\tiny
\fcBoundingBox{-0.1}{-0.2}{1.3}{2.1}
\fcFullDot{-0.1}{-0.1}
\fcFullDot{1}{2}
\uncover<7->{%
\rput[tl](1.7,1.9){$(x+h,y)$}
\fcFullDot{1.6}{2}
}%
\uncover<7>{%
\psline[linecolor=red, linewidth=2pt](1,2)(1.6, 2)
}%
\uncover<8->{%
\psline[linecolor=blue](1,2)(1.6, 2)
}%
\pscurve[arrows=->](-0.1,-0.1)(1,1)(1,2)
\uncover<1-3>{%
\pscurve[arrows=->](-0.1,-0.1)(0,1)(1,2)%
\rput[rt](0.3,1.55){$C'$} %
}%
\rput[l](0.9,0.5){$C$}
\rput[lt](0,-0.1){$A$}
\rput[br](1.1,2.1){$(x,y)$}
\end{pspicture}
\column{0.8\textwidth}
Let $\alert<13>{\fcv{F}=P\fcv i+Q\fcv j}$ be a smooth conservative field. \uncover<2->{ Fix pt. $A$ inside the domain of $\fcv {F}$.} \uncover<3->{Define $f$ by
\alert<3>{$ \alert<6>{f}(B) = \int_C \alert<5>{\fcv{F}} \cdot \fcv{\diff r}
$}, where \alert<4>{$C$ is any piecewise smooth curve from $A$ to $B$}.}
\uncover<5->{
\begin{theorem}
$ \alert<5>{\fcv{F}} = \alert<6>{\nabla f}$.
\end{theorem}
}
\end{columns}
\uncover<5->{
\begin{proof}
\uncover<7->{ Let $h>0$; for $h$ small, \alert<7>{the segment $S$ from $(x,y)$ to $(x+h,y)$} is in the domain of $\fcv F$.} \uncover<8->{$S$ is given by $\alert<14,15>{\fcv r(t)= (x+t)\fcv i+y\fcv j,  t\in [0,h]}$.} \uncover<15->{\alert<15>{On $S$, $\diff r=\fcv i\diff t$}.}

$
\begin{array}{r@{~}c@{~}l}
\uncover<9->{\frac{\partial}{\partial x} (f) &=& \lim_{h\to 0}\frac{\alert<10>{ f(x+h,y)}-\alert<11>{f(x,y)} }{h}} \uncover<10->{ =\lim_{h\to 0} \frac{1}{h}\left( \alert<10>{ \int_{\alert<12>{C}+S} \fcv F \cdot \diff \fcv r} - \alert<11>{\int_{\alert<12>{C} } \fcv F\cdot \diff \fcv r} \right)} \\
\uncover<12->{&=&\lim_{h\to 0}\frac{1}{h} \int_{ \alert<14, 15>{ S}} \alert<13>{ \fcv F} \cdot \alert<15>{\diff \fcv r}} \\
\uncover<13->{&=& \lim_{h\to 0}\frac{1}{ h} \int_{ \alert<14>{ t=0}}^{\alert<14>{t=h}} \alert<13>{ \left( P( \alert<14>{x+t,y})\alert<17>{\fcv i} + Q( \alert<14>{ x +t,y}) \alert<16>{\fcv j} \right)} \cdot \alert<15>{\alert<16,17>{ \fcv i} \diff t}} \\
\uncover<16->{&=&\lim_{h\to 0}\frac{1}{ h}  \int_{t=0}^h P(x+h,y)\diff t} \uncover<18->{=P(x,y),}
\end{array}
$
\uncover<18->{where the last equality is the single-variable Fundamental Theorem of Calculus.} \uncover<19->{Similarly it follows that $\frac{\partial}{\partial y} (f)=Q(x,y)$.}
\end{proof}
}
\end{frame}


\begin{frame}
\frametitle{Gradient Field $\Rightarrow$ Conservative Field }

\begin{theorem}[Fundamental Theorem of Calculus for Line Integrals]
$\displaystyle
\int_C (\nabla f) \cdot \fcv{\diff r} = f(B) - f(A) \; ,
$
for every smooth curve $C$ from $A$ to $B$.
\end{theorem}
\begin{proof}
$
\begin{array}{rcl}
\int_C (\nabla f) \cdot \fcv{\diff r} &=&\int_C f_x\diff x + f_y\diff y= \int_C \left( f_x x'(t) + f_y y'(t)\right)\diff t \\
&=&\int_a^b \frac{\diff }{\diff t} (f(\fcv{r}(t))  \diff t = f(B) - f(A).
\end{array}
$
\end{proof}

\begin{definition}
If $\fcv{F}= \nabla f$ then $f$ is called \emph{scalar potential} of $\fcv{F}$; $\fcv F$ is called \emph{gradient field}.
\end{definition}

Let $\fcv{F} = \nabla f$ be gradient field. For a curve $C$ joining points $A$ and $B$
\[
\int_C \fcv{F} \cdot \fcv{\diff r} =  \int_C (\nabla f )\cdot \fcv{\diff r}= f(B) - f(A)\;
\]
depends only on $A$ and $B$, but not on $C$ $\Rightarrow$ $\fcv{F}$ is conservative.
\end{frame}