\begin{frame}
\small
  \frametitle{Divergence and Curl in Polar Coordinates}

Suppose that in polar coordinates $(r, \theta)$, the field $\fcv{F}$ is given by
%
$$\fcv{F}(r, \theta) = P(r, \theta) \fcv{e}_{r} + Q(r, \theta) \fcv{e}_{\theta} \; ,$$
%
where $\fcv{e}_{r} = \fcv{e}_{r}(r, \theta)$ and $\fcv{e}_{\theta}=\fcv{e}_{\theta}(r, \theta)$ are the unit polar coordinate vectors.

%
$$\divg \fcv{F} =  \frac{1}{r} \frac{\partial (r P)}{\partial r} + \frac{1}{r} \frac{\partial Q}{\partial \theta} \quad \text{ and } \quad
   \curl_{\fcv{k}} \fcv{F} =  \frac{1}{r} \frac{\partial (r Q)}{\partial r} - \frac{1}{r} \frac{\partial P}{\partial \theta}
$$

\begin{itemize}
  \item \pause If $\fcv{F}$ is a radial field given in polar coordinates by $\fcv{F} = f(r,\theta)\fcv{e}_r$, then
%
$$\curl_{\fcv{k}} \fcv{F} = \frac{1}{r} \frac{\partial (r 0)}{\partial r} - \frac{1}{r}\frac{\partial f(r,\theta)}{\partial \theta} = - \frac{1}{r} \frac{\partial f}{\partial \theta} \, .$$
\item \pause Radial fields $\fcv{F}(r,\theta) = f(r)\fcv{e}_r$ are irrotational.
\end{itemize}

\end{frame}