\begin{frame}
\frametitle{Definition of Line Integral}
\begin{columns}
\column{0.3\textwidth}
\psset{xunit=0.8cm, yunit=0.8cm}
\begin{pspicture}(-0.1,-1.5)(3.5,3)
\tiny%
\renewcommand{\fcScreen}{[-1 4 -1] 0}%
\fcBoundingBox{-0.1}{-1}{3}{3}%
%\fcAxesIIId{2}{2}{2}
\newcommand{\numIterationsPlusOne}{6}%
\newcommand{\numIterations}{5}%
\pstVerb{
10 dict begin
/theCurveX {t 2 exp } def
/theCurveY {t 180 mul sin  } def
/tmax 2 def
/tmin 0 def
/Delta tmax tmin sub \numIterations\space div def
/fOnTheCurve {theCurveY theCurveX 3 div add 1 add} def
}%
\pscustom*[linecolor=orange]{%
\fcCurveIIId{tmin}{tmax}{[theCurveX theCurveY fOnTheCurve]}%
\fcCurveIIId{tmax}{tmin}{[theCurveX theCurveY 0]}%
}%
\fcCurveIIId[linecolor=red]{tmax}{tmin}{[theCurveX theCurveY 0]}%
\fcCurveIIId[linecolor=brown]{tmin}{tmax}{[theCurveX theCurveY fOnTheCurve]}%
\pstVerb{1 dict begin /t tmin def}
\fcLineIIId[linecolor=brown]{[theCurveX theCurveY fOnTheCurve]}{[theCurveX theCurveY 0]}%
\pstVerb{/t tmax def}
\fcLineIIId[linecolor=brown]{[theCurveX theCurveY fOnTheCurve]}{[theCurveX theCurveY 0]}%
\pstVerb{/t 1.7 def}
\fcPutIIId[t]{[theCurveX theCurveY -0.2]}{$C:\left|\begin{array}{r@{}c@{}l} x&=& h(t)\\ y&=&g(t)\end{array} \right.$}
\fcPutIIId[b]{[theCurveX theCurveY fOnTheCurve]}{$f(x,y)$}
\pstVerb{/t 0.1 def}
\fcPutIIId[l]{[theCurveX theCurveY 0.3]}{$\displaystyle \int_C f(x,y)\diff s $}
\pstVerb{end}
\pstVerb{end}
\end{pspicture}
\column{0.7\textwidth}
\begin{definition}
Suppose the limit $$\lim_{\max_k(\text{segment length}) \to 0} \sum_{k=1}^N f(P_k) \cdot \text{length}(D_k)$$
exists and is finite. Then we call this limit the \emph{line integral of $f$ on $C$ with respect to arclength}, and we denote it by 
\[
\int_C f(x,y) \diff s\quad .
\]
\end{definition}
\end{columns}
\uncover<2->{
The line integral is guaranteed to exists if $f$ is a continuous functions or is bounded and continuous except at a finite number of points.
}
\end{frame}