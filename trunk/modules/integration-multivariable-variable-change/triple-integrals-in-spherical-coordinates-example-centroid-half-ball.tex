
\begin{frame}
\begin{example}[Centroid of (filled) hemisphere]
\begin{columns}
\column{0.6\textwidth}
\begin{pspicture}(-1,-1)(1,1)
\tiny
\fcBoundingBox{-1.3}{-0.5}{1.3}{1.4}
\fcStartIIIdScene
\fcSurfaceInScene{90}{0}{0}{360}{
4 dict begin
/rho 1 def
/x u cos v cos mul rho mul def
/y u cos v sin mul rho mul def
/z u sin rho mul def
[x y z]
end
}{}
\fcAxesIIIdInScene{1.4}{1.4}{1.4}
\fcFinishIIIdScene[fastsort=true]
\end{pspicture}
\raisebox{0.85cm}{
$\fcv f:\left|\begin{array}{r@{}c@{}l}x&=&\rho \sin\phi \cos\theta \\y&=&\rho \sin\phi \sin\theta \\z&=&\rho \cos\phi \end{array}\right.$
}
\column{0.4\textwidth}
Find the centroid (geometric center) of a (filled) hemisphere. 
\end{columns}
\begin{itemize}
\item Introduce Cartesian coordinates as illustrated.
\item Let the coordinates of the centroid be $\left(x_C, y_C, z_C\right)$.
\item Region is symmetric with respect to the $z-$axis $\Rightarrow$ centroid is on $z$-axis $\Rightarrow$  $x_C=y_C=0$.
\item $z_C$, is the ``average'' of the $z$ coordinates of the figure:

\hfil $\displaystyle z_C = \frac{1}{\text{Vol}(\mathcal S)} \iiint_{\mathcal S} z ~\diff x~\diff y~\diff z .
$\hfil 
\item Let $\mathcal R$ be the reparametrization of the region in spherical coordinates, $\fcv f (\mathcal R)=\mathcal S$. 
$\mathcal R = \{ (\rho, \phi, \theta) | 0 \leq \rho \leq R, 0 \leq \phi \leq \frac{\pi}{2}, 0 \leq \theta \leq 2\pi \}.$
\end{itemize}
\end{example}
\end{frame}

\begin{frame}
\begin{example}[Centroid of (filled) hemisphere]
\begin{columns}
\column{0.6\textwidth}
\begin{pspicture}(-1,-1)(1,1)
\tiny
\fcBoundingBox{-1.3}{-0.5}{1.3}{1.4}
\fcStartIIIdScene
\fcSurfaceInScene{90}{0}{0}{360}{
4 dict begin
/rho 1 def
/x u cos v cos mul rho mul def
/y u cos v sin mul rho mul def
/z u sin rho mul def
[x y z]
end
}{}
\fcAxesIIIdInScene{1.4}{1.4}{1.4}
\fcFinishIIIdScene[fastsort=true]
\end{pspicture}
\raisebox{0.85cm}{
$\fcv f:\left|\begin{array}{r@{}c@{}l}x&=&\rho \sin\phi \cos\theta \\y&=&\rho \sin\phi \sin\theta \\z&=&\rho \cos\phi \end{array}\right.$
}
\column{0.4\textwidth}
Find the centroid (geometric center) of a (filled) hemisphere. 
\end{columns}
$\mathcal R = \{ (\rho, \phi, \theta) | 0 \leq \rho \leq R, 0 \leq \phi \leq \frac{\pi}{2}, 0 \leq \theta \leq 2\pi \}.$. Therefore
\[\begin{array}{r@{~}c@{~}l}
z_C & =&\displaystyle \frac{1}{\text{Vol}(\cR)} \iiint_{\mathcal R} z ~ \diff x \diff y \diff z  = \frac{3}{2\pi R^3} \iiint_{\mathcal R} \rho \cos\phi \cdot \rho^2\sin\phi \, \diff \rho\, \diff \phi\,\diff \theta  \\
& =&\displaystyle \frac{3}{2\pi R^3} \int_{\theta =0}^{\theta=2\pi} \left( \int_{\phi = 0}^{\phi=\pi/2} \left( \int_{\rho = 0}^{\rho = R} \rho^3 \sin\phi\cos\phi \, \diff \rho \right) \, \diff \phi \right) \, \diff \theta  \\
& =&\displaystyle \frac{3}{2\pi R^3} \left(\int_{\theta =0}^{\theta=2\pi} \diff \theta \right) \left( \int_{\phi = 0}^{\phi=\pi/2} \sin\phi\cos\phi\, \diff \phi \right) \left( \int_{\rho = 0}^{\rho = R} \rho^3 \, \diff \rho \right)  \\
& =&\displaystyle \frac{3}{2\pi R^3} \cdot 2\pi \cdot \left( \left. \frac{1}{2}\sin^2\phi \right|_{\phi=0}^{\phi=\pi/2}\right) \cdot \left( \left. \frac{\rho^4}{4} \right|_{\rho=0}^{\rho=R} \right) = \frac{3}{8} \, R\; .
\end{array}
\]

\end{example}



\end{frame}