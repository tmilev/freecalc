\begin{frame}
\begin{itemize}
\item Recall the polar coordinate variable change
\[
\begin{array}{rcl}
x&=&r\cos \theta \\
y&=&r\sin \theta.
\end{array}
\]
\item<2-> This variable change can be thought of as two functions: $x=h(r, \theta)= r\cos \theta $ and $y=g(r,\theta)=r\sin \theta$.
\item<3-> The functions $h,g$ map the two-dimensional plane with coordinates $r,\theta$  into the two-dimensional plane with coordinates $x,y$.
\item<4-> Let $\fcv f:\mathbb R^n\to \mathbb R^n$ be an infinitely differentiable map.
\item<5-> In other words, $\fcv f$ takes $n$ scalar inputs and produces $n$ scalar outputs.
\uncover<6->{
\begin{definition}[Infinitely Smooth Variable Change]
An infinitely differentiable map $\fcv f:\mathbb R^n\to \mathbb R^n$ is called an (infinitely) smooth variable change. 
\end{definition}
}
\end{itemize}
\end{frame}