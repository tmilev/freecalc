\solution{\ref{problemintegralxydxdyoverx=3,x+1=y^2,x=y^2+2y+3}

\begin{pspicture}(-1.5, -3)(6,3)
\fcLabels{6}{3}
\pscustom*[linecolor=\fcColorAreaUnderGraph]{
\parametricplot{-2}{0}{t t mul 2 t mul add 3 add t }
\psline(3, 0)(3,2)
\parametricplot{2}{-2}{t t mul 1 sub t }
}
\pscustom*[linecolor=green]{
\psline(3, 0)(3,2)
\parametricplot{2}{0}{t t mul 1 sub t }
\psline(-1,0)(3,0)
}
\parametricplot[linecolor=\fcColorGraph]{-2.4}{2.4}{t t mul 1 sub t }
\rput[l](0.3, -2){$x= y^2-1$}
\parametricplot[linecolor=\fcColorGraph]{-2.661325}{0.661325}{t t mul 2 t mul add 3 add t }
\rput[lb](4, -2){$x= y^2+2y+3$}
\psline[linecolor=\fcColorGraph](3, -2.3)(3,3)
\rput[r](3, 2.9){$x= 3~$}
\fcFullDot{3}{-2}
\fcFullDot{3}{2}
\fcFullDot{3}{0}
\psline[linecolor=black, arrows=<->](-0.75,0.5)(3, 0.5)
\fcFullDot[linecolor=black]{-0.75}{0.5}
\fcFullDot[linecolor=black]{3}{0.5}
\rput[b](-0.75, 1.5 ){$(y^2-1,y)$}
\psline[linestyle=dotted, arrows=->](-0.75, 1.5)(-0.75, 0.5)
\rput[bl](3, 0.7 ){$~(3,y)$}
\rput(2, 1){$Q_1$}
\rput(1.5, -1){$Q_2$}
\fcAxesStandardNoFrame{-1.5}{-3}{6}{3}
\end{pspicture}

We start by plotting the region. $ x=y^2-1$ is a parabola symmetric across the $x$ axis; $x=y^2+2y+3$ is a parabola with vertex at $x=3$, $y=-2$. The two parabolas intersect when 
\[
\begin{array}{rcl}
x=y^2-1&=& y^2+2y+3\\
2y+4&=&0\\
y&=&-2\\
x&=&y^2-1=(-2)^2-1=3,
\end{array}
\]
i.e., when $(x,y)=(3, -2)$. The line $x=3$ intersects $x+1=y^2$ when $y^2= 3+1 = 4$, i.e., when $(x,y)=(3,\pm 2)$. The line $x=3$ intersects $x=y^2+2y+3$ when $3= y^2+2y+3$. This implies $y(y-2)=0$ and finally we conclude the intersections of $x=3$ with $x=y^2+2y+3$ are $(x,y)=(3,0)$ and $(x,y)= (3,2)$. We can conclude that there three curves are plotted as indicated in the figure. Of the 8 regions bounded by the curves only two are bounded, and only one of them is bounded by all three curves. Since no further instruction is given in the problem, we assume that the intended region is the one bounded by all three curves, i.e., the region indicated in the figure above (the other bounded region can be enclosed using two of the curves only). Let $Q_1$ and $Q_2$ be the regions indicated in the figure above. Those two regions are curvilinear trapezoids with vertical bases. Consider the region $ Q_1$. Fix the $y$ coordinate of a point in $Q_1$. The figure shows that, for that fixed value of $y$, $x$ varies between $y^2-1$ and $3$. For $Q_2$, it similarly follows that, for a fixed $y$, $x$ varies between $y^2-1$ and $y^2+2y+3$. The points in $Q_1$ have $y$ coordinates in the range $y\in [ 0 , 2 ] $, and similarly, in $Q_2$ we have that the $y$ coordinate varies in the range $y\in [ -2 , 0 ] $. Thus our regions are parametrized as
\[
\begin{array}{rcl}
Q_1&=&\{(x,y)| 0\leq y\leq 2, y^2-1\leq x\leq 3 \}\\
Q_1&=&\{(x,y)| -2\leq y\leq 0, y^2-1\leq x\leq y^2+2y+3 \}.\\
\end{array}
\]

Finally our integral becomes 
\[
\begin{array}{rcl}
\displaystyle \iint_{\mathcal R} xy\diff x \diff y&=&\displaystyle \iint_{Q_1}xy \diff x \diff y + \iint_{Q_2}\diff x\diff y \\
&=&\displaystyle \int_{y=0}^{y=2} \left(\int_{x=y^2-1}^{x=3} xy \diff x\right) \diff y+ \int_{y=-2}^{y=0} \left( \int_{x=y^2-1}^{x=y^2+2y+3} xy \diff x \right) \diff y\\
&=&\displaystyle \int_{y=0}^{y=2}\left[ \frac{x^2y}{2}\right]_{x=y^2-1}^{x=3}\diff y+ \int_{y=0}^{y=3}\left[ \frac{x^2y}{2}\right]_{x=y^2-1}^{x=y^2+2y+3}\diff y\\
&=&\displaystyle \int_{y=-2}^{y=0}  \left(-\frac{1}{2} y (y^{2}-1)^{2}+\frac{9}{2} y \right)\diff y  +\int_{y=-2}^{y=0}\left( \frac{1}{2} y (y^{2}+2 y+3)^{2}-\frac{1}{2} y (y^{2}-1)^{2}\right)\diff y\\
&=&\displaystyle\int_{y=0}^{y=2}  \left(-\frac{1}{2} y^{5}+y^{3}+4 y  \right)\diff y  +\int_{y=-2}^{y=0}\left( 2 y^{4}+6 y^{3}+6 y^{2}+4 y\right)\diff y\\
&=&\displaystyle \left[-\frac{1}{12} y^{6}+\frac{1}{4} y^{4}+2 y^{2}\right]_{0}^{2} +\left[\frac{2}{5} y^{5}+\frac{3}{2} y^{4}+2 y^{3}+2 y^{2} \right]_{y=-2}^{y=0}\\
&=&\displaystyle \frac{20}{3} -\frac{16}{5} =\frac{52}{15}
\end{array}
\]

}