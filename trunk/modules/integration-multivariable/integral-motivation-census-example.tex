\begin{frame}
\frametitle{A Cheaper ``Census''}
Imagine we want cheap procedure to estimate population in region $\mathcal{R}$.
\begin{itemize}
\item<2-> We decompose $\mathcal{R}$ into pairwise non-overlapping smaller regions $D_k$ (states, counties, finer division...).
\uncover<3->{%
\[
\text{population}(\mathcal{R}) = \sum_k \text{population}(D_k) = \sum_k \text{density}(D_k) \cdot \text{area}(D_k)
\]
}
\item<4-> To find the population density in $D_k$ we need to count everyone (what an actual census does).
\item<5-> Instead, we estimate the population density as follows.
\begin{itemize}
\item<6-> We pick a sample point $P_k$ in each region $D_k$.
\item<7-> We estimate the population density $\text{density}(D_k)$ by counting people in a small region around $P_k$ ($\text{density\_near}(P_k)$).
\end{itemize}
\item<8-> Our population estimate becomes
\[
\text{population}(\mathcal{R}) = \sum \text{pop.}(D_k) \simeq \sum \text{density\_near}(P_k) \text{area}(D_k).
\]
\end{itemize}
\end{frame}