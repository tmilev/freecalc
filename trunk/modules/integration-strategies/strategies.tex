% begin module strategies


\begin{frame}
\frametitle{Table of Integral Formulas}
\small
\begin{columns}
\begin{column}{0.5\textwidth}
\begin{enumerate}
\item[1.] $ \ds \int x^n \dx = \frac{1}{n+1}x^{n+1}, \hspace{1ex} n\neq-1 $

\item[3.]$ \ds \int e^x \diff x = e^x  $

\item[5.]$ \ds \int \sin x \diff x = -\cos x $

\item[7.]$ \ds \int \sec^2 x \diff x = \tan x $

\item[9.]$ \ds \int \sec x \tan x \diff x = \sec x $

\item[11.]$ \ds \int \sec x \diff x = \ln |\sec x + \tan x| $

\item[13.]$ \ds \int \tan x \diff x = \ln |\sec x|  $

\item[15.]$ \ds \int \sinh x \diff x = \cosh x $

\item[17.]$ \ds \int \frac{a}{a^2+x^2}\diff x = \tan^{-1}\frac{x}{a} $

%\item[19.]$ \ds \int u \hspace{2pt} \diff{v} = uv - \int v \diff u $
\end{enumerate}
\end{column}
\begin{column}{0.5\textwidth}
\begin{enumerate}
\item[2.]$ \ds \int \frac{1}{x}\diff x = \ln |x| $
\item[4.]$ \ds \int a^x \diff x = \frac{1}{\ln a} a^x $
\item[6.]$ \ds \int \cos x \diff x = \sin x $
\item[8.]$ \ds \int \csc^2 x \diff x = -\cot x $
\item[10.]$ \ds \int \csc x \cot x \diff x = -\csc x $
\item[12.]$ \ds \int \csc x \diff x = -\ln |\csc x + \cot x|  $

\item[14.]$ \ds \int \cot x \diff x = -\ln |\csc x|  $

\item[16.]$ \ds \int \cosh x \diff x = \sinh x $

\item[18.]$ \ds \int \frac{1}{\sqrt{a^2-x^2}} \diff x = \sin^{-1} \frac{x}{a} $

\end{enumerate}
\end{column}

\end{columns}



\end{frame}


\begin{frame}
\frametitle{Integration}
Once you are armed with the  basic integration formulas (``Friendly Forms") in the table, if you don't immediately see
how to attack a given integral, you might try the following four-step strategy. \pause 
\begin{enumerate}
\item[1.] \textbf{Simplify the Integrand  if Possible}.\\
 Algebraic manipulation or trig identities sometimes simplify the integrand, making method of
integration more obvious. \\ \pause 
\[
\int \sqrt{x}(1-\sqrt{x})\; dx = \int \sqrt{x}-x\; dx
\]
\pause 

\begin{align*}
\int (\sin(x)+\cos(x))^2\; dx & = \int \sin^2(x)+2\sin(x)\cos(x)+\cos^2(x)\; dx\\
& = \int 1+2\sin(x)\cos(x)\; dx
\end{align*} 


\end{enumerate}

\end{frame}

% % % % % % % % % % % % % % % % % % % % % % % % % % % % % % % %
\begin{frame}
\frametitle{}

\begin{enumerate}
\item[2.] \textbf{Look for an Obvious Substitution}.\\
 Try to find some function in the integrand whose differential also occurs, up to multiplication by a constant. \pause 
For instance if for the integral, 
\[
\int \frac{x}{x^2-4} dx
\]
 we lat $ u=x^2-4, $ then $ du= 2x\; dx $. So we use substitution (instead of partial fractions).
 \end{enumerate} 
 \end{frame}
 
 \begin{frame}
 \begin{enumerate}
\item[3.] \textbf{Classify the Integrand According to Its Form.}\\
 If Steps 1 and 2 have not led to the
solution, then we take a look at the form of the integrand.\\
\begin{enumerate} \pause 
\item[(a)]  \textit{Trigonometric functions}. If is a product of powers of trig functions, then use the methods described in the Trigonometric Integrals lecture(s). \pause 
\item[(b)] \textit{Rational functions}. If is a rational function, try partial fractions.
\pause 
\item[(c)] \textit{Integration by parts}. If is a product of a power of $ x $ (or a polynomial) and a transcendental function (such as a trigonometric, exponential, or logarithmic function), then try IBP,using the LIPET rule.  
\pause 
\item[(d)] \textit{Radicals}. \\
For $ \sqrt{\pm a^2 \pm x^2} $ use trig substitution.\\
For $ \sqrt[n]{g(x)} $ try the rationalizing substitution $ u= \sqrt[n]{g(x)}. $
\pause 
\end{enumerate} 
 \item[4.] lather, rinse, and repeat....Try Again...\\
 
 If the first three steps have not produced the answer, remember that
 there are basically only two methods of integration: substitution and parts.
\end{enumerate}

\end{frame}

% end module integration strategies
% % % % % % % % % % % % % % % % % % % % % % % % % % % % % % % %
% % % % % % % % % % % % % % % % % % % % % % % % % % % % % % % %

% % % % % % % % % % % % % % % % % % % % % % % % % % % % % % % %

