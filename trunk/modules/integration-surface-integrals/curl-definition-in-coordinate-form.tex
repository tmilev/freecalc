\begin{frame}
\frametitle{Curl}
Let $\fcv X=P\fcv i+Q\fcv j +R\fcv k$ be a smooth vector field.

\begin{definition}[Curl, coordinate definition]
The \emph{curl} of a vector field $\fcv X$, denoted by $\curl \fcv X$, is defined by

\hfil $\curl \fcv X = \left(\partial_y R - \partial_z Q\right) \fcv{i} +  (\partial_z P - \partial_x R) \, \fcv{j} + (\partial_x Q - \partial_y P) \, \fcv{k} .$
\end{definition}

$$\curl \fcv{X} = 
\left| \begin{array}{ccc}
\fcv{i} & \fcv{j} & \fcv{k} \\
\partial_x & \partial_y & \partial_z \\
P & Q & R
\end{array}\right| = \nabla \times \fcv{X}\; .$$

%\begin{proposition}[can be used as alternative definition]
%$\curl X$ is the unique vector field that satisfies
%$$\fcv{curl}\, \fcv{X}\cdot \fcv{n} = \text{curl}_{\fcv{n}} (\orth_{\fcv{n}} \fcv{X})$$
%\end{proposition}

\begin{itemize}
\item Just like $\divg$, $\curl$ can be equipped with a coordinate-free definition (in this case the above definition becomes a theorem).
%\item Conversely, the coordinate-free properties of $\curl$ will be a consequence of the above definition.
\end{itemize}

\end{frame}