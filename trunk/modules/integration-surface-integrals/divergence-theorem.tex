\begin{frame}
\frametitle{Divergence}
\begin{itemize}
\item Recall that the divergence of $\fcv{X}$ at $p$ is the density of flux:
$(\divg \fcv{X})(p) = \lim_{D\to \{p\}} \frac{1}{\text{vol}(D)} \iint_S \fcv{X} \cdot \fcv{N}\diff S \; ,$
if the limit exists. 
\item<2->  The limit does exist when $X$ is reasonably smooth.
\item<3-> We have seen similar limits when we talked about average values:
$f(P) = \lim_{D\to\{p\}} \frac{1}{\text{vol}(D)} \iiint_D f(Q)\, \diff V\; .$
\item<4-> The definition of $\divg \fcv{X}$ involves a surface integral and the definition of average involves a triple integral.
\item<5-> The next theorem establishes a connection between surface integrals and triple integrals.
\end{itemize}
\end{frame}

\begin{frame}
  \frametitle{Divergence Theorem}

\begin{theorem}
Let $D$ be a compact set in space with boundary $S$ a piecewise smooth parametrized surface, oriented by the outward normal, and let $\fcv{X}$ be a smooth vector field on $D$ given by
\[\fcv{X}(x,y,z)= P (x,y,z) \fcv{i} + Q(x,y,z) \, \fcv{j} + R(x,y,z)\, \fcv{k}\quad .
\]
Then
\[
\iint_S \fcv{X} \cdot \fcv{\diff S} = \iiint_D \left(\frac{\partial P}{\partial x}+ \frac{\partial Q}{\partial y} + \frac{\partial R}{\partial z} \right) \, \diff V
\]
\end{theorem}
\begin{corollary}
$
\begin{array}{r@{~}c@{~}l}
  (\divg \fcv{X}) (p) &= &
\lim\limits_{D\to \{P\}} \frac{1}{\text{vol}(D)}\iint_S \fcv{X} \cdot \fcv{\diff S}   \\
&= & \lim\limits_{D\to \{P\}} \frac{1}{\text{vol}(D)}\iiint_D \left(\frac{\partial P}{\partial x}+ \frac{\partial Q}{\partial y} + \frac{\partial R}{\partial z} \right) \, \diff V =
  \frac{\partial P}{\partial x}+ \frac{\partial Q}{\partial y} + \frac{\partial R}{\partial z}\; .
\end{array}
$
\end{corollary}
\end{frame}

\begin{frame}
  \frametitle{Divergence Theorem}

%
If $\fcv{X} = P\, \fcv{i} + Q\, \fcv{j} +R\, \fcv{k}$, then
%
$$\divg \fcv{X} = \frac{\partial P}{\partial x}+ \frac{\partial Q}{\partial y} + \frac{\partial R}{\partial z}  = ``\langle \partial_x, \partial_y, \partial_z \rangle \cdot \langle P, Q, R\rangle" =  ``\nabla \cdot \fcv{X}"\; .$$

Intuitive notation:
%
$$\divg \fcv{X} = \nabla \cdot \fcv{X}\, .$$

\pause Confirmed by the example of $\fcv{X} = ax \, \fcv{i} + by\, \fcv{j} + cz\, \fcv{k}$ $\Longrightarrow$ $\divg \fcv{X} = a+b+c$.\pause

\begin{theorem}
  %
$$\iint_S \fcv{X} \cdot \fcv{\diff S} = \iiint_D \divg \fcv{X} \, \diff V$$
\end{theorem}

\begin{itemize}
  \item \pause If $(\divg \fcv{X})(p)>0$, then \pause $p$ acts as a source;
  \item \pause If $(\divg \fcv{X})(p)<0$, then \pause $p$ acts as a sink;
  \item \pause If $\divg \fcv{X} \equiv 0$ on some domain $D$, then \pause $\fcv{X}$ is incompressible on $D$.
\end{itemize}

\end{frame}
