\begin{frame}
  \frametitle{Divergence}

\underline{Recall}: The \emph{divergence}\index{divergence} of $\textbf{X}$ at $p$ is the density of flux\index{density!of flux}:
%
$$(\divg \textbf{X})(p) = \lim_{D\to \{p\}} \frac{1}{\text{vol}(D)} \iint_S \bm{X} \cdot \bm{N}dS \; ,$$
%
if the limit exists. \pause The limit does exist when $X$ is reasonably smooth.

\pause We have seen similar limits when we talked about average values:
%
$$f(P) = \lim_{D\to\{p\}} \frac{1}{\text{vol}(D)} \int\!\!\!\!\int\!\!\!\!\int_D f(Q)\, dV\; .$$
%
\pause However:
\begin{itemize}
  \item the definition of $\divg \textbf{X}$ involves a surface integral
  \item the definition of average involves a triple integral
\end{itemize}

\pause We should somehow transform the surface integral into a triple integral.
\end{frame}

\begin{frame}
  \frametitle{Divergence Theorem}

\begin{theorem}{\rm
  Let $D$ be a compact set in space with boundary $S$ a piecewise smooth parametrized surface, oriented by the outward normal, and let
%
$$\textbf{X}(x,y,z) = P(x,y,z) \, \textbf{i} + Q(x,y,z) \, \textbf{j} + R(x,y,z)\, \textbf{k}$$
%
be a smooth vector field defined on $D$. Then
%
$$\iint_S \bm{X} \cdot \bm{dS} = \int\!\!\!\!\!\int\!\!\!\!\!\int_D \left(\frac{\partial P}{\partial x}+ \frac{\partial Q}{\partial y} + \frac{\partial R}{\partial z} \right) \, dV$$
}\end{theorem}
%
\pause \underline{Consequence}:
%
\begin{align*}
  (\divg \textbf{X}) (p) = &
\lim_{D\to \{P\}} \frac{1}{\text{vol}(D)}\int\!\!\!\!\!\int_S \textbf{X} \cdot \textbf{dS}  = \\
= & \lim_{D\to \{P\}} \frac{1}{\text{vol}(D)}\int\!\!\!\!\!\int\!\!\!\!\!\int_D \left(\frac{\partial P}{\partial x}+ \frac{\partial Q}{\partial y} + \frac{\partial R}{\partial z} \right) \, dV =
  \frac{\partial P}{\partial x}+ \frac{\partial Q}{\partial y} + \frac{\partial R}{\partial z}\; .
\end{align*}


\end{frame}

\begin{frame}
  \frametitle{Divergence Theorem}

%
If $\textbf{X} = P\, \textbf{i} + Q\, \textbf{j} +R\, \textbf{k}$, then
%
$$\divg \textbf{X} = \frac{\partial P}{\partial x}+ \frac{\partial Q}{\partial y} + \frac{\partial R}{\partial z}  = ``\langle \partial_x, \partial_y, \partial_z \rangle \cdot \langle P, Q, R\rangle" =  ``\nabla \cdot \textbf{X}"\; .$$

Intuitive notation:
%
$$\divg \textbf{X} = \nabla \cdot \textbf{X}\, .$$

\pause Confirmed by the example of $\textbf{X} = ax \, \textbf{i} + by\, \textbf{j} + cz\, \textbf{k}$ $\Longrightarrow$ $\divg \textbf{X} = a+b+c$.\pause

\begin{theorem}
  %
$$\iint_S \bm{X} \cdot \bm{dS} = \int\!\!\!\!\!\int\!\!\!\!\!\int_D \divg \textbf{X} \, dV$$
\end{theorem}

\begin{itemize}
  \item \pause If $(\divg \textbf{X})(p)>0$, then \pause $p$ acts as a source;
  \item \pause If $(\divg \textbf{X})(p)<0$, then \pause $p$ acts as a sink;
  \item \pause If $\divg \textbf{X} \equiv 0$ on some domain $D$, then \pause $\textbf{X}$ is incompressible on $D$.
\end{itemize}

\end{frame}
