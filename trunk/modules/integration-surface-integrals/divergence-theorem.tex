
%commented out: this spells out the obvious:
% we already said we have two competing definitions for divergence
%that are equivalent by a theorem
%\begin{frame}
%\frametitle{Divergence}
%\begin{itemize}
%\item Recall that the divergence of $\fcv{X}$ at $p$ is the density of flux: $(\divg \fcv{X})(p) = \lim_{D\to \{p\}} \frac{1}{\text{vol}(D)} \iint_S \fcv{X} \cdot \fcv{N}\diff S \; ,$ if the limit exists. 
%\item<2->  The limit does exist when $X$ is reasonably smooth.
%\item<3-> We have seen similar limits when we talked about average values: $f(P) = \lim_{D\to\{p\}} \frac{1}{\text{vol}(D)} \iiint_D f(Q)\, \diff V\; .$
%\item<4-> The definition of $\divg \fcv{X}$ involves a surface integral and the definition of average involves a triple integral.
%\item<5-> The next theorem establishes a connection between surface integrals and triple integrals.
%\end{itemize}
%\end{frame}

\begin{frame}
%\frametitle{Divergence Theorem}

\begin{theorem}[Divergence Theorem]
Let $D$ be a compact set in space with boundary $S$ a piecewise smooth parametrized surface, oriented by the outward normal, and let $\fcv{X}$ be a smooth vector field on $D$ given by

\hfil $
\fcv{X}(x,y,z)= P (x,y,z) \fcv{i} + Q(x,y,z) \, \fcv{j} + R(x,y,z)\, \fcv{k}\quad .
$

Then
\[
\iint_S \fcv{X} \cdot \fcv{\diff S} = \iiint_D \left(\frac{\partial P}{\partial x}+ \frac{\partial Q}{\partial y} + \frac{\partial R}{\partial z} \right) \, \diff V
\]
\end{theorem}
\begin{corollary}[May serve as alternative definition of $\divg$]
$
\begin{array}{@{\!\!}r@{}c@{}l}
  (\divg \fcv{X}) (p) &= &\displaystyle
\lim\limits_{D\to \{p\}} \frac{1}{\text{vol}(D)}\iint_S \fcv{X} \cdot \fcv{\diff S}   \\
&= &\displaystyle \lim\limits_{D\to \{p\}} \frac{1}{\text{vol}(D)}\iiint_D \!\! \left(\frac{\partial P}{\partial x}+ \frac{\partial Q}{\partial y} + \frac{\partial R}{\partial z} \right)  \diff V \\
&=&\displaystyle
  \frac{\partial P}{\partial x}+ \frac{\partial Q}{\partial y} + \frac{\partial R}{\partial z}.
\end{array}
$
\end{corollary}
\end{frame}

\begin{frame}
\frametitle{Divergence Theorem}
\begin{itemize}
\item Let  $\fcv{X} = P\, \fcv{i} + Q\, \fcv{j} +R\, \fcv{k}$.
\item Recall our notation
\[
\begin{array}{rcl}
\displaystyle \divg \fcv{X} &=&\displaystyle \frac{\partial P}{\partial x}+ \frac{\partial Q}{\partial y} + \frac{\partial R}{\partial z}  = \left( \partial_x, \partial_y, \partial_z \right) \cdot \left( P, Q, R\right) \\
\displaystyle \divg \fcv X&=&\displaystyle \nabla \cdot \fcv{X} .
\end{array}
\]

\end{itemize}

\begin{theorem}[Divergence Theorem]
$\displaystyle\iint_S \fcv{X} \cdot \fcv{\diff S} = \iiint_D \divg \fcv{X} \, \diff V$
\end{theorem}

\begin{itemize}
\item If $(\divg \fcv{X})(p)>0$, then $p$ acts as a source;
\item If $(\divg \fcv{X})(p)<0$, then $p$ acts as a sink;
\item If $\divg \fcv{X} \equiv 0$ on some domain $D$, then $\fcv{X}$ is incompressible on $D$.
\end{itemize}

\end{frame}
