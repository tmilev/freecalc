\begin{frame}
\begin{example}[Orientation of the equator of a sphere]
\begin{columns}
\column{0.25\textwidth}
\begin{pspicture}(-1,-1)(1,1)
\tiny%
\fcBoundingBox{-1.3}{-1.21}{1.7}{1.1}%
\fcStartIIIdScene%
\only<1-12>{%
\fcSurfaceInScene[colorUV=pink]{0}{0}{90}{360}{%
4 dict begin%
/rho 1 def%
/x u sin v cos mul rho mul def%
/y u sin v sin mul rho mul def%
/z u cos rho mul def%
[x y z]%
end%
}{}%
}%
\only<13->{%
\fcSurfaceInScene[colorUV=pink]{90}{0}{180}{360}{%
4 dict begin%
/rho 1 def%
/x u sin v cos mul rho mul def%
/y u sin v sin mul rho mul def%
/z u cos rho mul def%
[x y z]%
end%
}{}%
}%
\only<11>{%
\fcCurveIIIdInScene[arrows=->, linecolor=red, linewidth=2]{0}{90}{[t cos t sin 0]}%
\fcCurveIIIdInScene[arrows=->, linecolor=red, linewidth=2]{90}{180}{[t cos t sin 0]}%
\fcCurveIIIdInScene[arrows=->, linecolor=red, linewidth=2]{180}{270}{[t cos t sin 0]}%
\fcCurveIIIdInScene[arrows=->, linecolor=red, linewidth=2]{270}{360}{[t cos t sin 0]}%
}%
\only<12>{%
\fcCurveIIIdInScene[arrows=->, linecolor=blue, linewidth=1]{0}{90}{[t cos t sin 0]}%
\fcCurveIIIdInScene[arrows=->, linecolor=blue]{90}{180}{[t cos t sin 0]}%
\fcCurveIIIdInScene[arrows=->, linecolor=blue]{180}{270}{[t cos t sin 0]}%
\fcCurveIIIdInScene[arrows=->, linecolor=blue]{270}{360}{[t cos t sin 0]}%
}%
\only<13->{%
\fcCurveIIIdInScene[arrows=->, linecolor=blue, linewidth=1]{90}{0}{[t cos t sin 0]}%
\fcCurveIIIdInScene[arrows=->, linecolor=blue]{180}{90}{[t cos t sin 0]}%
\fcCurveIIIdInScene[arrows=->, linecolor=blue]{270}{180}{[t cos t sin 0]}%
\fcCurveIIIdInScene[arrows=->, linecolor=blue]{360}{270}{[t cos t sin 0]}%
}%
\fcAxesIIIdInScene[linecolor=black, linewidth=1, arrows=->]{2}{2}{1.2}%
\fcFinishIIIdScene[fastsort=true]%
\uncover<3->{\fcLineIIId[arrows=->, linewidth=2pt, linecolor=blue]{[1 0 0]}{[2 0 0]}%
\fcPutIIId[b]{[1.8 0 0.1]}{\alert<9,10>{$\fcv n$}}%
}%
\uncover<3,9,11,12,14 >{\fcLineIIId[arrows=->, linewidth=3pt, linecolor=red]{[1 0 0]}{[2 0 0]}}%
\uncover<4-12>{\fcLineIIId[arrows=->, linewidth=2pt, linecolor=cyan]{[1 0 0]}{[1 1 0]}}%
\uncover<13->{\fcLineIIId[arrows=->, linewidth=2pt, linecolor=gray]{[1 0 0]}{[1 1 0]}}%
\uncover<10,11,12>{\fcLineIIId[arrows=->, linewidth=3pt, linecolor=red]{[1 0 0]}{[1 1 0]}}%
\uncover<4-9,13->{\fcLineIIId[arrows=->, linewidth=2pt, linecolor=cyan]{[1 0 0]}{[1 -1 0]}}%
\uncover<14>{\fcLineIIId[arrows=->, linewidth=3pt, linecolor=red]{[1 0 0]}{[1 -1 0]}}%
\uncover<10-12>{\fcLineIIId[arrows=->, linewidth=2pt, linecolor=gray]{[1 0 0]}{[1 -1 0]}}%
\uncover<2->{\fcDotIIId{[1 0 0]}}%
\uncover<5->{\fcLineIIId[arrows=->, linewidth=2pt, linecolor=blue!50]{[1 0 0]}{[1 0 -1]}}%
\uncover<7,9>{\fcLineIIId[arrows=->, linewidth=3pt, linecolor=red]{[1 0 0]}{[1 0 -1]}}%
\uncover<5-6>{\fcLineIIId[arrows=->, linewidth=2pt, linecolor=blue!50]{[1 0 0]}{[1 0 1]}}%
\uncover<7->{\fcLineIIId[arrows=->, linewidth=2pt, linecolor=gray]{[1 0 0]}{[1 0 1]}}%
\uncover<7->{\fcPutIIId{[1 0.3 -1]}{$\alert<7,9,10>{\fcv N}$}}%
\end{pspicture}

\column{0.75\textwidth}
Let $S$ be the unit sphere $x^2+y^2+z^2 = 1$ oriented by the outward normal $\fcv{n}$, $D = S \cap \{z\geq 0\}$ be the upper hemisphere. Introduce an orientation on the boundary $C=\partial D$.

\end{columns}
\begin{itemize}
\item<2-> \fcQuestion{2}{At the point $(1,0,0)$ the normal to the surface $\fcv{n}$ equals} \fcAnswer{3}{$\fcv{i}$.}
\item<4-> Let $\fcv T$ be a unit tangent to $C$ at $(1,0,0)$; then \alert<4>{$\fcv T=\fcv j$ or $-\fcv j$.}
\item<5-> Let $\fcv{N}$ be unit vector perpendicular to $\fcv n$ and $\fcv T$, \fcQuestion{6}{pointing towards $D$} \uncover<6->{$\Rightarrow$} \fcQuestion{6}{$\fcv N$ equals} \fcAnswer{7}{$-\fcv k$.} 
\item<8-> $\Rightarrow$ positively oriented tangent to $C$ is $\fcv{T}= \alert<9>{ \fcv{n} \times \fcv{N}} \fcQuestion{9}{= \fcv{i} \times (-\fcv{k})=} \fcAnswer{10}{ \fcv{j}.}$
\item<12-> A viewer, standing along $\fcv n$ with feet on surface, and facing in the direction of the tangent, has the upper hemisphere on the left side.
\item<13-> Change $D$ to be lower hemisphere: we get $\fcv{N}=\fcv{k}$, $\fcv{T}=\fcv{n}\times \fcv{N} = -\fcv{j}$.
\item<14-> A viewer, standing along $\fcv n$ with feet on surface, and facing in the direction of the tangent, has the lower hemisphere again to the left.
\end{itemize}
\end{example}
\end{frame}