\begin{frame}
\begin{example}[Orientation of the equator of a sphere]
\begin{columns}
\column{0.4\textwidth}
\begin{pspicture}(-1,-1)(1,1)
\tiny
\fcBoundingBox{-1.3}{-0.5}{1.3}{1.4}
\fcStartIIIdScene
\fcSurfaceInScene{90}{0}{0}{360}{
4 dict begin
/rho 1 def
/x u cos v cos mul rho mul def
/y u cos v sin mul rho mul def
/z u sin rho mul def
[x y z]
end
}{}
\fcAxesIIIdInScene{1.4}{1.4}{1.4}
\fcFinishIIIdScene[fastsort=true]
\end{pspicture}

\column{0.6\textwidth}
Let $S$ be the unit sphere $x^2+y^2+z^2 = 1$ oriented by the outward normal $\fcv{n}$, $D = S \cap \{z\geq 0\}$ be the upper hemisphere. Introduce an orientation on the boundary of $D$ is the circle $C= S \cap \{z=0\}$.

\end{columns}


\pause At the point $p=(1,0,0)$
\begin{itemize}
  \item the normal $\fcv{n}$ is the vector \pause $\fcv{i}$;
  \item the normal $\fcv{N}$ is the vector \pause $-\fcv{k}$;
  \item positively oriented tangent to $C$:\pause
%
$$\fcv{T}= \fcv{n} \times \fcv{N} = \fcv{i} \times (-\fcv{k}) = \fcv{k}\times \fcv{i} = \fcv{j} \;,$$
%
\item \pause which orients $C$ counterclockwise in the $\{\fcv{i},\fcv{j}\}-$plane.
\end{itemize}

\pause If instead we use the lower hemisphere to orient $C$: at $p=(1,0,0)$
 \begin{itemize}
   \item $\fcv{n}=\fcv{i}$, $\fcv{N}=\fcv{k}$, $\fcv{T}=\fcv{n}\times \fcv{N} = \fcv{i} \times \fcv{k} = -\fcv{j}$;
   \item which orients $C$ clockwise in the $\{\fcv{i},\fcv{j}\}-$plane.
 \end{itemize}
\end{example}
\end{frame}