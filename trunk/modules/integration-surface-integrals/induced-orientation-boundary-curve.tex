\begin{frame}
\frametitle{Induced Orientation on a Boundary Curve}
\begin{itemize}
\item Let $S$ be smooth surface, oriented by unit normal vector $\fcv{n}$.
\item Let $D$ be region in $S$, bounded by a curve $C=\partial D$.
\item Let $\fcv{N}$ denote the unit vector field on $C$ which is
\begin{itemize}
\item tangent to $S$;
\item normal to $C$;
\item pointing outward of $D$.
\end{itemize}
\item Let $\fcv{T}$ be unit tangent vector to $C$ (and hence tangent to $S$).
\item Then $\fcv{N}$ orients the tangents of $C$ and thus $C$ itself.
\begin{definition}
We say that $\fcv{T}$ is \emph{positively oriented} if the triple $\{\fcv{n}, \fcv{N}, \fcv{T}\}$ is positively oriented in space.
\end{definition}
\item Since $\fcv T, \fcv n, \fcv N$ are pairwise orthogonal unit vectors, positive orientation is equivalent to $\fcv{T}=\fcv{n}\times \fcv{N}$.
\item If we view the plane tangent to $S$ from the tip of $\fcv n$, then  $\{\fcv{N}, \fcv{T}\}$ is positively oriented in that plane.
\end{itemize}
\end{frame}