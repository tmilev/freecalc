\begin{frame}
\begin{example}
\begin{columns}
\column{0.3\textwidth}
\begin{pspicture}(-1,-1)(1,1)
\tiny
\fcBoundingBox{-1.3}{-0.5}{1.3}{1.4}
\fcStartIIIdScene
\fcSurfaceInScene{90}{0}{0}{360}{
4 dict begin
/rho 1 def
/x u cos v cos mul rho mul def
/y u cos v sin mul rho mul def
/z u sin rho mul def
[x y z]
end
}{}
\fcAxesIIIdInScene{1.4}{1.4}{1.4}
\fcFinishIIIdScene[fastsort=true]
\end{pspicture}
\column{0.7\textwidth}
Find the centroid of a hemisphere $S$ of radius $R$. \uncover<2->{ By definition, the centroid of surface is $\frac{1}{\text{Area(S)}}\iint_{S} \fcv f \diff S $, where $\fcv f(u,v)$ is the position vector of $S$.}
\end{columns}
\uncover<3->{
By symmetry, the centroid is on the $z$-axis; let its $z$-coordinate be $h$. \uncover<4->{Parametrize 
$S:\left|\begin{array}{r@{~}c@{~}l}
\fcv f(x,y)&=&\left(x,y,\sqrt{R^2-x^2-y^2}\right)\\
(x,y)&\in& D =\text{disk radius }R
\end{array}\right.$. }
\[
\begin{array}{r@{~}c@{~}l}
\displaystyle \uncover<5->{\diff S&=&\displaystyle |\fcv f_x \times \fcv f_y|  \diff x  \diff y = \sqrt{1+z_x^2+z_y^2} \diff x \diff y = \frac{R}{\sqrt{R^2-x^2-y^2}}  \diff x \diff y}\\
\displaystyle h &=&\displaystyle \frac{1}{\text{area}(S)}\, \iint_S z ~ \diff S =\displaystyle \frac{1}{2\pi R^2}  \iint_{D}  \frac{R\sqrt{R^2-x^2-y^2}}{\sqrt{R^2-x^2-y^2}} \diff x \diff y \\&=&\displaystyle \frac{1}{2\pi R} \iint_D  \diff x \diff y = \frac{\pi R^2}{2\pi R}=\frac{R}{2}\quad .
\end{array}
\]

}
\end{example}
\vskip 10cm

\end{frame}

\begin{frame}
\begin{example}
\begin{columns}
\column{0.3\textwidth}
\begin{pspicture}(-1,-1)(1,1)
\tiny
\fcBoundingBox{-1.3}{-0.5}{1.3}{1.4}
\fcStartIIIdScene
\fcSurfaceInScene{90}{0}{0}{360}{
4 dict begin
/rho 1 def
/x u cos v cos mul rho mul def
/y u cos v sin mul rho mul def
/z u sin rho mul def
[x y z]
end
}{}
\fcAxesIIIdInScene{1.4}{1.4}{1.4}
\fcFinishIIIdScene[fastsort=true]
\end{pspicture}
\column{0.7\textwidth}
Find the centroid of a hemisphere $S$ of radius $R$. By definition, the centroid of surface is $\frac{1}{\text{Area(S)}}\iint_{S} \fcv f \diff S $, where $\fcv f(u,v)$ is the position vector of $S$.
\end{columns}

The hemisphere can be also obtained by revolving the quarter of circle $(x,z) = (f(u), g(u)) = (R\cos{u}, R\sin{u})$, $0 \leq u \leq \frac{\pi}{2}$ about the $z-$axis.  

\hfil $\displaystyle S: (R\cos{u}\cos{v}, R\cos{u}\sin{v}, R\sin{u}), u\in \left[0,\frac{\pi}{2}\right], v\in \left[0,2\pi\right].$

\vskip -0.2cm
\[
\begin{array}{rcl}
\diff S &=& |f|\sqrt{|f'|^2+|g'|^2} = R^2\cos{u}\;\\
\iint_{S} z\diff S& = & \iint_{D} R\sin{u} R^2\cos{u}\diff u \diff v \\
&=&  R^3 \left(\int\limits_{u=0}^{u=\frac{\pi}{2}} \sin{u}\cos{u}\, \diff u\right) \left( \int\limits_{v=0}^{v=2\pi}  \diff v \right)  \\
& =& 2\pi R^3 \left. \frac{\sin^2{u}}{2} \right|_{u=0}^{u=\frac{\pi}{2}} =\pi R^3\; .\\
\text{centroid }z-\text{coord}&=& \frac{1}{\text{area}(S)}\, \iint_S z \, \diff S = \frac{1}{2\pi R^2} \, \pi R^3 = \frac{R}{2}\;  
\end{array}
\]
\end{example}


\vskip 10cm
\end{frame}