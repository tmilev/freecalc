\begin{frame}
  \frametitle{Vector-valued Integrals}

\begin{itemize}
  \item \pause If the density of force acting on a surface is variable, then the total force is expressed as a surface integral of a vector quantity.
\end{itemize}

\begin{itemize}
  \item \pause $S$: the surface of an inflated balloon.
  \item \pause pressure inside the balloon is $p_1$ and the the pressure outside is $p_2 < p_1$;
  \item \pause $Q$: point on the surface of the balloon;
  \item On an infinitesimal region around $Q$:
  \begin{itemize}
    \item \pause difference in pressure $p = p_1-p_2$ determines a force:
        \begin{itemize}
          \item \pause magnitude $p\, dS$;
          \item \pause direction $\textbf{N}$: unit normal to $S$ at $Q$, pointing outward.
        \end{itemize}
    \item \pause infinitesimal force $d\textbf{F} = p \textbf{N} \, dS$
  \end{itemize}
  \item \pause total force
%
$$\bm{F} =  \iint_S \, d\textbf{F} = \iint_S p \textbf{N} \, dS = \iint_S p\, \bm{dS} \; .$$
\end{itemize}
\end{frame}