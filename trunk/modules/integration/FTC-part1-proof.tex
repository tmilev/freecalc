% begin module FTC1-proof
\begin{frame}
\begin{theorem}
Let $f$ be a continuous function on $[a,b]$. Then $f$ is integrable over $[a,b]$.
\end{theorem}
\begin{itemize}
\item The proof of this theorem is not difficult, but is outside of the scope of Calculus I and II.
\item The only ``difficulty'' in the proof stems from the fact that we have not rigorously constructed the real numbers. 
\item We already (silently) assumed a construction of the real numbers when we defined limits. 
\item Such a construction is also (silently) assumed in most regular high school mathematics courses.
\item  The proof of the Theorem uses the fact that every set bounded above has a least upper bound. The latter fact is either taken as an axiom of the real numbers, or is proven from equivalent set of axioms. This fact is the only one a calculus student is missing to prove the above theorem.
\item A student interested in a proof of the theorem should google ``Darboux integral''.
\end{itemize}

\end{frame}

\begin{frame}
\begin{theorem}[The Fundamental Theorem of Calculus]
Let $f$ be a function continuous on $[a, b]$ and let $G(x) = \displaystyle \int_a^x f(t) \diff t$ for all $x\in [a,b]$. Then $G$ is differentiable and $G'(x) = f(x)$.  
\end{theorem}
\begin{proof}[\only<1-2>{Informal} Proof]
\begin{columns}
\column{0.3\textwidth}
\psset{xunit=1cm, yunit=1cm}
\begin{pspicture}(-0.5, -5)(2.5,5) 
\psframe*[linecolor=white](-0.5,-5)(5,5) 
\tiny 
%\psLabels{3}{2.5}

\pscustom*[linecolor=cyan]{
\psplot[linecolor=\psColorGraph, plotpoints=1000]{0.5}{2}{1 -0.5 x add 3 exp 0.333333 mul add -0.5 x add 2 exp -0.5 mul add }
\psline(2,0)(0.5,0)(0.5,0.833333)
}
\pscustom*[linecolor=blue]{
\psplot[linecolor=\psColorGraph, plotpoints=1000]{2}{2.2}{1 -0.5 x add 3 exp 0.333333 mul add -0.5 x add 2 exp -0.5 mul add }
\psline(2.2,1.192667)(2.2,0)(2,0)
}
\psaxes[ticks=none, labels=none]{<->}(0,0)(-0.5,-0.5)(3,2.5)

%Function formula: -1/2 (x-1/2)^{2}+1/3 (x-1/2)^{3}+1 
\psplot[linecolor=\psColorGraph, plotpoints=1000]{0.5}{2.5}{1 -0.5 x add 3 exp 0.333333 mul add -0.5 x add 2 exp -0.5 mul add }
\rput[b](1.3, 1.2){$y=f(x)$}
\psline(2,0)(2.2,0)(2.2,1.192667)(2,1.192667)(2,0)
\psXTickWithLabel{0.5}{$a$}
\psXTickWithLabel{2.5}{$b$}
\end{pspicture}
\column{0.7\textwidth}
\[
\begin{array}{rcl}
G'(x)&=&\lim\limits_{h\to 0} \frac{G(x+h)-G(x)}h\\
&=&\frac{\int_{a}^{x+h} f(x)dx-\int_{a}^{x}f(x)dx}h\\
&=&\lim_{n\to \infty} \sum_{i=1}^n f(x_i^*)\Delta x
\end{array}
\]

Let $\varepsilon>0$ be an (arbitrarily small) positive number. Then there exists $\delta_1>0$ such that $|f(x+h)-f(x)|<\frac{\varepsilon}2$ for all $|h|<\delta_1$.

By Theorem from preceding slide, $f$ is continuous and therefore integrable. Therefore the Riemann sum limit exists and is independent of the sampling point. Therefore there exists $\delta_2>0$ such that $\left|\int_{a}^x f(x)dx-\sum_{i=1}^n f(x_i^*)\Delta x \right| <\frac\varepsilon 2$.

Therefore 
\[G(x+h)
\]

\uncover<3->{}

\end{columns}
\end{proof} 
\end{frame}
% end module FTC1-proof
