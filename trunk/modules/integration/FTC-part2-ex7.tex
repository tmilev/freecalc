% begin module FTC-part2-ex7
\begin{frame}
\begin{example} %[Example 2, p. 357]
\begin{columns}
\column{0.4\textwidth}
\psset{xunit=0.7cm, yunit=0.7cm}
\begin{pspicture}(-1.000000, -1.3)(6.2,1.3) 
\tiny 
\psframe*[linecolor=white](-1.000000,-1.1)(6.2,1.2) 
\pscustom*[linecolor=cyan]{ %Function formula: \cos{}x 
\psplot[linecolor=\psColorGraph, plotpoints=1000]{0}{1}{x 57.29578 mul cos }\psline(1.000000, 0)(0.000000, 0)}
%Function formula: \cos{}x 
\psplot[linecolor=\psColorGraph, plotpoints=1000]{0}{6}{x 57.29578 mul cos }
\psaxes[arrows=<->, ticks=none, labels=none](0,0)(-0.6,-1.1)(6,1.1)
\psLabels{6}{1.1}
\psXTickWithLabel{1}{$b$}
\end{pspicture} 
\column{0.6\textwidth}
Find the area under the cosine curve from $0$ to $b$, where $0 \leq b \leq \frac{\pi}{2}$.
\end{columns}
\begin{itemize}
\item<2->  $\cos x$ is continuous on $[0, \frac{\pi}{2}]$ (in fact, it's continuous everywhere).
\item<3-| alert@3-4>  An antiderivative is \uncover<4->{$\sin x$.}
\[
\uncover<5->{%
\int_{\alert<handout:0| 5>{0}}^{\alert<handout:0| 5>{b}} \cos x \ \diff x = \left[ \sin x \right]_{\alert<handout:0| 5>{0}}^{\alert<handout:0| 5>{b}} %
}%
\uncover<6->{%
 = \sin (b) - \sin (0)%
}%
\uncover<7->{%
 = \sin b%
}%
\]
\end{itemize}
\end{example}
\end{frame}
% end module FTC-part2-ex7
