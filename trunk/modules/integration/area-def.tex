% begin module area-def
\begin{frame}
\begin{definition}[Area Under a Curve]
The area of the region $S$ that lies under the curve $y = f(x)$ is the limit of the sum of the areas of the approximating rectangles:
\abovedisplayskip=0pt
\belowdisplayskip=0pt
\[
A = \lim_{n\to\infty} R_n = \lim_{n\to\infty} [ f(x_1)\Delta x + f(x_2) \Delta x + \cdots + f(x_n) \Delta x]
\]
\end{definition}
\begin{itemize}
\item<2->  This limit always exists if $f$ is continuous.
\item<3->  If $f$ is continuous, we get the same limit if we use left endpoints:
\abovedisplayskip=0pt
\belowdisplayskip=0pt
\[
A = \lim_{n\to\infty} L_n = \lim_{n\to\infty} [ f(x_0)\Delta x + f(x_1) \Delta x + \cdots + f(x_{n-1}) \Delta x]
\]
\item<4->  If $f$ is continuous, we get the same limit if we use any number $x_i^*$ in the interval $[x_{i-1},x_i]$.  $x_i^*$ is called a sample point.
\abovedisplayskip=0pt
\belowdisplayskip=0pt
\[
A = \lim_{n\to\infty} [ f(x_1^*)\Delta x + f(x_2^*) \Delta x + \cdots + f(x_{n}^*) \Delta x]
\]
\end{itemize}
\uncover<5->{%
\begin{definition}[Riemann Sum]
A Riemann sum is any sum of the form
\abovedisplayskip=0pt
\belowdisplayskip=0pt
\[
f(x_1^*)\Delta x + f(x_2^*) \Delta x + \cdots + f(x_{n}^*) \Delta x.
\]
\end{definition}
}%
\end{frame}
% end module area-def
