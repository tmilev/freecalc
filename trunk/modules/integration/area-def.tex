% begin module area-def
\begin{frame}
Estimate the area under the curve $y = f(x)$ between $a$ and $b$.
\begin{columns}
\column{.55\textwidth}
\only<handout:1| -1>{%
\includegraphics[height=3.5cm]{integration/pictures/05-01-rectangles-right.pdf}%
}%
\only<handout:2| 2->{%
\includegraphics[height=3.5cm]{integration/pictures/05-01-rectangles-left.pdf}%
}%

\begin{itemize}
\item<1->  The width of the interval is $b-a$.
\item<1->  The width of each of the $n$ strips is $\Delta x = \frac{b-a}{n}$.
\item<1->  The strips divide $[a,b]$ into $n$ subintervals:\\ $[x_0, x_1]$, $[x_1,x_2]$, $\ldots ,$ $[x_{n-1},x_n]$,\\ where $x_0 = a$ and $x_n = b$.
\end{itemize}
\column{.45\textwidth}
\begin{itemize}
\item<1->  The \only<handout:1| -1>{right}\only<handout:2| 2->{\alert<handout:2| 2>{left}} endpoints of the subintervals are
\end{itemize}
\abovedisplayskip=0pt
\belowdisplayskip=0pt
\begin{eqnarray*}
x_{\only<handout:1| -1>{1}\only<handout:2| 2->{\alert<handout:2| 2>{0}}} & = & a \only<handout:1| -1>{+ \Delta x}\\
x_{\only<handout:1| -1>{2}\only<handout:2| 2->{\alert<handout:2| 2>{1}}} & = & a + \only<handout:1| -1>{2}\Delta x\\
x_{\only<handout:1| -1>{3}\only<handout:2| 2->{\alert<handout:2| 2>{2}}} & = & a + \only<handout:1| -1>{3}\only<handout:2| 2->{2}\Delta x\\
& \vdots & 
\end{eqnarray*}
\begin{itemize}
\item<1->  The height of the $i$th rectangle is $f(x_{i\only<handout:2| 2->{\alert<handout:2| 2>{-1}}})$.
\item<1->  The area of the $i$th rectangle is $f(x_{i\only<handout:2| 2->{\alert<handout:2| 2>{-1}}})\Delta x$.
\end{itemize}
\end{columns}
\abovedisplayskip=0pt
\belowdisplayskip=0pt
\[
\only<handout:1| -1>{R_n}\only<handout:2| 2->{\alert<handout:2| 2>{L_n}} = f(x_{\only<handout:1| -1>{1}\only<handout:2| 2->{\alert<handout:2| 2>{0}}})\Delta x + f(x_{\only<handout:1| -1>{2}\only<handout:2| 2->{\alert<handout:2| 2>{1}}})\Delta x + f(x_{\only<handout:1| -1>{3}\only<handout:2| 2->{\alert<handout:2| 2>{2}}})\Delta x + \cdots + f(x_{n\only<handout:2| 2->{\alert<handout:2| 2>{-1}}})\Delta x
\]
\end{frame}

\begin{frame}
\begin{definition}[Area Under a Curve]
The area of the region $S$ that lies under the curve $y = f(x)$ is the limit of the sum of the areas of the approximating rectangles:
\abovedisplayskip=0pt
\belowdisplayskip=0pt
\[
A = \lim_{n\to\infty} R_n = \lim_{n\to\infty} [ f(x_1)\Delta x + f(x_2) \Delta x + \cdots + f(x_n) \Delta x]
\]
\end{definition}
\begin{itemize}
\item<2->  This limit always exists if $f$ is continuous.
\item<3->  We get the same answer if we use left endpoints:
\abovedisplayskip=0pt
\belowdisplayskip=0pt
\[
A = \lim_{n\to\infty} L_n = \lim_{n\to\infty} [ f(x_0)\Delta x + f(x_1) \Delta x + \cdots + f(x_{n-1}) \Delta x]
\]
\item<4->  We get the same answer if we use any number $x_i^*$ in the interval $[x_{i-1},x_i]$.  $x_i^*$ is called a sample point.
\abovedisplayskip=0pt
\belowdisplayskip=0pt
\[
A = \lim_{n\to\infty} [ f(x_1^*)\Delta x + f(x_2^*) \Delta x + \cdots + f(x_{n}^*) \Delta x]
\]
\end{itemize}
\uncover<5->{%
\begin{definition}[Riemann Sum]
A Riemann sum is any sum of the form
\abovedisplayskip=0pt
\belowdisplayskip=0pt
\[
f(x_1^*)\Delta x + f(x_2^*) \Delta x + \cdots + f(x_{n}^*) \Delta x.
\]
\end{definition}
}%
\end{frame}
% end module area-def
