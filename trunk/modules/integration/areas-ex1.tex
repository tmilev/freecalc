% begin module areas-ex1
\begin{frame}
\begin{example}[Example 1, p. 289]
Find the sum of the areas of the four approximating rectangles obtained using right endpoints.
\begin{columns}[c]
\column{.5\textwidth}
\begin{itemize}
\item<2->  Let $R_4$ denote the sum of the areas of the rectangles.
\item<3-| alert@3-4>  Each rectangle has width \uncover<4->{$\frac{1}{4}$.}
\item<5-| alert@5-6>  The heights are \uncover<6->{$\left( \frac{1}{4}\right)^2, \left( \frac{1}{2}\right)^2, \left(\frac{3}{4}\right)^2$, and $1^2$.}
\item<handout:2-| 9->  A similar calculation works for $L_4$, the sum of the areas of the left endpoint rectangles.
\end{itemize}
\column{.5\textwidth}
\only<handout:1| -8>{%
\includegraphics[height=4.5cm]{integration/pictures/05-01-rightb.pdf}%
}%
\only<handout:2| 9->{%
\includegraphics[height=4.5cm]{integration/pictures/05-01-leftb.pdf}%
}%
\end{columns}
\uncover<7->{%
\abovedisplayskip=0pt
\belowdisplayskip=0pt
\[
R_4 = \frac{1}{4}\cdot \left(\frac{1}{4}\right)^2%
 + \frac{1}{4}\cdot \left(\frac{1}{2}\right)^2%
 + \frac{1}{4}\cdot \left(\frac{3}{4}\right)^2%
 + \frac{1}{4}\cdot \left( 1\right)^2%
\uncover<8->{ = \frac{15}{32} = 0.46875}%
\]
}%
\uncover<handout:2-| 9->{%
\abovedisplayskip=0pt
\belowdisplayskip=0pt
\[
L_4 = \frac{1}{4}\cdot \left( 0\right)^2%
 + \frac{1}{4}\cdot \left(\frac{1}{4}\right)^2%
 + \frac{1}{4}\cdot \left(\frac{1}{2}\right)^2%
 + \frac{1}{4}\cdot \left( \frac{3}{4}\right)^2%
\uncover<8->{ = \frac{7}{32} = 0.21875}%
\]
}%
\vspace{-.1in}
\end{example}
\end{frame}
% end module areas-ex1
