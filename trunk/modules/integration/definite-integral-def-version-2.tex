% begin module definite-integral-def
\begin{frame}\frametitle{ %(5.2) 
The Definite Integral}
\begin{definition}[Definite Integral]
\begin{itemize}
\item  Let $f:[a,b]\to {\mathbb{R}}$.
\item  Divide $[a,b]$ into $n$ subintervals of equal width $\Delta x = (b-a)/n$.
\item  Let $a=x_0$, $x_1,\ldots ,$ $x_n=b $ be the endpoints of the subintervals, so $ x_i=a+i\Delta x $ .
\item  Let $x_1^*, x_2^*, \ldots , x_n^*$ be any sample points, $x_i^*\in [x_{i-1},x_i]$.  
\end{itemize}
\pause 
\abovedisplayskip=0pt
\belowdisplayskip=0pt
If the limit \[\lim\limits_{n\to \infty} \sum\limits_{i=1}^n f(x_i^*)\Delta x\] 
exists and is independent of the choice of sample points $x_i^*$, then we say $ f $ is \textbf{integrable} on [a,b], and we call this limit \textbf{the integral} of $f$ over $[a,b]$ and write
\[
\int_a^b f(x) \diff x = \lim_{n\to \infty} \sum_{i=1}^n f(x_i^*)\Delta x \quad .
\]
\end{definition}
\end{frame}

\begin{frame}
\begin{definition}[Definite Integral Cont'd: Sample points may be chosen in many ways]
%\[
%\int_a^b f(x) \diff x = \lim_{n\to \infty} \sum_{i=1}^n f(x_i^*)\Delta x \quad .
%\] \pause
Using the right hand approximation, $ x_i^*=x_i $, so we would have 
\[
\int_a^b{f(x)}  \diff x = \lim_{n\to \infty} R_n  = \lim_{n\to \infty} \sum_{i=1}^n f(x_i)\Delta x,
\]\pause
using the left hand approximation, $ x_i^*=x_{i-1} $, so we would have 
\[
\int_a^b{f(x)}  \diff x = \lim_{n\to \infty} L_n  = \lim_{n\to \infty} \sum_{i=1}^n f(x_{i-1})\Delta x,
\]
\pause
choosing $ x_i^*$ to be the midpoint of the interval $ =\overline{x}_i=\frac{1}{2}(x_{i-1}+x_i)$  we have
\[
\int_a^b f(x)\diff x =\lim_{n\to \infty}M_n=\lim_{n\to \infty} \sum_{i=1}^n f(\overline{x}_i)\Delta x
\]

\end{definition}
\end{frame}

\begin{frame}{Notation/Terminology of the definite integral}
\[
\mathop{\alert<handout:0| 2>{\int}}_{\alert<handout:0| 4>{a}}^{\alert<handout:0| 4>{b}} \alert<handout:0| 3>{f(x)} \alert<handout:0| 0>{\diff x} = \lim_{n\to \infty} \sum_{i=1}^n f(x_i^*)\Delta x,
\]
\begin{itemize}
\item<1-| alert@2>  $\int$ is called the integration sign.
\item<1-| alert@3>  $f(x)$ is called the integrand.
\item<1-| alert@4>  $a$ and $b$ are called the limits of integration.
\item<5->  The definite integral is a number.  It does not depend on $x$.  We could use any variable instead of $x$.
\end{itemize}
\uncover<5->{%
\[
\int_a^b f(x) \diff x = \int_a^b f(t)\diff t = \int_a^b f(r)\diff r = \int_a^b f(\theta )\diff \theta
\]
}%

\uncover<6->{\begin{definition}[Riemann Sum]
A Riemann sum is any sum of the form
\abovedisplayskip=0pt
\belowdisplayskip=0pt
\[
\sum_{i=1}^nf(x_i^*) \Delta x = f(x_1^*)\Delta x + f(x_2^*) \Delta x + \cdots + f(x_{n}^*) \Delta x.
\]
\end{definition}
}
\end{frame}
% end module definite-integral-def
