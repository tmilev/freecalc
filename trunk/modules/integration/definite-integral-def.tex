% begin module definite-integral-def
\begin{frame}\frametitle{ %(5.2) 
The Definite Integral}
\begin{definition}[Definite Integral]
\begin{itemize}
\item  Let $f$ be a function defined for $a\leq x\leq b$.
\item  Divide the interval $[a,b]$ into $n$ subintervals of equal width $\Delta x = (b-a)/n$ nd set $x_0=a$, $x_n=b$.
\item  Let $x_0$, $x_1,\ldots ,$ $x_n $ be the endpoints of the subintervals.
\item  Let $x_1^*, x_2^*, \ldots , x_n^*$ be any sample points in these subintervals; that is, $x_i^*$ is in $[x_{i-1},x_i]$.  
\end{itemize}
\abovedisplayskip=0pt
\belowdisplayskip=0pt
Suppose the limit $\lim\limits_{n\to \infty} \sum\limits_{i=1}^n f(x_i^*)\Delta x$ exists and is independent of the choice of sample points $x_i^*$. Then we call this limit the integral of $f$ over $[a,b]$ and write
\[
\int_a^b f(x) \diff x = \lim_{n\to \infty} \sum_{i=1}^n f(x_i^*)\Delta x \quad .
\]
\end{definition}
\end{frame}

\begin{frame}
\[
\mathop{\alert<handout:0| 2>{\int}}_{\alert<handout:0| 4>{a}}^{\alert<handout:0| 4>{b}} \alert<handout:0| 3>{f(x)} \alert<handout:0| 0>{\diff x} = \lim_{n\to \infty} \sum_{i=1}^n f(x_i^*)\Delta x,
\]
\begin{itemize}
\item<1-| alert@2>  $\int$ is called the integration sign.
\item<1-| alert@3>  $f(x)$ is called the integrand.
\item<1-| alert@4>  $a$ and $b$ are called the limits of integration.
\item<5->  The definite integral is a number.  It does not depend on $x$.  We could use any variable instead of $x$.
\end{itemize}
\uncover<5->{%
\[
\int_a^b f(x) \diff x = \int_a^b f(t)\diff t = \int_a^b f(r)\diff r = \int_a^b f(\theta )\diff \theta
\]
}%
\end{frame}
% end module definite-integral-def
