Estimate the integral using a Riemann sum using the indicated sample points and interval length.
\begin{enumerate}[ref={\fcProblemRef}]
% Riemann sums
\item \label{problemRiemannSum-sqrt(8x+1)} $\displaystyle \int_0^4 \left(\sqrt{8x+1}\right)\diff x$. Use four intervals of equal width, choose the sample point to be the left endpoint of each interval. 

\answer{ $\Delta x = 1$ and $f(x) = \sqrt{8x+1}$. Thus ${\displaystyle \int_0^4 f(x) \diff x \approx 9 + \sqrt{17}}$.}

\item $\displaystyle \int_0^6 \frac{1}{x^2+1} \diff x$. Use three intervals of equal width, choose the sample point to be the left endpoint. 

\answer{ $\Delta x = 2$ and $f(x) = \frac{1}{x^2+1}$. Thus ${\displaystyle \int_0^6 f(x) \diff x \approx \frac{214}{85}}$.}
\item \label{problemRiemannSum-1div1plusxsquared} $\displaystyle \int\limits_{-3.5}^{-0.5} \frac{\diff x}{x^2+1} $. Use three intervals of equal width, choose the sample point to be the midpoint of each interval. 

\answer{ $\Delta x = 1$ and $f(x) = \frac{1}{x^2+1}$. Thus $\displaystyle \int \limits_{-3.5}^{-0.5} f(x) \diff x  \approx \Delta x\left(f{} \left(-3 \right)+ f{}\left( -2\right)+f{}\left(-1\right)\right)=\frac{4}{5}=0.8$.}
\item $\displaystyle\int_{0}^2 \frac{\diff x}{1+x+x^3}$. Use $\Delta x=\frac{1}2 $ and right endpoint sampling points.

\answer{$ \frac{1}{2}\left(\frac{8}{13}+\frac{1}{3}+\frac{8}{47}+\frac{1}{11}\right)=\frac{12197}{20163}\approx 0.604920$}

\item $\displaystyle\int_{-2}^{0} \frac{\diff x}{1+x+x^2}$. Use $\Delta x=\frac23 $ and left endpoint sampling points.

\answer{$\frac23\left(\frac{1}{3}+\frac{9}{13}+\frac{9}{7}\right)=\frac{1262}{819}\approx 1.540904$}

\item $\displaystyle \int\limits_0^2 \frac{\diff x}{1+x^3}$. Use four intervals of equal width, choose the sample point to be the left endpoint of each interval. 

\answer{ $\Delta x = 0.5$ and $f(x) = \frac{1}{1+x^3}$. Thus $\displaystyle \int\limits_0^2 f(x) \diff x  \approx \Delta x\left(f{}\left(0\right)+f{}\left(1\right)+f{}\left(\frac{1}{2}\right)+f{}\left(\frac{3}{2}\right)\right)=\frac{1649}{1260}\approx 1.30873$.}

\item $\displaystyle \int\limits_{-2}^{0} \frac{\diff x}{x^4+1} $. Use four intervals of equal width, choose the sample point to be the right endpoint. 

\answer{ $\Delta x = 0.5$ and $f(x) = \frac{1}{1+x^3}$. Thus $\displaystyle \int\limits_0^2 f(x) \diff x  \approx \Delta x\left(f{}\left(-\frac{3}{2}\right)+f{}\left(-1\right)+f{}\left(-\frac{1}{2}\right)+f{}\left(0\right)\right)=\frac{8595}{6596}\approx 1.303062$.}

\item  \label{problemRiemannSum1/(3x^2+1)from-1to0with3intervalsLeftEndpt}
$\displaystyle\int_{-1}^0\frac{1}{3 {{x}}^{2}+1}\diff x
$. Use \textbf{$3$ intervals} of equal width, choose the sampling points to be the \textbf{left endpoints} of each interval. 
Simplify your answer to a rational number (single fraction of two integers).

\answer{ $\Delta x = \frac{1}{3}$ and $f(x) =\frac{1}{3 {{x}}^{2}+1}$. Thus $\displaystyle \int\limits_{-1}^0 f(x) \diff x$  is approximated by $\Delta x \left(f{}\left(-1\right)+f{}\left( -\frac{2}{3}\right)+f{}\left( -\frac{1}{3}\right)\right)=\frac{10}{21}$.}

\end{enumerate}


