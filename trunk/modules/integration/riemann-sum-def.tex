\begin{frame}
Estimate the area under $y = f(x)$ between $a$ and $b$ using $n$ strips.
\begin{columns}
\column{.55\textwidth}
\psset{xunit=3cm, yunit=3cm}
\begin{pspicture}(-0.15,-0.15)(1.75,0.95)
\psframe*[linecolor=white](-0.15,-0.15)(1.75,0.95)
\tiny
\newcommand{\functionPlotted}{
-27 16 div x 3 exp mul
729 160 div x 2 exp mul
-171 50 div x mul  
11 10 div
add add add
}
\uncover<handout:1|1>{
\psline*[linecolor=\fcColorAreaUnderGraph, linewidth=0.1pt](0.425, 0)(0.425, 0.339931)(0.3, 0.339931)(0.3, 0)(0.55, 0)(0.55, 0.316508)(0.425, 0.316508)(0.425, 0)(0.675, 0)(0.675, 0.348456)(0.55, 0.348456)(0.55, 0)(0.8, 0)(0.8, 0.416)(0.675, 0.416)(0.675, 0)(0.925, 0)(0.925, 0.499364)(0.8, 0.499364)(0.8, 0)(1.05, 0)(1.05, 0.578773)(0.925, 0.578773)(0.925, 0)(1.175, 0)(1.175, 0.634452)(1.05, 0.634452)(1.05, 0)(1.3, 0)(1.3, 0.646625)(1.175, 0.646625)(1.175, 0)(1.425, 0)(1.425, 0.595517)(1.3, 0.595517)(1.3, 0)(1.55, 0)(1.55, 0.461352)(1.425, 0.461352)(1.425, 0)
\psline[linecolor=blue, linewidth=0.1pt](0.425, 0)(0.425, 0.339931)(0.3, 0.339931)(0.3, 0)(0.55, 0)(0.55, 0.316508)(0.425, 0.316508)(0.425, 0)(0.675, 0)(0.675, 0.348456)(0.55, 0.348456)(0.55, 0)(0.8, 0)(0.8, 0.416)(0.675, 0.416)(0.675, 0)(0.925, 0)(0.925, 0.499364)(0.8, 0.499364)(0.8, 0)(1.05, 0)(1.05, 0.578773)(0.925, 0.578773)(0.925, 0)(1.175, 0)(1.175, 0.634452)(1.05, 0.634452)(1.05, 0)(1.3, 0)(1.3, 0.646625)(1.175, 0.646625)(1.175, 0)(1.425, 0)(1.425, 0.595517)(1.3, 0.595517)(1.3, 0)(1.55, 0)(1.55, 0.461352)(1.425, 0.461352)(1.425, 0)
}
\uncover<handout:2|2->{
\psline*[linecolor=\fcColorAreaUnderGraph, linewidth=0.1pt](0.3, 0)(0.3, 0.4385)(0.425, 0.4385)(0.425, 0)(0.425, 0)(0.425, 0.339931)(0.55, 0.339931)(0.55, 0)(0.55, 0)(0.55, 0.316508)(0.675, 0.316508)(0.675, 0)(0.675, 0)(0.675, 0.348456)(0.8, 0.348456)(0.8, 0)(0.8, 0)(0.8, 0.416)(0.925, 0.416)(0.925, 0)(0.925, 0)(0.925, 0.499364)(1.05, 0.499364)(1.05, 0)(1.05, 0)(1.05, 0.578773)(1.175, 0.578773)(1.175, 0)(1.175, 0)(1.175, 0.634452)(1.3, 0.634452)(1.3, 0)(1.3, 0)(1.3, 0.646625)(1.425, 0.646625)(1.425, 0)(1.425, 0)(1.425, 0.595517)(1.55, 0.595517)(1.55, 0)
\psline[linecolor=blue, linewidth=0.1pt](0.3, 0)(0.3, 0.4385)(0.425, 0.4385)(0.425, 0)(0.425, 0)(0.425, 0.339931)(0.55, 0.339931)(0.55, 0)(0.55, 0)(0.55, 0.316508)(0.675, 0.316508)(0.675, 0)(0.675, 0)(0.675, 0.348456)(0.8, 0.348456)(0.8, 0)(0.8, 0)(0.8, 0.416)(0.925, 0.416)(0.925, 0)(0.925, 0)(0.925, 0.499364)(1.05, 0.499364)(1.05, 0)(1.05, 0)(1.05, 0.578773)(1.175, 0.578773)(1.175, 0)(1.175, 0)(1.175, 0.634452)(1.3, 0.634452)(1.3, 0)(1.3, 0)(1.3, 0.646625)(1.425, 0.646625)(1.425, 0)(1.425, 0)(1.425, 0.595517)(1.55, 0.595517)(1.55, 0)
}
\rput[t](0.3,-0.03){$a$}
\rput[t](0.425,-0.03){$x_1$}
\rput[t](0.55,-0.03){$x_2$}
%\rput[t](0.675,-0.03){}
%\rput[t](0.8,-0.03){$\frac{4}{5}$}
\rput[t](0.7375,-0.06){$\dots$}
\rput[t](0.925,-0.03){$x_{i-1}$}
\rput[t](1.05,-0.03){$x_{i}$}
%\rput[t](1.175,-0.03){$x_i$}
%\rput[t](1.3,-0.03){$\frac{13}{10}$}
\rput[t](1.3,-0.06){$\dots$}
%\rput[t](1.425,-0.03){$\frac{57}{40}$}
\rput[t](1.55,-0.03){$b$}
%Function formula: -171/50 (x)-27/16 ((x)^{3})+11/10+729/160 ((x)^{2})
\psplot[linecolor=red, plotpoints=1000]{0.3}{1.55}{x 2 exp 4.55625 mul 1.1 add x 3 exp -1.6875 mul add x -3.42 mul add }
\psline{<->}(1.6,0)(1.6, 0.578773)
\psline[linestyle=dotted](1.6, 0.578773)(1.05, 0.578773)
\rput[l](1.6, 0.3343865){$f(x_i)$}
\rput[b](0.9875,0.70773){$\Delta x$}
\psline(0.925,0.668773)(1.05, 0.668773)
\psline(0.925,0.648773)(0.925,0.688773)
\psline(1.05, 0.648773)(1.05, 0.688773)
\psaxes[ticks=none, labels=none]{<->}(0,0)(-0.1,-0.1)(1.7,0.9)

%\psbrace[linecolor=red,ref=lC](2)(3){Text I}
%\uput{:U}{$\overbrace{}^\text{\normalsize A line}$}
\end{pspicture}
%\only<handout:1| -1>{%
%\includegraphics[height=3.5cm]{integration/pictures/05-01-rectangles-right.pdf}%
%}%
%\only<handout:2| 2->{%
%\includegraphics[height=3.5cm]{integration/pictures/05-01-rectangles-left.pdf}%
%}%



\begin{itemize}
\item<1->  The width of the interval is $b-a$.
\item<1->  The width one strip is $\Delta x = \frac{b-a}{n}$.
\item<1->  $[a,b]$ is divided into $n$ subintervals:\\ $[x_0, x_1]$, $[x_1,x_2]$, $\ldots ,$ $[x_{n-1},x_n]$,\\ where $x_0 = a$ and $x_n = b$.
\end{itemize}
\column{.45\textwidth}
\begin{itemize}
\item<1->  The \only<handout:1| -1>{right}\only<handout:2| 2->{\alert<handout:2| 2>{left}} endpoints of the subintervals are
\end{itemize}
\abovedisplayskip=0pt
\belowdisplayskip=0pt
\begin{eqnarray*}
x_{\only<handout:1| -1>{1}\only<handout:2| 2->{\alert<handout:2| 2>{0}}} & = & a \only<handout:1| -1>{+ \Delta x}\\
x_{\only<handout:1| -1>{2}\only<handout:2| 2->{\alert<handout:2| 2>{1}}} & = & a + \only<handout:1| -1>{2}\Delta x\\
x_{\only<handout:1| -1>{3}\only<handout:2| 2->{\alert<handout:2| 2>{2}}} & = & a + \only<handout:1| -1>{3}\only<handout:2| 2->{2}\Delta x\\
& \vdots &
\end{eqnarray*}
\begin{itemize}
\item<1->  The height of the $i$th rectangle is $f(x_{i\only<handout:2| 2->{\alert<handout:2| 2>{-1}}})$.
\item<1->  The area of the $i$th rectangle is $f(x_{i\only<handout:2| 2->{\alert<handout:2| 2>{-1}}})\Delta x$.
\end{itemize}
\end{columns}
\abovedisplayskip=0pt
\belowdisplayskip=0pt
\[
\only<handout:1| -1>{R_n}\only<handout:2| 2->{\alert<handout:2| 2>{L_n}} = f(x_{\only<handout:1| -1>{1}\only<handout:2| 2->{\alert<handout:2| 2>{0}}})\Delta x + f(x_{\only<handout:1| -1>{2}\only<handout:2| 2->{\alert<handout:2| 2>{1}}})\Delta x + f(x_{\only<handout:1| -1>{3}\only<handout:2| 2->{\alert<handout:2| 2>{2}}})\Delta x + \cdots + f(x_{n\only<handout:2| 2->{\alert<handout:2| 2>{-1}}})\Delta x
\]
%\begin{definition}[Riemann sum]
%A Riemann sum is an expression of the form $\Delta x\left(f(x_1^*)+\dots +f(x_n^*) \right) $, where $x_i^*$ lies in  $[x_{i-1}, x_{i}]$ and is chosen arbitrarily within that interval.
%\end{definition}
\end{frame}