% begin module summation-formulas

\begin{frame}
\begin{definition}
 {\bf Sigma Notation:} The sum of $n$ terms $a_1,a_2,\ldots,a_n$ is written as
\[
\sum_{i=1}^n a_i=a_1+a_2+\cdots+a_n
\]                                     
where $i$ is 	the {\it index of summation}, $a_i$ is the {\it{$i$'th term}}, and the 
{\it{upper and lower bounds of summation}} are $n$ and $1$ respectively.
\end{definition}
NOTE:  The lower bound doesn't have to be 1. 
 Any integer less than or equal to the upper bound is legitimate.\\
Also, $i$ may be replaced with any other index symbol, usually $j$ or $k$.\\
\pause
\begin{example}
\[
\sum_{j=3}^7 j^2= \pause 9\pause + 16\pause+25+36+49 
\]
\end{example}

\end{frame}

\begin{frame}{Linearity of Summations}
If $ c $ is a constant then 
\begin{enumerate}
\item $\ds  \sum_{i=1}^n c  =  nc $\\ \pause
\item $\ds \sum_{i=1}^n ca_i =  c\sum_{i=1}^n a_i $\\ \pause
\item $\ds  \sum_{i=1}^n (a_i+b_i)  =  \sum_{i=1}^n a_i + \sum_{i=1}^n b_i $
\end{enumerate}

\end{frame}


\begin{frame}
The following \textbf{power sums} will be useful in what follows:
\begin{enumerate}
\item $\ds  \sum_{i=1}^n i  =  \frac{n(n+1)}{2} $\pause
\item $\ds  \sum_{i=1}^n i^2  =  \frac{n(n+1)(2n+1)}{6} $\pause
\item $ \ds \sum_{i=1}^n i^3  =  \left[\frac{n(n+1)}{2}\right]^2 $
\end{enumerate}
\end{frame}


\begin{frame}
\begin{example}
\[
\sum_{i=1}^n \frac{i+1}{n^2} = \pause \sum_{i=1}^n \frac{1}{n^2}(i+1) \pause = \frac{1}{n^2} \sum_{i=1}^n (i+1) 
\]
\pause 
\[
= \frac{1}{n^2} \left[ \sum_{i=1}^n i+\sum_{i=1}^n 1 \right]
\]
\pause 
\[
= \frac{1}{n^2} \left[ \frac{n(n+1)}{2}+n \right] = \frac{n+3}{2n}
\]
\pause
So $ \ds \sum_{i=1}^{10} \frac{i+1}{10^2} =\frac{13}{20} $, \pause $ \ds \sum_{i=1}^{100} \frac{i+1}{100^2} =\frac{103}{200} $, \pause and $ \ds \sum_{i=1}^{1000} \frac{i+1}{1000^2} =\frac{1003}{2000} $ 
\end{example}
\end{frame}

% end module summation-formulas
