\solution{
\ref{problemFindIversef=(5x+6)/(4x+5)}. Set $f(x)=y$. Then
\[
\begin{array}{rcl}
y&=&\displaystyle \frac{5x+6}{4x+5}\\
y(4x+5)&=&5x+6\\
x(4y-5)&=&-5y+6\\
x&=&\displaystyle \frac{-5y+6}{4y-5} .
\end{array}
\]
Therefore the function $\displaystyle x=g(y)=\frac{-5y+6}{4y-5}$ is the inverse of $f(x)$. We write $g=f^{-1}$. The function $g=f^{-1}$ is defined for $\displaystyle y\neq \frac{5}{4}$. For our final answer we relabel the argument of $g$ to $x$:

\[
g(x)=f^{-1}(x)= \frac{-5x+6}{4x-5}\quad .
\]

Let us check our work. In order for $f$ and $g$ to be inverses, we need that $g(f(x))$ be equal to $x$.
\[
g(f(x))=  \frac{-5f(x) +6}{4f(x)-5}=  \frac{-5\frac{(5x+6)}{4x+5} +6}{4\frac{(5x+6)}{4x+5}-5}= \frac{-5(5x+6) +6(4x+5)}{4(5x+6)-5(4x+5)}=\frac{-x}{-1}=x\quad ,
\]
as expected.
}

\solution{\ref{problemFindInversef=(3x+5)/(2x-4)}
This is a concise solution written in form suitable for test taking.
\[
\begin{array}{rcl}
y & =& \displaystyle \frac{3x+5}{2x-4} \\
y(2x-4) & =& 3x+5 \\
2xy-4y & =& 3x+5 \\
2xy-3x & =& 4y+5 \\
x(2y-3) & =& 4y+5 \\
x & =& \displaystyle  \frac{4y+5}{2y-3} \\
\text{Therefore}\quad \displaystyle  f^{-1}(y) & =& \displaystyle  \frac{5+4y}{2y -3 } \\
\displaystyle f^{-1}(x) & =& \displaystyle \frac{5+4x}{2x-3}.
\end{array}
\]
}%