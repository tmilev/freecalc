\solution{\ref{problemFindInversey=ln(x+3)}
\begin{align*}
y & = \ln (x+3) \\
e^y & = e^{\ln (x+3)} \\
e^y & = x + 3 \\
e^y - 3 & = x \\
\text{Therefore} \quad f^{-1}(y) & = e^y - 3.
\end{align*}
The domain of $e^y$ is all real numbers, so the domain of $f^{-1}$ is all real numbers.  
}%

\solution{ \ref{problemFindInversey=(lnx)^2}
\[ 
\begin{array}{rcll|l}y&=&(\ln x)^2 &&\mathrm{take~ \sqrt{~} ~on~both~sides,~ } y\geq 0 \\ \sqrt{y}&=&\ln x&&\mathrm{ ~exponentiate} \\ e^{\sqrt{y}}&=&e^{\ln x}=x \\ f^{-1}(y)&=&e^{\sqrt{y}} \\f^{-1}(x)&=&e^{\sqrt{x}} \end{array}
\]
}

\solution{\ref{problemFindInversey=e^x/(1+2e^x)}
\begin{align*}
y & = \frac{e^x}{1+2e^x} \\
y(1+2e^x) & = e^x \\
y & = e^x(1-2y) \\
\frac{y}{1-2y} & = e^x \\
\ln\frac{y}{1-2y} & = \ln e^x \\
\ln\frac{y}{1-2y} & = x \\
\text{Therefore} \quad f^{-1}(y) & = \ln\frac{y}{1-2y}.
\end{align*}
The natural logarithm function is only defined for positive input values.  
Therefore the domain is the set of all $y$ for which 
\begin{align*}
\frac{y}{1-2y} & > 0.
\end{align*}
This inequality holds if the numerator and denominator are both positive or both negative.  
This happens if either
\begin{enumerate}
\item  $y > 0$ and $y < \frac{1}{2}$, or 
\item  $y < 0$ and $y > \frac{1}{2}$.
\end{enumerate}
The latter option is impossible, so the domain is $\{ y \in \mathbb{R} \ | \ 0 < y < \frac{1}{2}\}$.  
}%
