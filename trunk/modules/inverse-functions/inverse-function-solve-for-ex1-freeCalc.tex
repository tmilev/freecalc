% begin module inverse-function-solve-for
\begin{frame}
%\frametitle{Inverse of a ne-to-one Function}
\begin{example}[\uncover<16->{\alert<handout:0| 16,17>{What if we change the problem to $x\leq -\frac{2}3$?}}]
Given: $\alert<handout:0| 2>{f(x) =  3x^2+4x-7}$ \alert<handout:0| 3,16,17>{with domain $x\only<1-16| handout:0>{\geq}\only<17->{\leq} -\frac{2}{3}$}.  Find $f^{-1}(x)$.
\begin{columns}
\column{0.4\textwidth}
\psset{xunit=0.35cm, yunit=0.35cm}
\begin{pspicture}(-9,-9)(5,5)
\psframe*[linecolor=white](-9,-9)(5,5)
\tiny
\psaxes[ticks=none, labels=none]{<->}(0,0)(-9,-9)(4.7,4.7)
\uncover<2-16| handout:0>{ %
\psplot[linecolor=\fcColorGraph, plotpoints=1000] {-0.66}{1.401612274}{x 2 exp 3 mul x 4 mul add -7 add }
}
\uncover<17->{ %
\psplot[linestyle=dashed, linecolor=gray!50, plotpoints=1000]{-0.66}{1.401612274}{x 2 exp 3 mul x 4 mul add -7 add }
}
\uncover<2,17>{ %
\psplot[linecolor=\fcColorGraph, plotpoints=1000]{-2.7349}{-0.67}{x 2 exp 3 mul x 4 mul add -7 add }
}
\uncover<3-16| handout:0>{ %
\psplot[linestyle=dashed, linecolor=gray!50, plotpoints=1000] {-2.7349}{-0.67}{x 2 exp 3 mul x 4 mul add -7 add }
}
\uncover<14->{
\psline[linecolor=blue, linestyle=dashed](-6.5, -6.5)(4.5,4.5)
}
\uncover<15-16| handout:0>{
\psplot[linecolor=red, plotpoints=1000]{-8.33333}{4.5}{-0.666667 25 x 3 mul add sqrt 0.333333 mul add }
}
\uncover<17->{
\psplot[linestyle=dashed, linecolor=gray!50, plotpoints=1000] {-8.33333}{4.5}{-0.666667 25 x 3 mul add sqrt 0.333333 mul add }
}
\uncover<15-16>{
\psplot[linecolor=gray!50, linestyle=dashed, plotpoints=1000] {-8.33333}{4.5}{-0.666667 25 x 3 mul add sqrt -0.333333 mul add }
}
\uncover<17->{
\psplot[linecolor=\fcColorGraph, plotpoints=1000] {-8.33333}{4.5}{-0.666667 25 x 3 mul add sqrt -0.333333 mul add }
}
\uncover<15-16| handout:0>{\rput[lb](-6, 1){$y=f^{-1}(x)$}}
\uncover<11->{\rput (-3.5, -8 ){$(-\frac{2}{3}, -\frac{25}{3})$}}
\uncover<2-16| handout:0>{\rput[tl](1.8, 4.45){$y=f(x)$}}
\uncover<14->{\rput[l] (-7.8, -2.2 ){$(-\frac{25}{3}, -\frac{2}{3})$}}
\uncover<17->{\rput[rt](4.5, -3){$y=f^{-1}(x)$}}
\uncover<17>{ \rput[tr](-3, 4.5){$y=f(x)$}}
\uncover<14->{\fcFullDot{-8.33333}{-0.666667}}
\fcFullDot{-0.666667}{-8.33333}
\end{pspicture}
\uncover<13->{Final }\uncover<12->{answer}\uncover<13->{, \alert<handout:0| 13>{relabelled}:}
\[
\uncover<12->{
f^{-1}(\only<12| handout:0>{y}\only<13->{\alert<handout:0| 13>{x}} )=-\frac{2}{3} \only<1-16| handout:0>{+}\only<17->{\alert<handout:0| 17>{-}} \frac{\sqrt{25 +3\only<12| handout:0>{y} \only<13->{\alert<handout:0| 13>{x}}\phantom{y} }}{3}\quad.
}
\]

\column{0.6\textwidth}

\[\begin{array}{rcl}
\uncover<4->{3x^2+4x-7&=&y } \\
\uncover<4->{\alert<handout:0| 7>{3}x^2+\alert<handout:0| 6>{4}x+\alert<handout:0| 8>{(-7-y)}&=&0 }
\end{array}
\]
\uncover<5->{That's \alert<handout:0| 6,7,8>{a quadratic equation in $x$}. Solve:}
\[\begin{array}{l}
\uncover<5->{
\phantom{=}\displaystyle \frac{-\alert<handout:0| 6>{4} \pm \sqrt{\alert<handout:0| 6>{4}^2-4\cdot\alert<handout:0| 7>{3}\cdot\alert<handout:0| 8>{(-y-7)} }}{2\cdot\alert<handout:0| 7>{3}} \\
%~&=& \frac{-2 \pm \sqrt{25+3y}}{3}\\
}
\\
\uncover<9->{=\displaystyle-\frac{2 \pm \sqrt{25+3y}}{3}=} \uncover<10->{\displaystyle-\frac{2}3 \pm \frac{\sqrt{25+3y}}{3}\quad .}
\end{array}
\]
\uncover<11->{
We are given $x\only<11-16| handout:0>{\geq}\only<17->{\alert<handout:0| 17>{\leq}}-\frac{2}3 $, therefore $x=-\frac{2}{3}\only<11-16| handout:0>{+}\only<17->{\alert<handout:0| 17>{-}}\frac{\sqrt{25+3y}}{3}=f^{-1}(y)$.
}
\end{columns}
\vspace{-10pt}
\end{example}
\end{frame}
% end module inverse-function-solve-for
