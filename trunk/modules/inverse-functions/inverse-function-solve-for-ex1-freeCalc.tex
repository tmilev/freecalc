% begin module inverse-function-solve-for
\begin{frame}
%\frametitle{Inverse of a ne-to-one Function}
\begin{example}[\uncover<16->{\alert<16,17>{What if we change problem to $x\leq -\frac{2}3$?} } ]
Given: $\alert<2>{f(x) =  3x^2+4x-7}$ \alert<3,16,17>{with domain $x\only<1-16>{\geq}\only<17->{\leq} -\frac{2}{3}$}.  Find $f^{-1}(x)$.
\begin{columns}
\column{0.4\textwidth}
\psset{xunit=0.35cm, yunit=0.35cm}
\begin{pspicture}(-5, -5)(5,5) 
\psframe*[linecolor=white](-10,-10)(6,6) \tiny
\psaxes[ticks=none, labels=none]{<->}(0,0)(-9,-9)(4.7,4.7)
%Function formula: -7+4 (x)+3 ((x)^{2}) 
\uncover<2-16>{ %
\psplot[linecolor=\psColorGraph, plotpoints=1000]{-0.66}{1.401612274}{x 2 exp 3 mul x 4 mul add -7 add }
\rput[tl](1.8, 4.45){$y=f(x) %=3x^2+4x-7  
$ %, $x\geq -2/3$
}
}
\uncover<17->{ %
\psplot[linestyle=dashed, linecolor=gray!50, plotpoints=1000]{-0.66}{1.401612274}{x 2 exp 3 mul x 4 mul add -7 add }
}
\uncover<2,17>{ %
\psplot[linecolor=\psColorGraph, plotpoints=1000]{-2.7349}{-0.67}{x 2 exp 3 mul x 4 mul add -7 add }
\uncover<17>{ \rput[tr](-3, 4.5){$y=f(x)$}}
}
\uncover<3-16>{ %
\psplot[linestyle=dashed, linecolor=gray!50, plotpoints=1000]{-2.7349}{-0.67}{x 2 exp 3 mul x 4 mul add -7 add }
}
\uncover<14->{
\psline[linecolor=blue, linestyle=dashed](-6.5, -6.5)(4.5,4.5)
}
\uncover<15-16>{
%Function formula: 1/3 (sqrt{}(3 (x)+25))-2/3 
\psplot[linecolor=red, plotpoints=1000]{-8.33333}{4.5}{-0.666667 25 x 3 mul add sqrt 0.333333 mul add }
}
\uncover<17->{
%Function formula: 1/3 (sqrt{}(3 (x)+25))-2/3 
\psplot[linestyle=dashed, linecolor=gray!50, plotpoints=1000]{-8.33333}{4.5}{-0.666667 25 x 3 mul add sqrt 0.333333 mul add }
\rput[rt](4.5, -3){$y=f^{-1}(x) %= \frac{-2+\sqrt{25+3x}}{3} 
$}
}
\uncover<15-16>{
\psplot[linecolor=gray!50, linestyle=dashed, plotpoints=1000]{-8.33333}{4.5}{-0.666667 25 x 3 mul add sqrt -0.333333 mul add }
\rput[lb](-6, 1){$y=f^{-1}(x) %= \frac{-2+\sqrt{25+3x}}{3} 
$}
}
\uncover<17->{
\psplot[linecolor=\psColorGraph, plotpoints=1000]{-8.33333}{4.5}{-0.666667 25 x 3 mul add sqrt -0.333333 mul add }
}
\psFullDot{-0.666667}{-8.33333}
\uncover<11->{
\rput (-3.5, -8 ){$(-\frac{2}{3}, -\frac{25}{3})$}
}
\uncover<14->{
\psFullDot{-8.33333}{-0.666667}
\rput[l] (-7.8, -2.2 ){$(-\frac{25}{3}, -\frac{2}{3})$}
}
\end{pspicture} 
\uncover<13->{Final }\uncover<12->{answer}\uncover<13->{, \alert<13>{relabelled}:}
\[
\uncover<12->{
f^{-1}(\only<12>{y}\only<13->{\alert<13>{x}} )=-\frac{2}{3} \only<1-16>{+}\only<17->{\alert<17>{-}} \frac{\sqrt{25 +3\only<12>{y} \only<13->{\alert<13>{x}}\phantom{y} }}{3}\quad.
}
\]

\column{0.6\textwidth}

\[\begin{array}{rcl}
\uncover<4->{3x^2+4x-7&=&y } \\
\uncover<4->{\alert<7>{3}x^2+\alert<6>{4}x+\alert<8>{(-7-y)}&=&0 }
\end{array}
\]
\uncover<5->{That's \alert<6,7,8>{quadratic equation in $x$}, solutions:}
\[\begin{array}{l}
\uncover<5->{
\phantom{=}\displaystyle \frac{-\alert<6>{4} \pm \sqrt{\alert<6>{4}^2-4*\alert<7>{3}*\alert<8>{(-y-7)} }}{2*\alert<7>{3}} \\
%~&=& \frac{-2 \pm \sqrt{25+3y}}{3}\\
}
\\
\uncover<9->{=\displaystyle-\frac{2 \pm \sqrt{25+3y}}{3}=} \uncover<10->{\displaystyle-\frac{2}3 \pm \frac{\sqrt{25+3y}}{3}\quad .}
\end{array}
\]
\uncover<11->{
We are given $x\only<11-16>{\geq}\only<17->{\alert<17>{\leq}}-\frac{2}3 $, therefore $x=-\frac{2}{3}\only<11-16>{+}\only<17->{\alert<17>{-}}\frac{\sqrt{25+3y}}{3}=f^{-1}(y)$.
}
\end{columns}
\end{example}
\end{frame}
% end module inverse-function-solve-for
