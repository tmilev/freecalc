% begin module inverse-notation-warning
\begin{frame}
\uncover<1->{The inverse of $f$ is denoted as $f^{-1}$.} \uncover<2->{This notation is one of the most frequent causes of student confusion.} \uncover<3->{\alert<3>{\textbf{WARNING:}}}
\uncover<3->{
\[
f^{\alert<4>{-1}} \alert<4>{(x)}  \text{ does not mean } \left(f \alert<5>{(x)}\right)^{\alert<5>{-1}} = \frac{1}{f(x)}\quad .
\]
}
\uncover<4->{The notations are  \alert<4,5>{different:} the superscript $-1$ has \alert<4,5>{different positions}.}
\begin{itemize}
\item<6->  $f^{-1}$ is the compositional inverse of $f$.
\item<7->  $\frac{1}{f(x)}$ is the multiplicative inverse of $f(x)$.
\item<8->  $f^{2}(x)$ is an abbreviation for $(f(x))^2$,  $f^{3}(x)$ is an abbreviation of $(f(x))^3$, and so on.
\item<9->  \alert<10>{\alert<9>{However, } $f^{-1}(x)$ is not the abbreviation of $\left(f(x)\right)^{-1}$ and does not follow this pattern.}
\end{itemize}

\only<10>{ \begin{quotation}
No one blamed English language of being logical.
\end{quotation}
-Bjarne Stroustrup, creator of the programming language C++
}

\uncover<11->{
\[
f^{n}(x)= \left\{ \begin{array}{ll} \alert<11>{\text{stands for } \left(f(x)\right)^n} & \alert<11>{ \text{when } n=1,2,3,\dots} \\
\alert<12>{\text{stands for inverse of } f \text{ applied to }x} & \alert<12>{ \text{when } n=-1} \\
\alert<13>{\text{should be avoided } } & \alert<13>{ \text{when } n\neq -1, 1,2,3,\dots .}
\end{array} \right.
\]
}
\uncover<14->{ To reduce confusion, if possible, use $\frac{1}{f(x)}$ instead of $\left(f(x)\right)^{-1}$.}


\vspace{8cm}
\end{frame}
% end module inverse-notation-warning
