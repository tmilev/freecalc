% begin module arcsin-ex1
\begin{frame}
\begin{example}[Example 1, p. 217]
\begin{columns}[t]
\column{.4\textwidth}
\[
\text{Find } \ \sin^{-1} \left( \frac{1}{2}\right) 
\]
\begin{itemize}
\item<2->  $\sin (\pi / 6) = 1/2$.
\item<3->  $-\pi /2 \leq \pi / 6 \leq \pi /2$.
\item<4->  Therefore $\sin^{-1} \left( \frac{1}{2}\right) = \frac{\pi}{6}$.
\end{itemize}
\column{.6\textwidth}
\[
\text{Find } \ \tan \left( \arcsin \left( \frac{1}{3}\right) \right)
\]
\begin{itemize}
\item<5->  Let $\theta = \arcsin (1/3)$, so $\sin \theta = 1/3$.
\item<6->  Draw a right triangle with opposite side $1$ and hypotenuse $3$.
\item<7->  \alert<handout:0| 7-8>{Length of adjacent side $ = \uncover<8->{\sqrt{3^2-1^2} = \sqrt{8} = 2\sqrt{2}.}$}
\item<9->  Then \alert<handout:0| 9-10>{$\tan (\arcsin (1/3)) = \uncover<10->{1/(2\sqrt{2}).}$}
\end{itemize}
\ \only<handout:0| -5>{%
\includegraphics[width=5cm]{inverse-trig/pictures/07-06-ex1a.pdf}%
}%
\only<handout:0| 6-7>{%
\includegraphics[width=5cm]{inverse-trig/pictures/07-06-ex1b.pdf}%
}%
\only<8->{%
\includegraphics[width=5cm]{inverse-trig/pictures/07-06-ex1c.pdf}%
}%
\end{columns}
\end{example}
\end{frame}
% end module arcsin-ex1
