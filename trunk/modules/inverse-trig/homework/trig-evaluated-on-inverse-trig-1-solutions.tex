\solution{\ref{problemsin(2arcsin x)}.
Let $y = \Arcsin x$.  Then $\sin y = x$, and we can draw a right triangle with opposite side length $x$ and hypotenuse length $1$ to find the other trigonometric ratios of $y$.  

\begin{center}
\psset{xunit=1.0cm,yunit=1.0cm,algebraic=true,dotstyle=o,dotsize=3pt 0,linewidth=0.8pt,arrowsize=3pt 2,arrowinset=0.25}
\begin{pspicture*}(-3.33,-6.11)(14.05,6.58)
\psline(0,0)(4,0)
\psline(0,0)(4,3)
\psline(4,3)(4,0)
\psline(4,0.2)(3.8,0.2)
\psline(3.8,0.2)(3.8,0)
\rput[tl](0.83,0.5){$y$}
\rput[tl](1.56,1.82){$1$}
\rput[tl](4.1,1.4){$x$}
\rput[tl](1.7,-0.05){$\sqrt{1-x^2}$}
\parametricplot{0.0}{0.6435011087932844}{1*0.66*cos(t)+0*0.66*sin(t)+0|0*0.66*cos(t)+1*0.66*sin(t)+0}
\end{pspicture*}
\end{center}

Then $\cos y = \sqrt{1-x^2}/1 = \sqrt{1-x^2}$.  
Now we use the double angle formula to find $\sin(2\Arcsin x)$.  

\begin{align*}
\sin (2 \Arcsin x) & = \sin (2y) \\
& = 2\sin y\cos y \\
& = 2x\sqrt{1-x^2}.
\end{align*}
}

\solution{\ref{problemsin(3arcsin x)}. Use the result of Problem \ref{problemsin(2arcsin x)}. This also requires the addition formula for sine: 
\[
\sin(A+B) = \sin A \cos B + \sin B\cos A,
\]
and the double angle formula for cosine:
\[
\cos (2y) = \cos^2 y - \sin^2 y.
\]  
\[
\begin{array}{r@{~}c@{~}ll|l}
\sin(3\Arcsin x) & =& \sin(3y) \\
& =& \sin (2y + y) \\
& =& \sin(2y)\cos y + \sin y \cos (2y) &&\text{Use addition formula }\\
& =& (2\sin y \cos y)\cos y + \sin y (\cos^2 y - \sin^2 y)&&\text{Use double angle formulas} \\
& =& 2\sin y \cos^2 y + \sin y \cos^2 y - \sin^3 y \\
& =& 3\sin y \cos^2 y - \sin^3 y \\
& =& 3x(1-x^2) - x^3 \\
& =& 3x - 4x^3.
\end{array}
\]
}