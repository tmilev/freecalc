% begin module limit-laws-fact
\begin{frame}
When computing a limit as $x$ approaches $a$, we don't care what happens when $x = a$.  This gives the following \alert{useful fact}:
\abovedisplayskip=0pt
\belowdisplayskip=-15pt
\abovedisplayshortskip=0pt
\belowdisplayshortskip=0pt
\begin{align*}
\text{If}\quad%
f(x) & = g(x)\\
\uncover<1->{%
\intertext{when $x\neq a$,}
}%
\text{then}\quad%
\lim_{x\to a} f(x) & = \lim_{x\to a} g(x),\\
\uncover<1->{%
\intertext{provided the limit exists.}
}%
\end{align*}
\uncover<1->{%
We can use this fact to find $\lim_{x\to a}f(x)$ when $f(a)$ has the form $\frac{0}{0}$.  
In such a case, we use algebra to find a function $g(x)$ that agrees with $f(x)$ at all points except $x = a$.  
There are three common techniques:
\begin{enumerate}
\item  Factoring.
\item  Using a conjugate radical. 
\item  Finding a common denominator.
\end{enumerate}
}%
\end{frame}
% end module limit-laws-fact
