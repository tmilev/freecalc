% begin module tangent-def
\begin{frame}
\begin{columns}[c]
\column{.4\textwidth}
\psset{xunit=1cm, yunit=1cm}
\begin{pspicture}(-5, -5)(5,5) 
\psframe*[linecolor=white](-5,-5)(5,5) 
\tiny
\psaxes[ticks=none, labels=none]{<->}(0,0)(-1.05,-0.52)(3.1,4.2)
\psLabelsWithOnes{3}{4.2}
\rput(-0.3,0.5){$y=x^{2}$} 
%Function formula: (x)^{2} 
\psplot[linecolor=red, plotpoints=1000]{-1}{2}{x 2 exp }
\psFullDot{1}{1}
\rput[lt](1.1,1){$P=(1,1)$}
\psline[linecolor=blue]( 0.25,-0.5)(2.55 ,4.1)
\end{pspicture} 
\column{.6\textwidth}
We say that the slope of the tangent is the limit of the slope of the secants (limit will be defined later in the lecture).  We write:
\[
\lim_{Q\rightarrow P} m_{PQ} = m, \qquad \lim_{x\rightarrow 1}\frac{x^2 - 1}{x - 1} = 2 .
\]
\uncover<2->{%
If the slope is indeed $2$, then the equation of the tangent is
\[
y - 1 = 2(x - 1), \qquad \text{ or }\qquad y = 2x - 1.
\]
}
\end{columns}
\end{frame}
% end module tangent-def
