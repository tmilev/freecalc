\begin{frame}
\begin{example}
Find an equation of the line passing through $(\alertNoH{13}{1},\alertNoH{13}{2})$ and $(\alertNoH{13}{2},\alertNoH{13}{3})$.
\[
\fcAnswerUncover{3}{10}{(\alertNoH{13}{4}-\alertNoH{13}{2})} \uncover<3->{(\alertNoH{12,4}{x-\alertNoH{13}{1}})=} \fcAnswerUncover{3}{9}{(\alertNoH{13}{2}-\alertNoH{13}{1})} \uncover<3->{(\alertNoH{12,5}{y-\alertNoH{13}{2}})}
\]
\begin{itemize}

\item<2-> It suffices to manufacture a linear equation such that when we plug in $(\alertNoH{13}{1},\alertNoH{13}{2})$ and $(\alertNoH{13}{2},\alertNoH{13}{4})$ we get an identity.
\item<3-> A (very simple) equation satisfied by $\alertNoH{4}{x=\alertNoH{13}{1}}$, $\alertNoH{5}{y=\alertNoH{13}{2}}$ is:

\hfil \hfil$
\alertNoH{4}{\alertNoH{6}{x}-\alertNoH{13}{1}}=\alertNoH{5}{\alertNoH{7}{y}-\alertNoH{13}{2}}.
$

\uncover<4->{This is so because both sides \alertNoH{4,5}{become zero} when $\alertNoH{4}{x=\alertNoH{13}{1}}$, $\alertNoH{5}{y=\alertNoH{13}{2}}$.}
\item<6-> If we plug in $\alertNoH{6}{x=\alertNoH{13}{2}}$ and $\alertNoH{7}{y=\alertNoH{13}{4}}$ in the above \alertNoH{8}{we don't get an identity}\uncover<9->{, but that can be easily fixed:}

\hfil \hfil $
\uncover<10->{\alertNoH{10}{(\alertNoH{13}{4}-\alertNoH{13}{2})}} \alertNoH{9}{(\alertNoH{6,13}{2}-\alertNoH{13}{1})} \only<10->{=}\only<handout:0|6-9>{\alertNoH{8}{\neq}}  \uncover<9->{\alertNoH{9}{(\alertNoH{13}{2}-\alertNoH{13}{1})}} \alertNoH{10}{(\alertNoH{7,13}{4}-\alertNoH{13}{2})} 
$
\item<11-> Perhaps the last modification caused $x=\alertNoH{13}{1}$, $y=\alertNoH{13}{2}$ to no longer be solutions? \uncover<12->{No - both sides are still zero when $\alertNoH{12}{x=\alertNoH{13}{1}}$, $\alertNoH{12}{y=\alertNoH{13}{2}}$.}
\end{itemize}
\end{example}
\end{frame}

\begin{frame}
\begin{example}
Find an equation of the line passing through $(\alertNoH{1}{x_1},\alertNoH{1}{y_1})$ and $(\alertNoH{1}{x_2}, \alertNoH{1}{y_2})$.
\[
(\alertNoH{1}{y_2}-\alertNoH{1}{y_1})(x-\alertNoH{1}{x_1})=(\alertNoH{1}{x_2}-\alertNoH{1}{x_1})(y-\alertNoH{1}{y_1})
\]
\begin{itemize}

\item It suffices to manufacture a linear equation such that when we plug in $(\alertNoH{1}{x_1},\alertNoH{1}{y_1})$ and $(\alertNoH{1}{x_2},\alertNoH{1}{y_2})$ we get an identity.
\item A (very simple) equation satisfied by $x=\alertNoH{1}{x_1}$, $ y=\alertNoH{1}{y_2}$ is:

\hfil \hfil$
x-\alertNoH{1}{x_1}=y-\alertNoH{1}{y_2}.
$

This is so because both sides become zero when $x=\alertNoH{1}{x_1}$, $y=\alertNoH{1}{y_1}$.
\item If we plug in $x=\alertNoH{1}{x_2}$ and $y=\alertNoH{1}{y_2}$ in the above we don't get an identity (necessarily), but that can be easily fixed:

\hfil \hfil $
(\alertNoH{1}{y_2}-\alertNoH{1}{y_1}) (\alertNoH{1}{x_2}-\alertNoH{1}{x_1}) =(\alertNoH{1}{x_2}-\alertNoH{1}{x_1})(\alertNoH{1}{y_2}-\alertNoH{1}{y_1}) 
$
\item Perhaps the last modification caused $x=\alertNoH{1}{x_1}$, $y=\alertNoH{1}{y_1}$ to no longer be solutions? No - both sides are still zero when $x=\alertNoH{1}{x_1}$, $y=\alertNoH{1}{y_1}$.
\end{itemize}
\end{example}
\end{frame}