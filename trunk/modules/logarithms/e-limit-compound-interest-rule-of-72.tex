\begin{frame}
\begin{example}
When quickly computing interest rate ``in the head'', financial advisors often use the following trick called the ``rule of $72$''. To find the time in years $t$ needed for a sum to double under compound interest rate of $k\%$, financial advisors simply approximate $t\approx \frac{72}{k}$. 

To illustrate the rule, under an interest rate of $2\%$, one needs approximately $\frac{72}{2}=36$ years for the sum to double. Under interest rate of $6\%$, the sum doubles after only about $\frac{72}{6}=12$ years. In $36$ years an interest of $6\%$ would double $3$ times, in other words would increase by a factor of $2^3=8$.


Using the approximation $e\approx \left(1+\frac{1}{n}\right)^n$ for large $n$, justify the rule of $72$.


\end{example}


\end{frame}