%begin module e-limit-problems-ex1
\begin{frame}
\begin{example}
Compute \[\lim_{x\to \infty} \left(\frac{x+3}{x} \right)^{x}\].

\[
\begin{array}{rcll|l}
\displaystyle\lim\limits_{x\to \infty}\left(\frac{x+3}{x}\right)^x &=&\displaystyle  \lim\limits_{x\to \infty}\left(1+\frac{3}{x}\right)^x \\
&=& \displaystyle \lim\limits_{x\to \infty}\left(1+\frac{1}{\frac{x}{3}}\right)^{3\frac{x}{3} } && \text{Set } \frac{x}{3}=y\\
&=&\displaystyle \lim\limits_{\substack{x\to \infty \\ \frac{x}{3}=y\to \infty }}\left(1+\frac{1}{y}\right)^{3y} \\
&=&\displaystyle \displaystyle\lim\limits_{y\to \infty}\left(\left(1+\frac{1}{y}\right)^y\right)^3= e^3 \quad . 
\end{array}
\]

\end{example}
\end{frame}
\begin{frame}
\begin{example}
Compute 
\[
\begin{array}{rll|l}
&\displaystyle \lim_{x\to \infty} \left(\frac{x}{x-2} \right)^{2x+2}\\
=&\displaystyle
\lim\limits_{x\to \infty}\left(\frac{x-2 +2}{x-2} \right)^{2x+2}
= 
\lim\limits_{x\to \infty}\left(1+\frac{2}{x-2} \right)^{2x+2} \\
=&\displaystyle \lim\limits_{x\to \infty}\left(1+\frac{1}{\frac{x-2}{2}} \right)^{2(x-2+2)+2}\\
=&\displaystyle \lim\limits_{x\to \infty}\left(1+\frac{1}{\frac{x-2}{2}} \right)^{4\frac{x-2}{2}+6} = \displaystyle \lim\limits_{\frac{x-2}{2}
= y\to\infty} \left(1+\frac{1}{y}\right)^{4y+6}  && \text{Set } y=\frac{x-2}{2}\\
=&\displaystyle \lim\limits_{y\to \infty} \left(\left(1+\frac{1}{y} \right)^{y}\right)^4 \lim\limits_{y\to\infty} \left(1+\frac{1}{y}\right)^6 =e^4\cdot (1+0)^6\\
=& e^4
\quad . 
\end{array}
\]

\end{example}
\end{frame}
%end module e-limit-problems-ex1
