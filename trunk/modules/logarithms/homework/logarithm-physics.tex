% begin homework logarithm-physics
A particle moves in such a way that, after $t$ seconds, it is $s(t) = \ln (2-t+t^2)$ m to the right of the origin.  
\begin{enumerate}
\item  What is the closest it comes to the origin?  

\solution{%
\begin{align*}
s'(t) & = \frac{-1+2t}{2-t+t^2}. \\
\text{Set } \quad s'(t) & = 0. \\
\frac{-1+2t}{2-t+t^2} & = 0 \\
-1+2t & = 0 \\
t & = 1/2.
\end{align*}

Therefore the position function has a critical number at $t = 1/2$.  
The parabola $2-t+t^2$ has a global minimum at $t = 1/2$, and the natural logarithm function is an increasing function, so $\ln(2-t+t^2)$ also has a global minimum at $t=1/2$.  
The minimum value is $s(1/2) = 2-1/2+(1/2)^2 = 7/4$m.  
}%

\item  What is its acceleration when it is closest to the origin?  

\item  For what values of $t$ is the position function $s(t)$ defined?  


\end{enumerate}

\end{enumerate}
% end homework logarithm-physics
