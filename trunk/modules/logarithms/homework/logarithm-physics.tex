% begin homework logarithm-physics
A particle moves in such a way that, after $t$ seconds, it is $s(t) = \ln \left(2-t+t^2\right)$ m to the right of the origin.  
\begin{enumerate}
\item  What is the closest it comes to the origin?  

\solution{%
\begin{align*}
s'(t) & = \frac{-1+2t}{2-t+t^2}. \\
\text{Set } \quad s'(t) & = 0. \\
\frac{-1+2t}{2-t+t^2} & = 0 \\
-1+2t & = 0 \\
t & = \frac{1}{2}.
\end{align*}

Therefore the position function has a critical number at $t = \frac{1}{2}$.  
The parabola $2-t+t^2$ has a global minimum at $t = \frac{1}{2}$, and the natural logarithm function is an increasing function, so $\ln\left(2-t+t^2\right)$ also has a global minimum at $t=\frac{1}{2}$.  
The minimum value is $s(\frac{1}{2}) = \ln\left(2- \frac{1}{2} + \left( \frac{ 1}{2} \right)^2\right) = \ln\left( \frac{7}{4} \right)\approx 0.5596$ m.  
}%

\item  What is its acceleration when it is closest to the origin?  

\solution{%
\begin{align*}
v(t) = s'(t) & = \frac{-1+2t}{2-t+t^2}. \\
a(t) = s''(t) & = \frac{(2-t+t^2)(2) - (-1+2t)(-1+2t)}{(2-t+t^2)^2} \\
 & = \frac{(4-2t+2t^2) - (1-4t+4t^2)}{(2-t+t^2)^2} \\
 & = \frac{3+2t-2t^2}{(2-t+t^2)^2}. \\
\text{Plug in $t = \frac{1}{2}$:} \quad a\left(\frac{1}{2}\right) & = \frac{3+2(\frac{1}{2})-2(\frac{1}{2})^2}{\left(\frac{7}{4}\right)^2} \\
& = \frac{3+1-\frac{1}{2}}{\frac{49}{16}} \\
& = \frac{\frac{7}{2}}{\frac{49}{16}} \\
& = \frac{8}{7}.
\end{align*}
Therefore the particle is accelerating at a rate of $\frac{8}{7}\approx 1.1429$ m/s$^2$ when it is closest to the origin.  
}%

\item  For which values of $t$ is the position function $s(t)$ defined?  

\solution{%
The natural logarithm function is defined for all positive input values.  
The formula $y = 2-t+t^2$ is always positive.  
To see this, note that its graph is a parabola with discriminant $b^2-4ac = (-1)^2-4(1)(2) = -7$, which is negative.  
This means that the graph of $y = 2-t+t^2$ never touches the $x$-axis, and hence it is either always positive or always negative.  
Since $y(0) = 2$ is positive, this means all values are positive.  
Therefore $\ln(2-t+t^2)$ is defined for all input values $t$.  
}%

\end{enumerate}
% end homework logarithm-physics
