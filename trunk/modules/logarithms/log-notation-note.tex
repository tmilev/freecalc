% begin module natural-logarithm-def
\begin{frame}
\alertNoH{1,2}{What does $\log x$ stand for?} \uncover<2->{ \alertNoH{2}{\textbf{WARNING:}} there are  \alertNoH{2}{ \textbf{two different}} accepted uses for $\log x$.}
\vskip -0.15cm
\begin{columns}
\column{0.4\textwidth}
\begin{itemize}
\item<3-> In some texts/applications $\log x$ stands for
\[
\log x = \log_{10} x\quad .
\]
\item<4-> Used in many engineering texts.
\item<4-> Used in many natural sciences texts.
\item<4-> Used in many high school textbooks.
\item<4-> Used in old math textbooks.
\end{itemize}
\column{0.63\textwidth}
\begin{itemize}
\item<5-> In other texts/applications $\log x$ stands for {\color{gray}(the principal branch of the)} \textbf{complex logarithm}

$
\alertNoH{7}{\log x =}\left\{\begin{array}{ll} \alertNoH{7}{\ln x} =\log_e x& \alertNoH{7}{ \text{if  } x>0} \\ \ln (-x) +\pi i&\text{if } x<0 \\ \textbf{?} &\text{for } x \notin \mathbb R\end{array}\right. \quad .
$
\begin{itemize}
\item<6-> Used in mathematical, many computer science texts.
\item<6-> Used in many natural science texts.
\item<6-> Used in most computer algebra systems.
\item<6-> This is the notation accepted by most mathematicians.
\end{itemize}
\item<7-> $\log $ and $\ln $ have different domains but else coincide: $\ln$ is defined for positive reals, and $\log$ - for non-zero complex.
\end{itemize}
\end{columns}
\end{frame}
% end module natural-logarithm-def
