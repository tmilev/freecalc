% begin module logarithm-def
\begin{frame}
\frametitle{Logarithmic Functions}
\begin{columns}[c]
\column{.3\textwidth}
\psset{xunit=0.7cm, yunit=0.7cm}
\begin{pspicture}(-5, -5)(5,5) 
\psframe*[linecolor=white](-5,-5)(5,5) 
\psaxes[ticks=none, labels=none]{<->}(0,0)(-2,-2.1)(4.2, 4.2)
\psline(-0.1, 1)(0.1,1)
\rput[r](-0.2, 1){\footnotesize$1$}
\rput(0.9, 3){\footnotesize$y=a^x$}
%Function formula: 2^{x} 
\psplot[linecolor=red, plotpoints=1000]{-2}{2}{2 x exp }
\uncover<8->{
\psplot[linecolor=red, plotpoints=1000]{0.25}{4}{x ln 0.693147181 div }
\psline[linestyle=dashed, linecolor=blue](-1.9, -1.9)(4,4) 
\rput[tl](2, 0.7){\footnotesize$y=\log_ax$}
}
\end{pspicture} 

\column{.7\textwidth}
\begin{itemize}
\item  Suppose $a > 0$, $a\neq 1$.
\item<2->  Let $f(x) = a^x$.
\item<3->  Then $f$ is either increasing or decreasing.
\item<4->  Therefore $f$ is one-to-one.
\item<5->  Therefore $f$ has an inverse function, $f^{-1}$.
\item<7->  The graph shows $y = a^x$ for $a > 1$.
\item<8->  The graph of $y = \log_a x$ is the reflection of this in the line $y = x$.
\end{itemize}
\end{columns}
\uncover<6->{%
\begin{definition}[$\log_a x$]
The inverse function of $f(x) = a^x$ is called the logarthmic function with base $a$, and is written $\log_a x$.  It is defined by the formula
\[
\log_a x = y \qquad \Leftrightarrow \qquad a^y = x .
\]
\end{definition}
}%
\end{frame}
% end module logarithm-def
