\begin{frame}
\begin{example}
The Richter magnitude $M_L$ of an earthquake is determined from the logarithm of the amplitude $A$ of waves recorded by seismographs (with adjustment to compensate for the distance between the measuring station and the estimated epicenter of the earthquake). The formula is

\hfil \hfil$
M_L=\log_{10}A -J_0(\delta),
$

\noindent where $J_0(\delta)$ depends on the distance $\delta$ from the epicenter. Compare the amplitudes $A_1$ and $A_2$ of the seism. waves of two hypothetical earthquakes of magnitudes $4$ and $7.5$ with the same epicenter.

\uncover<2->{
\hfil \hfil $
\begin{array}{rrcl}
&\log_{10}A_2 -J_0(\delta)&=&7.5\\
\cline{1-1}&\log_{10}A_1 -J_0(\delta)&=&4\\
& \log_{10}A_2-\log_{10}A_1 &=&3.5\\
&\displaystyle \log_{10}\left(\frac{A_2}{A_1}\right) &=&3.5\\
&\displaystyle \frac{A_2}{A_1} &=&10^{3.5}\\
&A_2&= & 10^{3.5}A_1\approx 3162 A_1.
\end{array}
$

The stronger earthquake has about $3160$ times larger wave amplitude.
}
\end{example}
\end{frame}