% begin module MVT-ex1
\begin{frame}
\begin{example}
Suppose that $ f(0)=-3 $ and $ f'(x)\le 5 $ for all $ x. $ What is the largest possible value of $ f(2) $?
\small 
\begin{itemize}
\setlength\itemsep{0.2em}
\item<2->  We are given that $f$ is differentiable (and therefore continuous) everywhere. 
\item<3-> In particular, we can apply the Mean Value Theorem on the interval $[0, 2]$. There exists a number $c$ such that
\[
f(2)-f(0) = f'(c)(2 -0), \text{ so, }
\]
\item<4->
\[
f(2)= f(0)+2f'(c)
\]
\item<5->  We are given that $f ' (x)\le 5$ for all $x$, so in particular we know that $f '(c)\le 5$, so
\item<6->
\[
f(2)  \le  f(0)+2(5)= -3+10=7 
\]
\end{itemize}
\end{example}
\end{frame}
% end module MVT-ex1
