% begin module MVT-proof
\begin{frame}[t]
\begin{theorem}[The Mean Value Theorem]
Let $f$ be a function that is \alertNoH{ 8}{continuous on $[a,b]$} and \alertNoH{ 9}{differentiable on $(a,b)$}.
Then there is a number $c$ in $(a,b)$ such that $\alertNoH{19}{f'(c) = \frac{f(b)-f(a)}{b-a}}$.
\end{theorem}

\begin{proof}
\begin{itemize}
\item<2->  Let $L$ be the secant line from $\alertNoH{ 5}{(a,f(a))}$ to $(b,f(b))$.
\item<3->  $L(x) =  \fcAnswerUncover{3}{ 6}{f(a)} + \fcAnswerUncover{3}{4}{\alertNoH{ 18}{ \frac{f(b)-f(a) }{b-a}}}(x-\fcAnswerUncover{3}{6}{a}).$
\item<7->  Consider the function $(f - L)(x) = f(x) - f(a) - \frac{f(b)-f(a)}{b-a}(x-a)$.
\item<8-| alert@9-10>  $L$ is linear, so it's continuous and differentiable everywhere.
\item<9->  \alertNoH{ 9}{$f-L$ is continuous on $[a,b]$} \uncover<10->{\alertNoH{ 10}{and differentiable on $(a,b)$.}}
\item<11-| alert@12-13>  $(f-L)(a) =$ \fcAnswerUncover{11 }{13}{$ f(a) - f(a) - \frac{f(b) -f(a )}{ b -a}(a-a) = 0$.}
\item<11-| alert@14-15>  $(f-L)(b) =$ \fcAnswerUncover{11 }{15}{$f(b) - f(a) - \frac{f(b)- f(a)}{b-a}(b-a) = 0$.}
\item<16->  Rolle's Theorem: There exists $c$ in $(a,b)$ such that
\abovedisplayskip=0pt
\belowdisplayskip=0pt
$
\uncover<16->{%
\alertNoH{19}{0} = (f-L)'(c) = f'(c) - \alertNoH{ 17-18}{L'(c)} } \alertNoH{19}{\uncover<17->{= f'(c) - } \fcAnswer{18}{ \frac{ f(b)-f(a)}{b-a}}}%
\uncover<19->{\qedhere} %
$
\end{itemize}
\end{proof}

\vspace{2cm} %guarantees top alignment of slide
\end{frame}
% end module MVT-proof
