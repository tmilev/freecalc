% begin module MVT-proof
\begin{frame}[t]
\begin{theorem}[The Mean Value Theorem]
Let $f$ be a function that is \alertNoH{ 8}{continuous on $[a,b]$} and \alertNoH{ 9}{differentiable on $(a,b)$}.
Then there is a number $c$ in $(a,b)$ such that $f'(c) = \frac{f(b)-f(a)}{b-a}$.
\end{theorem}

\begin{proof}
\begin{itemize}
\item<2->  Let $L$ be the secant line from $\alertNoH{ 5}{(a,f(a))}$ to $(b,f(b))$.
\item<2-| alert@3-4>  $L(x) =  \fcAnswerUncover{2}{ 5}{f(a)} \uncover<3->{+} \fcAnswerUncover{2}{4}{\alertNoH{ 19}{ \frac{f(b)-f(a) }{b-a}}}\uncover<3->{(x-\fcAnswer{5}{a}).}$
\item<6->  Consider the function $(f - L)(x) = f(x) - f(a) - \frac{f(b)-f(a)}{b-a}(x-a)$.
\item<7-| alert@8-9>  $L$ is linear, so it's continuous and differentiable everywhere.
\item<8->  \alertNoH{ 8}{$f-L$ is continuous on $[a,b]$} \uncover<9->{\alertNoH{ 9}{and differentiable on $(a,b)$.}}
\item<10-| alert@11-12>  $(f-L)(a) =$ \uncover<12->{$f(a) - f(a) - \frac{f(b)-f(a)}{b-a}(a-a) = 0$.}
\item<10-| alert@13-14>  $(f-L)(b) =$ \uncover<14->{$f(b) - f(a) - \frac{f(b)-f(a)}{b-a}(b-a) = 0$.}
\item<15->  Rolle's Theorem: There exists $c$ in $(a,b)$ such that
\abovedisplayskip=0pt
\belowdisplayskip=0pt
$
\uncover<15->{%
0 = (f-L)'(c) = \alertNoH{ 16-17}{f'(c)} - \alertNoH{ 18-19}{L'(c)} = %
}%
\uncover<17->{%
\alertNoH{ 17}{f'(c)}%
}%
\uncover<16->{%
 - %
}%
\uncover<19->{%
\alertNoH{ 19}{\frac{f(b)-f(a)}{b-a}}%
} \qedhere %
$
\end{itemize}
\end{proof}

\vspace{2cm} %guarantees top alignment of slide
\end{frame}
% end module MVT-proof
