% begin module closed-interval-method
\begin{frame}
\frametitle{The Closed Interval Method}

We know from the Extreme Value Theorem that a continuous function 
attains its absolute maximum and minimum on a closed interval $[a,b]$. 
The maximum might occur at an endpoint. The minimum might occur at an 
endpoint.

\uncover<2->{%
To find the absolute maximum and minimum values of a continuous function $f$ on a closed interval $[a,b]$:
\begin{enumerate}
\item  Find the values of $f$ at the critical numbers of $f$ in $[a,b]$.
\begin{itemize}
\item Find the values $c$ with $f'(c)=0$.
\item Find the values $c$ where $f'$ does not exist.
\end{itemize}
\item Find the values of $f$ at the endpoints $a$ and $b$.

\item The absolute maximum of $f$ is maximum of the preceding values; the absolute minimum value is the minimum.
\end{enumerate}
}%
\end{frame}
% end module closed-interval-method
