\solution{\ref{problemmaxminx^3-x+1over[-2,1]}
By the closed interval method the maximuma/minima over $[-2,1]$ are obtained either at the endpoints or at the critical points of $f(x)$. Since $f'(x)$ is defined over the entire interval, the only critical points are the ones for which  $f'(x)=0$.
\[
\begin{array}{rcl}
\displaystyle f'(x)&=&0\\
\displaystyle 3x^2-1&=&0\\
\displaystyle x^2&=&\displaystyle \frac{1}{3}\\
\displaystyle x&=&\displaystyle \pm\sqrt{\frac{1}{3}}=\pm\frac{\sqrt{3}}{3}
\end{array}
\]
The maximum and minimum of $f$ over $[-2,1]$ is attained at one of the points $x=-2, -\frac{\sqrt{3}}{3}, \frac{\sqrt{3}}{3}, 1$.

$\begin{array}{c|c}
x& f(x)\\\hline
-2&f(-2)=(-2)^3-(-2)+1=-5\\
-\frac{\sqrt{3}}{3}&f\left( -\frac{\sqrt{3}}{3}\right)= \left( -\frac{\sqrt{3 }}{3}\right)^3-\left( -\frac{\sqrt{3}}{3}\right)+1=\frac{2}{9}\sqrt{3}+1 \\
\frac{\sqrt{3}}{3}&f\left( \frac{\sqrt{3}}{3}\right)= \left( \frac{\sqrt{3}}{3}\right)^3-\left( \frac{\sqrt{3}}{3}\right)+1 =-\frac{2}{9}\sqrt{3}+1 \\
1&f(1)= \left( 1\right)^3-\left( 1\right)+1=1 \\
\end{array}
$

Observation of the table above shows that the maximum of $f$ over $[-2,1]$ is $f\left( -\frac{\sqrt{3}}{3}\right) =\frac{2}{9}\sqrt{3}+1$ attained for $x=-\frac{\sqrt{3}}{3}$ and the minimum is $f(-2)=-5$ (attained for $x=-2$).
}