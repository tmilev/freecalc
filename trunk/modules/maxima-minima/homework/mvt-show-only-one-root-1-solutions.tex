\textbf{Solution \ref{problemIVTandMVTx^3+4x+7}.}  $f(0) = -7$ and $f(2) = 9$. Since $f(x)$ is continuous and has both negative and positive outputs, it must have a zero. In other words, for some $c$ between $0$ and $2$, $f(c) = 0$. If there were solutions $x = a$ and $x = b$,  then we would have $f(a) = f(b)$, and Rolle's Theorem would guarantee that for some $x$-value, $f'(x) = 0$. However, $f'(x) = 3x^2 + 4$, which is never 0. Therefore there cannot be 2 or more solutions. 
(In other words: Note that $f'(x) = 3x^2 + 4$, which is always positive. Therefore, the graph of $f$ is increasing from left to right. So once the graph crosses the $x$-axis, it can never turn around and cross again, so there can only be a single zero (that is, a single solution to $f(x) = 0$).)


\textbf{Solution \ref{problemIVTandMVTcos3xdiv3+sinx-3x}.} $f(5)= \cos^3 \left(\frac{5}{3}\right) +\sin 5-15 \leq 2-15=-13<0 $ (because $\cos a, \sin b\in [-1,1]$ for arbitrary $a,b$). Similarly $f(-5)=\cos^3\left(-\frac{5}{3}\right) +\sin (-5)+15 \geq 15-2>0$. Therefore by the Intermediate Value Theorem $f(x)=0$ has at least one solution in the interval $[-5,5]$.

Suppose on the contrary to what we are trying to prove, $f(x)=0$ has two or more solutions; call the first 2 solutions $a,b$. That means that $f(a)=f(b)=0$, so by the Mean value theorem, there exists a $c\in (a,b)$ such that $f'(c)=(f(a)-f(b))/(a-b)=(0-0)/(a-b)=0$. On the other hand we may compute:
\[ 
f'(x)=-3+\cos x-\cos^{2}\left(\frac{x}3\right)\sin\left(\frac{x}{3}\right) \leq -1<0,
\] 
where the first inequality follows from the fact that $\sin x,\cos x\in [-1,1]$. So we got that $f'(c)=0$ for some $c$ but at the same time $f'(x)<0$ for all $x$, which is a contradiction. Therefore $f(x)=0$ has exactly one solution. 