\begin{frame}
\begin{itemize}
\item<1-> Previously we considered functions $\alert<2,6>{z}= g(\alert<3,5>{x,y})$  \alert<2>{with scalar output} and \alert<3>{two dimensional input}.
\item<4-> The graphs of such functions live in $\RR^3 = \RR^{\alert<5>{2} +\alert<6>{1}}$.
\item<5-> \alert<5>{2 dimensions} were used to represent the input.
\item<6-> \alert<6>{1 dimension} was used to represent the output.
\item<7-> To represent functions with 3 dimensional input (3 variables) and scalar output: need  $3+1=4$ dimensions.
\item<8-> That's difficult for eyes used to visualizing physical 3d- space.
\item<9-> Instead: label the level sets of the function with color or other means to indicate value.
\item<10-> In this way we represent the f-n graphically using dimension equal to the number of input variables.
\end{itemize}
\end{frame}
\begin{frame}
\frametitle{Example}
\begin{columns}
\column{0.4\textwidth}
\psset{xunit=0.4cm, yunit=0.4cm}
\begin{pspicture}(-4,-7)(6,10)
\renewcommand{\fcScreen}{[-1 1 -0.7] -1}
\fcBoundingBox{-4}{-8.5}{4}{10.5}
\tiny
\fcAxesIIIdFull{5}{5}{5}%
\uncover<5->{%
%d=4
\fcParallelogramIIId[linecolor=cyan!90]{[-2 -2 -8]}{[-2 3 -3]}{[5 -2 -1]}%
\fcPutIIId[l]{[2 2 0]}{$~~d=4$}%
\fcLineIIId{[0 0 0]}{[0 0 -4]}
\fcLineIIId{[0 0 0]}{[4 0 0]}
}%
\uncover<6->{%
%d=2
\fcParallelogramIIId[linecolor=cyan!70]{[-2 -2 -6]}{[-2 3 -1]}{[5 -2 1]}%
\fcPutIIId[l]{[2 2 2]}{$~~d=2$}%
\fcLineIIId{[0 0 0]}{[0 2 0]}
\fcLineIIId{[0 0 0]}{[2 0 0]}
\fcLineIIId{[0 0 0]}{[0 0 -2]}
}%
\uncover<7->{%
%d=0
\fcParallelogramIIId[linecolor=cyan!50]{[-2 -2 -4]}{[-2 3 1]}{[5 -2 3]}%
\fcPutIIId[l]{[2 2 4]}{$~~d=0$}%
\fcLineIIId{[0 0 0]}{[0 0 4]}
\fcLineIIId{[0 0 0]}{[-4 0 0]}
\fcLineIIId{[0 0 0]}{[0 -4 0]}
}%
\uncover<8->{%
%d=-2
\fcParallelogramIIId[linecolor=cyan!30]{[-2 -2 -2]}{[-2 3 3]}{[5 -2 5]}%
\fcPutIIId[l]{[2 2 6]}{$~~d=-2$}%
\fcLineIIId{[0 0 2]}{[0 0 4]}%
\fcLineIIId{[-2 0 0]}{[-4 0 0]}%
\fcLineIIId{[0 -2 0]}{[0 -5 0]}%
}%
\uncover<9->{%
%d=-4
\fcParallelogramIIId[linecolor=cyan!10]{[-2 -2 0]}{[-2 3 5]}{[5 -2 7]}%
\fcPutIIId[l]{[2 2 8]}{$~~d=-4$}
\fcPutIIId[l]{[2 2 6]}{$~~d=-2$}%
\fcLineIIId{[0 0 4]}{[0 0 5]}%
\fcLineIIId{[-4 0 0]}{[-5 0 0]}%
}%
\fcAxesIIIdFull[arrows=->, linestyle=dotted, dash=1pt]{5}{5}{5}
\end{pspicture}

\column{0.6\textwidth}
\begin{itemize}
\item<1-> Let $f(x,y,z) = x+y-z$.
\item<2-> The graph consists of the quadruples $(x,y,z,w)$ in $\RR^4$ such that $w = x+y-z$. Can't plot that graphically (yet).
\item<3-> However, can represent with labeled level sets.
\item<4-> The level set $f(x,y,z) = d$ is the surface $x+y-z = d$ in $\RR^3$, and that surface is a plane.
\item<5-> For varying values of $d$ we plot the level set. 
$f(x,y,z) = \only<5>{d=4}\only<6>{d=2} \only<7>{d=0}\only<8>{d=-2}
\only<9>{d=-4}$.

Darker color = larger $d$.
\end{itemize}
\end{columns}

\end{frame}

\begin{frame}
To understand surfaces in space we need the following.
\begin{remark} The level set $f(x,y,z)=0$ for the function
%
$$f(x,y,z) = ax+by-z+d$$
%
is the same as the graph of the function $g(x,y) = ax+by+c$.
\end{remark}
\begin{itemize}
\item<2-> Graph surfaces can always be represented as level surfaces
\item<3-> The converse is not true: level surfaces can't always be represented as graph surfaces. 
\item<4-> Example: a sphere centered at the origin is the
level surface of $f(x,y,z) = x^2+y^2+z^2$ but it ``fails the vertical line test in all directions'', so it cannot be globally represented as a graph surface, no matter how we change the coordinate system.
\item<5-> We'll show that under reasonable assumptions, level surfaces can \emph{locally} be described as graph surfaces. 
\end{itemize}
\end{frame}