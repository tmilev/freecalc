\begin{frame}
\frametitle{Limits along paths}
\begin{definition}
\begin{itemize}
\item<1-> Let $f: D\to \mathbb R$, where $D$ is a region in the plane;
\item<2-> let  $f$ be defined near point $P$ with position vector $\fcv p$.
\item<3-> Let $\fcv r(t)=(x(t), y(t))$, $t\in I$ be a continuous path such that:
\begin{itemize}
\item<3-> $0$ is in $I$, $\fcv{r}(0) = \fcv p$;
\item<4-> $\fcv{r}$ is continuous at $0$;
\item<5-> $\fcv{r}(t)$ lies in $D$ for $t\neq 0$
\end{itemize}
\end{itemize}
\uncover<6->{ 
We say that the one-variable limit
\[
\lim\limits_{t\to 0} f(\fcv{r}(t))
\]
is the limit of $f$ along the path $\fcv r(t)$.
}
\end{definition}
\end{frame}
\begin{frame}
\begin{theorem}
If the limit $\lim\limits_{Q\to P}f(Q) $ exists, then every path limit exists and all path limits are equal.
\end{theorem}
\begin{itemize}
\item<2-> If we pick our path to be of the form $\fcv{r}(t) = \fcv{r}_0+t\fcv{u}$, we see that the directional limit is a special case of the path limit.
\end{itemize}
\end{frame}