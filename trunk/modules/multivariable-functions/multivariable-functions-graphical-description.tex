\begin{frame}
\begin{itemize}
\item An analytical description is technically best, but not easy to interpret.
\item<2-> If output is a scalar, where does the function attain its extreme values (maxima, minima)?  
\item<3-> How do values change for nearby points - are they decreasing, increasing, how fast?
\item<4-> We will learn to decode this information from the analytical descriptions.
\item<5-> Even so, ``a picture is worth a thousand words''.
\end{itemize}
\end{frame}

\begin{frame}
\begin{center}
\psset{xunit=1cm,yunit=1cm}
\begin{pspicture}(-5.7,-3.5)(5.7,5.2)
\tiny
\renewcommand{\fcScreen}{[-1 -1 -0.4] 0}
\fcAxesIIId{5}{5}{5}
\fcPlotIIId[linewidth=0.3pt, linecolor=blue]{iterationsX=50, iterationsY=50}{-4}{-4}{4}{4}{ x 3 exp x y 4 exp mul add x 5 div sub 10 mul 2.729 x dup mul y dup mul add neg exp mul 2.729 x 1.225 sub dup mul y dup mul add neg exp add}
\rput[t](-3,4){Graph of a scalar function.}
\end{pspicture}
\end{center}

\end{frame}

\begin{frame}\frametitle{Graph of a function}
\begin{itemize}
\item For one variable function, $y=f(x)$, the graph of $f$ is a set of points in $\RR^2$: the set of points $(x,y)$ such that $y=f(x)$.
\item<2-> Example: if $f(x) = x^2$, then  $(3,9)$ is on the graph, because $9=3^2$, but $(2,5)$ is not because $5 \neq 2^2$.
\item<3-> We can extend this graphical representation for functions with two dimensional input and one dimensional (scalar) output. 
\item<4-> The \emph{graph} of the function $f\colon D \to \RR$, where $D$ is a region in $\RR^2$, is the set of points $P(x,y,z)$ in $\RR^3$ whose coordinates satisfy the condition 
\[
z=f(x,y)\quad .
\]
\item<5-> For example, the graph of $f(x,y) = 2x-y+3$ is the set
\[
\{ (x,y,z) \, | \, z= 2x-y+3\} \Longrightarrow \text{ plane } 2x-y-z+3=0 \; .
\]

\end{itemize}

 



\end{frame}
