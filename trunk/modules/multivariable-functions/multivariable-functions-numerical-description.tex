\begin{frame}
\frametitle{Numerical description}
\begin{itemize}
\item Verbal description is essential for understanding.
\item<2-> However this does not include quantitative or visual information.
\item<3-> A \emph{numerical} description gives output data for a relevant set of input data.
\item<4-> This facilitates construction/study of a mathematical model.
\item<5-> Numerical description is typically given by table.
\item<6-> This table contains numerical data collected through experiments at selected input levels. 
\end{itemize}
\end{frame}

\begin{frame}
\frametitle{Example: Describing Function Via Numerical Data}
\begin{itemize}
\item the following is Wind Chill Chart provided by NOOA. The table entries indicate the temperature felt on exposed skin under the corresponding wind speed and temperature.

\small
\begin{tabular}{|c@{~}c|@{~}c@{~}c@{~}c@{~}c@{~}c@{~}c@{~}c@{~}c@{~}c@{~}c@{~}c@{~}c@{~}c@{~}c@{~}c@{~}c@{~}c@{~}c@{}|}\hline
&&\multicolumn{18}{c|}{Temperature $^\circ$F}\\
\multirow{12}{0.2cm}{\begin{sideways}Wind (mph)\end{sideways}}    
&& 40 & 35 & 30 & 25 & 20 & 15   & 10  & 5   & 0   & -5  & -10 & -15 & -20 & -25 & -30 & -35 & -40 & -45 \\\hline
& 5  & 36 & 31 & 25 & 19 & 13 &  7   & 1   & -5  & -11 & -16 & -22 & -28 & -34 & -40 & -46 & -52 & -57 & -63 \\
& 10 & 34 & 27 & 21 & 15 & 9  &  3   & -4  & -10 & -16 & -22 & -28 & -35 & -41 & -47 & -53 & -59 & -66 & -72 \\
& 15 & 32 & 25 & 19 & 13 & 6  &  0   & -7  & -13 & -19 & -26 & -32 & -39 & -45 & -51 & -58 & -64 & -71 & -77 \\
& 20 & 30 & 24 & 17 & 11 & 4  & -2   & -9  & -15 & -22 & -29 & -35 & -42 & -48 & -55 & -61 & -68 & -74 & -81 \\
& 25 & 29 & 23 & 16 & 9  & 3  & -4   & -11 & -17 & -24 & -31 & -37 & -44 & -51 & -58 & -64 & -71 & -78 & -84 \\
& 30 & 28 & 22 & 15 & 8  & 1  & -5   & -12 & -19 & -26 & -33 & -39 & -46 & -53 & -60 & -67 & -73 & -80 & -87 \\
& 35 & 28 & 21 & 14 & 7  & 0  & -7   & -14 & -21 & -27 & -34 & -41 & -48 & -55 & -62 & -69 & -76 & -82 & -89 \\
& 40 & 27 & 20 & 13 & 6  & -1 & -8   & -15 & -22 & -29 & -36 & -43 & -50 & -57 & -64 & -71 & -78 & -84 & -91 \\
& 45 & 26 & 19 & 12 & 5  & -2 & -9   & -16 & -23 & -30 & -37 & -44 & -51 & -58 & -65 & -72 & -79 & -86 & -93 \\
& 50 & 26 & 19 & 12 & 4  & -3 & -10  & -17 & -24 & -31 & -38 & -45 & -52 & -60 & -67 & -74 & -81 & -88 & -95 \\
& 55 & 25 & 18 & 11 & 4  & -3 & -11  & -18 & -25 & -32 & -39 & -46 & -54 & -61 & -68 & -75 & -82 & -89 & -97 \\
& 60 & 25 & 17 & 10 & 3  & -4 & -11  & -19 & -26 & -33 & -40 & -48 & -55 & -62 & -69 & -76 & -84 & -91 & -98 \\\hline 
\end{tabular}
\end{itemize}
\end{frame}
\begin{frame}
\begin{itemize}
\item Another example is the Income Tax Table. Explain what the
input and output variables are in that case.
\end{itemize}
\end{frame}