\begin{frame}
\frametitle{A motivating example}
What does the graph $\Gamma$ of $g$ look like?
\[
g(x,y) =x^2+2y^2\quad .
\]

%\begin{pspicture}(-2,-2)(2,2)

%\end{pspicture}

By definition, the graph $\Gamma$ of $g$ is the set of points $P(x,y,z)$ in $\RR^3$ such that $z=x^2+2y^2$. This is no longer linear, so the set is not a plane. But what does it look like?

A method that provides significant insight is to use sections. The graph of $g$ lives in $\RR^3$ and we use an imaginary CT scan to cut it, understand the sections, and re-assemble the sections in order to get the graph.

What does it means to cut the graph by vertical planes parallel to the $Oyz-$plane? Such a plane has equation $x=a$, for some constant $a$. The intersection of such a plane with the graph $\Gamma$ is given by points of coordinates $(a, y, z=a^2+2y^2)$, and those points form a parabola in the plane $x=a$, with vertex at $(a,0,a^2)$. The vertex rises as $a$ gets away from 0. Intersecting the graph of $g$ with the plane $x=a$ means
essentially that we treat $x$ as a constant, and we study
the partial function $y\to z=g(a,y) = a^2+2y^2$.

Similarly, if we cut by planes $y=a$ we get partial
functions $x\to x^2+2a^2$, whose graphs are also
parabolas. Each such a parabola is now situated in the
vertical plane $y=a$. Hence vertical sections are
parabolas, with vertices rising as
we move farther away from the coordinate planes.

To get horizontal sections we have to
keep constant not an input variable, but the output
variable, $z$. Intersecting the graph of $g$ with
the plane $z=a$ we get a curve of equations $x^2+2y^2=a$,
$z=a$. For $a<0$ the intersection is empty, for $a=0$ it
consists of a single point, $(x,y,z) = (0,0,0)$, and for
$a>0$, the intersection is an ellipse in the plane $z=a$.

The \emph{level curve} corresponding to $z=a$ is the curve
$g(x,y)=a$ in the domain of $g$, with a label $a$ to
indicate the level of the output corresponding to
that level curve. Note that the level curve is not the
intersection of the graph of $g$ with the plane $z=a$;
instead, it is the projection of that intersection onto
the plane $Oxy$, where the domain of $g$ resides.

You should be familiarized with level curves if
you have ever seen a topographic map or if you
opened the newspaper at the weather section. What are the
functions in those cases?
\end{frame}


\subsection{Level sets}
The examples considered in the previous sections were all
examples of functions with scalar output and a two
dimensional input. Their graphs lived in
$\RR^3 = \RR^{2+1}$, with the 2 coming from the dimension
of the input and the 1 from the dimension of the output.
If we wanted to extend that graphic construction to
functions depending on three scalar variables,
we'd need $3+1=4$ dimensions for the graphs, and that
is momentarily impossible. The rescue comes from the
second graphical representation, using level sets,
because the level sets (with appropriate labeling of
the level) can be represented in a space with dimension
equal to the dimension of the input space.


For example, let $f(x,y,z) = ax+by-z$. The graph is the
quadruple $(x,y,z,w)$ in $\RR^4$ such that
$w = ax+by-z$, and we can't yet represent that
graphically. But the level set $f(x,y,z) = d$ is the
surface $ax+by-z = d$ in $\RR^3$, and that surface is a
plane. Level sets corresponding to equidistant values
$d_1$, $d_2$, and $d_3$ of the output are parallel
planes of equations
%
\begin{align*}
  \mathcal{P}_1 = f^{-1}(d_1) & \colon \; ax+by-z = d_1\\
  %
  \mathcal{P}_2 = f^{-1}(d_2) & \colon \; ax+by-z = d_2\\
  %
  \mathcal{P}_3 = f^{-1}(d_3) & \colon \; ax+by-z = d_3
\end{align*}
%
and these planes are also equidistant.

The following remark will play a very important role in
understanding surfaces in space.

\begin{remark}{\rm The level set $f(x,y,z)=0$ for the
function
%
$$f(x,y,z) = ax+by-z+d$$
%
is the same as the graph
of the function $g(x,y) = ax+by+c$.
}\end{remark}

We will see that while graph surfaces can always be
represented as level surfaces, the converse is not true:
level surfaces can't always be represented as graph
surfaces. However, we'll show that under some reasonable
assumptions, level surfaces can \emph{locally} be
described as graph surfaces. An illuminating example is
that of a sphere (centered at the origin): it is the
level surface of $f(x,y,z) = x^2+y^2+z^2$, but it can't
be globally represented as a graph surface. Try and see
what goes wrong!
