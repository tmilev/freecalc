%\begin{comment}
\begin{frame}
\frametitle{A motivating example}
$
\displaystyle g(x,y) =x^2+2y^2
$

\begin{itemize}
\item What does the graph $\Gamma$ of $g$ look like?
\item<2-> $\Gamma$ = points in $\RR^3$ such that $z=x^2+2y^2$. The set is not a plane: what does it look like?
\item<3-> To answer look at sections. Use imaginary CT scan to cut the graph; assemble resulting sections into a graph.
\end{itemize}

\vskip 10cm
\end{frame}

\begin{frame}
\frametitle{A motivating example}
$
g(x,y) =x^2+2y^2
$

\begin{columns}[T]
\column{0.4\textwidth}
\centering

\psset{xunit=0.75cm, yunit=0.75cm}
\begin{pspicture}(-3.4,-1.5)(3,4.3)
\tiny
\fcBoundingBox{-3.4}{-1.3}{3}{4.35}
\renewcommand{\fcScreen}{[-1 1 -0.5] -1}
\psline[linecolor=red!1](3, 4.3)(3.01,4.31)%
\psline[linecolor=red!1](-3.4, -1)(-3.41,-1.01)%
\uncover<2-5,9>{\fcParallelogramIIId{[1 -1.2 -0.5]}{[1 1.2 -0.5]}{[1 -1.2 4]}}%
\uncover<6>{\fcParallelogramIIId{[0 -1.2 -0.5]}{[0 1.2 -0.5]}{[0 -1.2 4]}} %
\uncover<7>{\fcParallelogramIIId{[0.333333 -1.2 -0.5]}{[0.333333 1.2 -0.5]}{[0.333333 -1.2 4]}} %
\uncover<8>{\fcParallelogramIIId{[0.666666 -1.2 -0.5]}{[0.666666 1.2 -0.5]}{[0.666666 -1.2 4]}} %
\fcAxesIIId{3}{3}{3}
\uncover<11->{\fcLineIIId[arrows=->, linecolor=black]{[0 0 0]}{[-3 0 0]}}
%parabola at x=1
\uncover<4-5,9->{\fcCurveIIId{-1.2}{1.2}{[1 t 2 t t mul mul 1 add]}} %
\uncover<5,9,10,12>{\fcDotIIId{[1 0 1]}} %
\uncover<2-5,9>{%
\fcDotIIId[linecolor=black]{[1 0 0]} %
\fcPutIIId[t]{[1 0 -0.3]}{$~~a$}%
} %
\uncover<5>{\fcPutIIId[bl]{[1 0 1.3]}{$~~~~\left(a,0,a^2\right)$}}%
%parabola at x=0
\uncover<6,10->{\fcCurveIIId{-1.2}{1.2}{[0 t 2 t t mul mul 0 add]}}
\uncover<6,10,12>{\fcDotIIId{[0 0 0]}}
\uncover<6>{ %
\fcDotIIId[linecolor=black]{[0 \space 0 0]}%
\fcPutIIId[t]{[0 0 -0.3]}{$~~a=0$}%
} %
%parabola at x=1/3
\uncover<7>{\fcCurveIIId{-1.2}{1.2}{[0.333333 t 2 t t mul mul 0.333333 dup mul add]}}
\uncover<7>{\fcDotIIId{[0.333333 0 0.333333 0.333333 mul]}}
\uncover<7>{ %
\fcDotIIId[linecolor=black]{[0.333333 0 0]}%
\fcPutIIId[t]{[0.333333 0 -0.3]}{$~~a$}%
}
%parabola at x=1/2
\uncover<10->{\fcCurveIIId{-1.2}{1.2}{[0.5 t 2 t t mul mul 0.5 dup mul add]}}
\uncover<10>{\fcDotIIId{[0.5 0 0.5 dup mul]}}
%parabola at x=2/3
\uncover<8>{\fcCurveIIId{-1.2}{1.2}{[0.666666 t 2 t t mul mul 0.666666 dup mul add]}}%
\uncover<8>{\fcDotIIId{[0.666666 0 0.666666 0.666666 mul]}}
\uncover<8>{ %
\fcDotIIId[linecolor=black]{[0.666666 0 0]}%
\fcPutIIId[t]{[0.666666  0 -0.3]}{$~~a$}%
} %
\uncover<11->{
%\fcCurveIIId{-1.2}{1.2}{[-0.333333 t 2 t t mul mul 0.333333 dup mul add]}
\fcCurveIIId{-1.2}{1.2}{[-0.5 t 2 t t mul mul 0.5 dup mul add]}
\fcCurveIIId{-1.2}{1.2}{[-1.0 t 2 t t mul mul 1.000000 dup mul add]}
}
\uncover<12>{
\fcCurveIIId[linecolor=blue]{-1.5}{1.5}{[t 0 t t mul]}
\fcDotIIId{[-0.5 0  -0.5 dup mul]}
\fcDotIIId{[0.5 0  0.5 dup mul]}
\fcDotIIId{[-1.000000 0  -1.000000 dup mul]}
}
\end{pspicture}

\uncover<2->{
\psset{xunit=0.4cm, yunit=0.4cm}
\begin{pspicture}(-1.6, -0.9)(1.6,4.3)
\tiny
\psframe*[linecolor=cyan!30](! -1.45 -0.75)(! 1.45 4.13)%
\psaxes[ticks=none, labels=none]{<->}(0,0)(-1.35,-0.65)(1.35,4.03)
\fcLabels[$y$][$z$]{1.35}{4.03}

\uncover<6>{
%Function formula: 2 x^{2}
\psplot[linecolor=\fcColorGraph, plotpoints=1000]{-1.2}{1.2}{x 2 exp 2 mul }
\fcFullDot{0}{0}
}

\uncover<7>{
%Function formula: 2 x^{2}+\frac{1}{9}
\psplot[linecolor=\fcColorGraph, plotpoints=1000]{-1.2}{1.2}{ 0.111111 x 2 exp 2 mul add }
\fcFullDot{0}{0.111111}
}
\uncover<8>{
%Function formula: 2 x^{2}+\frac{4}{9}
\psplot[linecolor=\fcColorGraph, plotpoints=1000]{-1.2}{1.2}{ 0.444444 x 2 exp 2 mul add }
\fcFullDot{0}{0.444444}
}
\uncover<3-5, 9->{ %
%Function formula: 2 x^{2}+1
\psplot[linecolor=\fcColorGraph, plotpoints=1000]{-1.2}{1.2}{ 1 x 2 exp 2 mul add } %
} %
\uncover<5, 9->{\fcFullDot{0}{1}} %
\end{pspicture}

The plane $x=a$.

\vskip 3cm %for alignment
} % uncover end

\column{0.6\textwidth}
\begin{itemize}
\item<2-> Cut by vertical planes $x=a$, $a$-constant, parallel to the $Oyz-$plane.
\item<3-> In other words, treat $x$ as constant and study the f-n $y\to z= a^2+2y^2=g(a,y) $.

\item<4-> The cross-sections are the curves:
$
\displaystyle \{(a, y, z)\quad \text{where } z=a^2+2y^2\}
$
\item<5-> These are parabolas lying inside the plane $x=a$ with vertices at $\left(a,0,a^2\right)$.

\item<6-> As $a$ moves away from $0$, the parabola vertex rises.
\item<12-> The vertices traverse the curve given by $\{ (a, 0, a^2) \}$.
\end{itemize}
\end{columns}

\vskip 10 cm
\end{frame}
%\end{comment}

%\begin{comment}
\begin{frame}
\frametitle{A motivating example}
$
g(x,y) =x^2+2y^2
$

\begin{columns}[T]
\column{0.4\textwidth}
\centering
\psset{xunit=0.75cm, yunit=0.75cm}
\begin{pspicture}(-3.4,-1.5)(3,4.3)
\tiny
\fcBoundingBox{-3.4}{-1.3}{3}{4.35}
\renewcommand{\fcScreen}{[-1 1 -0.5] -1}
\psline[linecolor=red!1](3, 4.3)(3.01,4.31)%
\psline[linecolor=red!1](-3.4, -1)(-3.41,-1.01)%
%parabola at y=1
\uncover<1-3,7>{\fcParallelogramIIId{[-1  1 -0.5]}{[1 1 -0.5]}{[-1 1 4]}}%
\uncover<1-3,7>{ %
\fcDotIIId[linecolor=black]{[0 1 0]}%
\fcPutIIId[t]{[0 1 -0.3]}{$~~a$}%
} %
\uncover<3,7->{\fcCurveIIId{-1}{1}{[t 1 2 t t mul mul 1 dup mul 2 mul add]}} %

%parabola at y=0
\uncover<4>{\fcParallelogramIIId{[-1  0 -0.5]}{[1 0 -0.5]}{[-1 0 4]}}%
\uncover<4>{ %
\fcDotIIId[linecolor=black]{[0 0 0]}%
\fcPutIIId[t]{[0 0 -0.3]}{$~~a$}%
} %
\uncover<4,8->{\fcCurveIIId{-1}{1}{[t 0 2 t t mul mul 0 dup mul 2 mul add]}} %

%parabola at y=1/3
\uncover<5>{\fcParallelogramIIId{[-1  0.333333 -0.5]}{[1 0.333333 -0.5]}{[-1 0.333333 4]}}%
\uncover<5>{%
\fcDotIIId[linecolor=black]{[0 0.333333 0]}%
\fcPutIIId[t]{[0 0.333333 -0.3]}{$~~a$}%
} %
\uncover<5>{\fcCurveIIId{-1}{1}{[t 0.333333 2 t t mul mul 0.333333 dup mul 2 mul add]}} %

%parabola at y=2/3
\uncover<6>{\fcParallelogramIIId{[-1  0.666666 -0.5]}{[1 0.666666 -0.5]}{[-1 0.666666 4]}}%
\uncover<6>{ %
\fcDotIIId[linecolor=black]{[0 0.666666 0]}%
\fcPutIIId[t]{[0 0.666666 -0.3]}{$~~a$}%
} %
\uncover<6,8->{\fcCurveIIId{-1}{1}{[t 0.666666 2 t t mul mul 0.666666 dup mul 2 mul add]}} %

\fcAxesIIId{3}{3}{3}%
\uncover<9->{%
\fcLineIIId[arrows=->]{[0 0 0]}{[0 -3 0]} %
\fcCurveIIId{-1}{1}{[t -0.666666 2 t t mul mul 0.666666 dup mul 2 mul add]} %
\fcCurveIIId{-1}{1}{[t -1 2 t t mul mul 1 dup mul 2 mul add]}%
}%
\uncover<10->{%
\fcCurveIIId[linecolor=blue]{-1.2}{1.2}{[0 t 2 t t mul mul]}%
\fcDotIIId{[0 0 0]}%
\fcDotIIId{[0 -0.5 0.5]}%
\fcDotIIId{[0 0.5 0.5]}%
\fcDotIIId{[0 1 2]}%
\fcDotIIId{[0 -1 2]}%
}
\end{pspicture}

\uncover<1->{
\psset{xunit=0.4cm, yunit=0.4cm}
\begin{pspicture}(-1.6, -0.9)(1.6,4.3)
\tiny
\psframe*[linecolor=cyan!30](! -1.45 -0.75)(! 1.45 4.13)%
\psaxes[ticks=none, labels=none]{<->}(0,0)(-1.35,-0.65)(1.35,4.03)
\fcLabels[$x$][$z$]{1.35}{4.03}
\uncover<4>{
%Function formula: 2 x^{2}
\psplot[linecolor=\fcColorGraph, plotpoints=1000]{-1}{1}{x 2 exp 2 mul }
}%
\uncover<5>{%
%Function formula: 2 x^{2}+\frac{2}{9}
\psplot[linecolor=\fcColorGraph, plotpoints=1000]{-1}{1}{ 2 9 div x 2 exp 2 mul add }%
}%
\uncover<6>{%
%Function formula: 2 x^{2}+2*\frac{4}{9}
\psplot[linecolor=\fcColorGraph, plotpoints=1000]{-1}{1}{ 8 9 div x 2 exp 2 mul add }%
}%
\uncover<3, 7->{%
%Function formula: 2 x^{2}+1
\psplot[linecolor=\fcColorGraph, plotpoints=1000]{-1}{1}{ 2 x 2 exp 2 mul add }%
}%
\end{pspicture}

The plane $x=a$.
%\vskip 3cm %for alignment
} % uncover end

\column{0.6\textwidth}
\begin{itemize}
\item<1-> Similarly, cut by vertical planes $y=a$, i.e., planes parallel to the $Oxz$ plane.

\item<2-> In other words, treat $y$ as constant and study the f-n $z=g(x,a) = x^2+2a^2$.

\item<3-> The cross-sections are the curves  $\{ (x,a,z) \text{  where  } z= x^2+2a^2\}$. These are parabolas lying inside the plane $y=a$.
\item<8-> $\Rightarrow$ the vertical sections along both the $x$ and $y$ axes are parabolas.
\item<10-> The vertices are rising as we move away from the origin.
\end{itemize}
\vskip 2.4cm
\end{columns}

\vskip 10 cm
\end{frame}
%\end{comment}


%\begin{comment}
\begin{frame}

\frametitle{A motivating example}
$
g(x,y) =x^2+2y^2
$

\begin{columns}
\column{0.4\textwidth}
\centering
\psset{xunit=0.65cm, yunit=0.65cm}
\begin{pspicture}(-3.5,-2.2)(3.5,4)
\tiny
\fcBoundingBox{-3.5}{-1.3}{3.5}{4.3}
\renewcommand{\fcScreen}{[-1 1 -0.5] -1}
\psline[linecolor=red!1](3, 4)(3.01,4.01)%
\psline[linecolor=red!1](-3, -2.2)(-3.01,-2.21)%
\uncover<1-2>{ %
\fcParallelogramIIId{[-2.5  -2.5 1]}{[2.5 -2.5 1]}{[-2.5 2.5 1]}
\fcDotIIId[linecolor=black]{[0 0 1]}%
\fcPutIIId[tl]{[0 0 1]}{$~~a$}%
} %
\uncover<3>{ %
\fcParallelogramIIId{[-2.5  -2.5 -1]}{[2.5 -2.5 -1]}{[-2.5 2.5 -1]}
\fcDotIIId[linecolor=black]{[0 0 -1]}%
\fcPutIIId[tl]{[0 0 -1]}{$~~a$}%
} %
\uncover<4>{ %
\fcParallelogramIIId{[-2.5  -2.5 0]}{[2.5 -2.5 0]}{[-2.5 2.5 0]}
} %
\uncover<5>{ %
\fcParallelogramIIId{[-2.5  -2.5 0.5]}{[2.5 -2.5 0.5]}{[-2.5 2.5 0.5]}
} %
\uncover<6>{ %
\fcParallelogramIIId{[-2.5  -2.5 1]}{[2.5 -2.5 1]}{[-2.5 2.5 1]}
} %
\uncover<7>{ %
\fcParallelogramIIId{[-2.5  -2.5 1.5]}{[2.5 -2.5 1.5]}{[-2.5 2.5 1.5]}
} %
\uncover<8>{ %
\fcParallelogramIIId{[-2.5  -2.5 2]}{[2.5 -2.5 2]}{[-2.5 2.5 2]}
} %

\fcAxesIIId{3}{3}{3}%
\fcLineIIId[arrows=->]{[0 0 0]}{[0 -3 0]} %
\fcLineIIId[arrows=->]{[0 0 0]}{[-3 0 0]} %
\fcLineIIId[arrows=->]{[0 0 0]}{[0 0 -1.5]} %

\uncover<4>{\fcDotIIId{[0 0 0]}}
\uncover<5,9->{ %
\fcCurveIIId{ 0}{6.2832}{[t 360 mul cos 0.500 sqrt mul t 360 mul sin 0.500 2 div sqrt mul 0.5]}
}
\uncover<5>{\fcDotIIId[linecolor=black]{[0 0 0.5]}}
\uncover<6,9->{ %
\fcCurveIIId{ 0}{6.2832}{[t 360 mul cos 1 sqrt mul t 360 mul sin 1 2 div sqrt mul 1]}
}
\uncover<6>{\fcDotIIId[linecolor=black]{[0 0 1]}}
\uncover<7,9->{ %
\fcCurveIIId{ 0}{6.2832}{[t 360 mul cos 1.5 sqrt mul t 360 mul sin 1.5 2 div sqrt mul 1.5]}
}
\uncover<7>{\fcDotIIId[linecolor=black]{[0 0 1.5]}}
\uncover<8,9->{ %
\fcCurveIIId{ 0}{6.2832}{[t 360 mul cos 2.0 sqrt mul t 360 mul sin 2.0 2 div sqrt mul 2.0]}
}
\uncover<8>{\fcDotIIId[linecolor=black]{[0 0 2.0]}}

\uncover<10->{
\fcCurveIIId[linewidth=1pt, linecolor=blue]{-1.414214 }{1.414214}{[t 0 t t mul]}
\fcCurveIIId[linewidth=1pt, linecolor=blue]{-1}{1}{[0 t t t 2 mul mul]}
}
\uncover<11->{
\fcCurveIIId[linewidth=2pt, linecolor=\fcColorGraph]{ 0}{6.2832}{[t 360 mul cos 0.500 sqrt mul t 360 mul sin 0.500 2 div sqrt mul 0.5]}
\fcCurveIIId[linewidth=2pt, linecolor=\fcColorGraph]{ 0}{6.2832}{[t 360 mul cos 1 sqrt mul t 360 mul sin 1 2 div sqrt mul 1]}
\fcCurveIIId[linewidth=2pt, linecolor=\fcColorGraph]{ 0}{6.2832}{[t 360 mul cos 1.5 sqrt mul t 360 mul sin 1.5 2 div sqrt mul 1.5]}
\fcCurveIIId[linewidth=2pt, linecolor=\fcColorGraph]{ 0}{6.2832}{[t 360 mul cos 2.0 sqrt mul t 360 mul sin 2.0 2 div sqrt mul 2.0]}
}
\end{pspicture}

\uncover<2->{
\psset{xunit=0.4cm, yunit=0.4cm}
\begin{pspicture}(-3.228413, -1.814212)(3.228427,1.914212)
\tiny
\psframe*[linecolor=cyan!30](! -3.078413 -1.664212)(! 3.078427 1.664212)%
\psaxes[ticks=none, labels=none]{<->}(0,0)(-2.978413,-1.564212) (2.978427,1.564212)
\fcLabels[$x$][$y$]{2.978427}{1.564212}
\uncover<4>{
\fcFullDot{0}{0}
}
\uncover<5>{
%Calculator input: plotCurve{}\left(\frac{\sqrt{2}}{2} \cos{}t, \frac{1}{2} \sin{}t, 0, 2 \pi\right)
\parametricplot[linecolor=\fcColorGraph, plotpoints=1000]{0}{6.28319}{t 57.29578 mul cos 0.707107 mul t 57.29578 mul sin 0.5 mul }
}
\uncover<6>{
%Calculator input: plotCurve{}\left(\cos{}t, \frac{\sqrt{2}}{2} \sin{}t, 0, 2 \pi\right)
\parametricplot[linecolor=\fcColorGraph, plotpoints=1000]{0}{6.28319}{t 57.29578 mul cos t 57.29578 mul sin 0.707107 mul }
}
\uncover<7>{
%Calculator input: plotCurve{}\left(\frac{\sqrt{2}\sqrt{3}}{2} \cos{}t, \frac{\sqrt{3}}{2} \sin{}t, 0, 2 \pi\right)
\parametricplot[linecolor=\fcColorGraph, plotpoints=1000]{0}{6.28319}{t 57.29578 mul cos 1.224745 mul t 57.29578 mul sin 0.866025 mul }
}
\uncover<8->{
%Calculator input: plotCurve{}\left(\sqrt{2} \cos{}t, \sin{}t, 0, 2 \pi\right)
\parametricplot[linecolor=\fcColorGraph, plotpoints=1000]{0}{6.28319}{t 57.29578 mul cos 1.414214 mul t 57.29578 mul sin }
}
\end{pspicture}

$x^2+2y^2= \only<1-3,5->{a} \only<4>{\alert<4>{0}}$, \only<1-3>{\alert<5>{$a<0$}}\only<5->{\alert<5>{$a>0$}}.
}

\column{0.6\textwidth}
\begin{itemize}
\item<1-> For horizontal sections keep constant the output variable, $z=a$.
\item<2-> When we intersect with $z=a$ we get the curve with equations $x^2+2y^2=a$, $z=a$.
\item<3-> For $a<0$ intersection is empty.
\item<4-> For $a=0$ intersection is $(0,0,0)$.
\item<5-> For $a>0$ intersection is an ellipse.
\item<10-> Figure is called ellipsoidal paraboloid.
\end{itemize}
\uncover<11->{
\begin{definition}
The sets $\{ (x,y,a)| g(x,y)=a\}$ are called \alert<11>{level curves} of the function $g$.
\end{definition}
}
\end{columns}

\vskip 10 cm
\end{frame}
\begin{frame}

\begin{itemize}
\item
You should be familiarized with level curves if you have ever seen a topographic map or from weather reports on the tv.
\item What are the functions in those cases?
\end{itemize}
\end{frame}
%\end{comment}
