\begin{frame}
\frametitle{Vector fields}
\begin{itemize}
\item \emph{Vector fields} are functions with multidimensional input and output. 
\item Input is point in space; output is a vector, which we plot as a vector with a tail at the input point. 
\item Examples
\begin{itemize}
  \item Velocity of fluid/air at given point;
  \item Electric force per unit of charge;
  \item Gravitational field;
  %\item Fundamental directions $\textbf{e}_\rho$, $\textbf{e}_\phi$, $\textbf{e}_\theta$, $\textbf{e}_r$
\end{itemize}
\end{itemize}
\end{frame}

\begin{frame}
\frametitle{Coordinate representation of vector fields}

\begin{itemize}
\item<1-> In rectangular coordinates a vector field $\textbf{F}$ can
be decomposed along the fundamental directions:
$$
\textbf{F}(x,y,z) = F_1(x,y,z) \textbf{i} + F_2(x,y,z) \textbf{j} + F_3(x,y,z) \textbf{k} \; .
$$
\item<2-> For regions in the plane 2-dim vector fields are defined in a similar fashion: as function from subsets of $\mathbb R^2$ to $\RR$:
$$
\textbf{F}(x,y) = F_1(x,y) \textbf{i} + F_2(x,y) \textbf{j}
$$


\end{itemize}
\end{frame}
\begin{frame}
\begin{itemize}
\item<1-> Example: define the vector field $\textbf{e}_r$ on $\mathbb{R}^2 \setminus \{ (0,0)\}$ via
$
\textbf{e}_r = \cos\theta\, \textbf{i} +
\sin\theta \,\textbf{j} = \frac{x}{r} \, \textbf{i} +
\frac{y}{r} \, \textbf{j} = \frac{x}{\sqrt{x^2+y^2}} \,
\textbf{i} + \frac{y}{\sqrt{x^2+y^2}} \, \textbf{j}
$

\item<2-> Similarly define the vector field $\textbf{e}_{\theta}$ by:
$\textbf{e}_\theta = -\sin\theta\, \textbf{i} +
\cos\theta \,\textbf{j} = -\frac{y}{r} \, \textbf{i} +
\frac{x}{r} \, \textbf{j} = -\frac{y}{\sqrt{x^2+y^2}} \,
\textbf{i} + \frac{x}{\sqrt{x^2+y^2}} \, \textbf{j} \; .
$
\uncover<3->{
\psset{xunit=0.3cm, yunit=0.3cm}
\begin{pspicture}(-5.2,-5.2)(5.2,5.2)
\tiny
\fcAxesStandard{-5.1}{-5.1}{5.1}{5.1}
\fcVectorField[linecolor=blue, arrows=->]{-5}{-5}{11}{11}{1} {1 dict begin /r x x mul y y mul add sqrt def  r 0 eq {0 0} {x r div  y r div} ifelse end}
\end{pspicture}
\psset{xunit=0.3cm, yunit=0.3cm}
~~~ \begin{pspicture}(-5.2,-5.2)(5.2,5.2)
\tiny
\fcAxesStandard{-5.1}{-5.1}{5.1}{5.1}
\fcVectorField[linecolor=blue, arrows=->]{-5}{-5}{11}{11 }{1}{1 dict begin /r x x mul y y mul add sqrt def  r 0 eq {0 0} {-1 y mul r div  x r div} ifelse end}
\end{pspicture}
}
\item<4-> From the picture it is evident what trajectory would be followed by an object that ``\alert<5>{flows along the vector field}''.
\item<5-> By ``\alert<5>{flowing}'' we mean an object whose \alert<5>{velocity} at each point is given by \alert<5>{the value of the field}.
\end{itemize}
\end{frame}
