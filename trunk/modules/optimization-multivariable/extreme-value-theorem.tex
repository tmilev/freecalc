\begin{frame}
  \frametitle{Extreme Value Theorem}

Global extreme points are guaranteed to exist if:

\begin{itemize}
  \item $f \colon D \to \mathbb{R}$ is continuous, and
  \item the domain $D$ has the following properties:
  \begin{itemize}
    \item $D$ is \emph{bounded}: The points in $D$ don't go farther than a certain fixed, finite distance from a fixed point.
    \item $D$ is \emph{closed}: $D$ contains all its boundary points.
  \end{itemize}
\end{itemize}
%

The statement above is the \textbf{Extreme Value Theorem}.

\begin{itemize}
  \item \pause Why does $D$ have to be bounded: \pause to exclude $f\colon \mathbb{R}^2 \to \mathbb{R}$, $f(x,y) = x$;
  \item \pause Why does $D$ have to be closed: \pause to exclude $f\colon \mathbb{R}^2\setminus\{(0,0)\} \to \mathbb{R}$, $f(x,y) = (x^2+y^2)^{-1}$. In this situation the boundary of $D$ is $\{(0,0)\}$ and is not included in $D$, so $D$ is not closed.
\end{itemize}
\end{frame}