\begin{frame}
\begin{definition}
The \emph{gradient vector} of $f$ at $P$ is the unique vector such that:
\begin{itemize}
\item its magnitude equals  the maximal rate of increase of $f$ at $P$;
\item if magn. $\neq 0$, its \alert<2>{direction is the one in which $f$ increases fastest}.
\end{itemize}
\end{definition}
\begin{itemize}
\item<3-> Recall that $\nabla f=\langle f_x, f_y, f_z\rangle$.
\item<2-> \alert<2>{ The increase of $f$} in \alert<5>{unit direction $\fcv u$} \alert<2>{is $D_{\fcv u} f $}. \uncover<3->{ We have:

\hfil $(D_{\fcv{u}}f)= (\nabla f) \cdot \fcv{u} \uncover<4->{= |\nabla f| \cdot  \alert<5>{| \fcv{u}|} \cos \alpha} \uncover<5->{ = |\nabla f| \alert<6>{\cos\alpha}\; ,}
$

\uncover<4->{where $\alpha$ is the angle between $\nabla f$ and $\fcv u$.}
} 
\item<6-> If $|\nabla f| \neq 0$, then $(D_{\fcv{u}}f)$ is maximal when $\alert<6>{\cos \alpha=1}$, i.e., $\alpha = 0$.
\item<7-> Therefore the maximum of $D_{\fcv u}f$ is achieved for $\fcv{u} =\frac{\nabla f}{|\nabla f|}$.
\item<8-> The maximum of $D_{\fcv u}f$ is then $|\nabla f| = |\langle f_x, f_y , f_z\rangle|$.
\uncover<9->{
\begin{theorem}[Coordinate Computation of gradient vector]
The gradient vector of $f$ equals $\nabla f=\langle f_x, f_y , f_z\rangle $.
\end{theorem}
}
\item<10-> In view of preceding thm., the gradient of $f$ is denoted by $\nabla f$.
\end{itemize}
\end{frame}