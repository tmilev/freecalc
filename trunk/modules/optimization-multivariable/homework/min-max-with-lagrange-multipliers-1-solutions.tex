\solution{\ref{problemFindExtremay+xUnderRestrictiony^2+y+x^2+x=1}
The restriction is $g(x,y)= y^2+y+x^2+x-1=0$. We use the method of Lagrange multipliers. We have that $\nabla f= (1,1)$ and $\nabla g = \left(2y+1,2x+1 \right)$. We have a local extremum when $\lambda\nabla f= \nabla g$, i.e., when
\[
\begin{array}{rcl}
\lambda &=& (2y+1)\\
\lambda &=& (2x+1)\\
y^2+y+x^2+x-1&=&0
\end{array}
\]
The first two equations imply $y=x$. We substitute that into the last equation to get that $2x^2+2x-1=0$. The solutions to the latter are $x=  \frac{ -2\pm \sqrt{2^2-4\cdot 2\cdot(-1)}}{4}= \frac{-1\pm \sqrt{3}}{2}$. The only restriction on $(x,y)$ is that they lie on the curve $y^2 + y+ x^2 +x=1$ (a circle). A circle is a bounded and closed set in space. Therefore $f$ must attain both its minimum and its maximum on it. Therefore the two critical points are maximum and minimum of $f$. Substitution of our answer in $f$ shows that $f$ attains its minimum at $\left(x,y\right)=\left(\frac{-1-\sqrt{3}}{2},\frac{-1-\sqrt{3}}{2}\right)$ and its maximum at $\left(x,y\right)=\left(\frac{-1+\sqrt{3}}{2} , \frac{ -1+\sqrt{3}}{2}\right)$. Our final answer is below.

\begin{tabular}{r|l}
$(x,y)$ & max or min\\\hline
$\left(\frac{-1-\sqrt{3}}{2},\frac{-1-\sqrt{3}}{2}\right) $ & minimum\\
$\left(\frac{-1+\sqrt{3}}{2} , \frac{ -1+\sqrt{3}}{2}\right)$ & maximum\\
\end{tabular}
}