\begin{frame}
\vskip -0.1cm
\begin{example}
A cone is folded from a wedge-shaped profile of radius $r$. Find the maximal possible volume $V$ of such a cone.

\begin{columns}[T]
\column{.27\textwidth}
%Please note: the below pictures are up to scale. The cone is an isogonal projection onto the tangent plane to the sphere centered at the origin and passing through point (0, 2, 0.4). The function used to draw the cone is
%the vector partition calculator function plotConeUsualProjection(3/4, \sqrt(1-DoubleValue{}(3/4)^2), 2, 0.4)
\psset{xunit=1.5cm, yunit=1.5cm}
\begin{pspicture}(-1.1, -1.05)(1.1,1)
\psframe*[linecolor=white](-1.1, -1.05)(1.1,1)
\tiny
\pscustom*[linecolor=cyan!30]{ \psparametricplot[algebraic] {2.35619}{7.06858} {0+1*cos(t)| 0+1*sin(t)} \psline(0.707107, 0.707107)(0, 0)(-0.707107, 0.707107)}

\psparametricplot[algebraic,linecolor=blue]{2.35619}{7.06858}{cos(t)| sin(t)}
\psline[linecolor=red](0.707107, 0.707107)(0, 0)(-0.707107, 0.707107)

\rput[t](0.35, 0.30){$r$}
\rput[lb](0.8,0.8){$B$}
\rput[rb](-0.8,0.8){$A$}
\rput[b](0,0.1){$O$}

\uncover<7->{
\psparametricplot[algebraic,linecolor=purple]{2.35619}{7.06858}{0.07*cos(t)| 0.07*sin(t)}
\rput[t](0, -0.1){$\alertNoH{7}{\theta}$}
}
\uncover<8->{
\psparametricplot[linewidth=1pt, algebraic,arrows=<->, linecolor =blue] {4.41238898} {5.01238898}{cos(t)| sin(t)}
}
\uncover<9->{
\rput[b](0, -0.9){\alertNoH{9}{$r\theta$}}
}
\psline[linecolor=red!1](-1.03,-1.03)(-1.03,-1.02)
\end{pspicture}

\psset{xunit=1.5cm, yunit=1.5cm}
\begin{pspicture}(-1.1, -0.2)(1.1,1)
\psframe*[linecolor=white](-1.1, -0.2)(1.1,1)
\tiny
\rput[b](0,0.7){$O$}
\rput[l](0.8, 0.0333562){$A$}

\psline[linecolor=\fcColorGraph](0.73046, 0.0333562)(0, 0.648593)
\psline[linecolor=\fcColorGraph](-0.73046, 0.0333562)(0, 0.648593)
\psparametricplot[algebraic,linecolor=blue]{-3.37036}{0.228769}{0.75*cos(t) |0.147087*sin(t)}
\psparametricplot[algebraic, linestyle=dashed, linecolor=blue] {0.228769}{2.91282}{0.75*cos(t) |0.147087*sin(t)}
\rput[bl](0.38,0.34 ){$r$}

\uncover<10,11>{
\psparametricplot[algebraic,linecolor=red]{-3.37036}{0.228769}{0.75*cos(t) |0.147087*sin(t)}
\psparametricplot[algebraic, linestyle=dashed, linecolor=red] {0.228769}{2.91282}{0.75*cos(t) |0.147087*sin(t)}
\rput[bl](0.38,0.34 ){$r$}
}

\psline[linecolor=red!1](-1, 0)(-0.99,0)
\psline[linecolor=red!1](0.99, 0)(1,0)
\uncover<2->{
\psline[linecolor=black](0,0)(0,0.648593)
\rput[r](-0.07,0.32){$h$}
}
\uncover<3->{
\psline[linecolor=black](0,0)(0.73046, 0.0333562)
\psline[linecolor=black](0.073046, 0.00333562) (0.073046, 0.0764577) (0, 0.0731221)
\rput[t](0.35, -0.02){$t$}
}
\end{pspicture}

\psset{xunit=1.5cm, yunit=1.5cm}
\begin{pspicture}(-1.1, -0.2)(1.2,1)
\psframe*[linecolor=white](-1.1, -0.2)(1.2,1)
\tiny
\uncover<13->{
\rput[b](0,0.7){$O$}
\rput[l](0.8, 0){$A$}
\rput[r](-0.05,0.32){\alertNoH{13,14}{$h$}}
\rput[t](0.38,-0.05){$\alertNoH{14}{t}$}
\rput[bl](0.38,0.34 ){$\alertNoH{14}{r}$}
\psline(0,0)(0.75,0)
\psline[linecolor=\fcColorGraph](0.75,0)(0, 0.648593)
\psline(0, 0.648593)(0,0)
\psline(0.075, 0)(0.075, 0.075)(0, 0.075)
}
\psline[linecolor=red!1](-1, -0.147087)(-0.99,-0.147087)
\psline[linecolor=red!1](0.99, 0.8)(1,0.8)
\end{pspicture}

\vspace{1cm}
\column{.73\textwidth}
\only<handout:1|1-20>{
\uncover<2->{
Set $h$ - cone height,} \uncover<3->{$t$ - cone radius.} \uncover<4->{Then $\alertNoH{4,17}{V=} \uncover<5->{\alertNoH{5}{\frac{1}3 (\alertNoH{6}{\text{area cone base}})h} }\uncover<6->{=\alertNoH{17}{ \frac13 \alertNoH{6}{\pi   t^2} h }}$.} \uncover<7->{ Let $\alertNoH{7}{\theta}$ - angle of the wedge.} \uncover<8->{Then $ \alertNoH{8,9}{\text{arc}{AB}=} \uncover<9->{\alertNoH{9,12}{r\theta}}$ \uncover<10->{\alertNoH{10,11}{= perimeter cone base =}} $\uncover<11->{\alertNoH{11,12}{2\pi t}.}$} \uncover<12->{Therefore $\alertNoH{12,15,18}{t=\frac{r\theta}{2\pi}}$.} \uncover<13->{Then

$\displaystyle
\alertNoH{13,14,19}{h=}  \uncover<14->{ \alertNoH{14}{ \sqrt{ r^2- \alertNoH{15}{t}^2 }}} \uncover<15->{= \sqrt{r^2- \left(\alertNoH{15}{ \frac{r\theta}{ 2\pi}}\right)^2}}\uncover<16->{=\alertNoH{19}{\frac{r}{2\pi }\sqrt{ 4\pi^2-\theta^2 }},}
$
}%13

\uncover<handout:2|17->{
and therefore

$
\begin{array}{rcl}
\alertNoH{17}{V}&\alertNoH{17}{=}&\displaystyle\alertNoH{17}{ \frac13\pi \alertNoH{18}{t}^2 \alertNoH{19}{h}}= \uncover<18->{\frac13\pi \left(\alertNoH{18}{\frac{r\theta}{2\pi}}\right)^2\alertNoH{19}{\frac{r}{2\pi}\sqrt{4\pi^2-\theta^2}}}\\
\uncover<20->{&=& \displaystyle \frac{r^3}{24\pi^2} \theta^2\sqrt{4\pi^2-\theta^2}\quad . }
\end{array}
$
}
}
\only<handout:3|21-24>{
\noindent We reduced the problem to: find the maximum of

$
V=\displaystyle \frac{r^3}{24\pi^2} \theta^2\sqrt{4\pi^2-\theta^2},\quad \quad  \uncover<22->{\alertNoH{22,23}{\uncover<23->{0} \leq \theta \leq \uncover<23->{2\pi}}}
$

as function of $\theta$ (using the \alertNoH{22, 23}{closed interval} method).

\uncover<24->{We need to find the critical points of $V$, i.e., the values of $\theta$ for which $\frac{\diff V}{\diff\theta}=0$ and the values of $\theta$ for which  $\frac{\diff V}{\diff \theta}$ is not defined.}
}
\only<handout:4|25-35>{
\noindent $
\begin{array}{rcl}
V&=&\displaystyle \frac{r^3}{24\pi^2} \theta^2\sqrt{4\pi^2-\theta^2}, \quad \quad \quad 0\leq \theta \leq2\pi \\
\displaystyle \uncover<26->{\displaystyle\frac{\diff V}{\diff \theta} &=&} \displaystyle \uncover<27->{\phantom{+}\left(\frac{r^3}{24\pi^2}\right)   \alertNoH{28,29}{\frac{\diff}{\diff \theta}\left(\theta^2\right)} \sqrt{4\pi^2-\theta^2}}\\
&&\uncover<27->{\displaystyle+\left(\frac{r^3}{24\pi^2}\right)\theta^2\alertNoH{30,31}{\frac{\diff}{ \diff \theta}\left(\sqrt{4\pi^2-\theta^2} \right)}} \\
\uncover<28->{&=& \displaystyle \phantom{+}\left(\frac{r^3}{24\pi^2}\right) \alertNoH{28,29}{( \uncover<29->{\alertNoH{29,33}{2\theta}})} \alertNoH{33}{\sqrt{4\pi^2-\theta^2}}}\\
&&\displaystyle\uncover<28->{ +\left(\frac{r^3}{24\pi^2}\right) \alertNoH{34}{\theta^2} \alertNoH{30,31,34}{\left(\uncover<31->{ \frac{1}{2} \frac{ \frac{\diff }{\diff \theta} (-\theta^2)}{\sqrt{4\pi^2-\theta^2}}}\right)}  } \\
\uncover<32->{&=&\displaystyle\left(\frac{r^3}{24\pi^2}\right)\frac{ \alertNoH{33}{ 2\theta (4 \pi^2-\theta^2)}\alertNoH{34}{-\theta^3} }{\alertNoH{33,34}{ \sqrt{ 4 \pi^2-\theta^2}}}}\\
\uncover<35->{&=&\displaystyle\left(\frac{r^3}{24\pi^2}\right)\frac{8 \theta \pi^2-3\theta^3 }{\sqrt{4\pi^2-\theta^2}}}
\end{array}
$
}
\only<handout:5|36-44>{
$
\begin{array}{rcl}
V&=&\displaystyle \frac{r^3}{24\pi^2} \theta^2\sqrt{4\pi^2-\theta^2}, \quad \quad \quad \alertNoH{44}{0\leq \theta \leq2\pi} \\
\displaystyle \frac{\diff V}{\diff \theta}&=& \displaystyle\left(\frac{r^3 } {24\pi^2} \right)\frac{\alertNoH{38}{8\theta\pi^2-3\theta^3} }{\sqrt{\alertNoH{43}{4\pi^2-\theta^2}}}
\end{array}
$

\uncover<37->{We have that $\frac{\diff V}{\diff \theta}=0$ when }
$
\begin{array}{rcl}
\uncover<38->{ \alertNoH{38}{8\theta\pi^2-3\theta^3}&=&0}\\
\uncover<39->{\theta(8\pi^2-3\theta^2)&=&0}\\
\uncover<40->{-3\alertNoH{41}{\theta}\alertNoH{42}{\left(\theta-\sqrt{\frac{8}{3}}\pi \right)} \alertNoH{44}{\left(\theta+\sqrt{\frac{8}{3}}\pi \right)}&=&0.}
\end{array}
$

\uncover<41->{Therefore $\theta$ is critical point for $V$ if $\alertNoH{41}{\theta= 0}$, $\alertNoH{42}{\theta=\sqrt{\frac83}\pi }$, or \alertNoH{43}{$\theta=2\pi$}} \uncover<44->{(note $\alertNoH{44}{\theta=-\sqrt{\frac83}\pi}$ is outside of the domain of $V$).} \uncover<45->{For $\theta=0,2\pi$ the volume $V$ is $0$, so the maximum
volume is attained at $\theta=\sqrt{\frac83}\pi$.}
} %frame36
\only<handout:6|46->{
\[
V(\alertNoH{47}{\theta} )=\displaystyle \frac{r^3}{24\pi^2} \alertNoH{47}{\theta}^2 \sqrt{4 \pi^2 - \alertNoH{47}{\theta}^2}
\]
Finally, the answer to the problem is
$
\begin{array}{rcl}
V_{max}&=&\displaystyle V\left(\alertNoH{47}{ \sqrt{\frac83} \pi} \right)\\
\uncover<47->{ &=&\displaystyle \frac{r^3}{\alertNoH{48}{24}\alertNoH{49}{\pi^2}} \left( \alertNoH{47}{ \alertNoH{48}{\sqrt{\frac83}} \alertNoH{49}{\pi}} \right)^{\alertNoH{48,49}{2}} \sqrt{4 \alertNoH{49}{\pi^2} -\left( \alertNoH{47}{ \sqrt{\frac83} \alertNoH{49}{\pi}} \right)^{\alertNoH{49}{2}}}}\\
\uncover<48->{ &=&\displaystyle \frac{r^3}{\alertNoH{48}{9}} \alertNoH{49}{\pi} \sqrt{\alertNoH{50}{4-\frac{8}3}}}\\
\uncover<50->{&=& \displaystyle \pi\frac{r^3}9 \sqrt{ \alertNoH{50}{\frac43}}} \uncover<51>{=\frac{2\pi r^3}{9\sqrt3}}
\end{array}
$
}

\end{columns}
\uncover<3>{}
\end{example}
\end{frame}
