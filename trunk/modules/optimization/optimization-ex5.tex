% begin module optimization-ex5
\begin{frame}
\vskip -0.1cm
\begin{example}
Find the largest possible area of a rectangle inscribed in a semicircle of radius $r$.
\begin{columns}[c]
\column{0.4\textwidth}
\psset{xunit=1.5cm, yunit=1.5cm}
\begin{pspicture}(-1.5,-0.5)(1.6,1.4)
\psframe*[linecolor=white](-1.5,-0.5)(1.6,1.4)
\tiny
\fcBoundingBox{-1.5}{-0.35}{1.5}{1.4}
\uncover<10->{\fcAxesStandardNoFrame{-1.5}{-0.35}{1.5}{1.3}}
\uncover<1-9>{\psline(-1,0)(1,0)}
%Function formula: sqrt{}(1- ((x)^{2}))
\psplot[linecolor=black, plotpoints=1000]{-1}{1}{x 2 exp -1 mul 1 add sqrt }
\rput[t](1,-0.1){$r$}
\uncover<10->{\rput[t](-1,-0.1){$-r$}}
\uncover<10->{\psline[linecolor=red](-0.866025404, 0)(-0.866025404, 0.5)(0.866025404, 0.5)(0.866025404, 0)}
\uncover<16->{
\psline[linecolor=red](-0.866025404, 0.5,)(0.866025404, 0.5)
\psline{<-}(-0.866025404, 0.4)(0.1, 0.4)
\rput(0.25, 0.4){$\alertNoH{16}{2x}$}
\psline{->}(0.4,0.4) (0.866025404, 0.4)
}
\uncover<handout:0|15,16>{\psline[linewidth=2pt, linecolor=purple](0.866025404, 0.5)(-0.866025404, 0.5)}
\uncover<11->{\fcFullDot{0.866025404}{0.5}
\rput[bl](0.866025404,0.6){$(x,\alertNoH{14}{y})$}
}
\uncover<handout:0|13,14>{\psline[linewidth=2pt, linecolor=purple](0.866025404, 0.5)(0.866025404, 0)}
\uncover<handout:0|2>{
\psline[linecolor=red](-0.9, 0) (-0.9, 0.43589) (0.9, 0.43589) (0.9, 0)
}
\uncover<handout:0|3>{
\psline[linecolor=red](-0.8, 0) (-0.8, 0.6) (0.8, 0.6) (0.8, 0)
}
\uncover<handout:0|4>{
\psline[linecolor=red](-0.7, 0) (-0.7, 0.714143) (0.7, 0.714143) (0.7, 0)
}
\uncover<handout:0|5>{
\psline[linecolor=red](-0.6, 0) (-0.6, 0.8) (0.6, 0.8) (0.6, 0)
}
\uncover<handout:0|6>{
\psline[linecolor=red](-0.5, 0) (-0.5, 0.866025) (0.5, 0.866025) (0.5, 0)
}
\uncover<handout:0|7>{
\psline[linecolor=red](-0.4, 0) (-0.4, 0.916515) (0.4, 0.916515) (0.4, 0)
}
\uncover<handout:0|8>{
\psline[linecolor=red](-0.3, 0) (-0.3, 0.953939) (0.3, 0.953939) (0.3, 0)
}
\uncover<handout:0|9>{
\psline[linecolor=red](-0.2, 0) (-0.2, 0.979796) (0.2, 0.979796) (0.2, 0)
}
\end{pspicture}

\uncover<17->{To eliminate $y$, use that \alertNoH{18}{$(x,y)$ lies on the semicircle.}}%
\abovedisplayskip=0pt
\belowdisplayskip=0pt
\abovedisplayshortskip=0pt
\belowdisplayshortskip=0pt
\vskip -0.1cm
\[
\begin{array}{@{}r@{}c@{}l@{}l|l}
\uncover<18->{\displaystyle  \alertNoH{18}{ \uncover<handout:0|18,19>{\alertNoH{19}{x^2+} } y^{\alertNoH{21}{2}}} & \displaystyle \alertNoH{18}{=}  & \alertNoH{18}{ r^2\uncover<20->{ \alertNoH{20}{ -x^2}}}}\\
\displaystyle \uncover<21->{\alertNoH{ 24}{y} & \alertNoH{ 24}{=}&\displaystyle  \onlyNoH{21}{\color{red}}  \uncover<handout:0|21,22>{\alertNoH{22}{\pm}} \alertNoH{24}{\sqrt{ \onlyNoH{21}{\color{black}} r^2-x^2}} \uncover<handout:0| 22-23>{ & &\alertNoH{22,23}{ y>0}}}
\end{array}
\]
%\uncover<20->{%
%\abovedisplayskip=0pt
%\belowdisplayskip=0pt
%\[
%\begin{array}{r|r}
%x & A(x)\\
%\hline
%\alertNoH{ 21-22}{0} & \alertNoH{ 22}{\uncover<22->{0}}\\
%\alertNoH{ 23-24,27}{600} & \alertNoH{ 24,27}{\uncover<24->{720,000}}\\
%\alertNoH{ 25-26}{2400} & \uncover<26->{\alertNoH{ 26}{0}}
%\end{array}
%\]
%}%
\column{.6\textwidth}
\uncover<10->{Let the semicircle have center at the origin.}  \uncover<11->{Let $(x,y)$ -coord. of top right corner of rectangle.} \uncover<12->{Let $A$ be its area.}%

\vskip 0.1cm

$
\begin{array}{@{}r@{}c@{}l}
\displaystyle \uncover<12->{\alertNoH{25,26} {A} & \alertNoH{25,26}{=} & \alertNoH{15, 16}{\text{base}} \cdot \alertNoH{13, 14}{\text{height}} }\\
\uncover<13->{ &\alertNoH{0}{=} &\displaystyle \fcAnswerUncover{13}{16}{2x} \cdot  \fcAnswerUncover{13}{14}{\alertNoH{ 17-24}{y}}  \uncover<17->{= \alertNoH{25,26}{ 2x\cdot \fcAnswerUncover{17}{ 24}{ \sqrt{r^2 -x^2 } }}}} \\
\uncover<25->{\alertNoH{ 25-26}{A'} &  \alertNoH{ 25-26}{=} &\displaystyle  \fcAnswer{26}{2 \alertNoH{27}{\sqrt{r^2-x^2}} - \frac{2x^2}{\sqrt{r^2-x^2}}}} \\
\uncover<27->{& \alertNoH{0}{=}&\displaystyle \onlyNoH{27}{\color{red}} \frac{\onlyNoH{27}{\color{black}} \alertNoH{29}{2} \onlyNoH{27}{\color{red}} (\alertNoH{27}{r^2 \alertNoH{29}{- x^2} })}{ \alertNoH{27, 28}{\sqrt{ r^2-x^2}} } \onlyNoH{27}{\color{black}} \alertNoH{29}{-} \frac{ \alertNoH{29}{2 x^2}}{\alertNoH{28}{ \sqrt{r^2-x^2}}}\uncover<28-> {=  \frac{\alertNoH{29} {2}( \alertNoH{32}{ r^2 \alertNoH{29}{- 2x^2}} )}{ \alertNoH{28}{\sqrt{\alertNoH{33}{r^2-x^2}}}}}}
\end{array}
$
\uncover<30->{%
Critical numbers: \alertNoH{ 30-31}{$x =  \fcAnswer{31}{ \alertNoH{32}{ \frac{ r}{\sqrt{2}}} \text{ and } \alertNoH{33}{r} }$.}
}%
\end{columns}
\uncover<34->{%
We have $0\leq x\leq r$ and so the critical numbers together with the endpoints are $x=0, \frac{r}{\sqrt{2}}, r $.} \uncover<35->{Since   $A(0) = 0 = A(r)$, the max is achieved at $x=y=\frac{r}{ \sqrt{2}}$.} \uncover<36->{ The max area is $A(\frac{r}{\sqrt{2}}) =  2\frac{r}{ \sqrt{2}}\sqrt{r^2 - \frac{r^2}{2}} = r^2$.
}%
\end{example}
\end{frame}
% end module optimization-ex5
