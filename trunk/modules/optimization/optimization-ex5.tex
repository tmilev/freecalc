% begin module optimization-ex5
\begin{frame}
\begin{example}[Example 5, p. 303]
Find the area of the largest rectangle that can be inscribed in a semicircle of radius $r$.
\begin{columns}[c]
\column{.5\textwidth}
\ \only<handout:0| -2>{%
\uncover<2>{%
\includegraphics[width=5cm]{optimization/pictures/04-07-ex5a.pdf}%
}}%
\only<handout:0| 3>{%
\includegraphics[width=5cm]{optimization/pictures/04-07-ex5b.pdf}%
}%
\only<handout:0| 4>{%
\includegraphics[width=5cm]{optimization/pictures/04-07-ex5c.pdf}%
}%
\only<handout:0| 5>{%
\includegraphics[width=5cm]{optimization/pictures/04-07-ex5d.pdf}%
}%
\only<handout:0| 6>{%
\includegraphics[width=5cm]{optimization/pictures/04-07-ex5e.pdf}%
}%
\only<handout:0| 7>{%
\includegraphics[width=5cm]{optimization/pictures/04-07-ex5f.pdf}%
}%
\only<handout:0| 8>{%
\includegraphics[width=5cm]{optimization/pictures/04-07-ex5g.pdf}%
}%
\only<handout:0| 9>{%
\includegraphics[width=5cm]{optimization/pictures/04-07-ex5h.pdf}%
}%
\only<10->{%
\includegraphics[width=5cm]{optimization/pictures/04-07-ex5i.pdf}%
}%

\uncover<12->{To eliminate $y$, use the fact that $(x,y)$ lies on the semicircle.}%
\abovedisplayskip=0pt
\belowdisplayskip=0pt
\abovedisplayshortskip=0pt
\belowdisplayshortskip=0pt
\begin{align*}
\uncover<12->{y^2} & \uncover<12->{=}  \uncover<12->{r^2-x^2}\\
\uncover<13->{\alert<handout:0| 15>{y}} & \uncover<13->{\alert<handout:0| 15>{=}}  \uncover<13->{\alert<handout:0| 15>{\sqrt{r^2-x^2}}}%
\end{align*}
%\uncover<20->{%
%\abovedisplayskip=0pt
%\belowdisplayskip=0pt
%\[
%\begin{array}{r|r}
%x & A(x)\\
%\hline
%\alert<handout:0| 21-22>{0} & \alert<handout:0| 22>{\uncover<22->{0}}\\
%\alert<handout:0| 23-24,27>{600} & \alert<handout:0| 24,27>{\uncover<24->{720,000}}\\
%\alert<handout:0| 25-26>{2400} & \uncover<26->{\alert<handout:0| 26>{0}}
%\end{array}
%\]
%}%
\column{.5\textwidth}
\uncover<10->{%
Let the semicircle have center at the origin.  Let $(x,y)$ be the coordinates of the top right corner of the rectangle.  Let $A$ be its area.
}%

\uncover<14->{Notice that $0\leq x \leq r$.}
\abovedisplayskip=0pt
\belowdisplayskip=0pt
\abovedisplayshortskip=0pt
\belowdisplayshortskip=0pt
\begin{align*}
\uncover<11->{A} & \uncover<11->{=}  \uncover<11->{2x\alert<handout:0| 15>{y}}  \uncover<15->{=}  \uncover<15->{2x\alert<handout:0| 15>{\sqrt{r^2-x^2}}}\\
\uncover<16->{\alert<handout:0| 16-17>{A'}} & \uncover<16->{\alert<handout:0| 16-17>{=}}  \uncover<17->{\alert<handout:0| 17>{2\sqrt{r^2-x^2} - \frac{2x^2}{\sqrt{r^2-x^2}}}}\\
& \uncover<18->{=}  \uncover<18->{\frac{2(r^2-2x^2)}{\sqrt{r^2-x^2}}}
\end{align*}

\uncover<19->{%
Critical number: \alert<handout:0| 19-20>{$x = $ \uncover<20->{$\frac{r}{\sqrt{2}}$.}} 
}%
\end{columns}
\uncover<21->{%
There is a local max. here because $A(0) = 0 = A(r)$.  Therefore the maximum area is $A(\frac{r}{\sqrt{2}}) = $ $ 2\frac{r}{\sqrt{2}}\sqrt{r^2 - \frac{r^2}{2}} = r^2$.
}%
\end{example}
\end{frame}
% end module optimization-ex5
