\begin{frame}
\frametitle{Implicit vs Explicit (Parametric) Curve Equations}
\begin{itemize}
\item Consider the parametric curve 
\[\alert<3->{
\left|\begin{array}{rcl}x  & = & -t^2 + 2\\
y&=&t-1 \quad .
\end{array}\right.
}\]
\item<2-> As we saw in preceding slides/lectures, all points $(x,y)$ on the image of this curve satisfy the equation 
\[ \alert<4>{x+(y+1)^2 -2=0}\]
\item<3-> Equations of the first form are called explicit (parametric) curve equations.
\item<4-> Equations of the second form are called implicit equations of the curve image.
\item<5-> Explicit (parametric) curve equations have the advantage that it is easy to generate points on the curve. 
\item<6-> Implicit curve equations have the advantage that it is easy to check whether a point is on the curve.

\end{itemize}
\end{frame}