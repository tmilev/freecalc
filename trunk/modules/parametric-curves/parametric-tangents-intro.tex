% begin module parametric-tangents-intro
\begin{frame}
\frametitle{Tangents}
\begin{itemize}
\item  Suppose we have a parametric curve $\alertNoH{ 4}{x = f(t)}$, $y = g(t)$.
\item  We saw previously that we can sometimes eliminate the parameter and write $y = F(x)$.
\item<2->  Substitute $x = f(t)$ and $y = g(t)$ in this:
\end{itemize}
\begin{eqnarray*}
\uncover<2->{%
g(t)%
}%
& \uncover<2->{ = } &%
\uncover<2->{%
F(f(t))%
}\\%
\uncover<3->{%
g'(t)%
}%
& \uncover<3->{ = } &%
\uncover<3->{%
F'(\alertNoH{ 4}{f(t)})f'(t)%
}\\%
\uncover<4->{%
g'(t)%
}%
& \uncover<4->{ = } &%
\uncover<4->{%
F'(\alertNoH{ 4}{x})f'(t)%
}\\%
\uncover<5->{%
\textrm{If } f'(t)\neq 0, \ \ F'(x)%
}%
& \uncover<5->{ = } &%
\uncover<5->{%
\frac{g'(t)}{f'(t)}%
}%
\end{eqnarray*}
\uncover<6->{%
This is easy to remember using Leibniz notation:
\[
\frac{\diff y}{\diff x} = \frac{\frac{\diff y}{\diff t}}{\frac{\diff x}{\diff t}} \qquad \textrm{if} \ \frac{\diff x}{\diff t} \neq 0.
\]
}%
\end{frame}


\begin{frame}
\[
\frac{\diff \alertNoH{ 7}{y}}{\diff x} = \frac{\alertNoH{ 3}{\frac{\diff \alertNoH{ 7}{y}}{\diff t}}}{\alertNoH{ 5}{\frac{\diff x}{\diff t}}}  \qquad \textrm{if} \ \frac{\diff x}{\diff t} \neq 0.%
\]
\begin{itemize}
\item<1-| alert@2-3>  Horizontal tangent if \uncover<3->{$\frac{\diff y}{\diff t} = 0$ and $\frac{\diff x}{\diff t} \neq 0$.}
\item<1-| alert@4-5>  Vertical tangent if \uncover<5->{$\frac{\diff x}{\diff t} = 0$ and $\frac{\diff y}{\diff t} \neq 0$.}
\item<6->  How do we find the second derivative?
\end{itemize}
\begin{eqnarray*}
\uncover<6->{%
\frac{\diff^2 y}{\diff x^2}%
}%
& \uncover<6->{ = } &%
\uncover<6->{%
\frac{\diff}{\diff x}\left( \alertNoH{ 7}{\frac{\diff y}{\diff x}}\right)%
}\\%
& \uncover<7->{ = } &%
\uncover<7->{%
\frac{\frac{\diff}{\diff t}\left( \alertNoH{ 7}{\frac{\diff y}{\diff x}}\right)}{\frac{\diff x}{\diff t}}%
}%
\end{eqnarray*}
\end{frame}
% end module parametric-tangents-intro
