\begin{frame}
\frametitle{Differentiability}

If $y=h(x)$ is a function of one variable, then
%
$$L_{h,x_0}(x) = h(x_0) + h'(x_0) (x-x_0)$$
%
$$
  \lim_{x\to x_0} \frac{|h(x)-L_{h,x_0}(x)|}{|x-x_0|} = \lim_{x \to x_0} \left| \frac{h(x)-h(x_0)}{x-x_0} -h'(x_0) \right| = 0
$$
%
\pause \underline{One variable}: the linear approximation is a good approximation.

\pause \underline{Several variables}:

$f_x(x_0,y_0)$, $f_y(x_0,y_0)$ exist $\Longrightarrow$
$f$ has a linear approximation $L_{f, (x_0,y_0)}$.

But is that a \emph{good} linear approximation? \pause
Unfortunately, \textcolor[rgb]{0.98,0.00,0.00}{not always}!
\end{frame}