\begin{frame}
\frametitle{Multivariable Differentiability Definition}

\begin{itemize}
\item Let $(x_0,y_0)$ be a fixed point and $a$ and $b$ be arbitrary numbers.
\item Define $\varepsilon_{f, a, b}(x,y) = f(x,y)- f(x_0,y_0)-a (x-x_0) -b(y-y_0)$.
\item $\varepsilon_{f, a, b}$ measures how well does $f(x_0,y_0)+a(x-x_0)+b(y-y_0)$ approximate $f$.
\end{itemize}
For the particular case:
$
\begin{array}{@{}r@{}c@{}l@{}}
a&=&\frac{\partial f}{\partial x}(x_0,y_0) \\
b&=&\frac{\partial f}{\partial y}(x_0,y_0)\\
\multicolumn{3}{l}{\text{we have that}}\\
\varepsilon_{f, a, b}(x,y) &=& f(x,y)- f(x_0,y_0)-\frac{\partial f}{\partial x}(x_0,y_0) (x-x_0) -\frac{\partial f}{\partial x}(x_0,y_0)(y-y_0).
\end{array}
$

 
\begin{definition}
$f$ is called \emph{differentiable at $(x_0,y_0)$} if there exist $a$ and $b$ such that \[\lim_{(x,y)\to (0,0)}  \frac{\varepsilon_{f, a, b} (x,y)}{|(x-x_0, y-y_0)|}=0 \]
\end{definition}
\begin{observation}
If a function $f$ is differentiable, then the numbers $a$
\end{observation}

\pause Example: $f(x,y) = x^2+xy+2y^2$ is differentiable at $(4,1)$.

\medskip

\pause \underline{Fact}: If $f$ has a good linear approximation at $(x_0,y_0)$, then the good linear approximation is necessarily the linearization of $f$ at $(x_0,y_0)$.
\end{frame}