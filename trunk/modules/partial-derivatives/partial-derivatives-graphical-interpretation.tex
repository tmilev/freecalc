\begin{frame}
\frametitle{Graphical Interpretation}
\begin{columns}
\column{0.4\textwidth}
\begin{pspicture}(-2, -2)(2,2)
\renewcommand{\fcScreen}{[-1 1 -2] 0}
\fcStartIIIdScene
\fcSurfaceInScene[iterationsU=6, iterationsV=6]{-1.2}{-1.2}{1.2}{1.2}{[u v u u mul v v mul sub]}
\fcCurveIIIdInScene{-3}{3}{[0 0 t]}
\fcFinishIIIdScene
\end{pspicture}
\column{0.6\textwidth}
\begin{itemize}
\item Recall the graph of $f$ is the surface whose points are $\{( x,y, f(x,y))\}$.
\item The vertical plane containing the line $\textbf{r}=\textbf{r}_0 + t\textbf{i}$ is the plane $y=y_0$.
\item Intersection of graph with the plane $y=y_0$ is the curve
\[\gamma(t) = \langle t, y_0, f(t,y_0) \rangle.
\]
\item  The graph of $z=h(x)$
\item The direction of tangent line to $\gamma$ is: $\gamma'(x_0) = \langle 1,0,f_x(x_0,y_0) \rangle$

\item In the $xz-$plane $y=y_0$, the slope of this line is $h'(x_0) = f_x(x_0,y_0)$.
\end{itemize}
\end{columns}
\end{frame}