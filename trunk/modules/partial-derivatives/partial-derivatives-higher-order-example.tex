\begin{frame}
\frametitle{Example}
$\alert<1-10>{f(x,y) = x^2y^3}$. Then
\[
\begin{array}{rclllrcl}
\alert<1,2>{f_x(x,y) }&\alert<1,2>{=}& \uncover<2->{ \alert<2>{ 2xy^3}}  &&&  \alert<7,8>{f_{y}(x,y)} & \alert<7,8>{=}&  \uncover<8->{ \alert<8>{3x^2y^2}} \\
\alert<3,4>{ f_{xx}(x,y)} &\alert<3,4>{=}& \uncover<4->{ \alert<3,4>{(2xy^3)_x = 2y^3}}  &&& \alert<9,10,13>{ f_{xy}( x,y) } &\alert<9,10,13>{=}& \uncover<10->{ \alert<10,13>{(2xy^3)_y = 6xy^2}} \\
\alert<5,6,13>{ f_{yx}( x,y)} &\alert<5,6,13>{=}& \uncover<6->{ \alert<6,13>{ (3x^2y^2)_x = 6xy^2}}  &&& \alert<11,12>{f_{yy}(x,y) }&\alert<11,12>{=}& \uncover<12->{ \alert<12>{(3x^2y^2)_y = 6x^2y}}
\end{array}
\]

\uncover<13->{ Notice that $\alert<13>{f_{xy} = f_{yx}}$. That is not a coincidence.}

\uncover<14->{
\begin{theorem}[Clairaut, (1713-1765)] If the second order derivatives $f_{xy}$ and $f_{yx}$ are continuous on an \alert<15>{open set}, then they are equal everywhere on that set.
\end{theorem}
}
\begin{itemize}
\item<15-> An \alert<15>{open set} is a connected set that contains a small open disk around all of its points, for example an open disk.
\item<16-> An analogous theorem is valid in $n$ dimensions.
\end{itemize}
\end{frame}