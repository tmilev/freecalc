\begin{frame}
\frametitle{Partial Derivatives}
\begin{itemize} 
\item Let $f\colon D \to \RR$,  $P_0(x_0,y_0)$ inside $D$.
\item<2-> Consider the line $\fcv{r}(t) = \fcv{r}_0 + t\fcv {i} = \langle x_0+t, y_0 \rangle$.
\item<3-> Set $g(t) = f(\fcv{r}(t)) = f(x_0+t, y_0)$.
\item<4-> Then $(D_{\fcv{i}} f)(x_0,y_0) = \lim\limits_{t\to 0} \frac{g(t)-g(0)}{t}$.
\item<5-> Define $ \frac{\partial}{\partial x}$ to be the differential operator $ D_{\fcv{i}}$, and similarly define $ \frac{\partial}{\partial y}$ to be the differential operator $ D_{\fcv{j}}$.

\begin{definition}[partial derivatives]
The partial derivatives $\frac{\partial}{\partial x} $, $\frac{\partial}{\partial y} $  of $f$ are defined as the directional derivatives of $f$ in the direction of the unit vector along the $x$, $y$ axes, i.e., 
\[
\begin{array}{rcl}
\displaystyle\frac{\partial}{\partial x}(f)&=&(D_{\fcv{i}})(f)\\
\displaystyle\frac{\partial}{\partial y}(f)&=&(D_{\fcv{j}})(f)\quad .
\end{array}
\]
\end{definition}
\end{itemize}
\end{frame}