\begin{frame}
  \frametitle{Rates of Change along Lines}

  $L$: line through $P_0(\textbf{r}_0)$.

  \pause
  \begin{center}
    How does $f(\textbf{r}) = |\textbf{r}|^2$
  change \textcolor[rgb]{0.98,0.00,0.00}{along $L$}?
  \end{center}

  \pause $\textbf{r} \colon \RR \to L$: smooth parametrization of $L$,
  $\textbf{r}(0) = \textbf{r}_0$
  %
  $$g\colon \RR \to \RR, \quad g(t) = f(\textbf{r}(t))$$
  %
  \begin{center}
    Rate of change of $f$ along $L$ = rate of change of $g$
  \end{center}

  \pause With respect to $t$:
  %
  $$\lim_{t\to 0} \frac{f(\textbf{r}(t))-f(\textbf{r}(0))}{t} =
  \lim_{t\to 0} \frac{g(t)-g(0)}{t} = g'(0)$$

  \pause Still ambiguous: \pause depends on the parametrization $\textbf{r}$.\pause

  Solution: \pause Arclength parametrization! \pause

  Almost solves the problem: \pause orientation still matters.
\end{frame}
