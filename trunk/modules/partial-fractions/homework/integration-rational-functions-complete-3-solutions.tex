\solution{\ref{problemIntegrate x/(2x^2+x-1)dx}
The quadratic in the denominator has real roots and therefore can be factored using real numbers. We therefore use partial fractions.
\[
\begin{array}{rcll|l}
\displaystyle \int \frac{x }{2x^2+x-1}\diff{}x&=&\displaystyle \int \frac{\frac{1}{2}x}{\left( x+1\right)\left(x-\frac{1}{2}\right)} \diff x &&\text{partial fractions, see below}\\
&=&\displaystyle  \int \frac{\frac{1}{3} }{\left( x+1\right)}\diff {}x +\int \frac{\frac{1}{6}}{ \left(x - \frac{1}{2}\right)} \diff {}x\\
&=&\displaystyle \frac{1}{3} \ln |x+1|+\frac{1}{6} \ln \left|x-\frac{1}{2}\right| +C\quad .
\end{array}
\]
Except for showing how the partial fraction decomposition was obtained, our solution is complete. 

We proceed to compute the partial fraction decomposition used above. In what follows, we will use the most straightforward technique - the method of coefficient comparison. This technique is the most laborious for a human but is perhaps the easiest to implement on a computer. The computations below were indeed carried out by a computer program written for the purpose. We note that the method of coefficient comparison is fast enough for a human, but  techniques such as the one used in the solution of Problem \ref{problemint(3x^2+2x-1)/((x-1)(x^2+1))dx} are much easier when not equipped with a computer.

We aim to decompose into partial fractions the following function (the denominator has been factored). 
\[
\frac{x }{2x^{2}+x -1}=\frac{x }{ \left(x +1\right)\left(2x -1\right)} = \frac{A_1}{x+1}+\frac{A_2}{2x-1}\quad .
\]
After clearing denominators, we get the following equality. 
\[x = A_{1} (2x -1)+A_{2} (x +1)
\]
After rearranging we get that the following polynomial must vanish. Here, by ``vanish'' we mean that the coefficients of the powers of $x$ must be equal to zero.
\[(A_{2} +2A_{1} -1)x +(A_{2} -A_{1} )\quad .
\]
In other words, we need to solve the following system.
\[
\begin{array}{llll} & 2A_{1} & +A_{2} & =1\\ & -A_{1} & +A_{2} & =0\\\end{array}
\] 

\begin{longtable}{cc} System status&Action \\\hline $\begin{array}{llll} & 2A_{1} & +A_{2} & =1\\ & -A_{1} & +A_{2} & =0\\\end{array}$ & Selected pivot column 2. Eliminated the non-zero entries in the pivot column. \\\hline $\begin{array}{llll} & A_{1} & +\frac{A_{2} }{2} & =\frac{1}{2}\\ & & \frac{3}{2}A_{2} & =\frac{1}{2}\\\end{array}$& Selected pivot column 3. Eliminated the non-zero entries in the pivot column. \\\hline $\begin{array}{llll} & A_{1} & & =\frac{1}{3}\\ & & A_{2} & =\frac{1}{3}\\\end{array}$& Final result.\\ \end{longtable}
Therefore, the final partial fraction decomposition is: \[\frac{\frac{x }{2}}{x^{2}+\frac{x }{2} -\frac{1}{2} } =\frac{ \frac{ 1}{3}}{(x +1)}+ \frac{\frac{1}{3}}{(2x -1)}\]
}

\solution{ \ref{problemIntegral x^4/((x^2+2)(x+1)^2)}
We are trying to integrate a rational function; we aim to decompose into partial fractions the following function. 
\[
\frac{x^{4}}{x^{4}+2x^{3}+3x^{2}+4x +2}=\frac{x^{4}}{ \left(x +1\right)^2 \left(x^{2}+2\right)}
\]

Since the numerator of the function is of degree greater than or equal to the denominator, we start the partial fraction decomposition by polynomial division.
\renewcommand{\arraystretch}{1.2}\begin{longtable}{|cccccc|} \hline&\multicolumn{5}{|c|}{\textbf{Remainder}}\\\multicolumn{1}{|c|}{} & &$\color{red}{-2x^{3}}\color{black}$ & $\color{red}{-3x^{2}}\color{black}$ & $\color{red}{-4x }\color{black}$ & $\color{red}{-2}\color{black}$ \\\hline\textbf{Divisor(s)} &\multicolumn{5}{|c|}{\textbf{Quotient(s)}}\\$x^{4}+2x^{3}+3x^{2}+4x +2$& \multicolumn{5}{|l|}{$1$}\\\hline& \multicolumn{5}{|c|}{\textbf{Dividend}}\\\multicolumn{1}{|c|}{$\underline{~}$} &$x^{4}$ & &&&\\&$x^{4}$ & $+2x^{3}$ & $+3x^{2}$ & $+4x $ & $+2$ \\\cline{2-6}&&$\color{red}{-2x^{3}}\color{black}$ & $\color{red}{-3x^{2}}\color{black}$ & $\color{red}{-4x }\color{black}$ & $\color{red}{-2}\color{black}$ \\\hline\end{longtable}
Therefore we have
\[
\begin{array}{rcl}
\displaystyle \frac{x^{4}}{x^{4}+2x^{3}+3x^{2}+4x +2}&=&\displaystyle 1+ \frac{-2x^3-3x ^2-4x-2}{x^{4}+2x^{3} +3x^{2} +4x +2}\\
\displaystyle \frac{-2x^3-3x^2-4x-2}{x^{4}+2x^{3}+3x^{2}+4x +2} & =&\displaystyle  \frac{-2x^3-3x^2-4x-2 }{\left(x +1\right)^2\left(x^{2}+2\right)}\\
&=&\displaystyle  \frac{A_1}{(x+1)}+\frac{A_2}{(x+1)^2}+\frac{A_3+A_4x}{(x^2+2)}
\end{array}
\]
We seek to find $A_i$'s that turn the above expression into an identity. After clearing denominators, we get the following equality. \[-2x^{3}-3x^{2}-4x -2 = A_{1} (x +1)(x^{2}+2)+A_{2} (x^{2}+2)+(A_{3} + A_{4} x)(x +1)^{2}\]After rearranging we get that the following polynomial must vanish. Here, by ``vanish'' we mean that the coefficients of the powers of $x$ must be equal to zero.\[(A_{4} +A_{1} +2)x^{3}+(2A_{4} +A_{3} +A_{2} +A_{1} +3)x^{2}+(A_{4} +2A_{3} +2A_{1} +4)x +(A_{3} +2A_{2} +2A_{1} +2) 
\]
In other words, we need to solve the following system. \[\begin{array}{llllll} & A_{1} & & & +A_{4} & =-2\\ & A_{1} & +A_{2} & +A_{3} & +2A_{4} & =-3\\ & 2A_{1} & & +2A_{3} & +A_{4} & =-4\\ & 2A_{1} & +2A_{2} & +A_{3} & & =-2\\\end{array}\] \begin{longtable}{cc} System status&Action \\\hline $\begin{array}{llllll} & A_{1} & & & +A_{4} & =-2\\ & A_{1} & +A_{2} & +A_{3} & +2A_{4} & =-3\\ & 2A_{1} & & +2A_{3} & +A_{4} & =-4\\ & 2A_{1} & +2A_{2} & +A_{3} & & =-2\\\end{array}$& Selected pivot column 2. Eliminated the non-zero entries in the pivot column. \\\hline $\begin{array}{llllll} & A_{1} & & & +A_{4} & =-2\\ & & A_{2} & +A_{3} & +A_{4} & =-1\\ & & & 2A_{3} & -A_{4} & =0\\ & & 2A_{2} & +A_{3} & -2A_{4} & =2\\\end{array}$& Selected pivot column 3. Eliminated the non-zero entries in the pivot column. \\\hline $\begin{array}{llllll} & A_{1} & & & +A_{4} & =-2\\ & & A_{2} & +A_{3} & +A_{4} & =-1\\ & & & 2A_{3} & -A_{4} & =0\\ & & & -A_{3} & -4A_{4} & =4\\\end{array}$& Selected pivot column 4. Eliminated the non-zero entries in the pivot column. \\\hline $\begin{array}{llllll} & A_{1} & & & +A_{4} & =-2\\ & & A_{2} & & +\frac{3}{2}A_{4} & =-1\\ & & & A_{3} & -\frac{A_{4} }{2} & =0\\ & & & & -\frac{9}{2}A_{4} & =4\\\end{array}$& Selected pivot column 5. Eliminated the non-zero entries in the pivot column. \\\hline $\begin{array}{llllll} & A_{1} & & & & =-\frac{10}{9}\\ & & A_{2} & & & =\frac{1}{3}\\ & & & A_{3} & & =-\frac{4}{9}\\ & & & & A_{4} & =-\frac{8}{9}\\\end{array}$& Final result.\\ \end{longtable}Therefore, the final partial fraction decomposition is the following. \[\frac{x^{4}}{x^{4}+2x^{3}+3x^{2}+4x +2}=1+ \frac{-2x^{3}-3x^{2}-4x -2}{x^{4}+2x^{3}+3x^{2}+4x +2}=1+ \frac{-\frac{10}{9}}{(x +1)}+\frac{\frac{1}{3}}{(x +1)^{2}}+\frac{-\frac{8}{9}x -\frac{4}{9}}{(x^{2}+2)}\]

Therefore we can integrate as follows.
\[
\begin{array}{rcl}
\displaystyle \int \frac{x^4}{(x^2+2)(x+1)^2} \diff x &=&\displaystyle \int \left( 1+ \frac{-\frac{10}{9}}{(x +1)}+\frac{\frac{1}{3}}{(x +1)^{2}}+\frac{-\frac{8}{9}x -\frac{4}{9}}{(x^{2}+2)}\right)\diff x\\
&=&\displaystyle \int \diff x-\frac{10}{9}\int \frac{1}{(x +1)}\diff x + \frac{ 1}{3}\int \frac{ 1}{(x +1)^{2}}\diff x \\
&&\displaystyle -\frac{8}{9} \int \frac{x}{x^{2}+2}\diff x-\frac{4}{9} \int \frac{ 1}{x^{2}+2}\diff x \\
&=&\displaystyle x -\frac{1}{3} (x+1)^{-1}-\frac{10}{9} \log{}\left(x+1\right)\\
&&\displaystyle -\frac{4}{9} \log{}\left(x^{2} +2\right) -\frac{ 2}{ 9} \sqrt{ 2} \arctan{}\left(\frac{\sqrt{2}}{2} x\right)+C
\end{array}
\]
}

\solution{\ref{problemint(3x^2+2x-1)/((x-1)(x^2+1))dx}. 
This is a concise solution written in a form suitable for exam taking. We set up the partial fraction decomposition as follows.
\[
\displaystyle \frac{3x^2 + 2x - 1}{(x-1)(x^2+1)} = \frac{A}{x-1} + \frac{Bx+C}{x^2+1}\quad .
\]
Therefore $3x^2 + 2x - 1 = A(x^2+1) + (Bx+C)(x-1)$. 
\begin{itemize}
\item We set $x = 1$ to get $4 = 2A$, so $A = 2$.
\item We set $x = 0$ to get $-1=A-c$, so $C=3$.
\item Finally, set $x = 2$ to get $15=5A+2B+C$, so $B=1$.
\end{itemize}
We can now compute the integral as follows.
\[
\displaystyle \int\left( \frac{2}{x-1} + \frac{x+3}{x^2+1} \right) \diff x = 2 \ln(|x-1|) + \frac{1}{2} \ln(x^2+1) + 3 \Arctan x+K\quad .
\]
}
