\solution{
\ref{problemint(x^6-x^5+9/2x^4-4x^3+13/2x^2-7/2x+11/4)/(x^5-x^4+3x^3-3x^2+9/4x-9/4)dx}.

\noindent\textbf{Step 1.} The first step of our algorithm is to reduce the fraction so that numerator has smaller degree than the denominator. This is done using polynomial long division as follows.

Variable name(s): $x $1 division steps total.\renewcommand{\arraystretch}{1.2}\begin{longtable}{|cccccccc|} \hline&\multicolumn{7}{|c|}{\textbf{Remainder}}\\\multicolumn{1}{|c|}{} & &&$\color{red}{\frac{3}{2}x^{4}}\color{black}$ & $\color{red}{-x^{3}}\color{black}$ & $\color{red}{+\frac{17}{4}x^{2}}\color{black}$ & $\color{red}{-\frac{5}{4}x }\color{black}$ & $\color{red}{+\frac{11}{4}}\color{black}$ \\\hline\textbf{Divisor(s)} &\multicolumn{7}{|c|}{\textbf{Quotient(s)}}\\$x^{5}-x^{4}+3x^{3}-3x^{2}+\frac{9}{4}x -\frac{9}{4}$& \multicolumn{7}{|l|}{${\color{blue}{x}} $}\\\hline& \multicolumn{7}{|c|}{\textbf{Dividend}}\\\multicolumn{1}{|c|}{$\underline{~}$} &$x^{6}$ & $-x^{5}$ & $+\frac{9}{2}x^{4}$ & $-4x^{3}$ & $+\frac{13}{2}x^{2}$ & $-\frac{7}{2}x $ & $+\frac{11}{4}$ \\&$x^{6}$ & $-x^{5}$ & $+3x^{4}$ & $-3x^{3}$ & $+\frac{9}{4}x^{2}$ & $-\frac{9}{4}x $ & \\\cline{2-8}&&&$\color{red}{\frac{3}{2}x^{4}}\color{black}$ & $\color{red}{-x^{3}}\color{black}$ & $\color{red}{+\frac{17}{4}x^{2}}\color{black}$ & $\color{red}{-\frac{5}{4}x }\color{black}$ & $\color{red}{+\frac{11}{4}}\color{black}$ \\\hline\end{longtable}

In other words, 

\noindent$
\begin{array}{@{}r@{}c@{}l}
x^{6}-x^{5}+\frac{9}{2} x^{4}-4 x^{3}+\frac{13}{2} x^{2}-\frac{7}{2} x + \frac{11}{4}  &=& (x^{5}-x^{4}+3 x^{3}-3 x^{2} + \frac{9}{4} x-\frac{9}{4}) {\color{blue}{x}} \\&&+{\color{red}{\frac{3}{2} x^{4}-x^{3} +\frac{17}{4} x^{2}-\frac{5}{4} x +\frac{11}{4}}} \quad ,
\end{array}
$

\noindent and therefore

\noindent$
\begin{array}{@{}r@{}c@{}l@{}}
\displaystyle \frac{x^{6}-x^{5}+\frac{9}{2} x^{4}-4 x^{3}+\frac{13}{2} x^{2}-\frac{7}{2} x+\frac{11}{4}}{x^{5}-x^{4}+3 x^{3}-3 x^{2}+\frac{9}{4} x-\frac{9}{4}} &=&\displaystyle {\color{blue}{x}} +\frac{\color{red}{\frac{3}{2} x^{4}-x^{3} +\frac{17}{4} x^{2}-\frac{5}{4} x +\frac{11}{4}}}{x^{5}-x^{4}+3 x^{3}-3 x^{2}+\frac{9}{4} x-\frac{9}{4} }
\\ 
&=&\displaystyle  x+\frac{6 x^{4}-4 x^{3}+17 x^{2}-5 x+11}{4x^{5}-4 x^{4}+12 x^{3}-12 x^{2}+9 x-9}.
\end{array}
$

\noindent Set
\[
N(x)= 6 x^{4}-4 x^{3}+17 x^{2}-5 x+11
\]
and
\[
D(x)= 4x^{5}-4 x^{4}+12 x^{3}-12 x^{2}+9 x-9\quad .
\]

\noindent\textbf{Step 2.} (Split into partial fractions). Factor the denominator $D(x)=4x^{5}-4 x^{4}+12 x^{3}-12 x^{2}+9 x-9$. 

We recall from elementary algebra that there is a trick to find all rational roots of $D(x)$ on condition $D(x)$ has integer coefficients. It is well known that when $\frac{p}{q}$ is a rational number, then $\pm \frac{p}{q}$ may be a root of the integer coefficient polynomial $D(x)$ only if $p$ is a divisor of the constant term of $D(x)$, and $q$ is a divisor of the leading coefficient of $D(x)$. Since in our case the leading coefficient is 4 and the constant term is -9, the only possible rational roots of $D(x)$ are $\pm 1, \pm 3, \pm 9, \pm \frac{1}{2}, \pm \frac{3}{2}, \pm \frac{9}{2}, \pm \frac{1}{4}, \pm \frac{3}{4}, \pm \frac{9}{4}$. A rational number $r$ is a root of $D(x)$ if and only if substituting $x=r$ yields 0. Direct check shows that, for example,  $D(-1)=-50$. However, $D(1)=0$ and therefore using polynomial division we get that $D(x)=(x-1)(4x^{4}+12x^{2}+9)$. We recognize that the second multiplicand is an exact square and therefore $D(x)=(x-1)(2x^2+3)^2$.


So far we got
\[
\frac{N(x)}{D(x)}= \frac{6 x^{4}-4 x^{3}+17 x^{2}-5 x+11}{(x-1)(2x^2+3)^2}\quad .
\]
In order to split $\frac{N(x)}{D(x)}$ into partial fractions, we need to find numbers $A, B, C, D, E$ such that
\[
\frac{6 x^{4}-4 x^{3}+17 x^{2}-5 x+11}{(x-1)(2x^2+3)^2}= \frac{A}{(x-1)}+\frac{Bx+C}{(2x^2+3)}+\frac{Dx+E}{(2x^2+3)^2}\quad .
\]
After clearing denominators, we see that this amounts to finding $A, B, C, D, E$ such that
\[
6 x^{4}-4 x^{3}+17 x^{2}-5 x+11= A(2x^2+3)^2+ (Bx+C)(2x^2+3)(x-1) + (Dx+E)(x-1)\quad .
\]
Plugging in $x=1$ we see that $25=25A $ and so $A=1$. We may plug back $A=1$ and regroup to get
\[
2x^{4}-4x^{3}+5x^{2}-5x+2= (Bx+C)(2x^2+3)(x-1) + (Dx+E)(x-1)\quad .
\]
Dividing both sides by $(x-1)$ we get
\[
2x^{3}-2x^{2}+3x-2= (Bx+C)(2x^2+3)+Dx+E\quad .
\]
Regrouping we get
\[
x^{3}(2- 2B) + x^2(-2-2C)+x(3-3B-D)+(-2-3C-E)=0\quad.
\]
As $x$ is an indeterminate, the above expression may vanish only if all coefficients in the preceding expression vanish. Therefore we get the system
\[
\left| \begin{array}{rcl}
2-2B&=&0\\
-2-C&=&0\\
3-3B-D&=&0\\
-2-3C-E&=&0\quad .
\end{array}   \right.
\]
We may solve the above linear system using the standard algorithm for solving linear systems (the algorithm is called row reduction and is also known as Gaussian elimination). The latter algorithm is studied in any standard the Linear algebra course. Alternatively, we see from the first equations $B=1$, $C=-1$, and substituting in the remaining equations we see $D=0$, $E=1$. Finally, we check that
\[
\frac{x^{6}-x^{5}+\frac{9}{2} x^{4}-4 x^{3}+\frac{13}{2} x^{2}-\frac{7}{2} x+\frac{11}{4}}{x^{5}-x^{4}+3 x^{3}-3 x^{2}+\frac{9}{4} x-\frac{9}{4}}
=x+\frac{1}{(x-1)}+\frac{x-1}{(2x^2+3)}+\frac{1}{(2x^2+3)^2}\quad .
\]
\textbf{Step 3.} (Find the integral of each partial fraction).
\[
\begin{array}{rcl}
\displaystyle\int x \diff x &= &\displaystyle \frac{x ^2}2+C\\
\displaystyle\int \frac{1}{x-1} \diff x &=&\displaystyle  \ln|x-1|+C\\
\displaystyle\int \frac{x-1}{2x^2+3} \diff x &=& \displaystyle \int\frac{x}{2x^2+3}\diff x -\frac{1}{3}\int \frac{1}{\frac23x^2+1}\diff x\\
&=& \displaystyle  \int \frac{\diff\left(\frac{x^2}2\right) }{2x^2+3}\diff x-\frac{1}{3} \int \frac{1}{\left(\sqrt{\frac23}x\right)^2+1} \diff x\\
&=&\displaystyle  \frac{1}{4} \int \frac{\diff(2x^2+3)}{2x^2+3}\diff x- \frac{1}{3}\int \frac{\frac{\diff \left(\sqrt{\frac23}x\right)} {\sqrt{\frac23}}} {\left( \sqrt{\frac23}x\right)^2+1} \\
&=&\displaystyle   \frac{1}{4}\ln (2x^2+3)-\frac{\sqrt{6}}{6}\Arctan \left(\sqrt{\frac23}x\right)+C
\quad .
\end{array}
\]
The last integral is
\[
\begin{array}{rcll|l}
\displaystyle \int \frac{1}{(2x^2+3)^2}\diff x&=&\displaystyle  \frac{1}{9} \int \frac{\frac{\diff\left(\sqrt{\frac{2}{3}}x\right)}{\sqrt{\frac23}}}{\left(\left(\sqrt{\frac23}x\right)^2+1\right)^2}\\
&=&\displaystyle  \frac{\sqrt{6}}{18}\int \frac{\diff\left(\sqrt{\frac23}x\right)}{\left(\left(\sqrt{\frac23}x\right)^2+1\right)^2} &&\text{Set }y=\sqrt{\frac23}x\\
&=&\displaystyle  \frac{\sqrt{6}}{18}\int \frac{\diff y}{(y^2+1)^2}\quad.
\end{array}
\]
The general form of the integral $\displaystyle\int \frac{\diff y}{(y^2+1)^2}$ \refBad{\ref{eqBuildingBlock3N}}{is solved in the theoretical discussion}{is solved in \eqref{eqBuildingBlock3N}} by integration by parts. As a review of the theory, we redo the computations directly.
\[
\begin{array}{rcl}
C+\arctan y &=&\displaystyle \int \frac{\diff y}{y^2+1}\\
&=&\displaystyle \frac{y}{y^2+1} +\int \frac{2y^2dy }{( y^2+ 1 )^2}= \frac{y}{y^2+1}+\int \frac{2(y^2+1-1)\diff y}{(y^2+1)^2}\\
&=&\displaystyle \frac{y}{y^2+1} + 2\int \frac{\diff y}{ ( y^2 +1)}- 2\int \frac{\diff y}{(y^2+1)^2}\quad.
\end{array}
\]
Transferring summands we get
\[
\int \frac{\diff y}{(y^2+1)^2}= \frac{1}{2} \left( \frac{ y }{y^2+1} +\arctan y\right) +C\quad .
\]
We recall that $y=\sqrt{\frac{2}{3}}x$ and therefore
\[
\int \frac{\diff x}{(2x^2+3)^2}=\frac{\sqrt{6}}{36}\left(\frac{\sqrt{\frac{2}{3}}x}{\left(\sqrt{\frac{2}{3}}x\right)^2+1}+\arctan \left( \sqrt{\frac{2}{3}}x\right)\right) +C
\]

To get the final answer we need to collect all terms, to get a final answer:
\[
\frac{1}{6}\left(\frac{x}{2x^2+3}\right) - \frac{5\sqrt{6}}{36} \arctan \left(\sqrt{\frac{2}{3}}x \right) +\frac{1}{4} \ln (2x^2+3) +\ln|x-1|+\frac{x ^2 } 2+ C\quad .
\]
}