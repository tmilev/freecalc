%begin module Building block III

\begin{frame}
\frametitle{Building block III}
In the preceding slides we solved the partial case of building block integral III: $\int \frac{x}{x^2+1}\diff x$. The general case is not much harder.

\begin{example}
\[
\begin{array}{rcl}
\displaystyle \int \frac{\alert<2>{ x} }{(x^2+1)^n} \alert<2>{\diff x} \uncover<2->{&=&\displaystyle  \int \frac{1}{(\alert<3>{x^2+1})^n} \alert<2>{\frac{\diff \left(\alert<3>{x^2+1} \right )}{2}}} \\
\uncover<3->{&=& \displaystyle \frac{1}{2}\int {\alert<3>{u}}^{-n}\diff \alert<3>{ u}} \\
\uncover<4->{&=&\left\{\begin{array}{ll}\displaystyle
\uncover<5->{\alert<5>{ \frac{1}{2}\ln (x^2+1) +C}} & \alert<4,5>{\text{if }n=1} \\
\uncover<7->{\alert<7>{\displaystyle \frac{1}{2} \frac{(x^2+1)^{-n+1}}{(-n+1)}+C}} &\alert<6>{ \text{if }n\neq -1}
\end{array} 
\right. ,}
\end{array}
\]
\uncover<3->{where we used the substitution $\alert<3>{u=x^2+1}$.}

\end{example}

\end{frame}

%end module Building block III