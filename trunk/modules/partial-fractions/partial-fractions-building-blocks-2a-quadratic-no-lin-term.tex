\begin{frame}
\frametitle{Linear substitutions leading to block IIa }
Building block IIa: $ \int \frac{x}{1+x^2}\diff x = \frac{1}{2}\ln(1+x^2)+C$.

\uncover<2->{\alert<2>{``Theoretical way''} to solve example below: transform to IIa; this \alert<2>{is slow}. } \uncover<3->{ \alert<3>{\textbf{Feel free to skip slide, we will redo in next slide with a shortcut.}}}
\begin{example}
\[
\renewcommand{\arraystretch}{0}
\begin{array}{r{c}ll|l}
 \int \frac{ x }{2x^2+3} 
\diff x \uncover<4->{
&=& \int\frac{x}{3\left(\frac{2}{3} x^2+1\right)} \diff x = \int\frac{x}{3\left(\left(\sqrt{\frac{2}{3}} x\right)^2+1\right)} \diff x\\
&=& \frac{3}{2}\int\frac{\sqrt{\frac{2}{3}} x}{3\left(\left(\sqrt{\frac{2}{3}} x\right)^2+1\right)} \diff \left( \sqrt{\frac{2}{3}}x \right)  && \text{Set } u=\sqrt{\frac{2}{3}}x\\
&=&\frac{1}{2}\int \frac{u}{u^2+1}\diff u= \frac{1}{4} \ln (1+u^2)+C\\
&=& \frac{1}{4} \ln \left(\frac{1}{3} (2x^2+3)\right) +C\\
&=&\frac{1}{4} \ln (2x^2+3) + \frac{\ln \left(\frac{1}{3}\right)}{4}+C\\
&=&\frac{1}{4} \ln (2x^2+3)+K\quad .
}
\end{array}
\]

\end{example}



\vspace{4cm}

\end{frame}


\begin{frame}
\frametitle{Linear substitutions leading to blocks IIa}
Building block IIa: $ \int \frac{x}{1+x^2}\diff x = \frac{1}{2}\ln(1+x^2)+C$.

The example below can be done directly, without transforming to block IIa.
%Building block IIIa: $ \int \frac{1}{1+x^2 }\diff x=\Arctan x+C.$


\begin{example}
\[
\begin{array}{rcll|l}
\displaystyle \int \frac{\alert<2>{ x} }{2x^2+3} \alert<2>{ \diff x} \uncover<2->{&=&\displaystyle \int\frac{1}{2x^2+3} \alert<2,3>{ \diff \left( \frac{x^2}{2} \right)}} \\
\uncover<3->{&=&\displaystyle \int\frac{1}{\alert<5>{2x^2+3}} \alert<3,4>{\diff} \left(\frac{\alert<5>{\alert<3>{ 2x^2} \uncover<4->{\alert<4>{+3}}} }{ \alert<3>{4}}\right)} \uncover<5->{&&\text{Set } \alert<5,7>{ u=2x^2+3} } \\
\uncover<5->{&=&\displaystyle \frac{1}{4}\alert<6>{ \int \frac{1}{\alert<5>{u}}\diff \alert<5>{u}}}\\
\uncover<6->{&=&\displaystyle \frac{1}{4}\alert<6>{ \ln \alert<7>{|u|}}+C}\\
\uncover<7->{&=&\displaystyle \frac{1}{4}\ln (\alert<7>{2x^2+3})+C}
\end{array}
\]

\end{example}
\vspace{4cm}

\end{frame}