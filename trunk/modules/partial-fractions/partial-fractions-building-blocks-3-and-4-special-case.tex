%begin module partial-fractions-building-blocks-3-and-4-intro
\begin{frame}
\frametitle{Building blocks III and IV for $n=1$}
Special case building block III: $\int \frac{x}{1+x^2}\diff x.$

Special case building block IV: $\int \frac{1}{1+x^2 }\diff x.$
\begin{example}
Integrate 
\[
\begin{array}{rcl}
\displaystyle \int \frac{x}{1+x^2 }\diff x &=&\displaystyle \int \frac{1}{(1+x^2) }\frac{\diff (x^2)}{2} = \int \frac{1}{1+x^2 }\frac{\diff (1+x^2)}{2} \\
&=&\displaystyle\frac12\ln (1+x^2)+C\quad .
\end{array}
\]
\end{example}
\begin{example}
Integrate 
\[
\displaystyle \int \frac{1}{1+x^2 }\diff x =\Arctan x +C
\]
\end{example}
\vspace{2cm} 
\end{frame}
\begin{frame}
\frametitle{Linear substitutions leading to blocks III and IV}
Special case building block III: $ \int \frac{x}{1+x^2}\diff x = \frac{1}{2}\ln(1+x^2)+C$.

Special case building block IV: $ \int \frac{1}{1+x^2 }\diff x=\Arctan x+C.$


\begin{example}
\[
\begin{array}{rcll|l}
\displaystyle \int \frac{1}{x^2+2}\diff x&=&\displaystyle \int\frac{1}{2 \left(\frac{1}{2}x^2+1\right)}\diff x \\
&=&\displaystyle \int \frac{1}{2\left(\left(\frac{x}{\sqrt{2}}\right)^2+1  \right)} \sqrt{2}\diff\left(\frac{ x}{\sqrt{2}}\right) &&\text{ Set } u=\frac{x}{\sqrt{2}} \\
&=&\displaystyle \frac{\sqrt{2}}{2}\int \frac{1}{1+u^2}du\\
&=&\frac{\sqrt{2}}{2}\Arctan (u)+C \\&=&\frac{\sqrt{2}}{2} \Arctan\left(\frac{x}{\sqrt{2}}\right)+C
\end{array}
\]

\end{example}
\vspace{2cm}

\end{frame}

\begin{frame}
\frametitle{Linear substitutions leading to blocks III and IV}
Special case building block III: $ \int \frac{x}{1+x^2}\diff x = \frac{1}{2}\ln(1+x^2)+C$.

Special case building block IV: $ \int \frac{1}{1+x^2 }\diff x=\Arctan x+C.$


\begin{example}
\[
\begin{array}{rcll|l}
\displaystyle \int \frac{x}{2x^2+3}\diff x&=&\displaystyle \int\frac{1}{2x^2+3}\diff \left(\frac{x^2}{2}\right) \\
&=&\displaystyle \int\frac{1}{2x^2+3}\diff \left(\frac{2x^2+3}{4}\right)&&\text{Set } u=2x^2+3 \\
&=&\displaystyle \frac{1}{4}\int \frac{1}{u}du\\
&=&\frac{1}{4}\ln (u)+C\\
&=&\frac{1}{4}\ln (2x^2+3)+C
\end{array}
\]

\end{example}
\vspace{4cm}

\end{frame}
\begin{frame}
\frametitle{Linear substitutions leading to blocks III and IV}
Special case building block III: $ \int \frac{x}{1+x^2}\diff x = \frac{1}{2}\ln(1+x^2)+C$.

Special case building block IV: $\int \frac{1}{1+x^2 }\diff x=\Arctan x+C.$

\begin{itemize}

\item Let $ax^2+bx+c$ be a quadratic with no real roots.
\item Then $ax^2+bx+c$ can be transformed to the form $r(u^2+1)$ by a linear substitution $u=px+q$ (for some number $r, p, q$). 
\item To find such substitution start by completing  the square. 
\item After completing the square proceed as in preceding examples.
\item In this way, integrals of the form $\displaystyle \int \frac{Ax+B}{ax^2+bx+c} \diff x$ can be expressed via building blocks III and IV.

\item We show examples, and then proceed in full generality.
\end{itemize}
\vspace{5cm}
\end{frame}


\begin{frame}
\frametitle{Linear substitutions leading to blocks III and IV}
Special case building block III: $ \int \frac{x}{1+x^2}\diff x = \frac{1}{2}\ln(1+x^2)+C$.

Special case building block IV: $\int \frac{1}{1+x^2 }\diff x=\Arctan x+C.$


\begin{example}
\uncover<7->{Let \alert<7,27,30>{$u= x+ \frac{1}{2} $}}\uncover<16->{, let \alert<16,28>{$z=\frac{2u}{\sqrt{3}}$}.} Integrate 
\[
\begin{array}{rcl}
\displaystyle \int\frac{x}{x^2+\alert<3>{x}+1}\diff x 
\only<1>{{~~~~~~~~~~~~~~~~~~~~~~~~~~~~~~~~~~~~~~~~~~~~~~~~~~~~~~} {~~~~~~~~~~~~~~~~~~~~~~~~~~~~~~~~~~~~~~~~~~~~~~~~~~~~~~} }
\only<2-10>{&=& \displaystyle \int \frac{x}{\alert<4>{ x^2+\alert<3>{2\frac{1}{2}x} +\frac{1}{4} } \alert<5>{-\frac{1}{4} +1} } \alert<6>{\diff x}  {~~~~~~~~~~~~~~~~~~~~~~~~~~~~~~~~~~~~~~~~~~~~~~~~~~~~~~} \\
\uncover<4->{&=&\displaystyle \int \frac{\alert<7>{x+\frac{1}{2}}-\frac{1}{2}}{ \alert<4>{\left(\alert<7>{x+\frac{1}2}\right)^2}+\alert<5>{\frac{3}{4}} }\alert<6>{\diff \left(\alert<7>{x+\frac{1}{2}}\right) }}\\
\uncover<7->{&=&\displaystyle \int \frac{\alert<7,8>{u} \alert<9>{ -\frac{1}{2}} }{{\alert<7>{u}}^2+\frac{3}{4}}\diff \alert<7>{u}} \\
\uncover<8->{&=&\alert<10>{ \displaystyle \int \frac{\alert<8>{u}}{u^2+\frac{3}{4}}\diff u\alert<9>{-\frac{1}{2}}\int \frac{1}{u^2+\frac{3}{4}}\diff u}}
}

\only<11->{
&=&\alert<11>{\displaystyle\alert<25>{ \int \frac{u}{u^2+\frac{3}{4}}\diff u} -\frac{1}{2} \alert<12,18>{\int \frac{1}{u^2+\frac{3}{4}}\diff u}} {~~~~~~~~~~~~~~~~~~~~~~~~~~~~~~~~~~~~~~~~~~~~~~~~~~~~~~} \\
}
\only<11-18>{
\only<12->{\displaystyle \alert<12,18>{ \int \frac{1}{\alert<13>{u^2+\frac{3}{4}} }\diff u}\uncover<13->{&=& \displaystyle \int \frac{1}{\alert<13>{\frac{3}{4}\left( \alert<14>{\frac{4}{3}u^2} +1\right)}}\alert<15>{\diff u}}}\\
\uncover<14->{&=&\displaystyle \int \frac{1}{\frac{3}{4}\left(\alert<14>{ \left(\alert<16>{ \frac{2u}{ \sqrt{3}}}\right)^2} +1\right)} \alert<15>{ \frac{\sqrt{3} }{2} \diff \left(\alert<16>{ \frac{2u}{\sqrt{3}}}\right) }}\\
\uncover<16->{&=&\displaystyle \frac{2\sqrt{3}}{3}\int \frac{1}{\alert<16>{z}^2+1}\diff \alert<16>{z}}\uncover<17->{ = \alert<18>{\frac{2\sqrt{3}}{3} \Arctan z}+C}
}
\only<19->{
&=&\displaystyle \only<19-24>{\alert<20>{\int \frac{u}{u^2+\frac{3}{4}}\diff u}} \only<25->{\alert<25>{ \frac{1}{2}\ln \left(\alert<27>{u}^2+\frac{3}{4}\right) } }-\frac{1}{2}\alert<19>{ \frac{2\sqrt{3}}{3} \Arctan \alert<28>{z} }+C\\
\only<26->{
\uncover<27->{&=&\displaystyle \frac{1}{2}\ln  \left( \alert<29>{ {\alert<27>{\left(x+\frac{1}{2}\right)}}^2 + \frac{3}{4}}\right) - \frac{\sqrt{3}}{3} \Arctan \left( \alert<28>{ \frac{\alert<30>{2u }}{ \sqrt{3}}} \right)+C} \\
\uncover<29->{&=&\displaystyle \frac{1}{2}\ln \left(\alert<29>{x^2+x+1}\right) - \frac{\sqrt{3}}{3} \Arctan \left(\frac{\alert<30>{2x+1}}{\sqrt{3}}\right)+C
}
}
}
\only<20-25>{
\displaystyle \alert<20,25>{\int \frac{\alert<21>{ u} }{u^2+\frac{3}{4}}\alert<21>{ \diff u} } \uncover<21->{ &=& \displaystyle\int \frac{1}{u^2+\frac{3}{4}}\diff \alert<21>{\left(\frac{u^2}{\alert<22>{2}}\right)}}\\
\uncover<22->{&=&\displaystyle \alert<22>{\frac{1}{2}}\int \frac{1}{\alert<24>{u^2+\frac{3}{4}}}\diff \left(\alert<24>{ u^2\uncover<23->{ \alert<23>{ +\frac{3}{4}} }} \right)}\uncover<24->{ =\alert<25>{ \frac{1}{2}\ln \left(\alert<24>{u^2 +\frac{3}{4}} \right)}+C}
}
\end{array}
\]
\end{example}

\vspace{8cm}

\end{frame}

%end module partial-fractions-building-blocks-3-and-4-intro