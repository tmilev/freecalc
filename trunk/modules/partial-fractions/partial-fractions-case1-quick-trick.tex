% begin module partial-fractions-case1-quick-trick
\begin{frame}
NOTE:  There is a quick trick to find $A, B$, and $C$.
\abovedisplayskip=2pt
\belowdisplayskip=0pt
\[
x^2 + 2x - 1 = A(\alert<handout:0| 3>{2x-1})(\alert<handout:0| 4>{x+2}) + B\alert<handout:0| 2>{x}(\alert<handout:0| 4>{x+2}) + C\alert<handout:0| 2>{x}(\alert<handout:0| 3>{2x-1})
\]
\begin{columns}[t]
\column{.4\textwidth}
\begin{itemize}
\item<2-| alert@5-7>  To find $A$, plug in \alert<handout:0| 2>{$0$}.
\item<3-| alert@8-10>  To find $B$, plug in \alert<handout:0| 3>{$\frac{1}{2}$}.
\item<4-| alert@11-13>  To find $C$, plug in \alert<handout:0| 4>{$-2$}.
\end{itemize}
\column{.6\textwidth}
\abovedisplayskip=0pt
\belowdisplayskip=0pt
\begin{eqnarray*}
\uncover<8->{%
\left( \frac{1}{2}\right)^2 + 2\cdot \frac{1}{2} - 1%
}%
& \uncover<8->{ = } & %
\uncover<8->{%
B\left( \frac{1}{2}\right)\left(\frac{1}{2} + 2\right)%
}\\%
\uncover<9->{%
 \frac{1}{4}%
}%
& \uncover<9->{ = } & %
\uncover<9->{%
\frac{5}{4} B%
}\\%
\uncover<10->{%
 B%
}%
& \uncover<10->{ = } & %
\uncover<10->{%
\frac{1}{5}%
}\\%
\end{eqnarray*}

\end{columns}
\begin{columns}[t]
\column{.3\textwidth}
\begin{eqnarray*}
\uncover<5->{%
0^2 + 2\cdot 0 - 1%
}%
& \uncover<5->{ = } & %
\uncover<5->{%
A(2\cdot 0 - 1)(0 + 2)%
}\\%
\uncover<6->{%
 - 1%
}%
& \uncover<6->{ = } & %
\uncover<6->{%
-2 A%
}\\%
\uncover<7->{%
 A%
}%
& \uncover<7->{ = } & %
\uncover<7->{%
\frac{1}{2}%
}\\%
\end{eqnarray*}

\column{.7\textwidth}
\vspace{.5in}
\begin{eqnarray*}
\uncover<11->{%
(-2)^2 + 2(-2) - 1%
}%
& \uncover<11->{ = } & %
\uncover<11->{%
C(-2)(2(-2) - 1)%
}\\%
\uncover<12->{%
 - 1%
}%
& \uncover<12->{ = } & %
\uncover<12->{%
10C%
}\\%
\uncover<13->{%
 C%
}%
& \uncover<13->{ = } & %
\uncover<13->{%
-\frac{1}{10}%
}\\%
\end{eqnarray*}
\end{columns}
\end{frame}
% end module partial-fractions-case1-quick-trick
