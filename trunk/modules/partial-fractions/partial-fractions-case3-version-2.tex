% begin module partial-fractions-case3
\begin{frame}\frametitle{Case 3: Distince Irreducible Quadratics}
%\begin{enumerate}
%\setcounter{enumi}{2}
%\item  
Suppose $Q(x)$ contains irreducible quadratic factors, none of which is repeated.
%\end{enumerate}

If $Q(x)$ has the factor $ax^2 + bx + c$, where $b^2-4ac < 0$, then, in addition to the partial fractions arising from linear factors, the expression for $R(x)/Q(x)$ will have a term of the form
\[
\frac{Ax+B}{ax^2+bx+c}
\]

This term can be integrated by completing the square and using the formula
\[
\int \frac{\diff x}{x^2+a^2} = \frac{1}{a} \Arctan \left( \frac{x}{a}\right) + C
\]
\end{frame}
% end module partial-fractions-case3
