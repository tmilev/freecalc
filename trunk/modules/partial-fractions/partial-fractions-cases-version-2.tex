% begin module partial-fractions-cases-ver2
\begin{frame}\frametitle{Step 2: Factor the denominator}
The second step is to factor the denominator $Q(x)$ as much as possible.  \\
We state the Fundamental Theorem of algebra without proving it.
\begin{theorem}[The Fundamental Theorem of Algebra]
Every polynomial has at least one complex root.
\end{theorem}
\begin{corollary} [Corollary to the Fundamental Theorem of Algebra]
Let $Q(x)$ be a polynomial (with real coefficients). Then $Q(x)$ can be factored as a product of terms of the form $(ax+b)^n$ (powers of linear terms) and product of terms of the form $(ax^2+bx+c)^n$ with $b^2-4ac<0$ (powers of irreducible quadratic terms).
\end{corollary}  

\end{frame}



\begin{frame}\frametitle{Step 3: Express in partial fraction form}
The third step is to express the rational function $R(x)/Q(x)$ as a sum of partial fractions of the form
\[
\frac{A}{(ax+b)^i} \qquad \textrm{or}\qquad \frac{Ax+B}{(ax^2+bx+c)^j}
\]

\uncover<2->{%
There are four cases that occur:
\begin{enumerate}
\item  $Q(x)$ is a product of distinct linear factors.
\item  $Q(x)$ is a product of linear factors, some of which are repeated.
\item  $Q(x)$ contains irreducible quadratic factors, none of which is repeated.
\item  $Q(x)$ contains irreducible quadratic factors, some of which are repeated.
\end{enumerate}
}%


\end{frame}
% end module partial-fractions-cases-ver2
