% begin module partial-fractions-intro
\begin{frame}
\frametitle{Integration of Rational Functions by Partial Fractions}

A rational function is a function which can be written as ratio of polynomials, $\frac{P(x)}{Q(x)}$. To integrate any rational function, we will express $\frac{P(x)}{Q(x)}$ as a sum of simpler expressions called partial fractions. We start with an example. 

Put $2/(x-1)$ and $1/(x+2)$ over a common denominator:
\[
\alert<5>{\frac{2}{x-1} - \frac{1}{x+2} } = %
\uncover<2->{%
\frac{2(x+2) - (x-1)}{(x-1)(x+2)} = %
}%
\uncover<3->{%
\alert<5>{ \frac{x + 5}{x^2+x-2} }%
}%
\]

\uncover<4->{%
We can now solve the following integral:
\[
\int \alert<5>{ \frac{x+5}{x^2+x-2}}\diff x = %
\uncover<5->{%
\int \left(\alert<5>{\frac{2}{x-1} - \frac{1}{x+2}} \right) \diff x = %
}%
\uncover<6->{%
2\ln | x - 1| - \ln | x + 2| + C
}%
\]
}%

\end{frame}
% end module partial-fractions-intro
