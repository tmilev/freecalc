% begin module partial-fractions-long-division
\begin{frame}
\frametitle{Review of polynomial notation}
\begin{itemize}
\item Recall that a rational function is a function of the form
\[
f(x) = \frac{P(x)}{Q(x)}
\]
where $P$ and $Q\neq 0$ are polynomials. 
\item<2->Recall that the degree of $P$ is the highest power of $x$ in $P$ that has a non-zero coefficient.
\end{itemize}
\end{frame}

\begin{frame}\frametitle{Ensure denominator degree $>$ numerator degree}
\begin{itemize}
\item To decompose $\frac{P(x)}{Q(x)}$ in partial fractions we ensure first the degree of the numerator is smaller than the degree of the denominator.
\item<2-> We recall that to divide the \alertNoH{4}{dividend $P(x)$} by the \alertNoH{5}{divisor $Q(x)$} to get \alertNoH{6}{quotient $S(x)$} with \alertNoH{7}{remainder $R(x)$} means \uncover<3->{ to find polynomials  $S(x), R(x)$ such that \alertNoH{7}{$\deg R<\deg Q$} and
\[
\begin{array}{rcll|l}
\displaystyle \alertNoH{4} {P(x)}& =&\displaystyle  \alertNoH{6}{S(x)} \alertNoH{5}{Q(x)} + \alertNoH{7}{ R(x)} \uncover<8->{&&\alertNoH{8}{ \text{divide by } Q(x)}}\\
\displaystyle \uncover<8->{\frac{P(x)}{\alertNoH{8}{Q(x)} }&=&\displaystyle  \frac{S(x)\fcCancel{9}{Q(x)} }{\fcCancel{9}{\alertNoH{8}{Q(x)}}} +\frac{R(x)}{ \alertNoH{8}{ Q(x)}}} \\
\uncover<9->{
\displaystyle \frac{P(x)}{Q(x) }&=&\displaystyle  S(x) +\frac{R(x)}{ Q(x)}} \\
\end{array}
\]
}
\item<10-> The above transforms $\frac{P(x)}{Q(x)}$ to a polynomial plus a fraction in which the numerator has degree smaller than the denominator.
\item<11-> The polynomials $Q(x)$ and $S(x)$ are computed via polynomial long division. We recall the procedure through examples.
\end{itemize}
\end{frame}
% end module partial-fractions-long-division
