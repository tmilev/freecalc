\begin{frame}
\begin{itemize}
\item The next step in producing a partial fraction decomposition is to factor the denominator $Q(x)$.
\item Factoring of $Q(x)$ can always be done in quadratic and linear terms: 
\begin{corollary} [Corollary to the Fundamental Theorem of Algebra]
Let $Q(x)$ be a polynomial (with real coefficients). Then $Q(x)$ can be factored as a product of terms of the form $(ax+b)^n$ (powers of linear terms) and product of terms of the form $(ax^2+bx+c)^n$ with $b^2-4ac<0$ (powers of quadratic terms).
\end{corollary}  
\item The above result is a corollary to the Fundamental Theorem of Algebra. We state the Fundamental Theorem of algebra without proving it.
\begin{theorem}[The Fundamental Theorem of Algebra]
Every polynomial has at least one complex root.
\end{theorem}
\end{itemize}
\end{frame}

\begin{frame}
Suppose we have already factored the denominator $Q(x)$ into factors of the form 
\[
(ax+b)^N\qquad \text{ and }\qquad (ax^2+bx+c)^N
\]
\uncover<2->{Then we can split the fraction $R(x)/Q(x)$ into sum of partial fractions of the form 
\[
\frac{A}{(ax+b)^i} \qquad \text{or}\qquad \frac{Ax+B}{(ax^2+bx+c)^i}\quad ,
\]
where the exponent $i$ in the partial fraction does not exceed the exponent $N$ of the corresponding term in $Q(x)$.
}

\uncover<3->{
The cases when the factorization of $Q(x)$ has terms appearing with power $N>1$ are treated differently from the the cases where all terms of the factorization of $Q(x)$ are distinct.
}
\end{frame}
