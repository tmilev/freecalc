\solution{\ref{problemWriteInPolarx+sqrt(3)y=2}

Polar coordinates are given by
\[
\left|\begin{array}{rcl}
x&=& r\cos \theta\\
y&=& r\sin \theta 
\end{array}\right. .
\]
All we need to do to obtain polar equations for our line is substitute the above expressions in the equation for the line. 
\[
r\cos \theta + \sqrt{3}r\sin \theta= 2.
\]
This is a perfectly good answer, but we can transform the equation to make it look more compact:
\[
\begin{array}{rcll|l}
\displaystyle r\cos \theta + \sqrt{3}r\sin \theta&=&\displaystyle 2\\
\displaystyle r\underbrace{\frac{1}{2}}_{=\cos \left(\frac{\pi}{3}\right)} \cos\theta +r \underbrace{\frac{\sqrt{3}}{2}}_{=\sin \left(\frac{\pi}{3}\right)} \sin\theta &=&\displaystyle  1 \\
\displaystyle r\cos \left(- \frac{\pi }{3}\right)\cos \theta -\sin \left(- \frac{\pi }{3}\right)\sin \theta &=& 1&&   \text{use } \cos (a+b)= \cos a \cos b - \sin a \sin b\\
\displaystyle r\cos (\theta -\frac{\pi}{3})&=&1\\
r&=&\displaystyle \frac{1}{\cos \left(\theta - \frac{\pi}{3} \right)}\\
&=&\displaystyle \sec\left (\theta - \frac{\pi}{3}\right )\quad .

\end{array}
\]


}