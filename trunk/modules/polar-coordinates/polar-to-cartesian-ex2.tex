% begin module polar-to-cartesian-ex2
\begin{frame}
\begin{example} 
Convert the point $(\alert<handout:0| 4>{2}, \alert<handout:0| 6>{\frac{\pi}{3}})$ from polar to Cartesian coordinates.
\[
\uncover<2->{%
x = \alert<handout:0| 3-4>{r}\cos \alert<handout:0| 5-6>{\theta} = %
}%
\uncover<3->{%
\alert<handout:0| 3-4>{\uncover<4->{2}}\alert<handout:0| 7-8>{ \cos } \alert<handout:0| 5-8>{\uncover<6->{\frac{\pi}{3}}} %
}%
\uncover<7->{%
= 2\left( \alert<handout:0| 7-8>{\uncover<8->{\frac{ 1}{ 2 } } } \right) \uncover<9->{ = 1}%
}%
\]
\[
\uncover<2->{%
y = r\sin \theta = %
}%
\uncover<10->{%
2\alert<handout:0| 11-12>{\sin \frac{\pi}{3}}%
}%
\uncover<11->{%
= 2\left( \alert<handout:0| 11-12>{ \uncover<12->{ \frac{ \sqrt{ 3 }}{2}}}\right) \uncover<13->{ = \sqrt{3}}%
}%
\]
\uncover<14->{%
Therefore the point with polar coordinates $(2,\frac{\pi}{3})$ has Cartesian coordinates $(1,\sqrt{3})$.
}%
\end{example}
\end{frame}
% end module polar-to-cartesian-ex2
