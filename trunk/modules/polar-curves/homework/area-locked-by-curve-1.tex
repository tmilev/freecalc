\begin{enumerate}
\item \label{problem-Area-swept-by-r=1+sin2theta} The curve given in polar coordinates by $r=1+\sin 2\theta$ is plotted below by computer. Find the area lying outside of this curve and inside of the circle $x^2+y^2=1$.
\psset{xunit=1cm, yunit=1cm}
\begin{pspicture}(-2.016386, -2.016424)(2.016407,2.116335) 
\tiny 
\psaxesStandard{-1.766386}{-1.766424}{1.766407}{1.766335}
%Calculator command: drawPolar{}(\sin{}(2 t)+1, 0, 2 \pi) 
\parametricplot[linecolor=\psColorGraph, plotpoints=1000, algebraic=false]{0}{6.28319}{ 1 t 2 mul 57.29578 mul sin add t 57.29578 mul cos mul 1 t 2 mul 57.29578 mul sin add t 57.29578 mul sin mul }
\end{pspicture} 
\answer{$A=2-\frac{\pi}{4}$}
\item \label{problem-Area-swept-by-r=cos2theta} The curve given in polar coordinates by $r=\cos (2\theta)$ is plotted below by computer. Find the area lying inside the curve and outside of the circle $x^2+y^2=\frac14$.
\psset{xunit=1cm, yunit=1cm}
\begin{pspicture}(-1.399902, -1.399975)(1.4,1.499975) 
\tiny 
\psaxesStandard{-1.149902}{-1.149975}{1.15}{1.149975}
%Calculator command: drawPolar{}(\cos{}(2 t), 0, 2 \pi) 
\parametricplot[linecolor=\psColorGraph, plotpoints=1000, algebraic=false]{0}{6.28319}{t 2 mul 57.29578 mul cos t 57.29578 mul cos mul t 2 mul 57.29578 mul cos t 57.29578 mul sin mul }
\end{pspicture} 
\end{enumerate}
\solution{~
\begin{itemize}
\item[\ref{problem-Area-swept-by-r=1+sin2theta}] A computer generated plot of the area is drawn below. The circle $x^2+y^2=1$ has one-to-one polar representation given by $r=1, \theta\in [0,2\pi)$. Except the origin, which is traversed four times by the curve $r=1+\sin (2\theta)$, the second curve is in a one-to-one correspondence with points in the $r,\theta$-plane given by the equation $r=1+\sin (2\theta), \theta\in [0,2\pi)$. Since the two curves do not intersect in the origin, we may conclude that the two curves may intersect only when their values for $r$ and $\theta$ coincide. Therefore the two curves intersect when 
\[\begin{array}{rcll|l}
1+\sin (2\theta)&=&1\\
\sin (2\theta)&=&0\\
\theta &=& 0,\frac{\pi}{2}, \pi, \frac{3\pi}{2}&&\text{because } \theta\in [0,2\pi) \\
\end{array}
\]
Therefore the two curves intersect in the points $(0,1)(-1,0)$ and $(0,-1),(1,0)$. 

Let the region whose area we are seeking to compute be denoted by $A$. From the computer-generated plot, it is clear that when a point has polar coordinates $\theta\in [\frac{\pi}{2}, \pi] \cup[\frac{3\pi}{2}, 2\pi]$, $r\in [1+\sin(2\theta),1]$ it lies in $A$. Furthermore, the points $r,\theta$ satisfying the above inequalities are in one-to-one correspondence with the points in $A$. Therefore the area of $A$ is computed via the integrals

\[
\begin{array}{rcll|l}
A&=&\displaystyle \int\limits_{\frac{\pi}{2}}^{\pi} \frac{1}{2} \left(1^2- (1+\sin(2\theta))^2 \right)\diff \theta + \int \limits_{ \frac{3\pi}{2}}^{ 2\pi} \frac{1}{2} \left(1^2- (1+\sin(2\theta) )^2 \right) \diff \theta &&\text{use the symmetry of } A\\
&=&\displaystyle  \int\limits_{\frac{\pi}{2}}^{\pi} \left(1^2-(1+\sin(2\theta))^2\right)\diff \theta= \int\limits_{\frac{\pi}{2}}^{\pi} \left( - 2\sin(2\theta) - \sin^2(2\theta)\right) \diff \theta  &&\text{use } \sin^2 z=\frac{1-\cos (2z)}{2} \\
&=&\displaystyle  \int\limits_{\frac{\pi}{2}}^{\pi}  \left( -2\sin(2\theta) -\frac{1}{2} +\frac{1}{2}\cos (4\theta)\right)\diff \theta = \left[\cos (2\theta) -\frac{1}{2}\theta -\frac{1}{8}\sin (4\theta) \right]_{\frac{\pi}{2}}^{\pi} \\
&=&2-\frac{\pi}{4}
\end{array}
\]

\psset{xunit=1cm, yunit=1cm}
\begin{pspicture}(-2.016386, -2.016424)(2.016407,2.116335) 
\tiny 
\psaxesStandard{-1.766386}{-1.766424}{1.766407}{1.766335}
\pscustom*[linecolor=\psColorAreaUnderGraph]{
%Calculator command: drawPolar{}(1, 1/2 \pi, \pi) 
\parametricplot[linecolor=\psColorGraph, plotpoints=1000, algebraic=false]{1.5708}{3.14159}{ 1 t 57.29578 mul cos mul 1 t 57.29578 mul sin mul }
%Calculator command: drawPolar{}(\sin{}(2 t)+1, 1/2 \pi, \pi) 
\parametricplot[linecolor=\psColorGraph, plotpoints=1000, algebraic=false]{1.5708}{3.14159}{ 1 t 2 mul 57.29578 mul sin add t 57.29578 mul cos mul 1 t 2 mul 57.29578 mul sin add t 57.29578 mul sin mul }
} %pscustom
\pscustom*[linecolor=\psColorAreaUnderGraph]{
%Calculator command: drawPolar{}(\sin{}(2 t)+1, -1/2 \pi, 0) 
\parametricplot[linecolor=\psColorGraph, plotpoints=1000, algebraic=false]{-1.5708}{0}{ 1 t 2 mul 57.29578 mul sin add t 57.29578 mul cos mul 1 t 2 mul 57.29578 mul sin add t 57.29578 mul sin mul }
%Calculator command: drawPolar{}(1, -1/2 \pi, 0) 
\parametricplot[linecolor=\psColorGraph, plotpoints=1000, algebraic=false]{-1.5708}{0}{ 1 t 57.29578 mul cos mul 1 t 57.29578 mul sin mul }
} %pscustom

%Calculator command: drawPolar{}(1, 0, 2 \pi) 
\parametricplot[linecolor=\psColorGraph, plotpoints=1000, algebraic=false]{0}{6.28319}{ 1 t 57.29578 mul cos mul 1 t 57.29578 mul sin mul }
%Calculator command: drawPolar{}(\sin{}(2 t)+1, 0, 2 \pi) 
\parametricplot[linecolor=\psColorGraph, plotpoints=1000, algebraic=false]{0}{6.28319}{ 1 t 2 mul 57.29578 mul sin add t 57.29578 mul cos mul 1 t 2 mul 57.29578 mul sin add t 57.29578 mul sin mul }
\end{pspicture} 
\item[\ref{problem-Area-swept-by-r=cos2theta}] A computer generated plot of the area is drawn below. 

Points with polar coordinates $(r_1, \theta_1) $ and $(r_2,\theta_2)$ coincide if one of the three holds:
\begin{itemize}
\item $r_1=r_2\neq 0$ and $\theta_1=\theta_2+2k\pi, k\in \mathbb Z $,
\item $r_1=-r_2\neq 0$ and $\theta_1=\theta_2+(2k+1)\pi, k\in \mathbb Z$,
\item $r_1=r_2=0 $ and $\theta$ is arbitrary.
\end{itemize}
Two curves given in polar coordinates via $r=f(\theta)$

\psset{xunit=2cm, yunit=2cm}
\begin{pspicture}(-1.399902, -1.399975)(1.4,1.499975) 
\tiny 
\pscustom*[linecolor=\psColorAreaUnderGraph]{
%Calculator command: drawPolar{}(1/2, 1/3 \pi, 2/3 \pi) 
\parametricplot[linecolor=\psColorGraph, plotpoints=1000, algebraic=false]{1.0472}{2.0944}{ 0.5 t 57.29578 mul cos mul 0.5 t 57.29578 mul sin mul }
%Calculator command: drawPolar{}(\cos{}(2 t), 5/3 \pi, 7/4 \pi) 
\parametricplot[linecolor=\psColorGraph, plotpoints=1000, algebraic=false]{5.23599}{5.49779}{t 2 mul 57.29578 mul cos t 57.29578 mul cos mul t 2 mul 57.29578 mul cos t 57.29578 mul sin mul }
%Calculator command: drawPolar{}(\cos{}(2 t), 7/6 \pi, 5/4 \pi) 
\parametricplot[linecolor=\psColorGraph, plotpoints=1000, algebraic=false]{3.66519}{3.92699}{t 2 mul 57.29578 mul cos t 57.29578 mul cos mul t 2 mul 57.29578 mul cos t 57.29578 mul sin mul }
%Calculator command: drawPolar{}(\cos{}(2 t), 2/3 \pi, 3/4 \pi) 
\parametricplot[linecolor=\psColorGraph, plotpoints=1000, algebraic=false]{2.0944}{2.35619}{t 2 mul 57.29578 mul cos t 57.29578 mul cos mul t 2 mul 57.29578 mul cos t 57.29578 mul sin mul }
%Calculator command: drawPolar{}(1/2, -1/6 \pi, 1/6 \pi) 
\parametricplot[linecolor=\psColorGraph, plotpoints=1000, algebraic=false]{-0.523599}{0.523599}{ 0.5 t 57.29578 mul cos mul 0.5 t 57.29578 mul sin mul }
%Calculator command: drawPolar{}(\cos{}(2 t), 1/6 \pi, 1/4 \pi) 
\parametricplot[linecolor=\psColorGraph, plotpoints=1000, algebraic=false]{0.523599}{0.785398}{t 2 mul 57.29578 mul cos t 57.29578 mul cos mul t 2 mul 57.29578 mul cos t 57.29578 mul sin mul }
%Calculator command: drawPolar{}(\cos{}(2 t), -1/4 \pi, -1/6 \pi) 
\parametricplot[linecolor=\psColorGraph, plotpoints=1000, algebraic=false]{-0.785398}{-0.523599}{t 2 mul 57.29578 mul cos t 57.29578 mul cos mul t 2 mul 57.29578 mul cos t 57.29578 mul sin mul }
%Calculator command: drawPolar{}(\cos{}(2 t), 1/4 \pi, 1/3 \pi) 
\parametricplot[linecolor=\psColorGraph, plotpoints=1000, algebraic=false]{0.785398}{1.0472}{t 2 mul 57.29578 mul cos t 57.29578 mul cos mul t 2 mul 57.29578 mul cos t 57.29578 mul sin mul }
%Calculator command: drawPolar{}(1/2, 4/3 \pi, 5/3 \pi) 
\parametricplot[linecolor=\psColorGraph, plotpoints=1000, algebraic=false]{4.18879}{5.23599}{ 0.5 t 57.29578 mul cos mul 0.5 t 57.29578 mul sin mul }
%Calculator command: drawPolar{}(\cos{}(2 t), 3/4 \pi, 5/6 \pi) 
\parametricplot[linecolor=\psColorGraph, plotpoints=1000, algebraic=false]{2.35619}{2.61799}{t 2 mul 57.29578 mul cos t 57.29578 mul cos mul t 2 mul 57.29578 mul cos t 57.29578 mul sin mul }
%Calculator command: drawPolar{}(1/2, 5/6 \pi, 7/6 \pi) 
\parametricplot[linecolor=\psColorGraph, plotpoints=1000, algebraic=false]{2.61799}{3.66519}{ 0.5 t 57.29578 mul cos mul 0.5 t 57.29578 mul sin mul }
%Calculator command: drawPolar{}(\cos{}(2 t), 5/4 \pi, 4/3 \pi) 
\parametricplot[linecolor=\psColorGraph, plotpoints=1000, algebraic=false]{3.92699}{4.18879}{t 2 mul 57.29578 mul cos t 57.29578 mul cos mul t 2 mul 57.29578 mul cos t 57.29578 mul sin mul }
}
\parametricplot[linecolor=\psColorGraph, plotpoints=1000, algebraic=false]{0}{6.28319}{ 0.5 t 57.29578 mul cos mul 0.5 t 57.29578 mul sin mul }
%Calculator command: drawPolar{}(\cos{}(2 t), 0, 2 \pi) 
\parametricplot[linecolor=\psColorGraph, plotpoints=1000, algebraic=false]{0}{6.28319}{t 2 mul 57.29578 mul cos t 57.29578 mul cos mul t 2 mul 57.29578 mul cos t 57.29578 mul sin mul }
\psaxes[ticks=none, labels=none, arrows = <->](0,0)(-1.149902,-1.149975)(1.15,1.149975)
\psLabels{1.15}{1.149975}
\end{pspicture} 
%Calculator command: drawPolar{}(1/2, 0, 2 \pi) 


\end{itemize}

}