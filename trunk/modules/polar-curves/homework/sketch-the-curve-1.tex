\begin{enumerate}
\item Sketch the curve given in polar coordinates by $r=2\sin \theta $. What kind of a figure is this curve? Find an equation satisfied by the curve in the $(x,y)$-coordinates.
\item Sketch the curve given in polar coordinates by $r=4\cos \theta $. What kind of a figure is this curve? Find an equation satisfied by the curve in the $(x,y)$-coordinates.
\item \label{problemPolarSketchr=2sec(theta)}  Sketch the curve given in polar coordinates by $r=2\sec \theta $. What kind of a figure is this curve? Find an equation satisfied by the curve in the $(x,y)$-coordinates.
\item Sketch the curve given in polar coordinates by $r=2\csc \theta $. What kind of a figure is this curve? Find an equation satisfied by the curve in the $(x,y)$-coordinates.
\item Sketch the curve given in polar coordinates by $r=2\sec \left(\theta+\frac{\pi}{4}\right) $. What kind of a figure is this curve? Find an equation satisfied by the curve in the $(x,y)$-coordinates.
\item Sketch the curve given in polar coordinates by $r=2\csc\left(\theta +\frac{\pi}{6}\right)$. What kind of a figure is this curve? Find an equation satisfied by the curve in the $(x,y)$-coordinates.

\end{enumerate}

\solution{\ref{problemPolarSketchr=2sec(theta)}. 
Recall from trigonometry that if we draw a unit circle as shown below, $\sec \theta$ is given by the signed distance as indicated on the figure. Therefore it is clear that the curve given in polar coordinates by $y=\sec \theta$ is the vertical line passing through $x=1$. Analogous considerations can be made for a circle of radius $2$, from where it follows that $y=2\sec \theta$ is the vertical line passing through $x=2$.  
\psset{xunit=1cm, yunit=1cm}
\begin{pspicture}(-1.39998, -1.399995)(1.4,2.7) 
\tiny 
\psaxesStandard{-1.14998}{-1.149995}{1.15}{2.6}
%Calculator command: drawPolar{}(1, 0, 2 \pi) 
\parametricplot[linecolor=\psColorGraph, plotpoints=1000, algebraic=false]{0}{6.28319}{ 1 t 57.29578 mul cos mul 1 t 57.29578 mul sin mul }
\psAngle{0}{1.107149}{0.2}{$\theta$}
\psline(0,0)(1,2)
\psline(1,-1)(1,2.6)
\psLengthIndicator{-0.1}{0.05}{0.9}{2.05}{}
\rput[r](0.5,1.1){$\sec \theta$}
\end{pspicture} 


}