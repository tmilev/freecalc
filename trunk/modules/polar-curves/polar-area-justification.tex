% begin module polar-area-justification
{% to make sure the newcommand below has limited scope.
\newcommand{\thePolarCurve}{t 2 div 1 add}
\begin{frame}[t]
\begin{columns}
\frametitle{Area swept by a polar curve: justification}
\column{0.5\textwidth}
\psset{xunit=1cm, yunit=1cm, algebraic=false}
\begin{pspicture}(-2.65,-1)(1.4,2.3)%
\tiny%
\psline[linecolor=red!1](1.4, 2.3)(1.39, 2.3)% 
\psline[linecolor=red!1](-2.65, -1)(-2.649, -1)% 
\rput[t](0,-0.1){$O$}%
\uncover<1>{
\pscustom*[linecolor=cyan]{%
\drawPolar{0}{3.5}{\thePolarCurve}{linecolor=red, plotpoints=1000}%
\psline(-2.575256, -0.964654)(0,0)(1,0)%
}%
}%
\rput[t](1.2,-0.1){$x$}%
\uncover<4-15>{%
\rput[b](-0.832294, 1.85){$P_2$}%
\rput[bl](0.84, 1.3){$P_1$}%
\psFullDot{\polarCurveEvaluateX{1}{\thePolarCurve}}{\polarCurveEvaluateY{1}{\thePolarCurve}}%
\psFullDot{\polarCurveEvaluateX{2}{\thePolarCurve}}{\polarCurveEvaluateY{2}{\thePolarCurve} }%
}%
\uncover<5-15>{%
\psline[linecolor=cyan](0,0)(! \polarCurveEvaluateXY{1}{\thePolarCurve})%
\psline[linecolor=cyan](0,0)(! \polarCurveEvaluateXY{2}{\thePolarCurve})%
}%
\uncover<7-15>{%
\polarWedge{1}{2}{\thePolarCurve}%
}%
\uncover<8,9,10>{%
\psline[linecolor=red, linewidth=2pt](0,0)(! \polarCurveEvaluateXY{1}{\thePolarCurve})%
\psline[linecolor=red, linewidth=2pt](0,0)(! \polarCurveEvaluateXY{2}{\thePolarCurve})%
}%
\uncover<5-15>{%
\rput[bl](0.50, 0.6){$r_1$}%
\rput[l](-0.45, 1.1){$r_2$}%
}%
\uncover<6-15>{%
\psAngle{1}{2}{0.7}{}%
\psAngle{0}{2}{0.45}{}%
\psAngle{0}{1}{0.15}{}%
\rput[bl](0.15, 0.05){$\alert<6>{\theta_1}$}%
\rput[b](0, 0.46){$\alert<6>{\theta_2}$}%
\rput[b](0, 0.73){$ \alert<9,10>{\Delta}$}%
}%
\uncover<11-15>{%
\polarWedgeSequence{0}{1}{3}{\thePolarCurve}%
}%
\uncover<16>{%
\polarWedgeSequence{0}{0.75}{4}{\thePolarCurve}%
}%
\uncover<17>{%
\polarWedgeSequence{0}{0.5}{6}{\thePolarCurve}%
}%
\uncover<18>{%
\polarWedgeSequence{0}{0.3}{10}{\thePolarCurve}%
}%
\uncover<19>{%
\polarWedgeSequence{0}{0.2}{15}{\thePolarCurve}%
}%
\uncover<20->{%
\polarWedgeSequence{0}{0.1}{30}{\thePolarCurve}%
}%
\drawPolar{0}{3.5}{\thePolarCurve}{linecolor=red, plotpoints=1000}%
\psline[arrows=->](0,0)(1.2, 0)%
\end{pspicture}

\column{0.5\textwidth}
\uncover<2->{Split $[a,b]$ into $N$ equal segments via points $a=\theta_0 \leq \theta_1 \leq \dots \leq \theta_{N-1} \leq \theta_N=b$.} \uncover<3->{The length of each segment is $\Delta=\frac{b-a}{N}$.} \uncover<4->{Let $r_i=f(\theta_i)$. Then each $\theta_i$ gives a point $P_i$ with polar coordinates $(\alert<5>{r_i},\alert<6>{\theta_i})$.}
\end{columns}

\only<1-12>{ \uncover<7->{The area swept by the curve is approximated by sum of areas of triangles given by connecting the origin with two consecutive vertices. Consider one such triangle, say, $OP_1P_2$.} \uncover<8->{By Euclidean geometry, the area of $\triangle OP_1P_2 $ is $\alert<8,9>{\frac{|OP_1| |OP_2| \uncover<8>{\alert<8>{ \textbf{?} }} \uncover<9->{\alert<9>{ \sin \Delta}}}{2}} \uncover<10->{=\frac{ r_1 r_2 \sin \Delta}{2}= \frac{ f(\theta_1) f(\theta_2) \sin \Delta}{2}} $.} 
}

\uncover<11->{\alert<12,13>{ Therefore the area swept by the curve \only<1-14>{is approximated by} \only<15->{\alert<15>{equals \alert<16->{the limit} of}} the sum:}
\[\begin{array}{rcl}
\uncover<15->{ A&=&}\alert<12,13>{\uncover<15->{\lim\limits_{\Delta\to 0}} \sum\limits_{i=0}^{N-1} \frac{f(\theta_i)f(\theta_{i+1}) \alert<14>{ \sin \Delta} }{2}} \uncover<14->{= \uncover<15->{ \lim \limits_{\Delta\to 0}} \alert<14>{ \frac{ \sin\Delta}{ \Delta}} \sum\limits_{i=0}^{N-1} \frac{ f( \theta_i) f(\theta_{i} + \Delta)\alert<14>{\Delta}}{2}}\\
\uncover<21->{\uncover<24>{\alert<24>{\text{{\tiny(can be proved)}}}}  &=&   \alert<22,23>{ \lim \limits_{\Delta\to 0}\frac{ \sin\Delta}{ \Delta  }}  \lim \limits_{\alert<24>{\Delta\to 0} } \sum\limits_{i=0}^{N-1} \frac{ f( \theta_i) f( \alert<24>{ \theta_{i} + \Delta} )\Delta}{2} \uncover<22->{=\uncover<22>{\alert<22>{\textbf{?}}} \uncover<23->{\alert<23>{1}}\cdot \lim \limits_{\Delta\to 0} \sum\limits_{i=0}^{N-1} \frac{\alert<25>{ f( \theta_i) f(\alert<24>{ \theta_{i}} )} \Delta}{2}}} \\
\uncover<25->{\uncover<27->{{\tiny\text{\alert<27>{(Riemann sum)}}}} &=& \lim \limits_{\Delta\to 0} \sum \limits_{i=0}^{N-1} \frac{ \alert<25>{ f^2( \theta_i)}  \Delta }{2}}\uncover<26->{=\uncover<26>{\alert<26>{\textbf{?}}} \uncover<27->{\alert<27>{  \int\limits_{a}^b \frac{ f^2(\theta)}{2}\diff \theta}} } 
\end{array}
\]
}
\end{frame}
}% end of scoping block
%end module polar-area-justification