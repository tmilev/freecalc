

\begin{frame}

\begin{pspicture*}(-3,-1)(1.5,2.3) %
\psline[linecolor=red!1](1.5, 2.3)(1.49, 2.3) 
\psline[linecolor=red!1](-3, -1)(2.99, -1) 
\uncover<1>{%
\polarWedgeSequence{0}{1}{3}{t 2 div 1 add}%
}%
\uncover<2>{%
\polarWedgeSequence{0}{0.75}{4}{t 2 div 1 add}%
}%
\uncover<3>{%
\polarWedgeSequence{0}{0.5}{6}{t 2 div 1 add}%
}%
\uncover<4>{%
\polarWedgeSequence{0}{0.3}{10}{t 2 div 1 add}%
}%
\uncover<5>{%
\polarWedgeSequence{0}{0.2}{15}{t 2 div 1 add}%
}%
\uncover<6->{%
\polarWedgeSequence{0}{0.1}{30}{t 2 div 1 add}%
}%
\drawPolar{0}{3.5}{t 2 div 1 add}{linecolor=red, plotpoints=1000}%
\end{pspicture*}
\uncover<1-6>{}

Let $r=f(\theta), \theta\in [a,b]$ be a curve given in polar coordinates.  Let $A$ denote the figure given by the union of the segments connecting each point the curve to the origin. Suppose no two points on the curve lie on the same ray from the origin. Then the area of $A$ is given by 
\[
A= \int\limits_{\theta=a}^b \frac{f(\theta)^2}{2}d\theta\quad .
\]

A motivation for the above definition may be given as follows. 

\end{frame}
