% begin module polar-area-justification
\begin{frame}
\begin{columns}
\column{0.4\textwidth}
\psset{xunit=0.6cm, yunit=0.6cm, algebraic=false}
\begin{pspicture*}(-3,-1)(1.5,2.3)%
\psline[linecolor=red!1](1.5, 2.3)(1.49, 2.3)% 
\psline[linecolor=red!1](-3, -1)(2.99, -1)% 
\uncover<1>{ %
\polarWedge{0}{1}{t 2 div 1 add}%
}%
\uncover<1>{%
\polarWedgeSequence{0}{1}{3}{t 2 div 1 add}%
}%
\uncover<2>{%
\polarWedgeSequence{0}{0.75}{4}{t 2 div 1 add}%
}%
\uncover<3>{%
\polarWedgeSequence{0}{0.5}{6}{t 2 div 1 add}%
}%
\uncover<4>{%
\polarWedgeSequence{0}{0.3}{10}{t 2 div 1 add}%
}%
\uncover<5>{%
\polarWedgeSequence{0}{0.2}{15}{t 2 div 1 add}%
}%
\uncover<6->{%
\polarWedgeSequence{0}{0.1}{30}{t 2 div 1 add}%
}%
\drawPolar{0}{3.5}{t 2 div 1 add}{linecolor=red, plotpoints=1000}%
\end{pspicture*}
\column{0.6\textwidth}

\end{columns}
\uncover<1-6>{}

Let $N$ be a large number and split the interval $[a,b]$ using $N+1$ equally spaced points $a=\theta_0\leq\theta_1\leq\dots\leq\theta_{N-1}\leq \theta_N=b$ into $N$ equal segments, each of the form $[\theta_j,\theta_{j+1}] $. Denote the length of each such segment by $\Delta\theta$, i.e., let $\Delta\theta= \frac{b-a}{N}$. 
Then the area of $A$ is approximated by triangles with vertices on the curve. Those are of the form $(f(\theta_j)\cos \theta_j, f(\theta_{j+1} )\cos \theta_{j+1})$ as drawn in the figure. Consider one such triangle, $OPQ$, as indicated in the figure. 
The area of triangle $OPQ $ is $\frac{|OP| |OQ|\sin (\Delta\theta)}{2}= \frac{f(\theta_1)f(\theta_2)\sin (\Delta\theta)}{2} $. In other words the area of $A$ is approximated by
\[
\sum_{j=0}^{N-1} \frac{f(\theta_j)f(\theta_{j+1})\sin (\Delta\theta)}{2}= \frac{\sin(\Delta\theta)}{\Delta\theta}\sum_{j=0}^{N-1} \frac{f(\theta_j)f(\theta_{j}+\Delta\theta)\Delta\theta}{2}\quad .
\]
In the above sum, the multiplicand $\frac{\sin(\Delta\theta)}{\Delta\theta}$ tends to $1$ as $\Delta\theta$ tends to $0$. Therefore as $\Delta$ tends to zero, the expression tends to the limit of the second multiplicand. On the other hand, one can show that the second multiplicand approximates the integral $\int\limits_{\theta=a}^b \frac{f(\theta)^2}{2}d\theta$.
\end{frame}
