\begin{frame}
\frametitle{Factoring polynomials in practice}
\begin{itemize}
\item In theory every polynomial can be factored.
\[a_0 x^{n}+a_1x^{n-1}+\dots + a_n=a_0 (x- \alertNoH{2}{x_1} )\dots (x-\alertNoH{2}{ x_n} )
\]
\item<2-> Theory guarantees numerical approximations for roots $\alertNoH{2}{ x_1,\dots, x_n} $. 
\item<3-> \alertNoH{3,4,6}{Can we find algebraic formulas} for $x_1, \dots, x_n$?
\item<4-> \alertNoH{4}{No}, if using finitely many operations $+,-,*,/,\sqrt[n]{~}$.
\item<5-> First (advanced) proof by Norwegian Niels Henrik Abel(1824) based on work of Italian Paolo Ruffini(1799). 
\item<6-> \footnotesize{ \alertNoH{6}{Yes}, with extra operations. Difficult: google Galois Theory to get started.}
\end{itemize}
\end{frame}
