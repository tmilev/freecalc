\begin{frame}
\frametitle{Factoring polynomials in practice}
\begin{itemize}
\item In theory every polynomial can be factored.
\[a_0 x^{n}+a_1x^{n-1}+\dots + a_n=a_0 (x-x_1)\dots (x-x_n)
\]
\item Theory guarantees numerical approximations for roots $x_1,\dots, x_n$. 
\item Can we find algebraic formulas for $x_1, \dots, x_n$?
\item No, if using finitely many operations $+,-,*,/,\sqrt[n]{~}$. 
\item \footnotesize{ Yes, with extra operations. Difficult: google Galois Theory to get started.}
\end{itemize}
\end{frame}
\begin{frame}
\frametitle{What does factorization mean?}
\begin{itemize}
\item Depending on context, ``to factor a polynomial'' means one of three things:
\begin{itemize}
\item  Factor the polynomial over the rational numbers. Use integers/quotients, but no $\sqrt[n]{~}$).
\item Factor the polynomial over the real numbers. Use radicals and/or numerical approximations, no use of $i=\sqrt{-1}$.
\item Fully factor the polynomial using complex numbers.
\end{itemize}
\end{itemize}
\begin{tabular}{|c|c|}\hline
These poly's are equal&Type of factorization\\\hline
$x^4+1$& factored over rationals\\\hline
$(x^2-\sqrt{2}x+1)(x^2+\sqrt{2}x+1)$& factored over the reals\\\hline
$\begin{array}{r}
\left(x-\left(\frac{\sqrt{2}}{2}+\frac{\sqrt{2}}{2}i\right)\right)
\left(x-\left(-\frac{\sqrt{2}}{2}+\frac{\sqrt{2}}{2}i\right)\right)\\
\left(x-\left(\frac{\sqrt{2}}{2}-\frac{\sqrt{2}}{2}i\right)\right)
\left(x-\left(-\frac{\sqrt{2}}{2}-\frac{\sqrt{2}}{2}i\right)\right)
\end{array}$& full complex factorization\\\hline
\end{tabular}


\end{frame}