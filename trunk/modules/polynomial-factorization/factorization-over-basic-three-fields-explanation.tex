\begin{frame}
\frametitle{What does factorization mean?}
\begin{itemize}
\item Based on context, ``to factor a polynomial'' means one of:
\begin{itemize}
\item<2-> Factor the polynomial \alertNoH{3}{over the rational numbers}. Use integers/quotients, but no $\sqrt[n]{~}$).
\item<4-> Factor the polynomial \alertNoH{5}{over the real numbers}. Use radicals and/or numerical approximations, \alertNoH{5}{no use of $i=\sqrt{-1}$}.
\item<6-> Fully factor the polynomial \alertNoH{7}{using complex numbers}.
\end{itemize}
\end{itemize}
\begin{tabular}{|c|c|}\hline
These poly's are equal&Type of factorization\\\hline
 $\alertNoH{3}{x^4+1}$&\uncover<2->{ \alertNoH{3}{factored over rationals}}\\\hline
$\alertNoH{5}{(x^2-\sqrt{2}x+1)(x^2+\sqrt{2}x+1)}$&\uncover<4->{ \alertNoH{5}{factored over the reals}}\\\hline
$\begin{array}{r}
\alertNoH{7}{ \left(x-\left(\frac{\sqrt{2}}{2}+\frac{\sqrt{2}}{2}i\right)\right)
\left(x-\left(-\frac{\sqrt{2}}{2}+\frac{\sqrt{2}}{2}i\right)\right) }\\
\alertNoH{7}{ \left(x-\left(\frac{\sqrt{2}}{2}-\frac{\sqrt{2}}{2}i\right)\right)
\left(x-\left(-\frac{\sqrt{2}}{2}-\frac{\sqrt{2}}{2}i\right)\right)}
\end{array}$&\uncover<6->{ \alertNoH{7}{ full complex factorization}} \\\hline
\end{tabular}


\end{frame}