\begin{frame}
\begin{theorem}[The Fundamental Theorem of Algebra]
Every polynomial can be factored into product of linear terms
\[
p(x)=\alertNoH{4}{a_0} x^{\alertNoH{3}{n}}+a_1x^{n-1}+\dots + a_n=\alertNoH{2}{\alertNoH{4}{a_0} (x-\alertNoH{5}{x_1})\dots (x-\alertNoH{5}{x_n})},
\] 
where $x_1,\dots, x_n$ are the (not necessarily different) roots of $p(x)$. 
\end{theorem}
\begin{itemize}
\only<1-7>{
\item<2-> Every pol. of \alertNoH{3}{deg. $n$} can be factored as \alertNoH{2}{product of $n$ linear factors}.
}
\item<5-> \alertNoH{7,8}{ \alertNoH{5}{$x_1,\dots, x_n$ may be complex} numbers. Reminder: complex numbers are of the form $p+qi$, where $i^2=-1$ and $\sqrt{-1}=i$.}
\only<1-7>{
\item<6-7> While we can find $x_1, \dots x_n$ with arbitrary precision, \alertNoH{6}{there may not be exist a formula involving radicals for computing each $x_1,\dots, x_n$}.
}
\end{itemize}
\uncover<9->{
\begin{Corollary}
Every real polynomial can be factored into a product of real linear terms and real quadratic terms with no real roots, i.e., factors of form 
\begin{itemize}
\item $(x-r)$, where $r$ is real and
\item $ax^2+bx+c$ with $b^2-4ac<0$ where $a,b,c$ are real.
\end{itemize}
\end{Corollary}
}

\vskip 10cm
\end{frame}