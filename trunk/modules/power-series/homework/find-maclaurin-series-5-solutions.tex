\solution{\ref{problemMaclaurin(1/(1-x)^2)}
\[
\begin{array}{rcll|l}
\displaystyle \frac{1}{1-x}&=&\displaystyle \frac{\diff }{\diff x}\left(1+x+x^2+x^3+\dots  \right)&&
\begin{array}{l}
\text{geometric series,}\\
\text{converges if and only if}\\
|x|<1
\end{array}
\\
\displaystyle \frac{\diff }{\diff x}\left(\frac{1}{1-x} \right)&=& \displaystyle \frac{\diff }{\diff x}\left(1+x+x^2+x^3+\dots  \right)&&\text{apply }\frac{\diff }{\diff x}\\
\displaystyle  -\frac{(1-x)'}{(1-x)^2}= \frac{1}{(1-x)^2}&=& \displaystyle 1+2x+3x^2+\dots \\
\displaystyle \frac{1}{(1-x)^2} &=&\displaystyle \sum\limits_{n=0}^\infty (n+1)x^n&& \text{rewrite in } \sum 
\text{ notation.}
\end{array}
\]
The radius of convergence of the geometric series is $1$. Differentiating does not change the radius of convergence. We have that the radius of convergence of $1+x+x^2+\dots$ is $1$ and therefore we have that $\frac{1}{ (1-x)^2}= \sum \limits_{ n=0}^ \infty (n+1)x^n$ converges for $|x|<1$ and the radius of convergence is $R=1$.

The problem does not ask us to determine the interval of convergence, however let us do it for exercise. The endpoints of the interval of convergence are $-1$ and $1$. The series is divergent for both of them: indeed at $x=-1$ the series becomes $\sum\limits_{n=0}^\infty (-1)^n(n+1)x^n $ and at $x=1$ the series becomes $\sum\limits_{n=0}^\infty (n+1)x^n $. Both of these series are divergent as their terms do not tend to zero as $n$ tends to infinity. Thus the interval of convergence is $(-1,1)$.

\refBad{\ref{problemMaclaurin 1/(1-x)^k}}{}{We generalize this problem in Problem \ref{problemMaclaurin 1/(1-x)^k}.}

}

\solution{\ref{problemMaclaurinSeriesln(1-x)}
\[
\begin{array}{rcll|l}
\displaystyle \frac{\diff }{\diff x} \left( \ln(1 -x) \right) &=&\displaystyle  \frac{-1}{1-x}&&\begin{array}{@{}l} \text{expand as geometric series}\\\text{for }|x|< 1\end{array}\\
&=&\displaystyle  -\left(1+x+x^2+x^3+\dots\right) &&\text{Integrate indefinitely, } |x|< 1 \\
\displaystyle 
\int 
\frac{\diff }{\diff x}(\ln(1-x))\diff x  &=& \displaystyle -\int\left(1+x+x^2+x^3+\dots \right)\diff x  && \begin{array}{@{}l} \text{For power series, }\\
\text{integral of infinite}\\
\text{sum equals}\\
\text{infinite sum of integrals} \\
\text{inside the convergence radius}
\end{array}
\\
\ln(1-x)&=&\displaystyle -\left(x+\frac{x^2}{2}+\frac{x^3}{3} +\dots \right)+C&&\text{To find }C \text{ set }x=0\\~\\
0=\ln 1&=&-0+C=C\\~\\
\ln (1-x)&=&-\left(x+\frac{x^2}{2}+\frac{x^3}{3}+\dots \right)\\
&=&\displaystyle - \sum_{n=1}^{\infty}\frac{x^n}{n} \quad .
\end{array}
\]
The radius of convergence of the geometric series $1+x+x^2+\dots$ is $1$. Since the series for $\ln (1-x)$ is obtained from the geometric series via integration, its radius of convergence is again $1$. 

We note that the interval of convergence for the series  $-\sum\limits_{n=1}^\infty \frac{x^n}{n}$ is $\left[-1, 1 \right)$ - the series is convergent at $x=-1$ by the alternating series test and divergent at $x=1$ (at $x=1$ the series is minus the harmonic series). This shows that integration of power series can change convergence at the endpoints of the interval of convergence.  
}

\solution{\ref{problemMaclaurin(ln(3-2x^2))}. 
We solve this problem by reducing it to Problem \ref{problemMaclaurinSeriesln(1-x)}, which asserts the power series expansion $\displaystyle \ln (1-y) = -\sum\limits_{n=1}^\infty \frac{y^n}{n}$ for $|y|<1$.
\[
\begin{array}{rcll|l}
\ln \left(3-2x^2\right)&=& \displaystyle \ln \left(3 \left(1 -\frac{2}{3} x^2\right)\right)\\
&=&\displaystyle \ln 3+ \ln  \left(1 -\frac{2}{3} x^2\right)&&\text{Set } y=\frac{2}{3}x^2\\
&=&\displaystyle \ln 3+\ln (1-y)\\
&=&\displaystyle \ln 3-\sum\limits_{n=1}^\infty \frac{y^n}{n}&& \begin{array}{l}
\ln (1-y)= -\sum\limits_{n=1}^\infty \frac{y^n}{n} \text{ for } |y|<1\\
\text{above does not hold for } |y|>1\\
\text{above may (not) hold for } y=\pm 1
\end{array}
\\
&=&\displaystyle \ln 3-\sum\limits_{n=1}^\infty \left(\frac{2}{3}\right)^n\frac{x^{2n}}{n}\quad . &&\text{Substituted back } y=\frac{2}{3}x^2\quad .
\end{array}
\]
As indicated above, the equality $\ln (1-y)= -\sum\limits_{n=1}^\infty \frac{y^n}{n}$ holds for $|y|<1$ and fails for $|y|>1$ (for $|y|>1$ the series $\sum\limits_{ n=1}^\infty \frac{y^n}{n}$ diverges). Therefore interval of convergence is given by 

\[\begin{array}{rcll|l}
|y|&<&1 &&\text{use }y=\frac{2}{3}x^2\\
\left|\frac{2}{3 }x^2\right|&<&1\\
|x^2|&<&\frac{3}{2}\\
|x|&<&\sqrt{\frac{3}{2}},
\end{array}
\]
i.e., the radius of convergence is $R=\sqrt{\frac{3}{2}}$.
}