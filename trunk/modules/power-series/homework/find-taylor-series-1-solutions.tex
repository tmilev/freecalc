\solution{\ref{problemTaylorSeries a=1 ln(sqrt(x^2-2x+2))}
\[
\begin{array}{rcll|l}
\displaystyle \ln \left( \sqrt{x^2-2x+2} \right) &=&\displaystyle  \frac{ 1}{2} \ln \left( (x-1)^2+1\right)&&\text{use } \ln(1+y) =  \sum \limits_{n=1 }^ \infty (-1)^{n+1} \frac{y^n}{n}, |y|< 1 \\
&=&\displaystyle \frac{1}{2}\sum\limits_{n=1}^{\infty} (-1)^{n+1}\frac{\left((x-1)^2\right)^{n}}{n}\\
&=&\displaystyle \sum \limits_{n=1}^{\infty} (-1)^{n+1} \frac{( x-1)^{2n}}{2n}\quad .
\end{array}
\]

Although the problem does not ask us to do this, we will determine the interval of convergence of the series for exercise. If we use the fact that $\ln(1+y) =  \sum \limits_{n=1 }^ \infty (-1)^{n+1} \frac{y^n}{n}$ holds for $ -1<y\leq 1$, it follows immediately that the above equality holds for $ 0<(x-1)^2\leq 1$, which holds for $x\in[0,2]$. Let us however compute the interval of convergence without using the aforementioned fact.

Let $a_n$ be the $n^{th}$ term of our series, i.e., let 
\[
a_n= (-1)^{n+1} \frac{( x-1)^{2n}}{2n}\quad .
\]
We use the ratio test:
\[
\begin{array}{rcl}
\displaystyle \lim \limits_{ n\to \infty}\left| \frac{a_{n + 1} }{a_n }\right|&=&\displaystyle \lim \limits_{ n\to \infty }\left| \frac{(-1)^{n+2}(x-1)^{2n+2} }{ (2n+2)} \frac{ 2 n}{ (-1)^{n+1} ( x-1)^{2n} } \right|\\
&=&\displaystyle \lim_{n\to \infty} (x-1)^2 \frac{n}{n+1}\\
&=&\displaystyle (x-1)^2\quad .
\end{array}
\] 
By the ratio test, the series is divergent for $ (x-1)^2>1$, i.e., for $|x-1|>1$, and convergent for $(x-1)^2<1$, i.e., for $|x-1|<1$. The ratio test is inconclusive at only two points: $x-1=1$, i.e., $x=2$ and $x-1=-1$, i.e., $x=0$. At both points the series becomes $\displaystyle \sum\limits_{n=1}^{\infty} (-1)^{n+1} \frac{ 2^{ 2n}}{2n}$ and the series is convergent at both points by the alternating series test.
}

\solution{\ref{problemTaylorlnxarounda=2}
This solution is similar to the solution of \ref{problemTaylorSeries a=1 ln(sqrt(x^2-2x+2))}, but we have written it in a concise fashion suitable for test taking. 

Denote Taylor series at $a$ by $\taylor_a$ and recall that the Maclaurin series of are just $\taylor_0$, the Taylor series at $0$.
\[
\begin{array}{rcll|l}
\displaystyle \taylor_2(\ln x)&=& \displaystyle \taylor_2 (\ln \left((x-2)+2\right))\\
&=&\displaystyle \taylor_2\left(\ln \left(2\left(\frac{x-2}{2}+1\right) \right)\right)\\
&=&\displaystyle \taylor_2\left(\ln 2+ \ln \left(1+\frac{x-2}{2} \right) \right)&& T_0(\ln (1+y))=\sum\limits_{n=1}^\infty \frac{(-1)^{n+1}y^n}{n}\\
&=&\displaystyle \ln 2 + \sum_{n=1}^{\infty }\frac{(-1)^{n+1} \left(\frac{x-2}{2}\right)}{n}\\
&=&\displaystyle \ln 2 +\sum_{n=1}^{\infty}\frac{(-1)^{n+1}}{2^n} (x-2)^n\quad .
\end{array}
\]
}