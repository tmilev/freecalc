% begin module power-series-as-function-ex2
\begin{frame}
\begin{example}
Find a power series representation for $\frac{1}{x+2}$.
\abovedisplayskip=0pt
\belowdisplayskip=0pt
\[
\begin{array}{rcll|l}
\displaystyle \uncover<2->{\frac{1}{2+x}}
& \uncover<2->{ = } &\displaystyle %
\uncover<2->{\frac{1}{2\left( 1 \alertNoH{3}{+} \frac{x}{2}\right)}}\\
&\uncover<3->{ = }&\displaystyle \uncover<3->{\frac{1 }{2} \cdot \alertNoH{13}{ \frac{1}{\left( 1 \alertNoH{3}{-} \alertNoH{4 }{ \left( \alertNoH{3}{-} \frac{x}{2}\right)} \right)}}%
}%
\uncover<4->{ = }% 
\uncover<4->{ \frac{1}{\alertNoH{9-10}{2}} \alertNoH{13} {\sum_{n=0}^\infty \left( \alertNoH{4 }{ \alertNoH{5-6}{- }\frac{\alertNoH{7-8}{x}}{ \alertNoH{9- 10}{2 }}} \right)^{ \alertNoH{5-10}{n} } }} \uncover<4->{&& \alertNoH{13}{ \begin{array}{l} \text{if \& only if} \\ \left|-\frac{x}{2}\right|<1 \end{array} } }\\%
&\uncover<5->{ = }&\displaystyle %
\uncover<5->{%
\sum_{n=0}^\infty \frac{ \fcAnswer{6}{(-1)^n}}{ \fcAnswerUncover{5}{10}{2^{n+1}}} \fcAnswerUncover{5}{8}{x^n}%
}\\%
 & \uncover<11->{ = } &\displaystyle%
\uncover<11->{%
\frac{1}{2} - \frac{x}{4} + \frac{x^2}{8} - \frac{x^3}{16} + \dots%
}%
\end{array}
\]
\uncover<12->{To find interval of convergence:}%
\[
\begin{array}{rcl}
\displaystyle \fcAnswer{13}{\left| - \frac{x}{2}\right|}%
 & \uncover<12->{\alertNoH{12,13}{<}} &\displaystyle %
\uncover<13->{%
\fcAnswer{13}{1}%
}\\%
\displaystyle \uncover<14->{\alertNoH{15}{|x|}}%
 & \uncover<14->{ \alertNoH{12-13,15}{<} } &\displaystyle %
\uncover<14->{%
\alertNoH{12-13,15}{2}%
}%
\end{array}
\]
\uncover<15->{Therefore the interval of convergence is $\alertNoH{15}{ x\in (-2, 2)}$.}%
\end{example}
\end{frame}
% end module power-series-as-function-ex2
