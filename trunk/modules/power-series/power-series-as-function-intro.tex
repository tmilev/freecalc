% begin module power-series-as-function-intro
\begin{frame}
\frametitle{Representations of Functions as Power Series}
\[
\sum_{n=0}^\infty x^n = 1 + x + x^2 + x^3 + \cdots 
\]
\begin{itemize}
\item<2->  This is a geometric series with \alert<handout:0| 2-3,9>{$a = $ \uncover<3->{$1$}} and \alert<handout:0| 4-5,10>{$r = $ \uncover<5->{$x$.}}
\item<6->  It is convergent if \uncover<7->{\alert<handout:0| 7,12>{$|x| < 1$}} and divergent otherwise.
\item<8->  If it converges, the sum is $\frac{\uncover<9->{\alert<handout:0| 9>{1}}}{1 - \uncover<10->{\alert<handout:0| 10>{x}}}$.
\item<11->  The thing that is new in this section is the we now regard the series $\sum_{n=0}^\infty x^n$ as expressing the function $f(x) = \frac{1}{1-x}$.
\item<12->  This only works if $-1 < x < 1$.
\end{itemize}
\end{frame}
% end module power-series-as-function-intro
