% begin module power-series-calculus-ex6
\begin{frame}
\vskip -0.1cm
\begin{example} 
Find a power series for $\ln (1-x)$ and state its radius of convergence.
$
\begin{array}{rcll|l}
\displaystyle \ln (1-\alertNoH{17,18}{x})%
& \uncover<2->{ = } &\displaystyle \uncover<2->{ \alertNoH{2}{ \int \alertNoH{3}{\diff}} \alertNoH{3} {( \ln (1-x))}} \uncover<3->{ =\int \alertNoH{3}{ \alertNoH{4,5} {(\ln (1-x))'} \diff x }} \uncover<2->{&&\alertNoH{2,13}{ \text{up to const.}}}\\  
&\uncover<4->{ = }&\displaystyle \uncover<4->{ \int \fcAnswer{5}{ \left(\alertNoH{6}{-}\alertNoH{7,8}{\frac{1}{1-x}}\right)} \diff x} \\%
&\uncover<6->{=}&\displaystyle \uncover<6->{ \alertNoH{6}{-} \alertNoH{9,10,11,12}{\int} \left(\fcAnswerUncover{6}{ 8}{\alertNoH{9}{1} +\alertNoH{10}{x}+\alertNoH{11}{x^2}+\alertNoH{12}{x^3}+\cdots }\right) \alertNoH{9,10,11,12}{\diff x} } \uncover<7->{&& \alertNoH{7,8,19}{\text{for } |x|<1}} \\%
 & \uncover<9->{ = } &\displaystyle %
\uncover<9->{\alertNoH{14}{-}\left(\alertNoH{15,16}{ \alertNoH{9}{ x} + \alertNoH{10}{\frac{x^2}{2}} + \alertNoH{11}{\frac{x^3}{3}} + \alertNoH{12}{\frac{x^4}{4}} + \cdots} \right) \uncover<13->{\alertNoH{ 13,14}{+C} } } \\
\uncover<14->{&=&\displaystyle  \alertNoH{ 14}{C -} \alertNoH{15,16}{\sum_{n=1}^\infty \fcAnswerUncover{14}{16}{ \frac{ {\alertNoH{17, 18}{x}}^n}{n}}} 
}
\end{array}
$
\vskip -0.3cm
\begin{itemize}
\item<17-> To find $C$, plug in $\alertNoH{17,18}{x = 0} $: $\alertNoH{17, 18}{C=}\fcAnswer{18}{0.} $
\item<19-> Therefore the theorem on integrating power series implies that $\displaystyle  \ln (1-x) = - \sum_{n=1}^\infty \frac{x^n}{n}$, for $|x|<1$.
\item<20-> By the same theorem, the radius of convergence remains $R = 1$.
\end{itemize}
\end{example}
\end{frame}
% end module power-series-calculus-ex6
