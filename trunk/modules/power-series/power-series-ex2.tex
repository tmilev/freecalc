% begin module power-series-ex2
\begin{frame}
\begin{example}[Example 2, p. 759]
For what values of $x$ is the series $\sum_{n=1}^\infty \frac{(x-3)^n}{n}$ convergent?
\begin{itemize}
\item<2->  Use the Ratio Test.
\item<3->  The $n$th term is $a_n = \frac{(x-3)^n}{n}$.
\end{itemize}
\abovedisplayskip=0pt
\belowdisplayskip=0pt
\begin{eqnarray*}
\uncover<4->{%
\lim_{n\to\infty} \left| \frac{a_{n+1}}{a_n}\right|%
}%
& \uncover<4->{ = } &%
\uncover<4->{%
\lim_{n\to\infty} \left| \frac{\alert<handout:0| 5-6>{(x-3)^{n+1}}}{\alert<handout:0| 7-8>{n+1}} \cdot \frac{\alert<handout:0| 7-8>{n}}{\alert<handout:0| 5-6>{(x-3)^n}}\right|%
}\\%
& \uncover<5->{ = } &%
\uncover<5->{%
\lim_{n\to\infty} \alert<handout:0| 6>{\uncover<6->{|x-3|}}\alert<handout:0| 8>{\uncover<8->{\frac{n}{\alert<handout:0| 9>{n}+1}}}\uncover<9->{\alert<handout:0| 9>{\cdot \frac{\frac{1}{n}}{\frac{1}{n}}}}%
}%
 \uncover<10->{ = } %
\uncover<10->{%
\lim_{n\to\infty} |x-3| \frac{1}{1 + \frac{1}{n}}%
}%
 \uncover<11->{ = } %
\uncover<11->{%
|x-3|%
}%
\end{eqnarray*}
\begin{itemize}
\item<12->  Therefore by the Ratio Test the series \alert<handout:0| 12-13>{converges absolutely if \uncover<13->{$|x-3| < 1$}} and \alert<handout:0| 14-15>{diverges if \uncover<15->{$|x-3|>1$.}}
\end{itemize}
\abovedisplayskip=0pt
\belowdisplayskip=0pt
\[
\uncover<16->{%
|x-3| < 1 \quad \Leftrightarrow \quad -1 < x -3 < 1 \quad %
}%
\uncover<17->{%
\Leftrightarrow \quad 2 < x < 4%
}%
\]
\vspace{-.3in}
\begin{itemize}
\item<18-| alert@18-19>  If we put $x = 4$ in the series, we get $\sum \frac{1}{n}$, which is \uncover<19->{divergent.}
\item<18-| alert@20-21>  If we put $x = 2$ in the series, we get $\sum \frac{(-1)^n}{n}$, which is \uncover<21->{convergent.}
\item<22->  The series converges if $2 \leq x < 4$ and diverges otherwise.
\end{itemize}
\end{example}
\end{frame}
% end module power-series-ex2
