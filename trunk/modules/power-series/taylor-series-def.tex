% begin module taylor-series-def
\begin{frame}
\begin{theorem}[Coefficients of a Power Series]
If $f$ has a power series representation at $a$, that is, if
\abovedisplayskip=0pt
\belowdisplayskip=0pt
\[
f(x) = \sum_{n=0}^\infty c_n (x-a)^n, \qquad |x-a| < R,
\]
then its coefficients are given by the formula
\abovedisplayskip=0pt
\belowdisplayskip=0pt
\[
\alert<handout:0| 2>{c_n = \frac{f^{(n)}(a)}{n!}}.
\]
\end{theorem}
\uncover<2->{%
Here is what we get if we plug these coefficients into the power series:
\abovedisplayskip=0pt
\belowdisplayskip=0pt
\begin{eqnarray*}
f(x) & = & \sum_{n=0}^\infty \alert<handout:0| 2>{\frac{f^{(n)}(a)}{n!}} (x-a)^n\\
& = & f(a) + \frac{f'(a)}{1!}(x-a) + \frac{f''(a)}{2!}(x-a)^2 + \frac{f'''(a)}{3!}(x-a)^3 + \cdots 
\end{eqnarray*}
}%
\uncover<3->{%
\begin{definition}[Taylor Series]
This series is called the Taylor series of $f$.
\end{definition}
}%
\end{frame}
% end module taylor-series-def
