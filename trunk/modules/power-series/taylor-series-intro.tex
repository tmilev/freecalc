% begin module taylor-series-intro
\begin{frame}
\frametitle{(12.10)  Taylor and Maclaurin Series}
\begin{itemize}
\item  Let $f$ be a function that can be represented by a power series: 
\item  $f(x) = \alert<handout:0| 3>{c_0} + c_1 (x-a) + c_2 (x-a)^2 + c_3(x-a)^3 + \cdots$
\item<2-| alert@2-3>  $f(a) = $ \uncover<3->{$c_0$.}
\item<4->  $f'(x) = \alert<handout:0| 6>{c_1}  + 2c_2 (x-a) + 3c_3(x-a)^2 + 4c_4 (x-a)^3 + \cdots$
\item<5-| alert@5-6>  $f'(a) = $ \uncover<6->{$c_1$.}
\item<7->  $f''(x) =  \alert<handout:0| 9>{2c_2}  + 2\cdot 3c_3(x-a) + 3\cdot 4c_4 (x-a)^2 + 4\cdot 5 c_5 (x-a)^3 + \cdots$
\item<8-| alert@8-9>  $f''(a) = $ \uncover<9->{$2 c_2$.}
\item<10->  $f'''(x) =   \alert<handout:0| 12>{2\cdot 3c_3} + 2\cdot 3\cdot 4c_4 (x-a) + 3\cdot 4\cdot 5  c_5(x-a)^2 + \cdots$
\item<11-| alert@11-12>  $f'''(a) = $ \uncover<12->{$2\cdot 3 c_3 = 3! c_3$.}
\item<13-| alert@13-14>  $f^{(n)}(a) = $ \uncover<14->{$n! c_n$.}
\item<15->  Therefore $c_n = \frac{f^{(n)}(a)}{n!}$.
\end{itemize}
\end{frame}
% end module taylor-series-intro
