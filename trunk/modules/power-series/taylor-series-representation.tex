% begin module taylor-series-representation
\begin{frame}
\begin{itemize}
\item  We have found the Maclaurin series for $e^x$:
\abovedisplayskip=0pt
\belowdisplayskip=0pt
\[
\alert<handout:0| 1-2>{\sum_{n=0}^\infty \uncover<2->{\frac{1}{n!}}x^n}
\]
\item<3->  If $e^x$ has a power series representation about $0$, then it is equal to the Maclaurin series.
\item<3->  Does $e^x$ have a power series representation about $0$?
\item<4->  Yes.  A proof of this fact can be found on p. 774.  It uses Taylor's Inequality, from p. 773.
\item<5->  Therefore
\uncover<5->{%
\abovedisplayskip=0pt
\belowdisplayskip=0pt
\[
e^x = \sum_{n=0}^\infty \frac{1}{n!} x^n
\]
}%
\item<6->  This equality holds for all $x$, because the series converges for all $x$.
\item<7->  Taylor's Inequality is used to show that many different functions are equal to their Maclaurin series.
\end{itemize}
\end{frame}
% end module taylor-series-representation
