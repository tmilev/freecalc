% begin module domains
\begin{frame}
\frametitle{A Note on Domains of Functions}
If the domain of a function isn't specified, it is implied to be all numbers $x$ for which the formula $f(x)$ is defined. There are some restrictions to consider:
\begin{itemize}
\item<2->  Can't divide by $0$.
\item<3->  Even roots of a negative number are not defined in this course ($\sqrt{-1} , \sqrt[4]{-2053}, \sqrt[6]{-15} \ldots$ not allowed).
\item<4->  Taking $\log x$ if $x \leq 0$ is not allowed in this course; taking  $\log 0$ is not allowed in any course.
\end{itemize}
\end{frame}

\begin{frame}
\begin{example}[Two Functions and Their Domains]
Find the implied domains of the following two functions:
\begin{columns}
\column{.5\textwidth}
\[
f(x) = \alert<handout:0| 5>{\sqrt[4]{x-2}} + \sqrt[3]{6-x}
\]
\begin{itemize}
\item<2->  Any risk of dividing by $0$?  \uncover<3->{No.}
\item<4->  Any risk of taking the even root of a negative number? \uncover<5->{\alert<handout:0| 5>{Yes.}}
\item<6->  $x - 2$  must not be negative.
\end{itemize}
\begin{eqnarray*}
\uncover<7->{x - 2} & \uncover<7->{\geq} & \uncover<7->{0}\\
\uncover<8->{x} & \uncover<8->{\geq} & \uncover<8->{2}
\end{eqnarray*}
\uncover<9->{Domain is all real numbers greater than or equal to $2$; that is, $[2,\infty )$.
}%
\column{.5\textwidth}
\[
g(x) = \frac{x^2 - 9}{\alert<handout:0| 11>{x^2 - x - 6}}
\]
\begin{itemize}
\item<10->  Any risk of dividing by $0$?  \uncover<11->{\alert<handout:0| 11>{Yes.}}
\item<12->  Any risk of taking the even root of a negative number? \uncover<13->{No.}
\item<14->  $x^2 - x - 6$ must not equal $0$.
\end{itemize}
\abovedisplayskip=0pt
\belowdisplayskip=0pt
\abovedisplayshortskip=0pt
\belowdisplayshortskip=0pt
\begin{eqnarray*}
\uncover<15->{x^2 - x - 6} & \uncover<15->{\neq} & \uncover<15->{0}\\
\uncover<16->{(x - 3)(x + 2)} & \uncover<16->{\neq} & \uncover<16->{0}\\
\uncover<17->{x} & \uncover<17->{\neq} & \uncover<17->{3 \textrm{ or } -2}\\
\end{eqnarray*}
\uncover<18->{Domain is all real numbers except $3$ and $-2$; that is, \\ $(-\infty , -2)$ $\cup (-2,3)$ $\cup (3,\infty )$.
}%
\end{columns}
\end{example}
\end{frame}
% end module domains
