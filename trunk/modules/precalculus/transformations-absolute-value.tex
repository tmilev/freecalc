% begin module transformations-absolute-value
\begin{frame}
What happens when we take the absolute value of a function?
\uncover<2->{
\[
|f(x)| = \left\{ \begin{array}{rcc}
f(x) & \textrm{if} & f(x) \geq 0\\
-f(x) & \textrm{if} & f(x) < 0
\end{array}\right.
\]
}
\uncover<3->{%
This tells us how to draw the graph of $y = |f(x)|$: the part of the graph above the $x$-axis remains the same; the part below the $x$-axis is reflected about the $x$-axis.
}
\uncover<4->{
\begin{example}[Example 5, p. 41]
Draw the graph of the function $f(x) = |x^2 - 1|$.
\begin{columns}[c]
\column{.4\textwidth}
\ \only<handout:0| -4>{%
\includegraphics[height=4cm]{precalculus/pictures/01-03-ex5z.pdf}%
}%
\only<handout:0| 5>{%
\includegraphics[height=4cm]{precalculus/pictures/01-03-ex5a.pdf}%
}%
\only<handout:0| 6>{%
\includegraphics[height=4cm]{precalculus/pictures/01-03-ex5b.pdf}%
}%
\only<7->{%
\includegraphics[height=4cm]{precalculus/pictures/01-03-ex5c.pdf}%
}%
\column{.6\textwidth}
\begin{itemize}
\item<5->  Draw the graph of $f(x) = x^2 - 1$.
\item<6->  Identify the part(s) below the $x$-axis.
\item<7->  Flip those parts over the $x$-axis.
\end{itemize}
\end{columns}
\end{example}
}
\end{frame}
% end module transformations-absolute-value
