\begin{frame}
To plot a parabola by hand roughly, we need to do the following.
\begin{columns}
\column{0.3\textwidth}
\begin{pspicture}(-2.2,-2.2)(2.2,2.2)%
\tiny%
\fcBoundingBox{-2.2}{-2.2}{2.2}{2.2}
\uncover<handout:0|1-4>{\fcAxesStandard{-0.9 }{-0.9}{1}{1}}%
\uncover<handout:1|5->{\fcAxesStandard{-2 }{-2}{2}{2}}%
\uncover<6->{
\psplot[linecolor=\fcColorGraph]{-1.05}{2}{x x mul x sub 1 sub}
}
\uncover<4->{
\fcFullDot{1 5 sqrt add 2 div}{0}
\fcFullDot{1 5 sqrt sub 2 div}{0}
}
\uncover<2->{\fcFullDot{0.5}{-5 4 div}}
\uncover<3->{\fcFullDot{0}{-1}}
\end{pspicture}
\column{0.7\textwidth}
\begin{itemize}
\item<2-> Find the vertex of the parabola.
\item<3-> Find the $y$ intercept.
\item<4-> Find the $x$ intercept(s) if any.
\item<5-> Select (or re-select) axes scale so all important points found in the preceding items fit in the plot.
\item<6-> Plot the parabola freehand, making sure that the parabola passes through all special points you found in the preceding items. 
\item<7-> If $a>0$ your parabola should open upwards, if $a<0$ your parabola should open downwards.
\item<8-> For $|a|>1$ we should aim to draw the graph steeper than $a=x^2$, for $|a|<1$ we should aim to draw the graph flatter than $a=x^2$.
\end{itemize}
\end{columns}
\end{frame}