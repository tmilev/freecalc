\begin{frame}
\begin{problem}[Quadratic equation formula]
Solve $ax^2+bx+c=0$. Recall that $D=b^2-4ac$.
$
\begin{array}{rcll|l}
\displaystyle ax^2+bx+c&=&0 \\
\displaystyle a\left(x +\frac{b }{2a}  \right)^2- \frac{D}{4a}&=& 0&&\text{comlete the square}\\
\displaystyle a\left(\left(x +\frac{b }{2a} \right)^2- \frac{D}{4a^2}\right)&=& 0\\
\displaystyle a\left(\left(x +\frac{b }{2a} \right)^2- \left( \frac{ \sqrt{D }}{2 a}\right)^2\right)&=& 0\\
\displaystyle a\left(x+\frac{b }{2a}-\frac{\sqrt{D}}{2a} \right)\left(x+\frac{b }{2a}+\frac{\sqrt{D}}{2a} \right)&=&0 &&\begin{array}{l} \text{use } A^2-B^2\\ =(A-B)(A+B)\end{array} \\
\end{array}
$
$
\begin{array}{rclcrcl}
\displaystyle x+\frac{b}{2a}-\frac{\sqrt{D}}{2a}&=&0 &\text{ or } & \displaystyle x+\frac{b}{2a}+\frac{\sqrt{D}}{2a}&=&0\\
x&=&\displaystyle \frac{-b+\sqrt{D}}{2a} & \text{ or } & x&=&\displaystyle \frac{-b-\sqrt{D}}{2a}.
\end{array}
$
\end{problem}
\end{frame}
\begin{frame}
\begin{theorem}
The solutions to the quadratic equation 
\[
ax^2+bx+c=0
\]
are the numbers 
\[
x_1=\frac{-b+ \sqrt{b^2-4ac}}{2a} \qquad x_2=\frac{-b- \sqrt{b^2-4ac}}{2a}.
\]
\end{theorem}
\begin{itemize}
\item Sometimes this formula is abbreviated as 
\[
x=\frac{-b\pm \sqrt{D}}{2a}=\frac{-b\pm \sqrt{b^2-4ac}}{2a}.
\]
\item If $D<0$ then $\sqrt{D}$ is not a real number and the quadratic equation has no real solutions.
\item If $D=0$ then $x_1=x_2$, the equation has only one zero (with multiplicity two). The zero is located at the vertex of the parabola.
\end{itemize}
\end{frame}