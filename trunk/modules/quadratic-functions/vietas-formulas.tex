\begin{frame}
\begin{proposition}[Vieta's formulas]
Let $ax^2+bx+c$ be a quadratic functions with zeros $x_1$ and $x_2$. Then:
\[
\begin{array}{rcl}
\displaystyle \alertNoH{2,3,6,7}{a}( \alertNoH{7}{x} \alertNoH{2,3,6}{- x_1})(\alertNoH{6}{ x} \alertNoH{2,7}{ -x_2})& \alertNoH{0}{=} &\displaystyle ax^2+ \alertNoH{8}{ b x} +\alertNoH{2}{c}\\
\uncover<handout:0|2-11>{\alertNoH{2}{\alertNoH{4}{a} \left(\alertNoH{3}{-} x_1 \right)\left( \alertNoH{3}{-} x_2\right)} &\alertNoH{2}{=}& \alertNoH{2}{c} }\\
\uncover<3->{\displaystyle \alertNoH{5,12}{ x_1x_2}& \alertNoH{5,12}{=}& \displaystyle \alertNoH{5,12}{\frac{c}{\alertNoH{4}{ a}}}} \\
\uncover<handout:0|6-11>{ \alertNoH{8}{\alertNoH{6,7,10}{a} \left( \alertNoH{6}{ \alertNoH{11}{ -} x_1} \alertNoH{7}{\alertNoH{11}{ -} x_2}\right) \alertNoH{6,7,9}{x}}  &\alertNoH{0}{=}& \alertNoH{8}{b\alertNoH{9}{ x}} }\\
\uncover<9->{\displaystyle \alertNoH{12}{ x_1+x_2} &\alertNoH{12}{=} &\displaystyle \alertNoH{12}{ \alertNoH{11}{-}\frac{b}{\alertNoH{10}{a}}}.}\\
\end{array}
\]
\end{proposition}
The two formulas are called Vieta's formulas (after Fran\c{c}ois Vi\`ete (1540-1603), Latinized name: Franciscus Vieta).

\end{frame}