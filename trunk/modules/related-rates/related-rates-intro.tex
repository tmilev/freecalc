% begin module related-rates-intro
\begin{frame}
\frametitle{(3.8)  Related Rates}
\begin{itemize}
\item  Suppose we are pumping a balloon with air.
\item  The balloon's volume is increasing.
\item  The balloon's radius is increasing.
\item  The rates of increase of these quantities are related to one another.
\item<2->  It is easier to measure the rate of increase of volume.
\item<3->  In a related rates problem, the idea is to compute the rate of change of one quantity in terms of the rate of change of another (which may be more easily measured).
\item<4->  Procedure:
\begin{enumerate}
\item<4->  Find an equation relating the two quantities.
\item<4->  Use the Chain Rule to differentiate both sides with respect to time.
\end{enumerate}
\end{itemize}
\end{frame}
% end module related-rates-intro
