% begin module sequence-ex1
\begin{frame}
Some sequences can by defined by giving a formula for the $n$th term $a_n$.  This example expresses four different sequences in three different ways: first, by using the preceding notation; second, by giving a formula; and third, by writing out the terms of the sequence.
\begin{example}
\[
\begin{array}{lll}
\left\{ \frac{n}{n+1}\right\} &%
a_n = \frac{n}{n+1} &%
\left\{ \frac{1}{2}, \frac{2}{3}, \frac{3}{4}, \frac{4}{5}, \ldots \right\}\\%
&&\\%
\left\{ \frac{(-1)^n(n+1)}{3^n}\right\} &%
a_n = \frac{(-1)^n(n+1)}{3^n} &%
\left\{ \frac{-2}{3}, \frac{3}{9}, \frac{-4}{27}, \frac{5}{81}, \ldots \right\}\\%
&&\\%
\left\{ \sqrt{n-3}\right\}_{n=3}^\infty &%
a_n = \sqrt{n-3}, n\geq 3&%
\left\{ 0, 1, \sqrt{2}, \sqrt{3}, \ldots \right\}\\%
&&\\%
\left\{ \cos \frac{n\pi}{6}\right\}_{n=0}^\infty &%
a_n = \cos \frac{n\pi}{6}, n\geq 0&%
\left\{ 1, \frac{\sqrt{3}}{2}, \frac{1}{2}, 0, \ldots \right\}\\%
\end{array}
\]
\end{example}
\end{frame}
% end module sequence-ex1
