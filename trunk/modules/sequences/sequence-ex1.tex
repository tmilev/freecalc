% begin module sequence-ex1
\begin{frame}
\frametitle{Sequences via formulas}
\begin{itemize}
\item Sequences can be defined by presenting a formula to obtain the $n^{th}$ term $a_n$ as a function of the index $n$.
\item<2-> Another frequently used notation: include the f-la in parenthesis and indicate the index ranges by super- and subscripts.
\item<3-> There is a third \alertNoH{3}{informal but frequently used notation}: list few terms of the sequence and \alertNoH{4}{let the reader guess the formula}. 
\end{itemize}
\vskip -0.15cm
\begin{example}
$
\renewcommand{\arraystretch}{1.7}
\begin{array}{lll}
\alertNoH{1}{a_n = \frac{n}{n+1}} & % 
\alertNoH{2}{ \left( \frac{n}{n+1}\right)_{n=1}^{\infty}} &
\uncover<3->{\alertNoH{3,4}{\left( \frac{1}{2}, \frac{2}{3}, \frac{3}{4}, \frac{4}{5}, \ldots \right)}} \\%
\alertNoH{1}{a_n = \frac{(-1)^n(n+1)}{3^n}} &%
\alertNoH{2}{ \left( \frac{(-1)^n(n+1)}{3^n} \right)_{n=1}^{\infty}} &%
\uncover<3->{\alertNoH{3,4}{\left( \frac{-2}{3}, \frac{3}{9}, \frac{-4}{27}, \frac{5}{81}, \ldots \right)}}\\%
\alertNoH{1}{a_n = \sqrt{n-3}, n\geq 3}&%
\alertNoH{2}{\left( \sqrt{n-3}\right)_{n=3}^\infty} &%
\uncover<3->{\alertNoH{3,4}{\left( 0, 1, \sqrt{2}, \sqrt{3}, \ldots \right)}}\\%
\alertNoH{1}{a_n = \cos \left(\frac{n\pi}{6}\right), n\geq 0}&%
\alertNoH{2}{ \left( \cos \frac{n\pi}{6}\right)_{n=0}^\infty} &%
\uncover<3->{\alertNoH{3,4}{\left( 1, \frac{\sqrt{3}}{2}, \frac{1}{2}, 0, \ldots \right)}}\\%
\end{array}
$
\end{example}
\end{frame}
% end module sequence-ex1
