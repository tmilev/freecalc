% begin module sequence-ex2
\begin{frame}
\frametitle{Sequences via recursion}
\begin{itemize}
\item Sequences can be defined by recursive formulas.
\item<2-> A sequence formula is recursive if it expresses the term $a_n$ via the preceding terms $a_1, a_2,\dots, a_{n-1}$, rather than directly as a function of $n$.
\uncover<3->{
\begin{example}[Defining sequences by recursion]
Define recursively the Fibonacci sequence $( f_n)_{n=1}^{\infty}$ by requesting that
\[
f_1 = \alertNoH{4}{1} \qquad f_2 = \alertNoH{4}{1} \qquad \alertNoH{5-14}{ f_n = f_{n-1} + f_{n-2}}, \quad n\geq 3.
\]
\uncover<4->{%
The first few terms are

\hfil \hfil$\alertNoH{4-6}{1},%
\alertNoH{4-8}{1},%
\uncover<handout:0|4->{
\fcAnswer{6}{\alertNoH{6-10}{{2},}}%
\fcAnswer{8}{\alertNoH{8-12}{{3},}}%
\fcAnswer{10}{\alertNoH{10-14}{{5},}}%
\fcAnswer{12}{\alertNoH{12-14}{{8},}}%
\fcAnswer{14}{{13},\ldots }%
}
$
}%
\end{example}
}
\item<15->
In fact the Fibonacci sequence can be described by a formula, but it is not very simple: $\displaystyle a_n=\frac{ \sqrt{5}}{ 5} \left( \left(\frac{1+\sqrt{5 }}{2} \right)^n- \left(\frac{ 1-\sqrt{ 5}}{2}\right)^n\right) $.%

\end{itemize}
\end{frame}

\begin{frame}
\frametitle{Sequences via inclusion criterion}
\begin{itemize}
\item A sequence can also be given by specifying a criterion to check whether a number should be included in the sequence or not.

\uncover<2->{
\begin{example}[Defining sequence by criterion]
Define $\left(p_n\right)_{n=1}^{\infty}$ as the sequence of all primes.

\hfil \hfil $\left(2,3,5,7,11, 13, 17, 19, 23, 29, 31,\dots\right)$

\end{example}
}
\item<3-> We know how to check whether a number is prime.
\item<4-> For example, a crude test for whether a number is prime is to check whether it is divisible by all positive numbers smaller than it.
\item<5-> Our sequence is well defined; we could generate it, say, by computer.
\item<6-> However, we have given no closed or even recursive formula to generate the entire sequence.
\end{itemize}
\end{frame}
\begin{frame}
\frametitle{Sequences defined indirectly}

\begin{itemize}
\item We note that in addition to the illustrated ways to define sequences, we are also free to use for the task any well-posed statement. 
\item<2-> Such ways to define a sequence may be very indirect or obscure and we will not use them in our course.
\item<3-> We hint the challenges that can arise by using arbitrary (but well-posed) definitions on a few examples.
\end{itemize}
\begin{example}
\begin{enumerate}
\item<4->  Let $a_n$ be the $n^{th}$ digit in the decimal expansion of the number $e$. \uncover<5->{The first few terms of $( a_n)$:
\begin{center}
$2, 7, 1, 8, 2, 8, 1, 8, 2, 8, 4, 5, \ldots $ 
\end{center} 
}%

\item<6-> Consider the sequence $( p_n)$, where $p_n$ is the population of the world as of January 1 of year $n$. 
\end{enumerate}
\end{example}
\end{frame}

% end module sequence-ex2



