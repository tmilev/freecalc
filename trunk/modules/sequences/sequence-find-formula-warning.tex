\begin{frame}
\frametitle{Warning about implied sequence formulas}
\begin{itemize}
\item We found the sequence 
$
\left( %
0,%
 \frac{1}{4},%
-\frac{2}{8},%
 \frac{3}{16},%
-\frac{4}{32},%
 \frac{5}{64},%
\ldots
\right)%
$
can be given by:
$
a_n =(-1)^n\frac{n-1}{2^n}
$
\item<2-> For any finite number of terms we can produce infinitely many different formulas that match them - but disagree on the terms after.
\item<3-> For example the sequence above can also be obtained by:
\[
a_n=\frac{27}{512} n^{5}-\frac{477}{512} n^{4}+\frac{3159}{512} n^{3}-\frac{9651}{512} n^{2}+\frac{6643}{256} n-\frac{793}{64}
\]
\uncover<4->{and that produces $a_7=\frac{363}{32} $.}

\item<5-> Bear in mind that using implied sequence formulas is \textbf{informal}. 
\begin{itemize}
\item<6-> It is acceptable to use the implied sequence notation only when we believe there is a single completely obvious pattern that will be recognized by every one.
\item<7-> The pattern should be obvious not only to us, but also to our potential readers.
\item<8-> If in doubt we should switch to a more rigorous notation. 
\end{itemize}  

\end{itemize}


\end{frame}