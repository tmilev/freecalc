% begin module alternating-def
\begin{frame}
\frametitle{Alternating Series}
\begin{definition}[Alternating Series]
An alternating series is a series whose terms are alternately positive and negative.
\end{definition}
\uncover<2->{%
\begin{examples}
Here are two examples:
\abovedisplayskip=0pt
\[
\begin{array}{r@{\ }c@{\ }c@{\ }c@{\ }c@{\ }c@{\ }c@{\ }c@{\ }c@{\ }c@{\ }c@{\ }c@{\ }c@{\ }l}
\alert<handout:0| 3-6>{1} &%
\alert<handout:0| 7-8>{-} &%
\frac{\alert<handout:0| 5-6>{1}}{\alert<handout:0| 3-4>{2}} &%
 + &%
\frac{\alert<handout:0| 5-6>{1}}{\alert<handout:0| 3-4>{3}} &%
\alert<handout:0| 7-8>{-} &%
\frac{\alert<handout:0| 5-6>{1}}{\alert<handout:0| 3-4>{4}} &%
 + &%
\frac{\alert<handout:0| 5-6>{1}}{\alert<handout:0| 3-4>{5}} &%
\alert<handout:0| 7-8>{-} &%
\frac{\alert<handout:0| 5-6>{1}}{\alert<handout:0| 3-4>{6}} &%
 + &%
\cdots &%
\displaystyle  = \sum_{n=1}^\infty \uncover<8->{\alert<handout:0| 8>{(-1)^{n-1}}}\frac{\uncover<6->{\alert<handout:0| 6>{1}}}{\uncover<4->{\alert<handout:0| 4>{n}}} \\%
\alert<handout:0| 13-14>{-} %
\frac{\alert<handout:0| 11-12>{1}}{\alert<handout:0| 9-10>{2}} &%
 + &%
\frac{\alert<handout:0| 11-12>{2}}{\alert<handout:0| 9-10>{3}} &%
\alert<handout:0| 13-14>{-} &%
\frac{\alert<handout:0| 11-12>{3}}{\alert<handout:0| 9-10>{4}} &%
 + &%
\frac{\alert<handout:0| 11-12>{4}}{\alert<handout:0| 9-10>{5}} &%
\alert<handout:0| 13-14>{-} &%
\frac{\alert<handout:0| 11-12>{5}}{\alert<handout:0| 9-10>{6}} &%
 + &%
\frac{\alert<handout:0| 11-12>{6}}{\alert<handout:0| 9-10>{7}} &%
\alert<handout:0| 13-14>{-} &%
\cdots &%
\displaystyle  = \sum_{n=1}^\infty \uncover<14->{\alert<handout:0| 14>{(-1)^{n}}}\frac{\uncover<12->{\alert<handout:0| 12>{n}}}{\uncover<10->{\alert<handout:0| 10>{n+1}}} %
\end{array}
\]
\end{examples}
}%
%\begin{itemize}
\uncover<15->{The $n$th term of an alternating series has the form}
\[
\uncover<15->{%
a_n = (-1)^{n-1}b_n \qquad \textrm{or}\qquad a_n = (-1)^n b_n
}%
\]
\uncover<15->{where $b_n$ is positive.}
%\end{itemize}
\end{frame}
% end module alternating-def
