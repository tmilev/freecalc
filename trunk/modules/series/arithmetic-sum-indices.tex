% begin module arithmetic-sum-indices
\begin{frame}
The indices in series notation don't always start at $n = 1$.  
\begin{example}[Different starting index]
Find the sum of the series
\abovedisplayskip=0pt
\belowdisplayskip=0pt
\[
\sum_{n=3}^{14} 4 + \frac{n}{2}.
\]
\alert<handout:0| 2-3>{\uncover<2->{This series is \uncover<3->{arithmetic.}}}
\uncover<4->{%
\begin{align*}
\alert<handout:0| 4-5>{\text{The first term is } a_3} & \alert<handout:0| 4-5>{= \uncover<5-| handout:0>{4+\frac{3}{2}=\frac{11}{2}.}} \\
\alert<handout:0| 6-7>{\text{The last term is } a_{14}} & \alert<handout:0| 6-7>{= \uncover<7-| handout:0>{4+\frac{14}{2}=11.}} \\
\alert<handout:0| 8-9>{\text{The number of terms is } M} & \alert<handout:0| 8-9>{= \uncover<9-| handout:0>{12.}} \\
\alert<handout:0| 10-11>{\text{The sum is } s} & \alert<handout:0| 10-11>{= \uncover<11-| handout:0>{\frac{11/2+11}{2}\cdot 12 = \frac{33}{4}\cdot 12 = 99.}}
\end{align*}
}%
\end{example}

\end{frame}
% end module arithmetic-sum-indices
