% begin module integral-test-below
\begin{frame}
\begin{columns}
\column{.56\textwidth}
\[
\alertNoH{15}{ \sum_{n=1}^\infty \alertNoH{5,11}{\frac{1}{\sqrt{n}}} =\alertNoH{6-10}{\frac{1}{\sqrt{1}}} \alertNoH{7-10}{+\frac{1}{\sqrt{2}}} \alertNoH{8-10}{+ \frac{1}{\sqrt{3}}} \alertNoH{9-10}{+ \frac{1}{\sqrt{4}}} \alertNoH{10}{+ \cdots} }
\]
\begin{itemize}
\item<2->  Use a computer to calculate partial sums.
\item<3->  Appears to be diverging.
\item<4->  How do we prove it?
\item<5->  Use $\alertNoH{5,11}{f(x) = \frac{1}{\sqrt{x}}}$.
\end{itemize}
%\begin{center}breaks in ubuntu
\uncover<5->{%
\psset{xunit=1.1cm, yunit=1.1cm}
\begin{pspicture}(-0.3,-0.8)(6.5,1.6)%
\tiny%
\fcBoundingBox{-0.3}{-0.8}{6.5}{1.6}%
\fcLabels{6.3}{1.5}%
\newcommand{\oneBlock}[2]{%
\pstVerb{1 dict begin /x ####1\space def}%
\psline*[linecolor=####2](! x 0)(!x 1 x sqrt div)(!x 1 add 1 x sqrt div)(! x 1 add 0)(! x 0)%
\psline[linecolor=brown](! x 0)(!x 1 x sqrt div)(!x 1 add 1 x sqrt div)(! x 1 add 0)(! x 0)%
\pstVerb{end}%
}%
\newcommand{\oneBlockLabel}[2]{%
\pstVerb{1 dict begin /theArgument ####1\space def}%
\psline[arrows=->, linewidth=0.5pt](! theArgument 0.5 add -0.2)(! theArgument 0.5 add 0.5 theArgument sqrt div)%
\rput[t](! theArgument 0.5 add -0.3 ){####2}%
\pstVerb{end}%
}%
\uncover<handout:1,3|6-11,15->{\oneBlock{1}{orange}}%
\uncover<handout:1|6-11>{\oneBlockLabel{1}{$\alertNoH{6}{a_1=1\vphantom{\frac{1}{\sqrt{2}}}}$}}%
\uncover<handout:1,3|7-11,15-> {\oneBlock{2}{orange}}%
\uncover<handout:1|7-11>{\oneBlockLabel{2}{$\alertNoH{7}{a_2=\frac{1}{\sqrt{2}}}$}}%
\uncover<handout:1,3|8-11,15->{\oneBlock{3}{orange}}%
\uncover<handout:1|8-11> {\oneBlockLabel{3}{$\alertNoH{8}{a_3=\frac{1}{\sqrt{3}}}$}}%
\uncover<handout:1,3|9-11,15-> {\oneBlock{4}{orange}}%
\uncover<handout:1|9-11> {\oneBlockLabel{4}{$\alertNoH{9}{a_4=\frac{1}{2}}$}}%
\uncover<handout:1,3|10-11,15->{\oneBlock{5}{orange}}%
\uncover<handout:1|10-11>{\oneBlockLabel{5}{$\alertNoH{10}{a_5=\frac{1}{\sqrt{5}}}$}}%
\uncover<handout:2|12,13,14>{%
\pscustom*[linecolor=orange]{
\psplot{1}{6.2}{1 x sqrt div}
\psline(6.2, 0)(1,0)(1,1)
}%
}%
\psplot[linecolor=\fcColorGraph]{1}{6.3}{1 x sqrt div}%
\rput(2, 1.2){$\alertNoH{5}{y=\frac{1}{\sqrt{x}}}$}%
\fcXTickWithLabel{1}{$1$}%
\fcXTickWithLabel{2}{$2$}%
\fcXTickWithLabel{3}{$3$}%
\fcXTickWithLabel{4}{$4$}%
\fcXTickWithLabel{5}{$5$}%
\fcXTickWithLabel{6}{$6$}%
\uncover<handout:2,3|14->{%
\rput[t](2.6, -0.3){\alertNoH{13,14,15}{infinite area}}%
\psline[arrows=->, linewidth=0.5pt](2.6, -0.2)(2.6, 0.4)%
}%
\fcAxesStandardNoFrame{-0.3}{-0.3}{6.3}{1.5}%
\end{pspicture}
}
%\end{center}
\column{.44\textwidth}
\uncover<2->{%
$
\begin{array}{|r@{\ }|r|}
\hline
n & s_n = \sum_{i=1}^n \frac{1}{\sqrt{i}}\\
\hline
     5 & 3.2317 \\
    10 & 5.0210 \\
    50 & 12.7524 \\
   100 & 18.5896 \\
   500 & 43.2834 \\
  1000 & 61.8010 \\
  5000 & 139.9681 \\
\hline
\end{array}
$
}%
\begin{itemize}
\item<6->  $\frac{1}{\sqrt{1}}$ is the area of a rectangle.
\item<7->  So is $\frac{1}{\sqrt{2}}$.
\item<handout:2-| 11->  \alertNoH{13,14}{\alertNoH{12}{$ \int_1^\infty \alertNoH{11}{\frac{1}{ \sqrt{x}}} \diff x$} is} \fcAnswerUncover{11}{14}{divergent.}
\item<handout:3| 15->  Therefore $\sum\limits_{n=1}^\infty \frac{1}{\sqrt{n}}$ is divergent.
\end{itemize}
\end{columns}
\end{frame}
% end module integral-test-below
