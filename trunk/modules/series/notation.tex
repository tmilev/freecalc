% begin module notation
\begin{frame}
\begin{example}[Series notation]
The series 
\abovedisplayskip=0pt
\belowdisplayskip=0pt
\[
2 + 4 + 6 + 8 + \ldots + 124
\]
can be written more concisely as
\abovedisplayskip=0pt
\belowdisplayskip=0pt
\[
\sum_{n=1}^{62} 2 + 2(n-1).
\]
$2+2(n-1)$ is the $n$th term, and the sigma sign $\sum$ tell us to add all these terms, starting from $n=1$ and going up to $n=62$.  
In this notation $n$ is called the index.  
\end{example}

\uncover<2->{%
\begin{example}[More series notation]
\[
\text{Write } \frac{2}{3} -\frac{4}{9} + \frac{8}{27} - \frac{16}{81} + \frac{32}{243} - \frac{64}{729} \text{ using series notation.}
\]
\abovedisplayskip=0pt
\belowdisplayskip=0pt
\[
\uncover<3->{\sum_{n=1}^{\uncover<4-| handout:0>{\alert<4>{6}}}} \uncover<5-| handout:0>{\alert<5>{\frac{2}{3}\Big(-\frac{2}{3}\Big)^{n-1}}}
\]
\end{example}
}%
\end{frame}
% end module notation
