% begin module ratio-test-inconclusive
\begin{frame}
\alert<handout:0| 21>{The Ratio Test is inconclusive if $\displaystyle \lim_{n\to\infty} \left| \frac{a_{n+1}}{a_n}\right| = 1.$}

\uncover<2->{%
\begin{example}
\begin{columns}
\column{.2\textwidth}
\abovedisplayskip=0pt
\belowdisplayskip=0pt
\[
\sum_{n=1}^\infty \frac{1}{n^2}
\]
\column{.8\textwidth}
\begin{itemize}
\item<3->  This is a $p$-series with \alert<handout:0| 3-4>{$p = \uncover<4->{2.}$}
\item<3->  Therefore it is \uncover<5->{\alert<handout:0| 5,21>{convergent}.}
\end{itemize}
\end{columns}
\abovedisplayskip=0pt
\belowdisplayskip=0pt
\[
\uncover<6->{%
\left| \frac{a_{n+1}}{a_n}\right|  = \frac{\frac{1}{(n+1)^2}}{\frac{1}{n^2}}%
}%
\uncover<7->{%
 = \frac{n^2}{(n+1)^2}\uncover<8->{\alert<handout:0| 8>{\cdot \frac{\frac{1}{n^2}}{\frac{1}{n^2}}}}%
}%
\uncover<9->{%
\alert<handout:0| 10-11>{%
 = \frac{1}{\left( 1 + \frac{1}{n}\right)^2}%
}%
}%
\uncover<10->{%
\alert<handout:0| 10-11>{%
 \to \uncover<11->{\alert<handout:0| 21>{1}} \qquad \textrm{ as } n\to\infty
}%
}%
\]
\end{example}

\begin{example}
\begin{columns}
\column{.2\textwidth}
\abovedisplayskip=0pt
\belowdisplayskip=0pt
\[
\sum_{n=1}^\infty \frac{1}{n}
\]
\column{.8\textwidth}
\begin{itemize}
\item<12->  This is a $p$-series with \alert<handout:0| 12-13>{$p = \uncover<13->{1.}$}
\item<12->  Therefore it is \uncover<14->{\alert<handout:0| 14,21>{divergent}.}
\end{itemize}
\end{columns}
\abovedisplayskip=0pt
\belowdisplayskip=0pt
\[
\uncover<15->{%
\left| \frac{a_{n+1}}{a_n}\right|  = \frac{\frac{1}{n+1}}{\frac{1}{n}}%
}%
\uncover<16->{%
 = \frac{n}{n+1}\uncover<17->{\alert<handout:0| 17>{\cdot \frac{\frac{1}{n}}{\frac{1}{n}}}}%
}%
\uncover<18->{%
\alert<handout:0| 19-20>{%
 = \frac{1}{ 1 + \frac{1}{n}}%
}%
}%
\uncover<19->{%
\alert<handout:0| 19-20>{%
 \to \uncover<20->{\alert<handout:0| 21>{1}} \qquad \textrm{ as } n\to\infty
}%
}%
\]
\end{example}

}%
\end{frame}
% end module ratio-test-inconclusive
