% begin module series-convergence
\begin{frame}
\begin{itemize}
\item  Does it make sense to add infinitely many numbers?
\item<2->  Sometimes yes, sometimes no.
\item<3->  Consider the series $\sum_{n=1}^\infty n$.
\end{itemize}
\uncover<3->{%
\abovedisplayskip=2pt
\[
\alertNoH{ 4-15}{ 1}%
\alertNoH{ 6-15}{+2}%
\alertNoH{ 8-15}{+3}%
\alertNoH{ 10-15}{+4}%
\alertNoH{ 12-15}{+5}%
\alertNoH{ 14-15}{+ \cdots + n}%
+ \cdots
\]
\belowdisplayskip=2pt
}%
\begin{itemize}
\item<4->  If we add the terms, we get the partial sums %
\uncover<5->{\alertNoH{ 5}{$1$},}%
\uncover<7->{\alertNoH{ 7}{$3$},}%
\uncover<9->{\alertNoH{ 9}{$6$},}%
\uncover<11->{\alertNoH{ 11}{$10$},}%
\uncover<13->{\alertNoH{ 13}{$15$}.}
\item<14->  After the $n$th term, we get \uncover<15->{\alertNoH{ 15}{$\frac{n(n+1)}{2}$}.}%
\item<16->  This goes to $\infty$ as $n$ gets bigger.
\item<17->  Now consider the series $\sum_{n=1}^\infty \frac{1}{2^n}$.
\end{itemize}
\uncover<17->{%
\abovedisplayskip=2pt
\[
\alertNoH{ 18-29}{ \frac{1}{2}}%
\alertNoH{ 20-29}{+\frac{1}{4}}%
\alertNoH{ 22-29}{+\frac{1}{8}}%
\alertNoH{ 24-29}{+\frac{1}{16}}%
\alertNoH{ 26-29}{+\frac{1}{32}}%
\alertNoH{ 28-29}{+ \cdots + \frac{1}{2^n}}%
+ \cdots
\]
\belowdisplayskip=2pt
}%
\begin{itemize}
\item<18->  If we add the terms, we get the partial sums %
\uncover<19->{\alertNoH{ 19}{$\frac{1}{2}$},}%
\uncover<21->{\alertNoH{ 21}{$\frac{3}{4}$},}%
\uncover<23->{\alertNoH{ 23}{$\frac{7}{8}$},}%
\uncover<25->{\alertNoH{ 25}{$\frac{15}{16}$},}%
\uncover<27->{\alertNoH{ 27}{$\frac{31}{32}$}.}
\item<28->  After the $n$th term, we get \uncover<29->{\alertNoH{ 29}{$1-\frac{1}{2^n}$}.}%
\item<30->  This gets closer and closer to $1$.  We write $\sum_{n=1}^\infty \frac{1}{2^n} = 1$.
\end{itemize}
\end{frame}


\begin{frame}
\begin{definition}[Partial Sum, Convergent, Divergent, Sum]
Given a series $\sum_{i=1}^\infty a_i = a_1 + a_2 + a_3 + \cdots$, let $s_n$ denote the $n$th partial sum:
\[
s_n = \sum_{i=1}^n a_i = a_1 + a_2 + \cdots + a_n
\]

If the sequence $\{ s_n\}$ is convergent and $\lim_{n\to \infty} s_n = s$, then we say that the series $\sum_{i=1}^\infty a_i$ is convergent, and we write
\[
\sum_{i=1}^\infty a_i = s.
\]
In this case, we call $s$ the sum of the series.

If the sequence $\{ s_n\}$ is divergent, then we say that the series $\sum_{i=1}^\infty a_i$ is divergent.
\end{definition}
\end{frame}
% end module series-convergence
