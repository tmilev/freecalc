% begin module series-convergence
\begin{frame}
\begin{itemize}
\item  Does it make sense to add infinitely many numbers?
\item<2->  Sometimes yes, sometimes no.
\item<3->  Consider the series $\sum_{n=1}^\infty n$.
\end{itemize}
\uncover<3->{%
\abovedisplayskip=2pt
\[
\alert<handout:0| 4-15>{ 1}%
\alert<handout:0| 6-15>{+2}%
\alert<handout:0| 8-15>{+3}%
\alert<handout:0| 10-15>{+4}%
\alert<handout:0| 12-15>{+5}%
\alert<handout:0| 14-15>{+ \cdots + n}%
+ \cdots 
\]
\belowdisplayskip=2pt
}%
\begin{itemize}
\item<4->  If we add the terms, we get the partial sums %
\uncover<5->{\alert<handout:0| 5>{$1$},}%
\uncover<7->{\alert<handout:0| 7>{$3$},}%
\uncover<9->{\alert<handout:0| 9>{$6$},}%
\uncover<11->{\alert<handout:0| 11>{$10$},}%
\uncover<13->{\alert<handout:0| 13>{$15$}.}
\item<14->  After the $n$th term, we get \uncover<15->{\alert<handout:0| 15>{$\frac{n(n+1)}{2}$}.}%
\item<16->  This goes to $\infty$ as $n$ gets bigger.
\item<17->  Now consider the series $\sum_{n=1}^\infty \frac{1}{2^n}$.
\end{itemize}
\uncover<17->{%
\abovedisplayskip=2pt
\[
\alert<handout:0| 18-29>{ \frac{1}{2}}%
\alert<handout:0| 20-29>{+\frac{1}{4}}%
\alert<handout:0| 22-29>{+\frac{1}{8}}%
\alert<handout:0| 24-29>{+\frac{1}{16}}%
\alert<handout:0| 26-29>{+\frac{1}{32}}%
\alert<handout:0| 28-29>{+ \cdots + \frac{1}{2^n}}%
+ \cdots 
\]
\belowdisplayskip=2pt
}%
\begin{itemize}
\item<18->  If we add the terms, we get the partial sums %
\uncover<19->{\alert<handout:0| 19>{$\frac{1}{2}$},}%
\uncover<21->{\alert<handout:0| 21>{$\frac{3}{4}$},}%
\uncover<23->{\alert<handout:0| 23>{$\frac{7}{8}$},}%
\uncover<25->{\alert<handout:0| 25>{$\frac{15}{16}$},}%
\uncover<27->{\alert<handout:0| 27>{$\frac{31}{32}$}.}
\item<28->  After the $n$th term, we get \uncover<29->{\alert<handout:0| 29>{$1-\frac{1}{2^n}$}.}%
\item<30->  This gets closer and closer to $1$.  We write $\sum_{n=1}^\infty \frac{1}{2^n} = 1$.
\end{itemize}
\end{frame}


\begin{frame}
\begin{definition}[Partial Sum, Convergent, Divergent, Sum]
Given a series $\sum_{i=1}^\infty a_i = a_1 + a_2 + a_3 + \cdots$, let $s_n$ denote the $n$th partial sum:
\[
s_n = \sum_{i=1}^n a_i = a_1 + a_2 + \cdots + a_n
\]

If the sequence $\{ s_n\}$ is convergent and $\lim_{n\to \infty} s_n = s$, then we say that the series $\sum_{i=1}^\infty a_i$ is convergent, and we write
\[
\sum_{i=1}^\infty a_i = s.
\]
In this case, we call $s$ the sum of the series.

If the sequence $\{ s_n\}$ is divergent, then we say that the series $\sum_{i=1}^\infty a_i$ is divergent.
\end{definition}
\end{frame}
% end module series-convergence
