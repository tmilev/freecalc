% begin module series-geometric-ex1
\begin{frame}
\begin{example}[Example 1, p. 724]
An important example is the geometric series
\abovedisplayskip=2pt
\belowdisplayskip=2pt
\[
a + ar + ar^2 + ar^3 + \cdots + ar^{n-1} + \cdots = \sum_{n = 1}^\infty ar^{n-1}, \qquad a\neq 0
\]
\begin{itemize}
\item<2->  If $r = 1$, then $s_n = a + a + \cdots + a = na \to \pm\infty$.
\item<3->  Since $\lim_{n\to\infty} s_n$ doesn't exist, the series is divergent when $r = 1$.
\item<4->  If $r\neq 1$, then
\end{itemize}
\only<handout:0| -5>{%
\uncover<4->{%
\abovedisplayskip=2pt
\belowdisplayskip=2pt
\[
\begin{array}{rcrrrrrr}
s_n & = & a & + ar & + ar^2 & + \cdots & + ar^{n-1} & \\
\invisible<1->{-}\qquad \uncover<5->{rs_n} & \uncover<5->{=} &  &  \uncover<5->{ar} & \uncover<5->{+ ar^2} & \uncover<5->{+ \cdots} & \uncover<5->{+ ar^{n-1}} & \uncover<5->{+ ar^n}\\
%\hline
\uncover<6->{s_n - rs_n} & \uncover<6->{=} & \multicolumn{6}{l}{\uncover<6->{a-ar^n}}\\
\uncover<7->{s_n} & \uncover<7->{=} & \multicolumn{6}{l}{\uncover<7->{\frac{a(1-r^n)}{1-r}}}
\end{array}
\]
}}%
\only<handout:1| 6->{%
\abovedisplayskip=2pt
\belowdisplayskip=2pt
\[
\begin{array}{rcrrrrrr}
\alert<handout:0| 8>{s_n} & = & \alert<handout:0| 9>{a} & + ar & + ar^2 & + \cdots & + ar^{n-1} & \\
\alert<handout:0| 8>{-\qquad \uncover<5->{rs_n}} & \uncover<5->{=} &  &  \uncover<5->{ar} & \uncover<5->{+ ar^2} & \uncover<5->{+ \cdots} & \uncover<5->{+ ar^{n-1}} & \uncover<5->{+ \alert<handout:0| 9>{ar^n}}\\
\hline
\uncover<7->{\alert<handout:0| 8>{s_n - rs_n}} & \uncover<7->{=} & \multicolumn{6}{l}{\uncover<7->{\alert<handout:0| 9>{a-ar^n}}}\\
\uncover<10->{s_n} & \uncover<10->{=} & \multicolumn{6}{l}{\uncover<10->{\frac{a(1-r^n)}{1-r}}}
\end{array}
\]
}%

\begin{itemize}
\item<11->  If $-1 < r < 1$, then $r^n\to 0$, so the geometric series is convergent and its sum is $a/(1-r)$.
\item<12->  If $r > 1$ or $r \leq -1$, then $r^n$ is divergent, so $\sum_{n=1}^\infty ar^{n-1}$ diverges.
\end{itemize}
\end{example}
\end{frame}
% end module series-geometric-ex1
