% begin module series-geometric-ex4
\begin{frame}
\begin{example}[Example 4, p. 726]
Write the number $2.3\overline{17} = 2.3171717\ldots$ as a quotient of integers.
\uncover<2->{%
\abovedisplayskip=0pt
\belowdisplayskip=0pt
\[
\alert<handout:0| 3>{2.3}%
\alert<handout:0| 4>{17}%
\alert<handout:0| 5>{17}%
\alert<handout:0| 6>{17}%
\ldots = 
\alert<handout:0| 3>{2.3}%
+ \alert<handout:0| 7-11>{\alert<handout:0| 4>{\frac{17}{10^3}}%
+ \alert<handout:0| 5>{\frac{17}{10^5}}%
+ \alert<handout:0| 6>{\frac{17}{10^7}}%
+ \cdots}%
\]
}%
\begin{itemize}
\item<7->  After the first term, we have a geometric series.
\item<8->  \alert<handout:0| 8-9,12-13>{$a =$ \uncover<9->{$\frac{17}{10^3}$}} and \alert<handout:0| 10-11,14-15>{$r =$ \uncover<11->{$\frac{1}{10^2}$.}}
\end{itemize}
\begin{eqnarray*}
\uncover<12->{%
2.3171717\ldots%
}%
& \uncover<12->{ = } &%
\uncover<12->{%
2.3 + \frac{\uncover<13->{\alert<handout:0| 13>{\frac{17}{10^3}}}}{1- \uncover<15->{\alert<handout:0| 15>{\frac{1}{10^2}}}}%
}%
 \uncover<16->{ = } %
\uncover<16->{%
2.3 + \frac{\uncover<13->{\alert<handout:0| 13>{\frac{17}{1000}}}}{\uncover<15->{\alert<handout:0| 15>{\frac{99}{100}}}}%
}\\%
& \uncover<17->{ = } &%
\uncover<17->{%
\frac{23}{10} + \frac{17}{990}%
}%
 \uncover<18->{ = } %
\uncover<18->{%
\frac{1147}{495}%
}%
\end{eqnarray*}
\end{example}
\end{frame}
% end module series-geometric-ex4
