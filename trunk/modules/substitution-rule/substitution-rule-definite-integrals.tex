% begin module substitution-rule-definite-integrals
\begin{frame}
\frametitle{Definite Integrals}
There are two ways to find a definite integral with the Substitution Rule:
\vspace{-.2in}
\begin{enumerate}
\item  First evaluate the indefinite integral, then use the FTC.
\end{enumerate}
\abovedisplayskip=0pt
\belowdisplayskip=0pt
\abovedisplayshortskip=0pt
\belowdisplayshortskip=0pt
\begin{align*}
\uncover<2->{%
\int_{\alertNoH{ 2}{0}}^{\alertNoH{ 2}{4}} \sqrt{2x+1} \ \diff x%
}%
 & \uncover<2->{ = } %
\uncover<2->{%
\left[ \alertNoH{3}{\int \sqrt{2x+1} \ \diff x} \right]_{\alertNoH{ 2}{0}}^{\alertNoH{ 2}{4}}%
}%
  \uncover<3->{ = } %
\uncover<3->{%
\left[\alertNoH{3}{ \frac{1}{3}(2\alertNoH{4,5}{x}+1)^{\frac{3}{2}}} \right]_{\alertNoH{ 2,5}{0}}^{\alertNoH{ 2,4}{4}}%
}\\%
 & \uncover<4->{ = \frac{1}{3}(\alertNoH{6}{ 2\cdot \alertNoH{4}{4} + 1} )^{\frac{3}{2}} - \frac{1}{3}(\alertNoH{7}{2\cdot \alertNoH{5}{0} + 1})^{\frac{3}{2}}%
}\\%
 & \uncover<6->{= \frac{1}{3}\alertNoH{8}{( \alertNoH{6}{9} )^{\frac{3}{2}}} - \frac{1}{3}(\alertNoH{7}{1})^{\frac{3}{2}}%
}%
\uncover<8->{ = \frac{1}{3}(\alertNoH{8}{27} - 1) \uncover<9->{= \frac{26}{3}}%
}%
\end{align*}
\begin{enumerate}
\setcounter{enumi}{1}
\item<10->  Change the limits of integration when the variable is changed.
\end{enumerate}
\uncover<11->{%
\begin{theorem}[The Substitution Rule for Definite Integrals]
If $g'$ is continuous on $[a,b]$ and $f$ is continuous on the range of $ g$, then
%If $u = g(x)$ and $f$ and $g'$ are continuous, then
\abovedisplayskip=0pt
\belowdisplayskip=0pt
\[
\int_{\alertNoH{ 11}{a}}^{\alertNoH{ 11}{b}} f(g(x))g'(x) \ \diff x = \int_{\alertNoH{11}{ g(a)}}^{\alertNoH{11}{g (b)}} f(u) \ \diff u%
\]
\end{theorem}
}%
\end{frame}
% end module substitution-rule-definite-integrals
