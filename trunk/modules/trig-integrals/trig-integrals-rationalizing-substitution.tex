%begin module trig-integrals-rationalizing-substitution
\begin{frame}
\frametitle{Integrals of the form $\int R(\cos \theta,\sin \theta) \diff \theta$, $R$}

Let $R$ be an arbitrary rational function in two variables (quotient of polynomials in two variables).
\begin{question}
Can we integrate $\int R(\cos \theta, \sin \theta)\diff \theta$?
\end{question}
\begin{itemize}
\item<2-> Yes. We will learn how in what follows.
\item<3-> The algorithm for integration is roughly:
\begin{itemize}
\item<4-> Apply the substitution $\theta=2\Arctan t$ to transform to integral of rational function.
\item<5-> Solve as previously studied.
\end{itemize}
\end{itemize}
\end{frame}

\begin{frame}
\frametitle{The rationalizing substitution $\theta= 2\Arctan t$}
\uncover<13->{
\noindent
Let $R$- rational function in two variables.
$\int R(\alertNoH{14,21}{\cos \theta,\sin \theta} ) \alertNoH{22}{\diff \theta} $
can be integrated via the  substitution $\alertNoH{14,15,18,23, 26}{ \theta=2\arctan t} $.
\uncover<14->{ How does this transform \alertNoH{14,21}{$\sin \theta$, $\cos\theta$}? }\uncover<22->{How does this transform $\alertNoH{22}{\diff \theta} $?} \uncover<26->{\alertNoH{26}{How is $t$ expressed via $\theta$?}}
\[
\begin{array}{rcl}
\uncover<14->{ \alertNoH{14,21}{\sin\alertNoH{15}{\theta}}} &\uncover<14->{=} &\displaystyle \uncover<15->{ \alertNoH{16}{\sin (\alertNoH{15}{ 2\Arctan t} )}} \uncover<16->{ \alertNoH{16}{= \frac{2 \alertNoH{17}{\tan\left( \Arctan t\right)} }{1 + {\alertNoH{17}{\tan}}^2 \alertNoH{17}{ \left(\Arctan t \right)}} }} \uncover<17->{\alertNoH{21}{ = \frac{2\alertNoH{17}{ t}}{1+ {\alertNoH{17}{t }}^2}}}\\
\uncover<14->{\alertNoH{14,21}{\cos \alertNoH{18}{\theta}}  } &\uncover<14->{=} &\displaystyle \alertNoH{19}{ \uncover<18-> {\alertNoH{19}{ \cos (\alertNoH{18}{2\Arctan t}) }} } \uncover<19->{ \alertNoH{19}{= \frac{1-{\alertNoH{20}{\tan} }^2 \alertNoH{20}{ (\Arctan t)}}{1+ {\alertNoH{20}{\tan}}^2 \alertNoH{20}{ (\Arctan t) }}}} \uncover<20->{  \alertNoH{21}{= \frac{1- {\alertNoH{20}{ t}}^2 }{1 +{\alertNoH{20}{t} }^2}}} \\
\only<handout:2|22->{
\uncover<22->{
\alertNoH{22,23}{\diff \theta}}&\uncover<23->{\alertNoH{23}{=}}& \displaystyle \uncover<23->{ \alertNoH{23,24,25}{2 \diff \left(\Arctan t\right)}}\uncover<24->{ \alertNoH{24,25}{=   \fcAnswer{25}{\frac{2}{ 1+t^2}} \diff t}}\\
\uncover<26->{\alertNoH{26,27}{t}&\alertNoH{26,27}{=}&}\displaystyle \uncover<27->{\alertNoH{27}{\tan \left(\frac{\theta}{2}\right)} }
}
\end{array}
\]
}

\only<handout:1|1-20>{
Recall the expression of $\sin (2z), \cos (2z)$ via $\tan z$:
\[
\begin{array}{rcl}
\uncover<1->{\alertNoH{1,2,16}{\sin \left(2z\right)}} &\uncover<1->{\alertNoH{1}{=}}&\displaystyle  \uncover<2->{ \alertNoH{2}{2\sin z\cos z}} \uncover<3->{=\frac{2 \alertNoH{5}{\sin z\cos z} \uncover<4->{\alertNoH{4}{ \frac{1}{\alertNoH{5}{ \cos^2z}}}}}{\alertNoH{3,6}{( \cos^2z +\sin^2z) }\uncover<4->{\alertNoH{4,6}{\frac{1}{\cos^2z}}}}} \uncover<5->{\alertNoH{16}{= \frac{2\alertNoH{5}{\tan z} }{ \alertNoH{6}{ 1+ \tan^2z}} } \quad .}\\
\uncover<1->{\alertNoH{7,8,19}{\cos (2z)} }& \uncover<7->{ \alertNoH{7,8}{= }}&\displaystyle\uncover<8->{\alertNoH{8}{ \cos^2z-\sin^2z}} \uncover<9->{= \frac{ \alertNoH{11}{ \left(\cos^2 z-\sin^2 z\right) \uncover<10->{\alertNoH{10}{ \frac{1}{ \cos^2z} }}}}{\alertNoH{12}{ \alertNoH{9}{\left(\cos^2z +\sin^2 z\right)} \uncover<10->{\alertNoH{10}{ \frac{1}{ \cos^2z }}} }}} \uncover<11->{\alertNoH{19}{ =\frac{\alertNoH{11}{ 1-\tan^2 z} }{\alertNoH{12}{1+\tan^2z}}} ~ .}
\end{array}
\]
}



\uncover<28->{
\begin{theorem}
The substitution given above  transforms $ \int R(\cos \theta, \sin\theta)\diff \theta$ to an integral of a rational function of $t$.
\end{theorem}
}
\vspace{10cm}
\end{frame}
%end module trig-integrals-rationalizing-substitution
