%begin module trig-integrals-without-rationalizing-substitution
\begin{frame}
\frametitle{Trigonometric Integrals - quick ad hoc techniques}
\begin{itemize}
\item As we saw, every rational trigonometric expression can be integrated with the substitution $\theta=2\Arctan t$.
\item<2-> This integration technique results in rather long computations. 
\item<3-> Particular integral types may be computable with quicker ad hoc techniques.
\item<4-> We illustrate such techniques on examples. 
\item<5-> Examples to which our ad hoc techniques apply arise from integrals needed outside of the subject of Calculus II, so these techniques are important.
\item<6-> The trigonometric integral we saw, $\int \frac{\diff \theta}{2\sin \theta -\cos\theta+5}$, will not work with any of following ad-hoc techniques, so the general method is important as well.
\end{itemize}
\end{frame}

%end module trig-integrals-without-rationalizing-substitution
