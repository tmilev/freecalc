%begin module area-under-hyperbola-ex1


\begin{frame}
\begin{example}
Find the area locked b-n the hyperbolas $\alert<2,3>{ y=\pm \sqrt{ x^2+1}}$ and $x=\pm 2\sqrt{ 2}$.
\begin{columns}
\column{.5\textwidth}
\psset{xunit=0.7cm, yunit=0.7cm}
\begin{pspicture}(-3.328427, -3)(3.328427,3)
\psframe*[linecolor=white](-3.328427,-3)(3.328427,3)
\tiny
\uncover<31->{
\pscustom*[linecolor=\fcColorAreaUnderGraph]{
\psplot[linecolor=\fcColorGraph, plotpoints = 1000 ] {-2.828427} {2.828427}{1 x 2 exp add 0.5 exp }
\psline[linecolor=\fcColorGraph](2.828427,-3)(2.828427,3)
\psplot[linecolor=\fcColorGraph, plotpoints=1000] { 2.828427 } {-2.828427}{1 x 2 exp add 0.5 exp -1 mul }
\psline[linecolor=\fcColorGraph](-2.828427,-3)(-2.828427,3)
}
}
\uncover<1-26,28->{
\psaxes[arrows=<->,ticks=none, labels=none](0,0)(-3,-3)(3,3)
}
\psline[linecolor=red!1](3.301,2)(3.302,2)
\psline[linecolor=red!1](-3.301,2)(-3.302,2)

%Function formula: - (x^{2}+1)^{1/2}
\psplot[linecolor=\fcColorGraph, plotpoints=1000]{-2.828427}{2.828427}{1 x 2 exp add 0.5 exp -1 mul }
\uncover<3-4>{\rput[tl](-2.2, -2.4){ \alert<3>{ $y= - \sqrt{ x^2 +1 }$}}}

%Function formula: (x^{2}+1)^{1/2}
\psplot[linecolor=\fcColorGraph, plotpoints=1000]{-2.828427}{ 2.828427 }{1 x 2 exp add 0.5 exp }
\uncover<2-4>{\rput[bl](-2.1, 2.4){\alert<2>{ $y=\sqrt{ x^2 +1} $}}}

\uncover<29->{
\psline[linecolor=\fcColorGraph](-2.828427,3)(-2.828427,-3)
}
\uncover<30->{
\psline[linecolor=\fcColorGraph](2.828427,3)(2.828427,-3)
}
\uncover<25-27>{
\psline{<->}(-2.9,2.9)(2.9,-2.9)
\rput[t](-2.1, 1.7){$\begin{array}{l} \alert<25>{v=0} \\\uncover<1-26>{\alert<25>{y+x=0}} \end{array}$}
}
\uncover<15-27>{
\psline{<->}(-2.9,-2.9)(2.9,2.9)
\rput[b](-2.1, -1.9){$\begin{array}{l} \uncover<1-26>{ \alert<15>{ y-x=0 }}\\\uncover<16->{\alert<16>{u=0}} \end{array}$}
}
\uncover<17-26>{
\fcFullDot{1.4}{1.4}
\rput[l]( 1.6, 1.4){$(\frac{y+x}{2},\frac{y+x}{2})$}
}
\uncover<14-26>{
\fcFullDot{0.6}{2.2}
\rput[lb](0.65, 2.2){$(x,y)$}
}
\uncover<26>{
\psline(0.6,2.2)(-0.8,0.8)
\psline(-0.7, 0.9)(-0.6, 0.8)(-0.7, 0.7)
\rput[rb](-0.3, 1.3){\alert<26>{$v$}}
}
\uncover<18-26>{
\psline(0.6,2.2)(1.4, 1.4)
\psline(1.3, 1.5)(1.2,1.4)(1.3, 1.3)
}
\uncover<23-26>{
\rput[tr](0.95, 1.8){\alert<23>{$u$}}
}
\uncover<14-26>{
\fcFullDot{2.2}{0.6}
\rput[lt]( 2.2, 0.65){$(y,x)$}
}
\end{pspicture}

\vbox to 3.0cm {
\uncover<18->{\alert<18>{
\uncover<22->{\alert<22>{Signed}} distance b-n $(x,y)$ and line $u=0$ equals}}
\only<1-23>{
$\uncover<19->{\uncover<22->{\alert<22>{\pm}} \alert<19>{ \sqrt{ \alert<20>{ \left(x-\frac{(x+y)}{2} \right)^2+ \left( y- \frac{(x+y )}{2} \right)^2}}}}
$
$\uncover<20->{=\uncover<22->{\alert<22>{\pm}} \sqrt{ \alert<20>{ \frac{1}{2}(y-x)^2 }}} \uncover<21->{= \alert<21>{ \uncover<1-21>{\pm} \alert<23>{ \frac{\sqrt{2 }}{ 2 } ( y-x)}}} \uncover<23>{ \alert<23>{=}}$
} %only<1-23>
\uncover<23->{ \alert<23,24>{$u $}.}
\only<24->{\uncover<25->{
Similarly compute that \alert<26>{signed distance b-n $(x,y)$ and the \alert<25>{line $v=0$} equals $v$}.
\uncover<27->{$\Rightarrow$ $y^2-x^2=1$ is the \alert<27>{ hyperbola $v=\frac{1/2}{u}$} in the $(u,v)$-plane.}
}}

\vfil
} %vbox

\column {.5\textwidth}
\only<1-27>{
\uncover<4->{We studied $\alert<27>{v=\frac{1/2}{u}}$ is called a hyperbola:}\uncover<3->{ why do we call $y= \sqrt{ x^2 +1}$ hyperbola?} \uncover<5->{Compute:}
\[
\begin{array}{rcl}
\uncover<5->{\sqrt{x^2+1} &=& y}\\
\uncover<6->{ x^2+1 &=& y^2}\\
\uncover<7->{y^2-x^2&=&1}\\
\uncover<8->{\uncover<9>{\alert<9>{\frac{1}{2}}} \uncover<10->{\alert<10,11>{\frac{\sqrt{2}}{2}}} \alert<11>{(y-x)} \uncover<10->{\alert<10,12>{\frac{\sqrt{2}}{2}}} \alert<12>{(y+x)}&=&\uncover<9->{\alert<9>{\frac{1}{2}}} \uncover<8>{1}}\\
\uncover<11->{\alert<11>{u}\alert<12>{v}&=& \frac{1}{2}}\\
\uncover<13->{\alert<27>{v}&\alert<27>{=}& \alert<27>{\frac{1/2}{u}},}
\end{array}
\]
\uncover<11->{where $\begin{array}{|l}
\alert<11,16,23>{u=\frac{\sqrt{2}}{2} \left(y-x\right)}\\
\alert<12,25>{v=\frac{\sqrt{2}}{2}\left(y+x\right)}
\end{array}$. } \uncover<14->{Consider an arbitrary point $(x,y)$.}
} %only<1-27>
\only<28->{
The area in question is:
$
\begin{array}{l}
\displaystyle\phantom{=} \int \limits^{{{\uncover<28,29>{\alert<29>{ \textbf{?}}}\uncover<30->{\alert<30>{ 2 \sqrt{ 2} } } } } }_{\uncover<28>{\alert<28>{\textbf{?}}}\uncover<29->{ -2\sqrt{2}}} 2\sqrt{x^2+1}\diff x \\
\displaystyle \uncover<32->{= \uncover<33->{\alert<33>{2}} \left[x\sqrt{x^2+1} \vphantom{\ln \left(\sqrt{x^2+1}+x\right) }\right.}\\
\displaystyle \uncover<32->{\left. +\ln \left(\sqrt{x^2+1}+x\right)\right]^{2\sqrt{2}}_{\only<33->{\alert<33>{0}} \uncover<1-32>{-2\sqrt{2}}}}\\
\uncover<34->{=2\left(2\sqrt{2} \sqrt{(2\sqrt{2})^2+1}\right.} \\
\uncover<34->{\left.+ \ln \left(\sqrt{(2\sqrt{2})^2+1}+2\sqrt{2} \right) \right)}\\
\uncover<35->{=12\sqrt{2} +2\ln \left(3+2\sqrt{2}\right )}\\
\uncover<36->{\approx 20.496}
\end{array}
$
}
\end{columns}

\end{example}

\end{frame}

%end module area-under-hyperbola-ex1
