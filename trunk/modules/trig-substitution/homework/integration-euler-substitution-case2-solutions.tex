\solution{\ref{problemintsqrt(1-x^2)dx} 

\textbf{Variant I.} This integral is possibly fastest to solve directly using a trig substitution. In the next variant of the solution we show the Euler substitution.

\noindent $
\begin{array}{rcl@{}l@{}|l}
\displaystyle \int \sqrt{1-x^2}\diff x &=&\displaystyle  \int \sqrt{1-\cos^2\theta}\diff (\cos \theta) &&\text{Set }x=\cos \theta, \theta\in[0,\pi]\\
&=&\displaystyle \int \sqrt{\sin^2\theta } (-\sin \theta)\diff \theta &&\theta\in [0,\pi]\Rightarrow \sin \theta\geq 0\\
&=&\displaystyle -\int \sin^2\theta \diff \theta && \sin^2\theta= \frac{1-\cos (2\theta) }{2}\\
&=&\displaystyle-\int \frac{1-\cos (2\theta)}{2}\diff \theta\\
&=&\displaystyle - \frac{\theta}{2}+\frac{ \sin (2\theta)}{4}+C\\
&=&\displaystyle - \frac{\theta}{2}+\frac{ 2\sin \theta\cos \theta }{4}+C
&&
\begin{array}{@{}r@{}c@{}l}
x &=&\cos \theta\\
\theta &=&\Arccos x\\
\sin \theta&=&\sin \left(\Arccos x\right)\\
&=&\sqrt{1-x^2}
\end{array}
\\
&=&\displaystyle -\frac{\Arccos x}{2}+ \frac{x\sqrt{1-x^2} }{2} + C\\
&=&\displaystyle \frac{\Arcsin x}{2}+\frac{x\sqrt{1- x^2} }{2} + K\quad ,
\end{array}
$
where for the last equality we recall that the derivative of $\Arcsin x$ is minus the derivative of $\Arccos x$.



\textbf{Variant II.} We show how to do this integral via the Euler substitution $x=\cos (2\Arctan t)$.

$
\begin{array}{rcl@{}l@{}|l}
\displaystyle \int \sqrt{1-x^2}\diff x &=&\displaystyle  \int \sqrt{1-\cos^2\theta}\diff (\cos \theta) &&\begin{array}{@{}r@{}c@{}l} &&\text{Set }\\
x&=&\cos(2\Arctan t)\\
\frac{1}{2}\Arccos x&=&\Arctan t\\ 
x&=&\frac{1-t^2}{1+t^2}\\
&=& \frac{2}{1+t^2}-1\\
\sqrt{1-x^2}&=&\frac{2t}{1+t^2}\\
\end{array}\\
&=&\displaystyle \int \frac{2t}{1+t^2}\diff \left(\frac{1-t^2}{1+t^2} \right) \\
&=&\displaystyle \int \frac{2t}{1+t^2} \left(\frac{-4t}{\left(1+t^2\right)^2}\right)\diff t &&\begin{array}{l}
\text{Integral rational}\\
\text{function}\\
\text{we skip details}
\end{array} \\
&=&\displaystyle \frac{-t}{t^{2}+1}+\frac{2 t}{ \left(t^{2}+1\right)^{2}}\\
&&\displaystyle - \Arctan{}t+C \\
&=&\displaystyle -\frac{1}{2}\sqrt{1-x^2} +\frac{\sqrt{1-x^2}}{t^2+1} \\
&&- \Arctan t+C\\
&=&\displaystyle \frac{1}{2} \sqrt{1-x^2} \left(\frac{ 2}{ t^2 +1} -1\right)\\
&&-\Arctan t+C\\
&=&\displaystyle \frac{x\sqrt{1-x^2}}{2}-\frac{1}{2}\arccos x+C\\
&=&\displaystyle \frac{x\sqrt{1-x^2}}{2}+\frac{1}{2}\arcsin x+K,
\end{array}
$
where for the very last equality we used the fact that the derivatives of $\arcsin x$ and $\arccos x$ are negatives of one another.

\textbf{Variant III. } We show how to do this integral geometrically, provided already know the area of a sector of circle. Of course, here we assume we have already derived the formula for an area of a circle. We warn the reader that if we did use an integral to derive the formula for sector area, it is possible we are making a circular reasoning argument. The danger is of course not real we did the integral purely algebraically in the preceding solution variants. In this way, the present solution Variant is simply a geometric interpretation of the problem.

By the Fundamental Theorem of Calculus, the indefinite integral measures up to a constant the area locked under the graph of $\sqrt{1-x^2}$. This graph is a part of a circle. Therefore, up to a constant, $\int\sqrt{1-t^2}\diff t$ equals $\int_{0}^{x}\sqrt{1-t^2}\diff t$. In turn $\int_{0}^{x}\sqrt{1-t^2}\diff t$ is given by the area highlighted in the picture below.

\psset{xunit=2cm, yunit=2cm}
\begin{pspicture}
\tiny
\pscustom*[linecolor=\fcColorAreaUnderGraph]{
\psplot[linecolor=\fcColorGraph]{0}{0.75}{1 x x mul sub sqrt}
\psline(! 0.75 1 0.75 0.75 mul sub sqrt)(0.75, 0)
\psline(0.75, 0)(0,0)
}
\fcAxesStandardNoFrame{-0.3}{-0.3}{1.2}{1.2}

\psplot[linecolor=\fcColorGraph, linewidth=1.5pt]{0}{0.75}{1 x x mul sub sqrt}
\psplot{0.75}{1}{1 x x mul sub sqrt}
\rput[t](0.75,-0.05){$P$}
\fcFullDot{0.75}{0}
\rput(0.375, -0.1){$x$}
\psline(0,0)(! 0.75 1 0.75 dup mul sub sqrt)(0.75, 0)
\rput[l](0.2, 0.5){$A$}
\rput[b](0.5, 0.2){$B$}
\rput[lb](0.45, 0.9){$\Arcsin x $}
\fcFullDot{0}{1}
\rput[r](-0.1, 1){$P$}
\rput[l](0.8, 0.3){$\sqrt{1-x^2}$} 
\rput[l](0.8, 0.65){$Q$}
\fcFullDot{0.75}{1 0.75 0.75 mul sub sqrt}
\rput[tr](-0.05, -0.05){$O$}
\end{pspicture}
\[
\begin{array}{rcll|l}
\text{Area}(A)&=&\displaystyle  \frac{\text{length }\left( \stackrel{\frown}{ PQ}
\right) }{2\pi } \pi= \frac{\text{length }\left(
\stackrel{\frown}{ PQ}
\right)}{2} = \frac{\Arcsin x}{2}\\
\text{Area}(B)&=&\displaystyle \text{Area}(\triangle OPQ)= \frac{x \sqrt{1-x^2} }{2}\\
\displaystyle \int_{0}^x\sqrt{1-t^2}\diff t&=&\displaystyle  \text{Area}(A)+\text{Area}(B)\\
&=&\displaystyle \frac{\Arcsin x}{2}+\frac{x\sqrt{1-x^2}}{2}\\
&\Rightarrow\\
\displaystyle\int\sqrt{1-x^2}\diff x&=&  \displaystyle \frac{\Arcsin x}{2} + \frac{x\sqrt{1-x^2}}{2}+C\quad .
\end{array}
\] 
}

\solution{\ref{problemintsqrt(1-x^2)/(1+x)dx} In this problem solution we use the standard Euler substitution $x=\cos (2\Arctan t)$. We recall \refBad{\ref{eqEulerSubx=cos(2arctant)}}{that}{from \eqref{eqEulerSubx=cos(2arctant)} that}

\noindent $\begin{array}{rcl}
\displaystyle x& =&\displaystyle \cos(2\Arctan t)= \frac{1-t^2}{1+t^2}\\
\Arccos (x)&=& 2 \Arctan t\\
\displaystyle \diff x &=&\displaystyle  -\frac{4t}{(1+t^2)^2}\diff t\\
\displaystyle \sqrt{1-x^2}&=&\displaystyle \sin (2\Arctan t)= \frac{2t}{1+t^2}\\
t&=&\displaystyle \frac{\sqrt{1-x^2}}{x+1}\quad .
\end{array}
$

\noindent $
\begin{array}{@{}r@{}c@{}l@{}l@{}|l}
\displaystyle \int  \frac{\sqrt{1-x^2}}{1+x}\diff x&=& \displaystyle \int t \left(-\frac{4t}{\left(1+t^2\right)^2} \right)\diff t&&\begin{array}{l} \text{Set }x=\frac{1-t^2}{1+t^2}\\
\text{Use f-las above}
\end{array}
\\
&=&\displaystyle -4\int\frac{t^2}{\left(1+t^2\right)^2}\diff t\\
&=&\displaystyle -4\int\frac{1+t^2-1}{\left(1+t^2\right)^2 }\diff t\\
&=&\displaystyle -4\int\left(\frac{1}{1+t^2} -\frac{1}{\left(1+t^2\right)^2}\right)\diff t\\
&=&\displaystyle -4\left(\Arctan t - \frac{1}{2}\left(\Arctan t+ \frac{t}{1+t^2} \right) \right)+C\\
&=&\displaystyle -2\left(\Arctan {}t - \frac{t}{1+t^2} \right) +C\\
&=&\displaystyle -2\left( \Arctan {}\left(\frac{\sqrt{1-x^2}}{1+x} \right) - \frac{1}{2}\sqrt{1-x^2} \right) +C\\
&=&\displaystyle -2 \Arctan {}t  +\sqrt{1-x^2} +C &&\text{Use f-las above}
\\
&=&\displaystyle -\Arccos x+ \sqrt{1-x^2}  +C\\
&=&\displaystyle \Arcsin x +\sqrt{1-x^2}+K\quad .
\end{array}
$

We have included the last equality to remind the student that derivatives of $\Arcsin(x)$ and $\Arccos x$ are negatives of one another.


}