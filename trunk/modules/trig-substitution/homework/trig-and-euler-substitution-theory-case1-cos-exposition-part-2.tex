We recall that the substitution $\theta = 2\arctan t$ transforms a trigonometric integral into an integral of a rational function. We now apply the substitution $2\arctan t$ after the substitution $x=\cos \theta$:

\begin{equation*}
\begin{array}{rcll|l}
x&=&\displaystyle \cos \theta &&\text{use } \theta=2\arctan t\\
&=&\displaystyle \cos (2\arctan t) &&\text{use }\refBad{\ref{eqSinCosViaTan}}{~}{\eqref{eqSinCosViaTan}:}  \displaystyle \cos (2z) = \frac{ 1-\tan^2 z }{1+ \tan^2 z}\\
&=&\displaystyle \frac{1-\tan^2(\arctan t)}{1+\tan^2( \arctan t)} \\
&=&\displaystyle \frac{1-t^2}{1+t^2} \quad .
\end{array}
\end{equation*}
We can furthermore compute
\begin{equation}\label{eqsqrt1minusxsquaredE2}
\begin{array}{rcll|l}
\sqrt{-x^2+1 }&=&\displaystyle \sqrt{1- \left(\frac{1-t^2}{1+t^2}\right)^2}\\
&=&\displaystyle \sqrt{\frac{(1+t^2)^2-(1-t^2)^2}{(1+t^2)^2} }\\
&=&\displaystyle \sqrt{\frac{4t^2}{(1+t^2)^2}} &&\displaystyle \sqrt{4t^2}=2t\text{ because } t\geq 0\\
&=&\displaystyle \frac{2t}{1+t^2}\quad .\\
\end{array}
\end{equation}
The differential $\diff x$ can be computed from $x=\frac{1-t^2}{1+t^2}$. Finally, we can express $t$ via $x$ with a little algebra:

\[
\begin{array}{rcll|l}
\displaystyle x&=&\displaystyle \frac{1-t^2}{1+t^2}\\
\displaystyle (1+t^2)x&=&\displaystyle 1-t^2\\
\displaystyle t^2(x+1)&=&\displaystyle 1-x\\
\displaystyle t^2&=&\displaystyle \frac{1-x}{1+x}\\
\displaystyle t&=&\displaystyle \sqrt{\frac{1-x}{1+x}}&& \text{here we use } t>0\\
\displaystyle t&=&\displaystyle \frac{\sqrt{1-x}}{\sqrt{ 1+x}} \frac{ \sqrt{1+x}}{\sqrt{1+x}} \\
\displaystyle t&=&\displaystyle \frac{\sqrt{-x^2+1}}{x+1}\quad .
\end{array}
\]
The Euler substitution $x= \cos (2\arctan t)$ can be now summarized as:
\[
\begin{array}{rcl}
x&=&\displaystyle \frac{1-t^2}{1+t^2}\\
\sqrt{-x^2+1}&=&\displaystyle \frac{2t}{1+t^2}  \\
\diff x&=&\displaystyle  -\frac{4 t}{(t^{2}+1)^{2}} \diff t\\
t&=&\displaystyle \frac{\sqrt{-x^2+1}}{x+1} \quad .
\end{array}
\]