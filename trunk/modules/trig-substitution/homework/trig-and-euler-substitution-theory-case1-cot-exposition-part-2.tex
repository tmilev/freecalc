We recall that the substitution $\theta=2\arctan t$ transforms a trigonometric integral into an integral of a rational function. We now apply the substitution $\theta=2\arctan t$ after the substitution $x=\cot\theta$:
\[
\begin{array}{rcll|l}
x&=&\displaystyle \cot \theta &&\text{use } \theta=2\arctan t\\
&=& \displaystyle \cot \left(2\arctan t\right) &&\displaystyle  \text{use\refBad{\ref{eqSinCosViaTan}}{ }{ \eqref{eqSinCosViaTan}: }} \cot 2z=\frac{\cos (2z)}{\sin (2z)}=\frac{1-\tan^2z}{2\tan z } \\
&=&\displaystyle \frac{1-\tan^2 (\arctan t)}{2 \tan (\arctan t)} \\
&=&\displaystyle \frac{1-t^2}{2t}\\
&=&\displaystyle \frac{1}{2}\left(\frac{1}t -t\right)\quad .
\end{array}
\]
We can furthermore compute
\begin{equation} \label{eqsqrtx2plus1Euler2}
\begin{array}{rcll|l}
\displaystyle \sqrt{x^2+1}&=& \displaystyle  \sqrt{ \frac{1}{4} \left(\frac{1}t -t \right)^2 +1}\\
&=&\displaystyle \frac{1}{2} \sqrt{\left( \frac{1}{t} +t \right)^2} & &\displaystyle \sqrt{\left(\frac{1}{t}+t\right)^2} = \frac{1}{t} +t \text{ because }t>0\\
&=&\displaystyle \frac{1}{2}\left(\frac{1}{t}+t\right)\quad .
\end{array}
\end{equation}
The differential $\diff x$ can via $\diff x $ as follows.
\[
\diff x=\diff \left(\frac{1}{2} \left( \frac{1}{t} - t\right)\right) = -\frac{1}{2} \left(\frac{1}{t^2}-1\right)\quad .
\]
Finally, we can subtract $\displaystyle x=\frac{1}{2} \left( \frac{1}{t} - t\right)$ from  $\displaystyle \sqrt{x^2+1}= \frac{1}{2} \left( \frac{1 }{ t} +t\right)$ to get that \[t=\sqrt{x^2+1}-x \quad .\]
The Euler substitution $x=\cot \theta= \cot (\arctan 2t)$ can be now summarized as:
\begin{equation}\label{eqEulerSub-case1-cot(2arctant)}
\begin{array}{rcl}
x&=&\displaystyle \frac12\left(\frac{1}{t}- t\right)\\
\displaystyle\sqrt{x^2+1}&=& \displaystyle \frac12 \left(\frac1t +t\right)\\
\displaystyle \diff x&=&\displaystyle -\frac12\left(\frac{1}{t^2}+1\right) \diff t\\
t &=&\sqrt{x^2+1}-x\quad .
\end{array}
\end{equation}