The trigonometric substitution $x=\sec \theta$ is given by
\[
\begin{array}{rcll|l}
\displaystyle \sqrt{x^2-1}&=&\displaystyle \sqrt{\sec^2\theta-1}=\sqrt{\frac{1}{\cos^2\theta}-1}\\
&=&\displaystyle \sqrt{\frac{\sin^2\theta}{\cos^2\theta}} =\sqrt{ \tan^2 \theta} &&\begin{array}{l} \text{when }\theta\in \theta \in \left[0, \frac{ \pi}{2 }\right)\cup \left[\pi, \frac{3\pi}{2}\right) \text{we have}\\
\tan \theta\geq 0 \text{ and so } \sqrt{\tan^2\theta}=\tan \theta
\end{array} \\
&=&\displaystyle \tan \theta\quad .
\end{array}
\]
The differential $\diff x$ can be expressed via $\diff \theta $ from $x=\sec \theta$. The substitution $x=\sec \theta $ can be now summarized as:
\begin{equation*}
\begin{array}{rcl}
\displaystyle x&=&\displaystyle \sec\theta= \frac{1}{\cos \theta}\\
\displaystyle \sqrt{x^2-1}& =&\displaystyle  \tan \theta\\
\displaystyle \diff x&=&\displaystyle \frac{\sin\theta}{ \cos^2\theta} \diff \theta= \sec\theta\tan\theta  \diff \theta\\
\displaystyle \theta&=&\Arcsec x \quad .
\end{array}
\end{equation*}