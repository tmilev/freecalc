%begin module quadratic-radicals-linear-substitution-preparation-ex1
\begin{frame}
\frametitle{Linear substitutions to simplify radicals $\sqrt{ay^2+by+c}$}
\begin{itemize}
\item Using linear substitutions, radicals of form  $\sqrt{ay^2+by+c}$, $a\neq 0$, $b^2-4ac\neq 0$ can be transformed to (multiple of):
\begin{itemize}
\item $\sqrt{x^2+1}$ 
\item $\sqrt{-x^2+1}$
\item $\sqrt{x^2-1}$.
\end{itemize}
\item We already studied how to do that using completing the square when dealing with rational functions. 
\end{itemize}
\end{frame}
\begin{frame}
Recall: linear substitution is subst. of the form $u=px+q$.
\begin{example}
Use linear substitution to transform $\sqrt{x^2+x+1}$ to multiple of $\sqrt{u^2+1}$. 

\noindent $
\begin{array}{rcl}
\sqrt{x^2+x+1}&=&\displaystyle \uncover<2->{ \sqrt{ x^2+2\frac{1}{2}x + \uncover<3->{ \alert<3>{ \frac{1}{4} } } \uncover<2>{ \alert<2>{ \textbf{?}}} \uncover<2->{ \alert<2,3>{-} } \uncover<3->{ \alert<3>{ \frac{1}{4}}} \uncover<2>{\alert<2>{\textbf{?}}} +1}} \\
\uncover<4->{&=&\displaystyle \sqrt{ {\left(x+\uncover<5->{\alert<5>{\frac{1}{2}}} \uncover<4>{ \alert<4>{ \textbf{?}}} \right)}^2- \uncover<4>{\alert<4>{\textbf{?} }} \uncover<5->{ \alert<5,6>{ \frac{4}{3}}} }} \\
\uncover<6->{&=&\displaystyle \sqrt{ \alert<6,7>{ \frac{3}{4}}\left( \alert<6,8>{\frac{4}{3}} \left(x+\frac{1}{2}\right)^{\alert<8>{2}} +\alert<6>{ 1} \right)}}\\
\uncover<7->{&=&\displaystyle \alert<7>{\frac{\sqrt{3}}{2}} \sqrt{\left(  \alert<10>{\alert<8>{\frac{2}{\sqrt{3}}} \left( x+ \frac{1}{2} \right)}\right)^{\alert<8>{2}}+1}}\\
\uncover<9->{ &=&\displaystyle \frac{\sqrt{3}}{2} \sqrt{ {\alert<10>{u}}^2+1},}
\end{array}
$

\noindent \uncover<9->{ where $\displaystyle \alert<9,10>{u=}\uncover<10->{ \alert<10>{ \frac{2}{\sqrt{3}}\left( x+\frac{1}{2}\right)}} \uncover<9>{ \alert<9>{ \textbf{?}}} \uncover<10->{ =\frac{2\sqrt{3}}{3}x +\frac{\sqrt{3}}{3}} $\uncover<10>{.}}
\end{example}
\vspace{5cm}
\end{frame}
%end module quadratic-radicals-linear-substitution-preparation-ex1