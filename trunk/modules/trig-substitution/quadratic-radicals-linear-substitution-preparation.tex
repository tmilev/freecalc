%begin module Euler substitution
\begin{frame}
\frametitle{Euler substitution}
\begin{itemize}
\item Using linear substitutions, radicals of form  $\sqrt{ay^2+by+c})$, $a\neq 0$, $b^2-4ac\neq 0$ can be transformed to (multiple of):
\begin{itemize}
\item $\sqrt{x^2+1}$ 
\item $\sqrt{-x^2+1}$
\item $\sqrt{x^2-1}$.
\end{itemize}
\item We already studied how to do that using completing the square. 
\end{itemize}
\end{frame}
\begin{frame}
Recall that a (real) linear substitution is a substitution of the form $u=px+q$, $p,q$- (real) constants.
\begin{example}
Use a linear substitution to transform $\sqrt{x^2+x+1}$ to a multiple of an expression of the form $\sqrt{u^2+1}$. 

\[
\begin{array}{rcl}
\sqrt{x^2+x+1}&=&\sqrt{ x^2+2\frac{1}{2}x +\frac{1}{4}\textbf{?}-\frac{1}{4}\textbf{?} +1} \\
&=& \sqrt{ \left(x+\frac{1}{2}\textbf{?} \right)^2-\textbf{?} }\\
&=&\sqrt{\frac{3}{4}\left( \frac{4}{3} \left(x+\frac{1}{2}\right)^2+1 \right)}\\
&=&\frac{\sqrt{3}}{2}\sqrt{\left(\frac{2}{\sqrt{3}}\left( x+\frac{1}{2}\right)\right)^2+1}\\
&=& \frac{\sqrt{3}}{2} \sqrt{u^2+1},
\end{array}
\]
where $u=\frac{2}{\sqrt{3}}\left( x+\frac{1}{2}\right) \textbf{?}=\frac{2\sqrt{3}}{3}x+\frac{\sqrt{3}}{3}$.
\end{example}
\vspace{5cm}
\end{frame}
\begin{frame}
Recall that a (real) linear substitution is a substitution of the form $u=px+q$, $p,q$- (real) constants.
\begin{example}
Use a linear substitution to transform $\sqrt{-2x^2+x+1}$ to a multiple of an expression of the form $\sqrt{-u^2+1}$. 

\[
\begin{array}{rcl}
\sqrt{-2x^2+x+1}&=&\sqrt{ -2\left(x^2-\frac{1}{2}x -\frac{1}{2}\right) } \\
&=& \sqrt{ -2\left(x^2-2\frac{1}{4}x +\frac{1}{16}-\frac{1}{16}-\frac{1}{2}\right) }\\
&=&\sqrt{-2\left(\left(x-\frac{1}{16}\right)^2-\frac{9}{16} \right)}\\
&=&\sqrt{\frac{9}{8}\left(-\frac{16}{9}\left(x-\frac{1}{16}\right)^2+1 \right)}\\
&=&\frac{3}{\sqrt{8}}\sqrt{-\left(\frac{4}{3}\left(x-\frac{1}{16}\right)\right)^2+1 }\\
&=&\frac{ 3}{\sqrt{8}} \sqrt{-u^2+1}
\end{array}
\]
where $u=\frac{4}{3}\left(x-\frac{1}{16}\right)  \textbf{?}=\frac{4}{3}x-\frac{1}{12}$.
\end{example}
\end{frame}
%end module Euler substitution.