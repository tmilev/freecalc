%begin module trig-substitutions-euler-substitutions-table

\begin{frame}
\begin{itemize}
\item Let $R$ be a rational function in two variables.
\item<2-> So far, with linear transformations we converted all integrals of the form $\displaystyle\int R(x, \sqrt{ax^2+bx+c})\diff x$ to one of the three forms:

\alert<4,9>{$\int R(x, \sqrt{x^2+1})\diff x$}, \alert<5,10>{$\int R(x, \sqrt{-x^2+1})\diff x$} , \alert<6,11>{$\int R(x, \sqrt{x^2-1}) \diff x$}.
\item<3-> Each of the above integrals can be transformed to a rational trigonometric integral using 3 pairs of substitutions:

\alert<4,9>{$x=\tan\theta $, $x=\cot \theta$;}  
\alert<5,10>{$x=\sin\theta $, $x=\cos \theta$;}
\alert<6,11>{$x=\csc\theta $, $x=\sec \theta$.}
\item<7-> We studied that trigonometric integrals are converted to rational function integrals via $\theta=2\arctan t$.
\item<8-> The resulting 3 pairs of substitutions are called Euler substitutions:
\alert<9,13>{$x=\tan (2\arctan t) $, $x=\cot (2\arctan t)$;}  
\alert<10,13>{$x=\sin(2\arctan t) $, $x=\cos (2\arctan t)$;}
\alert<11,13>{$x=\csc(2\arctan t) $, $x=\sec (2\arctan t)$.}
\item<12-> The Euler substitutions directly transform the integral to a rational function integral.
\item<13-> We will demonstrate that the Euler substitutions are \alert<13>{rational}.
\end{itemize}

\end{frame}

\begin{frame}
\frametitle{Trigonometric substitution and Euler substitution}
{\tabcolsep=0.11cm
\noindent\begin{tabular}{|l|l|l|r|}
\hline
Expression & Substitution& Variable range & Relevant identity\\\hline
\multirow{2}{*}{$\sqrt{x^2+1}$} & $x = \tan \theta$ &  $ \theta\in \left(-\frac{\pi}{2} , \frac{\pi}{2}\right)$ & $1 + \tan^2 \theta = \sec^2 \theta$\\
&$x=\cot \theta$ &$ \theta\in (0, \pi) $ & $1+\cot^2\theta =\csc^2\theta $ \\ \hline 
\multirow{2}{*}{ $\sqrt{-x^2+1 }$} & $x = \sin \theta$ &  $ \theta\in \left[ -\frac{\pi}{2} ,\frac{\pi}{2}\right]$ & $1 - \sin^2 \theta = \cos^2 \theta$\\
& $x = \cos \theta$ & $\theta\in (0,\pi)$& $1-\cos^2\theta=\cos^2\theta$ \\\hline 
\multirow{2}{*}{$\sqrt{x^2-1}$} &$x=\csc \theta$ &$\theta\in \left[0, \frac{\pi}{2} \right) \cup \left[ \pi, \frac{3\pi}{2}\right)$ &  $\csc^2\theta-1=\cot^2\theta $ \\
&$x = \sec \theta$ & 
$\theta\in \left[0, \frac{\pi}{2}\right)\cup \left[\pi, \frac{ 3 \pi}{2}\right)$
& $\sec^2\theta - 1 = \tan^2\theta$
\\
\hline
\multicolumn{4}{c}{Euler substitution by applying in addition $\theta=2\arctan t$}\\
\hline
\multirow{2}{*}{$\sqrt{x^2+1}$} & $ x =\frac{2t}{1-t^2}$ & $-1< t< 1$ & (?) \\
&$ x=\frac{1}{2} \left(\frac{1}{t}-t\right)$ & $0<t $ &  (?)\\ \hline 
\multirow{2}{*}{ $\sqrt{-x^2+1 }$} & $x=\frac{2t}{1+t^2} $ & $-1\leq t\leq 1 $ & (?)\\
& $x =\frac{1-t^2}{1+t^2} $ & $0<t$&  (?)\\\hline 
\multirow{2}{*}{ $\sqrt{x^2-1}$} & $x=\frac{1}{2}\left(\frac{ 1}{t}+t\right)$ & $t\in (-\infty, -1)\cup [0,1)$&(?)\\
& $x =\frac{1+t^2}{1-t^2} $ & $t \in (-\infty,-1)\cup [0,1)$ & (?)\\\hline
\end{tabular}
}
\end{frame}
%end module trig-substitutions-euler-substitutions-table