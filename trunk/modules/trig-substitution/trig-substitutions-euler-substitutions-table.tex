%begin module trig-substitutions-euler-substitutions-table
\begin{frame}
\begin{itemize}
\item Let $R$ be a rational function in two variables.
\item<2-> So far, with linear transformations we converted all integrals of the form $\displaystyle\int R(x, \sqrt{ax^2+bx+c})\diff x$ to one of the three forms:

\alert<4,9>{$\int R(x, \sqrt{x^2+1})\diff x$}, \alert<5,10>{$\int R(x, \sqrt{-x^2+1})\diff x$} , \alert<6,11>{$\int R(x, \sqrt{x^2-1}) \diff x$}.
\item<3-> Each of the above integrals can be transformed to a rational trigonometric integral using 3 pairs of substitutions:

\alert<4,9>{$x=\tan\theta $, $x=\cot \theta$;}  
\alert<5,10>{$x=\sin\theta $, $x=\cos \theta$;}
\alert<6,11>{$x=\csc\theta $, $x=\sec \theta$.}
\item<7-> We studied that trigonometric integrals are converted to rational function integrals via $\theta=2\arctan t$.
\item<8-> The resulting 3 pairs of substitutions are called Euler substitutions:
\alert<9>{$x=\tan (2\arctan t) $, $x=\cot (2\arctan t)$;}  
\alert<10>{$x=\sin(2\arctan t) $, $x=\cos (2\arctan t)$;}
\alert<11>{$x=\csc(2\arctan t) $, $x=\sec (2\arctan t)$.}
\item<12-> The Euler substitutions directly transform the integral to a rational function integral.
\item<13-> The Euler substitutions are rational.
\end{itemize}

\end{frame}

\begin{frame}
\frametitle{Trigonometric substitution and Euler substitution}

\end{frame}
%end module trig-substitutions-euler-substitutions-table