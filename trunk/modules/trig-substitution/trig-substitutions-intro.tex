% begin module trig-substitutions-intro
\begin{frame}
\frametitle{Trigonometric Substitution}
\begin{itemize}
\item  To find the area of a circle or ellipse, one needs to compute $\int \sqrt{a^2 - x^2} \diff x$.
\item<2->  For $\int x\sqrt{a^2 - x^2}\diff x$, the substitution $u = a^2 - x^2$ would work.
\item<3->  For $\int \sqrt{a^2 - x^2}\diff x$, we need a more elaborate substitution.
\item<4-| alert@6>  Instead, substitute $x = a\sin \theta$.
\end{itemize}
\[
\uncover<5->{%
\sqrt{a^2-\alert<handout:0| 6>{x^2}} = %
}%
\uncover<6->{%
\sqrt{a^2-\alert<handout:0| 6>{a^2\sin^2 \theta}} = %
}%
\uncover<7->{%
\sqrt{a^2(1 - \sin^2 \theta )} = %
}%
\uncover<8->{%
\sqrt{a^2\cos^2 \theta} = a|\cos \theta |.%
}%
\]
\begin{itemize}
\item<9->  With $u = a^2 - x^2$, the new variable is a function of the old one.
\item<10->  With $x = a\sin \theta$, the old variable is a function of the new one.
%Greg: the below remarks seem redundant to me. Students know how to compute \diff x, I'd think they'd be more confused than englightened by ``substitutions in reverse''.
%\item<11->  To make a substitution of the form $x = g(t)$, use the substitution rule in reverse.
%\item<12->  We call this inverse substitution.
\end{itemize}
\end{frame}
% end module trig-substitutions-intro
