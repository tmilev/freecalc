\begin{frame}
\frametitle{Angles and the coordinate system}
\begin{columns}
\column{0.2\textwidth}
\begin{pspicture}(-1.2,-1.2)(1.3, 1.2)%
\tiny%
\fcBoundingBox{-1.2}{-1.3}{1.3}{1.2}%
\pstVerb{20 dict begin}%
\pstVerb{/startAngle 0 def /endAngle 60 def}%
\only<handout:0|8>{\pstVerb{/endAngle 240 def}}%
\only<handout:0|9,14>{\pstVerb{/endAngle 420 def}}%
\only<handout:0|10>{\pstVerb{/endAngle -60 def}}%
\only<handout:0|11>{\pstVerb{/endAngle -240 def}}%
\fcAxesStandardNoFrame{-1.2}{-1.2}{1.3}{1.3}%
\fcLabels{1.3}{1.3}%
\uncover<5->{\parametricplot[linecolor=\fcColorGraph, arrows=->]{startAngle}{endAngle }{t cos t sin}}%
\uncover<6->{\rput[b](! startAngle endAngle add 2 div dup cos 1.1 mul exch sin 1.2 mul){$\alpha$}}%
\uncover<7->{\psline[arrows=->](0,0)(! endAngle cos 1.5 mul endAngle sin 1.5 mul)}%
\uncover<8->{\parametricplot[linecolor=blue, arrows=->]{startAngle}{endAngle}{t cos 0.3 t 2000 div add mul  t sin 0.3 t 2000 div add mul}}%
\uncover<3->{\psline[arrows=->](0,0)(1.5,0)}%
\uncover<2->{\fcFullDot{0}{0}}%
\uncover<5->{\fcFullDot{endAngle cos}{endAngle sin}}%
\uncover<4->{%
\fcFullDot{1}{0}%
\rput[tr](1,-0.1){$(1,0)$}%
}%
%(! endAngle cos 1.2 mul endAngle sin 1.2 mul)%
\pstVerb{end}%
\end{pspicture}
\column{0.8\textwidth}
\begin{itemize}
\item<1-> Given an angle measure $\alpha$ between $(-\pi, \pi]$, there is a conventional way to select a geometric angle with that measure.
\begin{itemize}
\item<2-> Select geometric angle's vertex to be the origin.
\item<3-> Select the initial arm of the angle on the $x$-axis, pointing in the positive direction.
\end{itemize}
\end{itemize}
\end{columns}

\begin{itemize}
\item[]
\begin{itemize}
\item<4-> \alertNoH{7}{Select the terminal arm} by rotating \alertNoH{4}{the point $(1,0)$} on the initial arm by $|\alpha|$ radians: go clockwise if $\alpha<0$, counter-clockwise if $\alpha>0$.
\item<5-> To rotate the point, \alertNoH{5}{move it along the circle} with radius $1$ \alertNoH{6}{for $\alpha$ units} of arc-length.
\end{itemize}
\item<8-> The construction also works for angle measures greater than \alertNoH{8,9}{$\pi \text{ rad}$}/\alertNoH{10}{smaller than $-\pi\text{ rad}$}.
\item<12-> In this way to every real $\alpha$ we can assign a geometric angle.
\item<13-> If $\alpha$ is in the interval $(-\pi, \pi]$ the so obtained geometric angle does have measure $\alpha$, else the \alertNoH{14}{measure of the geometric angle differs} from $\alpha$ \alertNoH{14}{by an even multiple of $\pi$}.
\end{itemize}
\end{frame}