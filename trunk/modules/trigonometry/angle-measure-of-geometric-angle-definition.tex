\begin{frame}
%\frametitle{Angle measure}
\vskip -0.2cm
\begin{columns}
\column{0.25\textwidth}
\begin{pspicture}(-1,-1)(6, 4)
\tiny
\pstVerb{20 dict begin /endAngle 60 def}
\psline[arrows=->](0,0)(2,0)
\psline[arrows=->](0,0)(! endAngle cos 2 mul endAngle sin 2 mul )
\parametricplot[arrows=->, linecolor=blue]{0}{endAngle}{t cos 0.2 mul t sin 0.2 mul}
\parametricplot[linestyle=dashed, linecolor=red!60]{endAngle}{360}{t cos t sin}
\parametricplot[linecolor=red]{0}{endAngle}{t cos t sin}
\fcFullDot[linecolor=blue]{0}{0}%
\rput[t] (-0.2, -0.2){$O$}%
\fcFullDot{1}{0}%
\rput[t] (1, -0.2){$A$}%
\fcFullDot{0.5 }{3 sqrt 0.5 mul}%
\rput[rt] (0.6, 1.2){$B$}%
\pstVerb{end}
\end{pspicture}

\column{0.75\textwidth}
\begin{definition}[Radian measure of geometric angle]
The measure of a geometric angle is a number determined as follows.
\begin{itemize}
\item<2-> The magnitude the measure of a geometric angle is the length of the short arc of a unit circle cut off from a circle of radius $1$ by the angle.  
\item<3-> If when traversing the arc from the initial arm to the terminal we move clockwise, the measure is taken with negative sign, else with positive.
\end{itemize}
\end{definition}
\begin{itemize}
%\only<handout:1|1-5>{
\item<4-> The unit of this angle measure is called radians. 
\item<5-> A circle of radius $1$ has circumference $2\pi $.
\item<6-> By convention the half-turn angle is measured with $\pi$ (rather than $-\pi$).
\item<7-> Therefore a geometric angle is measured with a number between $(-\pi, \pi]$.
%}
%\only<handout:2|6->{
%}
\end{itemize}
\end{columns}
\vskip 10cm
\end{frame}