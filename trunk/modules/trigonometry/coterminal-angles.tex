\begin{frame}
\vskip -0.1cm
\begin{definition}[Coterminal Angles]
Two angles (angle measures) are called coterminal if the corresponding geometric angles have the same initial and terminal sides. 
\end{definition}
\begin{center}
\psset{xunit=2cm, yunit=2cm}
\begin{pspicture}(-0.9, -1.1)(2,0.5)
\tiny
%force a boudning box:
%\psline[linecolor=red!1](-0.1, -0.1)(-0.21,0.2)
%\psline[linecolor=red!1](1.1, 0.6)(1.1,0.61)
\fcFullDotBlue{0}{0}
%Calculator input: plotCurve{}(1/10 \cos{}t, 1/10 \sin{}t, 0, -3/4 \pi)
\parametricplot[arrows=->, linecolor=\fcColorGraph, plotpoints=100]{0}{225} {t cos 0.3 mul t sin 0.3 mul}
\parametricplot[arrows=->, linecolor=\fcColorGraph, plotpoints=100]{0}{-135} {t cos 0.2 mul t sin 0.2 mul}
\parametricplot[arrows=->, linecolor=\fcColorGraph, plotpoints=400] {0} {585} {t cos 0.4 t 2000 div add mul t sin 0.4 t 2000 div add mul}
\rput[r](0, 0.7){$\theta=\frac{13\pi}{4}$}
\rput[t] (0,-0.1){$O$}
\rput[l](-0.2, 0.15){$\theta=\frac{5\pi}{4}$}
\rput[lt](0, -0.2){$\theta=-\frac{3\pi}{4}$}
\psline{->}(0,0)(1,0)
\psline[arrows=->, linecolor=blue](0,0)(-0.707106781, -0.707106781)
\end{pspicture}
\end{center}
\vskip -0.1cm
\begin{observation}
The set of angles coterminal with $\alpha$ consists of the angles $\alpha+2k\pi$, where $k$ runs over the set of integers. In other words, the angles coterminal with $\alpha$ are the angles:
\[
\dots, \alpha-6\pi, \alpha-4\pi,\alpha-2\pi,\alpha, \alpha+2\pi, \alpha+4\pi, \alpha+6\pi, \dots.
\]
\end{observation}

\end{frame}