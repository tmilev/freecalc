\begin{frame}
\begin{center}
\psset{xunit=2cm, yunit=2cm}
\begin{pspicture}(-1.9,-1.4)(2.2,1.4)
\tiny
\fcAxesStandard{-1.4}{-1.4}{1.6}{1.4}
\fcLabels{1.6}{1.3}
\parametricplot{0}{360}{t cos t sin}
%\newcommand{\rayFormat}{}
\newcommand{\theLine}[1]{%
\psline[linecolor=blue](! ####1 cos 0.9 mul ####1 sin 0.9 mul)(! ####1 cos 1.1 mul ####1 sin 1.1 mul)%
\psline[linestyle=dashed, linewidth=0.5pt, linecolor=blue](0,0)(! ####1 cos 1 mul ####1 sin 1 mul)%
}%
\theLine{30}%
\theLine{45}%
\theLine{60}%
\theLine{90}%
\theLine{120}%
\theLine{135}%
\theLine{150}%
\theLine{180}%
\theLine{210}%
\theLine{225}%
\theLine{240}%
\theLine{270}%
\theLine{300}%
\theLine{315}%
\theLine{330}%
\theLine{360}%
\rput[l](! 30 cos 30 sin){$~~~~\fcAnswer{4}{\frac{\pi}{6}}\uncover<3->{ =} 30^\circ$}%
\rput[l](! 45 cos 45 sin){$~~~~\fcAnswer{6}{\frac{\pi}{4}} \uncover<5->{=} 45^\circ$}%
\rput[bl](! 60 cos 60 sin 0.1 add){$\fcAnswer{8}{\frac{\pi}{3}}\uncover<7->{=} 60^\circ$}%
\rput[bl](0,1.02){$\fcAnswer{10}{\frac{\pi}{2}} \uncover<9->{=} 90^\circ$}%
\rput[br](! 120 cos 120 sin 0.1 add){$\fcAnswer{12}{\frac{2\pi}{3}} \uncover<11->{=} 120^\circ$}%
\rput[br](! 135 cos 135 sin 0.1 add){$\fcAnswer{14}{\frac{3\pi}{4}} \uncover<13->{=} 135^\circ$}%
\rput[br](! 150 cos 150 sin 0.1 add){$\fcAnswer{16}{\frac{5\pi}{6}} \uncover<15->{=} 150^\circ$}%
\rput[br](! -1 0.05){$\fcAnswer{18}{\pi} \uncover<17->{=} 180^\circ$}%
\rput[tr](! 210 cos 210 sin 0.1 sub){$\uncover<20->{\frac{7\pi}{6}} \uncover<20->{=} 210^\circ$}%
\rput[tr](! 225 cos 225 sin 0.1 sub){$\uncover<20->{\frac{5\pi}{4}} \uncover<20->{=} 225^\circ$}%
\rput[tr](! 240 cos 240 sin 0.1 sub){$\uncover<20->{\frac{4\pi}{3}} \uncover<20->{=} 240^\circ$}%
\rput[tr](! 0 -1.1){$\fcAnswer{20}{\frac{3\pi}{2}} \uncover<19->{=} 270^\circ$}%
\rput[tl](! 300 cos 300 sin 0.1 sub){$\uncover<22->{\frac{5\pi}{3}} \uncover<22->{=} 300^\circ$}%
\rput[tl](! 315 cos 315 sin 0.1 sub){$\uncover<22->{\frac{7\pi}{4}} \uncover<22->{=} 315^\circ$}%
\rput[tl](! 330 cos 330 sin 0.1 sub){$\uncover<22->{\frac{11\pi}{6}} \uncover<22->{=} 330^\circ$}%
\rput[tl](! 1 -0.1){$\fcAnswer{22}{2\pi} \uncover<21->{=} 360^\circ$}%
\rput[lb](1.2, 1.2){Quadrant I}
\rput[rb](-1.2,1.2){Quadrant II}
\rput[rt](-1.2,-1.2){Quadrant III}
\rput[lt](1.2,-1.2){Quadrant IV}
\end{pspicture}
\end{center}


The most frequently encountered angles are given in the table below.
\[
\begin{array}{|c@{ \ }|c@{ \ }|c@{ \ }|c@{ \ }|c@{ \ }|c@{ \ }|c@{ \ }|c@{ \ }|c@{ \ }|c@{ \ }|c@{ \ }|c@{ \ }|}
\hline
\text{Deg.} &
\alertNoH{1-2}{0^\circ } &
\alertNoH{3-4}{30^\circ} &
\alertNoH{5-6}{45^\circ} &
\alertNoH{7-8}{60^\circ} &
\alertNoH{9-10}{90^\circ} &
\alertNoH{11-12}{120^\circ}&
\alertNoH{13-14}{135^\circ}&
\alertNoH{15-16}{150^\circ}&
\alertNoH{17-18}{180^\circ}&
\alertNoH{19-20}{270^\circ}&
\alertNoH{21-22}{360^\circ}\\
\hline
\textrm{Rad.} &
\fcAnswer{2}{ 0} &
\displaystyle \fcAnswer{4}{\frac{ \pi}{6}} &
\displaystyle \fcAnswer{6}{\frac{ \pi}{4}} &
\displaystyle \fcAnswer{8}{\frac{ \pi}{3}} &
\displaystyle \fcAnswer{10}{\frac{ \pi}{2}} &
\displaystyle \fcAnswer{12}{\frac{2\pi}{3}} &
\displaystyle \fcAnswer{14}{\frac{3\pi}{4}} &
\displaystyle \fcAnswer{16}{\frac{5\pi}{6}} &
\displaystyle \fcAnswer{18}{\pi} &
\displaystyle \fcAnswer{20}{\frac{3\pi}{2}} &
\displaystyle \fcAnswer{22}{2\pi} \\
\hline
\end{array}
\]


\end{frame}

\begin{frame}
\begin{center}
\psset{xunit=2cm, yunit=2cm}
\begin{pspicture}(-1.9,-1.4)(1.9,1.4)
\tiny
\fcBoundingBox{-1.8}{-1.3}{1.8}{1.3}
\fcAxesStandard{-1.7}{-1.3}{1.7}{1.3}
\fcLabels{1.6}{1.3}
\parametricplot{0}{360}{t cos t sin}
%\newcommand{\rayFormat}{}
\newcommand{\theLine}[1]{%
\psline[linecolor=blue](! ####1 cos 0.9 mul ####1 sin 0.9 mul)(! ####1 cos 1.1 mul ####1 sin 1.1 mul)%
\psline[linestyle=dashed, linewidth=0.5pt, linecolor=blue](0,0)(! ####1 cos 1 mul ####1 sin 1 mul)%
}%
\uncover<1->{\theLine{57.295779513 1 mul}}%
\uncover<2->{\theLine{57.295779513 2 mul}}%
\uncover<3->{\theLine{57.295779513 3 mul}}%
\uncover<4->{\theLine{57.295779513 4 mul}}%
\uncover<5->{\theLine{57.295779513 5 mul}}%
\uncover<6->{\theLine{57.295779513 6 mul}}%
\uncover<7->{\theLine{57.295779513 7 mul}}%
\uncover<8->{\theLine{57.295779513 8 mul}}%
\uncover<1->{\rput[l](! 57.295779513 1 mul dup cos exch sin 0.1 add){$1 \text{ rad} \approx 57.3^\circ$}}%
\uncover<2->{\rput[r](! 57.295779513 2 mul dup cos exch sin 0.1 add){$2 \text{ rad} \approx 114.6^\circ$}}%
\uncover<3->{\rput[r](! 57.295779513 3 mul dup cos exch sin 0.1 add){$3 \text{ rad} \approx 171.9^\circ$}}%
\uncover<4->{\rput[t](! 57.295779513 4 mul dup cos exch sin -0.1 add){$4 \text{ rad} \approx 229.2^\circ$}}%
\uncover<5->{\rput[t](! 57.295779513 5 mul dup cos exch sin -0.1 add){$5 \text{ rad} \approx 286.5^\circ$}}%
\uncover<6->{\rput[tl](! 57.295779513 6 mul dup cos exch sin){$6 \text{ rad} \approx 343.8^\circ$}}%
\uncover<7->{\rput[l](! 57.295779513 7 mul dup cos exch sin){$7 \text{ rad} \approx 401.1^\circ$}}%
\uncover<8->{\rput[b](! 57.295779513 8 mul dup cos exch sin 0.1 add){$8 \text{ rad} \approx 458.4^\circ$}}%
\end{pspicture}
\end{center}
\begin{itemize}
\item Integer quantities of radians are not rational multiples of (the measure of) a half-turn and are not easy to compute with. 
\item<9-> For example to determine in which quadrant is an angle of $k$ radians located one needs to know the numerical value of $\frac{k}{\pi}$, which requires knowledge of $\pi$ with great numerical accuracy.
\end{itemize}
\end{frame}