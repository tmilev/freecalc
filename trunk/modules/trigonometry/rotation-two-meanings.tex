\begin{frame}
\begin{columns}
\column{0.25\textwidth}
\begin{pspicture}(-1,-1)(6, 4)%
\tiny%
\fcBoundingBox{-1.4}{-0.5}{1.4}{1.4}%
\pstVerb{20 dict begin 
/endAngle 60 def
/startAngle 0 def
}%
\only<2>{\pstVerb{/endAngle 20 def}}%
\only<3>{\pstVerb{/endAngle 40 def}}%
\only<4>{\pstVerb{/endAngle 60 def}}%
\only<5>{\pstVerb{/endAngle 80 def}}%
\only<6>{\pstVerb{/endAngle 100 def}}%
\only<7->{\pstVerb{/endAngle 120 def}}%
\fcFullDot[linecolor=blue]{0}{0}%
\rput[t] (-0.2, -0.2){$O$}%
\fcFullDot{startAngle cos}{startAngle sin}%
\rput[t] (1, -0.2){$A$}%
\uncover<2-7,9->{%
\parametricplot[arrows=->, linecolor=red]{startAngle}{endAngle}{t cos t sin}%
\fcFullDot{endAngle cos}{ endAngle sin}%
\rput[b] (! endAngle cos endAngle sin 0.1 add){$B$}%
}%
\pstVerb{end}%
\end{pspicture}

\column{0.75\textwidth}
\begin{itemize}
\item The term rotation refers to two distinct objects:
\begin{itemize}
\item<2-> \emph{continuous rotation} (\emph{rotation} for short) - a gradual with respect to time transformation of space and
\item<8-> \emph{rotation} - an instantaneous transformation of space. \uncover<9->{ All points transition from their initial to their final positions instantaneously.}
\end{itemize}
\item<10-> In mathematics, the term rotation usually refers to ``instantaneous'' rotation.
\item<11-> In physics, the term rotation usually refers to continuous rotation (time is explicitly parametrized). 
\item<12-> Whether the term rotation refers to continuous rotation or to ``instantaneous'' rotation should be inferred from context.
\end{itemize}

\end{columns}
\vskip 10cm

\end{frame}