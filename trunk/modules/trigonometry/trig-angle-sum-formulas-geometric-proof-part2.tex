\begin{frame}
\frametitle{Trig Functions of Sums and Differences of Angles}
\begin{theorem}
$\begin{array}{rcl}
\alertNoH{1}{\sin (\alpha+ \beta)}&\alertNoH{1}{= }&\alertNoH{1}{ \sin\alpha\cos\beta +\cos \alpha\sin \beta }\\
\uncover<6->{ \alertNoH{1}{\sin (\alpha- \beta)}&\alertNoH{1}{= }&\alertNoH{1}{ \sin\alpha\cos\beta -\cos \alpha\sin \beta }}\\
\alertNoH{1}{\cos(\alpha+\beta)}&\alertNoH{1}{=}&\alertNoH{1}{\cos\alpha\cos \beta -\sin\alpha\sin\beta}\\
\uncover<6->{\alertNoH{1}{\cos(\alpha-\beta)}&\alertNoH{1}{=}&\alertNoH{1}{\cos\alpha\cos \beta +\sin\alpha\sin\beta}}
\end{array}
$
\end{theorem}
\begin{itemize}
\item<2-> We gave a geometric proof of the sum formulas when the two angles are acute and their sum is less than $\pi=90^\circ$. 
\item<3-> The theorem holds for all angles $\alpha,\beta$ without any restrictions.
\item<4-> This can be shown by combining the preceding proof with identities such as $\cos \left(\frac{\pi}{2}-\alpha\right)=\sin \alpha$, $\cos\left(\frac{\pi}{2}+\alpha\right)=-\sin\alpha$.
\item<5-> There is a theoretically more advanced (but algebraically simpler) proof using Euler's formula (to be studied later/in another course).
\item<6-> The difference formulas are a consequence of the sum formulas and the fact that $\sin$ is an odd function and $\cos $ is even.
\end{itemize}


\vskip 10cm
\end{frame}
