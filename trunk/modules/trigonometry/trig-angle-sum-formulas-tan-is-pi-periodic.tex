\begin{frame}

\begin{example}
Show that $\tan (\pi+x)=\tan x$ using the angle sum formulas. 

\uncover<2->{
\[
\begin{array}{rcl}
\alertNoH{16}{\tan (\pi+x)}&\alertNoH{0}{=}&\displaystyle \frac{\alertNoH{3}{\sin (\pi+x)}}{\alertNoH{4}{\cos (\pi+x)}}\\
\uncover<3->{&\alertNoH{0}{=}&\displaystyle \frac{\alertNoH{3}{ \alertNoH{5,6}{\sin \pi} \cos x +\alertNoH{7,8}{\cos \pi} \sin x}}{\alertNoH{4}{\alertNoH{9,10}{\cos \pi} \cos x- \alertNoH{11,12}{\sin \pi} \sin x}}}\\
\uncover<5->{&\alertNoH{0}{=}& \displaystyle \frac{\fcAnswerUncover{5}{6}{0}\cdot\cos x +\fcAnswerUncover{5}{8}{(-1)}\cdot\sin x}{\fcAnswerUncover{5}{10}{(-1)} \cdot \cos x- \fcAnswerUncover{5}{12}{0} \cdot \sin x} }\\
\uncover<13->{&\alertNoH{0}{=}&\displaystyle \frac{\alertNoH{14}{-}\sin x}{\alertNoH{14}{-}\cos x}}\\
\uncover<14->{&\alertNoH{0}{=}&\displaystyle \alertNoH{15}{\frac{\sin x}{\cos x}}}\\
\uncover<15->{&\alertNoH{15,16}{=}&\displaystyle \alertNoH{15,16}{\tan x}},\\
\end{array}
\]
\uncover<16->{as desired.}
}


\end{example}
\end{frame}

\begin{frame}
\begin{proposition}[$\tan, \cot$ are $\pi$-periodic]
The tangent and cotangent functions are $\pi$-periodic, in other words, 
\[
\begin{array}{rcl}
\tan (\theta+\pi)&=&\tan\theta \\
\cot (\theta+\pi)&=&\cot\theta \\
\end{array}
\]
\end{proposition}
\end{frame}