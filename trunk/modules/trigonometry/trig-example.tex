% begin module trig-example
\begin{frame}
\begin{example}
\begin{columns}[c]
\column{.5\textwidth}

\psset{xunit=1.8cm,yunit=1.8cm}
\begin{pspicture}(-2.3,-0.5)(0.5,2.3)
\tiny%
\fcAxesStandard{-2.3}{-0.5}{0.5}{2.2}%
\psline[linecolor=blue](0,0)(-1,1.732)%
\psline[linecolor=blue](0,0)(0.5,0)%
\uncover<3->{%
\fcPerpendicular[linestyle=dotted]{[-1 3 sqrt]}{[-1 0]}{0.1}%
\fcPerpendicular[linestyle=dotted]{[-1 3 sqrt]}{[0 1]}{0.1}%
}%
\fcFullDot{-1}{3 sqrt}%
\rput[l](0.15, 0.35){$\frac{2\pi}{3}$}%
\psarc[linecolor=red](0,0){0.5}{0}{120}%
\uncover<2->{%
\rput(-0.25, 0.15){$\frac{\pi}{3}$}%
\psarc[linecolor=red](0,0){0.3}{120}{180}%
}%
\uncover<3->{%
\rput[br](-1,1.732){$(-1,\sqrt{3})$}%
\alertNoH{5,7,11,13}{\rput[lb](-0.45, 0.85){$2$}}%
\alertNoH{5,9,11,15}{\rput[r](-1.1, 0.85){$\sqrt{3}$}}%
\rput[t](-1, -0.05){$(\alertNoH{7,9,13,15}{-1},0)$}%
}%
\end{pspicture}
\column{.5\textwidth}
Find the exact trigonometric ratios for $\displaystyle \theta = \frac{2\pi }{3} =120^\circ$.
\end{columns}
\[
\begin{array}{@{}r@{}c@{}lr@{}c@{}lr@{}c@{}l}
\displaystyle \alert<handout:0| 4-5>{\sin\left( \frac{2\pi}{3} \right)} & \alertNoH{4-5}{=}& \displaystyle  \fcAnswer{5}{ \frac{\sqrt{3}}{2}}  &
\displaystyle  \alert<handout:0| 6-7>{\cos\left( \frac{2\pi}{3}\right)} &\alertNoH{6-7}{=}& \displaystyle   \fcAnswer{7}{-\frac{1}{2}} &
\displaystyle  \alert<handout:0| 8-9>{\tan \left(\frac{2\pi}{3}\right)} &\alertNoH{8-9}{=}&\displaystyle   \fcAnswer{9}{\frac{\sqrt{3}}{-1}= -\sqrt{3}} \\
\displaystyle  \alert<handout:0| 10-11>{\csc \left(\frac{2\pi}{3}\right)} &\alertNoH{10-11}{=}&\displaystyle   \fcAnswer{11}{\frac{2}{\sqrt{3}}} &\displaystyle  
\alert<handout:0| 12-13>{\sec \left(\frac{2\pi}{3}\right)} &\alertNoH{12-13}{=}&\displaystyle   \fcAnswer{13}{-\frac{2}{1}=-2} &\displaystyle  
\alert<handout:0| 14-15>{\cot \left(\frac{2\pi}{3}\right)} & \alertNoH{14-15}{ =}&\displaystyle   \fcAnswer{15}{-\frac{1}{\sqrt{3}}}
\end{array}
\]
\end{example}
\end{frame}
% end module trig-example
