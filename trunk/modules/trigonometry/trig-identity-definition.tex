\begin{frame}
\frametitle{Trigonometric Identities}
\begin{definition}[Trigonometric Identity]
A trigonometric identity is an equality between the trigonometric functions in one or more variables that holds for all values of the involved variables in the domains of all of the expressions.
\end{definition}
\begin{itemize}
\item<2-> By convention, when dealing with trigonometric identities we do not account for the domains of the involved expressions.
\item<3-> For example, $\frac{\sin\theta}{\sin \theta}=1$ is considered a valid trigonometric identity, although, when considered as a function, the right hand side is not defined for $\theta\neq 0$.
\end{itemize}
\end{frame}
