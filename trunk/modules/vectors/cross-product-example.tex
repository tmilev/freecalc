\begin{frame}
 \frametitle{Example}

$$\textbf{u} \times \textbf{v} = \langle u_1, u_2, u_3 \rangle \times \langle v_1, v_2, v_3 \rangle =
\left|  \begin{array}{ccc}
      \textbf{i} & \textbf{j} & \textbf{k} \\
      u_1 & u_2 & u_3 \\
      v_1 & v_2 & v_3
        \end{array}
\right|$$

If $\textbf{u} = \langle 1,2,3\rangle$ and $\textbf{v} = \langle 6,5,4 \rangle$, then
%
$$\textbf{u} \times \textbf{v} = \langle 1, 2, 3 \rangle \times \langle 6, 5, 4 \rangle =
\left|  \begin{array}{ccc}
      \textbf{i} & \textbf{j} & \textbf{k} \\
      1 & 2 & 3 \\
      6 & 5 & 4
        \end{array}
\right| = $$
%
$$= \left| \begin{array}{cc}
           2 & 3\\
	   5 & 4
          \end{array}
\right| \textbf{i} - \left| \begin{array}{cc}
           1 & 3\\
	   6 & 4
          \end{array}
\right| \textbf{j} + \left| \begin{array}{cc}
           1 & 2\\
	   6 & 5
          \end{array}
\right| \textbf{k}  = $$
%
$$=(2\cdot 4 -3\cdot 5) \textbf{i} - (1\cdot 4 - 3\cdot 6) \textbf{j} + (1 \cdot 5 - 2\cdot 6) \textbf{k} =$$
%
$$= -7 \textbf{i} + 14 \textbf{j} -7 \textbf{k} \; .$$
%
\end{frame}