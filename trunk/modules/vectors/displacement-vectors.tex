\begin{frame}
\frametitle{Displacement Vectors}
\begin{columns}\column{0.3\textwidth}
\begin{pspicture}(-0.2,-1.1)(2.3,1.1)
\fcBoundingBox{-0.2}{-1.1}{2.3}{1.1}
\uncover<1>{\fcFullDot{2}{1}}
\uncover<1>{\fcFullDot{1}{-1}}
\rput[b](1.2, -1){$A$}
\rput[t](2.2, 1){$B$}
%\uncover<4>{\fcFullDotBlack{0}{0}}
\uncover<2->{\psline[arrows=->](1,-1)(2, 1)}
\uncover<4>{\psline[linecolor=red, arrows=->>](1,-1)(2, 1)}
\end{pspicture}
\column{0.7\textwidth}
\begin{definition}
A displacement vector is an ordered pair of points $(A, B)$.
\end{definition}
\end{columns}
\begin{itemize}
\item<2-> When $A\neq B$, represent as arrow, $A$ - tail $B$- head.
\item<3-> Define displacement vector magnitude $(A,B)$ to be the length of the segment $|AB|$. 
\item<4-> If $A\neq B$ the direction of the displacement vector is defined as the ray starting at $A$ and passing through $B$.
\item<5-> If $A=B$:
\begin{itemize}
\item<6-> displacement vector has zero magnitude and non-specified direction
\item<7-> $(A,A)$: zero displacement vector at point $A$.
\end{itemize}
\end{itemize}
\end{frame}