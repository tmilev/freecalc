\begin{frame}
\frametitle{Line from Point and Direction}
\begin{columns}
\column{0.3\textwidth}
\psset{xunit=1cm, yunit=1cm}
\begin{pspicture}(-0.2,-0.2)(3.5,2)
\tiny
\fcFullDot{0}{0}
\rput[tl](0,-0.1){$O$}
\psline(0, 1.7)(3.5,0.3)
\fcFullDot{0.5}{1.5}
\rput[bl](0.5, 1.5){$P_0$}
\uncover<4->{
\psline[arrows=->, linecolor=red](0.5,1.5)(2.5,0.7)
}
\psline[arrows=->, linecolor=blue](1,1.3)(2, 0.9)
\rput[b](1.5,1.2) {$\fcv u$}
\rput[l](3.5,0.3){$L$}

\uncover<2->{\psline[arrows=->](0,0)(0.5,1.5)
\rput[l](0.25, 0.75){$~~\fcv r_0$}
}
\uncover<3->{%
\psline[arrows=->](0,0)(2.5, 0.7)
\rput[b](2.5, 0.75){$P$}
\rput[t](1.25, 0.2){$\fcv r$}
}
\end{pspicture}

\column{0.7\textwidth}
\begin{itemize}
\item Suppose we have line $L$ that passes through point $P_0$ and has non-zero direction $\textbf{u}$.
\item<2-> Denote by $\fcv r_0=\fcv{OP}_0$ the position vector of $P_0$.
\item<3->$P$ with position vector $\textbf{r}$ is on $L$ $\Leftrightarrow$ 
\item<4->$\textbf{P}_0\textbf{P}$ has the same direction as $\textbf{u}$ $\Leftrightarrow$
\item<5-> $\textbf{P}_0\textbf{P}$ is a scalar multiple of $\textbf{u}$ $\Leftrightarrow$
\item<6-> $\textbf{r}-\textbf{r}_0 = t\textbf{u}$  for some real number $t$.
\end{itemize}
\end{columns}
\uncover<7->{
\begin{definition}
The equation 
\[
\fcv{r} = \fcv{r}_0+t\fcv{u}
\]
is called a parametric vectorial equation of the the line $L$.
\end{definition}
}
\end{frame}

\begin{frame}
\frametitle{Line from Point and Direction}
\begin{columns}
\column{0.3\textwidth}
\psset{xunit=1.4cm, yunit=1.4cm}
\begin{pspicture}(-0.2,-0.2)(2,2)
\renewcommand{\fcScreen}{[-3 -1 -0.2] 0}
\tiny
\fcAxesIIId{2}{2}{2}
\fcLineIIId{[0.5 0.5 1]}{[3 3 0.5]}
\fcLineIIId[arrows=->]{[0 0 0]}{[1 1 0.9]}
\fcPutIIId[tl]{[0.5 0.5 0.45]}{$~\fcv r_0$}

\fcLineIIId[linecolor=blue, arrows=->]{[1.5 1.5 0.8]}{[2 2 0.7]}
\fcPutIIId[bl]{[1.75 1.75 0.8]}{$\fcv u$}

\fcLineIIId[arrows=->]{[0 0 0]}{[2.5 2.5 0.6]}
\fcPutIIId[t]{[1.25 1.25 0.2]}{$\fcv r$}

\fcDotIIId{[1 1 0.9]}
\fcPutIIId[lb]{[1 1 0.95]}{$P_0$}
\fcDotIIId{[2.5 2.5 0.6]}
\fcPutIIId[lb]{[2.5 2.5 0.65]}{$P$}
\fcPutIIId[t]{[3 3 0.5]}{$~L$}
\end{pspicture}

$L$- line with direction $\textbf{u}$ passing through $P_0$

\column{0.7\textwidth}
\begin{itemize}
\item Point $P_0(x_0,y_0,z_0)$, $\fcv{r}_0=\langle x_0,y_0,z_0\rangle$;
\item Direction $\textbf{u}=\langle u_1,u_2,u_3\rangle$.
\uncover<2->{$P(x,y,z)$ with position vector $\textbf{r}$ is on $L$ $\Leftrightarrow$ \\ }
\uncover<3->{\medskip $\textbf{r} = \textbf{r}_0+t\textbf{u}$ $\Leftrightarrow$\\}
\uncover<4->{\medskip $\langle x,y,z\rangle = \langle x_0,y_0,z_0\rangle + t\langle u_1,u_2,u_3\rangle$ $\Leftrightarrow$\\}
\end{itemize}
\end{columns}
\begin{definition}
\uncover<5->{%
The equations 
\[
\left| \begin{array}{ll}
x & = x_0 + t u_1 \\
y & = y_0 + t u_2 \\
z & = z_0 + t u_3
\end{array}
\right., t\in \mathbb R
\]
are called \alert<5>{parametric scalar equations} of the line $L$.
}
\end{definition}

\end{frame}

\begin{frame}
\uncover<1->{%
\[
\left|
\begin{array}{ll}
x & = x_0 + t u_1 \\
y & = y_0 + t u_2 \\
z & = z_0 + t u_3
\end{array}
\right. \Longrightarrow \boxed{\frac{x-x_0}{u_1} = \frac{y-y_0}{u_2} = \frac{z-z_0}{u_3}} \text{ \textcolor[rgb]{0.98,0.00,0.00}{Symmetric equations}}\]
}%uncover
\begin{itemize}
\item<2->Caution! Symmetric equations are valid for $u_1,u_2,u_3\neq 0$. For example if $u_2=0$ the equations should be:
\[
\frac{x-x_0}{u_1} = \frac{z-z_0}{u_3} \quad  \text{ and } \quad y=y_0 
\]
\end{itemize}

\end{frame}