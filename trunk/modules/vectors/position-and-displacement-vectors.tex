\begin{comment}
\begin{frame}
\frametitle{Position vectors via displacement vectors}
\begin{itemize}
\item<1-> Suppose we have space without chosen origin.
\item<2-> To each displacement vector $(A,B)$, \uncover<3->{assign \alert<3>{position vector} by choosing \alert<3>{origin to be the tail $A$} and giving the position vector by the head $B$.}
\item<4-> We are ready to give ``origin-free'' alternative definition/interpretation of vector.
\end{itemize}
\uncover<4->{
\begin{definition}[Alternative definition/interpretation of position vector]
Define a position vector as the set that consists all displacement vectors equivalent to one fixed displacement vector.
\end{definition}
}
\begin{columns}
\column{0.15\textwidth}
\psset{xunit=1cm, yunit=1cm}
\begin{pspicture}(-0.4,-0.4)(2, 1.3)
\fcBoundingBox{-0.4}{-0.4}{2}{1.3}
\uncover<2->{\fcFullDot[linecolor=gray]{0}{0}}
\uncover<3->{\fcFullDot[linecolor=red]{0}{0}}
\uncover<2->{
\rput[tl](1, 0.9){$B$}
\rput[tl](0,-0.1){$\alert<3>{ A\uncover<3->{=O}}$}
\fcFullDot[linecolor=gray]{1}{1}
\psline[arrows=->](0,0)(1,1)
}
\end{pspicture}
\column{0.85\textwidth}
\begin{itemize}
\item<5-> Definitions are technically different but equivalent.  \item<6-> We choose which def. to use according to application.
\item<7-> The set of zero displacement vectors with arbitrary tail points = zero  position vector, $\textbf{0}$. 
\end{itemize}
\end{columns}
\end{frame}
\end{comment}
\begin{frame}
\frametitle{Additional notation for position vectors}
\begin{itemize}
\item In preceding slide: each position vector $\bm u$ can be thought of as a set of equivalent displacement vectors.
\item<2-> So we can represent position vectors via displacement vectors.
\item<3-> For two points $A, B$ define the position vector $\overrightarrow{AB}$ or $\bm {AB}$ as the vector represented by the displacement vector $(A,B)$.
\item<4-> $\Rightarrow$ it's allowed to represent position vectors as arrows with tails not necessarily at origin.
%\item<5->  We write $\bm {AB}= \bm{CD}=\bm u$.
\end{itemize}
\begin{figure}
\begin{pspicture}(-1, -2)(2,2)
\fcBoundingBox{-1}{-2.5}{3}{2}
\uncover<1->{
%\fcFullDot{0}{0}
\psline[arrows=->](0,0)(1,1)
\rput[l](0.5, -0.5){$\bm u\uncover<3->{=\bm {AB}}\uncover<4->{=\bm{CD}}$} 
}
\uncover<2->{
\rput[tl](0,-0.1){$A$}
\rput[tl](1,0.9){$B$}
}
\uncover<4->{
%\fcFullDot[lin]{1}{0}
\psline[arrows=->](1,0)(2,1)
\rput[lt](1.1, 0){$C$}
\rput[lt](2, 0.9){$D$}
}
\end{pspicture}
\end{figure}
\end{frame}