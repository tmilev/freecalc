\begin{frame}
\frametitle{Torque}
\begin{columns}
\begin{pspicture}(-0.2, -1.2)(2, 1.2)
\tiny%
\fcBoundingBox{-0.2}{-1.2}{2}{1.2}%
\uncover<1-2>{\fcLineIIId{[0 0 -0.2]}{[0 0 -1]}}%
\uncover<3->{\fcLineIIId[arrows=->]{[0 0 -0.2]}{[0 0 -1]}}%
\uncover<3->{\fcLineIIId[arrows=->]{[0 0 1]}{[0 0 0.1]}}%
%\fcAxesIIId{1}{1}{1}
\fcPolyLineIIId{[0.2  0 0.1] [0.1 0.173205 0.1] [-0.1 0.173205 0.1] [-0.2 0 0.1] [-0.1 -0.173205 0.1] [0.1 -0.173205 0.1] [0.2 0 0.1]}%
\fcPolyLineIIId{[-0.1 -0.173205 -0.1] [0.1 -0.173205 -0.1] [0.2 0 -0.1] [0.1 0.173205 -0.1]}%
\fcLineIIId{[-0.1 -0.173205 -0.1]}{[-0.1 -0.173205 0.1]}%
\fcLineIIId{[0.1 -0.173205 -0.1]}{[0.1 -0.173205 0.1]}%
\fcLineIIId{[0.2 0 -0.1]} {[0.2 0 0.1]}%
\fcLineIIId{[0.1 0.173205 -0.1]}{[0.1 0.173205 0.1]}%
\uncover<2->{\fcPolyLineIIId[linecolor=blue]{[0.15 -0.259808 0] [0.3 0 0] [0.15 0.259808 0] [0.04 0.259808 0]}%
\fcLineIIId[linecolor=blue]{[0.15 0.259808 0]}{[0.15 0.259808 0] 5 \fcVectorTimesScalar}%
}%
\uncover<4->{\fcLineIIId[linecolor=blue, arrows=->]{[0.15 0.259808 0]} {[0.15 0.259808 0] 5 \fcVectorTimesScalar}%
\fcPutIIId[b]{[0.32 0.65 0.1]}{$\fcv r$}%
}%
\uncover<3->{\fcAngleIIId[arrows=->, linecolor=blue]{[0.15 0.259808 0]}{[0.4 0.2 0]}{1.2}}%
\uncover<4->{\fcLineIIId[arrows=->, linecolor=red]{[0.75 1.29904 0]}{[1 0.5 1]}}%
\uncover<5->{\fcLineIIId[arrows=->]{[0.75 1.29904 0]}{[0.75 1.29904 1]}
\fcPutIIId[l]{[0.75 1.3 0.5]}{$~~\fcv F_o$}
}%
\uncover<6->{\fcLineIIId[arrows=->]{[0.75 1.29904 0]}{[0.466506 0.808013 0]}%
\fcPutIIId[b]{[0.61 1.05 0]}{$~~\fcv F_\rho$}%
}%
\uncover<7->{\fcLineIIId[arrows=->]{[0.75 1.29904 0]}{[1.283494 0.991027 0]}%
\fcPutIIId[l]{[1 1.15 0]}{$~~\fcv F_\theta$}%
}%
\uncover<9->{\fcPutIIId[r]{[0 0 -0.5]}{$\tau~$}}%
\end{pspicture}

\column{0.7\textwidth}
\begin{itemize}
\item If we tighten a bolt \uncover<2->{using a wrench,} \uncover<3->{it moves in direction perpendicular to the motion of the wrench.}
\item<4-> Let arm of the wrench: given by vector $\fcv r$. 
\end{itemize}
\end{columns}
\begin{itemize}
\item<4-> Suppose we are applying a force $\fcv F$ at arm of the wrench. The force has three components: 
\begin{itemize}
\item<5-> component $\fcv F_o$ orthogonal to the plane of rotation
\item<6-> component $\fcv F_\rho$ in the plane of rotation towards/away from the center
\item<7-> component $\fcv F_\theta$ tangent to the motion of the wrench.
\end{itemize}
\item<8-> Only $\fcv F_\theta$ contributes to the bolt motion.
\item<9-> The force of bolt motion $\fcv \tau$ is proportional to length of wrench.
\item<10-> It turns out $\fcv \tau = \fcv r \times (\fcv F_\rho+\fcv F_\theta)$, where \alert<10>{$\times$ is the vector cross product}.
\end{itemize}
\end{frame}