
\begin{frame}
\[
V = 2\pi rh\Delta r.
\]
\uncover<2->{
If we are approximating the solid obtained by rotating the region under $f(x)$, let the height $h = f(r)$.  Then
\[
V = 2\pi rf(r)\Delta r .
\]
}
\uncover<3->{
If there are $n$ cyclindrical shells $C_1, \ldots , C_n$, let $x_1, \ldots , x_n$ be their radii.  Then
\[
\sum_{i=1}^n 2\pi x_i f(x_i)\Delta x
\]
is the volume of the approximating cylinders.
}

\uncover<4->{
Taking the limit as the number of shells goes to $\infty$, we get
\[
V = \lim_{n\rightarrow\infty} \sum_{i=1}^n 2\pi x_i f(x_i)\Delta x = \int_{\uncover<5->{\alertNoH{ 5}{a}}}^{\uncover<5->{\alertNoH{ 5}{b}}}2\pi xf(x) \diff x .
\]
\uncover<5->{
If we are considering the region bounded by $f(x)$ between $a$ and $b$, this gives us the endpoints.
}
}
\end{frame}

\begin{frame}
\begin{definition}[Volume by Cylindrical Shells]
The volume of the solid obtained by rotating around the $y$-axis the region under the curve $y = f(x)$ from $a$ to $b$ is
\[
V = \int_a^b 2\pi xf(x)\diff x .
\]
\end{definition}
\end{frame}