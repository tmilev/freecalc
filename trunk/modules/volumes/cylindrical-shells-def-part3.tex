
\begin{frame}
\begin{columns}
\column{0.37\textwidth}
\begin{pspicture}(-2.2,-1.2)(2.6,2.5)%
\tiny%
\renewcommand{\fcScreenStyle}{x}%
\fcBoundingBox{-2.2}{-1.1}{2.5}{2.4}%
\renewcommand{\fcScreen}{[0 0 -1] 0}%
\renewcommand{\fcScreen}{[-0.28 dup -1] 0}%
\newcommand{\theFun}{x x mul 2 mul x x x mul mul sub\space}%
\newcommand{\theYAxis}{0\space}%
\newcommand{\theSurfaceFla}{1 dict begin /x u def v cos u mul \theFun v sin u mul end\space}%
\fcStartIIIdScene%
\fcAxesIIIdInScene[zLabel={}, arrows=->]{2.2}{2.2}{3}%
\fcPutIIId[t]{[0 0 3.1]}{$z$}%
\only<handout:1|1,2>{%
\fcSurfaceInScene[colorVU= {1 0.5 0.5}, iterationsU=6, iterationsV=12, arrows=none, linewidth=0.3, linecolor=black, forceForeground=true ]{0.03}{0}{2}{360}{[\theSurfaceFla]}{}%
}%
\fcCurveIIIdInScene[linewidth=1, arrows=(none), linecolor= \fcColorGraph]{0}{2}{1 dict begin /x {t} def  [t \theFun 0] end\space}%
\fcFinishIIIdScene%
%\fcCurveIIId{-0.8}{2.2}{1 dict begin /u {t} def  [t \theFun 0] end}%
%\fcLineIIId[linecolor=black, linewidth=0.4pt]{[-0.05 1 0]}{[0.05 1 0]}%
%\fcLineIIId[linecolor=black, linewidth=0.4pt]{[1 -0.05 0]}{[1 0.05 0]}%
\newcommand{\numIterations}{5}%
\only<handout:0|10->{\renewcommand{\numIterations}{10}}%
\only<handout:0|11->{\renewcommand{\numIterations}{15}}%
\only<handout:0|12->{\renewcommand{\numIterations}{20}}%
\only<handout:2|2->{%
\multido{\na=1+1}{\numIterations}{%
\pstVerb{%
/numIterations \numIterations\space def
/theCounter \na\space 1 sub def%
/theRad theCounter numIterations div 2 mul def
/theRadNext theCounter 1 add numIterations div 2 mul def
/theRadNextNext theCounter 2 add numIterations div 2 mul def
1 dict begin /x theRad theRadNext add 2 div def \theFun end 
/theHeight exch def
1 dict begin /x theRadNext theRadNextNext add 2 div def \theFun end 
/theHeightNext exch def
/theHeightNext theHeightNext 0 lt {0}{theHeightNext}ifelse def
/goodAngleMax 160 def
/goodAngleMin -10 def
}%
\fcCurveIIId[linecolor=cyan, linewidth=0.3pt]{0}{360}{  [t cos theRad mul theHeight t sin theRad mul ] }%
%\fcCurveIIId[linecolor=red, linewidth=0.3pt]{0}{360}{  [t cos theRad mul 0 t sin theRad mul ] }%
\fcLineIIId[linecolor=cyan]{[goodAngleMax cos theRadNext mul theHeight goodAngleMax sin theRadNext mul]}{[goodAngleMax cos theRadNext mul theHeightNext goodAngleMax sin theRadNext mul ]}%
\fcCurveIIId[linecolor=cyan, linewidth=0.3pt]{0 }{360}{  [t cos theRadNext mul theHeight t sin theRadNext mul ] }%
\fcLineIIId[linecolor=cyan]{[goodAngleMin cos theRadNext mul theHeight goodAngleMin sin theRadNext mul]}{[goodAngleMin cos theRadNext mul theHeightNext goodAngleMin sin theRadNext mul ]}%
\only<-10>{%
\fcLineIIId[linecolor=black, linewidth=0.4pt]{[theRad -0.05 0]}{[ theRad  0.05 0]}%
\fcLineIIId[linecolor=black, linewidth=0.4pt]{[theRad theRadNext add 2 div -0.03 0  ]}{[theRad theRadNext add 2 div 0.03 0 ]}%
}%
%\fcCurveIIId[linecolor=red, linewidth=0.3pt]{0 }{360}{  [t cos theRadNext mul 0 t sin theRadNext mul ] }%
%\fcLineIIId[linecolor=\fcColorGraph]{[goodAngleMin cos theRad mul goodAngleMin sin theRad mul theHeightNext]}{[goodAngleMin cos theRad mul goodAngleMin sin theRad mul theHeight]}%
\only<5>{%
\fcLineIIId[linewidth=0.3pt]{[theRadNext theRad add 2 div 0 0]}{[theRadNext theRad add 2 div theHeight 0]}%
\fcLineIIId[linewidth=0.3pt]{[theRad theHeight 0]}{[theRadNext theHeight 0]}%
}%
\only<6-10>{\fcPutIIId[t]{[theRadNext theRad add 2 div -0.1 0]}{$x_{\na}$}}%
}%
\only<handout:0|5>{\fcLineIIId[linecolor=purple, linewidth=1.5pt]{[theRad theRadNext add 2 div 0 0]}{[theRad theRadNext add 2 div theHeight 0]}%
\fcPutIIId[t]{[theRad theRadNext add 2 div -0.1 0]}{$r$}%
}%
\fcCurveIIId[linecolor=cyan, linewidth=0.3pt]{0}{360}{[t cos 2 mul 0 t sin 2 mul]}%
\only<handout:0|14->{%
\fcLineIIId[linecolor=blue, linewidth=1.5pt]{[0 0 0]}{[2 0 0]}%
\fcDotIIId[linecolor=blue]{[0 0 0]}%
\fcDotIIId[linecolor=blue]{[2 0 0]}%
\fcPutIIId{[2 -0.1 0]}{$b$}
\fcPutIIId{[0 -0.1 0]}{$a$}
}%
}%only
\end{pspicture}
\column{0.65\textwidth}

Consider a solid obtained by rotating the region under $f(x)$ around the $y$ axis. \uncover<handout:2|2->{ Approximate the volume by cylindrical shells.  \uncover<5->{Select the height of an individual shell to be $h = f(r)$ ($r$=average outer \& inner radius).}

\hfil \hfil $
\uncover<4->{ V_{\text{shell}} = 2\pi r \alertNoH{5}{h}\Delta r} \uncover<5->{ = 2\pi \alertNoH{7}{r} \alertNoH{5}{f(\alertNoH{7}{r})} \alertNoH{8}{\Delta r} .}
$
}
\end{columns}
\uncover<handout:2|6->{
Suppose there are $n$ cyclindrical shells and let $x_1, \ldots , x_n$ be the averages of outer and inner radii. The shell volume sum is:

\hfil \hfil $\displaystyle
V_{\text{approx}}=\sum_{i=1}^n 2\pi \alertNoH{7}{x_i} f(\alertNoH{7}{x_i})\alertNoH{8}{ \Delta x}.
$
}

\uncover<handout:2|9->{
Take the limit as the number of shells goes to $\infty$ to get
\[
V = \lim_{n\to \infty} V_{\text{approx}}= \lim_{n\rightarrow\infty} \sum_{i=1}^n 2\pi x_i f(x_i)\Delta x \uncover<13->{= \int_{\alertNoH{ 14}{a}}^{\alertNoH{ 14}{b}}2\pi xf(x) \diff x .}
\]
\uncover<14->{
The endpoints of integration are the endpoints of the rotated region.
}
}
\end{frame}

\begin{frame}
\begin{definition}[Volume by Cylindrical Shells]
The volume of the solid obtained by rotating around the $y$-axis the region under the curve $y = f(x)$ from $a$ to $b$ is
\[
V = \int_a^b 2\pi xf(x)\diff x .
\]
\end{definition}
\end{frame}